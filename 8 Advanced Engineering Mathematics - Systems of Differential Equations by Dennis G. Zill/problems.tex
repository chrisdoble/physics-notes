\documentclass{article}
\usepackage{amsmath} % For align*
\usepackage{amsfonts} % For open face letters
\usepackage{enumitem} % For customisable list labels
\usepackage{graphicx} % For images
\usepackage{siunitx} % For units
\graphicspath{{./images/}}

\setlist[enumerate, 1]{label={(\alph*)}}
\setlist[enumerate, 2]{label={(\roman*)}}

\title{Advanced Engineering Mathematics Systems of Differential Equations by Dennis G. Zill Problems}
\author{Chris Doble}
\date{June 2023}

\begin{document}

\maketitle

\tableofcontents

\setcounter{section}{9}
\section{Systems of Linear Differential Equations}

\subsection{Theory of Linear Systems}

\subsubsection{}

\[\mathbf{X}' = \begin{pmatrix}
    3 & -5 \\
    4 & 8
  \end{pmatrix} \mathbf{X}\]

\setcounter{subsubsection}{2}
\subsubsection{}

\[\mathbf{X}' = \begin{pmatrix}
    -3 & 4  & -9 \\
    6  & -1 & 0  \\
    10 & 4  & 3
  \end{pmatrix} \mathbf{X}\]

\setcounter{subsubsection}{4}
\subsubsection{}

\[\mathbf{X}' = \begin{pmatrix}
    1 & -1 & 1  \\
    2 & 1  & -1 \\
    1 & 1  & 1
  \end{pmatrix} \mathbf{X} + \begin{pmatrix}
    t - 1  \\
    -3 t^2 \\
    t^2 - t + 2
  \end{pmatrix}\]

\setcounter{subsubsection}{6}
\subsubsection{}

\begin{align*}
  \frac{d x}{d t} & = 4 x + 2 y + e^t \\
  \frac{d y}{d t} & = -x + 3 y - e^t
\end{align*}

\setcounter{subsubsection}{8}
\subsubsection{}

\begin{align*}
  \frac{d x}{d t} & = x - y + 2 z + e^{-t} - 3 t      \\
  \frac{d y}{d t} & = 3 x - 4 y + z + 2 e^{-t} + t    \\
  \frac{d z}{d t} & = -2 x + 5 y + 6 z + 2 e^{-t} - t
\end{align*}

\setcounter{subsubsection}{10}
\subsubsection{}

\begin{align*}
  3 (e^{-5 t}) - 4 (2 e^{-5 t}) & = -5 e^{-5 t}     \\
                                & = \frac{d x}{d t} \\
  4 (e^{-5 t}) - 7 (2 e^{-5 t}) & = -10 e^{-5 t}    \\
                                & = \frac{d y}{d t}
\end{align*}

\setcounter{subsubsection}{12}
\subsubsection{}

\begin{align*}
  -(-e^{-3 t / 2}) + \frac{1}{4} (2 e^{-3 t / 2}) & = \frac{3}{2} e^{-3 t / 2} \\
                                                  & = \frac{d x}{d t}          \\
  (-e^{-3 t / 2}) - (2 e^{-3 t / 2})              & = -3 e^{-3 t / 2}          \\
                                                  & = \frac{d y}{d t}
\end{align*}

\setcounter{subsubsection}{16}
\subsubsection{}

\begin{align*}
  W(\mathbf{X}_1, \mathbf{X}_2) & = \begin{vmatrix}
                                      e^{-2 t} & e^{-6 t}  \\
                                      e^{-2 t} & -e^{-6 t}
                                    \end{vmatrix}                     \\
                                & = -e^{-8 t} - e^{-8 t}                     \\
                                & = -2 e^{-8 t}                              \\
                                & \ne 0 \text{ for } t \in (-\infty, \infty)
\end{align*}

Yes, they form a fundamental set.

\setcounter{subsubsection}{18}
\subsubsection{}

\begin{align*}
  W(\mathbf{X}_1, \mathbf{X}_2, \mathbf{X}_3) & = \begin{vmatrix}
                                                    1 + t    & 1  & 3 + 2 t  \\
                                                    -2 + 2 t & -2 & -6 + 4 t \\
                                                    4 + 2 t  & 4  & 12 + 4 t
                                                  \end{vmatrix} \\
                                              & = 0
\end{align*}

No, they don't form a fundamental set.

\setcounter{subsubsection}{20}
\subsubsection{}

\begin{align*}
  x               & = 2 t + 5                             \\
  y               & = -t + 1                              \\
  \frac{d x}{d t} & = (2 t + 5) + 4 (-t + 1) + 2 t - 7    \\
                  & = 2                                   \\
  \frac{d y}{d t} & = 3 (2 t + 5) + 2 (-t + 1) - 4 t - 18 \\
                  & = -1
\end{align*}

\setcounter{subsubsection}{22}
\subsubsection{}

\begin{align*}
  x               & = e^t + t e^t                               \\
  x'              & = 2 e^t + t e^t                             \\
  y               & = e^t - t e^t                               \\
  y'              & = - t e^t                                   \\
  \frac{d x}{d t} & = 2 (e^t + t e^t) + (e^t - t e^t) - e^t     \\
                  & = 2 e^t + t e^t                             \\
  \frac{d y}{d t} & = 3 (e^t + t e^t) + 4 (e^t - t e^t) - 7 e^t \\
                  & = -t e^t
\end{align*}

\end{document}