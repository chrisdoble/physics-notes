\documentclass{article}
\usepackage{amsmath} % For align*
\usepackage{amsfonts} % For open face letters
\usepackage{enumitem} % For customisable list labels
\usepackage{graphicx} % For images
\usepackage{siunitx} % For units
\graphicspath{{./images/}}

\setlist[enumerate, 1]{label={(\alph*)}}
\setlist[enumerate, 2]{label={(\roman*)}}

\title{Advanced Engineering Mathematics Systems of Differential Equations by Dennis G. Zill Problems}
\author{Chris Doble}
\date{June 2023}

\begin{document}

\maketitle

\tableofcontents

\setcounter{section}{9}
\section{Systems of Linear Differential Equations}

\subsection{Theory of Linear Systems}

\subsubsection{}

\[\mathbf{X}' = \begin{pmatrix}
    3 & -5 \\
    4 & 8
  \end{pmatrix} \mathbf{X}\]

\setcounter{subsubsection}{2}
\subsubsection{}

\[\mathbf{X}' = \begin{pmatrix}
    -3 & 4  & -9 \\
    6  & -1 & 0  \\
    10 & 4  & 3
  \end{pmatrix} \mathbf{X}\]

\setcounter{subsubsection}{4}
\subsubsection{}

\[\mathbf{X}' = \begin{pmatrix}
    1 & -1 & 1  \\
    2 & 1  & -1 \\
    1 & 1  & 1
  \end{pmatrix} \mathbf{X} + \begin{pmatrix}
    t - 1  \\
    -3 t^2 \\
    t^2 - t + 2
  \end{pmatrix}\]

\setcounter{subsubsection}{6}
\subsubsection{}

\begin{align*}
  \frac{d x}{d t} & = 4 x + 2 y + e^t \\
  \frac{d y}{d t} & = -x + 3 y - e^t
\end{align*}

\setcounter{subsubsection}{8}
\subsubsection{}

\begin{align*}
  \frac{d x}{d t} & = x - y + 2 z + e^{-t} - 3 t      \\
  \frac{d y}{d t} & = 3 x - 4 y + z + 2 e^{-t} + t    \\
  \frac{d z}{d t} & = -2 x + 5 y + 6 z + 2 e^{-t} - t
\end{align*}

\setcounter{subsubsection}{10}
\subsubsection{}

\begin{align*}
  3 (e^{-5 t}) - 4 (2 e^{-5 t}) & = -5 e^{-5 t}     \\
                                & = \frac{d x}{d t} \\
  4 (e^{-5 t}) - 7 (2 e^{-5 t}) & = -10 e^{-5 t}    \\
                                & = \frac{d y}{d t}
\end{align*}

\setcounter{subsubsection}{12}
\subsubsection{}

\begin{align*}
  -(-e^{-3 t / 2}) + \frac{1}{4} (2 e^{-3 t / 2}) & = \frac{3}{2} e^{-3 t / 2} \\
                                                  & = \frac{d x}{d t}          \\
  (-e^{-3 t / 2}) - (2 e^{-3 t / 2})              & = -3 e^{-3 t / 2}          \\
                                                  & = \frac{d y}{d t}
\end{align*}

\setcounter{subsubsection}{16}
\subsubsection{}

\begin{align*}
  W(\mathbf{X}_1, \mathbf{X}_2) & = \begin{vmatrix}
                                      e^{-2 t} & e^{-6 t}  \\
                                      e^{-2 t} & -e^{-6 t}
                                    \end{vmatrix}                     \\
                                & = -e^{-8 t} - e^{-8 t}                     \\
                                & = -2 e^{-8 t}                              \\
                                & \ne 0 \text{ for } t \in (-\infty, \infty)
\end{align*}

Yes, they form a fundamental set.

\setcounter{subsubsection}{18}
\subsubsection{}

\begin{align*}
  W(\mathbf{X}_1, \mathbf{X}_2, \mathbf{X}_3) & = \begin{vmatrix}
                                                    1 + t    & 1  & 3 + 2 t  \\
                                                    -2 + 2 t & -2 & -6 + 4 t \\
                                                    4 + 2 t  & 4  & 12 + 4 t
                                                  \end{vmatrix} \\
                                              & = 0
\end{align*}

No, they don't form a fundamental set.

\setcounter{subsubsection}{20}
\subsubsection{}

\begin{align*}
  x               & = 2 t + 5                             \\
  y               & = -t + 1                              \\
  \frac{d x}{d t} & = (2 t + 5) + 4 (-t + 1) + 2 t - 7    \\
                  & = 2                                   \\
  \frac{d y}{d t} & = 3 (2 t + 5) + 2 (-t + 1) - 4 t - 18 \\
                  & = -1
\end{align*}

\setcounter{subsubsection}{22}
\subsubsection{}

\begin{align*}
  x               & = e^t + t e^t                               \\
  x'              & = 2 e^t + t e^t                             \\
  y               & = e^t - t e^t                               \\
  y'              & = - t e^t                                   \\
  \frac{d x}{d t} & = 2 (e^t + t e^t) + (e^t - t e^t) - e^t     \\
                  & = 2 e^t + t e^t                             \\
  \frac{d y}{d t} & = 3 (e^t + t e^t) + 4 (e^t - t e^t) - 7 e^t \\
                  & = -t e^t
\end{align*}

\subsection{Homogeneous Linear Systems}

\subsubsection{}

\[\mathbf{X} = c_1 \begin{pmatrix}
    1 \\
    2
  \end{pmatrix} e^{5 t} + c_2 \begin{pmatrix}
    -1 \\
    1
  \end{pmatrix} e^{-t}\]

\setcounter{subsubsection}{2}
\subsubsection{}

\[\mathbf{X} = c_1 \begin{pmatrix}
    2 \\
    1
  \end{pmatrix} e^{-3 t} + c_2 \begin{pmatrix}
    \frac{2}{5} \\
    1
  \end{pmatrix} e^{t}\]

\setcounter{subsubsection}{4}
\subsubsection{}

\[\mathbf{X} = c_1 \begin{pmatrix}
    1 \\
    4
  \end{pmatrix} e^{-10 t} + c_2 \begin{pmatrix}
    5 \\
    2
  \end{pmatrix} e^{8 t}\]

\setcounter{subsubsection}{6}
\subsubsection{}

\[\mathbf{X} = c_1 \begin{pmatrix}
    2 \\
    3 \\
    1
  \end{pmatrix} e^{2 t} + c_2 \begin{pmatrix}
    1 \\
    0 \\
    2
  \end{pmatrix} e^{-t} + c_3 \begin{pmatrix}
    1 \\
    0 \\
    0
  \end{pmatrix} e^t\]

\setcounter{subsubsection}{12}
\subsubsection{}

\[\mathbf{X} = 2 \begin{pmatrix}
    0 \\
    1
  \end{pmatrix} e^{-t / 2} + 3 \begin{pmatrix}
    1 \\
    1
  \end{pmatrix} e^{t / 2}\]

\setcounter{subsubsection}{14}
\subsubsection{}

\begin{enumerate}
  \item

        \begin{align*}
          \frac{d x_1}{d t}    & = -\frac{3}{100} x_1 + \frac{1}{100} x_2 \\
          \frac{d x_2}{d t}    & = \frac{2}{100} x_1 - \frac{2}{100} x_2  \\
          \begin{pmatrix}
            \frac{d x_1}{d t} \\
            \frac{d x_2}{d t}
          \end{pmatrix} & = \begin{pmatrix}
                              -\frac{3}{100} & \frac{1}{100}  \\
                              \frac{2}{100}  & -\frac{2}{100}
                            \end{pmatrix} \begin{pmatrix}
                                            x_1 \\
                                            x_2
                                          \end{pmatrix}
        \end{align*}

  \item

        \[\begin{pmatrix}
            x_1 \\
            x_2
          \end{pmatrix} = -\frac{35}{3} \begin{pmatrix}
            -1 \\
            1
          \end{pmatrix} e^{-t / 25} + \frac{50}{3} \begin{pmatrix}
            \frac{1}{2} \\
            1
          \end{pmatrix} e^{-t / 100}\]
\end{enumerate}

\setcounter{subsubsection}{20}
\subsubsection{}

\[\mathbf{X} = c_1 \begin{pmatrix}
    1 \\
    3
  \end{pmatrix} + c_2 \left[ \begin{pmatrix}
      1 \\
      3
    \end{pmatrix} t +
    \begin{pmatrix}
      0 \\
      -1
    \end{pmatrix} \right]\]

\setcounter{subsubsection}{22}
\subsubsection{}

\[\mathbf{X} = c_1 \begin{pmatrix}
    1 \\
    1
  \end{pmatrix} e^{2 t} + c_2 \left[ \begin{pmatrix}
      1 \\
      1
    \end{pmatrix} t + \begin{pmatrix}
      0 \\
      \frac{1}{3}
    \end{pmatrix} \right] e^{2 t}\]

\setcounter{subsubsection}{24}
\subsubsection{}

\[\mathbf{X} = c_1 \begin{pmatrix}
    1 \\
    0 \\
    1
  \end{pmatrix} e^{2 t} + c_2 \begin{pmatrix}
    1 \\
    1 \\
    0
  \end{pmatrix} e^{2 t} + c_3 \begin{pmatrix}
    1 \\
    1 \\
    1
  \end{pmatrix} e^t\]

\setcounter{subsubsection}{30}
\subsubsection{}

\[\mathbf{X} = -\frac{1}{2} \begin{pmatrix}
    2 \\
    1
  \end{pmatrix} e^{4 t} + 13 \left[ \begin{pmatrix}
      2 \\
      1
    \end{pmatrix} t + \begin{pmatrix}
      0 \\
      \frac{1}{2}
    \end{pmatrix} \right] e^{4 t}\]

\setcounter{subsubsection}{32}
\subsubsection{}

\begin{align*}
  \mathbf{K}_1 & = \begin{pmatrix}
                     1 \\
                     0 \\
                     0 \\
                     0 \\
                     0
                   \end{pmatrix} \\
  \mathbf{K}_1 & = \begin{pmatrix}
                     0 \\
                     0 \\
                     0 \\
                     1 \\
                     0
                   \end{pmatrix} \\
  \mathbf{K}_3 & = \begin{pmatrix}
                     0 \\
                     0 \\
                     1 \\
                     0 \\
                     0
                   \end{pmatrix} \\
\end{align*}

\setcounter{subsubsection}{34}
\subsubsection{}

\begin{align*}
  \mathbf{X} & = c_1 \begin{pmatrix}
                       1 \\
                       2 - i
                     \end{pmatrix} e^{(4 + i) t} + c_2 \begin{pmatrix}
                                                         1 \\
                                                         2 + i
                                                       \end{pmatrix} e^{(4 - i) t}                                                                               \\
             & = c_1 \left[ \begin{pmatrix}
                                1 \\
                                2
                              \end{pmatrix} \cos t - \begin{pmatrix}
                                                       0 \\
                                                       -1
                                                     \end{pmatrix} \sin t \right] e^{4 t} + c_2 \left[ \begin{pmatrix}
                                                                                                         0 \\
                                                                                                         -1
                                                                                                       \end{pmatrix} \cos t + \begin{pmatrix}
                                                                                                                                1 \\
                                                                                                                                2
                                                                                                                              \end{pmatrix} \sin t \right] e^{4 t} \\
             & = c_1 \begin{pmatrix}
                       \cos t \\
                       2 \cos t + \sin t
                     \end{pmatrix} e^{4 t} + c_2 \begin{pmatrix}
                                                   \sin t \\
                                                   2 \sin t - \cos t
                                                 \end{pmatrix} e^{4 t}
\end{align*}

\setcounter{subsubsection}{36}
\subsubsection{}

\begin{align*}
  \mathbf{X} & = c_1 \begin{pmatrix}
                       1 \\
                       -1 + i
                     \end{pmatrix} e^{(4 + i) t} + c_2 \begin{pmatrix}
                                                         1 \\
                                                         -1 - i
                                                       \end{pmatrix} e^{(4 - i) t}                                                                               \\
             & = c_1 \left[ \begin{pmatrix}
                                1 \\
                                -1
                              \end{pmatrix} \cos t - \begin{pmatrix}
                                                       0 \\
                                                       1
                                                     \end{pmatrix} \sin t \right] e^{4 t} + c_2 \left[ \begin{pmatrix}
                                                                                                         0 \\
                                                                                                         1
                                                                                                       \end{pmatrix} \cos t + \begin{pmatrix}
                                                                                                                                1 \\
                                                                                                                                -1
                                                                                                                              \end{pmatrix} \sin t \right] e^{4 t} \\
             & = c_1 \begin{pmatrix}
                       \cos t \\
                       -\cos t - \sin t
                     \end{pmatrix} e^{4 t} + c_2 \begin{pmatrix}
                                                   \sin t \\
                                                   \cos t - \sin t
                                                 \end{pmatrix} e^{4 t}
\end{align*}

\setcounter{subsubsection}{38}
\subsubsection{}

\begin{align*}
  \mathbf{X} & = c_1 \begin{pmatrix}
                       5 \\
                       4 - 3 i
                     \end{pmatrix} e^{3 i} + c_2 \begin{pmatrix}
                                                   5 \\
                                                   4 + 3 i
                                                 \end{pmatrix} e^{-3 i}                                                                                  \\
             & = c_1 \left[ \begin{pmatrix}
                                5 \\
                                4
                              \end{pmatrix} \cos 3 t - \begin{pmatrix}
                                                         0 \\
                                                         -3
                                                       \end{pmatrix} \sin 3 t \right] + c_2 \left[ \begin{pmatrix}
                                                                                                     0 \\
                                                                                                     -3
                                                                                                   \end{pmatrix} \cos 3 t + \begin{pmatrix}
                                                                                                                              5 \\
                                                                                                                              4
                                                                                                                            \end{pmatrix} \sin 3 t \right] \\
             & = c_1 \begin{pmatrix}
                       5 \cos 3 t \\
                       4 \cos 3 t + 3 \sin 3 t
                     \end{pmatrix} + c_2 \begin{pmatrix}
                                           5 \sin 3 t \\
                                           4 \sin 3 t - 3 \cos 3 t
                                         \end{pmatrix}
\end{align*}

\setcounter{subsubsection}{46}
\subsubsection{}

\begin{align*}
  \mathbf{X} & = c_1 \begin{pmatrix}
                       25 \\
                       -7 \\
                       6
                     \end{pmatrix} e^t + c_2 \begin{pmatrix}
                                               1 + 5 i \\
                                               1       \\
                                               1
                                             \end{pmatrix} e^{5 i t} + c_3 \begin{pmatrix}
                                                                             1 - 5 i \\
                                                                             1       \\
                                                                             1
                                                                           \end{pmatrix} e^{-5 i t}         \\
             & = c_1 \begin{pmatrix}
                       25 \\
                       -7 \\
                       6
                     \end{pmatrix} e^t + c_2 \left[ \begin{pmatrix}
                                                        1 \\
                                                        1 \\
                                                        1
                                                      \end{pmatrix} \cos 5 t - \begin{pmatrix}
                                                                                 5 \\
                                                                                 0 \\
                                                                                 0
                                                                               \end{pmatrix} \sin 5 t \right] \\
             & \qquad + c_3 \left[ \begin{pmatrix}
                                       5 \\
                                       0 \\
                                       0
                                     \end{pmatrix} \cos 5 t + \begin{pmatrix}
                                                                1 \\
                                                                1 \\
                                                                1
                                                              \end{pmatrix} \sin 5 t \right]                  \\
             & = c_1 \begin{pmatrix}
                       25 \\
                       -7 \\
                       6
                     \end{pmatrix} e^t + c_2 \begin{pmatrix}
                                               \cos 5 t - 5 \sin 5 t \\
                                               \cos 5 t              \\
                                               \cos 5 t
                                             \end{pmatrix} + c_3 \begin{pmatrix}
                                                                   5 \cos 5 t + \sin 5 t \\
                                                                   \sin 5 t              \\
                                                                   \sin 5 t
                                                                 \end{pmatrix}                      \\
             & = -\begin{pmatrix}
                    25 \\
                    -7 \\
                    6
                  \end{pmatrix} e^t - \begin{pmatrix}
                                        \cos 5 t - 5 \sin 5 t \\
                                        \cos 5 t              \\
                                        \cos 5 t
                                      \end{pmatrix} + 6 \begin{pmatrix}
                                                          5 \cos 5 t + \sin 5 t \\
                                                          \sin 5 t              \\
                                                          \sin 5 t
                                                        \end{pmatrix}                               \\
\end{align*}

\setcounter{subsubsection}{48}
\subsubsection{}

\begin{enumerate}
  \item \[\begin{pmatrix}
            x_1' \\
            x_2' \\
            x_3'
          \end{pmatrix} = \begin{pmatrix}
            -\frac{1}{20} & 0             & \frac{1}{10}  \\
            \frac{1}{20}  & -\frac{1}{20} & 0             \\
            0             & \frac{1}{20}  & -\frac{1}{10}
          \end{pmatrix} \begin{pmatrix}
            x_1 \\
            x_2 \\
            x_3
          \end{pmatrix}\]

  \item

        \begin{align*}
          \mathbf{X} & = c_1 \begin{pmatrix}
                               2 \\
                               2 \\
                               1
                             \end{pmatrix} + c_2 \begin{pmatrix}
                                                   -1 - i \\
                                                   i      \\
                                                   1
                                                 \end{pmatrix} e^{\left( -\frac{1}{10} + \frac{1}{20} i \right) t} + c_3 \begin{pmatrix}
                                                                                                                           -1 + i \\
                                                                                                                           -i     \\
                                                                                                                           1
                                                                                                                         \end{pmatrix} e^{\left( -\frac{1}{10} - \frac{1}{20} i \right) t} \\
                     & = c_1 \begin{pmatrix}
                               2 \\
                               2 \\
                               1
                             \end{pmatrix} + c_2 \left[ \begin{pmatrix}
                                                            -1 \\
                                                            0  \\
                                                            1
                                                          \end{pmatrix} \cos \frac{1}{20} t - \begin{pmatrix}
                                                                                                -1 \\
                                                                                                1  \\
                                                                                                0
                                                                                              \end{pmatrix} \sin \frac{1}{20} t \right] e^{-t / 10}                                          \\
                     & \qquad + c_3 \left[ \begin{pmatrix}
                                               -1 \\
                                               1  \\
                                               0
                                             \end{pmatrix} \cos \frac{1}{20} t + \begin{pmatrix}
                                                                                   -1 \\
                                                                                   0  \\
                                                                                   1
                                                                                 \end{pmatrix} \sin \frac{1}{20} t \right] e^{-t / 10}                                                       \\
                     & = c_1 \begin{pmatrix}
                               2 \\
                               2 \\
                               1
                             \end{pmatrix} + c_2 \begin{pmatrix}
                                                   \sin \frac{t}{20} - \cos \frac{t}{20} \\
                                                   -\sin \frac{t}{20}                    \\
                                                   \cos \frac{t}{20}
                                                 \end{pmatrix} e^{-t / 10}                                                                                                     \\
                     & \qquad + c_3 \begin{pmatrix}
                                      -\cos \frac{t}{20} - \sin \frac{t}{20} \\
                                      \cos \frac{t}{20}                      \\
                                      \sin \frac{t}{20}
                                    \end{pmatrix} e^{-t / 10}                                                                                                                 \\
                     & = 11 \begin{pmatrix}
                              2 \\
                              2 \\
                              1
                            \end{pmatrix} - 6 \begin{pmatrix}
                                                \sin \frac{t}{20} - \cos \frac{t}{20} \\
                                                -\sin \frac{t}{20}                    \\
                                                \cos \frac{t}{20}
                                              \end{pmatrix} e^{-t / 10}                                                                                                        \\
                     & \qquad - 2 \begin{pmatrix}
                                    -\cos \frac{t}{20} - \sin \frac{t}{20} \\
                                    \cos \frac{t}{20}                      \\
                                    \sin \frac{t}{20}
                                  \end{pmatrix} e^{-t / 10}
        \end{align*}
\end{enumerate}

\subsection{Solution by Diagonalization}

\subsubsection{}

\[\mathbf{X} = \begin{pmatrix}
    3 & -2 \\
    1 & 3
  \end{pmatrix} \begin{pmatrix}
    c_1 e^{7 t} \\
    c_2 e^{-4 t}
  \end{pmatrix} = \begin{pmatrix}
    3 c_1 e^{7 t} - 2 c_2 e^{-4 t} \\
    c_1 e^{7 t} + 3 c_2 e^{-4 t}
  \end{pmatrix}\]

\setcounter{subsubsection}{2}
\subsubsection{}

\[\mathbf{X} = \begin{pmatrix}
    1 & 1  \\
    2 & -2
  \end{pmatrix} \begin{pmatrix}
    c_1 e^{3 t / 2} \\
    c_2 e^{t / 2}
  \end{pmatrix} = \begin{pmatrix}
    c_1 e^{3 t / 2} + c_2 e^{t / 2} \\
    2 c_1 e^{3 t / 2} - 2 c_2 e^{t / 2}
  \end{pmatrix}\]

\setcounter{subsubsection}{4}
\subsubsection{}

\[\mathbf{X} = \begin{pmatrix}
    0 & -1 & 1 \\
    0 & 1  & 1 \\
    1 & 0  & 1
  \end{pmatrix} \begin{pmatrix}
    c_1 e^{6 t}  \\
    c_2 e^{-4 t} \\
    c_3 e^{2 t}
  \end{pmatrix} = \begin{pmatrix}
    -c_2 e^{-4 t} + c_3 e^{2 t} \\
    c_2 e^{-4 t} + c_3 e^{2 t}  \\
    c_1 e^{6 t} + c_3 e^{2 t}
  \end{pmatrix}\]

\setcounter{subsubsection}{10}
\subsubsection{}

\begin{enumerate}
  \item \[\begin{pmatrix}
            m_1 & 0   \\
            0   & m_2
          \end{pmatrix} \begin{pmatrix}
            x_1'' \\
            x_2''
          \end{pmatrix} + \begin{pmatrix}
            k_1 + k_2 & -k_2 \\
            -k_2      & k_2
          \end{pmatrix} \begin{pmatrix}
            x_1 \\
            x_2
          \end{pmatrix}\]

        $\mathbf{M}$ has an inverse because it has a nonzero determinant (the product of the diagonal entries).

  \item \[\begin{pmatrix}
            x_1'' \\
            x_2''
          \end{pmatrix} + \begin{pmatrix}
            \frac{k_1 + k_2}{m_1} & -\frac{k_2}{m_1} \\
            -\frac{k_2}{m_2}      & \frac{k_2}{m_2}
          \end{pmatrix} \begin{pmatrix}
            x_1 \\
            x_2
          \end{pmatrix} = \mathbf{0}\]

  \item

        \begin{align*}
          \mathbf{X}                      & = \mathbf{P Y}                              \\ \\
          \mathbf{P Y'' + B P Y}          & = \mathbf{0}                                \\
          \mathbf{Y'' + P^{-1} B P Y}     & = \mathbf{0}                                \\
          \mathbf{Y'' + D Y}              & = \mathbf{0}                                \\
          \begin{pmatrix}
            y_1'' \\
            y_2''
          \end{pmatrix} + \begin{pmatrix}
                            6 & 0 \\
                            0 & 1
                          \end{pmatrix} \begin{pmatrix}
                                          y_1 \\
                                          y_2
                                        \end{pmatrix} & = \mathbf{0}                    \\
          y_1'' + 6 y_1                   & = 0                                         \\
          y_1                             & = c_1 \cos \sqrt{6} t + c_2 \sin \sqrt{6} t \\
          y_2'' + y_2                     & = 0                                         \\
          y_2                             & = c_3 \cos t + c_4 \sin t                   \\ \\
        \end{align*}

        \begin{align*}
          \mathbf{X} & = \mathbf{P Y}                                                                                                                \\
                     & = \begin{pmatrix}
                           -2 & 1 \\
                           1  & 2
                         \end{pmatrix} \begin{pmatrix}
                                         c_1 \cos \sqrt{6} t + c_2 \sin \sqrt{6} t \\
                                         c_3 \cos t + c_4 \sin t
                                       \end{pmatrix}                                                                     \\
                     & = \begin{pmatrix}
                           -2 c_1 \cos \sqrt{6} t - 2 c_2 \sin \sqrt{6} t + c_3 \cos t + c_4 \sin t \\
                           c_1 \cos \sqrt{6} t + c_2 \sin \sqrt{6} t + 2 c_3 \cos t + 2 c_4 \sin t
                         \end{pmatrix}                                                    \\
                     & = c_3 \begin{pmatrix}
                               1 \\
                               2
                             \end{pmatrix} \cos t + c_4 \begin{pmatrix}
                                                          1 \\
                                                          2
                                                        \end{pmatrix} \sin t + c_1 \begin{pmatrix}
                                                                                     -2 \\
                                                                                     1
                                                                                   \end{pmatrix} \cos \sqrt{6} t + c_2 \begin{pmatrix}
                                                                                                                         -2 \\
                                                                                                                         1
                                                                                                                       \end{pmatrix} \sin \sqrt{6} t
        \end{align*}
\end{enumerate}

\subsection{Nonhomogeneous Linear Systems}

\subsubsection{}

\begin{align*}
  \begin{pmatrix}
    x' \\
    y'
  \end{pmatrix} & = \begin{pmatrix}
                      2  & 3  \\
                      -1 & -2
                    \end{pmatrix} \begin{pmatrix}
                                    x \\
                                    y
                                  \end{pmatrix} + \begin{pmatrix}
                                                    -7 \\
                                                    5
                                                  \end{pmatrix}                        \\
  \mathbf{X}_c    & = c_1 \begin{pmatrix}
                            -1 \\
                            1
                          \end{pmatrix} e^{-t} + c_2 \begin{pmatrix}
                                                       -3 \\
                                                       1
                                                     \end{pmatrix} e^t                  \\
  \mathbf{X}_p    & = \begin{pmatrix}
                        a_1 \\
                        a_2
                      \end{pmatrix}                                                    \\
  \begin{pmatrix}
    0 \\
    0
  \end{pmatrix} & = \begin{pmatrix}
                      2  & 3  \\
                      -1 & -2
                    \end{pmatrix} \begin{pmatrix}
                                    a_1 \\
                                    a_2
                                  \end{pmatrix} + \begin{pmatrix}
                                                    -7 \\
                                                    5
                                                  \end{pmatrix}                        \\
                  & = \begin{pmatrix}
                        2 a_1 + 3 a_2 - 7 \\
                        -a_1 - 2 a_2 + 5
                      \end{pmatrix}                                                 \\
  \mathbf{X}_p    & = \begin{pmatrix}
                        -1 \\
                        3
                      \end{pmatrix}                                                    \\
  \mathbf{X}      & = \mathbf{X}_c + \mathbf{X}_p                                       \\
                  & = c_1 \begin{pmatrix}
                            -1 \\
                            1
                          \end{pmatrix} e^{-t} + c_2 \begin{pmatrix}
                                                       -3 \\
                                                       1
                                                     \end{pmatrix} e^t + \begin{pmatrix}
                                                                           -1 \\
                                                                           3
                                                                         \end{pmatrix} \\
\end{align*}

\setcounter{subsubsection}{2}
\subsubsection{}

\begin{align*}
  \mathbf{X}_c     & = c_1 \begin{pmatrix}
                             1 \\
                             1
                           \end{pmatrix} e^{4 t} + c_2 \begin{pmatrix}
                                                         -1 \\
                                                         1
                                                       \end{pmatrix} e^{-2 t}                                                \\
  \mathbf{X}_p     & = \begin{pmatrix}
                         a_3 t^2 + a_2 t + a_1 \\
                         b_3 t^2 + b_2 t + b_1
                       \end{pmatrix}                                                                                 \\
  \begin{pmatrix}
    2 a_3 t + a_2 \\
    2 b_3 t + b_2
  \end{pmatrix} & = \begin{pmatrix}
                      1 & 3 \\
                      3 & 1
                    \end{pmatrix} \begin{pmatrix}
                                    a_3 t^2 + a_2 t + a_1 \\
                                    b_3 t^2 + b_2 t + b_1
                                  \end{pmatrix} + \begin{pmatrix}
                                                    -2 t^2 \\
                                                    t + 5
                                                  \end{pmatrix}                                                             \\
                   & = \begin{pmatrix}
                         (a_3 + 3 b_3 - 2) t^2 + (a_2 + 3 b_2) t + (a_1 + 3 b_1) \\
                         (3 a_3 + b_3) t^2 + (3 a_2 + b_2 + 1) t + (3 a_1 + b_1 + 5)
                       \end{pmatrix}                                            \\
  \begin{pmatrix}
    0 \\
    0
  \end{pmatrix}  & = \begin{pmatrix}
                       (a_3 + 3 b_3 - 2) t^2 + (a_2 - 2 a_3 + 3 b_2) t + (a_1 - a_2 + 3 b_1) \\
                       (3 a_3 + b_3) t^2 + (3 a_2 + b_2 - 2 b_3 + 1) t + (3 a_1 + b_1 - b_2 + 5)
                     \end{pmatrix}                                \\
  a_3              & = -\frac{1}{4}                                                                                          \\
  b_3              & = \frac{3}{4}                                                                                           \\
  a_2              & = \frac{1}{4}                                                                                           \\
  b_2              & = -\frac{1}{4}                                                                                          \\
  a_1              & = -2                                                                                                    \\
  b_1              & = \frac{3}{4}                                                                                           \\
  \mathbf{X}_p     & = \begin{pmatrix}
                         -\frac{1}{4} t^2 + \frac{1}{4} t - 2 \\
                         \frac{3}{4} t^2 - \frac{1}{4} t + \frac{3}{4}
                       \end{pmatrix}                                                          \\
  \mathbf{X}       & = c_1 \begin{pmatrix}
                             1 \\
                             1
                           \end{pmatrix} e^{4 t} + c_2 \begin{pmatrix}
                                                         -1 \\
                                                         1
                                                       \end{pmatrix} e^{-2 t} + \begin{pmatrix}
                                                                                  -\frac{1}{4} t^2 + \frac{1}{4} t - 2 \\
                                                                                  \frac{3}{4} t^2 - \frac{1}{4} t + \frac{3}{4}
                                                                                \end{pmatrix}
\end{align*}

\setcounter{subsubsection}{4}
\subsubsection{}

\begin{align*}
  \mathbf{X}_c      & = c_1 \begin{pmatrix}
                              \frac{1}{9} \\
                              1
                            \end{pmatrix} e^{7 t} + c_2 \begin{pmatrix}
                                                          -\frac{1}{3} \\
                                                          1
                                                        \end{pmatrix} e^{3 t}                     \\
  \mathbf{X}_p      & = \begin{pmatrix}
                          a \\
                          b
                        \end{pmatrix} e^t                                                         \\
  \begin{pmatrix}
    a \\
    b
  \end{pmatrix} e^t & = \begin{pmatrix}
                          4 & \frac{1}{3} \\
                          9 & 6
                        \end{pmatrix} \begin{pmatrix}
                                        a \\
                                        b
                                      \end{pmatrix} e^t + \begin{pmatrix}
                                                            -3 \\
                                                            10
                                                          \end{pmatrix} e^t                       \\
  \begin{pmatrix}
    a \\
    b
  \end{pmatrix}   & = \begin{pmatrix}
                        4 a + \frac{1}{3} b - 3 \\
                        9 a + 6 b + 10
                      \end{pmatrix}                                                     \\
  \begin{pmatrix}
    0 \\
    0
  \end{pmatrix}   & = \begin{pmatrix}
                        3 a + \frac{1}{3} b - 3 \\
                        9 a + 5 b + 10
                      \end{pmatrix}                                                     \\
  \mathbf{X}_p      & = \begin{pmatrix}
                          \frac{55}{36} \\
                          -\frac{19}{4}
                        \end{pmatrix} e^t                                                         \\
  \mathbf{X}        & = c_1 \begin{pmatrix}
                              \frac{1}{9} \\
                              1
                            \end{pmatrix} e^{7 t} + c_2 \begin{pmatrix}
                                                          -\frac{1}{3} \\
                                                          1
                                                        \end{pmatrix} e^{3 t} + \begin{pmatrix}
                                                                                  \frac{55}{36} \\
                                                                                  -\frac{19}{4}
                                                                                \end{pmatrix} e^t
\end{align*}

\setcounter{subsubsection}{8}
\subsubsection{}

\begin{align*}
  \mathbf{X}_c    & = c_1 \begin{pmatrix}
                            -2 \\
                            3
                          \end{pmatrix} e^{2 t} + c_2 \begin{pmatrix}
                                                        -1 \\
                                                        1
                                                      \end{pmatrix} e^t                  \\
  \mathbf{X}_p    & = \begin{pmatrix}
                        a \\
                        b
                      \end{pmatrix}                                                     \\
  \begin{pmatrix}
    0 \\
    0
  \end{pmatrix} & = \begin{pmatrix}
                      -1 & -2 \\
                      3  & 4
                    \end{pmatrix} \begin{pmatrix}
                                    a \\
                                    b
                                  \end{pmatrix} + \begin{pmatrix}
                                                    3 \\
                                                    3
                                                  \end{pmatrix}                         \\
                  & = \begin{pmatrix}
                        -a - 2 b + 3 \\
                        3 a + 4 b + 3
                      \end{pmatrix}                                                     \\
  \mathbf{X}_p    & = \begin{pmatrix}
                        -9 \\
                        6
                      \end{pmatrix}                                                     \\
  \mathbf{X}      & = c_1 \begin{pmatrix}
                            -2 \\
                            3
                          \end{pmatrix} e^{2 t} + c_2 \begin{pmatrix}
                                                        -1 \\
                                                        1
                                                      \end{pmatrix} e^t + \begin{pmatrix}
                                                                            -9 \\
                                                                            6
                                                                          \end{pmatrix} \\
  \begin{pmatrix}
    -4 \\
    5
  \end{pmatrix} & = c_1 \begin{pmatrix}
                          -2 \\
                          3
                        \end{pmatrix} + c_2 \begin{pmatrix}
                                              -1 \\
                                              1
                                            \end{pmatrix} + \begin{pmatrix}
                                                              -9 \\
                                                              6
                                                            \end{pmatrix}               \\
  \begin{pmatrix}
    0 \\
    0
  \end{pmatrix} & = \begin{pmatrix}
                      -2 c_1 - c_2 - 5 \\
                      3 c_1 + c_2 + 1
                    \end{pmatrix}                                                     \\
  \mathbf{X}      & = 4 \begin{pmatrix}
                          -2 \\
                          3
                        \end{pmatrix} e^{2 t} - 13 \begin{pmatrix}
                                                     -1 \\
                                                     1
                                                   \end{pmatrix} e^t + \begin{pmatrix}
                                                                         -9 \\
                                                                         6
                                                                       \end{pmatrix}    \\
\end{align*}

\setcounter{subsubsection}{10}
\subsubsection{}

\begin{enumerate}
  \item \[\begin{pmatrix}
            x_1' \\
            x_2'
          \end{pmatrix} = \begin{pmatrix}
            -\frac{3}{100} & \frac{1}{100} \\
            \frac{1}{50}   & -\frac{1}{25}
          \end{pmatrix} \begin{pmatrix}
            x_1 \\
            x_2
          \end{pmatrix} + \begin{pmatrix}
            0 \\
            1
          \end{pmatrix}\]

  \item

        \begin{align*}
          \mathbf{X}_c    & = c_1 \begin{pmatrix}
                                    -\frac{1}{2} \\
                                    1
                                  \end{pmatrix} e^{-t / 20} + c_2 \begin{pmatrix}
                                                                    1 \\
                                                                    1
                                                                  \end{pmatrix} e^{-t / 50}                                      \\
          \mathbf{X}_p    & = \begin{pmatrix}
                                a \\
                                b
                              \end{pmatrix}                                                                                     \\
          \begin{pmatrix}
            0 \\
            0
          \end{pmatrix} & = \begin{pmatrix}
                              -\frac{3}{100} & \frac{1}{100} \\
                              \frac{1}{50}   & -\frac{1}{25}
                            \end{pmatrix} \begin{pmatrix}
                                            a \\
                                            b
                                          \end{pmatrix} + \begin{pmatrix}
                                                            0 \\
                                                            1
                                                          \end{pmatrix}                                                         \\
                          & = \begin{pmatrix}
                                -\frac{3}{100} a + \frac{1}{100} b \\
                                \frac{1}{50} a - \frac{1}{25} b + 1
                              \end{pmatrix}                                                                 \\
          \mathbf{X}_p    & = \begin{pmatrix}
                                10 \\
                                30
                              \end{pmatrix}                                                                                     \\
          \mathbf{X}      & = c_1 \begin{pmatrix}
                                    -\frac{1}{2} \\
                                    1
                                  \end{pmatrix} e^{-t / 20} + c_2 \begin{pmatrix}
                                                                    1 \\
                                                                    1
                                                                  \end{pmatrix} e^{-t / 50} + \begin{pmatrix}
                                                                                                10 \\
                                                                                                30
                                                                                              \end{pmatrix}                     \\
          \begin{pmatrix}
            60 \\
            10
          \end{pmatrix} & = c_1 \begin{pmatrix}
                                  -\frac{1}{2} \\
                                  1
                                \end{pmatrix} + c_2 \begin{pmatrix}
                                                      1 \\
                                                      1
                                                    \end{pmatrix} + \begin{pmatrix}
                                                                      10 \\
                                                                      30
                                                                    \end{pmatrix}                                               \\
          \begin{pmatrix}
            0 \\
            0
          \end{pmatrix} & = \begin{pmatrix}
                              -\frac{1}{2} c_1 + c_2 - 50 \\
                              c_1 + c_2 + 20
                            \end{pmatrix}                                                                          \\
          \mathbf{X}      & = -\frac{140}{3} \begin{pmatrix}
                                               -\frac{1}{2} \\
                                               1
                                             \end{pmatrix} e^{-t / 20} + \frac{80}{3} \begin{pmatrix}
                                                                                        1 \\
                                                                                        1
                                                                                      \end{pmatrix} e^{-t / 50} + \begin{pmatrix}
                                                                                                                    10 \\
                                                                                                                    30
                                                                                                                  \end{pmatrix}
        \end{align*}

  \item

        \begin{align*}
          \lim_{t \rightarrow \infty} x_1(t) & = \lim_{t \rightarrow \infty} \frac{70}{3} e^{-t / 20} + \frac{80}{3} e^{-t / 50} + 10   \\
                                             & = 10                                                                                     \\
          \lim_{t \rightarrow \infty} x_1(t) & = \lim_{t \rightarrow \infty} -\frac{140}{3} e^{-t / 20} + \frac{80}{3} e^{-t / 50} + 30 \\
                                             & = 30
        \end{align*}
\end{enumerate}

\setcounter{subsubsection}{12}
\subsubsection{}

\begin{align*}
  \begin{pmatrix}
    x' \\
    y'
  \end{pmatrix} & = \begin{pmatrix}
                      3 & -3 \\
                      2 & -2
                    \end{pmatrix} \begin{pmatrix}
                                    x \\
                                    y
                                  \end{pmatrix} + \begin{pmatrix}
                                                    4 \\
                                                    -1
                                                  \end{pmatrix}                  \\
  \mathbf{X}_c    & = c_1 \begin{pmatrix}
                            3 \\
                            2
                          \end{pmatrix} e^t + c_2 \begin{pmatrix}
                                                    1 \\
                                                    1
                                                  \end{pmatrix}                  \\
  \mathbf{X}_p    & = \mathbf{\Phi}(t) \int \mathbf{\Phi}^{-1}(t) \mathbf{F} \,dt \\
                  & = \begin{pmatrix}
                        3 e^t & 1 \\
                        2 e^t & 1
                      \end{pmatrix} \int \begin{pmatrix}
                                           e^{-t} & -e^{-t} \\
                                           -2     & 3
                                         \end{pmatrix} \begin{pmatrix}
                                                         4 \\
                                                         -1
                                                       \end{pmatrix} \,dt         \\
                  & = \begin{pmatrix}
                        3 e^t & 1 \\
                        2 e^t & 1
                      \end{pmatrix} \int \begin{pmatrix}
                                           5 e^{-t} \\
                                           -11
                                         \end{pmatrix} \,dt                       \\
                  & = \begin{pmatrix}
                        3 e^t & 1 \\
                        2 e^t & 1
                      \end{pmatrix} \begin{pmatrix}
                                      -5 e^{-t} \\
                                      -11 t
                                    \end{pmatrix}                                \\
                  & = \begin{pmatrix}
                        -15 - 11 t \\
                        -10 - 11 t
                      \end{pmatrix}                                              \\
  \mathbf{X}      & = c_1 \begin{pmatrix}
                            3 \\
                            2
                          \end{pmatrix} e^t + c_1 \begin{pmatrix}
                                                    1 \\
                                                    1
                                                  \end{pmatrix} + \begin{pmatrix}
                                                                    -15 - 11 t \\
                                                                    -10 - 11 t
                                                                  \end{pmatrix}
\end{align*}

\setcounter{subsubsection}{14}
\subsubsection{}

\begin{align*}
  \mathbf{X}'  & = \begin{pmatrix}
                     3           & -5 \\
                     \frac{3}{4} & -1
                   \end{pmatrix} \mathbf{X} + \begin{pmatrix}
                                                1 \\
                                                -1
                                              \end{pmatrix} e^{t / 2}                                                \\
  \mathbf{X}_c & = c_1 \begin{pmatrix}
                         \frac{10}{3} \\
                         1
                       \end{pmatrix} e^{3 t / 2} + c_2 \begin{pmatrix}
                                                         2 \\
                                                         1
                                                       \end{pmatrix} e^{t / 2}                                       \\
  \mathbf{X}_p & = \begin{pmatrix}
                     \frac{1}{2} (-15 - 13 t) \\
                     \frac{1}{4} (-9 - 13 t)
                   \end{pmatrix} e^{t / 2}                                                                          \\
  \mathbf{X}   & = c_1 \begin{pmatrix}
                         \frac{10}{3} \\
                         1
                       \end{pmatrix} e^{3 t / 2} + c_2 \begin{pmatrix}
                                                         2 \\
                                                         1
                                                       \end{pmatrix} e^{t / 2} + \frac{1}{4} \begin{pmatrix}
                                                                                               -30 - 26 t \\
                                                                                               -9 - 13 t
                                                                                             \end{pmatrix} e^{t / 2}
\end{align*}

\setcounter{subsubsection}{32}
\subsubsection{}

\begin{align*}
  \mathbf{X}_c    & = c_1 \begin{pmatrix}
                            -1 \\
                            1
                          \end{pmatrix} e^{4 t} + c_2 \begin{pmatrix}
                                                        1 \\
                                                        1
                                                      \end{pmatrix} e^{2 t}                                                                             \\
  \mathbf{X}_p    & = \begin{pmatrix}
                        2 e^{2 t} t - 2 e^{4 t} t - e^{2 t} + e^{4 t} \\
                        2 e^{2 t} t + 2 e^{4 t} t + e^{2 t} + e^{4 t} \\
                      \end{pmatrix}                                                                                     \\
                  & = \begin{pmatrix}
                        -2 \\
                        2
                      \end{pmatrix} t e^{4 t} + \begin{pmatrix}
                                                  1 \\
                                                  1
                                                \end{pmatrix} e^{4 t} + \begin{pmatrix}
                                                                          2 \\
                                                                          2
                                                                        \end{pmatrix} t e^{2 t} + \begin{pmatrix}
                                                                                                    -1 \\
                                                                                                    1
                                                                                                  \end{pmatrix} e^{2 t}                                 \\
  \mathbf{X}      & = c_1 \begin{pmatrix}
                            -1 \\
                            1
                          \end{pmatrix} e^{4 t} + c_2 \begin{pmatrix}
                                                        1 \\
                                                        1
                                                      \end{pmatrix} e^{2 t} + \begin{pmatrix}
                                                                                -2 \\
                                                                                2
                                                                              \end{pmatrix} t e^{4 t} + \begin{pmatrix}
                                                                                                          1 \\
                                                                                                          1
                                                                                                        \end{pmatrix} e^{4 t} + \begin{pmatrix}
                                                                                                                                  2 \\
                                                                                                                                  2
                                                                                                                                \end{pmatrix} t e^{2 t} \\
                  & \qquad + \begin{pmatrix}
                               -1 \\
                               1
                             \end{pmatrix} e^{2 t}                                                                                                      \\
  \begin{pmatrix}
    1 \\
    1
  \end{pmatrix} & = c_1 \begin{pmatrix}
                          -1 \\
                          1
                        \end{pmatrix} + c_2 \begin{pmatrix}
                                              1 \\
                                              1
                                            \end{pmatrix} + \begin{pmatrix}
                                                              1 \\
                                                              1
                                                            \end{pmatrix} + \begin{pmatrix}
                                                                              -1 \\
                                                                              1
                                                                            \end{pmatrix}                                                              \\
  \begin{pmatrix}
    0 \\
    0
  \end{pmatrix} & = \begin{pmatrix}
                      -c_1 + c_2 - 1 \\
                      c_1 + c_2 + 1
                    \end{pmatrix}                                                                                                                      \\
  \mathbf{X}      & = \begin{pmatrix}
                        2 \\
                        0
                      \end{pmatrix} e^{4 t} + \begin{pmatrix}
                                                -2 \\
                                                2
                                              \end{pmatrix} t e^{4 t} + \begin{pmatrix}
                                                                          2 \\
                                                                          2
                                                                        \end{pmatrix} t e^{2 t} + \begin{pmatrix}
                                                                                                    -1 \\
                                                                                                    1
                                                                                                  \end{pmatrix} e^{2 t}
\end{align*}

\setcounter{subsubsection}{34}
\subsubsection{}

\begin{align*}
  \mathbf{X}_c    & = c_1 \begin{pmatrix}
                            -3 \\
                            1
                          \end{pmatrix} e^{-12 t} + c_2 \begin{pmatrix}
                                                          1 \\
                                                          3
                                                        \end{pmatrix} e^{-2 t}                                                                                 \\
  \mathbf{X}_p    & = \frac{1}{29} \begin{pmatrix}
                                     -76 \cos t + 332 \sin t \\
                                     -168 \cos t + 276 \sin t
                                   \end{pmatrix}                                                                                                     \\
  \mathbf{X}      & = c_1 \begin{pmatrix}
                            -3 \\
                            1
                          \end{pmatrix} e^{-12 t} + c_2 \begin{pmatrix}
                                                          1 \\
                                                          3
                                                        \end{pmatrix} e^{-2 t} - \frac{4}{29} \begin{pmatrix}
                                                                                                19 \\
                                                                                                42
                                                                                              \end{pmatrix} \cos t + \frac{4}{29} \begin{pmatrix}
                                                                                                                                    83 \\
                                                                                                                                    69
                                                                                                                                  \end{pmatrix} \sin t         \\
  \begin{pmatrix}
    0 \\
    0
  \end{pmatrix} & = c_1 \begin{pmatrix}
                          -3 \\
                          1
                        \end{pmatrix} + c_2 \begin{pmatrix}
                                              1 \\
                                              3
                                            \end{pmatrix} - \frac{4}{29} \begin{pmatrix}
                                                                           19 \\
                                                                           42
                                                                         \end{pmatrix}                                                                        \\
                  & = \begin{pmatrix}
                        -3 c_1 + c_2 - \frac{76}{29} \\
                        c_1 + 3 c_2 - \frac{168}{29}
                      \end{pmatrix}                                                                                                             \\
  \mathbf{X}      & = -\frac{6}{29} \begin{pmatrix}
                                      -3 \\
                                      1
                                    \end{pmatrix} e^{-12 t} + 2 \begin{pmatrix}
                                                                  1 \\
                                                                  3
                                                                \end{pmatrix} e^{-2 t} - \frac{4}{29} \begin{pmatrix}
                                                                                                        19 \\
                                                                                                        42
                                                                                                      \end{pmatrix} \cos t + \frac{4}{29} \begin{pmatrix}
                                                                                                                                            83 \\
                                                                                                                                            69
                                                                                                                                          \end{pmatrix} \sin t
\end{align*}

\setcounter{subsubsection}{36}
\subsubsection{}

\begin{align*}
  \mathbf{A}      & = \begin{pmatrix}
                        5  & -2 \\
                        21 & -8
                      \end{pmatrix}                                                         \\
  \mathbf{F}      & = \begin{pmatrix}
                        6 \\
                        4
                      \end{pmatrix}                                                         \\
  \mathbf{P}      & = \begin{pmatrix}
                        2 & 1 \\
                        7 & 3
                      \end{pmatrix}                                                         \\
  \mathbf{G}      & = \begin{pmatrix}
                        -14 \\
                        34
                      \end{pmatrix}                                                         \\
  \begin{pmatrix}
    y_1' \\
    y_2'
  \end{pmatrix} & = \begin{pmatrix}
                      -2 y_1 - 14 \\
                      -y_2 + 34
                    \end{pmatrix}                                                           \\
  \begin{pmatrix}
    y_1 \\
    y_2
  \end{pmatrix} & = \begin{pmatrix}
                      c_1 e^{-2 t} - 7 \\
                      c_2 e^{-t} + 34
                    \end{pmatrix}                                                         \\
  \mathbf{X}      & = c_1 \begin{pmatrix}
                            2 \\
                            7
                          \end{pmatrix} e^{-2 t} + c_2 \begin{pmatrix}
                                                         1 \\
                                                         3
                                                       \end{pmatrix} e^{-t} + \begin{pmatrix}
                                                                                20 \\
                                                                                53
                                                                              \end{pmatrix}
\end{align*}

\setcounter{subsubsection}{38}
\subsubsection{}

\begin{align*}
  \mathbf{A}  & = \begin{pmatrix}
                    5 & 5 \\
                    5 & 5
                  \end{pmatrix}                                                                                                                   \\
  \mathbf{P}  & = \begin{pmatrix}
                    1 & -1 \\
                    1 & 1
                  \end{pmatrix}                                                                                                                   \\
  \mathbf{G}  & = \begin{pmatrix}
                    4 + t \\
                    4 - t
                  \end{pmatrix}                                                                                                                   \\
  \mathbf{Y}' & = \mathbf{D Y + G}                                                                                                                 \\
              & = \begin{pmatrix}
                    10 y_1 + 4 + t \\
                    4 - t
                  \end{pmatrix}                                                                                                                   \\
  \mathbf{Y}  & = \begin{pmatrix}
                    c_1 e^{10 t} - \frac{1}{10} t - \frac{41}{100} \\
                    -\frac{1}{2} t^2 + 4t + c_2
                  \end{pmatrix}                                                                                   \\
  \mathbf{X}  & = \mathbf{P Y}                                                                                                                     \\
              & = c_1 \begin{pmatrix}
                        1 \\
                        1
                      \end{pmatrix} e^{10 t} + c_2 \begin{pmatrix}
                                                     -1 \\
                                                     1
                                                   \end{pmatrix} + \frac{1}{2} \begin{pmatrix}
                                                                                 1 \\
                                                                                 -1
                                                                               \end{pmatrix} t^2 + \begin{pmatrix}
                                                                                                     -\frac{41}{10} \\
                                                                                                     \frac{39}{10}
                                                                                                   \end{pmatrix} t - \frac{41}{100} \begin{pmatrix}
                                                                                                                                      1 \\
                                                                                                                                      1
                                                                                                                                    \end{pmatrix}
\end{align*}

\subsection{Matrix Exponential}

\subsubsection{}

\[\begin{pmatrix}
    e^t & 0       \\
    0   & e^{2 t}
  \end{pmatrix}\]

\[\begin{pmatrix}
    e^{-t} & 0        \\
    0      & e^{-2 t}
  \end{pmatrix}\]

\setcounter{subsubsection}{2}
\subsubsection{}

\[\begin{pmatrix}
    t + 1 & t     & t        \\
    t     & t + 1 & t        \\
    -2 t  & -2 t  & -2 t + 1
  \end{pmatrix}\]

\setcounter{subsubsection}{4}
\subsubsection{}

\[\begin{pmatrix}
    c_1 e^t \\
    c_2 e^{2 t}
  \end{pmatrix}\]

\setcounter{subsubsection}{6}
\subsubsection{}

\[\mathbf{X} = c_1 \begin{pmatrix}
    t + 1 \\
    t     \\
    -2 t
  \end{pmatrix} + c_2 \begin{pmatrix}
    t     \\
    t + 1 \\
    -2 t
  \end{pmatrix} + c_3 \begin{pmatrix}
    t \\
    t \\
    -2 t + 1
  \end{pmatrix}\]

\setcounter{subsubsection}{8}
\subsubsection{}

\[\mathbf{X} = c_1 \begin{pmatrix}
    1 \\
    0
  \end{pmatrix} e^t + c_2 \begin{pmatrix}
    0 \\
    1
  \end{pmatrix} e^{2 t} + \begin{pmatrix}
    -3 \\
    \frac{1}{2}
  \end{pmatrix}\]

\setcounter{subsubsection}{10}
\subsubsection{}

\[\mathbf{X} = c_1 \begin{pmatrix}
    \cosh t \\
    \sinh t
  \end{pmatrix} + c_2 \begin{pmatrix}
    \sinh t \\
    \cosh t
  \end{pmatrix} - \begin{pmatrix}
    1 \\
    1
  \end{pmatrix}\]

\setcounter{subsubsection}{12}
\subsubsection{}

\begin{align*}
  \mathbf{X}      & = c_1 \begin{pmatrix}
                            t + 1 \\
                            t     \\
                            -2 t
                          \end{pmatrix} + c_2 \begin{pmatrix}
                                                t     \\
                                                t + 1 \\
                                                -2 t
                                              \end{pmatrix} + c_3 \begin{pmatrix}
                                                                    t \\
                                                                    t \\
                                                                    -2 t + 1
                                                                  \end{pmatrix} \\
  \begin{pmatrix}
    1  \\
    -4 \\
    6
  \end{pmatrix} & = c_1 \begin{pmatrix}
                          1 \\
                          0 \\
                          0
                        \end{pmatrix} + c_2 \begin{pmatrix}
                                              0 \\
                                              1 \\
                                              0
                                            \end{pmatrix} + c_3 \begin{pmatrix}
                                                                  0 \\
                                                                  0 \\
                                                                  1
                                                                \end{pmatrix}   \\
  \mathbf{X}      & = \begin{pmatrix}
                        t + 1 \\
                        t     \\
                        -2 t
                      \end{pmatrix} - 4 \begin{pmatrix}
                                          t     \\
                                          t + 1 \\
                                          -2 t
                                        \end{pmatrix} + 6 \begin{pmatrix}
                                                            t \\
                                                            t \\
                                                            -2 t + 1
                                                          \end{pmatrix}         \\
\end{align*}

\setcounter{subsubsection}{14}
\subsubsection{}

\[e^{\mathbf{A} t} = \begin{pmatrix}
    \frac{1}{2} e^{-2 t} (3 e^{4 t} - 1) & \frac{3}{4} e^{-2 t} (e^{4 t} - 1)  \\
    e^{-2 t} - e^{2 t}                   & -\frac{1}{2} e^{-2 t} (e^{4 t} - 3) \\
  \end{pmatrix}\]

\setcounter{subsubsection}{16}
\subsubsection{}

\[e^{\mathbf{A} t} = \begin{pmatrix}
    e^{2 t} (1 + 3 t) & -9 e^{2 t} t      \\
    e^{2 t} t         & e^{2 t} (1 - 3 t)
  \end{pmatrix}\]

\setcounter{subsubsection}{24}
\subsubsection{}

\begin{align*}
  \mathbf{X} & = e^{\mathbf{A} t} \mathbf{C}                                                \\
             & = \mathbf{P} e^{\mathbf{D} t} \mathbf{P}^{-1} \mathbf{C}                     \\
             & = \begin{pmatrix}
                   1 & 1 \\
                   3 & 1
                 \end{pmatrix} \begin{pmatrix}
                                 e^{5 t} & 0       \\
                                 0       & e^{3 t}
                               \end{pmatrix} \begin{pmatrix}
                                               -\frac{1}{2} & \frac{1}{2}  \\
                                               \frac{3}{2}  & -\frac{1}{2}
                                             \end{pmatrix} \mathbf{C}                    \\
             & = \begin{pmatrix}
                   -\frac{1}{2} e^{3 t} (-3 + e^{2 t}) & \frac{1}{2} e^{3 t} (-1 + e^{2 t})   \\
                   -\frac{3}{2} e^{3 t} (-1 + e^{2 t}) & \frac{1}{2} e^{3 t} (-1 + 3 e^{2 t})
                 \end{pmatrix} \mathbf{C}
\end{align*}

\subsection{Chapter in Review}

\subsubsection{}

$\frac{1}{3}$

\setcounter{subsubsection}{4}
\subsubsection{}

\[\mathbf{X} = c_1 \begin{pmatrix}
    -1 \\
    1
  \end{pmatrix} e^t + c_2 \left[ \begin{pmatrix}
      -1 \\
      1
    \end{pmatrix} t e^t + \begin{pmatrix}
      -1 \\
      0
    \end{pmatrix} e^t \right]\]

\setcounter{subsubsection}{6}
\subsubsection{}

\begin{align*}
  \mathbf{X} & = c_1 \begin{pmatrix}
                       -i \\
                       1
                     \end{pmatrix} e^{1 + 2 i} + c_2 \begin{pmatrix}
                                                       i \\
                                                       1
                                                     \end{pmatrix} e^{1 - 2 i}                                                                                   \\
             & = c_3 \left[ \begin{pmatrix}
                                0 \\
                                1
                              \end{pmatrix} \cos 2 t - \begin{pmatrix}
                                                         -1 \\
                                                         0
                                                       \end{pmatrix} \sin 2 t \right] e^t + c_4 \left[ \begin{pmatrix}
                                                                                                         -1 \\
                                                                                                         0
                                                                                                       \end{pmatrix} \cos 2 t + \begin{pmatrix}
                                                                                                                                  0 \\
                                                                                                                                  1
                                                                                                                                \end{pmatrix} \sin 2 t \right] e^t \\
             & = \begin{pmatrix}
                   c_3 e^t \sin 2 t - c_4 e^t \cos 2 t \\
                   c_3 e^t \cos 2 t + c_4 e^t \sin 2 t
                 \end{pmatrix}                                                                                                             \\
             & = c_3 \begin{pmatrix}
                       \sin 2 t \\
                       \cos 2 t
                     \end{pmatrix} e^t + c_4 \begin{pmatrix}
                                               -\cos 2 t \\
                                               \sin 2 t
                                             \end{pmatrix} e^t
\end{align*}

\setcounter{subsubsection}{8}
\subsubsection{}

\[\mathbf{X} = c_1 \begin{pmatrix}
    0 \\
    1 \\
    1
  \end{pmatrix} e^{4 t} + c_2 \begin{pmatrix}
    -7  \\
    -12 \\
    16
  \end{pmatrix} e^{-3 t} + c_3 \begin{pmatrix}
    -2 \\
    3  \\
    1
  \end{pmatrix} e^{2 t}\]

\section{Systems of Nonlinear Differential Equations}

\subsection{Autonomous Systems}

\subsubsection{}

\begin{align*}
  x' & = y         \\
  y' & = -9 \sin x
\end{align*}

Critical points at $(n \pi, 0), n \in \mathbb{Z}$.

\setcounter{subsubsection}{2}
\subsubsection{}

\begin{align*}
  x' & = y                 \\
  y' & = x^2 - y (1 - x^3) \\ \\
  0  & = y                 \\
  0  & = x^2 - y (1 - x^3) \\
     & = x^2               \\
  0  & = x
\end{align*}

Critical point at $(0, 0)$.

\setcounter{subsubsection}{4}
\subsubsection{}

\begin{align*}
  x' & = y                         \\
  y' & = \epsilon x^3 - x          \\ \\
  0  & = y                         \\
  0  & = \epsilon x^3 - x          \\
     & = \epsilon x^2 - 1          \\
  x  & = \sqrt{\frac{1}{\epsilon}}
\end{align*}

Critical points at $(0, 0)$ and $\left( \pm \sqrt{\frac{1}{\epsilon}}, 0 \right)$.

\setcounter{subsubsection}{6}
\subsubsection{}

$x' = x + x y$ can only be $0$ if $x = 0$ or $y = -1$. If $x = 0$, $y' = -y - x y$ is $0$ if $y = 0$. If $y = -1$, it's $0$ if $x = -1$. Therefore the critical points are $(0, 0)$ and $(-1, -1)$.

\setcounter{subsubsection}{8}
\subsubsection{}

\begin{align*}
  x'  & = 3 x^2 - 4 y              \\
  3 x^2 = 4 y                      \\
  x   & = \sqrt{\frac{4}{3} y}     \\ \\
  y'  & = x - y                    \\
  0   & = \sqrt{\frac{4}{3} y} - y \\
  y^2 & = \frac{4}{3} y            \\
  y   & = \frac{4}{3}
\end{align*}

The critical points are $(0, 0)$ and $\left( \frac{4}{3}, \frac{4}{3} \right)$.

\setcounter{subsubsection}{10}
\subsubsection{}

\begin{align*}
  x' & = x \left( 10 - x - \frac{1}{2} y \right) \\
  y' & = y (16 - y - x)
\end{align*}

$(0, 0)$, $(0, 16)$, $(10, 0)$, $(4, 12)$

\setcounter{subsubsection}{12}
\subsubsection{}

\begin{align*}
  x' & = x^2 e^y     \\
  y' & = y (e^x - 1)
\end{align*}

All points on the line $x = 0$.

\setcounter{subsubsection}{14}
\subsubsection{}

\begin{align*}
  x' & = x (1 - x^2 - 3 y^2) \\
  y' & = y (3 - x^2 - 3 y^2) \\ \\
\end{align*}

$(0, 0)$, $(0, \pm 1)$, $(\pm 1, 0)$

\setcounter{subsubsection}{16}
\subsubsection{}

\begin{enumerate}
  \item \[\mathbf{X} = c_1 \begin{pmatrix}
            1 \\
            2
          \end{pmatrix} e^{5 t} + c_2 \begin{pmatrix}
            -1 \\
            1
          \end{pmatrix} e^{-t}\]

  \item

        \begin{align*}
          \begin{pmatrix}
            2 \\
            -2
          \end{pmatrix} & = c_1 \begin{pmatrix}
                                  1 \\
                                  2
                                \end{pmatrix} + c_2 \begin{pmatrix}
                                                      -1 \\
                                                      1
                                                    \end{pmatrix} \\
          \begin{pmatrix}
            0 \\
            0
          \end{pmatrix} & = \begin{pmatrix}
                              c_1 - c_2 - 2 \\
                              2 c_1 + c_2 + 2
                            \end{pmatrix}                         \\
          \mathbf{X}      & = -2 \begin{pmatrix}
                                   -1 \\
                                   1
                                 \end{pmatrix} e^{-t}
        \end{align*}
\end{enumerate}

\setcounter{subsubsection}{18}
\subsubsection{}

\begin{enumerate}
  \item

        \begin{align*}
          x & = c_1 (4 \cos 3 t - 3 \sin 3 t) + c_2 (3 \cos 3 t + 4 \sin 3 t) \\
          y & = c_1 (5 \cos 3 t) + c_2 (5 \sin 3 t)
        \end{align*}

  \item

        \begin{align*}
          4   & = 4 c_1 + 3 c_2           \\
          5   & = 5 c_1                   \\
          c_1 & = 1                       \\
          c_2 & = 0                       \\ \\
          x   & = 4 \cos 3 t - 3 \sin 3 t \\
          y   & = 5 \cos 3 t
        \end{align*}
\end{enumerate}

\setcounter{subsubsection}{20}
\subsubsection{}

\begin{enumerate}
  \item

        \begin{align*}
          x & = c_1 (-\cos t + \sin t) e^{4 t} + c_2 (-\cos t - \sin t) e^{4 t} \\
          y & = c_1 (2 \cos t) e^{4 t} + c_2 (2 \sin t) e^{4 t}
        \end{align*}

  \item

        \begin{align*}
          -1  & = -c_1 - c_2                \\
          2   & = 2 c_1                     \\
          c_1 & = 1                         \\
          c_2 & = 0                         \\ \\
          x   & = (\sin t - \cos t) e^{4 t} \\
          y   & = 2 \cos t e^{4 t}
        \end{align*}
\end{enumerate}

\setcounter{subsubsection}{22}
\subsubsection{}

\begin{align*}
  \frac{d r}{d t}               & = \frac{1}{r} \{ x [-y - x (x^2 + y^2)^2] + y [x - y (x^2 + y^2)^2] \}    \\
                                & = \frac{1}{r} [-x y - x^2 r^4 + x y - y^2 r^4]                            \\
                                & = -r^5                                                                    \\
  \frac{1}{r^5} \frac{d r}{d t} & = -1                                                                      \\
  -\frac{1}{4} \frac{1}{r^4}    & = c_1 - t                                                                 \\
  \frac{1}{r^4}                 & = 4 t + c_1                                                               \\
  r                             & = \frac{1}{\sqrt[4]{4 t + c_1}}                                           \\ \\
  \frac{d \theta}{d t}          & = \frac{1}{r^2} \{ -y [-y - x (x^2 + y^2)^2] + x [x - y (x^2 + y^2)^2] \} \\
                                & = \frac{1}{r^2} [y^2 + x y r^2 + x^2 - x y r^2]                           \\
                                & = 1                                                                       \\
  \theta                        & = t + c_2                                                                 \\ \\
  4                             & = \frac{1}{\sqrt[4]{c_1}}                                                 \\
  c_1                           & = \frac{1}{256}                                                           \\ \\
  0                             & = c_2                                                                     \\ \\
  r                             & = \frac{1}{\sqrt[4]{4 t + \frac{1}{256}}}                                 \\
                                & = \frac{4}{\sqrt[4]{1024 t + 1}}                                          \\
  \theta                        & = t
\end{align*}

\setcounter{subsubsection}{24}
\subsubsection{}

\begin{align*}
  \frac{d r}{d t}      & = \frac{1}{r} \{ x [-y + x (1 - x^2 - y^2)] + y [x + y (1 - x^2 - y^2)] \}    \\
                       & = \frac{1}{r} [-x y + x^2 (1 - r^2) + x y + y^2 (1 - r^2)]                    \\
                       & = r (1 - r^2)                                                                 \\
                       & = -r^3 + r                                                                    \\
  r                    & = \pm \frac{e^t}{\sqrt{e^{2 t} + c_1}}                                        \\ \\
  \frac{d \theta}{d t} & = \frac{1}{r^2} \{ -y [-y + x (1 - x^2 - y^2)] + x [x + y (1 - x^2 - y^2)] \} \\
                       & = \frac{1}{r^2} [y^2 - x y (1 - r^2) + x^2 + x y (1 - r^2)]                   \\
                       & = 1                                                                           \\
  \theta               & = t + c_2                                                                     \\ \\
  1                    & = \pm \frac{1}{\sqrt{1 + c_1}}                                                \\
  c_1                  & = 0                                                                           \\
  c_2                  & = 0                                                                           \\
  r                    & = 1                                                                           \\
  \theta               & = t                                                                           \\ \\
  2                    & = \pm \frac{1}{\sqrt{1 + c_1}}                                                \\
  c_1                  & = -\frac{3}{4}                                                                \\
  c_2                  & = 0                                                                           \\
  r                    & = \frac{e^t}{\sqrt{e^{2 t} - \frac{3}{4}}}                                    \\
  \theta               & = t
\end{align*}

\setcounter{subsubsection}{26}
\subsubsection{}

No periodic solutions.

\subsection{Stability of Linear Systems}

\subsubsection{}

Stable node

\setcounter{subsubsection}{2}
\subsubsection{}

Unstable spiral

\setcounter{subsubsection}{4}
\subsubsection{}

Degenerate stable node

\setcounter{subsubsection}{6}
\subsubsection{}

Saddle point

\setcounter{subsubsection}{8}
\subsubsection{}

Saddle point

\setcounter{subsubsection}{10}
\subsubsection{}

Saddle point

\setcounter{subsubsection}{12}
\subsubsection{}

Degenerate stable node

\setcounter{subsubsection}{14}
\subsubsection{}

Stable spiral

\setcounter{subsubsection}{16}
\subsubsection{}

\[-1 + \mu^2 < 0 \Rightarrow |\mu| < 1\]

\setcounter{subsubsection}{18}
\subsubsection{}

Saddle point when $\mu < -1$, unstable spiral when $-1 < \mu < 3$, unstable node when $\mu \ge 3$.

\setcounter{subsubsection}{22}
\subsubsection{}

\begin{enumerate}
  \item $(3, 4)$

  \item Saddle point
\end{enumerate}

\setcounter{subsubsection}{24}
\subsubsection{}

\begin{enumerate}
  \item $(0.5, 2)$

  \item Unstable spiral
\end{enumerate}

\subsection{Linearization and Local Stability}

\subsubsection{}

The Jacobian is \[\begin{pmatrix}
    \alpha    & -\beta + 2 y \\
    \beta - y & \alpha - x
  \end{pmatrix}.\]

At $(0, 0)$ this is \[\begin{pmatrix}
    \alpha & -\beta \\
    \beta  & \alpha
  \end{pmatrix}.\]

The eigenvalues of this matrix are $\alpha \pm i \beta$ so if $\alpha > 0$ then $(0, 0)$ is an unstable critical point and if $\alpha < 0$ it is a stable critical point.

\setcounter{subsubsection}{2}
\subsubsection{}

\begin{align*}
  g(x)      & = k x (n + 1 - x)       \\
  g'(x)     & = k (n + 1 - x) - k x   \\
            & = k (n + 1 - 2 x)       \\
  g'(0)     & = k (n + 1)             \\
  g'(n + 1) & = k (n + 1 - 2 (n + 1)) \\
            & = -k (n + 1)
\end{align*}

$x = 0$ is unstable, $x = n + 1$ is stable.

\setcounter{subsubsection}{4}
\subsubsection{}

\begin{align*}
  g(T)    & = k (T - T_0) \\
  g'(T)   & = k           \\
  g'(T_0) & = k
\end{align*}

$T = T_0$ is unstable.

\setcounter{subsubsection}{6}
\subsubsection{}

\begin{align*}
  g(x)       & = k (\alpha - x) (\beta - x), \alpha > \beta \\
  g'(x)      & = -k (\beta - x) - k (\alpha - x)            \\
  g'(\alpha) & = -k (\beta - \alpha) - k (\alpha - \alpha)  \\
             & = k (\alpha - \beta)                         \\
  g'(\beta)  & = -k (\beta - \beta) - k (\alpha - \beta)    \\
             & = -k (\alpha - \beta)
\end{align*}

$x = \alpha$ is unstable, $x = \beta$ is stable.

\setcounter{subsubsection}{8}
\subsubsection{}

\begin{align*}
  g(P)                         & = P (a - b P) (1 - c P^{-1}), P > 0, a < b c                                                                                                                          \\
  g'(P)                        & = (a - b P) (1 - c P^{-1}) - b P (1 - c P^{-1}) + c P^{-1} (a - b P)                                                                                                  \\
  g'\left( \frac{a}{b} \right) & = \left( a - b \frac{a}{b} \right) \left( 1 - c \frac{b}{a} \right) - b \frac{a}{b} \left( 1 - c \frac{b}{a} \right) + c \frac{b}{a} \left( a - b \frac{a}{b} \right) \\
                               & = -a \left( 1 - \frac{b c}{a} \right)                                                                                                                                 \\
  g'(c)                        & = (a - b c) \left( 1 - c \frac{1}{c} \right) - b c \left( 1 - c \frac{1}{c} \right) + c \frac{1}{c} (a - b c)                                                         \\
                               & = a - b c
\end{align*}

$P = \frac{a}{b}$ is unstable, $P = c$ is stable.

\setcounter{subsubsection}{10}
\subsubsection{}

\begin{align*}
  x'                                                        & = 1 - 2 x y      \\
  y'                                                        & = 2 x y - y      \\ \\
  0                                                         & = 1 - 2 x y      \\
  2 x y                                                     & = 1              \\ \\
  0                                                         & = 1 - y          \\
  y                                                         & = 1              \\
  x                                                         & = \frac{1}{2}    \\ \\
  \mathbf{A}                                                & = \begin{pmatrix}
                                                                  -2 y & -2 x    \\
                                                                  2 y  & 2 x - 1
                                                                \end{pmatrix} \\
  \left. \mathbf{A} \right|_{\left( \frac{1}{2}, 1 \right)} & = \begin{pmatrix}
                                                                  -2 & -1 \\
                                                                  2  & 0
                                                                \end{pmatrix}
\end{align*}

$\left( \frac{1}{2}, 1 \right)$ is a stable spiral.

\setcounter{subsubsection}{12}
\subsubsection{}

\begin{align*}
  x' & = y - x^2 + 2                             \\
  y' & = 2 x y - y                               \\ \\
  0  & = y - x^2 + 2                             \\
  y  & = x^2 - 2                                 \\ \\
  0  & = 2 x (x^2 - 2) - (x^2 - 2)               \\
     & = 2 x^3 - x^2 - 4 x + 2                   \\
     & = (x^2 - 2) (2 x - 1)                     \\
     & = (x + \sqrt{2}) (x - \sqrt{2}) (2 x - 1) \\
  x  & = \frac{1}{2} \text{ or } \pm \sqrt{2}    \\
  y  & = -\frac{7}{4} \text{ or } 0
\end{align*}

Critical points are $\left( \frac{1}{2}, -\frac{7}{4} \right)$ and $(\pm \sqrt{2}, 0)$.

\begin{align*}
  \mathbf{A}                                                           & = \begin{pmatrix}
                                                                             -2 x & 1       \\
                                                                             2 y  & 2 x - 1
                                                                           \end{pmatrix}               \\
  \left. \mathbf{A} \right|_{\left( \frac{1}{2}, -\frac{7}{4} \right)} & = \begin{pmatrix}
                                                                             -1           & 1 \\
                                                                             -\frac{7}{2} & 0
                                                                           \end{pmatrix}             \\
  \left. \mathbf{A} \right|_{(-\sqrt{2}, 0)}                           & = \begin{pmatrix}
                                                                             2 \sqrt{2} & 1               \\
                                                                             0          & -2 \sqrt{2} - 1
                                                                           \end{pmatrix} \\
  \left. \mathbf{A} \right|_{(\sqrt{2}, 0)}                            & = \begin{pmatrix}
                                                                             -2 \sqrt{2} & 1              \\
                                                                             0           & 2 \sqrt{2} - 1
                                                                           \end{pmatrix}
\end{align*}

$\left( \frac{1}{2}, -\frac{7}{4} \right)$ is a stable spiral, $(\pm \sqrt{2}, 0)$ are saddle points.

\setcounter{subsubsection}{14}
\subsubsection{}

\begin{align*}
  x'  & = -3 x + y^2 + 2          \\
  y'  & = x^2 - y^2               \\ \\
  0   & = -3 x + y^2 + 2          \\
  y^2 & = 3 x - 2                 \\ \\
  0   & = x^2 - 3 x + 2           \\
      & = (x - 2) (x - 1)         \\
  x   & = 1 \text{ or } 2         \\ \\
  y   & = \pm 1 \text{ or } \pm 2
\end{align*}

Critical points are $(1, \pm 1)$ and $(2, \pm 2)$.

\begin{align*}
  \mathbf{A}                          & = \begin{pmatrix}
                                            -3  & 2 y  \\
                                            2 x & -2 y
                                          \end{pmatrix} \\
  \left. \mathbf{A} \right|_{(1, -1)} & = \begin{pmatrix}
                                            -3 & -2 \\
                                            2  & 2
                                          \end{pmatrix} \\
  \left. \mathbf{A} \right|_{(1, 1)}  & = \begin{pmatrix}
                                            -3 & 2  \\
                                            2  & -2
                                          \end{pmatrix} \\
  \left. \mathbf{A} \right|_{(2, -2)} & = \begin{pmatrix}
                                            -3 & -4 \\
                                            4  & 4
                                          \end{pmatrix} \\
  \left. \mathbf{A} \right|_{(2, 2)}  & = \begin{pmatrix}
                                            -3 & 4  \\
                                            4  & -4
                                          \end{pmatrix} \\
\end{align*}

$(1, -1)$ is a saddle point, $(1, 1)$ is a stable node, $(2, -2)$ is an unstable spiral, $(2,2)$ is a saddle point.

\setcounter{subsubsection}{22}
\subsubsection{}

It's not possible to classify $x = 0$.

\setcounter{subsubsection}{24}
\subsubsection{}

It's not possible to classify $x = 0$ but $x = \pm \sqrt{\frac{1}{\epsilon}}$ are saddle points.

\setcounter{subsubsection}{28}
\subsubsection{}

\begin{enumerate}
  \item The critical point at $(0, 0)$ is a stable spiral.
\end{enumerate}

\setcounter{subsubsection}{32}
\subsubsection{}

\begin{enumerate}
  \item

        \begin{align*}
          x' & = 2 x y              \\
          y' & = 1 - x^2 + y^2      \\ \\
          0  & = 1 - x^2 + y^2      \\
          x  & = \sqrt{y^2 + 1}     \\ \\
          0  & = 2 \sqrt{y^2 + 1} y \\
             & = 4 (y^2 + 1) y^2    \\
             & = 4 y^4 + 4 y^2      \\
          y  & = 0                  \\ \\
          x  & = \pm 1
        \end{align*}

        Critical points are $(\pm 1, 0)$.

        \begin{align*}
          \mathbf{A}                          & = \begin{pmatrix}
                                                    2 y  & 2 x \\
                                                    -2 x & 2 y
                                                  \end{pmatrix} \\
          \left. \mathbf{A} \right|_{(-1, 0)} & = \begin{pmatrix}
                                                    0 & -2 \\
                                                    2 & 0
                                                  \end{pmatrix} \\
          \left. \mathbf{A} \right|_{(1, 0)}  & = \begin{pmatrix}
                                                    0  & 2 \\
                                                    -2 & 0
                                                  \end{pmatrix}
        \end{align*}

        The trace of both the matrices is $0$ and the determinant is $4$, so we know the eigenvalues are pure imaginary but don't know the nature of the critical points.

  \item

        \begin{align*}
          \frac{d y}{d x} & = \frac{d y / d t}{d x / d t}      \\
                          & = \frac{1 - x^2 + y^2}{2 x y}      \\
          y               & = \pm \sqrt{-x^2 + c_1 x - 1}      \\
          y^2             & = -x^2 + c_1 x - 1                 \\
                          & = -x^2 + 2 c_2 x - 1               \\
                          & = -(x^2 - 2 c_2 x + c^2) + c^2 - 1 \\
                          & = -(x - c)^2 + c^2 - 1             \\
          (x - c)^2 + y^2 & = c^2 - 1
        \end{align*}
\end{enumerate}

\setcounter{subsubsection}{36}
\subsubsection{}

\begin{align*}
  L q'' + R q' + \alpha q + \beta q^3 & = 0                                         \\
  q'                                  & = r                                         \\
  r'                                  & = -\frac{1}{L} (R r + \alpha q + \beta q^3)
\end{align*}

When $\beta > 0$ the only critical point is $(q, r) = (0, 0)$. When $\beta < 0$ the critical points are $(0, 0)$ and $\left( \pm \sqrt{-\frac{\alpha}{\beta}}, 0 \right)$.

\begin{align*}
  \mathbf{A}                                                                     & = \begin{pmatrix}
                                                                                       0                                   & 1            \\
                                                                                       -\frac{1}{L} (\alpha + 3 \beta q^2) & -\frac{R}{L}
                                                                                     \end{pmatrix} \\
  \left. \mathbf{A} \right|_{(0, 0)}                                             & = \begin{pmatrix}
                                                                                       0                 & 1            \\
                                                                                       -\frac{\alpha}{L} & -\frac{R}{L}
                                                                                     \end{pmatrix}                   \\
  \left. \mathbf{A} \right|_{\left( \pm \sqrt{-\frac{\alpha}{\beta}}, 0 \right)} & = \begin{pmatrix}
                                                                                       0                  & 1            \\
                                                                                       \frac{2 \alpha}{L} & -\frac{R}{L}
                                                                                     \end{pmatrix}
\end{align*}

The eigenvalues of $\left. \mathbf{A} \right|_{(0, 0)}$ are \[\frac{-R \pm \sqrt{R^2 - 4 L \alpha}}{2 L}\] both of which are negative, so $(0, 0)$ is stable.

The eigenvalues of $\left. \mathbf{A} \right|_{\left( \pm \sqrt{-\frac{\alpha}{\beta}}, 0 \right)}$ are \[\frac{-R \pm \sqrt{R^2 + 8 L \alpha}}{2 L}\] of which one is negative and the other positive meaning $\left(\pm \sqrt{-\frac{\alpha}{\beta}}, 0 \right)$ are saddle points.

\setcounter{subsubsection}{38}
\subsubsection{}

\begin{enumerate}
  \item

        \begin{align*}
          \theta'' + \sin \theta & = \frac{1}{2}                        \\ \\
          \theta'                & = r                                  \\
          r'                     & = \frac{1}{2} - \sin \theta          \\ \\
          \theta'(\pi / 6, 0)    & = 0                                  \\
          r'(\pi / 6, 0)         & = \frac{1}{2} - \sin \frac{\pi}{6}   \\
                                 & = 0                                  \\
          \theta'(5 \pi / 6, 0)  & = 0                                  \\
          r'(5 \pi / 6, 0)       & = \frac{1}{2} - \sin \frac{5 \pi}{6} \\
                                 & = 0                                  \\
        \end{align*}

  \item

        \begin{align*}
          \mathbf{A}                                                    & = \begin{pmatrix}
                                                                              0            & 1 \\
                                                                              -\cos \theta & 0
                                                                            \end{pmatrix}                      \\
          \left. \mathbf{A} \right|_{\left( \frac{\pi}{6}, 0 \right)}   & = \begin{pmatrix}
                                                                              0                   & 1 \\
                                                                              -\frac{\sqrt{3}}{2} & 0
                                                                            \end{pmatrix} \\
          \left. \mathbf{A} \right|_{\left( \frac{5 \pi}{6}, 0 \right)} & = \begin{pmatrix}
                                                                              0                  & 1 \\
                                                                              \frac{\sqrt{3}}{2} & 0
                                                                            \end{pmatrix}
        \end{align*}

        The eigenvalues of $\left. \mathbf{A} \right|_{\left( \frac{\pi}{6}, 0 \right)}$ are $\pm \frac{3^{1 / 4}}{\sqrt{2}} i$ so $\left( \frac{\pi}{6}, 0 \right)$ is a center, a stable spiral, or an unstable spiral.

        The eigenvalues of $\left. \mathbf{A} \right|_{\left( \frac{5 \pi}{6} \right)}$ are $\pm \frac{3^{1 / 4}}{\sqrt{2}}$ so $\left( \frac{5 \pi}{6}, 0 \right)$ is a saddle point.
\end{enumerate}

\end{document}