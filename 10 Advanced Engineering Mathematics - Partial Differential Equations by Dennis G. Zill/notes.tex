\documentclass{article}
\usepackage{amsmath} % For align*
\usepackage{siunitx} % For units

\title{Advanced Engineering Mathematics Partial Differential Equations by Dennis G. Zill Notes}
\author{Chris Doble}
\date{November 2023}

\begin{document}

\maketitle

\tableofcontents

\setcounter{section}{11}
\section{Orthogonal Functions and Fourier Series}

\subsection{Orthogonal Functions}

\begin{itemize}
  \item The \textbf{inner product} of two functions $f_1$ and $f_2$ on an interval $[a, b]$ is the number \[(f_1, f_2) = \int_a^b f_1(x) f_2(x) \,d x.\]

  \item Two functions $f_1$ and $f_2$ are said to be orthogonal on an interval if $(f_1, f_2) = 0$.

  \item A set of real-valued functions $\{\phi_1(x), \phi_2(x), \ldots, \phi_n(x)\}$ is said to be \textbf{orthogonal} on an interval if \[(\phi_i, \phi_j) = 0 \text{ for } i \ne j.\]

  \item The \textbf{square norm} of a function is \[||\phi_n(x)||^2 = (\phi_n, \phi_n)\] and thus its \textbf{norm} is \[||\phi_n(x)|| = \sqrt{(\phi_n, \phi_n)}.\]

  \item An \textbf{orthonormal set} of functions is an orthogonal set of functions that all have a norm of $1$.

  \item An orthogonal set can be made into an orthonormal set by dividing each member by its norm.

  \item If $\{\phi_n(x)\}$ is an infinite orthogonal set of functions on an interval $[a, b]$ and $f(x)$ is an arbitrary function, then it's possible to determine a set of coefficients $c_n, n = 0, 1, 2, \ldots$ such that \[f(x) = \sum_{n = 0}^\infty c_n \phi_n(x) = c_0 \phi_0(x) + c_1 \phi_1(x) + \ldots + c_n \phi_n(x) + \ldots\] This is called an \textbf{orthogonal series expansion} of $f$ or a \textbf{generalized Fourier series} where the coefficients are given by \[c_n = \frac{(f, \phi_n)}{||\phi_n||^2}.\]

  \item A set of real-valued functions $\{\phi_n(x)\}$ is said to be \textbf{orthogonal with respect to a weight function} $w(x)$ on the interval $[a, b]$ if \[\int_a^b w(x) \phi_m(x) \phi_n(x) \,d x = 0,\ m \ne n.\]
\end{itemize}

\subsection{Fourier Series}

\begin{itemize}
  \item The \textbf{Fourier series} of a function $f$ defined on the interval $(-p, p)$ is given by \[f(x) = \frac{a_0}{2} + \sum_{n = 1}^\infty \left( a_n \cos \frac{n \pi}{p} x + b_n \sin \frac{n \pi}{p} x \right)\] where \begin{align*}
          a_0 & = \frac{1}{p} \int_{-p}^p f(x) \,d x                        \\
          a_n & = \frac{1}{p} \int_{-p}^p f(x) \cos \frac{n \pi}{p} x \,d x \\
          b_n & = \frac{1}{p} \int_{-p}^p f(x) \sin \frac{n \pi}{p} x \,d x
        \end{align*}

  \item At points of discontinuity in $f$, the Fourier series takes on the average of the values either side of it.

  \item The Fourier series of a function $f$ gives a \textbf{periodic extension} of the function outside the interval $(-p, p)$.
\end{itemize}

\subsection{Fourier Cosine and Sine Series}

\begin{itemize}
  \item A function $f$ is said to be \textbf{even} if \[f(-x) = f(x)\] and \textbf{odd} if \[f(-x) = -f(x).\]

  \item Even and odd functions have some interesting properties:

        \begin{itemize}
          \item The product of two even functions is even.

          \item The product of two odd functions is even.

          \item The product of an even function and an odd function is odd.

          \item If $f$ is even, then $\int_{-a}^a f(x) \,d x = 2 \int_0^a f(x) \,d x$.

          \item If $f$ is odd, then $\int_{-a}^a f(x) \,d x = 0$.
        \end{itemize}

  \item In light of this, if a function $f$ is even its Fourier coefficients are \begin{align*}
          a_0 & = \frac{2}{p} \int_0^p f(x) \,d x                        \\
          a_n & = \frac{2}{p} \int_0^p f(x) \cos \frac{n \pi}{p} x \,d x \\
          b_n & = 0.
        \end{align*} The series consists of cosine terms and is called the \textbf{Fourier cosine series}.

  \item Similarly, if $f$ is odd then \begin{align*}
          a_n & = 0,\ n = 0, 1, 2, \ldots                                 \\
          b_n & = \frac{2}{p} \int_0^p f(x) \sin \frac{n \pi}{p} x \,d x.
        \end{align*} The series consists of sine terms and is called the \textbf{Fourier sine series}.

  \item Sometimes a Fourier series ``overshoots'' the original value of the function near discontinuities. This is called the \textbf{Gibbs phenomenon}.

  \item Taking the Fourier cosine series of a function $f$ over the interval $[0, L]$ effectively mirrors the function around the vertical axis.

  \item Taking the Fourier sine series of a function $f$ over the interval $[0, L]$ effectively rotates it $\ang{180}$ around the origin.

  \item A particular solution for a nonhomogeneous differential equation with a periodic driving force can be found by taking the Fourier transform of the driving force then using the method of undetermined coefficients to determine the coefficients.
\end{itemize}

\subsection{Complex Fourier Series}

\begin{itemize}
  \item The \textbf{complex Fourier series} of a function $f$ defined on an interval $(-p, p)$ is given by \[f(x) = \sum_{n = -\infty}^\infty c_n e^{i n \pi x / p}\] where \[c_n = \frac{1}{2 p} \int_{-p}^p f(x) e^{-i n \pi x / p} \,d x,\ n = 0, \pm 1, \pm 2, \ldots\]

  \item The \textbf{fundamental period} of a Fourier series is $T = 2 p$.

  \item The \textbf{fundamental angular frequency} of a Fourier series is $\omega = \frac{2 \pi}{T}$.

  \item A \textbf{frequency spectrum} is a plot of the points $(n \omega, |c_n|)$ where $\omega$ is the fundamental angular frequency and $c_n$ are the coefficients of the complex Fourier series. This can be useful to see how each harmonic contributes.
\end{itemize}

\end{document}