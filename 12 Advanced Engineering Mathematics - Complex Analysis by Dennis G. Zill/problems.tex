\documentclass{article}
\usepackage{amsmath} % For align*
\usepackage{enumitem} % For customisable list labels

\renewcommand{\Im}{\operatorname{Im}}
\renewcommand{\Re}{\operatorname{Re}}

\setlist[enumerate, 1]{label={(\alph*)}}
\setlist[enumerate, 2]{label={(\roman*)}}

\title{Advanced Engineering Mathematics Complex Analysis by Dennis G. Zill Problems}
\author{Chris Doble}
\date{February 2023}

\begin{document}

\maketitle

\tableofcontents

\setcounter{section}{16}
\section{Functions of a Complex Variable}

\subsection{Complex Numbers}

\subsubsection{}

\[3 + 3 i\]

\setcounter{subsubsection}{2}
\subsubsection{}

\[i^8 = (i^2)^4 = (-1)^4 = 1\]

\setcounter{subsubsection}{4}
\subsubsection{}

\[7 - 13 i\]

\setcounter{subsubsection}{6}
\subsubsection{}

\[-7 + 5 i\]

\setcounter{subsubsection}{8}
\subsubsection{}

\[11 - 10 i\]

\setcounter{subsubsection}{10}
\subsubsection{}

\[-5 + 12 i\]

\setcounter{subsubsection}{12}
\subsubsection{}

\[-2 i\]

\setcounter{subsubsection}{14}
\subsubsection{}

\begin{align*}
  \frac{2 - 4 i}{3 + 5 i} & = \frac{(2 - 4 i) (3 - 5 i)}{34}  \\
                          & = \frac{-14 - 22 i}{34}           \\
                          & = -\frac{7}{17} - \frac{11}{17} i
\end{align*}

\setcounter{subsubsection}{16}
\subsubsection{}

\begin{align*}
  \frac{(3 - i) (2 + 3 i)}{1 + i} & = \frac{9 + 7 i}{1 + i}       \\
                                  & = \frac{(9 + 7 i) (1 - i)}{2} \\
                                  & = \frac{16 - 2 i}{2}          \\
                                  & = 8 - i
\end{align*}

\setcounter{subsubsection}{26}
\subsubsection{}

\begin{align*}
  \frac{1}{z}                    & = \frac{\overline{z}}{z \overline{z}} \\
                                 & = \frac{x - i y}{x^2 + y^2}           \\
  \Re \left( \frac{1}{z} \right) & = \frac{x}{x^2 + y^2}
\end{align*}

\setcounter{subsubsection}{28}
\subsubsection{}

\begin{align*}
  2 z + 4 \overline{z} - 4 i       & = 2 (x + i y) + 4 (x - i y) - 4 i \\
                                   & = 6 x - 2 (y + 2) i               \\
  \Im (2 z + 4 \overline{z} - 4 i) & = -2 y - 4
\end{align*}

\setcounter{subsubsection}{30}
\subsubsection{}

\begin{align*}
  z - 1 - 3 i & = x + i y - 1 - 3 i            \\
              & = (x - 1) + (y - 3) i          \\
  |z|         & = \sqrt{(x - 1)^2 + (y - 3)^2}
\end{align*}

\setcounter{subsubsection}{32}
\subsubsection{}

\begin{align*}
  2 z & = i (2 + 9 i)      \\
      & = -9 + 2 i         \\
  z   & = -\frac{9}{2} + i
\end{align*}

\setcounter{subsubsection}{34}
\subsubsection{}

\begin{align*}
  (x + i y)^2 & = x^2 + 2 x y i - y^2        \\
              & = (x^2 - y^2) + 2 x y i      \\
  x^2         & = y^2                        \\
  x           & = y                          \\
  2 x y       & = 1                          \\
  x^2         & = \frac{1}{2}                \\
  x           & = \frac{\sqrt{2}}{2}         \\
  z           & = \frac{\sqrt{2}}{2} (1 + i)
\end{align*}

\setcounter{subsubsection}{36}
\subsubsection{}

\begin{align*}
  z + 2 \overline{z}    & = x + i y + 2 x - 2 i y          \\
                        & = 3 x - i y                      \\
  \frac{2 - i}{1 + 3 i} & = \frac{(2 - i) (1 - 3 i)}{10}   \\
                        & = \frac{-1 - 7 i}{10}            \\
  3 x - i y             & = \frac{-1 - 7 i}{10}            \\
  x                     & = -\frac{1}{30}                  \\
  y                     & = \frac{7}{10}                   \\
  z                     & = -\frac{1}{30} + \frac{7}{10} i
\end{align*}

\setcounter{subsubsection}{38}
\subsubsection{}

\begin{align*}
  |10 + 8 i| & \approx 12.8 \\
  |11 - 6 i| & \approx 12.5
\end{align*}

$11 - 6 i$ is closer.

\subsection{Powers and Roots}

\subsubsection{}

\[2 (\cos 0 + i \sin 0)\]

\setcounter{subsubsection}{2}
\subsubsection{}

\[-3 [\cos (-\pi / 2) + i \sin (-\pi / 2)]\]

\setcounter{subsubsection}{4}
\subsubsection{}

\[\sqrt{2} [\cos (\pi / 4) + i \sin (\pi / 4)]\]

\setcounter{subsubsection}{6}
\subsubsection{}

\[2 [\cos (5 \pi / 6) + i \sin (5 \pi / 6)]\]

\setcounter{subsubsection}{8}
\subsubsection{}

\begin{align*}
  \frac{3}{-1 + i} & = \frac{3 (-1 - i)}{2}                                         \\
                   & = \frac{-3 - 3 i}{2}                                           \\
                   & = -\frac{3}{2} - \frac{3}{2} i                                 \\
                   & = \frac{3 \sqrt{2}}{2} [\cos (5 \pi / 4) + i \sin (5 \pi / 4)]
\end{align*}

\setcounter{subsubsection}{10}
\subsubsection{}

\[-\frac{5 \sqrt{3}}{2} - \frac{5}{2} i\]

\setcounter{subsubsection}{12}
\subsubsection{}

\[5.54 + 2.30 i\]

\setcounter{subsubsection}{14}
\subsubsection{}

\begin{align*}
  8 [\cos (\pi / 2) + i \sin (\pi / 2)]             & = 8 i                                       \\
  \frac{1}{2} [\cos (-\pi / 4) + i \sin (-\pi / 4)] & = \frac{\sqrt{2}}{4} - \frac{\sqrt{2}}{4} i
\end{align*}

\setcounter{subsubsection}{20}
\subsubsection{}

\begin{align*}
  (1 + \sqrt{3} i)^9 & = \{ 2 [\cos (\pi / 3) + i \sin (\pi / 3)] \}^9 \\
                     & = 512 (\cos \pi + i \sin \pi)                   \\
                     & = -512
\end{align*}

\setcounter{subsubsection}{22}
\subsubsection{}

\begin{align*}
  \left( \frac{1}{2} + \frac{1}{2} i \right)^10 & = \left\{ \frac{\sqrt{2}}{2} [\cos (\pi / 4) + i \sin (\pi / 4)] \right\}^{10} \\
                                                & = \frac{1}{32} [\cos (\pi / 2) + i \sin (\pi / 2)]                             \\
                                                & = \frac{1}{32} i
\end{align*}

\setcounter{subsubsection}{26}
\subsubsection{}

\begin{align*}
  w_k & = 2 [\cos (2 \pi k / 3) + i \sin (2 \pi k / 3)] \\
  w_0 & = 2                                             \\
  w_1 & = -1 + \sqrt{3} i                               \\
  w_2 & = -1 - \sqrt{3} i
\end{align*}

\setcounter{subsubsection}{28}
\subsubsection{}

\begin{align*}
  w_k & = \cos (\pi / 4 + k \pi) + i \sin (\pi / 4 + k \pi) \\
  w_0 & = \frac{\sqrt{2}}{2} (1 + i)                        \\
  w_1 & = -\frac{\sqrt{2}}{2} (1 + i)
\end{align*}

\setcounter{subsubsection}{30}
\subsubsection{}

\begin{align*}
  w_k & = \sqrt{2} [\cos (\pi / 3 + k \pi) + i \sin (\pi / 3 + k \pi)] \\
  w_0 & = \frac{\sqrt{2}}{2} + \frac{\sqrt{6}}{2} i                    \\
  w_1 & = -\frac{\sqrt{2}}{2} - \frac{\sqrt{6}}{2} i
\end{align*}

\setcounter{subsubsection}{32}
\subsubsection{}

\begin{align*}
  z^4 + 1 & = 0                                                       \\
  z^4     & = -1                                                      \\
  w_k     & = \cos (\pi / 4 + k \pi / 2) + \sin (\pi / 4 + k \pi / 2) \\
  w_0     & = \frac{\sqrt{2}}{2} + \frac{\sqrt{2}}{2} i               \\
  w_1     & = -\frac{\sqrt{2}}{2} + \frac{\sqrt{2}}{2} i              \\
  w_2     & = -\frac{\sqrt{2}}{2} - \frac{\sqrt{2}}{2} i              \\
  w_3     & = \frac{\sqrt{2}}{2} - \frac{\sqrt{2}}{2} i
\end{align*}

\end{document}