\documentclass{article}
\usepackage{amsmath} % For align*
\usepackage{enumitem} % For customisable list labels

\renewcommand{\Im}{\operatorname{Im}}
\renewcommand{\Re}{\operatorname{Re}}

\setlist[enumerate, 1]{label={(\alph*)}}
\setlist[enumerate, 2]{label={(\roman*)}}

\title{Advanced Engineering Mathematics Complex Analysis by Dennis G. Zill Problems}
\author{Chris Doble}
\date{February 2023}

\begin{document}

\maketitle

\tableofcontents

\setcounter{section}{16}
\section{Functions of a Complex Variable}

\subsection{Complex Numbers}

\subsubsection{}

\[3 + 3 i\]

\setcounter{subsubsection}{2}
\subsubsection{}

\[i^8 = (i^2)^4 = (-1)^4 = 1\]

\setcounter{subsubsection}{4}
\subsubsection{}

\[7 - 13 i\]

\setcounter{subsubsection}{6}
\subsubsection{}

\[-7 + 5 i\]

\setcounter{subsubsection}{8}
\subsubsection{}

\[11 - 10 i\]

\setcounter{subsubsection}{10}
\subsubsection{}

\[-5 + 12 i\]

\setcounter{subsubsection}{12}
\subsubsection{}

\[-2 i\]

\setcounter{subsubsection}{14}
\subsubsection{}

\begin{align*}
  \frac{2 - 4 i}{3 + 5 i} & = \frac{(2 - 4 i) (3 - 5 i)}{34}  \\
                          & = \frac{-14 - 22 i}{34}           \\
                          & = -\frac{7}{17} - \frac{11}{17} i
\end{align*}

\setcounter{subsubsection}{16}
\subsubsection{}

\begin{align*}
  \frac{(3 - i) (2 + 3 i)}{1 + i} & = \frac{9 + 7 i}{1 + i}       \\
                                  & = \frac{(9 + 7 i) (1 - i)}{2} \\
                                  & = \frac{16 - 2 i}{2}          \\
                                  & = 8 - i
\end{align*}

\setcounter{subsubsection}{26}
\subsubsection{}

\begin{align*}
  \frac{1}{z}                    & = \frac{\overline{z}}{z \overline{z}} \\
                                 & = \frac{x - i y}{x^2 + y^2}           \\
  \Re \left( \frac{1}{z} \right) & = \frac{x}{x^2 + y^2}
\end{align*}

\setcounter{subsubsection}{28}
\subsubsection{}

\begin{align*}
  2 z + 4 \overline{z} - 4 i       & = 2 (x + i y) + 4 (x - i y) - 4 i \\
                                   & = 6 x - 2 (y + 2) i               \\
  \Im (2 z + 4 \overline{z} - 4 i) & = -2 y - 4
\end{align*}

\setcounter{subsubsection}{30}
\subsubsection{}

\begin{align*}
  z - 1 - 3 i & = x + i y - 1 - 3 i            \\
              & = (x - 1) + (y - 3) i          \\
  |z|         & = \sqrt{(x - 1)^2 + (y - 3)^2}
\end{align*}

\setcounter{subsubsection}{32}
\subsubsection{}

\begin{align*}
  2 z & = i (2 + 9 i)      \\
      & = -9 + 2 i         \\
  z   & = -\frac{9}{2} + i
\end{align*}

\setcounter{subsubsection}{34}
\subsubsection{}

\begin{align*}
  (x + i y)^2 & = x^2 + 2 x y i - y^2        \\
              & = (x^2 - y^2) + 2 x y i      \\
  x^2         & = y^2                        \\
  x           & = y                          \\
  2 x y       & = 1                          \\
  x^2         & = \frac{1}{2}                \\
  x           & = \frac{\sqrt{2}}{2}         \\
  z           & = \frac{\sqrt{2}}{2} (1 + i)
\end{align*}

\setcounter{subsubsection}{36}
\subsubsection{}

\begin{align*}
  z + 2 \overline{z}    & = x + i y + 2 x - 2 i y          \\
                        & = 3 x - i y                      \\
  \frac{2 - i}{1 + 3 i} & = \frac{(2 - i) (1 - 3 i)}{10}   \\
                        & = \frac{-1 - 7 i}{10}            \\
  3 x - i y             & = \frac{-1 - 7 i}{10}            \\
  x                     & = -\frac{1}{30}                  \\
  y                     & = \frac{7}{10}                   \\
  z                     & = -\frac{1}{30} + \frac{7}{10} i
\end{align*}

\setcounter{subsubsection}{38}
\subsubsection{}

\begin{align*}
  |10 + 8 i| & \approx 12.8 \\
  |11 - 6 i| & \approx 12.5
\end{align*}

$11 - 6 i$ is closer.

\subsection{Powers and Roots}

\subsubsection{}

\[2 (\cos 0 + i \sin 0)\]

\setcounter{subsubsection}{2}
\subsubsection{}

\[-3 [\cos (-\pi / 2) + i \sin (-\pi / 2)]\]

\setcounter{subsubsection}{4}
\subsubsection{}

\[\sqrt{2} [\cos (\pi / 4) + i \sin (\pi / 4)]\]

\setcounter{subsubsection}{6}
\subsubsection{}

\[2 [\cos (5 \pi / 6) + i \sin (5 \pi / 6)]\]

\setcounter{subsubsection}{8}
\subsubsection{}

\begin{align*}
  \frac{3}{-1 + i} & = \frac{3 (-1 - i)}{2}                                         \\
                   & = \frac{-3 - 3 i}{2}                                           \\
                   & = -\frac{3}{2} - \frac{3}{2} i                                 \\
                   & = \frac{3 \sqrt{2}}{2} [\cos (5 \pi / 4) + i \sin (5 \pi / 4)]
\end{align*}

\setcounter{subsubsection}{10}
\subsubsection{}

\[-\frac{5 \sqrt{3}}{2} - \frac{5}{2} i\]

\setcounter{subsubsection}{12}
\subsubsection{}

\[5.54 + 2.30 i\]

\setcounter{subsubsection}{14}
\subsubsection{}

\begin{align*}
  8 [\cos (\pi / 2) + i \sin (\pi / 2)]             & = 8 i                                       \\
  \frac{1}{2} [\cos (-\pi / 4) + i \sin (-\pi / 4)] & = \frac{\sqrt{2}}{4} - \frac{\sqrt{2}}{4} i
\end{align*}

\setcounter{subsubsection}{20}
\subsubsection{}

\begin{align*}
  (1 + \sqrt{3} i)^9 & = \{ 2 [\cos (\pi / 3) + i \sin (\pi / 3)] \}^9 \\
                     & = 512 (\cos \pi + i \sin \pi)                   \\
                     & = -512
\end{align*}

\setcounter{subsubsection}{22}
\subsubsection{}

\begin{align*}
  \left( \frac{1}{2} + \frac{1}{2} i \right)^10 & = \left\{ \frac{\sqrt{2}}{2} [\cos (\pi / 4) + i \sin (\pi / 4)] \right\}^{10} \\
                                                & = \frac{1}{32} [\cos (\pi / 2) + i \sin (\pi / 2)]                             \\
                                                & = \frac{1}{32} i
\end{align*}

\setcounter{subsubsection}{26}
\subsubsection{}

\begin{align*}
  w_k & = 2 [\cos (2 \pi k / 3) + i \sin (2 \pi k / 3)] \\
  w_0 & = 2                                             \\
  w_1 & = -1 + \sqrt{3} i                               \\
  w_2 & = -1 - \sqrt{3} i
\end{align*}

\setcounter{subsubsection}{28}
\subsubsection{}

\begin{align*}
  w_k & = \cos (\pi / 4 + k \pi) + i \sin (\pi / 4 + k \pi) \\
  w_0 & = \frac{\sqrt{2}}{2} (1 + i)                        \\
  w_1 & = -\frac{\sqrt{2}}{2} (1 + i)
\end{align*}

\setcounter{subsubsection}{30}
\subsubsection{}

\begin{align*}
  w_k & = \sqrt{2} [\cos (\pi / 3 + k \pi) + i \sin (\pi / 3 + k \pi)] \\
  w_0 & = \frac{\sqrt{2}}{2} + \frac{\sqrt{6}}{2} i                    \\
  w_1 & = -\frac{\sqrt{2}}{2} - \frac{\sqrt{6}}{2} i
\end{align*}

\setcounter{subsubsection}{32}
\subsubsection{}

\begin{align*}
  z^4 + 1 & = 0                                                       \\
  z^4     & = -1                                                      \\
  w_k     & = \cos (\pi / 4 + k \pi / 2) + \sin (\pi / 4 + k \pi / 2) \\
  w_0     & = \frac{\sqrt{2}}{2} + \frac{\sqrt{2}}{2} i               \\
  w_1     & = -\frac{\sqrt{2}}{2} + \frac{\sqrt{2}}{2} i              \\
  w_2     & = -\frac{\sqrt{2}}{2} - \frac{\sqrt{2}}{2} i              \\
  w_3     & = \frac{\sqrt{2}}{2} - \frac{\sqrt{2}}{2} i
\end{align*}

\subsection{Sets in the Complex Plane}

\subsubsection{}

A vertical line at $\Re(z) = 5$.

\setcounter{subsubsection}{2}
\subsubsection{}

A horizontal line at $\Im(z) = -3$.

\setcounter{subsubsection}{4}
\subsubsection{}

A circle of radius $2$ centred at $3 i$.

\setcounter{subsubsection}{6}
\subsubsection{}

A circle of radius $5$ centred at $4 - 3 i$.

\setcounter{subsubsection}{8}
\subsubsection{}

The region of the plane to the left of (but not including) $\Re (z) = -1$. It is a domain.

\setcounter{subsubsection}{10}
\subsubsection{}

The region of the plane above (but not including) $\Im (z) = 3$. It is a domain.

\setcounter{subsubsection}{12}
\subsubsection{}

The region of the plane between (but not including) $\Re (z) = 3$ and $\Re (z) = 5$. It is a domain.

\setcounter{subsubsection}{14}
\subsubsection{}

\begin{align*}
  z^2       & = (a + i b)^2         \\
            & = a^2 - b^2 + 2 i a b \\
  \Re (z^2) & = a^2 - b^2           \\
  \Re (z^2) & > 0                   \\
  a^2 - b^2 & > 0                   \\
  a^2       & > b^2
\end{align*}

The region between $y = x$ and $y = -x$. Not a domain.

\setcounter{subsubsection}{16}
\subsubsection{}

The region between $\theta = 0$ and $\theta = 2 \pi / 3$. Not a domain.

\setcounter{subsubsection}{18}
\subsubsection{}

The region outside a circle of radius $1$ centred at $i$. It is a domain.

\setcounter{subsubsection}{20}
\subsubsection{}

The region between the circles of radius $2$ and $3$ centred at $i$. It is a domain.

\setcounter{subsubsection}{22}
\subsubsection{}

\[y = -x\]

\setcounter{subsubsection}{24}
\subsubsection{}

\begin{align*}
  z^2 + \overline{z}^2 & = (a + i b)^2 + (a - i b)^2                 \\
                       & = a^2 + 2 i a b - b^2 + a^2 - 2 i a b - b^2 \\
                       & = 2 (a^2 - b^2)                             \\
  2 (a^2 - b^2)        & = 2                                         \\
  a^2 - b^2            & = 1                                         \\
  a^2                  & = b^2 + 1
\end{align*}

The hyperbola $x^2 - y^2 = 1$.

\subsection{Functions of a Complex Variable}

\subsubsection{}

\begin{align*}
  f(z)    & = z^2                              \\
          & = (x + i y)^2                      \\
          & = x^2 - y^2 + 2 i x y              \\
  u(x, y) & = x^2 - y^2                        \\
          & = x^2 - 4                          \\
  v(x, y) & = 2 x y                            \\
          & = 4 x                              \\
  x       & = \frac{v}{4}                      \\
  u       & = \left( \frac{v}{4} \right)^2 - 4 \\
          & = \frac{1}{16} v^2 - 4
\end{align*}

\setcounter{subsubsection}{2}
\subsubsection{}

\begin{align*}
  u & = -y^2 \\
  v & = 0
\end{align*}

Line on the left half of the real axis.

\setcounter{subsubsection}{4}
\subsubsection{}

\begin{align*}
  u & = 0     \\
  v & = 2 x^2
\end{align*}

Line on the top half of the imaginary axis.

\setcounter{subsubsection}{6}
\subsubsection{}

\[f(x) = (6 x - 5) + i (6 y + 9)\]

\setcounter{subsubsection}{8}
\subsubsection{}

\[f(z) = (x^2 - y^2 - 3 x) + i (2 x y - 3 y + 4)\]

\setcounter{subsubsection}{10}
\subsubsection{}

\[f(z) = (x^3 - 3 x y^2 - 4 x) + i (3 x^2 y - y^3 - 4 y)\]

\setcounter{subsubsection}{12}
\subsubsection{}

\[f(z) = \left( x + \frac{x}{x^2 + y^2} \right) i \left( y - \frac{y}{x^2 + y^2} \right)\]

\setcounter{subsubsection}{14}
\subsubsection{}

\begin{enumerate}
  \item $-4 + i$

  \item $3 - 9 i$

  \item $1 + 86 i$
\end{enumerate}

\setcounter{subsubsection}{16}
\subsubsection{}

\begin{enumerate}
  \item $14 - 20 i$

  \item $-13 + 43 i$

  \item $3 - 26 i$
\end{enumerate}

\setcounter{subsubsection}{18}
\subsubsection{}

\[6 - 5 i\]

\setcounter{subsubsection}{20}
\subsubsection{}

\[-4 i\]

\setcounter{subsubsection}{26}
\subsubsection{}

\[f'(z) = 12 z^2 - 2 (3 + i) z - 5\]

\setcounter{subsubsection}{28}
\subsubsection{}

\begin{align*}
  f'(z) & = 2 (z^2 - 4 z + 8 i) + (2 z + 1) (2 z - 4)  \\
        & = 2 z^2 - 8 z + 16 i + 4 z^2 - 8 z + 2 z - 4 \\
        & = 6 z^2 - 14 z - 4 + 16 i
\end{align*}

\setcounter{subsubsection}{30}
\subsubsection{}

\[f'(z) = 6 z (z^2 - 4 i)^2\]

\setcounter{subsubsection}{32}
\subsubsection{}

\begin{align*}
  f'(z) & = \frac{3 (2 z + i) - 2 (3 z - 4 + 8 i)}{(2 z + i)^2} \\
        & = \frac{6 z + 3 i - 6 z + 8 - 16 i}{(2 z + i)^2}      \\
        & = \frac{8 - 13 i}{(2 z + i)^2}
\end{align*}

\setcounter{subsubsection}{34}
\subsubsection{}

$3 i$

\setcounter{subsubsection}{36}
\subsubsection{}

$\pm 2 i$

\setcounter{subsubsection}{40}
\subsubsection{}

\begin{align*}
  \frac{d x}{d t} & = 2 x         \\
  x               & = c_1 e^{2 t} \\
  \frac{d y}{d t} & = 2 y         \\
  y               & = c_2 e^{2 t}
\end{align*}

\setcounter{subsubsection}{42}
\subsubsection{}

\begin{align*}
  f(z)            & = \frac{1}{\overline{z}}                      \\
                  & = \frac{1}{x - i y}                           \\
                  & = \frac{x + i y}{x^2 + y^2}                   \\
                  & = \frac{x}{x^2 + y^2} + i \frac{y}{x^2 + y^2} \\
  \frac{d x}{d t} & = \frac{x}{x^2 + y^2}                         \\
  \frac{d y}{d t} & = \frac{y}{x^2 + y^2}                         \\
  \frac{d y}{d x} & = \frac{y}{x}                                 \\
  \frac{d y}{y}   & = \frac{d x}{x}                               \\
  \ln y           & = \ln x + c_1                                 \\
  y               & = c_2 x
\end{align*}

\subsection{Cauchy-Riemann Equations}

\subsubsection{}

\begin{align*}
  f(z)                          & = z^3                                                 \\
                                & = (x + i y)^3                                         \\
                                & = (x^2 + 2 i x y - y^2) (x + i y)                     \\
                                & = x^3 + i x^2 y + 2 i x^2 y - 2 x y^2 - x y^2 - i y^3 \\
                                & = (x^3 - 3 x y^2) + i (3 x^2 y - y^3)                 \\
  \frac{\partial u}{\partial x} & = 3 x^2 - 3 y^2                                       \\
                                & = \frac{\partial v}{\partial y}                       \\
  \frac{\partial u}{\partial y} & = -6 x y                                              \\
                                & = -\frac{\partial v}{\partial x}
\end{align*}

\setcounter{subsubsection}{2}
\subsubsection{}

\begin{align*}
  f(z)                          & = \Re (z)                         \\
                                & = x                               \\
  \frac{\partial u}{\partial x} & = 1                               \\
                                & \ne \frac{\partial v}{\partial y}
\end{align*}

\setcounter{subsubsection}{4}
\subsubsection{}

\begin{align*}
  f(z)                          & = 4 z - 6 \overline{z} + 3        \\
                                & = 4 (x + i y) - 6 (x - i y) + 3   \\
                                & = (-2 x + 3) + 10 i y             \\
  \frac{\partial u}{\partial x} & = -2                              \\
                                & \ne \frac{\partial v}{\partial y}
\end{align*}

\setcounter{subsubsection}{6}
\subsubsection{}

\begin{align*}
  f(z)                          & = x^2 + y^2                       \\
  \frac{\partial u}{\partial x} & = 2 x                             \\
                                & \ne \frac{\partial v}{\partial y}
\end{align*}

\setcounter{subsubsection}{8}
\subsubsection{}

\begin{align*}
  f(z)                          & = e^x \cos y + i e^x \sin y \\
  u                             & = e^x \cos y                \\
  \frac{\partial u}{\partial x} & = e^x \cos y                \\
  \frac{\partial u}{\partial y} & = -e^x \sin y               \\
  v                             & = e^x \sin y                \\
  \frac{\partial v}{\partial x} & = e^x \sin y                \\
  \frac{\partial v}{\partial y} & = e^x \cos y                \\
\end{align*}

Analytic everywhere.

\setcounter{subsubsection}{10}
\subsubsection{}

\begin{align*}
  f(z)                          & = x + \sin x \cosh y + i (y + \cos x \sinh y) \\
  u                             & = x + \sin x \cosh y                          \\
  \frac{\partial u}{\partial x} & = 1 + \cos x \cosh y                          \\
  \frac{\partial u}{\partial y} & = \sin x \sinh y                              \\
  v                             & = y + \cos x \sinh y                          \\
  \frac{\partial v}{\partial x} & = -\sin x \sinh y                             \\
  \frac{\partial v}{\partial y} & = 1 + \cos x \cosh y
\end{align*}

Analytic everywhere.

\setcounter{subsubsection}{14}
\subsubsection{}

\begin{align*}
  f(z)                          & = 3 x - y + 5 + i (a x + b y - 3) \\
  u                             & = 3 x - y + 5                     \\
  \frac{\partial u}{\partial x} & = 3                               \\
  \frac{\partial u}{\partial y} & = -1                              \\
  v                             & = a x + b y - 3                   \\
  \frac{\partial v}{\partial x} & = a                               \\
  \frac{\partial v}{\partial y} & = b                               \\
  a                             & = 1                               \\
  b                             & = 3
\end{align*}

\setcounter{subsubsection}{16}
\subsubsection{}

\begin{align*}
  f(z)                          & = x^2 + y^2 + 2 i x y \\
  u                             & = x^2 + y^2           \\
  \frac{\partial u}{\partial x} & = 2 x                 \\
  \frac{\partial u}{\partial y} & = 2 y                 \\
  v                             & = 2 x y               \\
  \frac{\partial v}{\partial x} & = 2 y                 \\
  \frac{\partial v}{\partial y} & = 2 x
\end{align*}

Only differentiable when $y = 0$.

\setcounter{subsubsection}{18}
\subsubsection{}

\begin{align*}
  f(z)                          & = x^3 + 3 x y^2 - x + i (y^3 + 3 x^2 y - y) \\
  u                             & = x^3 + 3 x y^2 - x                         \\
  \frac{\partial u}{\partial x} & = 3 x^2 + 3 y^2 - 1                         \\
  \frac{\partial u}{\partial y} & = 6 x y                                     \\
  v                             & = y^3 + 3 x^2 y - y                         \\
  \frac{\partial v}{\partial x} & = 6 x y                                     \\
  \frac{\partial v}{\partial y} & = 3 y^2 + 3 x^2 - 1
\end{align*}

Only differentiable when $x = 0$ or $y = 0$.

\setcounter{subsubsection}{20}
\subsubsection{}

\begin{align*}
  f(z)  & = e^x \cos y + i e^x \sin y                                       \\
  f'(z) & = \frac{\partial u}{\partial x} + i \frac{\partial v}{\partial x} \\
        & = e^x \cos y + i e^x \sin y
\end{align*}

\setcounter{subsubsection}{22}
\subsubsection{}

\begin{align*}
  u                                 & = x             \\
  \frac{\partial^2 u}{\partial x^2} & = 0             \\
  \frac{\partial^2 u}{\partial y^2} & = 0             \\
  \frac{\partial v}{\partial y}     & = 1             \\
  v                                 & = y + h(x)      \\
  h'(x)                             & = 0             \\
  v                                 & = y + c         \\
  f(z)                              & = x + i (y + c)
\end{align*}

\setcounter{subsubsection}{24}
\subsubsection{}

\begin{align*}
  u                                 & = x^2 - y^2                   \\
  \frac{\partial^2 u}{\partial x^2} & = 2                           \\
  \frac{\partial^2 u}{\partial y^2} & = -2                          \\
  \frac{\partial v}{\partial y}     & = 2 x                         \\
  v                                 & = 2 x y + h(x)                \\
  2 y                               & = 2 y + h'(x)                 \\
  h'(x)                             & = 0                           \\
  h(x)                              & = c                           \\
  v                                 & = 2 x y + c                   \\
  f(z)                              & = (x^2 - y^2) + i (2 x y + c)
\end{align*}

\end{document}