\documentclass{article}
\usepackage{amsmath} % For align*
\usepackage{bookmark} % For links
\usepackage{enumitem} % For customisable list labels
\usepackage{esint} % For contour integrals

\hypersetup{
  colorlinks=true,
  linkcolor=blue,
  urlcolor=blue
}

\newcommand{\arcsinh}{\operatorname{arcsinh}}
\renewcommand{\Im}{\operatorname{Im}}
\newcommand{\Ln}{\operatorname{Ln}}
\renewcommand{\Re}{\operatorname{Re}}

\setlist[enumerate, 1]{label={(\alph*)}}
\setlist[enumerate, 2]{label={(\roman*)}}

\title{Advanced Engineering Mathematics Complex Analysis by Dennis G. Zill Problems}
\author{Chris Doble}
\date{February 2023}

\begin{document}

\maketitle

\tableofcontents

\setcounter{section}{16}
\section{Functions of a Complex Variable}

\subsection{Complex Numbers}

\subsubsection{}

\[3 + 3 i\]

\setcounter{subsubsection}{2}
\subsubsection{}

\[i^8 = (i^2)^4 = (-1)^4 = 1\]

\setcounter{subsubsection}{4}
\subsubsection{}

\[7 - 13 i\]

\setcounter{subsubsection}{6}
\subsubsection{}

\[-7 + 5 i\]

\setcounter{subsubsection}{8}
\subsubsection{}

\[11 - 10 i\]

\setcounter{subsubsection}{10}
\subsubsection{}

\[-5 + 12 i\]

\setcounter{subsubsection}{12}
\subsubsection{}

\[-2 i\]

\setcounter{subsubsection}{14}
\subsubsection{}

\begin{align*}
  \frac{2 - 4 i}{3 + 5 i} & = \frac{(2 - 4 i) (3 - 5 i)}{34}  \\
                          & = \frac{-14 - 22 i}{34}           \\
                          & = -\frac{7}{17} - \frac{11}{17} i
\end{align*}

\setcounter{subsubsection}{16}
\subsubsection{}

\begin{align*}
  \frac{(3 - i) (2 + 3 i)}{1 + i} & = \frac{9 + 7 i}{1 + i}       \\
                                  & = \frac{(9 + 7 i) (1 - i)}{2} \\
                                  & = \frac{16 - 2 i}{2}          \\
                                  & = 8 - i
\end{align*}

\setcounter{subsubsection}{26}
\subsubsection{}

\begin{align*}
  \frac{1}{z}                    & = \frac{\overline{z}}{z \overline{z}} \\
                                 & = \frac{x - i y}{x^2 + y^2}           \\
  \Re \left( \frac{1}{z} \right) & = \frac{x}{x^2 + y^2}
\end{align*}

\setcounter{subsubsection}{28}
\subsubsection{}

\begin{align*}
  2 z + 4 \overline{z} - 4 i       & = 2 (x + i y) + 4 (x - i y) - 4 i \\
                                   & = 6 x - 2 (y + 2) i               \\
  \Im (2 z + 4 \overline{z} - 4 i) & = -2 y - 4
\end{align*}

\setcounter{subsubsection}{30}
\subsubsection{}

\begin{align*}
  z - 1 - 3 i & = x + i y - 1 - 3 i            \\
              & = (x - 1) + (y - 3) i          \\
  |z|         & = \sqrt{(x - 1)^2 + (y - 3)^2}
\end{align*}

\setcounter{subsubsection}{32}
\subsubsection{}

\begin{align*}
  2 z & = i (2 + 9 i)      \\
      & = -9 + 2 i         \\
  z   & = -\frac{9}{2} + i
\end{align*}

\setcounter{subsubsection}{34}
\subsubsection{}

\begin{align*}
  (x + i y)^2 & = x^2 + 2 x y i - y^2        \\
              & = (x^2 - y^2) + 2 x y i      \\
  x^2         & = y^2                        \\
  x           & = y                          \\
  2 x y       & = 1                          \\
  x^2         & = \frac{1}{2}                \\
  x           & = \frac{\sqrt{2}}{2}         \\
  z           & = \frac{\sqrt{2}}{2} (1 + i)
\end{align*}

\setcounter{subsubsection}{36}
\subsubsection{}

\begin{align*}
  z + 2 \overline{z}    & = x + i y + 2 x - 2 i y          \\
                        & = 3 x - i y                      \\
  \frac{2 - i}{1 + 3 i} & = \frac{(2 - i) (1 - 3 i)}{10}   \\
                        & = \frac{-1 - 7 i}{10}            \\
  3 x - i y             & = \frac{-1 - 7 i}{10}            \\
  x                     & = -\frac{1}{30}                  \\
  y                     & = \frac{7}{10}                   \\
  z                     & = -\frac{1}{30} + \frac{7}{10} i
\end{align*}

\setcounter{subsubsection}{38}
\subsubsection{}

\begin{align*}
  |10 + 8 i| & \approx 12.8 \\
  |11 - 6 i| & \approx 12.5
\end{align*}

$11 - 6 i$ is closer.

\subsection{Powers and Roots}

\subsubsection{}

\[2 (\cos 0 + i \sin 0)\]

\setcounter{subsubsection}{2}
\subsubsection{}

\[-3 [\cos (-\pi / 2) + i \sin (-\pi / 2)]\]

\setcounter{subsubsection}{4}
\subsubsection{}

\[\sqrt{2} [\cos (\pi / 4) + i \sin (\pi / 4)]\]

\setcounter{subsubsection}{6}
\subsubsection{}

\[2 [\cos (5 \pi / 6) + i \sin (5 \pi / 6)]\]

\setcounter{subsubsection}{8}
\subsubsection{}

\begin{align*}
  \frac{3}{-1 + i} & = \frac{3 (-1 - i)}{2}                                         \\
                   & = \frac{-3 - 3 i}{2}                                           \\
                   & = -\frac{3}{2} - \frac{3}{2} i                                 \\
                   & = \frac{3 \sqrt{2}}{2} [\cos (5 \pi / 4) + i \sin (5 \pi / 4)]
\end{align*}

\setcounter{subsubsection}{10}
\subsubsection{}

\[-\frac{5 \sqrt{3}}{2} - \frac{5}{2} i\]

\setcounter{subsubsection}{12}
\subsubsection{}

\[5.54 + 2.30 i\]

\setcounter{subsubsection}{14}
\subsubsection{}

\begin{align*}
  8 [\cos (\pi / 2) + i \sin (\pi / 2)]             & = 8 i                                       \\
  \frac{1}{2} [\cos (-\pi / 4) + i \sin (-\pi / 4)] & = \frac{\sqrt{2}}{4} - \frac{\sqrt{2}}{4} i
\end{align*}

\setcounter{subsubsection}{20}
\subsubsection{}

\begin{align*}
  (1 + \sqrt{3} i)^9 & = \{ 2 [\cos (\pi / 3) + i \sin (\pi / 3)] \}^9 \\
                     & = 512 (\cos \pi + i \sin \pi)                   \\
                     & = -512
\end{align*}

\setcounter{subsubsection}{22}
\subsubsection{}

\begin{align*}
  \left( \frac{1}{2} + \frac{1}{2} i \right)^10 & = \left\{ \frac{\sqrt{2}}{2} [\cos (\pi / 4) + i \sin (\pi / 4)] \right\}^{10} \\
                                                & = \frac{1}{32} [\cos (\pi / 2) + i \sin (\pi / 2)]                             \\
                                                & = \frac{1}{32} i
\end{align*}

\setcounter{subsubsection}{26}
\subsubsection{}

\begin{align*}
  w_k & = 2 [\cos (2 \pi k / 3) + i \sin (2 \pi k / 3)] \\
  w_0 & = 2                                             \\
  w_1 & = -1 + \sqrt{3} i                               \\
  w_2 & = -1 - \sqrt{3} i
\end{align*}

\setcounter{subsubsection}{28}
\subsubsection{}

\begin{align*}
  w_k & = \cos (\pi / 4 + k \pi) + i \sin (\pi / 4 + k \pi) \\
  w_0 & = \frac{\sqrt{2}}{2} (1 + i)                        \\
  w_1 & = -\frac{\sqrt{2}}{2} (1 + i)
\end{align*}

\setcounter{subsubsection}{30}
\subsubsection{}

\begin{align*}
  w_k & = \sqrt{2} [\cos (\pi / 3 + k \pi) + i \sin (\pi / 3 + k \pi)] \\
  w_0 & = \frac{\sqrt{2}}{2} + \frac{\sqrt{6}}{2} i                    \\
  w_1 & = -\frac{\sqrt{2}}{2} - \frac{\sqrt{6}}{2} i
\end{align*}

\setcounter{subsubsection}{32}
\subsubsection{}

\begin{align*}
  z^4 + 1 & = 0                                                       \\
  z^4     & = -1                                                      \\
  w_k     & = \cos (\pi / 4 + k \pi / 2) + \sin (\pi / 4 + k \pi / 2) \\
  w_0     & = \frac{\sqrt{2}}{2} + \frac{\sqrt{2}}{2} i               \\
  w_1     & = -\frac{\sqrt{2}}{2} + \frac{\sqrt{2}}{2} i              \\
  w_2     & = -\frac{\sqrt{2}}{2} - \frac{\sqrt{2}}{2} i              \\
  w_3     & = \frac{\sqrt{2}}{2} - \frac{\sqrt{2}}{2} i
\end{align*}

\subsection{Sets in the Complex Plane}

\subsubsection{}

A vertical line at $\Re(z) = 5$.

\setcounter{subsubsection}{2}
\subsubsection{}

A horizontal line at $\Im(z) = -3$.

\setcounter{subsubsection}{4}
\subsubsection{}

A circle of radius $2$ centred at $3 i$.

\setcounter{subsubsection}{6}
\subsubsection{}

A circle of radius $5$ centred at $4 - 3 i$.

\setcounter{subsubsection}{8}
\subsubsection{}

The region of the plane to the left of (but not including) $\Re (z) = -1$. It is a domain.

\setcounter{subsubsection}{10}
\subsubsection{}

The region of the plane above (but not including) $\Im (z) = 3$. It is a domain.

\setcounter{subsubsection}{12}
\subsubsection{}

The region of the plane between (but not including) $\Re (z) = 3$ and $\Re (z) = 5$. It is a domain.

\setcounter{subsubsection}{14}
\subsubsection{}

\begin{align*}
  z^2       & = (a + i b)^2         \\
            & = a^2 - b^2 + 2 i a b \\
  \Re (z^2) & = a^2 - b^2           \\
  \Re (z^2) & > 0                   \\
  a^2 - b^2 & > 0                   \\
  a^2       & > b^2
\end{align*}

The region between $y = x$ and $y = -x$. Not a domain.

\setcounter{subsubsection}{16}
\subsubsection{}

The region between $\theta = 0$ and $\theta = 2 \pi / 3$. Not a domain.

\setcounter{subsubsection}{18}
\subsubsection{}

The region outside a circle of radius $1$ centred at $i$. It is a domain.

\setcounter{subsubsection}{20}
\subsubsection{}

The region between the circles of radius $2$ and $3$ centred at $i$. It is a domain.

\setcounter{subsubsection}{22}
\subsubsection{}

\[y = -x\]

\setcounter{subsubsection}{24}
\subsubsection{}

\begin{align*}
  z^2 + \overline{z}^2 & = (a + i b)^2 + (a - i b)^2                 \\
                       & = a^2 + 2 i a b - b^2 + a^2 - 2 i a b - b^2 \\
                       & = 2 (a^2 - b^2)                             \\
  2 (a^2 - b^2)        & = 2                                         \\
  a^2 - b^2            & = 1                                         \\
  a^2                  & = b^2 + 1
\end{align*}

The hyperbola $x^2 - y^2 = 1$.

\subsection{Functions of a Complex Variable}

\subsubsection{}

\begin{align*}
  f(z)    & = z^2                              \\
          & = (x + i y)^2                      \\
          & = x^2 - y^2 + 2 i x y              \\
  u(x, y) & = x^2 - y^2                        \\
          & = x^2 - 4                          \\
  v(x, y) & = 2 x y                            \\
          & = 4 x                              \\
  x       & = \frac{v}{4}                      \\
  u       & = \left( \frac{v}{4} \right)^2 - 4 \\
          & = \frac{1}{16} v^2 - 4
\end{align*}

\setcounter{subsubsection}{2}
\subsubsection{}

\begin{align*}
  u & = -y^2 \\
  v & = 0
\end{align*}

Line on the left half of the real axis.

\setcounter{subsubsection}{4}
\subsubsection{}

\begin{align*}
  u & = 0     \\
  v & = 2 x^2
\end{align*}

Line on the top half of the imaginary axis.

\setcounter{subsubsection}{6}
\subsubsection{}

\[f(x) = (6 x - 5) + i (6 y + 9)\]

\setcounter{subsubsection}{8}
\subsubsection{}

\[f(z) = (x^2 - y^2 - 3 x) + i (2 x y - 3 y + 4)\]

\setcounter{subsubsection}{10}
\subsubsection{}

\[f(z) = (x^3 - 3 x y^2 - 4 x) + i (3 x^2 y - y^3 - 4 y)\]

\setcounter{subsubsection}{12}
\subsubsection{}

\[f(z) = \left( x + \frac{x}{x^2 + y^2} \right) i \left( y - \frac{y}{x^2 + y^2} \right)\]

\setcounter{subsubsection}{14}
\subsubsection{}

\begin{enumerate}
  \item $-4 + i$

  \item $3 - 9 i$

  \item $1 + 86 i$
\end{enumerate}

\setcounter{subsubsection}{16}
\subsubsection{}

\begin{enumerate}
  \item $14 - 20 i$

  \item $-13 + 43 i$

  \item $3 - 26 i$
\end{enumerate}

\setcounter{subsubsection}{18}
\subsubsection{}

\[6 - 5 i\]

\setcounter{subsubsection}{20}
\subsubsection{}

\[-4 i\]

\setcounter{subsubsection}{26}
\subsubsection{}

\[f'(z) = 12 z^2 - 2 (3 + i) z - 5\]

\setcounter{subsubsection}{28}
\subsubsection{}

\begin{align*}
  f'(z) & = 2 (z^2 - 4 z + 8 i) + (2 z + 1) (2 z - 4)  \\
        & = 2 z^2 - 8 z + 16 i + 4 z^2 - 8 z + 2 z - 4 \\
        & = 6 z^2 - 14 z - 4 + 16 i
\end{align*}

\setcounter{subsubsection}{30}
\subsubsection{}

\[f'(z) = 6 z (z^2 - 4 i)^2\]

\setcounter{subsubsection}{32}
\subsubsection{}

\begin{align*}
  f'(z) & = \frac{3 (2 z + i) - 2 (3 z - 4 + 8 i)}{(2 z + i)^2} \\
        & = \frac{6 z + 3 i - 6 z + 8 - 16 i}{(2 z + i)^2}      \\
        & = \frac{8 - 13 i}{(2 z + i)^2}
\end{align*}

\setcounter{subsubsection}{34}
\subsubsection{}

$3 i$

\setcounter{subsubsection}{36}
\subsubsection{}

$\pm 2 i$

\setcounter{subsubsection}{40}
\subsubsection{}

\begin{align*}
  \frac{d x}{d t} & = 2 x         \\
  x               & = c_1 e^{2 t} \\
  \frac{d y}{d t} & = 2 y         \\
  y               & = c_2 e^{2 t}
\end{align*}

\setcounter{subsubsection}{42}
\subsubsection{}

\begin{align*}
  f(z)            & = \frac{1}{\overline{z}}                      \\
                  & = \frac{1}{x - i y}                           \\
                  & = \frac{x + i y}{x^2 + y^2}                   \\
                  & = \frac{x}{x^2 + y^2} + i \frac{y}{x^2 + y^2} \\
  \frac{d x}{d t} & = \frac{x}{x^2 + y^2}                         \\
  \frac{d y}{d t} & = \frac{y}{x^2 + y^2}                         \\
  \frac{d y}{d x} & = \frac{y}{x}                                 \\
  \frac{d y}{y}   & = \frac{d x}{x}                               \\
  \ln y           & = \ln x + c_1                                 \\
  y               & = c_2 x
\end{align*}

\subsection{Cauchy-Riemann Equations}

\subsubsection{}

\begin{align*}
  f(z)                          & = z^3                                                 \\
                                & = (x + i y)^3                                         \\
                                & = (x^2 + 2 i x y - y^2) (x + i y)                     \\
                                & = x^3 + i x^2 y + 2 i x^2 y - 2 x y^2 - x y^2 - i y^3 \\
                                & = (x^3 - 3 x y^2) + i (3 x^2 y - y^3)                 \\
  \frac{\partial u}{\partial x} & = 3 x^2 - 3 y^2                                       \\
                                & = \frac{\partial v}{\partial y}                       \\
  \frac{\partial u}{\partial y} & = -6 x y                                              \\
                                & = -\frac{\partial v}{\partial x}
\end{align*}

\setcounter{subsubsection}{2}
\subsubsection{}

\begin{align*}
  f(z)                          & = \Re (z)                         \\
                                & = x                               \\
  \frac{\partial u}{\partial x} & = 1                               \\
                                & \ne \frac{\partial v}{\partial y}
\end{align*}

\setcounter{subsubsection}{4}
\subsubsection{}

\begin{align*}
  f(z)                          & = 4 z - 6 \overline{z} + 3        \\
                                & = 4 (x + i y) - 6 (x - i y) + 3   \\
                                & = (-2 x + 3) + 10 i y             \\
  \frac{\partial u}{\partial x} & = -2                              \\
                                & \ne \frac{\partial v}{\partial y}
\end{align*}

\setcounter{subsubsection}{6}
\subsubsection{}

\begin{align*}
  f(z)                          & = x^2 + y^2                       \\
  \frac{\partial u}{\partial x} & = 2 x                             \\
                                & \ne \frac{\partial v}{\partial y}
\end{align*}

\setcounter{subsubsection}{8}
\subsubsection{}

\begin{align*}
  f(z)                          & = e^x \cos y + i e^x \sin y \\
  u                             & = e^x \cos y                \\
  \frac{\partial u}{\partial x} & = e^x \cos y                \\
  \frac{\partial u}{\partial y} & = -e^x \sin y               \\
  v                             & = e^x \sin y                \\
  \frac{\partial v}{\partial x} & = e^x \sin y                \\
  \frac{\partial v}{\partial y} & = e^x \cos y                \\
\end{align*}

Analytic everywhere.

\setcounter{subsubsection}{10}
\subsubsection{}

\begin{align*}
  f(z)                          & = x + \sin x \cosh y + i (y + \cos x \sinh y) \\
  u                             & = x + \sin x \cosh y                          \\
  \frac{\partial u}{\partial x} & = 1 + \cos x \cosh y                          \\
  \frac{\partial u}{\partial y} & = \sin x \sinh y                              \\
  v                             & = y + \cos x \sinh y                          \\
  \frac{\partial v}{\partial x} & = -\sin x \sinh y                             \\
  \frac{\partial v}{\partial y} & = 1 + \cos x \cosh y
\end{align*}

Analytic everywhere.

\setcounter{subsubsection}{14}
\subsubsection{}

\begin{align*}
  f(z)                          & = 3 x - y + 5 + i (a x + b y - 3) \\
  u                             & = 3 x - y + 5                     \\
  \frac{\partial u}{\partial x} & = 3                               \\
  \frac{\partial u}{\partial y} & = -1                              \\
  v                             & = a x + b y - 3                   \\
  \frac{\partial v}{\partial x} & = a                               \\
  \frac{\partial v}{\partial y} & = b                               \\
  a                             & = 1                               \\
  b                             & = 3
\end{align*}

\setcounter{subsubsection}{16}
\subsubsection{}

\begin{align*}
  f(z)                          & = x^2 + y^2 + 2 i x y \\
  u                             & = x^2 + y^2           \\
  \frac{\partial u}{\partial x} & = 2 x                 \\
  \frac{\partial u}{\partial y} & = 2 y                 \\
  v                             & = 2 x y               \\
  \frac{\partial v}{\partial x} & = 2 y                 \\
  \frac{\partial v}{\partial y} & = 2 x
\end{align*}

Only differentiable when $y = 0$.

\setcounter{subsubsection}{18}
\subsubsection{}

\begin{align*}
  f(z)                          & = x^3 + 3 x y^2 - x + i (y^3 + 3 x^2 y - y) \\
  u                             & = x^3 + 3 x y^2 - x                         \\
  \frac{\partial u}{\partial x} & = 3 x^2 + 3 y^2 - 1                         \\
  \frac{\partial u}{\partial y} & = 6 x y                                     \\
  v                             & = y^3 + 3 x^2 y - y                         \\
  \frac{\partial v}{\partial x} & = 6 x y                                     \\
  \frac{\partial v}{\partial y} & = 3 y^2 + 3 x^2 - 1
\end{align*}

Only differentiable when $x = 0$ or $y = 0$.

\setcounter{subsubsection}{20}
\subsubsection{}

\begin{align*}
  f(z)  & = e^x \cos y + i e^x \sin y                                       \\
  f'(z) & = \frac{\partial u}{\partial x} + i \frac{\partial v}{\partial x} \\
        & = e^x \cos y + i e^x \sin y
\end{align*}

\setcounter{subsubsection}{22}
\subsubsection{}

\begin{align*}
  u                                 & = x             \\
  \frac{\partial^2 u}{\partial x^2} & = 0             \\
  \frac{\partial^2 u}{\partial y^2} & = 0             \\
  \frac{\partial v}{\partial y}     & = 1             \\
  v                                 & = y + h(x)      \\
  h'(x)                             & = 0             \\
  v                                 & = y + c         \\
  f(z)                              & = x + i (y + c)
\end{align*}

\setcounter{subsubsection}{24}
\subsubsection{}

\begin{align*}
  u                                 & = x^2 - y^2                   \\
  \frac{\partial^2 u}{\partial x^2} & = 2                           \\
  \frac{\partial^2 u}{\partial y^2} & = -2                          \\
  \frac{\partial v}{\partial y}     & = 2 x                         \\
  v                                 & = 2 x y + h(x)                \\
  2 y                               & = 2 y + h'(x)                 \\
  h'(x)                             & = 0                           \\
  h(x)                              & = c                           \\
  v                                 & = 2 x y + c                   \\
  f(z)                              & = (x^2 - y^2) + i (2 x y + c)
\end{align*}

\subsection{Exponential and Logarithmic Functions}

\subsubsection{}

\[\frac{\sqrt{3}}{2} + \frac{1}{2} i\]

\setcounter{subsubsection}{2}
\subsubsection{}

\[e^{-1} \frac{\sqrt{2}}{2} (1 + i)\]

\setcounter{subsubsection}{4}
\subsubsection{}

\[-e^\pi\]

\setcounter{subsubsection}{6}
\subsubsection{}

\[e^{1.5} (\cos 2 + i \sin 2) = -1.865 + 4.075 i\]

\setcounter{subsubsection}{8}
\subsubsection{}

\[\cos 5 + i \sin 5 = 0.2836 - 0.9589 i\]

\setcounter{subsubsection}{10}
\subsubsection{}

\begin{align*}
  e^{1 + 5 \pi i / 4} e^{-1 - \pi i / 3} & = e^{11 \pi i / 12}                                 \\
                                         & = \cos \frac{11 \pi}{12} + i \sin \frac{11 \pi}{12} \\
                                         & = -0.9659 + 0.2588 i
\end{align*}

\setcounter{subsubsection}{12}
\subsubsection{}

\begin{align*}
  f(z) & = e^{-i z}                \\
       & = e^{-i (x + i y)}        \\
       & = e^{y - i x}             \\
       & = e^y (\cos x - i \sin x)
\end{align*}

\setcounter{subsubsection}{14}
\subsubsection{}

\begin{align*}
  f(z) & = e^{z^2}                                       \\
       & = e^{x^2 - y^2 + 2 i x y}                       \\
       & = e^{x^2 - y^2} [\cos (2 x y) + i \sin (2 x y)]
\end{align*}

\setcounter{subsubsection}{16}
\subsubsection{}

\begin{align*}
  e^z   & = e^{x + i y}                          \\
        & = e^x (\cos y + i \sin y)              \\
  |e^z| & = \sqrt{e^{2 x} [\cos^2 y + \sin^2 y]} \\
        & = e^x
\end{align*}

\setcounter{subsubsection}{18}
\subsubsection{}

\begin{align*}
  e^{z + \pi i} & = e^{x + i (y + \pi)}                     \\
                & = e^x [\cos (y + \pi) + i \sin (y + \pi)] \\
                & = e^x [-\cos y - i \sin y]                \\
                & = -e^x (\cos y + i \sin y)                \\
  e^{z - \pi i} & = e^{x + i (y - \pi)}                     \\
                & = e^x [\cos (y - \pi) + i \sin (y - \pi)] \\
                & = e^x (-\cos y - i \sin y)                \\
                & = -e^x (\cos y + i \sin y)
\end{align*}

\setcounter{subsubsection}{20}
\subsubsection{}

\begin{align*}
  e^{\overline{z}}              & = e^{x - i y}                     \\
                                & = e^x (\cos y - i \sin y)         \\
  u                             & = e^x \cos y                      \\
  v                             & = -e^x \sin y                     \\
  \frac{\partial u}{\partial x} & = e^x \cos y                      \\
                                & \ne \frac{\partial v}{\partial y}
\end{align*}

\setcounter{subsubsection}{22}
\subsubsection{}

\[\log_e 5 + i (\pi + 2 n \pi) = 1.6094 + i (\pi + 2 n \pi)\]

\setcounter{subsubsection}{24}
\subsubsection{}

\[\log_e (2 \sqrt{2}) + i \left( \frac{3}{4} \pi + 2 n \pi \right) = 1.0397 + i \left( \frac{3}{4} \pi + 2 n \pi \right)\]

\setcounter{subsubsection}{26}
\subsubsection{}

\[\log_e (2 \sqrt{2}) + i \left( \frac{1}{3} \pi + 2 n \pi \right) = 1.0397 + i \left( \frac{1}{3} \pi + 2 n \pi \right)\]

\setcounter{subsubsection}{28}
\subsubsection{}

\[\log_e (6 \sqrt{2}) - \frac{\pi}{4} i = 2.1383 - \frac{\pi}{4} i\]

\setcounter{subsubsection}{30}
\subsubsection{}

\[\log_e 13 + 2.7468 i = 2.5649 + 2.7468 i\]

\setcounter{subsubsection}{32}
\subsubsection{}

\[5 \left( \log_e 2 + \frac{\pi}{3} i \right) = 3.4657 - \frac{\pi}{3} i\]

\setcounter{subsubsection}{34}
\subsubsection{}

\[z = \log_e 4 + i \left( \frac{\pi}{2} + 2 n \pi \right) = 1.3863 + i \left( \frac{\pi}{2} + 2 n \pi \right)\]

\setcounter{subsubsection}{36}
\subsubsection{}

\begin{align*}
  z - 1 & = 2 + i \left( -\frac{\pi}{2} + 2 n \pi \right) \\
  z     & = 3 + i \left( -\frac{\pi}{2} + 2 n \pi \right)
\end{align*}

\setcounter{subsubsection}{38}
\subsubsection{}

\begin{align*}
  \ln (-i)   & = i \left( -\frac{\pi}{2} + 2 n \pi \right) \\
  (-i)^{4 i} & = e^{4 i \ln (-i)}                          \\
             & = e^{4 i \times i (-\pi / 2 + 2 n \pi)}     \\
             & = e^{2 \pi (1 - 4 n)}
\end{align*}

\setcounter{subsubsection}{40}
\subsubsection{}

\begin{align*}
  \ln (1 + i)       & = \log_e \sqrt{2} + i \left( \frac{\pi}{4} + 2 n \pi \right)                            \\
  (1 + i)^{(1 + i)} & = e^{(1 + i) \ln (1 + i)}                                                               \\
                    & = e^{(1 + i) [\log_e \sqrt{2} + i (\pi / 4 + 2 n \pi)]}                                 \\
                    & = e^{\log_e \sqrt{2} + i (\pi / 4 + 2 n \pi) + i \log_e \sqrt{2} - (\pi / 4 + 2 n \pi)} \\
                    & = e^{(\log_e \sqrt{2} - \pi / 4 - 2 n \pi) + i (\log_e \sqrt{2} + \pi / 4 + 2 n \pi)}   \\
                    & = e^{-2 n \pi} e^{(\log_e \sqrt{2} - \pi / 4) + i (\log_e \sqrt{2} + \pi / 4)}          \\
                    & = e^{-2 n \pi} e^{\log_e \sqrt{2} - \pi / 4} e^{i (\log_e \sqrt{2} + \pi / 4)}          \\
                    & = e^{-2 n \pi} (0.2739 + 0.5837 i)
\end{align*}

\setcounter{subsubsection}{42}
\subsubsection{}

\begin{align*}
  \Ln (-1)            & = \pi i                     \\
  (-1)^{(-2 i / \pi)} & = e^{(-2 i / \pi) \Ln (-1)} \\
                      & = e^{(-2 i / \pi) (\pi i)}  \\
                      & = e^2
\end{align*}

\setcounter{subsubsection}{46}
\subsubsection{}

\begin{enumerate}
  \item

        \begin{align*}
          (-1 + i)^2     & = -2 i                                  \\
          \Ln (-1 + i)^2 & = \Ln (-2 i)                            \\
                         & = \log_e 2 - \frac{\pi}{2} i            \\
          2 \Ln (-1 + i) & = 2 \log_e \sqrt{2} + \frac{3 \pi}{2} i \\
                         & \ne \Ln (-1 + i)^2
        \end{align*}

        Not true

  \item

        \begin{align*}
          \Ln i^3 & = \Ln (-i)          \\
                  & = -\frac{\pi}{2} i  \\
          3 \Ln i & = \frac{3 \pi}{2} i \\
                  & \ne \Ln i^3
        \end{align*}

        Not true

  \item

        \begin{align*}
          \ln i^3 & = i \left( -\frac{\pi}{2} + 2 n \pi \right)  \\
          3 \ln i & = 3 i \left( \frac{\pi}{2} + 2 n \pi \right) \\
                  & \ne \ln i^3
        \end{align*}

        Not true
\end{enumerate}

\subsection{Trigonometric and Hyperbolic Functions}

\subsubsection{}

\begin{align*}
  \cos (3 i) & = \cos 0 \cosh 3 - i \sin 0 \sinh 3 \\
             & = \cosh 3                           \\
             & = 10.0677
\end{align*}

\setcounter{subsubsection}{2}
\subsubsection{}

\begin{align*}
  \sin (\pi / 4 + i) & = \sin \frac{\pi}{4} \cosh 1 + i \cos \frac{\pi}{4} \sinh 1 \\
                     & = 1.0911 + 0.8309 i
\end{align*}

\setcounter{subsubsection}{4}
\subsubsection{}

\begin{align*}
  \tan i & = \frac{\sin i}{\cos i}                                                       \\
         & = \frac{\sin 0 \cosh 1 + i \cos 0 \sinh 1}{\cos 0 \cosh 1 + i \sin 0 \sinh 1} \\
         & = \frac{i \sinh 1}{\cosh 1}                                                   \\
         & = i \tanh 1                                                                   \\
         & = 0.7615 i
\end{align*}

\setcounter{subsubsection}{6}
\subsubsection{}

\begin{align*}
  \sec (\pi + i) & = \frac{1}{\cos (\pi + i)}                      \\
                 & = \frac{1}{\cos \pi \cosh 1 + \sin \pi \sinh 1} \\
                 & = -\frac{1}{\cosh 1}                            \\
                 & = -0.6480
\end{align*}

\setcounter{subsubsection}{8}
\subsubsection{}

\begin{align*}
  \cosh (\pi i) & = \cosh 0 \cos \pi + i \sinh 0 \sin \pi \\
                & = -1
\end{align*}

\setcounter{subsubsection}{10}
\subsubsection{}

\begin{align*}
  \sinh (1 + \pi i / 3) & = \sinh 1 \cos (\pi / 3) + i \cosh 1 \sin (\pi / 3) \\
                        & = 0.5876 + 1.3363 i
\end{align*}

\setcounter{subsubsection}{14}
\subsubsection{}

\begin{align*}
  \sin z                         & = 2                                               \\
  \frac{e^{i z} - e^{-i z}}{2 i} & = 2                                               \\
  e^{i z} - e^{-i z}             & = 4 i                                             \\
  e^{2 i z} - 1                  & = 4 i e^{i z}                                     \\
  e^{2 i z} - 4 i e^{i z} - 1    & = 0                                               \\
  e^{i z}                        & = \frac{4 i \pm \sqrt{-16 + 4}}{2}                \\
                                 & = (2 \pm \sqrt{3}) i                              \\
  i z                            & = \log_e (2 \pm \sqrt{3}) + i (\pi / 2 + 2 n \pi) \\
  z                              & = (\pi / 2 + 2 n \pi) - i \log_e (2 \pm \sqrt{3}) \\
\end{align*}

\setcounter{subsubsection}{16}
\subsubsection{}

\begin{align*}
  \sinh z                & = -i                                        \\
  \frac{e^z - e^{-z}}{2} & = -i                                        \\
  e^{2 z} + 2 i e^z - 1  & = 0                                         \\
  e^z                    & = \frac{-2 i \pm \sqrt{-4 + 4}}{2}          \\
                         & = -i                                        \\
  z                      & = \ln (-i)                                  \\
                         & = i \left( -\frac{\pi}{2} + 2 n \pi \right)
\end{align*}

\setcounter{subsubsection}{18}
\subsubsection{}

\begin{align*}
  \cos z                       & = \sin z                                   \\
  \frac{e^{i z} + e^{-i z}}{2} & = \frac{e^{i z} - e^{-i z}}{2 i}           \\
  e^{i z} + e^{-i z}           & = \frac{e^{i z} - e^{-i z}}{i}             \\
                               & = -i (e^{i z} - e^{-i z})                  \\
  e^{2 i z} + 1                & = -i (e^{2 i z} - 1)                       \\
  e^{2 i z} (1 + i)            & = -1 + i                                   \\
  e^{2 i z}                    & = \frac{-1 + i}{1 + i}                     \\
                               & = \frac{(-1 + i)(1 - i)}{(1 + i) (1 - i)}  \\
                               & = \frac{-1 + i + i + 1}{1 - i + i + 1}     \\
                               & = \frac{2 i}{2}                            \\
                               & = i                                        \\
  2 i z                        & = \ln i                                    \\
                               & = i \left( \frac{\pi}{2} + 2 n \pi \right) \\
  z                            & = \frac{\pi}{4} + n \pi
\end{align*}

\setcounter{subsubsection}{20}
\subsubsection{}

\begin{align*}
  \cos z                            & = \cosh 2         \\
  \cos x \cosh y - i \sin x \sinh y & = \cosh 2         \\
  y                                 & = \pm 2           \\
  x                                 & = 2 n \pi         \\
  z                                 & = 2 n \pi \pm 2 i
\end{align*}

\subsection{Inverse Trigonometric and Hyperbolic Functions}

\subsubsection{}

\begin{align*}
  \arcsin z            & = -i \ln [i z + (1 - z^2)^{1 / 2}]             \\
  \arcsin (-i)         & = -i \ln [i (-i) + (1 - (-i)^2)^{1 / 2}]       \\
                       & = -i \ln [1 \pm \sqrt{2}]                      \\
  \ln (1 + \sqrt{2})   & = \log_e (1 + \sqrt{2}) + 2 n \pi i            \\
  \ln (1 - \sqrt{2})   & = \ln \left( -\frac{1}{1 + \sqrt{2}} \right)   \\
                       & = -\ln [-(1 + \sqrt{2})]                       \\
                       & = -[\log_e (1 + \sqrt{2}) + i (\pi + 2 n \pi)] \\
                       & = -\log_e (1 + \sqrt{2}) + i (\pi + 2 n \pi)   \\
  \ln (1 \pm \sqrt{2}) & = (-1)^n \log_e (1 + \sqrt{2}) + n \pi i       \\
  \arcsin (-i)         & = -i [(-1)^n \log_e (1 + \sqrt{2}) + n \pi i]  \\
                       & = n \pi - (-1)^n i \log_e (1 + \sqrt{2})       \\
                       & = n \pi + (-1)^{n + 1} i \log_e (1 + \sqrt{2})
\end{align*}

\setcounter{subsubsection}{2}
\subsubsection{}

\begin{align*}
  \arcsin 0 & = -i \ln (\pm 1) \\
            & = -i (n \pi i)   \\
            & = n \pi
\end{align*}

\setcounter{subsubsection}{4}
\subsubsection{}

\begin{align*}
  \arccos 2            & = -i \ln [2 + i (1 - 2^2)^{1 / 2}]      \\
                       & = -i \ln [2 \pm \sqrt{3}]               \\
  \ln (2 + \sqrt{3})   & = \log_e (2 + \sqrt{3}) + 2 n \pi i     \\
  \ln (2 - \sqrt{3})   & = \log_e (2 - \sqrt{3}) + 2 n \pi i     \\
                       & = -\log_e (2 + \sqrt{3}) + 2 n \pi i    \\
  \ln (2 \pm \sqrt{3}) & = \pm \log_e (2 + \sqrt{3}) + 2 n \pi i \\
  \arccos 2            & = 2 n \pi \pm i \log_e (2 + \sqrt{3})
\end{align*}

\setcounter{subsubsection}{6}
\subsubsection{}

\begin{align*}
  \arccos \frac{1}{2}                                     & = -i \ln \left\{ \frac{1}{2} + i \left[ 1 - \left( \frac{1}{2} \right)^2 \right]^{1 / 2} \right\} \\
                                                          & = -i \ln \left( \frac{1}{2} \pm \frac{\sqrt{3}}{2} i \right)                                      \\
  \ln \left( \frac{1}{2} \pm \frac{\sqrt{3}}{2} i \right) & = i \left( \pm \frac{\pi}{3} + 2 n \pi \right)                                                    \\
  \arccos \frac{1}{2}                                     & = \pm \frac{\pi}{3} + 2 n \pi
\end{align*}

\setcounter{subsubsection}{8}
\subsubsection{}

\begin{align*}
  \arctan 1            & = \frac{i}{2} \ln \frac{1 + i}{-1 + i}       \\
  \frac{1 + i}{-1 + i} & = \frac{(1 + i) (-1 - i)}{(-1 + i) (-1 - i)} \\
                       & = -i                                         \\
  \ln (-i)             & = i \left( -\frac{\pi}{2} + 2 n \pi \right)  \\
  \arctan 1            & = \frac{\pi}{4} + n \pi
\end{align*}

\setcounter{subsubsection}{10}
\subsubsection{}

\begin{align*}
  \arcsinh \frac{4}{3}                         & = \ln \left\{ \frac{4}{3} + \left[ \left( \frac{4}{3} \right)^2 + 1 \right]^{1 / 2} \right\} \\
                                               & = \ln \left( \frac{4}{3} \pm \frac{5}{3} \right)                                             \\
  \ln \left( \frac{4}{3} + \frac{5}{3} \right) & = \ln \frac{9}{3}                                                                            \\
                                               & = \ln 3                                                                                      \\
                                               & = \log_e 3 + 2 n \pi i                                                                       \\
  \ln \left( \frac{4}{3} - \frac{5}{3} \right) & = \ln \left( -\frac{1}{3} \right)                                                            \\
                                               & = \log_e \frac{1}{3} + i (\pi + 2 n \pi)                                                     \\
                                               & = -\log_e 3 + i (\pi + 2 n \pi)                                                              \\
  \arcsinh \frac{4}{3}                         & = (-1)^n \log_e 3 + n \pi i                                                                  \\
\end{align*}

\subsection{Chapter in Review}

\subsubsection{}

$0$, $32$

\setcounter{subsubsection}{2}
\subsubsection{}

\begin{align*}
  \frac{3 + 4 i}{3 - 4 i}                   & = \frac{(3 + 4 i)^2}{(3 - 4 i) (3 + 4 i)} \\
                                            & = \frac{-7 + 24 i}{25}                    \\
                                            & = -\frac{7}{25} + \frac{24}{25} i         \\
  \Re \left( \frac{z}{\overline{z}} \right) & = -\frac{7}{25}
\end{align*}

\setcounter{subsubsection}{4}
\subsubsection{}

\begin{align*}
  \frac{4 i}{-3 - 4 i} & = \frac{(4 i) (-3 + 4 i)}{(-3 - 4 i) (-3 + 4 i)}                         \\
                       & = \frac{-16 - 12 i}{25}                                                  \\
                       & = -\frac{16}{25} - \frac{12}{25} i                                       \\
  |z|                  & = \sqrt{\left( \frac{16}{25} \right)^2 + \left( \frac{12}{25} \right)^2} \\
                       & = \frac{4}{5}
\end{align*}

\setcounter{subsubsection}{6}
\subsubsection{}

False

\setcounter{subsubsection}{8}
\subsubsection{}

\begin{align*}
  e^z & = 2 i                                                 \\
  z   & = \ln (2 i)                                           \\
      & = \log_e 2 + i \left( \frac{\pi}{2} + 2 n \pi \right)
\end{align*}

\setcounter{subsubsection}{10}
\subsubsection{}

\begin{align*}
  (1 + i)^{(2 + i)}                                        & = e^{(2 + i) \ln (1 + i)}                                                                                                                                    \\
  ln (1 + i)                                               & = \log_e \sqrt{2} + \frac{\pi}{4} i                                                                                                                          \\
  (2 + i) \left( \log_e \sqrt{2} + \frac{\pi}{4} i \right) & = 2 \log_e \sqrt{2} + \frac{\pi}{2} i + i \log_e \sqrt{2} - \frac{\pi}{4}                                                                                    \\
                                                           & = \left( 2 \log_e \sqrt{2} - \frac{\pi}{4} \right) + i \left( \log_e \sqrt{2} + \frac{\pi}{2} \right)                                                        \\
  (1 + i)^{(2 + i)}                                        & = e^{2 \log_e \sqrt{2} - \pi / 4} \left[ \cos \left( \log_e \sqrt{2} + \frac{\pi}{2} \right) + i \sin \left( \log_e \sqrt{2} + \frac{\pi}{2} \right) \right] \\
                                                           & \approx -0.3097 + 0.8576 i
\end{align*}

\setcounter{subsubsection}{12}
\subsubsection{}

False

\setcounter{subsubsection}{14}
\subsubsection{}

\[\Ln (-i e^3) = 3 - \frac{\pi}{2} i\]

\setcounter{subsubsection}{20}
\subsubsection{}

\begin{align*}
  z^2       & = x^2 - y^2 + 2 i x y \\
  \Im (z^2) & \le 2                 \\
  2 x y     & \le 2
\end{align*}

\setcounter{subsubsection}{22}
\subsubsection{}

\[\frac{1}{\sqrt{x^2 + y^2}} \le 1\]

\setcounter{subsubsection}{26}
\subsubsection{}

\begin{align*}
  z^4 & = 1 - i                                    \\
  z_k & = 2^{1 / 8} e^{(-\pi / 4 + 2 k \pi) i / 4} \\
      & = 2^{1 / 8} e^{i (k \pi / 2 - \pi / 16)}   \\
  z_0 & = 1.0695 - 0.2127 i                        \\
  z_1 & = 0.2127 + 1.0695 i                        \\
  z_2 & = -1.0695 + 0.2127 i                       \\
  x_3 & = -0.2127 - 1.0695 i
\end{align*}

\section{Integration in the Complex Plane}

\subsection{Contour Integrals}

\subsubsection{}

\begin{align*}
  z(t)              & = 2 t + i (4 t - 1)                                                     \\
  z'(t)             & = 2 + 4 i                                                               \\
  f(z(t))           & = (2 t + 3) + i (4 t - 1)                                               \\
  f(z(t)) z'(t)     & = [(2 t + 3) + i (4 t - 1)] (2 + 4 i)                                   \\
                    & = (2 t + 3) (2) + (2 t + 3) (4 i) + i (4 t - 1) (2) + i (4 t - 1) (4 i) \\
                    & = 4 t + 6 + 8 i t + 12 i + 8 i t - 2 i - 16 t + 4                       \\
                    & = (-12 t + 10) + i (16 t + 10)                                          \\
  \int_C f(z) \,d z & = \int_1^3 f(z(t)) z'(t) \,d t                                          \\
                    & = \int_1^3 (-12 t + 10) \,d t + i \int_1^3 (16 t + 10) \,d t            \\
                    & = -28 + 84 i
\end{align*}

\setcounter{subsubsection}{2}
\subsubsection{}

\begin{align*}
  z(t)              & = 3 t + 2 i t                                 \\
  z'(t)             & = 3 + 2 i                                     \\
  \int_C f(z) \,d z & = \int_{-2}^2 (3 t + 2 i t)^2 (3 + 2 i) \,d t \\
                    & = \int_{-2}^2 [(3 + 2 i) t]^2 (3 + 2 i) \,d t \\
                    & = (3 + 2 i)^3 \int_{-2}^2 t^2 \,d t           \\
                    & = (-9 + 46 i) \frac{16}{3}                    \\
                    & = -48 + \frac{736}{3} i
\end{align*}

\setcounter{subsubsection}{4}
\subsubsection{}

\begin{align*}
  z(t)              & = e^{i t}                                                                             \\
  z'(t)             & = i e^{i t}                                                                           \\
  \int_C f(z) \,d z & = \int_{-\pi / 2}^{\pi / 2} \frac{1 + e^{i t}}{e^{i t}} i e^{i t} \,d t               \\
                    & = i \int_{-\pi / 2}^{\pi / 2} (1 + e^{i t}) \,d t                                     \\
                    & = i \left[ t + \frac{1}{i} e^{i t} \right]_{-\pi / 2}^{\pi / 2}                       \\
                    & = i [t - i e^{i t}]_{-\pi / 2}^{\pi / 2}                                              \\
                    & = i \left( \frac{\pi}{2} - i e^{\pi i / 2} + \frac{\pi}{2} + i e^{-\pi i / 2} \right) \\
                    & = i (\pi + 2)
\end{align*}

\setcounter{subsubsection}{6}
\subsubsection{}

\begin{align*}
  z(t)              & = \cos t + i \sin t                                                      \\
  z'(t)             & = -\sin t + i \cos t                                                     \\
  \int_C f(z) \,d z & = \int_0^{2 \pi} \cos t (-\sin t + i \cos t) \,d t                       \\
                    & = \int_0^{2 \pi} \left( -\frac{1}{2} \sin 2 t + i \cos^2 t \right) \,d t \\
                    & = \pi i
\end{align*}

\setcounter{subsubsection}{8}
\subsubsection{}

\begin{align*}
  z(t)              & = (1 - t) + i t                                                                  \\
  z'(t)             & = -1 + i                                                                         \\
  \int_C f(z) \,d z & = \int_0^1 [(1 - t)^2 + i t^3] (-1 + i) \,d t                                    \\
                    & = \int_0^1 (1 - 2 t + t^2 + i t^3) (-1 + i) \,d t                                \\
                    & = \int_0^1 (-1 + i + 2 t - 2 i t - t^2 + i t^2 - i t^3 - t^3) \,d t              \\
                    & = \int_0^1 (-1 + 2 t - t^2 - t^3) \,d t + i \int_0^1 (1 - 2 t + t^2 - t^3) \,d t \\
                    & = -\frac{7}{12} + \frac{1}{12} i
\end{align*}

\setcounter{subsubsection}{16}
\subsubsection{}

\begin{align*}
  % 1
  z(t)                  & = 1 + i t                           \\
  z'(t)                 & = i                                 \\
  \int_{C_1} f(z) \,d z & = \int_0^1 i \,d t                  \\
                        & = i                                 \\
  % 2
  z(t)                  & = (1 - t) + i (1 - t)               \\
  z'(t)                 & = -(1 + i)                          \\
  \int_{C_2} f(z) \,d z & = -\int_0^1 (1 - t) (1 + i) \,d t   \\
                        & = -\int_0^1 (1 + i - t - i t) \,d t \\
                        & = -\frac{1}{2} - \frac{1}{2} i      \\
  % 3
  z(t)                  & = t                                 \\
  z'(t)                 & = 1                                 \\
  \int_{C_3} f(z) \,d z & = \int_0^1 t \,d t                  \\
                        & = \frac{1}{2}                       \\
  % Total
  \int_C f(z) \,d z     & = \frac{1}{2} i
\end{align*}

\setcounter{subsubsection}{18}
\subsubsection{}

\begin{align*}
  % 1
  z(t)                  & = 1 + i t                                     \\
  z'(t)                 & = i                                           \\
  \int_{C_1} f(z) \,d z & = \int_0^1 (1 + i t)^2 i \,d t                \\
                        & = i \int_0^1 (1 + 2 i t - t^2) \,d t          \\
                        & = -1 + \frac{2}{3} i                          \\
  % 2
  z(t)                  & = (1 + i) (1 - t)                             \\
  z'(t)                 & = -(1 + i)                                    \\
  \int_{C_2} f(z) \,d z & = -\int_0^1 [(1 + i) (1 - t)]^2 (1 + i) \,d t \\
                        & = -\int_0^1 (1 - t + i - i t)^2 (1 + i) \,d t \\
                        & = \frac{2}{3} - \frac{2}{3} i                 \\
  % 3
  z(t)                  & = t                                           \\
  z'(t)                 & = 1                                           \\
  \int_{C_3} f(z) \,d z & = \int_0^1 t^2 \,d t                          \\
                        & = \frac{1}{3}                                 \\
  % Total
  \int_C f(z) \,d z     & = 0
\end{align*}

\setcounter{subsubsection}{22}
\subsubsection{}

\begin{align*}
  z(t)              & = t + i (1 - t^2)             \\
  z'(t)             & = 1 - 2 i t                   \\
  \int_C f(z) \,d z & = \frac{4}{3} - \frac{5}{3} i
\end{align*}

\setcounter{subsubsection}{24}
\subsubsection{}

\begin{align*}
  L                                              & = 10 \pi                    \\
  |z^2 + 1|                                      & \ge |z^2| - 1               \\
  \left| \frac{e^z}{z^2 + 1} \right|             & \le \frac{|e^z|}{|z^2| - 1} \\
                                                 & = \frac{e^5}{24}            \\
                                                 & = M                         \\
  \left| \oint \frac{e^z}{z^2 + 1} \,d z \right| & \le M L                     \\
                                                 & = \frac{5 \pi e^5}{12}
\end{align*}

\setcounter{subsubsection}{26}
\subsubsection{}

\begin{align*}
  z(t)                                 & = (1 + i) t, \ 0 \le t \le 1 \\
  L                                    & = \sqrt{2}                   \\
  |z^2 + 4|                            & = |2 i t^2 + 4|              \\
                                       & \le |2 i + 4|                \\
                                       & = \sqrt{20}                  \\
                                       & = 2 \sqrt{5}                 \\
                                       & = M                          \\
  \left| \oint (z^2 + 4) \,d z \right| & \le M L                      \\
                                       & = 2 \sqrt{10}
\end{align*}

\setcounter{subsubsection}{32}
\subsubsection{}

\begin{align*}
  z(t)                        & = e^{i t}                                   \\
  z'(t)                       & = i e^{i t}                                 \\
  \oint \overline{f(z)} \,d z & = \int_0^{2 \pi} 2 e^{-i t} i e^{i t} \,d t \\
                              & = 4 \pi i
\end{align*}

The circulation is $0$ and the flux is $4 \pi$.

\subsection{Cauchy-Goursat Theorem}

\subsubsection{}

\begin{align*}
  z                          & = e^{i t},\ 0 \le t \le 2 \pi                                                     \\
  z'                         & = i e^{i t}                                                                       \\
  \int (z^3 - 1 + 3 i) \,d z & = \int_0^{2 \pi} [(e^{i t})^3 - 1 + 3 i] i e^{i t} \,d t                          \\
                             & = i \int_0^{2 \pi} (e^{4 i t} - e^{i t} + 3 i e^{i t}) \,d t                      \\
                             & = \left[ \frac{1}{4} e^{4 i t} - e^{i t} + 3 i e^{i t} \right]_0^{2 \pi}          \\
                             & = \frac{1}{4} e^{8 \pi i} - e^{2 \pi i} + 3 i e^{2 \pi i} - \frac{1}{4} + 1 - 3 i \\
                             & = \frac{1}{4} - 1 + 3 i - \frac{1}{4} + 1 - 3 i                                   \\
                             & = 0
\end{align*}

\setcounter{subsubsection}{8}
\subsubsection{}

\begin{align*}
  \int_C \frac{1}{z} \,d z & = \int_0^{2 \pi} \frac{1}{e^{i t}} i e^{i t} \,d t \\
                           & = 2 \pi i
\end{align*}

\setcounter{subsubsection}{10}
\subsubsection{}

\begin{align*}
  \varointctrclockwise_C \left( z + \frac{1}{z} \right) \,d z & = \varointctrclockwise_C \frac{1}{z^{-1}} \,d z + \varointctrclockwise_C \frac{1}{z} \,d z \\
                                                              & = 0 + 2 \pi i                                                                              \\
                                                              & = 2 \pi i
\end{align*}

\setcounter{subsubsection}{12}
\subsubsection{}

\begin{align*}
  z^2 - \pi^2                                        & = (z + \pi) (z - \pi)                                                                         \\
  \frac{z}{z^2 - \pi^2}                              & = \frac{1 / 2}{z + \pi} + \frac{1 / 2}{z - \pi}                                               \\
  \varointctrclockwise_C \frac{z}{z^2 - \pi^2} \,d z & = \frac{1}{2} \varointctrclockwise \left( \frac{1}{z + \pi} + \frac{1}{z - \pi} \right) \,d z \\
                                                     & = 0
\end{align*}

\setcounter{subsubsection}{14}
\subsubsection{}

\begin{enumerate}
  \item

        \begin{align*}
          \frac{2 z + 1}{z^2 + z}                              & = \frac{2 z + 1}{z (z + 1)}                                                 \\
                                                               & = \frac{1}{z} + \frac{1}{z + 1}                                             \\
          \varointctrclockwise_C \frac{2 z + 1}{z^2 + z} \,d z & = \varointctrclockwise_C \frac{2 z + 1}{z (z + 1)} \,d z                    \\
                                                               & = \varointctrclockwise_C \left( \frac{1}{z} + \frac{1}{z + 1} \right) \,d z \\
                                                               & = 2 \pi i
        \end{align*}

  \item $4 \pi i$

  \item $0$
\end{enumerate}

\setcounter{subsubsection}{16}
\subsubsection{}

\begin{enumerate}
  \item

        \begin{align*}
          \frac{-3 z + 2}{z^2 - 8 z + 12}                              & = \frac{1}{z - 1} - \frac{4}{z - 6}                                           \\
          \varointctrclockwise_C \frac{-3 z + 2}{z^2 - 8 z + 12} \,d z & = \varointctrclockwise \left( \frac{1}{z - 1} - \frac{4}{z - 6} \right) \,d z \\
                                                                       & = -8 \pi i
        \end{align*}

  \item $-6 \pi i$
\end{enumerate}

\setcounter{subsubsection}{18}
\subsubsection{}

\begin{align*}
  \frac{z - 1}{z (z - i) (z - 3 i)}                              & = \frac{1}{3 z} - \frac{1/2 - i / 2}{z - i} + \frac{1 / 6 - i / 2}{z - 3 i} \\
  \varointctrclockwise_C \frac{z - 1}{z (z - i) (z - 3 i)} \,d z & = -\left( \frac{1}{2} - \frac{i}{2} \right) 2 \pi i                         \\
                                                                 & = -\pi (1 + i)
\end{align*}

\setcounter{subsubsection}{20}
\subsubsection{}

\begin{align*}
  \frac{8 z - 3}{z^2 - z}           & = \frac{3}{z} + \frac{5}{z - 1}                                                                                                                                 \\
  \varointctrclockwise_C f(z) \,d z & = \varointctrclockwise_{C_1} f(z) \,d z - \varointctrclockwise_{C_2} f(z) \,d z                                                                                 \\
                                    & = \varointctrclockwise_{C_1} \left( \frac{3}{z} + \frac{5}{z - 1} \right) \,d z - \varointctrclockwise_{C_2} \left( \frac{3}{z} + \frac{5}{z - 1} \right) \,d z \\
                                    & = 6 \pi i - 10 \pi i                                                                                                                                            \\
                                    & = -4 \pi i
\end{align*}

\setcounter{subsubsection}{22}
\subsubsection{}

\begin{align*}
  \varointctrclockwise_C \left( \frac{e^z}{z + 3} - 3 \overline{z} \right) \,d z & = \varointctrclockwise_C \frac{e^z}{z + 3} \,d z - 3 \varointctrclockwise_C \overline{z} \,d z \\
                                                                                 & = -3 \varointctrclockwise_0^{2 \pi} e^{-i t} i e^{i t} \,d t                                   \\
                                                                                 & = -6 \pi i
\end{align*}

\subsection{Independence of the Path}

\subsubsection{}

\begin{enumerate}
  \item

        \begin{align*}
          z(t)                   & = i (t - 1),\ 0 \le t \le 2                                     \\
          z'(t)                  & = i                                                             \\
          \int_C (4 z - 1) \,d z & = \int_0^2 \{4 [i (t - 1)] - 1\} i \,d t                        \\
                                 & = \int_0^2 [4 (1 - t) - i] \,d t                                \\
                                 & = \left[ 4 \left( t - \frac{1}{2} t^2 \right) - i t \right]_0^2 \\
                                 & = -2 i
        \end{align*}

  \item

        \begin{align*}
          F(z)                   & = 2 z^2 - z                           \\
          \int_C (4 z - 1) \,d z & = F(i) - F(-i)                        \\
                                 & = [2 (i)^2 - (i)] - [2 (-i)^2 - (-i)] \\
                                 & = -2 - i + 2 - i                      \\
                                 & = -2 i
        \end{align*}
\end{enumerate}

\setcounter{subsubsection}{2}
\subsubsection{}

\begin{align*}
  z(-1)            & = -2 + 7 i                 \\
  z(1)             & = 2 - i                    \\
  \int_C 2 z \,d z & = z^2|_{-2 + 7 i}^{2 - i}  \\
                   & = (2 - i)^2 - (-2 + 7 i)^2 \\
                   & = 48 + 24 i
\end{align*}

\setcounter{subsubsection}{4}
\subsubsection{}

\begin{align*}
  \int_0^{3 + i} z^2 \,d z & = \left. \frac{1}{3} z^3 \right|_0^{3 + i} \\
                           & = \frac{1}{3} (3 + i)^3                    \\
                           & = 6 + \frac{26}{3} i
\end{align*}

\setcounter{subsubsection}{6}
\subsubsection{}

\begin{align*}
  \int_{1 - i}^{1 + i} z^3 \,d z & = \left. \frac{1}{4} z^4 \right|_{1 - i}^{1 + i} \\
                                 & = \frac{1}{4} [(1 + i)^4 - (1 - i)^4]            \\
                                 & = 0
\end{align*}

\setcounter{subsubsection}{8}
\subsubsection{}

\begin{align*}
  \int_{-i / 2}^{1 - i} (2 z + 1)^2 \,d z & = \left. z + 2 z^2 + \frac{4}{3} z^3 \right|_{-i / 2}^{1 - i} \\
                                          & = -\frac{7}{6} - \frac{22}{3} i
\end{align*}

\setcounter{subsubsection}{10}
\subsubsection{}

\begin{align*}
  \int_{i / 2}^i e^{\pi z} \,d z & = \left. \frac{1}{\pi} e^{\pi z} \right|_{i / 2}^i \\
                                 & = -\frac{1}{\pi} (1 + i)
\end{align*}

\setcounter{subsubsection}{12}
\subsubsection{}

\begin{align*}
  \int_{\pi}^{\pi + 2 i} \sin \frac{z}{2} \,d z & = \left. -2 \cos \frac{z}{2} \right|_\pi^{\pi + 2 i} \\
                                                & = -2 \cos \left( \frac{\pi}{2} + i \right)           \\
                                                & = 2 i \sinh 1                                        \\
                                                & \approx 2.3504 i
\end{align*}

\setcounter{subsubsection}{14}
\subsubsection{}

\begin{align*}
  \int_{\pi i}^{2 \pi i} \cosh z \,d z & = \sinh (2 \pi i) - \sinh (\pi i) \\
                                       & = 0
\end{align*}

\setcounter{subsubsection}{16}
\subsubsection{}

\begin{align*}
  \int_C \frac{1}{z} \,d z & = \Ln 4 e^{\pi i / 2} - \Ln 4 e^{-\pi i / 2}              \\
                           & = \log_e 4 + \frac{\pi}{2} i - \log_e 4 + \frac{\pi}{2} i \\
                           & = \pi i
\end{align*}

\setcounter{subsubsection}{18}
\subsubsection{}

\begin{align*}
  \int_{-4 i}^{4 i} \frac{1}{z^2} \,d z & = \left. -\frac{1}{z} \right|_{-4 i}^{4 i} \\
                                        & = -\frac{1}{4 i} - \frac{1}{4 i}           \\
                                        & = -\frac{1}{2 i}                           \\
                                        & = \frac{i}{2}
\end{align*}

\setcounter{subsubsection}{20}
\subsubsection{}

\begin{align*}
  \int_\pi^i e^z \cos z \,d z & = \frac{1}{2} \int_\pi^i [e^{z (1 + i)} + e^{z (1 - i)}] \,d z                                              \\
                              & = \frac{1}{2} \left( \left. \frac{e^{z (1 + i)}}{1 + i} + \frac{e^{z (1 - i)}}{1 - i} \right|_\pi^i \right) \\
                              & \approx 11.4928 + 0.9667 i
\end{align*}

\setcounter{subsubsection}{22}
\subsubsection{}

\begin{align*}
  \int_i^{1 + i} z e^z \,d z & = z e^z|_i^{1 + i} - \int_i^{1 + i} e^z \,d z \\
                             & = (1 + i) e^{1 + i} - i e^i - [e^z]_i^{1 + i} \\
                             & = (1 + i) e^{1 + i} - i e^i - e^{1 + i} + e^i \\
                             & \approx -0.9055 + 1.7698 i
\end{align*}

\subsection{Cauchy's Integral Formulas}

\subsubsection{}

$8 \pi i$

\setcounter{subsubsection}{2}
\subsubsection{}

$2 \pi i e^{\pi i} = -2 \pi i$

\setcounter{subsubsection}{4}
\subsubsection{}

$2 \pi i [(-2 i)^2 - 3 (-2 i) + 4 i] = 2 \pi i (-4 + 10 i) = -20 \pi - 8 \pi i$

\setcounter{subsubsection}{6}
\subsubsection{}

\begin{enumerate}
  \item


        \begin{align*}
          \varointctrclockwise_C \frac{z^2}{z^2 + 4} \,d z & = \varointctrclockwise_C \frac{\frac{z^2}{z + 2 i}}{z - 2 i} \,d z \\
                                                           & = 2 \pi i \frac{(2 i)^2}{(2 i) + 2 i}                              \\
                                                           & = -2 \pi
        \end{align*}

  \item

        \begin{align*}
          \varointctrclockwise_C \frac{z^2}{z^2 + 4} \,d z & = \varointctrclockwise_C \frac{\frac{z^2}{z - 2 i}}{z + 2 i} \,d z \\
                                                           & = 2 \pi i \frac{(-2 i)^2}{(-2 i) - 2 i}                            \\
                                                           & = 2 \pi i \frac{-4}{-4 i}                                          \\
                                                           & = 2 \pi
        \end{align*}
\end{enumerate}

\setcounter{subsubsection}{8}
\subsubsection{}

\begin{align*}
  \varointctrclockwise_C \frac{z^2 + 4}{z^2 - 5 i z - 4} \,d z & = \varointctrclockwise_C \frac{z^2 + 4}{(z - i) (z - 4 i)} \,d z     \\
                                                               & = \varointctrclockwise_C \frac{\frac{z^2 + 4}{z - i}}{z - 4 i} \,d z \\
                                                               & = -8 \pi
\end{align*}

\setcounter{subsubsection}{10}
\subsubsection{}

\begin{align*}
  \frac{2 \pi i}{2!} \left. \frac{d^2}{d z^2} (e^{z^2}) \right|_{z = i} & = \pi i \left. \frac{d}{d z} (2 z e^{z^2}) \right|_{z = i} \\
                                                                        & = \pi i \left. (2 e^{z^2} + 4 z^2 e^{z^2}) \right|_{z = i} \\
                                                                        & = \pi i (2 e^{-1} - 4 e^{-1})                              \\
                                                                        & = -\frac{2 \pi}{e} i
\end{align*}

\setcounter{subsubsection}{12}
\subsubsection{}

\begin{align*}
  \varointctrclockwise_C \frac{\cos 2 z}{z^5} \,d z & = \frac{2 \pi i}{4!} \left. \frac{d^4}{d z^4} (\cos 2 z) \right|_{z = 0} \\
                                                    & = \frac{\pi}{12} i (16 \cos 2 z)|_{z = 0}                                \\
                                                    & = \frac{4}{3} \pi i
\end{align*}

\setcounter{subsubsection}{18}
\subsubsection{}

\begin{align*}
  \varointctrclockwise_C \left( \frac{e^{2 i z}}{z^4} - \frac{z^4}{(z - i)^3} \right) \,d z & = \frac{2 \pi i}{3!} \left. \frac{d^3}{d z^3} (e^{2 i z}) \right|_{z = 0} - \frac{2 \pi i}{2!} \left. \frac{d^2}{d z^2} (z^4) \right|_{z = i} \\
                                                                                            & = \frac{\pi}{3} i (-8 i e^{2 i z})|_{z = 0} - \pi i (12 z^2)|_{z = i}                                                                         \\
                                                                                            & = \frac{8}{3} \pi + 12 \pi i
\end{align*}

\setcounter{subsubsection}{20}
\subsubsection{}

\begin{align*}
  \varointctrclockwise_C \frac{1}{z^3 (z - 1)^2} \,d z & = \varointclockwise_{C_1} \frac{\frac{1}{(z - 1)^2}}{z^3} \,d z + \varointctrclockwise_{C_2} \frac{\frac{1}{z^3}}{(z - 1)^2} \,d z \\
                                                       & = \frac{2 \pi i}{2!} \left. \frac{d^2}{d z^2} \left( \frac{1}{(z - 1)^2} \right) \right|_{z = 0}                                   \\
                                                       & \qquad + 2 \pi i \left. \frac{d}{d z} \left( \frac{1}{z^3} \right) \right|_{z = 1}                                                 \\
                                                       & = 6 \pi i - 6 \pi i                                                                                                                \\
                                                       & = 0
\end{align*}

\setcounter{subsubsection}{22}
\subsubsection{}

\begin{align*}
  \varointctrclockwise_C \frac{3 z + 1}{z (z - 2)^2} \,d z & = \varointctrclockwise_{C_1} \frac{\frac{3 z + 1}{z}}{(z - 2)^2} \,d z - \varointctrclockwise_{C_2} \frac{\frac{3 z + 1}{(z - 2)^2}}{z} \,d z \\
                                                           & = 2 \pi i \left. \frac{d}{d z} \left( \frac{3 z + 1}{z} \right) \right|_{z = 2} - 2 \pi i \left. \frac{3 z + 1}{(z - 2)^2} \right|_{z = 0}    \\
                                                           & = -\frac{1}{2} pi i - \frac{1}{2} \pi i                                                                                                       \\
                                                           & = -\pi i
\end{align*}

\subsection{Chapter in Review}

\subsubsection{}

True

\setcounter{subsubsection}{2}
\subsubsection{}

True

\setcounter{subsubsection}{4}
\subsubsection{}

$0$

\setcounter{subsubsection}{6}
\subsubsection{}

\begin{align*}
  \varointctrclockwise_C \frac{z^3 + e^z}{(z + \pi i)^3} \,d z & = \frac{2 \pi i}{2!} \left. \frac{d^2}{d z^2} (z^3 + e^z) \right|_{z = -\pi i} \\
                                                               & = \pi i (6 z + e^z)|_{z = -\pi i}                                              \\
                                                               & = \pi i (-6 \pi i - 1)                                                         \\
                                                               & = 6 \pi^2 - \pi i
\end{align*}

\setcounter{subsubsection}{8}
\subsubsection{}

\begin{align*}
  \varointctrclockwise_C \frac{1}{(z - z_0) (z - z_1)} \,d z & = \varointctrclockwise_{C_1} \frac{\frac{1}{z - z_0}}{z - z_1} \,d z + \varointctrclockwise_{C_2} \frac{\frac{1}{z - z_1}}{z - z_0} \,d z \\
                                                             & = 2 \pi i \frac{1}{z_1 - z_0} + 2 \pi i \frac{1}{z_0 - z_1}                                                                               \\
                                                             & = 0
\end{align*}

True

\setcounter{subsubsection}{10}
\subsubsection{}

$2 \pi i$ if $n = -1$, $0$ otherwise.

\setcounter{subsubsection}{12}
\subsubsection{}

\begin{align*}
  z_1                    & = -4 + i y, \ 0 \le y \le 2                                                                                                                      \\
  z_1'                   & = i                                                                                                                                              \\
  z_2                    & = x + 2 i, \ -4 \le x \le 3                                                                                                                      \\
  z_2'                   & = 1                                                                                                                                              \\
  z_3                    & = 3 + i (2 - y), \ 0 \le y \le 2                                                                                                                 \\
  z_3'                   & = -i                                                                                                                                             \\
  \int_C (x + i y) \,d z & = \int_0^2 (-4 + i y) i \,d y + \int_{-4}^3 (x + 2 i) \,d x + \int_0^2 [3 + i (2 - y)] (-i) \,d y                                                \\
                         & = \int_0^2 (-y - 4 i) \,d y + \int_{-4}^3 (x + 2 i) \,d x - \int_0^2 [(y - 2) + 3 i] \,d y                                                       \\
                         & = \left[ -\frac{1}{2} y^2 - 4 i y \right]_0^2 + \left[ \frac{1}{2} x^2 + 2 i x \right]_{-4}^3 - \left[ \frac{1}{2} y^2 - 2 y + 3 i y \right]_0^2 \\
                         & = -2 - 8 i + \frac{9}{2} + 6 i - 8 + 8 i - 2 + 4 - 6 i                                                                                           \\
                         & = -\frac{7}{2}
\end{align*}

\setcounter{subsubsection}{14}
\subsubsection{}

\begin{align*}
  \int_C |z^2| \,d z & = \int_0^2 |(t + i t^2)^2| (1 + 2 i t) \,d t \\
                     & = \int_0^2 (t^2 + t^4) (1 + 2 i t) \,d t     \\
                     & = \frac{136}{15} + \frac{88}{3} i
\end{align*}

\setcounter{subsubsection}{16}
\subsubsection{}

$0$

\setcounter{subsubsection}{18}
\subsubsection{}

\begin{align*}
  z(-1)               & = 1                          \\
  z(1)                & = 1 + 4 i                    \\
  \int_C \sin z \,d z & = -\cos z|_1^{1 + 4 i}       \\
                      & = \cos 1 - \cos (1 + 4 i)    \\
                      & \approx -14.2144 + 22.9637 i
\end{align*}

\setcounter{subsubsection}{20}
\subsubsection{}

$2 \pi i$

\setcounter{subsubsection}{22}
\subsubsection{}

\begin{align*}
  \varointctrclockwise_C \frac{e^{-2 z}}{z^4} \,d z & = \frac{2 \pi i}{3!} \left. \frac{d^3}{d z^3} (e^{-2 z}) \right|_{z = 0} \\
                                                    & = -\frac{8}{3} \pi i
\end{align*}

\setcounter{subsubsection}{24}
\subsubsection{}

\begin{align*}
  \varointctrclockwise_C \frac{1}{2 z^2 + 7 z + 3} \,d z & = \varointctrclockwise_C \frac{\frac{1}{z + 3}}{2 z + 1} \,d z             \\
                                                         & = \varointctrclockwise_C \frac{\frac{1}{2 (z + 3)}}{z + \frac{1}{2}} \,d z \\
                                                         & = 2 \pi i \frac{1}{2 \left( -\frac{1}{2} + 3 \right)}                      \\
                                                         & = \frac{2}{5} \pi i
\end{align*}

\setcounter{subsubsection}{26}
\subsubsection{}

$2 \pi$

\end{document}