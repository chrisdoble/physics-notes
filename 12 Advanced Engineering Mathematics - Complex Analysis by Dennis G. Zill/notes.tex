\documentclass{article}
\usepackage{amsmath} % For align*
\usepackage{graphicx} % For images

\graphicspath{{./images/}}

\newcommand{\Arg}{\operatorname{Arg}}
\renewcommand{\Im}{\operatorname{Im}}
\renewcommand{\Re}{\operatorname{Re}}

\title{Advanced Engineering Mathematics Complex Analysis by Dennis G. Zill Notes}
\author{Chris Doble}
\date{February 2024}

\begin{document}

\maketitle

\tableofcontents

\setcounter{section}{16}
\section{Functions of a Complex Variable}

\subsection{Complex Numbers}

\begin{itemize}
  \item A \textbf{complex number} is any number of the form \[z = a + i b\] where $a$ and $b$ are real numbers and $i$ is the imaginary unit such that $i^2 = -1$.

  \item The real number $a$ in the above complex number $z$ is called the \textbf{real part} of $z$ and the real number $b$ (not $i b$) is called the \textbf{imaginary part} of $z$.

  \item The real and imaginary parts of a complex number $z$ are denoted $\Re(z)$ and $\Im(z)$, respectively.

  \item A real constant multiple of the imaginary unit, e.g. $6 i$ is called a \textbf{pure imaginary number}.

  \item Two complex numbers are equal if their real and imaginary parts are equal.

  \item The addition and subtraction of complex numbers occur between the real and imaginary parts, e.g. \[(a + b i) + (c + d i) = (a + c) + (b + d) i.\]

  \item The multiplication of complex numbers occurs elementwise as normal, e.g. \[(a + b i) (c + d i) = a c + a d i + b c i - b d.\]

  \item The \textbf{conjugate} of a complex number $z = a + i b$ is \[\overline{z} = a - i b.\]

  \item The division of complex numbers occurs by multiplying the numerator and denominator by the conjugate of the denominator, e.g. \begin{align*}
          \frac{a + b i}{c + d i} & = \frac{(a + b i) (c - d i)}{(c + d i) (c - d i)}              \\
                                  & = \frac{a c - a d i + b c i + b d}{c^2 + d^2}                  \\
                                  & = \frac{a c + b d}{c^2 + d^2} + i \frac{b c - a d}{c^2 + d^2}.
        \end{align*}

  \item Conjugates have several interesting properties: \begin{align*}
          \overline{z_1 + z_2} & = \overline{z_1} + \overline{z_2}        \\
          \overline{z_1 - z_2} & = \overline{z_1} - \overline{z_2}        \\
          \overline{z_1 z_2}   & = \overline{z_1} \, \overline{z_2}       \\
          \frac{z_1}{z_2}      & = \frac{\overline{z_1}}{\overline{z_2}}.
        \end{align*}

  \item The sum and product of a complex number $z = x + i y$ with its conjugate are real numbers \begin{align*}
          z + \overline{z} & = 2 x       \\
          z \overline{z}   & = x^2 + y^2
        \end{align*} while the difference between a complex number and its conjugate is a purre imaginary number \[z - \overline{z} = 2 i y.\]

  \item The above properties let us define \[\Re(z) = \frac{z + \overline{z}}{2} \text{ and } \Im(z) = \frac{z - \overline{z}}{2 i}.\]

  \item The \textbf{complex plane} or \textbf{$z$-plane} is a coordinate system where the horizontal or $x$-axis is called the \textbf{real axis} and the vertical or $y$-axis is called the \textbf{imaginary axis}. Complex numbers can be plotted in this coordinate system by considering their real and imaginary parts an ordered pair corresponding their position.

  \item The \textbf{modulus} or \textbf{absolute value} of a complex number $z = x + i y$ denoted by $|z|$ is the real number \[|z| = \sqrt{x^2 + y^2} = \sqrt{z \overline{z}}.\] This is the distance between $z$ and the origin in the complex plane.

  \item If you consider two numbers in the complex plane as vectors, the length of their sum can't be longer than their individual lengths combined \[|z_1 + z_2| \le |z_1| + |z_2|.\] This extends to any finite sum \[|z_1 + z_2 + \cdots + z_n| \le |z_1| + |z_2| + \cdots + |z_n|\] and is known as the \textbf{triangle inequality}.
\end{itemize}

\subsection{Powers and Roots}

\begin{itemize}
  \item A complex number can be expressed in \textbf{polar form} \[z = (r \cos \theta) + i (r \sin \theta)\] where $r = |z|$ is the nonnegative modulus of $z$ and $\theta = \arg z$ is the \textbf{argument} of $z$ — the angle between $z$ and the positive real axis measured in the counterclockwise direction.

  \item The argument of a complex number $z$ isn't unique as any multiply of $2 \pi$ can be added to it. The \textbf{principle argument} of $z$ denoted $\Arg z$ is the argument of $z$ restricted to the intercal $-\pi \le \Arg z \le \pi$.

  \item Multiplication and division of complex numbers is simpler in polar form. For two complex numbers $z_1 = r_1 (\cos \theta_1 + i \sin \theta_1)$ and $z_2 = r_2 (\cos \theta_2 + i \sin \theta_2)$ we get \begin{align*}
          z_1 z_2         & = r_1 r_2 [\cos (\theta_1 + \theta_2) + i \sin (\theta_1 + \theta_2)]          \\
          \frac{z_1}{z_2} & = \frac{r_1}{r_2} [\cos (\theta_1 - \theta_2) + i \sin (\theta_1 - \theta_2)].
        \end{align*}

  \item The above formulas can be used to find integer powers of a complex number $z$ \[z^n = r^n (\cos n \theta + i \sin n \theta)\] where $n$ is an integer (including negative integers).

  \item \textbf{DeMoivre's formula} is a special case of the above where $r = 1$ so \[z^n = (\cos \theta + i \sin \theta)^n = \cos n \theta + i \sin n \theta.\]

  \item A number $w$ is said to be an \textbf{$\boldsymbol{n}$th root} of a nonzero complex number $z$ if $w^n = z$. The $n$th roots of a nonzero complex number $z = r (\cos \theta + i \sin \theta)$ are \[w_k = r^{1 / n} \left[ \cos \left( \frac{\theta + 2 k \pi}{n} \right) + i \sin \left( \frac{\theta + 2 k \pi}{n} \right) \right]\] where $k = 0, 1, 2, \ldots, n - 1$.

  \item The root $w$ of a complex number $z$ obtained by using the principle argument of $z$ with $k = 0$ is called the \textbf{principle $\boldsymbol{n}$th root} of $z$.

  \item Since the $n$th roots of a complex number have the same modulus they lie on a circle of radius $r^{1 / n}$. The arguments of subsequent roots differ by $2 \pi / n$ so they're also equally spaced around the circle.
\end{itemize}

\subsection{Sets in the Complex Plane}

\begin{itemize}
  \item The points $z = x + i y$ that satisfy the equation \[|z - z_0| = \rho\] for $\rho > 0$ lie on a circle of radius $\rho$ centred at the point $z_0$.

  \item The points $z$ satisfying the inequality $|z - z_0| < \rho$ for $\rho > 0$ lie within, but not on, a circle of radius $\rho$ centered at the point $z_0$. This set is called a \textbf{neighborhood} of $z_0$ or an \textbf{open disk}.

  \item A point $z_0$ is said to be an \textbf{interior point} of a set $S$ of the complex plane if there exists some neighborhood of $z_0$ that lies entirely within $S$.

  \item If every point $z$ of a set $S$ is an interior point, then $S$ is said to be an \textbf{open set}. An example of a set that isn't open is the set of points satisfying the inequality $\Re (z) \ge 0$. This isn't open because it includes the line $\Re (z) = 0$ and no points on that line are interior to the set because, no matter what $\rho$ you choose, some points in the neighborhood have $\Re (z) < 0$.

  \item If every neighborhood of a point $z_0$ contains at least one point that is in a set $S$ and at least one point that is not in $S$, then $z_0$ is said to be a \textbf{boundary point} of $S$.

  \item The \textbf{boundary} of a set $S$ in the complex plane is the set of all boundary points of $S$.

  \item If any pair of points in a set $S$ can be connected by a polygonal line that lies entirely within the set, then $S$ is said to be \textbf{connected}.

  \item An open connected set is called a \textbf{domain}.

  \item A \textbf{region} is a set in the complex plane with all, some, or none of its boundary points. A region containing all of its boundary points is said to be \textbf{closed}.
\end{itemize}

\end{document}