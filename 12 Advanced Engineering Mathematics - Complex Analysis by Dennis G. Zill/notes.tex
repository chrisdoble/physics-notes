\documentclass{article}
\usepackage{amsmath} % For align*
\usepackage{graphicx} % For images

\graphicspath{{./images/}}

\renewcommand{\Im}{\operatorname{Im}}
\renewcommand{\Re}{\operatorname{Re}}

\title{Advanced Engineering Mathematics Complex Analysis by Dennis G. Zill Notes}
\author{Chris Doble}
\date{February 2024}

\begin{document}

\maketitle

\tableofcontents

\setcounter{section}{16}
\section{Functions of a Complex Variable}

\subsection{Complex Numbers}

\begin{itemize}
  \item A \textbf{complex number} is any number of the form \[z = a + i b\] where $a$ and $b$ are real numbers and $i$ is the imaginary unit such that $i^2 = -1$.

  \item The real number $a$ in the above complex number $z$ is called the \textbf{real part} of $z$ and the real number $b$ (not $i b$) is called the \textbf{imaginary part} of $z$.

  \item The real and imaginary parts of a complex number $z$ are denoted $\Re(z)$ and $\Im(z)$, respectively.

  \item A real constant multiple of the imaginary unit, e.g. $6 i$ is called a \textbf{pure imaginary number}.

  \item Two complex numbers are equal if their real and imaginary parts are equal.

  \item The addition and subtraction of complex numbers occur between the real and imaginary parts, e.g. \[(a + b i) + (c + d i) = (a + c) + (b + d) i.\]

  \item The multiplication of complex numbers occurs elementwise as normal, e.g. \[(a + b i) (c + d i) = a c + a d i + b c i - b d.\]

  \item The \textbf{conjugate} of a complex number $z = a + i b$ is \[\overline{z} = a - i b.\]

  \item The division of complex numbers occurs by multiplying the numerator and denominator by the conjugate of the denominator, e.g. \begin{align*}
          \frac{a + b i}{c + d i} & = \frac{(a + b i) (c - d i)}{(c + d i) (c - d i)}              \\
                                  & = \frac{a c - a d i + b c i + b d}{c^2 + d^2}                  \\
                                  & = \frac{a c + b d}{c^2 + d^2} + i \frac{b c - a d}{c^2 + d^2}.
        \end{align*}

  \item Conjugates have several interesting properties: \begin{align*}
          \overline{z_1 + z_2} & = \overline{z_1} + \overline{z_2}        \\
          \overline{z_1 - z_2} & = \overline{z_1} - \overline{z_2}        \\
          \overline{z_1 z_2}   & = \overline{z_1} \, \overline{z_2}       \\
          \frac{z_1}{z_2}      & = \frac{\overline{z_1}}{\overline{z_2}}.
        \end{align*}

  \item The sum and product of a complex number $z = x + i y$ with its conjugate are real numbers \begin{align*}
          z + \overline{z} & = 2 x       \\
          z \overline{z}   & = x^2 + y^2
        \end{align*} while the difference between a complex number and its conjugate is a purre imaginary number \[z - \overline{z} = 2 i y.\]

  \item The above properties let us define \[\Re(z) = \frac{z + \overline{z}}{2} \text{ and } \Im(z) = \frac{z - \overline{z}}{2 i}.\]

  \item The \textbf{complex plane} or \textbf{$z$-plane} is a coordinate system where the horizontal or $x$-axis is called the \textbf{real axis} and the vertical or $y$-axis is called the \textbf{imaginary axis}. Complex numbers can be plotted in this coordinate system by considering their real and imaginary parts an ordered pair corresponding their position.

  \item The \textbf{modulus} or \textbf{absolute value} of a complex number $z = x + i y$ denoted by $|z|$ is the real number \[|z| = \sqrt{x^2 + y^2} = \sqrt{z \overline{z}}.\] This is the distance between $z$ and the origin in the complex plane.

  \item If you consider two numbers in the complex plane as vectors, the length of their sum can't be longer than their individual lengths combined \[|z_1 + z_2| \le |z_1| + |z_2|.\] This extends to any finite sum \[|z_1 + z_2 + \cdots + z_n| \le |z_1| + |z_2| + \cdots + |z_n|\] and is known as the \textbf{triangle inequality}.
\end{itemize}

\end{document}