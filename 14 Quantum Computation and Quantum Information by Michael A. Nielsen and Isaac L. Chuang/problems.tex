\documentclass{article}
\usepackage{amsmath} % For align*
\usepackage{amsfonts} % For mathbb
\usepackage{bookmark} % For links
\usepackage{braket} % For bra-ket notation
\usepackage{enumitem} % For customisable list labels
\usepackage{siunitx} % For units

\hypersetup{
  colorlinks=true,
  linkcolor=blue,
  urlcolor=blue
}

\setlist[enumerate, 1]{label={(\alph*)}}

\newcommand{\tr}{\operatorname{tr}}
\renewcommand{\vec}[1]{\boldsymbol{\mathbf{#1}}}

\title{Quantum Computation and Quantum Information by Michael A. Nielsen and Isaac L. Chuang Problems}
\author{Chris Doble}
\date{June 2024}

\begin{document}

\maketitle

\tableofcontents

\part{Fundamental concepts}

\setcounter{section}{1}
\section{Linear algebra}

\subsection*{Exercise 2.1}

\[(1, -1) + (1, 2) - (2, 1) = (0, 0)\]

\subsection*{Exercise 2.2}

\[A = \begin{bmatrix}
    0 & 1 \\
    1 & 0
  \end{bmatrix}\]

Using the basis $\ket{+} = (\ket{0} + \ket{1}) / \sqrt{2}$ and $\ket{-} = (\ket{0} - \ket{1}) / \sqrt{2}$ we get

\begin{align*}
  \ket{0}                                          & = \frac{\ket{+} + \ket{-}}{\sqrt{2}} \\
  \ket{1}                                          & = \frac{\ket{+} - \ket{-}}{\sqrt{2}} \\
  \begin{bmatrix}
    a_{0 0} & a_{0 1} \\
    a_{1 0} & a_{1 1}
  \end{bmatrix} \frac{1}{\sqrt{2}} \begin{bmatrix}
                                     1 \\
                                     1
                                   \end{bmatrix} & = \frac{1}{\sqrt{2}} \begin{bmatrix}
                                                                          1 \\
                                                                          -1
                                                                        \end{bmatrix}    \\
  a_{0 0} + a_{0 1}                                & = 1                                  \\
  a_{1 0} + a_{1 1}                                & = -1                                 \\
  \begin{bmatrix}
    a_{0 0} & a_{0 1} \\
    a_{1 0} & a_{1 1}
  \end{bmatrix} \frac{1}{\sqrt{2}} \begin{bmatrix}
                                     1 \\
                                     -1
                                   \end{bmatrix} & = \frac{1}{\sqrt{2}} \begin{bmatrix}
                                                                          1 \\
                                                                          1
                                                                        \end{bmatrix}    \\
  a_{0 0} - a_{0 1}                                & = 1                                  \\
  a_{1 0} - a_{1 1}                                & = 1                                  \\
  A                                                & = \begin{bmatrix}
                                                         1 & 0  \\
                                                         0 & -1
                                                       \end{bmatrix}
\end{align*}

\subsection*{Exercise 2.5}

\begin{align*}
  \begin{bmatrix}
    y_1^* & \ldots & y_n^*
  \end{bmatrix} \begin{bmatrix}
                  z_1    \\
                  \vdots \\
                  z_n
                \end{bmatrix} & = \begin{bmatrix}
                                    y_1^* & \ldots & y_n^*
                                  \end{bmatrix} \left( \begin{bmatrix}
                                                         z_1    \\
                                                         0      \\
                                                         \vdots \\
                                                         0
                                                       \end{bmatrix} + \begin{bmatrix}
                                                                         0      \\
                                                                         z_2    \\
                                                                         \vdots \\
                                                                         0
                                                                       \end{bmatrix} + \cdots + \begin{bmatrix}
                                                                                                  0      \\
                                                                                                  0      \\
                                                                                                  \vdots \\
                                                                                                  z_n
                                                                                                \end{bmatrix} \right) \\
                                & = \begin{bmatrix}
                                      y_1^* & \ldots & y_n^*
                                    \end{bmatrix} \begin{bmatrix}
                                                    z_1    \\
                                                    0      \\
                                                    \vdots \\
                                                    0
                                                  \end{bmatrix} + \cdots + \begin{bmatrix}
                                                                             y_1^* & \ldots & y_n^*
                                                                           \end{bmatrix} \begin{bmatrix}
                                                                                           0      \\
                                                                                           0      \\
                                                                                           \vdots \\
                                                                                           z_n
                                                                                         \end{bmatrix}               \\
                                & = z_1 \begin{bmatrix}
                                          y_1^* & \ldots & y_n^*
                                        \end{bmatrix} \begin{bmatrix}
                                                        1      \\
                                                        0      \\
                                                        \vdots \\
                                                        0
                                                      \end{bmatrix} + \cdots + z_n \begin{bmatrix}
                                                                                     y_1^* & \ldots & y_n^*
                                                                                   \end{bmatrix} \begin{bmatrix}
                                                                                                   0      \\
                                                                                                   0      \\
                                                                                                   \vdots \\
                                                                                                   1
                                                                                                 \end{bmatrix}       \\
  \begin{bmatrix}
    y_1^* & \ldots & y_n^*
  \end{bmatrix} \begin{bmatrix}
                  z_1    \\
                  \vdots \\
                  z_n
                \end{bmatrix} & = y_1^* z_1 + y_2^* z_2 + \ldots + y_n^* z_n                                          \\
                                & = (y_1 z_1^* + y_2 z_2^* + \ldots + y_n z_n^*)^*                                    \\
                                & = \left( \begin{bmatrix}
                                             z_1^* & \cdots & z_n^*
                                           \end{bmatrix} \begin{bmatrix}
                                                           y_1    \\
                                                           \vdots \\
                                                           y_n
                                                         \end{bmatrix} \right)^*                                      \\
  \begin{bmatrix}
    v_1^* & \cdots & v_n^*
  \end{bmatrix} \begin{bmatrix}
                  v_1    \\
                  \vdots \\
                  v_n
                \end{bmatrix} & = |v_1|^2 + \cdots + |v_n|^2                                                          \\
                                & \ge 0
\end{align*}

\subsection*{Exercise 2.6}

\begin{align*}
  \left( \sum_i \lambda_i \ket{w_i}, \ket{v} \right) & = \left( \ket{v}, \sum_i \lambda_i \ket{w_i} \right)^*   \\
                                                     & = \left( \sum_i \lambda_i (\ket{v}, \ket{w_i}) \right)^* \\
                                                     & = \sum_i \lambda_i^* (\ket{v}, \ket{w_i})^*              \\
                                                     & = \sum_i \lambda_i^* (\ket{w_i}, \ket{v})
\end{align*}

\subsection*{Exercise 2.7}

\begin{align*}
  \braket{w | v}              & = \begin{bmatrix}
                                    1 & 1
                                  \end{bmatrix} \begin{bmatrix}
                                                  1 \\
                                                  -1
                                                \end{bmatrix}      \\
                              & = (1) (1) + (1) (-1)                \\
                              & = 0                                 \\
  \frac{\ket{w}}{||\ket{w}||} & = \frac{1}{\sqrt{2}} \begin{bmatrix}
                                                       1 \\
                                                       1
                                                     \end{bmatrix} \\
  \frac{\ket{v}}{||\ket{v}||} & = \frac{1}{\sqrt{2}} \begin{bmatrix}
                                                       1 \\
                                                       -1
                                                     \end{bmatrix}
\end{align*}

\subsection*{Exercise 2.9}

\begin{align*}
  \sigma_0 & = \ket{0} \bra{0} + \ket{1} \bra{1}     \\
  \sigma_1 & = \ket{1} \bra{0} + \ket{0} \bra{1}     \\
  \sigma_2 & = i \ket{1} \bra{0} - i \ket{0} \bra{1} \\
  \sigma_3 & = \ket{0} \bra{0} - \ket{1} \bra{1}
\end{align*}

\subsection*{Exercise 2.11}

\begin{align*}
  \begin{vmatrix}
    -\lambda & 1        \\
    1        & -\lambda
  \end{vmatrix}        & = \lambda^2 - 1                              \\
  \lambda                       & = \pm 1                             \\
  \begin{bmatrix}
    0 & 1 \\
    1 & 0
  \end{bmatrix} \begin{bmatrix}
                  a \\
                  b
                \end{bmatrix} & = \begin{bmatrix}
                                    a \\
                                    b
                                  \end{bmatrix}                      \\
  b                             & = a                                 \\
  a                             & = b                                 \\
  X_1                           & = \frac{1}{\sqrt{2}} \begin{bmatrix}
                                                         1 \\
                                                         1
                                                       \end{bmatrix} \\
  \begin{bmatrix}
    0 & 1 \\
    1 & 0
  \end{bmatrix} \begin{bmatrix}
                  a \\
                  b
                \end{bmatrix} & = \begin{bmatrix}
                                    -a \\
                                    -b
                                  \end{bmatrix}                      \\
  b                             & = -a                                \\
  a                             & = -b                                \\
  X_2                           & = \frac{1}{\sqrt{2}} \begin{bmatrix}
                                                         1 \\
                                                         -1
                                                       \end{bmatrix}
\end{align*}

\subsection*{Exercise 2.12}

\begin{align*}
  \begin{vmatrix}
    1 - \lambda & 0           \\
    1           & 1 - \lambda
  \end{vmatrix} & = (1 - \lambda)^2  \\
  \lambda_1                    & = 1 \\
  \lambda_2                    & = 1
\end{align*}

The eigenvalue $1$ is degenerate. Because the matrix only has one eigenvector it can't diagonalised.

\subsection*{Exercise 2.13}

\[(\ket{w} \bra{v})^\dag = \bra{v}^\dag \ket{w}^\dag = \ket{v} \bra{w}\]

\subsection*{Exercise 2.16}

\begin{align*}
  P^2 & = \left( \sum_{i = 1}^k \ket{i} \bra{i} \right) \left( \sum_{j = 1}^k \ket{j} \bra{j} \right) \\
      & = \sum_{i = j = 1}^k \ket{i} \braket{i | j} \bra{j}                                           \\
      & = \sum_{i = j = 1}^k \ket{i} \delta_{i j} \bra{j}                                             \\
      & = \sum_{i = 1}^k \ket{i} \bra{i}                                                              \\
      & = P
\end{align*}

\subsection*{Exercise 2.17}

\begin{align*}
  A                                & = A^\dag                                               \\
  \sum_i \lambda_i \ket{i} \bra{i} & = \left( \sum_i \lambda_i \ket{i} \bra{i} \right)^\dag \\
                                   & = \sum_i \lambda_i^* \ket{i} \bra{i}
\end{align*}

$\lambda_i = \lambda_i^*$ implies the eigenvalues are real.

\subsection*{Exercise 2.18}

\begin{align*}
  U^\dag U                                                                                             & = I                      \\
  \left( \sum_i \lambda_i \ket{i} \bra{i} \right)^\dag \left( \sum_i \lambda_j \ket{j} \bra{j} \right) & = \sum_k \ket{k} \bra{k} \\
  \sum_{i j} \lambda_i^* \lambda_j \ket{i} \braket{i | j} \bra{j}                                      & = \sum_k \ket{k} \bra{k} \\
  \sum_{i j} \lambda_i^* \lambda_j \ket{i} \delta_{i j} \bra{j}                                        & = \sum_k \ket{k} \bra{k} \\
  \sum_i |\lambda_i|^2 \ket{i} \bra{i}                                                                 & = \sum_k \ket{k} \bra{k} \\
  |\lambda_i|^2                                                                                        & = 1                      \\
  \lambda_i                                                                                            & = e^{i \theta}
\end{align*}

\subsection*{Exercise 2.19}

\begin{align*}
  I^\dag   & = \begin{bmatrix}
                 1 & 0 \\
                 0 & 1
               \end{bmatrix}^\dag \\
           & = \begin{bmatrix}
                 1 & 0 \\
                 0 & 1
               \end{bmatrix}     \\
           & = I                  \\
  I^\dag I & = I I                \\
           & = I                  \\
  X^\dag   & = \begin{bmatrix}
                 0 & 1 \\
                 1 & 0
               \end{bmatrix}^\dag \\
           & = \begin{bmatrix}
                 0 & 1 \\
                 1 & 0
               \end{bmatrix}     \\
           & = X                  \\
  X^\dag X & = X X                \\
           & = I                  \\
  Y^\dag   & = \begin{bmatrix}
                 0 & -i \\
                 i & 0
               \end{bmatrix}^\dag \\
           & = \begin{bmatrix}
                 0 & -i \\
                 i & 0
               \end{bmatrix}     \\
           & = Y                  \\
  Y^\dag Y & = Y Y                \\
           & = I                  \\
  Z^\dag   & = \begin{bmatrix}
                 1 & 0  \\
                 0 & -1
               \end{bmatrix}^\dag \\
           & = \begin{bmatrix}
                 1 & 0  \\
                 0 & -1
               \end{bmatrix}     \\
           & = Z                  \\
  Z^\dag Z & = Z Z                \\
           & = I
\end{align*}

\subsection*{Exercise 2.22}

\begin{align*}
  \braket{v_1 | A | v_2} & = \braket{v_1 | A v_2}                       \\
                         & = \braket{v_1 | \lambda_2 v_2}               \\
                         & = \lambda_2 \braket{v_1 | v_2}               \\
  \braket{v_1 | A | v_2} & = \braket{A^\dag v_1 | v_2}                  \\
                         & = \braket{A v_1 | v_2}                       \\
                         & = \braket{\lambda_1 v_1 | v_2}               \\
                         & = \lambda_1 \braket{v_1 | v_2}               \\
  0                      & = (\lambda_2 - \lambda_1) \braket{v_1 | v_2} \\
                         & = \braket{v_1 | v_2}                         \\
\end{align*}

\subsection*{Exercise 2.23}

For each basis vector $\ket{i}, i = 1, \ldots k$, $P \ket{i} = \ket{i}$ and so they are eigenvectorrs of $P$ with eigenvalue $1$. For each basis vector $\ket{j}, j = k + 1, \ldots, d$, $P \ket{j} = 0$ and so they are eigenvectors of $P$ with eigenvalue of $0$. That is a total of $d$ eigenvectors so all eigenvalues are either $0$ or $1$.

\subsection*{Exercise 2.26}

\begin{align*}
  \ket{\psi}             & = \frac{\ket{0} + \ket{1}}{\sqrt{2}}                                                                                   \\
  \ket{\psi}^{\otimes 2} & = \left( \frac{\ket{0} + \ket{1}}{\sqrt{2}} \right) \left( \frac{\ket{0} + \ket{1}}{\sqrt{2}} \right)                  \\
                         & = \frac{\ket{00} + \ket{01} + \ket{10} + \ket{11}}{2}                                                                  \\
  \ket{\psi}^{\otimes 3} & = \left( \frac{\ket{00} + \ket{01} + \ket{10} + \ket{11}}{2} \right) \left( \frac{\ket{0} + \ket{1}}{\sqrt{2}} \right) \\
                         & = \frac{\ket{000} + \ket{001} + \ket{010} + \ket{011} + \ket{100} + \ket{101} + \ket{110} + \ket{111}}{2^{3 / 2}}      \\
  \ket{\psi}             & = \frac{1}{\sqrt{2}} \begin{bmatrix}
                                                  1 \\
                                                  1
                                                \end{bmatrix}                                                                                    \\
  \ket{\psi}^{\otimes 2} & = \frac{1}{2} \begin{bmatrix}
                                           1 \\
                                           1 \\
                                           1 \\
                                           1
                                         \end{bmatrix}                                                                                           \\
  \ket{\psi}^{\otimes 3} & = \frac{1}{2^{3 / 2}} \begin{bmatrix}
                                                   1 \\
                                                   1 \\
                                                   1 \\
                                                   1 \\
                                                   1 \\
                                                   1 \\
                                                   1 \\
                                                   1
                                                 \end{bmatrix}
\end{align*}

\subsection*{Exercise 2.27}

\begin{enumerate}
  \item

        \begin{align*}
          X           & = \begin{bmatrix}
                            0 & 1 \\
                            1 & 0
                          \end{bmatrix}  \\
          Z           & = \begin{bmatrix}
                            1 & 0  \\
                            0 & -1
                          \end{bmatrix}  \\
          X \otimes Z & = \begin{bmatrix}
                            (0) Z & (1) Z \\
                            (1) Z & (0) Z
                          \end{bmatrix}  \\
                      & = \begin{bmatrix}
                            0 & 0  & 1 & 0  \\
                            0 & 0  & 0 & -1 \\
                            1 & 0  & 0 & 0  \\
                            0 & -1 & 0 & 0
                          \end{bmatrix}
        \end{align*}

  \item

        \[I \otimes X = \begin{bmatrix}
            0 & 1 & 0 & 0 \\
            1 & 0 & 0 & 0 \\
            0 & 0 & 0 & 1 \\
            0 & 0 & 1 & 0
          \end{bmatrix}\]

  \item

        \[X \otimes I = \begin{bmatrix}
            0 & 0 & 1 & 0 \\
            0 & 0 & 0 & 1 \\
            1 & 0 & 0 & 0 \\
            0 & 1 & 0 & 0
          \end{bmatrix}\]

        No, the tensor product is not commutative.
\end{enumerate}

\subsection*{Exercise 2.28}

\begin{align*}
  (A \otimes B)^*       & = \begin{bmatrix}
                              A_{11} B & A_{12} B & \cdots & A_{1n} B \\
                              A_{21} B & A_{22} B & \cdots & A_{2n} B \\
                              \vdots   & \vdots   & \vdots & \vdots   \\
                              A_{m1} B & A_{m2} B & \cdots & A_{mn} B
                            \end{bmatrix}^*             \\
                        & = \begin{bmatrix}
                              A_{11}^* B^* & A_{12}^* B^* & \cdots & A_{1n}^* B^* \\
                              A_{21}^* B^* & A_{22}^* B^* & \cdots & A_{2n}^* B^* \\
                              \vdots       & \vdots       & \vdots & \vdots       \\
                              A_{m1}^* B^* & A_{m2}^* B^* & \cdots & A_{mn}^* B^*
                            \end{bmatrix} \\
                        & = A^* \otimes B^*                                     \\
  (A \otimes B)^T       & = \begin{bmatrix}
                              A_{11} B & A_{12} B & \cdots & A_{1n} B \\
                              A_{21} B & A_{22} B & \cdots & A_{2n} B \\
                              \vdots   & \vdots   & \vdots & \vdots   \\
                              A_{m1} B & A_{m2} B & \cdots & A_{mn} B
                            \end{bmatrix}^T             \\
                        & = \begin{bmatrix}
                              A_{11} B^T & A_{21} B^T & \cdots & A_{m1} B^T \\
                              A_{12} B^T & A_{22} B^T & \cdots & A_{m2} B^T \\
                              \vdots     & \vdots     & \vdots & \vdots     \\
                              A_{1n} B^T & A_{2n} B^T & \cdots & A_{mn} B^T
                            \end{bmatrix}       \\
                        & = A^T \otimes B^T                                     \\
  (A \otimes B)^\dagger & = [(A \otimes B)^*]^T                                 \\
                        & = (A^* \otimes B^*)^T                                 \\
                        & = (A^*)^T \otimes (B^*)^T                             \\
                        & = A^\dagger \otimes B^\dagger
\end{align*}

\subsection*{Exercise 2.29}

\begin{align*}
  (A \otimes B)^\dagger (A \otimes B) (\ket{a} \otimes \ket{b}) & = (A^\dagger \otimes B^\dagger) (A \ket{a} \otimes B \ket{b}) \\
                                                                & = A^\dagger A \ket{a} \otimes B^\dagger B \ket{b}             \\
                                                                & = \ket{a} \otimes \ket{b}
\end{align*}

\subsection*{Exercise 2.30}

\begin{align*}
  (A \otimes B)^\dagger (\ket{a} \otimes \ket{b}) & = (A^\dagger \otimes B^\dagger) (\ket{a} \otimes \ket{b}) \\
                                                  & = (A \otimes B) (\ket{a} \otimes \ket{b})
\end{align*}

\subsection*{Exercise 2.34}

\begin{align*}
  A         & = \begin{bmatrix}
                  4 & 3 \\
                  3 & 4
                \end{bmatrix}                            \\
  \lambda_1 & = 1                                         \\
  \vec{x}_1 & = \begin{bmatrix}
                  -1 \\
                  1
                \end{bmatrix}                            \\
  \lambda_2 & = 7                                         \\
  \vec{x}_2 & = \begin{bmatrix}
                  1 \\
                  1
                \end{bmatrix}                            \\
  P         & = \begin{bmatrix}
                  -1 & 1 \\
                  1  & 1
                \end{bmatrix}                            \\
  A         & = P D P^{-1}                                \\
  D         & = P^{-1} A P                                \\
            & = \begin{bmatrix}
                  1 & 0 \\
                  0 & 7
                \end{bmatrix}                            \\
  \sqrt{A}  & = P \sqrt{D} P^{-1}                         \\
            & = \frac{1}{2} \begin{bmatrix}
                              1 + \sqrt{7}  & -1 + \sqrt{7} \\
                              -1 + \sqrt{7} & 1 + \sqrt{7}
                            \end{bmatrix} \\
  \ln A     & = P \ln (D) P^{-1}                          \\
            & = \frac{\ln 7}{2} \begin{bmatrix}
                                  1 & 1 \\
                                  1 & 1
                                \end{bmatrix}
\end{align*}

\subsection*{Exercise 2.35}

\begin{align*}
  \vec{v}                                    & = (v_x, v_y, v_z)                                                   \\
  \vec{v} \cdot \vec{\sigma}                 & = v_x \sigma_1 + v_y \sigma_2 + v_z \sigma_3                        \\
                                             & = \begin{bmatrix}
                                                   v_z         & v_x - i v_y \\
                                                   v_x + i v_y & -v_z
                                                 \end{bmatrix}                                         \\
  A                                          & = i \theta \vec{v} \cdot \vec{\sigma}                               \\
  \lambda_1                                  & = -i \theta v                                                       \\
  \vec{x}_1                                  & = \begin{bmatrix}
                                                   (v_z - v) / (v_x + i v_y) \\
                                                   1
                                                 \end{bmatrix}                                         \\
  \lambda_2                                  & = i \theta v                                                        \\
  \vec{x}_2                                  & = \begin{bmatrix}
                                                   (v_z + v) / (v_x + i v_y) \\
                                                   1
                                                 \end{bmatrix}                                         \\
  P                                          & = \begin{bmatrix}
                                                   \vec{x}_1 & \vec{x}_2
                                                 \end{bmatrix}                                             \\
  \exp (i \theta \vec{v} \cdot \vec{\sigma}) & = P \begin{bmatrix}
                                                     e^{\lambda_1} & 0             \\
                                                     0             & e^{\lambda_2}
                                                   \end{bmatrix} P^{-1}                                   \\
                                             & = \begin{bmatrix}
                                                   \cos \theta + i v_z \sin \theta & i (v_x - i v_y) \sin \theta     \\
                                                   i (v_x + i v_y) \sin \theta     & \cos \theta - i v_z \sin \theta
                                                 \end{bmatrix} \\
                                             & = \cos \theta I + i \sin \theta \begin{bmatrix}
                                                                                 v_z         & v_x - i v_y \\
                                                                                 v_x + i v_y & -v_z
                                                                               \end{bmatrix}           \\
                                             & = \cos \theta I + i \sin \theta \vec{v} \cdot \vec{\sigma}
\end{align*}

\subsection*{Exercise 2.36}

\begin{align*}
  \tr I & = 1 + 1 \\
        & = 2     \\
  \tr X & = 0 + 0 \\
        & = 0     \\
  \tr Y & = 0 + 0 \\
        & = 0     \\
  \tr Z & = 1 - 1 \\
        & = 0
\end{align*}

\subsection*{Exercise 2.37}

\begin{align*}
  \tr (A B) & = \sum_i \braket{i | A B | i}                  \\
            & = \sum_i \braket{i | A I B | i}                \\
            & = \sum_i \braket{i | A i} \braket{i | B | i}   \\
            & = \sum_i \braket{i | B | i} \braket{i | A | i} \\
            & = \sum_i \braket{i | B I A | i}                \\
            & = \sum_i \braket{i | B A | i}                  \\
            & = \tr (B A)
\end{align*}

\subsection*{Exercise 2.38}

\begin{align*}
  \tr (A + B) & = \sum_i (A_{ii} + B_{ii})      \\
              & = \sum_i A_{ii} + \sum_i B_{ii} \\
              & = \tr A + \tr B                 \\
  \tr (z A)   & = \sum_i z A_{ii}               \\
              & = z \sum_i A_{ii}               \\
              & = z \tr A
\end{align*}

\subsection*{Exercise 2.42}

\begin{align*}
  \frac{[A, B] + \{A, B\}}{2} & = \frac{1}{2} (A B - B A + A B + B A) \\
                              & = A B
\end{align*}

\subsection*{Exercise 2.44}

\begin{align*}
  [A, B]                & = 0 \\
  A B - B A             & = 0 \\
  \{A, B\}              & = 0 \\
  A B + B A             & = 0 \\
  A B - B A + A B + B A & = 0 \\
  2 A B                 & = 0 \\
  A B                   & = 0 \\
  A^{-1} A B            & = 0 \\
  B                     & = 0
\end{align*}

\subsection*{Exercise 2.45}

\begin{align*}
  [A, B]^\dagger & = (A B - B A)^\dagger                       \\
                 & = (A B)^\dagger - (B A)^\dagger             \\
                 & = B^\dagger A^\dagger - A^\dagger B^\dagger \\
                 & = [B^\dagger, A^\dagger]
\end{align*}

\subsection*{Exercise 2.46}

\begin{align*}
  [A, B] & = A B - B A    \\
         & = -(B A - A B) \\
         & = -[B, A]
\end{align*}

\subsection*{Exercise 2.47}

\begin{align*}
  (i [A, B])^\dagger & = -i [B^\dagger, A^\dagger] \\
                     & = -i [B, A]                 \\
                     & = i [A, B]
\end{align*}

\subsection*{Exercise 2.48}

\begin{align*}
  % Positive
  J & = A \\
  K & = A \\
  U & = I \\
  % Unitary
  J & = I \\
  K & = I \\
  U & = A \\
  % Hermitian
  J & = A \\
  K & = A \\
  U & = I
\end{align*}

\subsection*{Exercise 2.51}

\begin{align*}
  H           & = \frac{1}{\sqrt{2}} \begin{bmatrix}
                                       1 & 1  \\
                                       1 & -1
                                     \end{bmatrix}        \\
  H^\dagger   & = \frac{1}{\sqrt{2}} \begin{bmatrix}
                                       1 & 1  \\
                                       1 & -1
                                     \end{bmatrix}        \\
  H^\dagger H & = \frac{1}{2} \begin{bmatrix}
                                1 & 1  \\
                                1 & -1
                              \end{bmatrix} \begin{bmatrix}
                                              1 & 1  \\
                                              1 & -1
                                            \end{bmatrix} \\
              & = \frac{1}{2} \begin{bmatrix}
                                2 & 0 \\
                                0 & 2
                              \end{bmatrix}               \\
              & = I
\end{align*}

\subsection*{Exercise 2.52}

\begin{align*}
  H^2 & = H H         \\
      & = H^\dagger H \\
      & I
\end{align*}

\subsection*{Exercise 2.53}

\begin{align*}
  H \vec{v}                                                                                              & = \lambda \vec{v}      \\
  \frac{1}{\sqrt{2}} \begin{bmatrix}
                       1 & 1  \\
                       1 & -1
                     \end{bmatrix} \vec{v} - \lambda \vec{v}                                                                & = 0 \\
  \begin{bmatrix}
    \frac{1}{\sqrt{2}} - \lambda & \frac{1}{\sqrt{2}}            \\
    \frac{1}{\sqrt{2}}           & -\frac{1}{\sqrt{2}} - \lambda
  \end{bmatrix} \vec{v}              & = 0                                          \\
  \left( \frac{1}{\sqrt{2}} - \lambda \right) \left( -\frac{1}{\sqrt{2}} - \lambda \right) - \frac{1}{2} & = 0                    \\
  -\frac{1}{2} + \lambda^2 - \frac{1}{2}                                                                 & = 0                    \\
  \lambda^2 - 1                                                                                          & = 0                    \\
  (\lambda - 1) (\lambda + 1)                                                                            & = 0
\end{align*}

The eigenvalues are $\pm 1$. The eigenvectors are $\begin{bmatrix}
    1 \\
    -1 \pm \sqrt{2}
  \end{bmatrix}$.

\subsection*{Exercise 2.54}

\begin{align*}
  \exp(A) \exp(B) & = \exp \left( \sum_i a_i \ket{i} \bra{i} \right) \exp \left( \sum_i b_i \ket{i} \bra{i} \right) \\
                  & = \exp \left( \sum_i a_i \ket{i} \bra{i} + \sum_i b_i \ket{i} \bra{i} \right)                   \\
                  & = \exp(A + B)
\end{align*}

\subsection*{Exercise 2.55}

\begin{align*}
  U(t_1, t_2)                     & = \exp \left[ \frac{-i H(t_2 - t_1)}{\hbar} \right] \\
  U(t_1, t_2)^\dagger             & = \exp \left[ \frac{i H(t_2 - t_1)}{\hbar} \right]  \\
  U(t_1, t_2)^\dagger U(t_1, t_2) & = 1
\end{align*}

\subsection*{Exercise 2.56}

\begin{align*}
  K         & = -i \ln (U)                                             \\
  \ln U     & = \ln \left( \sum_u u \ket{u} \bra{u} \right)            \\
            & = \sum_u \ln (u) \ket{u} \bra{u}                         \\
            & = \sum_u i \theta_u \ket{u} \bra{u}                      \\
  K         & = -i \ln (U)                                             \\
            & = \sum_u \theta_u \ket{u} \bra{u}                        \\
  K^\dagger & = \left( \sum_u \theta_u \ket{u} \bra{u} \right)^\dagger \\
            & = \sum_u \theta_u \ket{u} \bra{u}                        \\
            & = K
\end{align*}

\subsection*{Exercise 2.57}

Assuming the state to be measured is $\ket{\psi}$, then the state of the system after the first measurement is \[\ket{\psi'} = \frac{L_l \ket{\psi}}{\sqrt{\braket{\psi | L_l^\dagger L_l | \psi}}}.\] The state of the system after the second measurement is \begin{align*}
  \ket{\psi''} & = \frac{M_m \ket{\psi'}}{\sqrt{\braket{\psi' | M_m^\dagger M_m | \psi'}}}                                                                                                                                                                                          \\
               & = \frac{1}{\sqrt{\braket{\frac{L_l \ket{\psi}}{\sqrt{\braket{\psi | L_l^\dagger L_l | \psi}}} | M_m^\dagger M_m | \frac{L_l \ket{\psi}}{\sqrt{\braket{\psi | L_l^\dagger L_l | \psi}}}}}} \frac{M_m L_l \ket{\psi}}{\sqrt{\braket{\psi | L_l^\dagger L_l | \psi}}} \\
               & = \frac{M_m L_l \ket{\psi}}{\sqrt{\braket{\psi | L_l^\dagger M_m^\dagger M_m L_l | \psi}}}                                                                                                                                                                         \\
               & = \frac{N_{l m} \psi}{\sqrt{\braket{\psi | N_{l m}^\dagger N_{l m} | \psi}}}
\end{align*}

\subsection*{Exercise 2.58}

\begin{align*}
  \braket{M} & = m \\
  \Delta(M)  & = 0
\end{align*}

\subsection*{Exercise 2.59}

\begin{align*}
  \braket{X}   & = \braket{0 | X | 0}                           \\
               & = \begin{bmatrix}
                     1 & 0
                   \end{bmatrix} \begin{bmatrix}
                                   0 & 1 \\
                                   1 & 0
                                 \end{bmatrix} \begin{bmatrix}
                                                 1 \\
                                                 0
                                               \end{bmatrix}   \\
               & = 0                                            \\
  \braket{X^2} & = \braket{0 | X^2 | 0}                         \\
               & = \begin{bmatrix}
                     1 & 0
                   \end{bmatrix} \begin{bmatrix}
                                   0 & 1 \\
                                   1 & 0
                                 \end{bmatrix}^2 \begin{bmatrix}
                                                   1 \\
                                                   0
                                                 \end{bmatrix} \\
               & = \begin{bmatrix}
                     1 & 0
                   \end{bmatrix} \begin{bmatrix}
                                   1 & 0 \\
                                   0 & 1
                                 \end{bmatrix} \begin{bmatrix}
                                                 1 \\
                                                 0
                                               \end{bmatrix}   \\
               & = 1                                            \\
  \Delta(X)    & = \sqrt{\braket{X^2} - \braket{X}^2}           \\
               & = 1
\end{align*}

\subsection*{Exercise 2.60}

\begin{align*}
  \vec{v} \cdot \vec{\sigma} & = v_1 \sigma_x + v_2 \sigma_2 + v_3 \sigma_3                     \\
                             & = v_1 \begin{bmatrix}
                                       0 & 1 \\
                                       1 & 0
                                     \end{bmatrix} + v_2 \begin{bmatrix}
                                                           0 & -i \\
                                                           i & 0
                                                         \end{bmatrix} + v_3 \begin{bmatrix}
                                                                               1 & 0  \\
                                                                               0 & -1
                                                                             \end{bmatrix}     \\
                             & = \begin{bmatrix}
                                   v_3         & v_1 - i v_2 \\
                                   v_1 + i v_2 & -v_3
                                 \end{bmatrix}                                      \\
  0                          & = (v_3 - \lambda) (-v_3 - \lambda) - (v_1 - i v_2) (v_1 + i v_2) \\
                             & = \lambda^2 - v_1^2 - v_2^2 - v_3^2                              \\
                             & = \lambda^2 - 1                                                  \\
                             & = (\lambda - 1) (\lambda + 1)                                    \\
  \lambda                    & = \pm 1                                                          \\
\end{align*}

\subsection*{Exercise 2.62}

If the measurement operators and the POVM elements coincide then \[M_m = E_m = M_m^\dagger M_m\] and \[M_m^\dagger = (M_m^\dagger M_m)^\dagger = M_m^\dagger M_m = M_m,\] i.e. $M_m$ are Hermitian. Then \[E_m = M_m^\dagger M_m = M_m^2 = M_m,\] i.e. $M_m$ are projectors.

\subsection*{Exercise 2.63}

\begin{align*}
  M_m^\dagger M_m & = \sqrt{E_m} U_m^\dagger U_m \sqrt{E_m} \\
                  & = \sqrt{E_m} I \sqrt{E_m}               \\
                  & = E_m
\end{align*}

So $U_m$ can be arbitrary unitary operators.

\subsection*{Exercise 2.65}

\begin{align*}
  \frac{\ket{0} + \ket{1}}{\sqrt{2}} & = \ket{+} \\
  \frac{\ket{0} - \ket{1}}{\sqrt{2}} & = \ket{-}
\end{align*}

\end{document}