\documentclass{article}
\usepackage{amsmath} % For align*
\usepackage{amsfonts} % For mathbb
\usepackage{bookmark} % For links
\usepackage{braket} % For bra-ket notation
\usepackage{siunitx} % For units

\hypersetup{
  colorlinks=true,
  linkcolor=blue,
  urlcolor=blue
}

\title{Quantum Computation and Quantum Information by Michael A. Nielsen and Isaac L. Chuang Problems}
\author{Chris Doble}
\date{June 2024}

\begin{document}

\maketitle

\tableofcontents

\part{Fundamental concepts}

\setcounter{section}{1}
\section{Linear algebra}

\subsection*{Exercise 2.1}

\[(1, -1) + (1, 2) - (2, 1) = (0, 0)\]

\subsection*{Exercise 2.2}

\[A = \begin{bmatrix}
    0 & 1 \\
    1 & 0
  \end{bmatrix}\]

Using the basis $\ket{+} = (\ket{0} + \ket{1}) / \sqrt{2}$ and $\ket{-} = (\ket{0} - \ket{1}) / \sqrt{2}$ we get

\begin{align*}
  \ket{0}                                          & = \frac{\ket{+} + \ket{-}}{\sqrt{2}} \\
  \ket{1}                                          & = \frac{\ket{+} - \ket{-}}{\sqrt{2}} \\
  \begin{bmatrix}
    a_{0 0} & a_{0 1} \\
    a_{1 0} & a_{1 1}
  \end{bmatrix} \frac{1}{\sqrt{2}} \begin{bmatrix}
                                     1 \\
                                     1
                                   \end{bmatrix} & = \frac{1}{\sqrt{2}} \begin{bmatrix}
                                                                          1 \\
                                                                          -1
                                                                        \end{bmatrix}    \\
  a_{0 0} + a_{0 1}                                & = 1                                  \\
  a_{1 0} + a_{1 1}                                & = -1                                 \\
  \begin{bmatrix}
    a_{0 0} & a_{0 1} \\
    a_{1 0} & a_{1 1}
  \end{bmatrix} \frac{1}{\sqrt{2}} \begin{bmatrix}
                                     1 \\
                                     -1
                                   \end{bmatrix} & = \frac{1}{\sqrt{2}} \begin{bmatrix}
                                                                          1 \\
                                                                          1
                                                                        \end{bmatrix}    \\
  a_{0 0} - a_{0 1}                                & = 1                                  \\
  a_{1 0} - a_{1 1}                                & = 1                                  \\
  A                                                & = \begin{bmatrix}
                                                         1 & 0  \\
                                                         0 & -1
                                                       \end{bmatrix}
\end{align*}

\subsection*{Exercise 2.5}

\begin{align*}
  \begin{bmatrix}
    y_1^* & \ldots & y_n^*
  \end{bmatrix} \begin{bmatrix}
                  z_1    \\
                  \vdots \\
                  z_n
                \end{bmatrix} & = \begin{bmatrix}
                                    y_1^* & \ldots & y_n^*
                                  \end{bmatrix} \left( \begin{bmatrix}
                                                         z_1    \\
                                                         0      \\
                                                         \vdots \\
                                                         0
                                                       \end{bmatrix} + \begin{bmatrix}
                                                                         0      \\
                                                                         z_2    \\
                                                                         \vdots \\
                                                                         0
                                                                       \end{bmatrix} + \cdots + \begin{bmatrix}
                                                                                                  0      \\
                                                                                                  0      \\
                                                                                                  \vdots \\
                                                                                                  z_n
                                                                                                \end{bmatrix} \right) \\
                                & = \begin{bmatrix}
                                      y_1^* & \ldots & y_n^*
                                    \end{bmatrix} \begin{bmatrix}
                                                    z_1    \\
                                                    0      \\
                                                    \vdots \\
                                                    0
                                                  \end{bmatrix} + \cdots + \begin{bmatrix}
                                                                             y_1^* & \ldots & y_n^*
                                                                           \end{bmatrix} \begin{bmatrix}
                                                                                           0      \\
                                                                                           0      \\
                                                                                           \vdots \\
                                                                                           z_n
                                                                                         \end{bmatrix}               \\
                                & = z_1 \begin{bmatrix}
                                          y_1^* & \ldots & y_n^*
                                        \end{bmatrix} \begin{bmatrix}
                                                        1      \\
                                                        0      \\
                                                        \vdots \\
                                                        0
                                                      \end{bmatrix} + \cdots + z_n \begin{bmatrix}
                                                                                     y_1^* & \ldots & y_n^*
                                                                                   \end{bmatrix} \begin{bmatrix}
                                                                                                   0      \\
                                                                                                   0      \\
                                                                                                   \vdots \\
                                                                                                   1
                                                                                                 \end{bmatrix}       \\
  \begin{bmatrix}
    y_1^* & \ldots & y_n^*
  \end{bmatrix} \begin{bmatrix}
                  z_1    \\
                  \vdots \\
                  z_n
                \end{bmatrix} & = y_1^* z_1 + y_2^* z_2 + \ldots + y_n^* z_n                                          \\
                                & = (y_1 z_1^* + y_2 z_2^* + \ldots + y_n z_n^*)^*                                    \\
                                & = \left( \begin{bmatrix}
                                             z_1^* & \cdots & z_n^*
                                           \end{bmatrix} \begin{bmatrix}
                                                           y_1    \\
                                                           \vdots \\
                                                           y_n
                                                         \end{bmatrix} \right)^*                                      \\
  \begin{bmatrix}
    v_1^* & \cdots & v_n^*
  \end{bmatrix} \begin{bmatrix}
                  v_1    \\
                  \vdots \\
                  v_n
                \end{bmatrix} & = |v_1|^2 + \cdots + |v_n|^2                                                          \\
                                & \ge 0
\end{align*}

\subsection*{Exercise 2.6}

\begin{align*}
  \left( \sum_i \lambda_i \ket{w_i}, \ket{v} \right) & = \left( \ket{v}, \sum_i \lambda_i \ket{w_i} \right)^*   \\
                                                     & = \left( \sum_i \lambda_i (\ket{v}, \ket{w_i}) \right)^* \\
                                                     & = \sum_i \lambda_i^* (\ket{v}, \ket{w_i})^*              \\
                                                     & = \sum_i \lambda_i^* (\ket{w_i}, \ket{v})
\end{align*}

\subsection*{Exercise 2.7}

\begin{align*}
  \braket{w | v}              & = \begin{bmatrix}
                                    1 & 1
                                  \end{bmatrix} \begin{bmatrix}
                                                  1 \\
                                                  -1
                                                \end{bmatrix}      \\
                              & = (1) (1) + (1) (-1)                \\
                              & = 0                                 \\
  \frac{\ket{w}}{||\ket{w}||} & = \frac{1}{\sqrt{2}} \begin{bmatrix}
                                                       1 \\
                                                       1
                                                     \end{bmatrix} \\
  \frac{\ket{v}}{||\ket{v}||} & = \frac{1}{\sqrt{2}} \begin{bmatrix}
                                                       1 \\
                                                       -1
                                                     \end{bmatrix}
\end{align*}

\subsection*{Exercise 2.9}

\begin{align*}
  \sigma_0 & = \ket{0} \bra{0} + \ket{1} \bra{1}     \\
  \sigma_1 & = \ket{1} \bra{0} + \ket{0} \bra{1}     \\
  \sigma_2 & = i \ket{1} \bra{0} - i \ket{0} \bra{1} \\
  \sigma_3 & = \ket{0} \bra{0} - \ket{1} \bra{1}
\end{align*}

\subsection*{Exercise 2.11}

\begin{align*}
  \begin{vmatrix}
    -\lambda & 1        \\
    1        & -\lambda
  \end{vmatrix}        & = \lambda^2 - 1                              \\
  \lambda                       & = \pm 1                             \\
  \begin{bmatrix}
    0 & 1 \\
    1 & 0
  \end{bmatrix} \begin{bmatrix}
                  a \\
                  b
                \end{bmatrix} & = \begin{bmatrix}
                                    a \\
                                    b
                                  \end{bmatrix}                      \\
  b                             & = a                                 \\
  a                             & = b                                 \\
  X_1                           & = \frac{1}{\sqrt{2}} \begin{bmatrix}
                                                         1 \\
                                                         1
                                                       \end{bmatrix} \\
  \begin{bmatrix}
    0 & 1 \\
    1 & 0
  \end{bmatrix} \begin{bmatrix}
                  a \\
                  b
                \end{bmatrix} & = \begin{bmatrix}
                                    -a \\
                                    -b
                                  \end{bmatrix}                      \\
  b                             & = -a                                \\
  a                             & = -b                                \\
  X_2                           & = \frac{1}{\sqrt{2}} \begin{bmatrix}
                                                         1 \\
                                                         -1
                                                       \end{bmatrix}
\end{align*}

\subsection*{Exercise 2.12}

\begin{align*}
  \begin{vmatrix}
    1 - \lambda & 0           \\
    1           & 1 - \lambda
  \end{vmatrix} & = (1 - \lambda)^2  \\
  \lambda_1                    & = 1 \\
  \lambda_2                    & = 1
\end{align*}

The eigenvalue $1$ is degenerate. Because the matrix only has one eigenvector it can't diagonalised.

\subsection*{Exercise 2.13}

\[(\ket{w} \bra{v})^\dag = \bra{v}^\dag \ket{w}^\dag = \ket{v} \bra{w}\]

\subsection*{Exercise 2.16}

\begin{align*}
  P^2 & = \left( \sum_{i = 1}^k \ket{i} \bra{i} \right) \left( \sum_{j = 1}^k \ket{j} \bra{j} \right) \\
      & = \sum_{i = j = 1}^k \ket{i} \braket{i | j} \bra{j}                                           \\
      & = \sum_{i = j = 1}^k \ket{i} \delta_{i j} \bra{j}                                             \\
      & = \sum_{i = 1}^k \ket{i} \bra{i}                                                              \\
      & = P
\end{align*}

\subsection*{Exercise 2.17}

\begin{align*}
  A                                & = A^\dag                                               \\
  \sum_i \lambda_i \ket{i} \bra{i} & = \left( \sum_i \lambda_i \ket{i} \bra{i} \right)^\dag \\
                                   & = \sum_i \lambda_i^* \ket{i} \bra{i}
\end{align*}

$\lambda_i = \lambda_i^*$ implies the eigenvalues are real.

\subsection*{Exercise 2.18}

\begin{align*}
  U^\dag U                                                                                             & = I                      \\
  \left( \sum_i \lambda_i \ket{i} \bra{i} \right)^\dag \left( \sum_i \lambda_j \ket{j} \bra{j} \right) & = \sum_k \ket{k} \bra{k} \\
  \sum_{i j} \lambda_i^* \lambda_j \ket{i} \braket{i | j} \bra{j}                                      & = \sum_k \ket{k} \bra{k} \\
  \sum_{i j} \lambda_i^* \lambda_j \ket{i} \delta_{i j} \bra{j}                                        & = \sum_k \ket{k} \bra{k} \\
  \sum_i |\lambda_i|^2 \ket{i} \bra{i}                                                                 & = \sum_k \ket{k} \bra{k} \\
  |\lambda_i|^2                                                                                        & = 1                      \\
  \lambda_i                                                                                            & = e^{i \theta}
\end{align*}

\subsection*{Exercise 2.19}

\begin{align*}
  I^\dag   & = \begin{bmatrix}
                 1 & 0 \\
                 0 & 1
               \end{bmatrix}^\dag \\
           & = \begin{bmatrix}
                 1 & 0 \\
                 0 & 1
               \end{bmatrix}     \\
           & = I                  \\
  I^\dag I & = I I                \\
           & = I                  \\
  X^\dag   & = \begin{bmatrix}
                 0 & 1 \\
                 1 & 0
               \end{bmatrix}^\dag \\
           & = \begin{bmatrix}
                 0 & 1 \\
                 1 & 0
               \end{bmatrix}     \\
           & = X                  \\
  X^\dag X & = X X                \\
           & = I                  \\
  Y^\dag   & = \begin{bmatrix}
                 0 & -i \\
                 i & 0
               \end{bmatrix}^\dag \\
           & = \begin{bmatrix}
                 0 & -i \\
                 i & 0
               \end{bmatrix}     \\
           & = Y                  \\
  Y^\dag Y & = Y Y                \\
           & = I                  \\
  Z^\dag   & = \begin{bmatrix}
                 1 & 0  \\
                 0 & -1
               \end{bmatrix}^\dag \\
           & = \begin{bmatrix}
                 1 & 0  \\
                 0 & -1
               \end{bmatrix}     \\
           & = Z                  \\
  Z^\dag Z & = Z Z                \\
           & = I
\end{align*}

\subsection*{Exercise 2.22}

\begin{align*}
  \braket{v_1 | A | v_2} & = \braket{v_1 | A v_2}                       \\
                         & = \braket{v_1 | \lambda_2 v_2}               \\
                         & = \lambda_2 \braket{v_1 | v_2}               \\
  \braket{v_1 | A | v_2} & = \braket{A^\dag v_1 | v_2}                  \\
                         & = \braket{A v_1 | v_2}                       \\
                         & = \braket{\lambda_1 v_1 | v_2}               \\
                         & = \lambda_1 \braket{v_1 | v_2}               \\
  0                      & = (\lambda_2 - \lambda_1) \braket{v_1 | v_2} \\
                         & = \braket{v_1 | v_2}                         \\
\end{align*}

\subsection*{Exercise 2.23}

For each basis vector $\ket{i}, i = 1, \ldots k$, $P \ket{i} = \ket{i}$ and so they are eigenvectorrs of $P$ with eigenvalue $1$. For each basis vector $\ket{j}, j = k + 1, \ldots, d$, $P \ket{j} = 0$ and so they are eigenvectors of $P$ with eigenvalue of $0$. That is a total of $d$ eigenvectors so all eigenvalues are either $0$ or $1$.

\end{document}