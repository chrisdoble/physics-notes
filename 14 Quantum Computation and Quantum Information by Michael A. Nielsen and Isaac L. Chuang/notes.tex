\documentclass{article}
\usepackage{amsmath} % For align*
\usepackage{bookmark} % For links
\usepackage{braket} % For bra-ket notation

\hypersetup{
  colorlinks=true,
  linkcolor=blue,
  urlcolor=blue
}

\title{Quantum Computation and Quantum Information by Michael A. Nielsen and Isaac L. Chuang}
\author{Chris Doble}
\date{May 2024}

\begin{document}

\maketitle

\tableofcontents

\part{Fundamental concepts}

\section{Introduction and overview}

\setcounter{subsection}{1}
\subsection{Quantum bits}

\begin{itemize}
  \item The special states $\ket{0}$ and $\ket{1}$ form an orthonormal basis and are known as \textbf{computational basis states}.

  \item A quantum bit (\textbf{qubit}) is a linear combination of the computational basis states \[\ket{\psi} = \alpha \ket{0} + \beta \ket{1}\] where $\alpha$ and $\beta$ are complex numbers.

  \item When we measure a qubit we either get $\ket{0}$ with probability $|\alpha|^2$ or $\ket{1}$ with probability $|\beta|^2$. Thus, $|\alpha|^2 + |\beta|^2 = 1$ and a qubit can be thought of as a unit vector in a two-dimensional complex vector space.

  \item If a qubit is in the state \[\ket{+} = \frac{1}{\sqrt{2}} (\ket{0} + \ket{1})\] there's a 50/50 chance of measuring $\ket{0}$ or $\ket{1}$.

  \item If we let \[\alpha = e^{i \gamma} \cos \frac{\theta}{2}\] and \[\beta = e^{i \gamma} e^{i \varphi} \sin \frac{\theta}{2}\] then \begin{align*}
          |\alpha|^2 + |\beta|^2 & = \alpha^* \alpha + \beta^* \beta                   \\
                                 & = \cos^2 \frac{\theta}{2} + \sin^2 \frac{\theta}{2} \\
                                 & = 1
        \end{align*} so the qubit is still normalised and it can be written \[\ket{\psi} = e^{i \gamma} \left( \cos \frac{\theta}{2} \ket{0} + e^{i \varphi} \sin \frac{\theta}{2} \ket{1} \right).\] It turns out that $e^{i \gamma}$ has no observable effects and we can effectively write \[\ket{\psi} = \cos \frac{\theta}{2} \ket{0} + e^{i \varphi} \sin \frac{\theta}{2} \ket{1}.\] This defines a point on a three-dimensional sphere known as the \textbf{Bloch sphere} where $\theta$ and $\varphi$ take on their usual roles in a spherical coordinate system.

  \item Before measurement a qubit is in a linear combination of $\ket{0}$ and $\ket{1}$ but when measured you get one or the other and the state of the system changes to match the measured result.
\end{itemize}

\end{document}