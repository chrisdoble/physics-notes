\documentclass{article}
\usepackage{amsmath} % For align*
\usepackage{bookmark} % For links
\usepackage{braket} % For bra-ket notation

\hypersetup{
  colorlinks=true,
  linkcolor=blue,
  urlcolor=blue
}

\title{Quantum Computation and Quantum Information by Michael A. Nielsen and Isaac L. Chuang}
\author{Chris Doble}
\date{May 2024}

\begin{document}

\maketitle

\tableofcontents

\part{Fundamental concepts}

\section{Introduction and overview}

\setcounter{subsection}{1}
\subsection{Quantum bits}

\begin{itemize}
  \item The special states $\ket{0}$ and $\ket{1}$ form an orthonormal basis and are known as \textbf{computational basis states}.

  \item A quantum bit (\textbf{qubit}) is a linear combination of the computational basis states \[\ket{\psi} = \alpha \ket{0} + \beta \ket{1}\] where $\alpha$ and $\beta$ are complex numbers.

  \item When we measure a qubit we either get $\ket{0}$ with probability $|\alpha|^2$ or $\ket{1}$ with probability $|\beta|^2$. Thus, $|\alpha|^2 + |\beta|^2 = 1$ and a qubit can be thought of as a unit vector in a two-dimensional complex vector space.

  \item If a qubit is in the state \[\ket{+} = \frac{1}{\sqrt{2}} (\ket{0} + \ket{1})\] there's a 50/50 chance of measuring $\ket{0}$ or $\ket{1}$.

  \item If we let \[\alpha = e^{i \gamma} \cos \frac{\theta}{2}\] and \[\beta = e^{i \gamma} e^{i \varphi} \sin \frac{\theta}{2}\] then \begin{align*}
          |\alpha|^2 + |\beta|^2 & = \alpha^* \alpha + \beta^* \beta                   \\
                                 & = \cos^2 \frac{\theta}{2} + \sin^2 \frac{\theta}{2} \\
                                 & = 1
        \end{align*} so the qubit is still normalised and it can be written \[\ket{\psi} = e^{i \gamma} \left( \cos \frac{\theta}{2} \ket{0} + e^{i \varphi} \sin \frac{\theta}{2} \ket{1} \right).\] It turns out that $e^{i \gamma}$ has no observable effects and we can effectively write \[\ket{\psi} = \cos \frac{\theta}{2} \ket{0} + e^{i \varphi} \sin \frac{\theta}{2} \ket{1}.\] This defines a point on a three-dimensional sphere known as the \textbf{Bloch sphere} where $\theta$ and $\varphi$ take on their usual roles in a spherical coordinate system.

  \item Before measurement a qubit is in a linear combination of $\ket{0}$ and $\ket{1}$ but when measured you get one or the other and the state of the system changes to match the measured result.
\end{itemize}

\subsubsection{Multiple Bits}

\begin{itemize}
  \item A two qubit system has four computational basis state $\ket{00}$, $\ket{01}$, $\ket{10}$, and $\ket{11}$ so the general expression for the state of such a system is \[\ket{\psi} = \alpha_{00} \ket{00} + \alpha_{01} \ket{01} + \alpha_{10} \ket{10} + \alpha_{11} \ket{11}.\]

  \item If you were to measure the first qubit, you would get $\ket{0}$ with probability $|\alpha_{00}|^2 + |\alpha_{01}|^2$ and the system would be left in the state \[\ket{\psi} = \frac{\alpha_{00} \ket{00} + \alpha_{01} \ket{01}}{\sqrt{|\alpha_{00}|^2 + |\alpha_{01}|^2}},\] i.e. it is renormalised such that the normalisation condition still holds.
\end{itemize}

\subsection{Quantum Computation}

\subsubsection{Single Qubit Gates}

\begin{itemize}
  \item The quantum \textsc{not} changes $\ket{0}$ to $\ket{1}$ and $\ket{1}$ to $\ket{0}$. It acts linearly on superpositions of those states, i.e. it turns $\alpha \ket{0} + \beta \ket{1}$ into $\beta \ket{0} + \alpha \ket{1}$. If a quantum state $\ket{\psi} = \alpha \ket{0} + \beta \ket{1}$ is written in vector notation as \[\ket{\psi} = \begin{bmatrix}
            \alpha \\
            \beta
          \end{bmatrix}\] then the quantum \textsc{not} gate can be expressed in matrix form as \[X = \begin{bmatrix}
            0 & 1 \\
            1 & 0
          \end{bmatrix}.\]

  \item In order to preserve the normalisation condition, matrix representations of quantum gates must be unitary, i.e. $M^\dagger M = I$ where $I$ is the identity matrix.

  \item An arbitrary unitary 2x2 matrix can be decomposed into a finite set of other 2x2 matrices. This means an arbitrary single qubit gate can be generated by a finite set of other gates.
\end{itemize}

\subsubsection{Multiply Qubit Gates}

\begin{itemize}
  \item The controlled-\textsc{not} or \textsc{cnot} gate is a multi-qubit gate that has two input qubits known as the control qubit and the target qubit. If the control qubit is set to $\ket{0}$ the target qubit is left alone, but if it's set to $\ket{1}$ the target qubit is flipped. Another way of writing this is $\ket{A, B} \rightarrow \ket{A, A \oplus B}$ where $\oplus$ is modulo-two addition. Yet another way of writing this is in matrix form \[U_{CN} = \begin{bmatrix}
            1 & 0 & 0 & 0 \\
            0 & 1 & 0 & 0 \\
            0 & 0 & 0 & 1 \\
            0 & 0 & 1 & 0
          \end{bmatrix}\] where the first column describes what happens to the $\ket{00}$ basis state, etc.

  \item Other classical gates like \textsc{nand} or \textsc{xor} can't be represented as quantum gates as they're irreversibile. For example, given the output $A \oplus B$ from an \textsc{xor} gate it's not possible to determine what the inputs $A$ and $B$ were.

  \item Any multiple qubit logic gate may be composed from \textsc{cnot} and single qubit gates.
\end{itemize}

\end{document}