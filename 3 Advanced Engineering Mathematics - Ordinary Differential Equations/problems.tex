\documentclass{article}
\usepackage{amsmath} % For align*
\usepackage{amsfonts} % For open face letters
\usepackage{enumitem} % For customisable list labels
\usepackage{graphicx} % For images
\usepackage{siunitx} % For units
\graphicspath{{./images/}}

\setlist[enumerate, 1]{label={(\alph*)}}

\title{Advanced Engineering Mathematics Ordinary Differential Equations Notes}
\author{Chris Doble}
\date{February 2022}

\begin{document}

\maketitle

\tableofcontents

\section{Introduction to Differential Equations}

\subsection{Definitions and Terminology}

\subsubsection{1}

2, linear

\subsubsection{3}

4, linear

\subsubsection{5}

2, nonlinear

\subsubsection{7}

3, linear

\subsubsection{9}

no; yes

\subsubsection{15}

The domain of the function is $x \in [-2, \infty)$.

\[y' = 1 + \frac{2}{\sqrt{x + 2}}\]

The largest interval of definition of the solution is $x \in (-2, \infty)$.

\begin{align*}
  (y - x) y'                                            & = y - x + 8                  \\
  (x + 4 \sqrt{x + 2} - x) (1 + \frac{2}{\sqrt{x + 2}}) & = x + 4 \sqrt{x + 2} - x + 8 \\
  4 \sqrt{x + 2} + 8                                    & = 4 \sqrt{x + 2} + 8
\end{align*}

\subsubsection{17}

The domain of the function is $x \in \mathbb{R}, x \ne \pm 2$.

\[y' = \frac{2 x}{(4 - x^2)^2}\]

The largest intervals of definition of the solution are $(-\infty, -2)$, $(-2, 2)$, and $(2, \infty)$.

\begin{align*}
  y'                      & = 2 x y^2                                \\
  \frac{2 x}{(4 - x^2)^2} & = 2 x \left( \frac{1}{4 - x^2} \right)^2 \\
                          & = \frac{2x}{(4 - x^2)^2}
\end{align*}

\subsubsection{19}

\begin{align*}
  ln \frac{2 X - 1}{X - 1} & = t                       \\
  2 X - 1                  & = (X - 1) e^t             \\
  (2 - e^t) X              & = 1 - e^t                 \\
  X                        & = \frac{e^t - 1}{e^t - 2}
\end{align*}

The solutions intervals of validity are $(\infty, \ln 2)$ and $(\ln 2, \infty)$.

\begin{align*}
  \frac{dX}{dt}                                           & = (X - 1) (1 - 2 X)                                                                                   \\
  \frac{e^t}{e^t - 2} - \frac{e^t (e^t - 1)}{(e^t - 2)^2} & = \left( \frac{e^t - 1}{e^t - 2} - 1 \right) \left( 1 - 2 \frac{e^t - 1}{e^t - 2} \right)             \\
  \frac{e^t (e^t - 2) - e^t (e^t - 1)}{(e^t - 2)^2}       & = \left( \frac{e^t - 1 - e^t + 2}{e^t - 2} \right) \left( \frac{e^t - 2 - 2 e^t + 2}{e^t - 2} \right) \\
  \frac{e^{2t} - 2 e^t - e^{2t} + e^t}{(e^t - 2)^2}       & = \left( \frac{1}{e^t - 2} \right) \left( \frac{-e^t}{e^t - 2} \right)                                \\
  \frac{-e^t}{(e^t - 2)^2}                                & = \frac{-e^t}{(e^t - 2)^2}
\end{align*}

\subsubsection{31}

$m = -2$

\subsubsection{33}

$m = 2 \text{ or } 3$

\subsubsection{35}

$m = -1 \text{ or } 0$

\subsubsection{37}

$y = 2$

\subsubsection{39}

No constant solutions

\subsection{Initial Value Problems}

\subsubsection{1}

\begin{align*}
  y(0) = -\frac{1}{3} & = \frac{1}{1 + c_1 e^{-(0)}} \\
  -3                  & = 1 + c_1                    \\
  c_1                 & = -4
\end{align*}

\[y = \frac{1}{1 - 4 e^{-x}}\]

\subsubsection{3}

\begin{align*}
  y(2) = \frac{1}{3} & = \frac{1}{(2)^2 + c} \\
  3                  & = 4 + c               \\
  c                  & = -1
\end{align*}

\[y = \frac{1}{x^2 - 1}\]

$I = (1, \infty)$

\subsubsection{5}

\begin{align*}
  y(0) = 1 & = \frac{1}{(0)^2 + c} \\
  c        & = 1
\end{align*}

\[y = \frac{1}{x^2 + 1}\]

$I = (-\infty, \infty)$

\subsubsection{7}

\begin{align*}
  x(0) = -1 & = c_1 \cos 0 + c_2 \sin 0 \\
  c_1       & = -1
\end{align*}

\begin{align*}
  x'(0) = 8 & = -c_1 \sin 0 + c_2 \cos 0 \\
  c_2       & = 8
\end{align*}

\[x = -\cos t + 8 \sin t\]

\subsubsection{9}

\begin{align*}
  x' \left( \frac{\pi}{6} \right) = 0 & = -c_1 \sin \frac{\pi}{6} + c_2 \cos \frac{\pi}{6} \\
                                      & = -c_1 \frac{1}{2} + c_2 \frac{\sqrt{3}}{2}        \\
  c_1                                 & = \sqrt{3} c_2
\end{align*}

\begin{align*}
  x \left( \frac{\pi}{6} \right) = \frac{1}{2} & = c_1 \cos \frac{\pi}{6} + c_2 \sin \frac{\pi}{6} \\
                                               & = \frac{3}{2} c_2 + \frac{1}{2} c_2               \\
                                               & = 2 c_2                                           \\
  c_2                                          & = \frac{1}{4}
\end{align*}

\[y = \frac{\sqrt{3}}{4} \cos t + \frac{1}{4} \sin t\]

\subsubsection{11}

\begin{align*}
  y(0) = 1 & = c_1 e^{(0)} + c_2 e^{-(0)} \\
           & = c_1 + c_2                  \\
  c_1      & = 1 - c_2
\end{align*}

\begin{align*}
  y'(0) = 2 & = c_1 e^{(0)} - c_2 e^{-(0)} \\
            & = 1 - c_2 - c_2              \\
  c_2       & = -\frac{1}{2}
\end{align*}

\[y = \frac{3}{2} e^x - \frac{1}{2} e^{-x}\]

\subsubsection{13}

\begin{align*}
  y(-1) = 5 & = c_1 e^{(-1)} + c_2 e^{-(-1)} \\
            & = c_1 e^{-1} + c_2 e           \\
  c_1       & = 5 e - c_2 e^2
\end{align*}

\begin{align*}
  y'(-1) = -5   & = c_1 e^{(-1)} - c_2 e^{-(-1)} \\
                & = 5 e - c_2 e^2 - c_2 e        \\
  c_2 e (e + 1) & = 5 (e + 1)                    \\
  c_2           & = \frac{5}{e}
\end{align*}

\[y = 5 e^{-x - 1}\]

\subsubsection{15}

\[y = 0\]

\[y = x^3\]

\subsubsection{17}

\[f(x, y) = y^{2 / 3}\]

\[\frac{\partial f}{\partial y} = \frac{2}{3 y^{1 / 3}}\]

$y < 0$ or $y > 0$

\subsubsection{19}

\[f(x, y) = \frac{y}{x}\]

\[\frac{\partial f}{\partial y} = \frac{1}{x}\]

$x < 0$ or $x > 0$

\subsubsection{21}

\[f(x, y) = \frac{x^2}{4 - y^2}\]

\[\frac{\partial f}{\partial y} = \frac{2 x^2 y}{(4 - y^2)^2}\]

$y < -2$, $-2 < y < 2$, or $y > 2$

\subsubsection{23}

\[f(x, y) = \frac{y^2}{x^2 + y^2}\]

\[\frac{\partial f}{\partial y} = \frac{2 y}{x^2 + y^2} - \frac{2 y^3}{(x^2 + y^2)^2}\]

$x \ne 0$ and $y \ne 0$

\subsubsection{25}

\[f(x, y) = \sqrt{y^2 - 9}\]

\[\frac{\partial f}{\partial y} = \frac{y}{\sqrt{y^2 - 9}}\]

Yes

\subsubsection{27}

No

\subsubsection{29}

\begin{enumerate}
  \item $y = c x$

  \item

        \[f(x, y) = \frac{y}{x}\]

        \[\frac{\partial f}{\partial y} = \frac{1}{x}\]

        $x \ne 0$

  \item No, the function is not differentiable at $x = 0$
\end{enumerate}

\subsubsection{31}

\begin{enumerate}
  \item \[y' = \frac{1}{(x + c)^2} = y^2\]

  \item

        \[y(0) = 1 = -\frac{1}{(0) + c} \Rightarrow c = -1 \Rightarrow y = \frac{1}{1 - x}\]

        $I = (-\infty, 1)$

        \[y(0) = -1 = -\frac{1}{(0) + c} \Rightarrow c = 1 \Rightarrow y = -\frac{1}{x + 1}\]

        $I = (-1, \infty)$
\end{enumerate}

\subsubsection{39}

\begin{align*}
  y(0) = 0 & = c_1 \cos 3 (0) + c_2 \sin 3 (0) \\
  c_1      & = 0
\end{align*}

\begin{align*}
  y \left( \frac{\pi}{6} \right) = -1 & = c_2 \sin 3 \left( \frac{\pi}{6} \right) \\
  c_2                                 & = -1
\end{align*}

\[y = -\sin 3 x\]

\subsubsection{41}

\begin{align*}
  y'(0) = 0 & = -3 c_1 \sin 3 (0) + 3 c_2 \cos 3 (0) \\
  c_2       & = 0
\end{align*}

\begin{align*}
  y' \left( \frac{\pi}{4} \right) = 0 & = -3 c_1 \sin 3 \left( \frac{\pi}{4} \right) \\
                                      & = -\frac{3}{\sqrt{2}} c_1                    \\
  c_1                                 & = 0
\end{align*}

\[y = 0\]

\subsubsection{43}

\begin{align*}
  y(0) = 0 & = c_1 \cos 3 (0) + c_2 \sin 3 (0) \\
  c_1      & = 0
\end{align*}

\begin{align*}
  y(\pi) = 4 & = c_2 \sin 3 (\pi) \\
  4          & = 0
\end{align*}

No solution

\subsection{Differential Equations as Mathematical Models}

\subsubsection{1}

\[\frac{dP}{dt} = k P + r\]

\[\frac{dP}{dt} = k P - r\]

\subsubsection{3}

\[\frac{dP}{dt} = k_b P - k_d P^2\]

\subsubsection{7}

\[\frac{dx}{dt} = k x (1000 - x)\]

\subsubsection{9}

\[\frac{dA}{dt} = -\frac{A}{100}\]

\[A(0) = \qty{50}{lb}\]

\subsubsection{11}

\[\frac{dA}{dt} + \frac{7}{600 - t} A = 6\]

\subsubsection{13}

\begin{align*}
  \frac{dV}{dt}     & = -c A_h \sqrt{2 g h}                          \\
  A_w \frac{dh}{dt} & = -c A_h \sqrt{2 g h}                          \\
  \frac{dh}{dt}     & = -\frac{c A_h \sqrt{2 g}}{A_w} \sqrt{h}       \\
                    & = -\frac{c \pi r_h^2 \sqrt{2 g}}{A_w} \sqrt{h} \\
                    & = -\frac{c \pi}{430} \sqrt{h}
\end{align*}

\subsubsection{15}

\[L \frac{di}{dt} + R i = E\]

\subsubsection{17}

\[m \frac{dv}{dt} = m g - k v^2\]

\subsubsection{19}

\[m \frac{d^2 x}{d t^2} = -k x\]

\subsubsection{21}

\begin{align*}
  \frac{d}{d t} (m v)                   & = R - k v       \\
  \frac{d m}{d t} v + m \frac{d v}{d t} & = R - k v - m g
\end{align*}

\subsubsection{23}

\[g = \frac{k}{R^2} \Rightarrow k = g R^2\]

\[\frac{d^2 r}{d t^2} = -\frac{g R^2}{r^2}\]

\subsubsection{25}

\[\frac{d A}{d t} = k (M - A)\]

\subsubsection{27}

\[\frac{d x}{d t} = r - k x\]

\subsubsection{29}

\begin{align*}
  \frac{d y}{d x} & = \tan \theta                     \\
                  & = \tan \frac{\phi}{2}             \\
                  & = \frac{1 - \cos \phi}{\sin \phi} \\
                  & = \frac{1 - x / r}{y / r}         \\
                  & = \frac{r - x}{y}                 \\
                  & = \frac{\sqrt{x^2 + y^2} - x}{y}
\end{align*}

\end{document}