\documentclass{article}
\usepackage{amsmath} % For align*
\usepackage{enumitem} % For customisable list labels
\usepackage{siunitx} % For units

% To allow 5 levels of nested lists
\setlistdepth{5}
\setlist[itemize,1]{label=$\bullet$}
\setlist[itemize,2]{label=$\bullet$}
\setlist[itemize,3]{label=$\bullet$}
\setlist[itemize,4]{label=$\bullet$}
\setlist[itemize,5]{label=$\bullet$}
\renewlist{itemize}{itemize}{5}

\title{Advanced Engineering Mathematics Ordinary Differential Equations Formula Sheet}
\author{Chris Doble}
\date{February 2022}

\begin{document}

\tableofcontents

\section{Flow Chart}

\begin{itemize}
  \item Ordinary

        \begin{itemize}
          \item First order

                \begin{itemize}
                  \item Linear

                        \begin{itemize}
                          \item Homogeneous

                                \begin{itemize}
                                  \item Separation of variables
                                \end{itemize}

                          \item Nonhomogeneous

                                \begin{itemize}
                                  \item Bernoulli

                                  \item Exact

                                  \item Exact with integration constant

                                  \item Homogeneous substitution

                                  \item Reduction to separation of variables

                                  \item Riccati

                                  \item Variation of parameters
                                \end{itemize}
                        \end{itemize}

                  \item Nonlinear

                        \begin{itemize}
                          \item Separable

                                \begin{itemize}
                                  \item Separation of variables
                                \end{itemize}
                        \end{itemize}
                \end{itemize}

          \item Second order

                \begin{itemize}
                  \item Linear

                        \begin{itemize}
                          \item Homogeneous

                                \begin{itemize}
                                  \item Auxiliary/characteristic equation

                                  \item Cauchy/Euler

                                  \item Reduction of order
                                \end{itemize}

                          \item Nonhomogeneous

                                \begin{itemize}
                                  \item Cauchy/Euler

                                  \item Undetermined coefficients

                                  \item Variation of parameters
                                \end{itemize}
                        \end{itemize}

                  \item Nonlinear

                        \begin{itemize}
                          \item Reduction of order

                          \item Taylor series
                        \end{itemize}
                \end{itemize}

          \item Higher order

                \begin{itemize}
                  \item Linear

                        \begin{itemize}
                          \item Homogeneous

                                \begin{itemize}
                                  \item Auxiliary/characteristic equation

                                  \item Cauchy/Euler
                                \end{itemize}

                          \item Nonhomogeneous

                                \begin{itemize}
                                  \item Cauchy/Euler

                                  \item Undetermined coefficients

                                  \item Variation of parameters
                                \end{itemize}
                        \end{itemize}

                  \item Nonlinear

                        \begin{itemize}
                          \item Taylor series
                        \end{itemize}
                \end{itemize}
        \end{itemize}

  \item Partial
\end{itemize}

\section{First-order ODEs}

\textbf{Form:} IVP \[\frac{d y}{d x} = f(x, y)\] \[y(x_0) = y_0\] \\ \textbf{Test:} $f(x, y)$ and $\partial f / \partial y$ are continuous over $I$ \\ \textbf{Property:} A unique solution is guaranteed over $I$

\subsection{Separable Equations}

\textbf{Form:} \[\frac{d y}{d x} = g(x) h(y)\] \\ \textbf{Solution:} Divide by $h(y)$ then integrate with respect to $x$.

\begin{align*}
  \frac{d y}{d x}                          & = g(x) h(y)      \\
  \frac{1}{h(y)} \frac{d y}{d x}           & = g(x)           \\
  \int \frac{1}{h(y)} \frac{d y}{d x} \,dx & = \int g(x) \,dx \\
  \int \frac{1}{h(y)} \,dy                 & = \int g(x) \,dx \\
  H(y)                                     & = G(x) + c
\end{align*}

\subsection{Linear Equations}

\textbf{Form:} \[\frac{d y}{d x} + P(x) y = f(x)\] \\ \textbf{Solution:}

\begin{enumerate}
  \item Determine the integrating factor $e^{\int P(x) \,d x}$

  \item Multiply by the integrating factor

  \item Recognise that the left hand side of the equation is the derivative of the product of the integrating factor and $y$

  \item Integrate both sides of the equation

  \item Solve for $y$
\end{enumerate}

\subsection{Exact Equations}

\textbf{Form:} \[z = f(x, y) = c\] \[dz = \frac{\partial f}{\partial x} \,d x + \frac{\partial f}{\partial y} \,d y = M(x, y) \,d x + N(x, y) \,d y = 0\] \\ \textbf{Test:} \[\frac{\partial M}{\partial y} = \frac{\partial N}{\partial x}\] \\ \textbf{Solution:} \begin{enumerate}
  \item Integrate $M(x, y)$ with respect to $x$ to find an expression for $z = f(x, y)$

        \begin{align*}
          \frac{\partial f}{\partial x} & = M(x, y)                   \\
          f(x, y)                       & = \int M(x, y) \,d x + g(y)
        \end{align*}

  \item Differentiate $f(x, y)$ with respect to $y$ and equate it to $N(x, y)$ to find $g'(y)$

        \begin{align*}
          \frac{\partial f}{\partial y} = N(x, y) & = \frac{\partial}{\partial y} \int M(x, y) \,d x + g'(y)   \\
          g'(y)                                   & = N(x, y) - \frac{\partial}{\partial y} \int M(x, y) \,d x
        \end{align*}

  \item Integrate $g'(y)$ with respect to $y$ to find $g(y)$ and substitute it into $f(x, y)$

  \item Equate $f(x, y)$ with an unknown constant $c$
\end{enumerate} \textbf{Note:} The steps can be performed with $x$ and $y$ reversed, i.e. start by integrating $N(x, y)$ with respect to $y$, etc.

\subsection{Exact Equations with Integration Constant}

\textbf{Form:} \[M(x, y) \,d x + N(x, y) \,d y = 0\] \\ \textbf{Test:} $(M_y - N_x) / N$ is a function of $x$ alone or $(N_x - M_y) / M$ is a function of $y$ alone \\ \textbf{Solution:}

\begin{enumerate}
  \item Compute the integrating factor \[\mu(x) = e^{\int \frac{M_y - N_x}{N} \,d x}\] or \[\mu(y) = e^{\int \frac{N_x - M_y}{M} \,d y}\] as appropriate

  \item Multiple the equation by this factor

  \item The equation is now exact and can be solved as above
\end{enumerate}

\subsection{Homogeneous Equations}

\textbf{Form:} \[M(x, y) \,d x + N(x, y) \,d y = 0\] \\ \textbf{Test:} $M$ and $N$ are homogeneous functions of the same degree \\ \textbf{Solution:}

\begin{enumerate}
  \item Rewrite as \[M(x, y) = x^\alpha M(1, u) \text{ and } N(x, y) = x^\alpha N(1, u) \text{ where } u = y / x\] or \[M(x, y) = y^\alpha M(v, 1) \text{ and } N(x, y) = y^\alpha N(v, 1) \text{ where } v = x / y\]

  \item Substitute $y = u x$ and $d y = u \,d x + x \,d u$ or $x = v y$ and $d x = v \,d y + y \,d v$ as appropriate

  \item Solve the resulting first-order separable DE

  \item Substitude $u = y / x$ or $v = x / y$ as appropriate
\end{enumerate}

\subsection{Bernoulli's Equation}

\textbf{Form:} \[\frac{d y}{d x} + P(x) y = f(x) y^n\] \\ \textbf{Test:} $n \ne 0$ and $n \ne 1$ \\ \textbf{Solution:}

\begin{enumerate}
  \item Substitude $y = u^{1 / (1 - n)}$ and $\frac{d y}{d x} = \frac{d}{dx} (u^{1 / (1 - n)})$

  \item Solve the resulting linear equation

  \item Substitude $u = y^{1 - n}$
\end{enumerate}

\subsection{Reduction to Separation of Variables}

\textbf{Form:} \[\frac{d y}{d x} = f(A x + B y + C), B \ne 0\] \\ \textbf{Solution:} \begin{enumerate}
  \item Substitute \[A x + B y + C = u\]

  \item Solve the resulting separable equation

  \item Substitute \[u = A x + B y + C\]
\end{enumerate}

\subsection{Riccati's Equation}

\textbf{Form:} \[\frac{dy}{dx} = P(x) + Q(x) y + R(x) y^2\] \\ \textbf{Test:} You know a particular solution $y_1$ of the equation \\ \textbf{Solution:} \begin{enumerate}
  \item Substitute $y = y_1 + u$ and $y' = y_1' + u'$

  \item Solve the resulting Bernoulli equation

  \item Substitude $u = y - y_1$
\end{enumerate}

\section{Higher-order ODEs}

\subsection{Initial Value Problems}

\textbf{Form:} $n$-th order IVP \[a_n(x) \frac{d^ny}{dx^n} + a_{n - 1}(x) \frac{d^{n - 1}y}{dx_{n - 1}} + \cdots + a_1(x) \frac{dy}{dx} + a_0(x) y = g(x)\] subject to \[y(x_0) = y_0, \,y'(x_0) = y_1, \,\ldots, \,y^{(n - 1)}(x_0) = y_{n - 1}\] \\ \textbf{Test:} $a_n(x)$, $a_{n - 1}(x)$, $\ldots$, $a_0(x)$, and $g(x)$ are continuous on an interval $I$ and $a_n(x) \ne 0$ for every $x$ in $I$ \\ \textbf{Property:} A unique solution exists for every $x = x_0$ in $I$

\subsection{Linear Independence}

\textbf{Form:} A set of functions $f_1$, $f_2$, $\ldots$, $f_n$ \\ \textbf{Test:} The Wronskian $W(f_1, f_2, \ldots, f_n) \ne 0$ for every $x$ in an interval $I$ where \[W(f_1, f_2, \ldots, f_n) = \begin{vmatrix}
    f_1           & f_2           & \cdots & f_n           \\
    f_1'          & f_2'          & \cdots & f_n'          \\
    \vdots        & \vdots        &        & \vdots        \\
    f_1^{(n - 1)} & f_2^{(n - 1)} & \cdots & f_n^{(n - 1)}
  \end{vmatrix}\] \\ \textbf{Property:} The functions are linearly independent in $I$

\subsection{Linear Equations}

\subsubsection{Homogeneous Linear $n$th-Order Equations}

The general solution is of the form \[y = c_1 y_1 + c_2 y_2 + \cdots + c_n y_n\] where $c_i$ are arbitrary constants and $y_i$ are a fundamental set of solutions (i.e. a set of $n$ linearly independent solutions).

\subsubsection{Nonhomogeneous Linear $n$th-Order Equations}

The general solution is of the form \[y = y_c + y_p = c_1 y_1(x) + c_2 y_2(x) + \cdots + c_n y_n(x) + y_p(x)\] where $y_c$ is the complementary function (i.e. the general solution of the associated homogeneous equation) and $y_p$ is a particular solution.

\subsubsection{Reduction of Order}

\textbf{Form:} \[y'' + P(x) y' + Q(x) y = 0\] \\ \textbf{Test:} A non-trivial solution $y_1(x)$ is known \\ \textbf{Solution:} \begin{enumerate}
  \item Recognise that the ratio of two linearly independent functions isn't constant, i.e. \[u(x) = \frac{y_1(x)}{y_2(x)} \text{ or } y_2(x) = u(x) y_1(x)\]

  \item Substitute $y_2(x) = u(x) y_1(x)$ into the DE — this will result in a DE involving only $u''$ and $u'$ which can be treated as a linear first-order DE in $u' = w$

  \item Solve for $w$

  \item Substitute $w = u'$

  \item Integrate to find $u$

  \item Multiply by $y_1$ to find $y_2$
\end{enumerate} or equivalently \[y_2 = y_1(x) \int \frac{e^{-\int P(x) \,dx}}{y_1^2(x)} \,dx\]

\subsubsection{Homogeneous Linear Equations with Constant Coefficients}

\textbf{Form:} \[a_n y^{(n)} + a_{n - 1} y^{(n - 1)} + \cdots + a_1 y' + a_0 y = 0\] \\ \textbf{Solution:} \begin{enumerate}
  \item Assume the equation has a solution of the form $y = e^{mx}$, giving \[a_n m^n e^{mx} + a_{n - 1} m^{n - 1} e^{mx} + \cdots + a_1 m e^{mx} + a_0 e^{mx} = 0\]

  \item Divide by $e^{mx}$, giving the auxiliary/characteristic equation \[a_n m^n + a_{n - 1} m^{n - 1} + \cdots + a_1 m + a_0 = 0\]

  \item Solve for $m$, where

        \begin{itemize}
          \item A real root $m$ corresponds to a solution \[y = ce^{mx}\]

\item Complex roots $\alpha \pm i \beta$ correspond to solutions \[y = e^{\alpha x} (c_1 \cos \beta x + c_2 \sin \beta x)\]

          \item A root $m$ of multiplicity $k$ corresponds to the solutions \[e^{mx}, \: x e^{mx}, \: x^2 e^{mx}, \: \ldots, \: x^{k - 1} x^{mx}\]
        \end{itemize}
\end{enumerate}

\subsubsection{Method of Undetermined Coefficients}

\textbf{Form:} A nonhomogeneous linear DE where the input function ($g(x)$) is comprised of constants, polynomials, exponentials $e^{\alpha x}$, sines, and cosines \\ \textbf{Solution}: \begin{enumerate}
  \item Solve the associated homogeneous equation

  \item Assume the particular solution has the same form as the input function

  \item If a term in the proposed solution is present in the complementary function, multiply it by $x^n$ where $n$ is the smallest positive integer that removes the duplication

  \item Substitute the proposed solution into the DE

  \item Solve for the unknown constants
\end{enumerate}

\subsubsection{Variation of Parameters}

\textbf{Form:} A nonhomogeneous linear DE \\ \textbf{Solution:} \begin{enumerate}
  \item Solve the homogeneous equation to find the complementary function

  \item Assume the solution has the form \[y_p = u_1(x) y_1(x) + \cdots + u_n(x) y_n(x)\] where $n$ is the order of the equation and $y_i$ are the fundamental set of solutions from the complementary equation

  \item Convert to standard form by dividing by the leading coefficient \[y^{(n)} + a_{n - 1}(x) y^{(n - 1)} + \cdots + a_1(x) y' + a_0(x) y = f(x)\]

  \item Solve the system of linear equations

        \begin{align*}
          y_1 u_1' + \cdots + y_n u_n'                     & = 0    \\
          y_1' u_1' + \cdots + y_n' u_n'                   & = 0    \\
          \vdots \qquad \qquad                                      \\
          y_1^{(n - 1)} u_1' + \cdots + y_n^{(n - 1)} y_n' & = 0    \\
          y_1^{(n)} u_1' + \cdots + y_n^{(n)} u_n'         & = f(x)
        \end{align*}

        via Cramer's method:

        \begin{enumerate}
          \item Compute the Wronskian of $y_i$ \[W = \begin{vmatrix}
                    y_1       & \cdots & y_n       \\
                    y_1'      & \cdots & y_n'      \\
                    \vdots    & \ddots & \vdots    \\
                    y_1^{(n)} & \cdots & y_n^{(n)}
                  \end{vmatrix}\]

          \item Compute $u_i'$ for $i = 1, \: \ldots, \: n$ where \[u_i' = \frac{W_i}{W}\] and $W_i$ is the determinant of the matrix formed by replacing the $i$th column of the Wronskian matrix with the column vector \[\begin{bmatrix}
                    0      \\
                    \vdots \\
                    0      \\
                    f(x)
                  \end{bmatrix}\]
        \end{enumerate}

  \item Integrate each $u_i'$ to find $u_i$
\end{enumerate}

\subsubsection{Cauchy-Euler Equations}

\textbf{Form:} \[a_n x^n \frac{d^n y}{dx^n} + a_{n - 1} x^{n - 1} \frac{d^{n - 1} y}{dx^{n - 1}} + \cdots + a_1 x \frac{dy}{dx} + a_0 y = g(x)\] \\ \textbf{Solution:} \begin{itemize}
  \item If the equation is homogeneous:

        \begin{enumerate}
          \item Assume the equation has a solution of the form $y = x^m$, giving

                \begin{align*}
                  a_n x^n \frac{d^n y}{dx^n} & = a_n x^n m (m - 1) (m - 2) \cdots (m - n + 1) x^{m - n} \\
                                             & = a_n m (m - 1) (m - 2) \cdots (m - n + 1) x^m
                \end{align*}

                and the equation then becomes \[f(m) x^m = 0\] where $f(m)$ is a polynomial in $m$ known as the auxiliary or characteristic equation, the roots of which form the general solution

          \item Solve the auxiliary equation where

                \begin{itemize}
                  \item A real root $m$ corresponds to a solution \[y = c x^m\]

                  \item Complex roots $\alpha \pm i \beta$ correspond to solutions \[x^\alpha (c_1 \cos (\beta \ln x) + c_2 \sin (\beta \ln x))\]

                  \item A root $m$ of multiplicity $k$ corresponds to solutions \[x^m, \: x^m \ln x, \: x^m (\ln x)^2\, \:\ldots, \: x^m (\ln x)^{k - 1}\]
                \end{itemize}
        \end{enumerate}

  \item If the equation is nonhomogeneous:

        \begin{enumerate}
          \item Solve the associated homogeneous equation

          \item Find a particular solution via variation of parameters
        \end{enumerate}
\end{itemize}

\subsection{Nonlinear Equations}

\subsubsection{Reducation of Order}

\textbf{Form:} Nonlinear second-order DE \[F(x, y', y'') = 0\] i.e. $y$ is missing \\ \textbf{Solution:} \begin{enumerate}
  \item Substitute $u = y'$ (and thus $u' = y''$)

  \item Solve the resulting DE for $u$

  \item Integrate to find $y$
\end{enumerate} \textbf{Form:} Nonlinear second-order DE \[F(y, y', y'') = 0\] i.e. $x$ is missing \\ \textbf{Solution:} \begin{enumerate}
  \item Substitute $u = y'$ and \[y'' = \frac{du}{dy} \frac{dy}{dx} = u \frac{du}{dy}\]

  \item Solve the resulting DE for $u$

  \item Integrate to find $y$
\end{enumerate}

\subsubsection{Taylor Series}

\textbf{Form:} Nonlinear initial value problem \\ \textbf{Solution:} \begin{enumerate}
  \item Substitute the initial conditions into a Taylor series centred at $x_0$

  \item Take further derivatives of the equation and substitute the initial conditions in to find additional terms for the Taylor series
\end{enumerate}

\end{document}