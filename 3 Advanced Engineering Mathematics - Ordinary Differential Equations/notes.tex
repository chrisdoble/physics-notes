\documentclass{article}
\usepackage{amsmath} % For align*
\usepackage{enumitem} % For customisable list labels
\usepackage{graphicx} % For images
\usepackage{siunitx} % For units
\graphicspath{{./images/}}

\title{Advanced Engineering Mathematics Ordinary Differential Equations Notes}
\author{Chris Doble}
\date{February 2022}

\begin{document}

\maketitle

\tableofcontents

\section{Introduction to Differential Equations}

\subsection{Definitions and Terminology}

\begin{itemize}
\item An equation containing the derivatives of one or more dependent variables, with respect to one or more independent variables, is said to be a \textbf{differential equation} (DE)

\item An \textbf{ordinary DE} (ODE) is a DE that contains only ordinary (i.e. non-partial) derivatives of one or more functions with respect to a single independent variable

\item A \textbf{partial DE} is a DE that contains only partial derivatives of one or more functions of two or more independent variables

\item The \textbf{order} of a DE is the order of the highest derivative in the equation

  \item First order ODEs are sometimes written in the \textbf{differential form} \[M(x, y) \,dx + N(x, y) \,dy = 0\]

  \item $n$-th order ODEs in one dependent variable can be expressed by the \textbf{general form} \[F(x, y, y', \ldots, y^{(n)}) = 0\]

  \item It's possible to solve ODEs in the general form uniquely for the highest derivative $y^{(n)}$ in terms of the other $n + 1$ variables, allowing them to be expressed in the \textbf{normal form} \[\frac{d^n y}{d x^n} = f(x, y, y', \ldots, y^{(n - 1)})\]

  \item An $n$-th order ODE is said be \textbf{linear} in the variable $y$ if it can be expressed in the form \[a_n(x) y^{(n)} + a_{n-1}(x) y^{(n - 1)} + \cdots + a_1(x) y' + a_0(x) y - g(x) = 0\] i.e. the dependent variable $y$ and all of its derivatives aren't raised to a power or used in nonlinear functions like $e^y$ or $\sin y$, and the coefficients $a_0$, $a_1$, $\ldots$, $a_n$ depend at most on the independent variable $x$

  \item A \textbf{nonlinear} ODE is one that is not linear

  \item A \textbf{solution} to an ODE is a function $\phi$, defined on an interval $I$ and possessing at least $n$ derivatives that are continuous on $I$, such that \[F(x, \phi(x), \phi'(x), \ldots, \phi^{n}(x)) = 0 \text{ for all } x \text{ in } I.\]

  \item The \textbf{interval of definition}, \textbf{interval of validity}, or the \textbf{domain} of a solution is the interval over which the solution is valid

\item A solution of a DE that is $0$ on an interval $I$ is said to be a \textbf{trivial solution}

\item Because solutions to DEs must be differentiable over their interval of validity, discontinuities, etc. must be excluded from the interval

  \item An \textbf{explicit solution} to an ODE is one where the dependent variable is expressed solely in terms of the independent variable and constants

  \item An \textbf{implicit solution} to an ODE is a relation $G(x, y) = 0$ over an interval $I$ provided there exists at least one function $\phi$ that satisfies the relation as well as the ODE on $I$

  \item When solving a first-order ODE we usually obtain a solution containing a single arbitrary constant or parameter $c$. A solution containing an arbitrary constant represents a set of solution called a \textbf{one-parameter family of solutions}

\item When solving an $n$-th order DE we usually obtain an \textbf{$n$-parameter family of solutions}

\item A solution of a DE that is free from arbitrary parameters is called a \textbf{particular solution}

\item A \textbf{singular solution} is a solution to a DE that isn't a member of a family of solutions

  \item A \textbf{system of ODEs} is two or more equations involving the derivatives of two or more unknown functions of a single independent variable. A solution of such a system is a differentiable function for each equation defined on a common interval $I$ that satisfy each equation of the system on that interval
\end{itemize}

\subsection{Initial Value Problems}

\begin{itemize}
\item An \textbf{initial value problem} is the problem of solving a DE with some given \textbf{initial conditions}, e.g. solve \[\frac{d^n y}{dx^n} = f(x, y, y', \ldots, y^{(n - 1)})\] subject to \[y(x_0) = y_0, \,y'(x_0) = y_1, \,\ldots, \,y^{(n - 1)}(x_0) = y_{n - 1}\]

  \item The domain of $y = f(x)$ differs depending on how it's considered:

        \begin{itemize}
          \item As a function its domain is all real numbers for which it's defined

\item As a solution of a DE its domain is a single interval over which it's defined an differentiable

          \item As a solution of an initial value problem its domain is a single interval over which it's defined, differentiable, and contains the initial conditions
        \end{itemize}

  \item An initial value problem may not have any solutions. If it does it may have multiple.

  \item First-order initial value problems of the form \[\frac{dy}{dx} = f(x, y)\] \[y(x_0) = y_0\] are guaranteed to have a unique solution over an interval $I$ containing $x_0$ if $f(x, y)$ and $\partial f / \partial y$ are continuous
\end{itemize}

\subsection{Differential Equations as Mathematical Models}

\begin{itemize}
  \item A \textbf{mathematical model} is a mathematical description of a system or phenomenon

  \item The \textbf{level of resolution} of a model determines how many variables are included in the model

  \item A simple model of the growth of a population $P$ is \[\frac{dP}{dt} = k P\] where $k > 0$

  \item A simple model of radioactive decay of an amount of substance $A$ is \[\frac{dA}{dt} = k A\] where $k < 0$

  \item Newton's empirical law of cooling/warming states that the rate of change of the temperature of a body is proportional to the difference between the temperature of the body and the temperature of the surrounding medium \[\frac{dT}{dt} = k (T - T_m)\]
\end{itemize}

\section{First-Order Differential Equations}

\subsection{Solution Curves Without a Solution}

\begin{itemize}
  \item An ODE in which the independent variable doesn't appear is said to be \textbf{autonomous}, e.g. \[\frac{d y}{d x} = f(y)\]

  \item A real number $c$ is a \textbf{critical/equilibrium/stationary point} of an autonomous DE if it is a zero of $f$

  \item If $c$ is a critial point of an autonomous DE, then $y(x) = c$ is a solution

  \item A solution of the form $y(x) = c$ is called an \textbf{equilibrium solution}

  \item We can draw several conclusions about the solutions of an autonomous DE with $n$ critical points and $n + 1$ subregions bounded by the critical points:

        \begin{itemize}
          \item If $(x_0, y_0)$ is in a subregion, it remains in that subregion for all $x$

          \item By continuity, $f(y) < 0$ or $f(y) > 0$ for all $y$ in a subregion and thus $y(x)$ can't have maximum/minimum points or oscillate

          \item If $y(x)$ is bounded above by a critical point $c_1$, it must approach $y(x) = c_1$ as $x \rightarrow -\infty$ or $x \rightarrow \infty$

          \item If $y(x)$ is bounded above and below by critical points $c_1$ and $c_2$, it must approach $y(x) = c_1$ as $x \rightarrow -\infty$ and $y(x) = c_2$ as $x \rightarrow \infty$ or vice versa

          \item If $y(x)$ is bounded below by a critical point $c_1$, it must approach $y(x) = c_1$ as $x \rightarrow -\infty$ or $x \rightarrow \infty$
        \end{itemize}
\end{itemize}

\includegraphics*{attractors-and-repellers.png}

\begin{itemize}
  \item If $y(x)$ is a solution of an autonomous differential equation $d y / d x = f(y)$, then $y_1(x) = y (x - k)$, where $k$ is a constant, is also a solution
\end{itemize}

\end{document}