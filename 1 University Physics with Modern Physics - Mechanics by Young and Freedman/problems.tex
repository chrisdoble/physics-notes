\documentclass{article}
\usepackage{amsmath}
\usepackage{enumerate}

\title{University Physics with Modern Physics Mechanics Problems}
\author{Chris Doble}
\date{December 2022}

\begin{document}

\maketitle

\tableofcontents

\setcounter{section}{13}
\section{Periodic Motion}

\subsection{VP14.3.1}

\begin{enumerate}[a)]
  \item
        \[T = \frac{1}{f} = \frac{1}{4.15} = 0.241\,\textrm{s}\]
        \[\omega = 2\pi f = 2\pi(4.15) = 26.1\,\textrm{rad/s}\]

  \item
        \begin{align*}
          \omega            & = \sqrt{\frac{k}{m}} \\
          \omega^2          & = \frac{k}{m}        \\
          m\omega^2         & = k                  \\
          (0.400)(26.1)^2   & = k                  \\
          272\,\textrm{N/m} & = k
        \end{align*}

  \item
        \begin{align*}
          F & = kx               \\
            & = (272)(0.0200)    \\
            & = 5.44\,\textrm{N}
        \end{align*}
\end{enumerate}

\subsection{VP14.3.2}

\begin{enumerate}[a)]
  \item
        \begin{align*}
          T                             & = 2\pi\sqrt{\frac{m}{k}} \\
          \frac{T}{2\pi}                & = \sqrt{\frac{m}{k}}     \\
          \frac{T^2}{4\pi^2}            & = \frac{m}{k}            \\
          \frac{kT^2}{4\pi^2}           & = m                      \\
          \frac{(4.50)(1.20)^2}{4\pi^2} & = m                      \\
          0.164\,\textrm{kg}            & = m
        \end{align*}

  \item
        \begin{align*}
          F = ma_\textrm{max}          & = kA \\
          \frac{ma_\textrm{max}}{k}    & = A  \\
          \frac{(0.164)(1.20)}{(4.50)} & = A  \\
          0.0437\,\textrm{m}           & = A
        \end{align*}
\end{enumerate}

\subsection{VP14.3.3}

\begin{enumerate}[a)]
  \item
        \begin{align*}
          \frac{1}{2}mv^2 + \frac{1}{2}kx^2                  & = \frac{1}{2}kA^2 \\
          \sqrt{\frac{m}{k}v^2 + x^2}                        & = A               \\
          \sqrt{\frac{1}{(2\pi f)^2}v^2 + x^2}               & = A               \\
          \sqrt{\frac{1}{(2\pi(50.0))^2}(12.4)^2+(0.0300)^2} & = A               \\
          0.0496\,\textrm{m}                                 & = A
        \end{align*}

  \item
        \begin{align*}
          \frac{1}{2}mv^2 & = \frac{1}{2}kA^2     \\
          v               & = \sqrt{\frac{k}{m}}A \\
                          & = 2\pi fA             \\
                          & = 2\pi(50.0)(0.0496)  \\
                          & = 15.6\,\textrm{m/s}
        \end{align*}
\end{enumerate}

\subsection{VP14.3.4}

\begin{enumerate}[a)]
  \item
        \begin{align*}
          f & = \frac{1}{2\pi}\sqrt{\frac{k}{m}}      \\
            & = \frac{1}{2\pi}\sqrt{\frac{185}{5.00}} \\
            & = 0.968\,\textrm{Hz}
        \end{align*}

  \item
        \begin{align*}
          F = ma_\textrm{max}              & = kA \\
          \frac{ma_\textrm{max}}{k}        & = A  \\
          \frac{(5.00)(1.52)}{185}         & = A  \\
          4.11 \times 10^{-2} \,\textrm{m} & = A
        \end{align*}
\end{enumerate}

\subsection{VP14.4.1}

\begin{enumerate}[a)]
  \item
        \begin{align*}
          \frac{1}{2}mv^2                  & = \frac{1}{2}kA^2 \\
          \sqrt{\frac{m}{k}}v              & = A               \\
          \sqrt{\frac{0.150}{8.00}}0.350   & = A               \\
          4.79 \times 10^{-2} \,\textrm{m} & = A
        \end{align*}

  \item \[\frac{1}{2}mv^2 = \frac{1}{2}(0.150)(0.350)^2 = 9.19 \times 10^{-3} \,\textrm{J}\]

  \item
        \[U = \frac{1}{2}kx^2 = \frac{1}{2}(8.00)(0.0300)^2 = 3.60 \times 10^{-3} \,\textrm{J}\]
        \[K = E - U = 9.19 \times 10^{-3} - 3.60 \times 10^{-3} = 5.59 \times 10^{-3} \,\textrm{J}\]
\end{enumerate}

\subsection{VP14.4.2}

\begin{enumerate}[a)]
  \item
        \begin{align*}
          E                                         & = \frac{1}{2}kA^2 \\
          \frac{2E}{A^2}                            & = k               \\
          \frac{2(6.00 \times 10^{-2})}{(0.0440)^2} & = k               \\
          62.0 \,\textrm{N/m}                       & = k
        \end{align*}

  \item
        \begin{align*}
          E                                        & = \frac{1}{2}mv^2 + \frac{E}{2} \\
          \sqrt{\frac{E}{m}}                       & = v                             \\
          \sqrt{\frac{6.00 \times 10^{-2}}{0.300}} & = v                             \\
          0.447 \,\textrm{m/s}                     & = v
        \end{align*}
\end{enumerate}

\subsection{VP14.4.3}

\begin{enumerate}[a)]
  \item
        \begin{align*}
          E                                         & = \frac{1}{2}kA^2 \\
          \frac{2E}{A^2}                            & = k               \\
          \frac{2(4.00 \times 10^{-3})}{(0.0300)^2} & = k               \\
          8.89 \,\textrm{N/m}                       & = k
        \end{align*}

        \begin{align*}
          E                                        & = \frac{1}{2}mv^2 \\
          \frac{2E}{v^2}                           & = m               \\
          \frac{2(4.00 \times 10^{-3})}{(0.125)^2} & = m               \\
          0.512 \,\textrm{kg}                      & = m
        \end{align*}

  \item
        \begin{align*}
          F = ma & = kA                           \\
          a      & = \frac{kA}{m}                 \\
                 & = \frac{(8.89)(0.0300)}{0.512} \\
                 & = 0.521 \,\textrm{m/s}^2
        \end{align*}

  \item
        \begin{align*}
          U                   & = \frac{1}{2}kx^2 \\
          \sqrt{\frac{2U}{k}} & = x
        \end{align*}

        \begin{align*}
          F = ma & = kx                                                \\
          a      & = \frac{\sqrt{2kU}}{m}                              \\
                 & = \frac{\sqrt{2(8.89)(3.00 \times 10^{-3})}}{0.512} \\
                 & = 0.451 \,\textrm{m/s}^2
        \end{align*}
\end{enumerate}

\subsection{VP14.4.4}

\begin{enumerate}[a)]
  \item At maximum displacement from equilibrium the object's velocity is 0 and all energy is stored in elastic potential energy \[E = \frac{1}{2}kA^2.\] The object's kinetic energy will be equal to $\frac{1}{3}$ of its total mechanical energy when its potential energy is equal to $\frac{2}{3}$
        \begin{align*}
          U               & = \frac{2}{3}E                            \\
          \frac{1}{2}kx^2 & = \frac{2}{3}\left(\frac{1}{2}kA^2\right) \\
          x               & = \pm\sqrt{\frac{2}{3}}A.
        \end{align*}

  \item Similarly, the object's kinertic energy will be equal to $\frac{4}{5}$ of the total mechanical energy when its potential energy is equal to $\frac{1}{5}$
        \begin{align*}
          U               & = \frac{1}{5}E                            \\
          \frac{1}{2}kx^2 & = \frac{1}{5}\left(\frac{1}{2}kA^2\right) \\
          x               & = \pm\frac{A}{\sqrt{5}}.
        \end{align*}
\end{enumerate}

\subsection{VP14.9.1}

\begin{enumerate}[a)]
  \item The period is given by \[T = \frac{1}{f} = \frac{1}{0.609} = 1.64 \,\textrm{s}.\]

  \item Rearranging the formula for $f$ we find that
        \begin{align*}
          f                      & = \frac{1}{2\pi}\sqrt{\frac{g}{L}} \\
          (2\pi f)^2L            & = g                                \\
          (2\pi(0.609))^2(0.500) & = g                                \\
          7.32 \,\textrm{m/s}^2  & = g.
        \end{align*}
\end{enumerate}

\subsection{VP14.9.2}

The frequency of the air-track glider is given by \[f = \frac{1}{2\pi}\sqrt{\frac{k}{m}}.\]

The frequency of the simple pendulum is given by \[f = \frac{1}{2\pi}\sqrt{\frac{g}{L}}.\]

If their frequencies are to be equal then
\begin{align*}
  \frac{1}{2\pi}\sqrt{\frac{k}{m}} & = \frac{1}{2\pi}\sqrt{\frac{g}{L}} \\
  \frac{k}{m}                      & = \frac{g}{L}                      \\
  L                                & = \frac{gm}{k}                     \\
                                   & = \frac{(9.80)(0.350)}{8.75}       \\
                                   & = 0.392 \,\textrm{m}
\end{align*}

\subsection{VP14.9.3}

Assuming the tire is hung from its rim then it is a physical pendulum and
\begin{align*}
  f & = \frac{1}{2\pi}\sqrt{\frac{Mgd}{I}}     \\
    & = \frac{1}{2\pi}\sqrt{\frac{MgR}{2MR^2}} \\
    & = \frac{1}{2\pi}\sqrt{\frac{g}{2R}}.
\end{align*}

\subsection{VP14.9.4}

The rod is a physical pendulum. Rearranging the formula for period we find
\begin{align*}
  T                                                    & = 2\pi\sqrt{\frac{I}{mgd}} \\
  mgd\left(\frac{T}{2\pi}\right)^2                     & = I                        \\
  (0.600)(9.80)(0.500)\left(\frac{1.59}{2\pi}\right)^2 & = I                        \\
  0.188 \,\textrm{kg/m}^2                              & = I
\end{align*}

\subsection{Bridging Problem}

In order for the centre of mass of the cylinders to undergo simple harmonic motion it must experience a net force of the form \[F = -k'x.\]

The normal and weight forces cancel, leaving only the friction force from the cylinders on the ground $f$ and the restorative force from the spring $-kx$ \[F = f - kx.\]

The cylinders roll without slipping so
\begin{align*}
  \omega               & = -\frac{v}{R}               \\
  \frac{d}{dt}(\omega) & = \frac{d}{dt}(-\frac{v}{R}) \\
  \alpha               & = -\frac{a}{R}.
\end{align*}

The friction force generates a torque
\begin{align*}
  \tau = I\alpha                                        & = fR \\
  \left(\frac{1}{2}MR^2\right)\left(-\frac{a}{R}\right) & = fR \\
  -\frac{1}{2}Ma                                        & = f.
\end{align*}

Substituting this into the force equation we find that
\begin{align*}
  F = Ma        & = -\frac{1}{2}Ma - kx \\
  \frac{3}{2}Ma & = -kx                 \\
  Ma            & = -\frac{2}{3}kx
\end{align*}

which matches the required form of $F = -k'x$ where $k' = \frac{2}{3}k$.

This gives a period of
\begin{align*}
  T & = 2\pi\sqrt{\frac{m}{k'}}   \\
    & = 2\pi\sqrt{\frac{3m}{2k}}.
\end{align*}

\end{document}