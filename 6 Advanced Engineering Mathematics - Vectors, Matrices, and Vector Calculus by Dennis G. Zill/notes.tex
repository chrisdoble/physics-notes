\documentclass{article}
\usepackage{amsfonts} % For \mathbb
\usepackage{amsmath} % For align*
\usepackage{enumitem} % For customisable list labels
\usepackage{graphicx} % For images
\usepackage{siunitx} % For units
\graphicspath{{./images/}}

\title{Advanced Engineering Mathematics Vectors, Matrices, and Vector Calculus by Dennis G. Zill Notes}
\author{Chris Doble}
\date{June 2023}

\begin{document}

\maketitle

\tableofcontents

\section{Vectors}

\subsection{Vectors in 2-Space}

\begin{itemize}
  \item The zero vector can be assigned any direction

  \item The vectors $\textbf{i}$ and $\textbf{j}$ are known as the \textbf{standard basis vectors} for $\mathbb{R}^2$
\end{itemize}

\subsection{Vectors in 3-Space}

\begin{itemize}
  \item In $\mathbb{R}^3$ the octant in which all coordinates are positive is known as the \textbf{first octant}. There is no agreement for naming the other seven octants.
\end{itemize}

\subsection{Dot Product}

\begin{itemize}
  \item The \textbf{dot product} is also known as the \textbf{inner product} or the \textbf{scalar product} and is denoted $\mathbf{a} \cdot \mathbf{b}$

  \item Two non-zero vectors are orthogonal iff their dot product is $0$

  \item The zero vector is considered orthogonal to all vectors

  \item The angles $\alpha$, $\beta$, and $\gamma$ between a vector and the unit vectors $\mathbf{i}$, $\mathbf{j}$, and $\mathbf{k}$, respectively are called the \textbf{direction angles} of the vector

  \item The cosines of a vectors direction angles (the \textbf{direction cosines}) can be calculated as

        \begin{align*}
          \cos \alpha & = \frac{\mathbf{a} \cdot \mathbf{i}}{||\mathbf{a}|| ||\mathbf{i}||} \\
                      & = \frac{a_1}{||\mathbf{a}||}                                        \\
          \cos \beta  & = \frac{\mathbf{a} \cdot \mathbf{j}}{||\mathbf{a}|| ||\mathbf{j}||} \\
                      & = \frac{a_2}{||\mathbf{a}||}                                        \\
          \cos \gamma & = \frac{\mathbf{a} \cdot \mathbf{k}}{||\mathbf{a}|| ||\mathbf{k}||} \\
                      & = \frac{a_3}{||\mathbf{a}||}
        \end{align*}

        Equivalently, these can be calculated as the components of the unit vector $\mathbf{a} / ||\mathbf{a}||$.

  \item To find the component of a vector $\mathbf{a}$ in the direction of a vector $\mathbf{b}$ \[\text{comp}_\mathbf{b} \mathbf{a} = ||\mathbf{a}|| \cos \theta = \frac{\mathbf{a} \cdot \mathbf{b}}{||\mathbf{b}||}\]

  \item To project a vector $\mathbf{a}$ onto a vector $\mathbf{b}$ \[\text{proj}_\mathbf{b} \mathbf{a} = (\text{comp}_\mathbf{b} \mathbf{a}) \frac{\mathbf{b}}{||\mathbf{b}||} = \left( \frac{\mathbf{a} \cdot \mathbf{b}}{\mathbf{b} \cdot \mathbf{b}} \right) \mathbf{b}\]
\end{itemize}

\subsection{Cross Product}

\begin{itemize}
  \item The cross product is only defined in $\mathbb{R}^3$

  \item The \textbf{scalar triple product} of vectors $\mathbf{a}$, $\mathbf{b}$, and $\mathbf{c}$ is defined as \[\mathbf{a} \cdot (\mathbf{b} \times \mathbf{c}) = (\mathbf{a} \times \mathbf{b}) \cdot \mathbf{c} = \begin{vmatrix}
            a_1 & a_2 & a_3 \\
            b_1 & b_2 & b_3 \\
            c_1 & c_2 & c_3
          \end{vmatrix}\]

  \item The area of a parallelogram with sides $\mathbf{a}$ and $\mathbf{b}$ is $||\mathbf{a} \times \mathbf{b}||$

  \item The area of a triangle with sides $\mathbf{a}$ and $\mathbf{b}$ is $\frac{1}{2} ||\mathbf{a} \times \mathbf{b}||$

  \item The volume of a paralleleipied with sides $\mathbf{a}$, $\mathbf{b}$, and $\mathbf{c}$ is $|\mathbf{a} \cdot (\mathbf{b} \times \mathbf{c})|$

  \item $\mathbf{a} \cdot (\mathbf{b} \times \mathbf{c}) = 0$ iff $\mathbf{a}$, $\mathbf{b}$, and $\mathbf{c}$ are coplanar
\end{itemize}

\end{document}