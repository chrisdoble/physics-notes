\documentclass{article}
\usepackage{amsmath} % For align*
\usepackage{amsfonts} % For open face letters
\usepackage{enumitem} % For customisable list labels
\usepackage{graphicx} % For images
\usepackage{siunitx} % For units
\graphicspath{{./images/}}

\setlist[enumerate, 1]{label={(\alph*)}}
\setlist[enumerate, 2]{label={(\roman*)}}

\title{Advanced Engineering Mathematics Vectors, Matrices, and Vector Calculus by Dennis G. Zill Problems}
\author{Chris Doble}
\date{June 2023}

\begin{document}

\maketitle

\tableofcontents

\section{Vectors}

\subsection{Vectors in 2-Space}

\subsubsection{}

\begin{enumerate}
  \item $3 \mathbf{a} = 6 \mathbf{i} + 12 \mathbf{j}$

  \item $\mathbf{a} + \mathbf{b} = \mathbf{i} + 8 \mathbf{j}$

  \item $\mathbf{a} - \mathbf{b} = 3 \mathbf{i}$

  \item $||\mathbf{a} + \mathbf{b}|| = \sqrt{1 + 8^2} = \sqrt{65}$

  \item $||\mathbf{a} - \mathbf{b}|| = 3$
\end{enumerate}

\setcounter{subsubsection}{8}
\subsubsection{}

\begin{enumerate}
  \item $4 \mathbf{a} - 2 \mathbf{b} = \langle 6, -14 \rangle$

  \item $-3 \mathbf{a} - 5 \mathbf{b} = \langle 2, 4 \rangle$
\end{enumerate}

\setcounter{subsubsection}{14}
\subsubsection{}

$\overrightarrow{P_1 P_2} = \langle 2, 5 \rangle$

\setcounter{subsubsection}{18}
\subsubsection{}

$(1, 18)$

\setcounter{subsubsection}{20}
\subsubsection{}

\begin{enumerate}
  \item Yes

  \item Yes

  \item Yes

  \item No

  \item Yes

  \item Yes
\end{enumerate}

\setcounter{subsubsection}{24}
\subsubsection{}

\begin{enumerate}
  \item $\frac{\mathbf{a}}{||\mathbf{a}||} = \frac{\langle 2, 2 \rangle}{\sqrt{2^2 + 2^2}} = \frac{1}{2 \sqrt{2}} \langle 2, 2 \rangle = \langle \frac{1}{\sqrt{2}}, \frac{1}{\sqrt{2}} \rangle$

  \item $\langle -\frac{1}{\sqrt{2}}, -\frac{1}{\sqrt{2}} \rangle$
\end{enumerate}

\setcounter{subsubsection}{30}
\subsubsection{}

$2 \frac{\mathbf{a}}{||\mathbf{a}||} = 2 \frac{\langle 3, 7 \rangle}{\sqrt{3^2 + 7^2}} = \frac{2}{\sqrt{58}} \langle 3, 7 \rangle = \langle \frac{6}{\sqrt{58}}, \frac{14}{\sqrt{58}} \rangle$

\setcounter{subsubsection}{36}
\subsubsection{}

$\mathbf{x} = -(\mathbf{a} + \mathbf{b})$

\setcounter{subsubsection}{40}
\subsubsection{}

\begin{align*}
  \mathbf{a} & = 2 \mathbf{i} + 3 \mathbf{j}                                                                                     \\
  \mathbf{b} & = \mathbf{i} + \mathbf{j}                                                                                         \\
  \mathbf{c} & = \mathbf{i} - \mathbf{j}                                                                                         \\ \\
  \mathbf{i} & = \frac{1}{2} (\mathbf{b} + \mathbf{c})                                                                           \\
  \mathbf{j} & = \frac{1}{2} (\mathbf{b} - \mathbf{c})                                                                           \\
  \mathbf{a} & = 2 \left( \frac{1}{2} (\mathbf{b} + \mathbf{c}) \right) + 3 \left( \frac{1}{2} (\mathbf{b} - \mathbf{c}) \right) \\
             & = \mathbf{b} + \mathbf{c} + \frac{3}{2} \mathbf{b} - \frac{3}{2} \mathbf{c}                                       \\
             & = \frac{5}{2} \mathbf{b} - \frac{1}{2} \mathbf{c}
\end{align*}

\setcounter{subsubsection}{42}
\subsubsection{}

\begin{align*}
  y          & = \frac{1}{4} x^2 + 1                                        \\
  y(2)       & = 2                                                          \\
  y'         & = \frac{1}{2} x                                              \\
  y'(2)      & = 1                                                          \\
  \mathbf{v} & = \pm \langle \frac{1}{\sqrt{2}}, \frac{1}{\sqrt{2}} \rangle
\end{align*}

\setcounter{subsubsection}{44}
\subsubsection{}

\begin{enumerate}
  \item

        \begin{align*}
          \mathbf{F}_n                & = \mathbf{F} \cos \theta         \\
          \mathbf{F}_g                & = \mathbf{F} \sin \theta         \\
          ||\mathbf{F}_f||            & = \mu ||\mathbf{F}_n||           \\
          ||-\mathbf{F}_g||           & = \mu ||\mathbf{F}_n||           \\
          ||-\mathbf{F} \sin \theta|| & = \mu ||\mathbf{F} \cos \theta|| \\
          ||\mathbf{F}|| \sin \theta  & = \mu ||\mathbf{F}|| \cos \theta \\
          \tan \theta                 & = \mu
        \end{align*}

  \item $\theta = \arctan \mu \approx \ang{31}$
\end{enumerate}

\setcounter{subsubsection}{46}
\subsubsection{}

\begin{align*}
  F_x        & = \frac{q Q}{4 \pi \epsilon_0} \int_{-a}^a \frac{L \,dy}{2 a (L^2 + y^2)^{3 / 2}}  \\
             & = \frac{L q Q}{8 \pi \epsilon_0} \int_{-a}^a (L^2 + y^2)^{-3 / 2} \,dy             \\
             & = \frac{L q Q}{8 \pi \epsilon_0} \frac{2 a}{L^2 \sqrt{a^2 + L^2}}                  \\
             & = \frac{a q Q}{4 \pi \epsilon_0 L \sqrt{a^2 + L^2}}                                \\
  F_y        & = -\frac{q Q}{4 \pi \epsilon_0} \int_{-a}^a \frac{y \,dy}{2 a (L^2 + y^2)^{3 / 2}} \\
             & = 0                                                                                \\
  \mathbf{F} & = \langle \frac{1}{4 \pi \epsilon_0} \frac{q Q}{L \sqrt{a^2 + L^2}}, 0 \rangle
\end{align*}

\setcounter{subsubsection}{48}
\subsubsection{}

Let the three sides of the triangle be vectors $\mathbf{a}$, $\mathbf{b}$, and $\mathbf{c}$. The triangle is closed so it must be the case that \[\mathbf{a} + \mathbf{b} + \mathbf{c} = 0.\] This gives \[\mathbf{c} = -(\mathbf{a} + \mathbf{b}).\] The vector from the midpoint of side $\mathbf{a}$ to the midpoint of side $\mathbf{b}$ is \[\left( \mathbf{a} + \frac{1}{2} \mathbf{b} \right) - \frac{1}{2} \mathbf{a} = \frac{1}{2} (\mathbf{a} + \mathbf{b})\] which is parallel with $\mathbf{c}$ and half its length.

\subsection{Vectors in 3-Space}

\setcounter{subsubsection}{6}
\subsubsection{}

A plane at $z = 5$ parellel with the $x$-$y$ plane.

\setcounter{subsubsection}{8}
\subsubsection{}

A line parallel to the $z$ axis at $x = 2$ and $y = 3$.

\setcounter{subsubsection}{12}
\subsubsection{}

\begin{enumerate}
  \item $(0, 5, 4)$, $(-2, 0, 4)$, $(-2, 5, 0)$

  \item $(-2, 5, -2)$

  \item $(3, 5, 4)$
\end{enumerate}

\setcounter{subsubsection}{14}
\subsubsection{}

The planes $x = 0$, $y = 0$, and $z = 0$.

\setcounter{subsubsection}{16}
\subsubsection{}

$(-1, 2, -3)$

\setcounter{subsubsection}{18}
\subsubsection{}

The planes $z = \pm 5$.

\setcounter{subsubsection}{20}
\subsubsection{}

$\sqrt{(6 - 3)^2 + (4 + 1)^2 + (8 - 2)^2} = \sqrt{9 + 25 + 36} = \sqrt{70}$

\setcounter{subsubsection}{30}
\subsubsection{}

\begin{align*}
  \sqrt{(2 - x)^2 + (1 - 2)^2 + (1 - 3)^2} & = \sqrt{21}        \\
  (2 - x)^2 + 1 + 4                        & = 21               \\
  (2 - x)^2                                & = 16               \\
  2 - x                                    & = \pm 4            \\
  x                                        & = 2 \pm 4          \\
                                           & = -2 \text{ or } 6
\end{align*}

\setcounter{subsubsection}{32}
\subsubsection{}

$\left( 4, \frac{1}{2}, \frac{3}{2} \right)$

\setcounter{subsubsection}{36}
\subsubsection{}

$(-3, -6, 1)$

\subsection{Dot Product}

\subsubsection{}

$\mathbf{a} \cdot \mathbf{b} = 12$

\setcounter{subsubsection}{10}
\subsubsection{}

$\left( \frac{\mathbf{a} \cdot \mathbf{b}}{\mathbf{b} \cdot \mathbf{b}} \right) \mathbf{b} = \frac{12}{30} \mathbf{b} = \langle -\frac{2}{5}, \frac{4}{5}, 2 \rangle$

\setcounter{subsubsection}{12}
\subsubsection{}

$\mathbf{a} \cdot \mathbf{b} = ||\mathbf{a}|| ||\mathbf{b}|| \cos \theta = 25 \sqrt{2}$

\setcounter{subsubsection}{16}
\subsubsection{}

\begin{align*}
  \mathbf{a} \cdot \mathbf{v} & = 0                                            \\
  3 x_1 + y_1 - 1             & = 0                                            \\ \\
  \mathbf{b} \cdot \mathbf{v} & = 0                                            \\
  -3 x_1 + 2 y_2 + 2          & = 0                                            \\ \\
  3 y_2 + 1                   & = 0                                            \\
  y_2                         & = -\frac{1}{3}                                 \\ \\
  3 x_1 - \frac{1}{3} - 1     & = 0                                            \\
  x_1                         & = \frac{4}{9}                                  \\ \\
  \mathbf{v}                  & = \langle \frac{4}{9}, -\frac{1}{3}, 1 \rangle
\end{align*}

\setcounter{subsubsection}{18}
\subsubsection{}

\begin{align*}
  \mathbf{a} \cdot \mathbf{c} & = \mathbf{a} \cdot \left( \mathbf{b} - \frac{\mathbf{a} \cdot \mathbf{b}}{||\mathbf{a}||^2} \mathbf{a} \right)   \\
                              & = \mathbf{a} \cdot \mathbf{b} - \frac{\mathbf{a} \cdot \mathbf{b}}{||\mathbf{a}||^2} \mathbf{a} \cdot \mathbf{a} \\
                              & = 0
\end{align*}

\setcounter{subsubsection}{20}
\subsubsection{}

\begin{align*}
  ||\mathbf{a}||              & = \sqrt{3^2 + 1^2}                                                          \\
                              & = \sqrt{10}                                                                 \\
  ||\mathbf{b}||              & = \sqrt{2^2 + 2^2}                                                          \\
                              & = \sqrt{8}                                                                  \\
                              & = 2 \sqrt{2}                                                                \\
  \mathbf{a} \cdot \mathbf{b} & = 4                                                                         \\
  \theta                      & = \arccos \frac{\mathbf{a} \cdot \mathbf{b}}{||\mathbf{a}|| ||\mathbf{b}||} \\
                              & = \arccos \frac{4}{(\sqrt{10}) (2 \sqrt{2})}                                \\
                              & = \arccos \frac{1}{\sqrt{5}}                                                \\
                              & \approx \ang{63}
\end{align*}

\setcounter{subsubsection}{24}
\subsubsection{}

\begin{align*}
  ||\mathbf{a}|| & = \sqrt{1^2 + 2^2 + 3^3} \\
                 & = \sqrt{14}              \\
  \cos \alpha    & = \frac{1}{\sqrt{14}}    \\
  \alpha         & \approx \ang{75}         \\
  \cos \beta     & = \frac{2}{\sqrt{14}}    \\
  \beta          & \approx \ang{58}         \\
  \cos \gamma    & = \frac{3}{\sqrt{14}}    \\
  \gamma         & \approx \ang{37}
\end{align*}

\setcounter{subsubsection}{28}
\subsubsection{}

\begin{align*}
  \overrightarrow{A D}     & = \langle s, -s, s \rangle                                                                                          \\
  ||\overrightarrow{A D}|| & = s \sqrt{3}                                                                                                        \\
  \overrightarrow{A B}     & = \langle s, 0, 0 \rangle                                                                                           \\
  ||\overrightarrow{A B}|| & = s                                                                                                                 \\
  \theta                   & = \arccos \frac{\overrightarrow{A D} \cdot \overrightarrow{A B}}{||\overrightarrow{A D}|| ||\overrightarrow{A B}||} \\
                           & = \arccos \frac{s^2}{s^2 \sqrt{3}}                                                                                  \\
                           & = \arccos \frac{1}{\sqrt{3}}                                                                                        \\
                           & \approx \ang{55}
\end{align*}

\setcounter{subsubsection}{32}
\subsubsection{}

\begin{align*}
  \text{comp}_\mathbf{b} \mathbf{a} & = \frac{\mathbf{a} \cdot \mathbf{b}}{||\mathbf{b}||} \\
                                    & = \frac{5}{7}
\end{align*}

\setcounter{subsubsection}{36}
\subsubsection{}

\begin{align*}
  \text{comp}_{\overrightarrow{O P}} \mathbf{a} & = \frac{\mathbf{a} \cdot \overrightarrow{O P}}{||\overrightarrow{O P}||} \\
                                                & = \frac{72}{\sqrt{109}}
\end{align*}

\setcounter{subsubsection}{38}
\subsubsection{}

\begin{align*}
  \text{proj}_\mathbf{b} \mathbf{a} & = \left( \frac{\mathbf{a} \cdot \mathbf{b}}{\mathbf{b} \cdot \mathbf{b}} \right) \mathbf{b} \\
                                    & = \frac{35}{25} \mathbf{b}                                                                  \\
                                    & = \langle -\frac{21}{5}, \frac{28}{5} \rangle
\end{align*}

\setcounter{subsubsection}{42}
\subsubsection{}

\begin{align*}
  \mathbf{a} + \mathbf{b}                          & = \langle 3, 4 \rangle                                                                                                                                  \\
  \text{proj}_{\mathbf{a} + \mathbf{b}} \mathbf{a} & = \left( \frac{\mathbf{a} \cdot (\mathbf{a} + \mathbf{b})}{(\mathbf{a} + \mathbf{b}) \cdot (\mathbf{a} + \mathbf{b})} \right) (\mathbf{a} + \mathbf{b}) \\
                                                   & = \frac{24}{25} (\mathbf{a} + \mathbf{b})                                                                                                               \\
                                                   & = \langle \frac{72}{25}, \frac{96}{25} \rangle
\end{align*}

\setcounter{subsubsection}{44}
\subsubsection{}

$W = \mathbf{F} \cdot \mathbf{d} = F d \cos \theta = 1000$

\setcounter{subsubsection}{46}
\subsubsection{}

\begin{enumerate}
  \item $W = 0$

  \item

        \begin{align*}
          ||\mathbf{d}|| & = \sqrt{4^2 + 3^2}                           \\
                         & = 5                                          \\
          \mathbf{F}     & = F \hat{\mathbf{d}}                         \\
                         & = F \frac{\mathbf{d}}{||\mathbf{d}||}        \\
                         & = F \langle \frac{4}{5}, \frac{3}{5} \rangle \\
                         & = \langle 24, 18 \rangle                     \\
          W              & = \mathbf{F} \cdot \mathbf{d}                \\
                         & = \qty{150}{J}
        \end{align*}
\end{enumerate}

\subsection{Cross Product}

\subsubsection{}

\begin{align*}
  \mathbf{a} \times \mathbf{b} & = \begin{vmatrix}
                                     \mathbf{i} & \mathbf{j} & \mathbf{k} \\
                                     1          & -1         & 0          \\
                                     0          & 3          & 5
                                   \end{vmatrix}        \\
                               & = -5 \mathbf{i} - 5 \mathbf{j} + 3 \mathbf{k}
\end{align*}

\setcounter{subsubsection}{10}
\subsubsection{}

\begin{align*}
  \overrightarrow{P_1 P_2} \times \overrightarrow{P_1 P_3} & = \begin{vmatrix}
                                                                 \mathbf{i} & \mathbf{j} & \mathbf{k} \\
                                                                 -2         & 2          & -4         \\
                                                                 -3         & 1          & 1
                                                               \end{vmatrix}        \\
                                                           & = 6 \mathbf{i} + 14 \mathbf{j} + 4 \mathbf{k}
\end{align*}

\setcounter{subsubsection}{16}
\subsubsection{}

\begin{enumerate}
  \item

        \begin{align*}
          \mathbf{b} \times \mathbf{c}                     & = \begin{vmatrix}
                                                                 \mathbf{i} & \mathbf{j} & \mathbf{k} \\
                                                                 2          & 1          & 1          \\
                                                                 3          & 1          & 1
                                                               \end{vmatrix}  \\
                                                           & = \mathbf{j} - \mathbf{k}               \\
          \mathbf{a} \times (\mathbf{b} \times \mathbf{c}) & = \begin{vmatrix}
                                                                 \mathbf{i} & \mathbf{j} & \mathbf{k} \\
                                                                 1          & -1         & 2          \\
                                                                 0          & 1          & -1
                                                               \end{vmatrix}  \\
                                                           & = -\mathbf{i} + \mathbf{j} + \mathbf{k}
        \end{align*}
\end{enumerate}

\setcounter{subsubsection}{18}
\subsubsection{}

$2 \mathbf{k}$

\setcounter{subsubsection}{20}
\subsubsection{}

\begin{align*}
  \mathbf{k} \times (2 \mathbf{i} - \mathbf{j}) & = (\mathbf{k} \times 2 \mathbf{i}) - (\mathbf{k} \times \mathbf{j}) \\
                                                & = \mathbf{i} + 2 \mathbf{j}
\end{align*}

\setcounter{subsubsection}{22}
\subsubsection{}

\begin{align*}
  [(2 \mathbf{k}) \times (3 \mathbf{j})] \times (4 \mathbf{j}) & = (-6 \mathbf{i}) \times (4 \mathbf{j}) \\
                                                               & = -24 \mathbf{k}
\end{align*}

\setcounter{subsubsection}{36}
\subsubsection{}

$12 \mathbf{i} - 9 \mathbf{j} + 18 \mathbf{k}$

\setcounter{subsubsection}{52}
\subsubsection{}

\begin{align*}
  \mathbf{b} \times \mathbf{c}                    & = \begin{vmatrix}
                                                        \mathbf{i}  & \mathbf{j} & \mathbf{k}  \\
                                                        -2          & 6          & -6          \\
                                                        \frac{5}{2} & 3          & \frac{1}{2}
                                                      \end{vmatrix}        \\
                                                  & = 21 \mathbf{i} - 14 \mathbf{j} - 21 \mathbf{k} \\
  \mathbf{a} \cdot (\mathbf{b} \times \mathbf{c}) & = 4 \times 21 + 6 \times (-14)                  \\
                                                  & = 0
\end{align*}

They are coplanar.

\subsection{Lines and Planes in 3-Space}

\subsubsection{}

$\mathbf{r} = \langle 1, 2, 1 \rangle + t \langle 2, 3, -3 \rangle$

\setcounter{subsubsection}{6}
\subsubsection{}

\begin{align*}
  x & = 2 + 4 t \\
  y & = 3 - 4 t \\
  z & = 5 + 3 t
\end{align*}

\setcounter{subsubsection}{12}
\subsubsection{}

\begin{align*}
  x & = 1 + 9 t  \\
  y & = 4 + 10 t \\
  z & = -9 + 7 t \\
  \frac{x - 1}{9} = \frac{y - 4}{10} = \frac{z + 9}{7}
\end{align*}

\setcounter{subsubsection}{18}
\subsubsection{}

\begin{align*}
  x & = 4 + 3 t            \\
  y & = 6 + \frac{1}{2} t  \\
  z & = -7 - \frac{3}{2} t \\
  \frac{x - 4}{3} = \frac{y - 6}{1 / 2} = -\frac{z + 7}{3 / 2}
\end{align*}

\setcounter{subsubsection}{22}
\subsubsection{}

\begin{align*}
  x & = 6 + 2 t  \\
  y & = 4 - 3 t  \\
  z & = -2 + 6 t
\end{align*}

\setcounter{subsubsection}{24}
\subsubsection{}

\begin{align*}
  x & = 2 + t \\
  y & = -2    \\
  z & = 15
\end{align*}

\setcounter{subsubsection}{28}
\subsubsection{}

$(0, 5, 15), \,(5, 0, \frac{15}{2}), \,(10, -5, 0)$

\setcounter{subsubsection}{30}
\subsubsection{}

\begin{align*}
  4 + t_x    & = 6 + 2 t_x  \\
  t_x        & = -2         \\ \\
  5 + t_y    & = 11 + 4 t_y \\
  t_y        & = -2         \\ \\
  -1 + 2 t_z & = -3 + t_z   \\
  t_z        & = -2
\end{align*}

$(2, 3, -5)$

\setcounter{subsubsection}{34}
\subsubsection{}

\begin{align*}
  \mathbf{a}     & = \langle -1, 2, -2 \rangle                                                 \\
  ||\mathbf{a}|| & = 3                                                                         \\
  \mathbf{b}     & = \langle 2, 3, -6 \rangle                                                  \\
  ||\mathbf{b}|| & = 7                                                                         \\
  \theta         & = \arccos \frac{\mathbf{a} \cdot \mathbf{b}}{||\mathbf{a}|| ||\mathbf{b}||} \\
                 & \approx \ang{40.37}
\end{align*}

\setcounter{subsubsection}{36}
\subsubsection{}

\begin{align*}
  \mathbf{a}                   & = \langle 1, 1, 1 \rangle                                                                    \\
  \mathbf{b}                   & = \langle -2, 1, -5 \rangle                                                                  \\
  \mathbf{a} \times \mathbf{b} & = \begin{vmatrix}
                                     \hat{\mathbf{i}} & \hat{\mathbf{j}} & \hat{\mathbf{k}} \\
                                     1                & 1                & 1                \\
                                     -2               & 1                & -5
                                   \end{vmatrix} \\
                               & = \langle -6, 3, 3 \rangle                                                                   \\
  x                            & = 4 - 6 t                                                                                    \\
  y                            & = 1 + 3 t                                                                                    \\
  z                            & = 6 + 3 t
\end{align*}

\setcounter{subsubsection}{38}
\subsubsection{}

\begin{align*}
  \langle 2, -3, 4 \rangle \cdot (\mathbf{r} - \langle 5, 1, 3 \rangle) & = 0 \\
  2 (x - 5) - 3 (y - 1) + 4 (z - 3)                                     & = 0 \\
  2 x - 3 y + 4 z - 19                                                  & = 0
\end{align*}

\setcounter{subsubsection}{44}
\subsubsection{}

\begin{align*}
  \mathbf{a}                                                                             & = \langle 3, 5, 2 \rangle              \\
  \mathbf{b}                                                                             & = \langle 2, 3, 1 \rangle              \\
  \mathbf{c}                                                                             & = \langle -1, -1, 4 \rangle            \\
  \mathbf{a} - \mathbf{c}                                                                & = \langle 4, 6, -2 \rangle             \\
  \mathbf{b} - \mathbf{c}                                                                & = \langle 3, 4, -3 \rangle             \\
  (\mathbf{a} - \mathbf{c}) \times (\mathbf{b} - \mathbf{c})                             & = \begin{vmatrix}
                                                                                               \mathbf{i} & \mathbf{j} & \mathbf{k} \\
                                                                                               4          & 6          & -2         \\
                                                                                               3          & 4          & -3
                                                                                             \end{vmatrix} \\
                                                                                         & = \langle -10, 6, -2 \rangle           \\
  \mathbf{n} \cdot (\mathbf{r} - \mathbf{c})                                             & = 0                                    \\
  \langle -10, 6, -2 \rangle \cdot (\langle x, y, z \rangle - \langle -1, -1, 4 \rangle) & = 0                                    \\
  -10 (x + 1) + 6 (y + 1) - 2 (z - 4)                                                    & = 0                                    \\
  -10 x + 6 y - 2 z + 4                                                                  & = 0
\end{align*}

\setcounter{subsubsection}{50}
\subsubsection{}

\begin{align*}
  \langle 1, 1, -4 \rangle \cdot (\mathbf{r} - \langle 2, 3, -5 \rangle) & = 0  \\
  (x - 2) + (y - 3) - 4 (z + 5)                                          & = 0  \\
  x + y - 4 z                                                            & = 25
\end{align*}

\setcounter{subsubsection}{62}
\subsubsection{}

\begin{enumerate}
  \item Not perpendicular

  \item Not perpendicular

  \item Perpendicular

  \item Perpendicular
\end{enumerate}

\setcounter{subsubsection}{64}
\subsubsection{}

\begin{align*}
  5 x - 4 y - 9 t & = 8               \\
  x + 4 y + 3 t   & = 4               \\ \\
  6 x - 6 t       & = 12              \\
  x               & = 2 + t           \\ \\
  y               & = \frac{1}{2} - t \\ \\
  z               & = t
\end{align*}

\setcounter{subsubsection}{68}
\subsubsection{}

\begin{align*}
  2 (1 + 2 t) - 3 (2 - t) + 2 (-3 t) & = -7 \\
  t                                  & = -3 \\
  x                                  & = -5 \\
  y                                  & = 5  \\
  z                                  & = 9
\end{align*}

\setcounter{subsubsection}{72}
\subsubsection{}

\begin{align*}
  x + y - 4 t       & = 2        \\
  2 x - y + t       & = 10       \\ \\
  3 x - 3 t         & = 12       \\
  x                 & = 4 + t    \\ \\
  2 (4 + t) - y + t & = 10       \\
  8 + 2 t - y + t   & = 10       \\
  y                 & = -2 + 3 t \\ \\
  z                 & = t        \\ \\
  x                 & = 5 + t    \\
  y                 & = 6 + 3 t  \\
  z                 & = -12 + t
\end{align*}

\setcounter{subsubsection}{74}
\subsubsection{}

\begin{align*}
  \mathbf{n}                                              & = \begin{vmatrix}
                                                                \mathbf{i} & \mathbf{j} & \mathbf{k} \\
                                                                3          & -1         & 5          \\
                                                                1          & 1          & 1
                                                              \end{vmatrix} \\
                                                          & = \langle -6, 2, 4 \rangle             \\
  \mathbf{n} \cdot (\mathbf{r} - \langle 4, 0, 1 \rangle) & = 0                                    \\
  -6 (x - 4) + 2 y + 4 (z - 1)                            & = 0                                    \\
  -6 x + 2 y + 4 z                                        & = -20                                  \\
  3 x - y - 2 z                                           & = 10
\end{align*}

\subsection{Vector Spaces}

\subsubsection{}

Violates axiom 6

\setcounter{subsubsection}{2}
\subsubsection{}

Violates axiom 10

\setcounter{subsubsection}{4}
\subsubsection{}

Vector space

\setcounter{subsubsection}{6}
\subsubsection{}

Violates axiom 2

\setcounter{subsubsection}{8}
\subsubsection{}

Vector space

\setcounter{subsubsection}{10}
\subsubsection{}

Subspace

\setcounter{subsubsection}{12}
\subsubsection{}

Not a subspace

\setcounter{subsubsection}{14}
\subsubsection{}

Subspace

\setcounter{subsubsection}{16}
\subsubsection{}

Subspace

\setcounter{subsubsection}{18}
\subsubsection{}

Not a subspace

\setcounter{subsubsection}{22}
\subsubsection{}

\begin{enumerate}
  \item

        \begin{align*}
          k_1 \mathbf{u}_1 + k_2 \mathbf{u}_2 + k_3 \mathbf{u}_3                                  & = \mathbf{0} \\
          k_1 \langle 1, 0, 0 \rangle + k_2 \langle 1, 1, 0 \rangle + k_3 \langle 1, 1, 1 \rangle & = \mathbf{0} \\ \\
          k_3                                                                                     & = 0          \\ \\
          k_2 + k_3                                                                               & = 0          \\
          k_2                                                                                     & = 0          \\ \\
          k_1 + k_2 + k_3                                                                         & = 0          \\
          k_1                                                                                     & = 0
        \end{align*}

  \item \[\mathbf{a} = 7 \mathbf{u}_1 - 12 \mathbf{u}_2 + 8 \mathbf{u}_3\]
\end{enumerate}

\setcounter{subsubsection}{24}
\subsubsection{}

Dependent

\setcounter{subsubsection}{26}
\subsubsection{}

Independent

\setcounter{subsubsection}{28}
\subsubsection{}

$f(x)$ is undefined at $x = -3$ and $x = -1$.

\setcounter{subsubsection}{30}
\subsubsection{}

\begin{align*}
  ||x||      & = \sqrt{(x, x)}                                                      \\
             & = \sqrt{\int_0^{2 \pi} x^2 \,dx}                                     \\
             & = \sqrt{\left[ \frac{1}{3} x^3 \right]_0^{2 \pi}}                    \\
             & = \sqrt{\frac{8}{3} \pi^3}                                           \\
  ||\sin x|| & = \sqrt{(\sin x, \sin x)}                                            \\
             & = \sqrt{\int_0^{2 \pi} \sin^2 x \,dx}                                \\
             & = \sqrt{\left[ \frac{x}{2} - \frac{1}{4} \sin 2 x \right]_0^{2 \pi}} \\
             & = \sqrt{\pi}
\end{align*}

\end{document}