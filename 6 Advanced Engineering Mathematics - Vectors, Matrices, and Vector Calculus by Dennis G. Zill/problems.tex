\documentclass{article}
\usepackage{amsmath} % For align*
\usepackage{amsfonts} % For open face letters
\usepackage{enumitem} % For customisable list labels
\usepackage{graphicx} % For images
\usepackage{siunitx} % For units
\graphicspath{{./images/}}

\newcommand{\adj}{\operatorname{adj}}
\newcommand{\proj}{\operatorname{proj}}
\newcommand{\rank}{\operatorname{rank}}
\newenvironment{amatrix}[1]{%
  \left(\begin{array}{@{}*{#1}{c}|c@{}}
    }{%
  \end{array}\right)
}
\setlist[enumerate, 1]{label={(\alph*)}}
\setlist[enumerate, 2]{label={(\roman*)}}

\title{Advanced Engineering Mathematics Vectors, Matrices, and Vector Calculus by Dennis G. Zill Problems}
\author{Chris Doble}
\date{June 2023}

\begin{document}

\maketitle

\tableofcontents

\section{Vectors}

\subsection{Vectors in 2-Space}

\subsubsection{}

\begin{enumerate}
  \item $3 \mathbf{a} = 6 \mathbf{i} + 12 \mathbf{j}$

  \item $\mathbf{a} + \mathbf{b} = \mathbf{i} + 8 \mathbf{j}$

  \item $\mathbf{a} - \mathbf{b} = 3 \mathbf{i}$

  \item $||\mathbf{a} + \mathbf{b}|| = \sqrt{1 + 8^2} = \sqrt{65}$

  \item $||\mathbf{a} - \mathbf{b}|| = 3$
\end{enumerate}

\setcounter{subsubsection}{8}
\subsubsection{}

\begin{enumerate}
  \item $4 \mathbf{a} - 2 \mathbf{b} = \langle 6, -14 \rangle$

  \item $-3 \mathbf{a} - 5 \mathbf{b} = \langle 2, 4 \rangle$
\end{enumerate}

\setcounter{subsubsection}{14}
\subsubsection{}

$\overrightarrow{P_1 P_2} = \langle 2, 5 \rangle$

\setcounter{subsubsection}{18}
\subsubsection{}

$(1, 18)$

\setcounter{subsubsection}{20}
\subsubsection{}

\begin{enumerate}
  \item Yes

  \item Yes

  \item Yes

  \item No

  \item Yes

  \item Yes
\end{enumerate}

\setcounter{subsubsection}{24}
\subsubsection{}

\begin{enumerate}
  \item $\frac{\mathbf{a}}{||\mathbf{a}||} = \frac{\langle 2, 2 \rangle}{\sqrt{2^2 + 2^2}} = \frac{1}{2 \sqrt{2}} \langle 2, 2 \rangle = \langle \frac{1}{\sqrt{2}}, \frac{1}{\sqrt{2}} \rangle$

  \item $\langle -\frac{1}{\sqrt{2}}, -\frac{1}{\sqrt{2}} \rangle$
\end{enumerate}

\setcounter{subsubsection}{30}
\subsubsection{}

$2 \frac{\mathbf{a}}{||\mathbf{a}||} = 2 \frac{\langle 3, 7 \rangle}{\sqrt{3^2 + 7^2}} = \frac{2}{\sqrt{58}} \langle 3, 7 \rangle = \langle \frac{6}{\sqrt{58}}, \frac{14}{\sqrt{58}} \rangle$

\setcounter{subsubsection}{36}
\subsubsection{}

$\mathbf{x} = -(\mathbf{a} + \mathbf{b})$

\setcounter{subsubsection}{40}
\subsubsection{}

\begin{align*}
  \mathbf{a} & = 2 \mathbf{i} + 3 \mathbf{j}                                                                                     \\
  \mathbf{b} & = \mathbf{i} + \mathbf{j}                                                                                         \\
  \mathbf{c} & = \mathbf{i} - \mathbf{j}                                                                                         \\ \\
  \mathbf{i} & = \frac{1}{2} (\mathbf{b} + \mathbf{c})                                                                           \\
  \mathbf{j} & = \frac{1}{2} (\mathbf{b} - \mathbf{c})                                                                           \\
  \mathbf{a} & = 2 \left( \frac{1}{2} (\mathbf{b} + \mathbf{c}) \right) + 3 \left( \frac{1}{2} (\mathbf{b} - \mathbf{c}) \right) \\
             & = \mathbf{b} + \mathbf{c} + \frac{3}{2} \mathbf{b} - \frac{3}{2} \mathbf{c}                                       \\
             & = \frac{5}{2} \mathbf{b} - \frac{1}{2} \mathbf{c}
\end{align*}

\setcounter{subsubsection}{42}
\subsubsection{}

\begin{align*}
  y          & = \frac{1}{4} x^2 + 1                                        \\
  y(2)       & = 2                                                          \\
  y'         & = \frac{1}{2} x                                              \\
  y'(2)      & = 1                                                          \\
  \mathbf{v} & = \pm \langle \frac{1}{\sqrt{2}}, \frac{1}{\sqrt{2}} \rangle
\end{align*}

\setcounter{subsubsection}{44}
\subsubsection{}

\begin{enumerate}
  \item

        \begin{align*}
          \mathbf{F}_n                & = \mathbf{F} \cos \theta         \\
          \mathbf{F}_g                & = \mathbf{F} \sin \theta         \\
          ||\mathbf{F}_f||            & = \mu ||\mathbf{F}_n||           \\
          ||-\mathbf{F}_g||           & = \mu ||\mathbf{F}_n||           \\
          ||-\mathbf{F} \sin \theta|| & = \mu ||\mathbf{F} \cos \theta|| \\
          ||\mathbf{F}|| \sin \theta  & = \mu ||\mathbf{F}|| \cos \theta \\
          \tan \theta                 & = \mu
        \end{align*}

  \item $\theta = \arctan \mu \approx \ang{31}$
\end{enumerate}

\setcounter{subsubsection}{46}
\subsubsection{}

\begin{align*}
  F_x        & = \frac{q Q}{4 \pi \epsilon_0} \int_{-a}^a \frac{L \,dy}{2 a (L^2 + y^2)^{3 / 2}}  \\
             & = \frac{L q Q}{8 \pi \epsilon_0} \int_{-a}^a (L^2 + y^2)^{-3 / 2} \,dy             \\
             & = \frac{L q Q}{8 \pi \epsilon_0} \frac{2 a}{L^2 \sqrt{a^2 + L^2}}                  \\
             & = \frac{a q Q}{4 \pi \epsilon_0 L \sqrt{a^2 + L^2}}                                \\
  F_y        & = -\frac{q Q}{4 \pi \epsilon_0} \int_{-a}^a \frac{y \,dy}{2 a (L^2 + y^2)^{3 / 2}} \\
             & = 0                                                                                \\
  \mathbf{F} & = \langle \frac{1}{4 \pi \epsilon_0} \frac{q Q}{L \sqrt{a^2 + L^2}}, 0 \rangle
\end{align*}

\setcounter{subsubsection}{48}
\subsubsection{}

Let the three sides of the triangle be vectors $\mathbf{a}$, $\mathbf{b}$, and $\mathbf{c}$. The triangle is closed so it must be the case that \[\mathbf{a} + \mathbf{b} + \mathbf{c} = 0.\] This gives \[\mathbf{c} = -(\mathbf{a} + \mathbf{b}).\] The vector from the midpoint of side $\mathbf{a}$ to the midpoint of side $\mathbf{b}$ is \[\left( \mathbf{a} + \frac{1}{2} \mathbf{b} \right) - \frac{1}{2} \mathbf{a} = \frac{1}{2} (\mathbf{a} + \mathbf{b})\] which is parallel with $\mathbf{c}$ and half its length.

\subsection{Vectors in 3-Space}

\setcounter{subsubsection}{6}
\subsubsection{}

A plane at $z = 5$ parellel with the $x$-$y$ plane.

\setcounter{subsubsection}{8}
\subsubsection{}

A line parallel to the $z$ axis at $x = 2$ and $y = 3$.

\setcounter{subsubsection}{12}
\subsubsection{}

\begin{enumerate}
  \item $(0, 5, 4)$, $(-2, 0, 4)$, $(-2, 5, 0)$

  \item $(-2, 5, -2)$

  \item $(3, 5, 4)$
\end{enumerate}

\setcounter{subsubsection}{14}
\subsubsection{}

The planes $x = 0$, $y = 0$, and $z = 0$.

\setcounter{subsubsection}{16}
\subsubsection{}

$(-1, 2, -3)$

\setcounter{subsubsection}{18}
\subsubsection{}

The planes $z = \pm 5$.

\setcounter{subsubsection}{20}
\subsubsection{}

$\sqrt{(6 - 3)^2 + (4 + 1)^2 + (8 - 2)^2} = \sqrt{9 + 25 + 36} = \sqrt{70}$

\setcounter{subsubsection}{30}
\subsubsection{}

\begin{align*}
  \sqrt{(2 - x)^2 + (1 - 2)^2 + (1 - 3)^2} & = \sqrt{21}        \\
  (2 - x)^2 + 1 + 4                        & = 21               \\
  (2 - x)^2                                & = 16               \\
  2 - x                                    & = \pm 4            \\
  x                                        & = 2 \pm 4          \\
                                           & = -2 \text{ or } 6
\end{align*}

\setcounter{subsubsection}{32}
\subsubsection{}

$\left( 4, \frac{1}{2}, \frac{3}{2} \right)$

\setcounter{subsubsection}{36}
\subsubsection{}

$(-3, -6, 1)$

\subsection{Dot Product}

\subsubsection{}

$\mathbf{a} \cdot \mathbf{b} = 12$

\setcounter{subsubsection}{10}
\subsubsection{}

$\left( \frac{\mathbf{a} \cdot \mathbf{b}}{\mathbf{b} \cdot \mathbf{b}} \right) \mathbf{b} = \frac{12}{30} \mathbf{b} = \langle -\frac{2}{5}, \frac{4}{5}, 2 \rangle$

\setcounter{subsubsection}{12}
\subsubsection{}

$\mathbf{a} \cdot \mathbf{b} = ||\mathbf{a}|| ||\mathbf{b}|| \cos \theta = 25 \sqrt{2}$

\setcounter{subsubsection}{16}
\subsubsection{}

\begin{align*}
  \mathbf{a} \cdot \mathbf{v} & = 0                                            \\
  3 x_1 + y_1 - 1             & = 0                                            \\ \\
  \mathbf{b} \cdot \mathbf{v} & = 0                                            \\
  -3 x_1 + 2 y_2 + 2          & = 0                                            \\ \\
  3 y_2 + 1                   & = 0                                            \\
  y_2                         & = -\frac{1}{3}                                 \\ \\
  3 x_1 - \frac{1}{3} - 1     & = 0                                            \\
  x_1                         & = \frac{4}{9}                                  \\ \\
  \mathbf{v}                  & = \langle \frac{4}{9}, -\frac{1}{3}, 1 \rangle
\end{align*}

\setcounter{subsubsection}{18}
\subsubsection{}

\begin{align*}
  \mathbf{a} \cdot \mathbf{c} & = \mathbf{a} \cdot \left( \mathbf{b} - \frac{\mathbf{a} \cdot \mathbf{b}}{||\mathbf{a}||^2} \mathbf{a} \right)   \\
                              & = \mathbf{a} \cdot \mathbf{b} - \frac{\mathbf{a} \cdot \mathbf{b}}{||\mathbf{a}||^2} \mathbf{a} \cdot \mathbf{a} \\
                              & = 0
\end{align*}

\setcounter{subsubsection}{20}
\subsubsection{}

\begin{align*}
  ||\mathbf{a}||              & = \sqrt{3^2 + 1^2}                                                          \\
                              & = \sqrt{10}                                                                 \\
  ||\mathbf{b}||              & = \sqrt{2^2 + 2^2}                                                          \\
                              & = \sqrt{8}                                                                  \\
                              & = 2 \sqrt{2}                                                                \\
  \mathbf{a} \cdot \mathbf{b} & = 4                                                                         \\
  \theta                      & = \arccos \frac{\mathbf{a} \cdot \mathbf{b}}{||\mathbf{a}|| ||\mathbf{b}||} \\
                              & = \arccos \frac{4}{(\sqrt{10}) (2 \sqrt{2})}                                \\
                              & = \arccos \frac{1}{\sqrt{5}}                                                \\
                              & \approx \ang{63}
\end{align*}

\setcounter{subsubsection}{24}
\subsubsection{}

\begin{align*}
  ||\mathbf{a}|| & = \sqrt{1^2 + 2^2 + 3^3} \\
                 & = \sqrt{14}              \\
  \cos \alpha    & = \frac{1}{\sqrt{14}}    \\
  \alpha         & \approx \ang{75}         \\
  \cos \beta     & = \frac{2}{\sqrt{14}}    \\
  \beta          & \approx \ang{58}         \\
  \cos \gamma    & = \frac{3}{\sqrt{14}}    \\
  \gamma         & \approx \ang{37}
\end{align*}

\setcounter{subsubsection}{28}
\subsubsection{}

\begin{align*}
  \overrightarrow{A D}     & = \langle s, -s, s \rangle                                                                                          \\
  ||\overrightarrow{A D}|| & = s \sqrt{3}                                                                                                        \\
  \overrightarrow{A B}     & = \langle s, 0, 0 \rangle                                                                                           \\
  ||\overrightarrow{A B}|| & = s                                                                                                                 \\
  \theta                   & = \arccos \frac{\overrightarrow{A D} \cdot \overrightarrow{A B}}{||\overrightarrow{A D}|| ||\overrightarrow{A B}||} \\
                           & = \arccos \frac{s^2}{s^2 \sqrt{3}}                                                                                  \\
                           & = \arccos \frac{1}{\sqrt{3}}                                                                                        \\
                           & \approx \ang{55}
\end{align*}

\setcounter{subsubsection}{32}
\subsubsection{}

\begin{align*}
  \text{comp}_\mathbf{b} \mathbf{a} & = \frac{\mathbf{a} \cdot \mathbf{b}}{||\mathbf{b}||} \\
                                    & = \frac{5}{7}
\end{align*}

\setcounter{subsubsection}{36}
\subsubsection{}

\begin{align*}
  \text{comp}_{\overrightarrow{O P}} \mathbf{a} & = \frac{\mathbf{a} \cdot \overrightarrow{O P}}{||\overrightarrow{O P}||} \\
                                                & = \frac{72}{\sqrt{109}}
\end{align*}

\setcounter{subsubsection}{38}
\subsubsection{}

\begin{align*}
  \text{proj}_\mathbf{b} \mathbf{a} & = \left( \frac{\mathbf{a} \cdot \mathbf{b}}{\mathbf{b} \cdot \mathbf{b}} \right) \mathbf{b} \\
                                    & = \frac{35}{25} \mathbf{b}                                                                  \\
                                    & = \langle -\frac{21}{5}, \frac{28}{5} \rangle
\end{align*}

\setcounter{subsubsection}{42}
\subsubsection{}

\begin{align*}
  \mathbf{a} + \mathbf{b}                          & = \langle 3, 4 \rangle                                                                                                                                  \\
  \text{proj}_{\mathbf{a} + \mathbf{b}} \mathbf{a} & = \left( \frac{\mathbf{a} \cdot (\mathbf{a} + \mathbf{b})}{(\mathbf{a} + \mathbf{b}) \cdot (\mathbf{a} + \mathbf{b})} \right) (\mathbf{a} + \mathbf{b}) \\
                                                   & = \frac{24}{25} (\mathbf{a} + \mathbf{b})                                                                                                               \\
                                                   & = \langle \frac{72}{25}, \frac{96}{25} \rangle
\end{align*}

\setcounter{subsubsection}{44}
\subsubsection{}

$W = \mathbf{F} \cdot \mathbf{d} = F d \cos \theta = 1000$

\setcounter{subsubsection}{46}
\subsubsection{}

\begin{enumerate}
  \item $W = 0$

  \item

        \begin{align*}
          ||\mathbf{d}|| & = \sqrt{4^2 + 3^2}                           \\
                         & = 5                                          \\
          \mathbf{F}     & = F \hat{\mathbf{d}}                         \\
                         & = F \frac{\mathbf{d}}{||\mathbf{d}||}        \\
                         & = F \langle \frac{4}{5}, \frac{3}{5} \rangle \\
                         & = \langle 24, 18 \rangle                     \\
          W              & = \mathbf{F} \cdot \mathbf{d}                \\
                         & = \qty{150}{J}
        \end{align*}
\end{enumerate}

\subsection{Cross Product}

\subsubsection{}

\begin{align*}
  \mathbf{a} \times \mathbf{b} & = \begin{vmatrix}
                                     \mathbf{i} & \mathbf{j} & \mathbf{k} \\
                                     1          & -1         & 0          \\
                                     0          & 3          & 5
                                   \end{vmatrix}        \\
                               & = -5 \mathbf{i} - 5 \mathbf{j} + 3 \mathbf{k}
\end{align*}

\setcounter{subsubsection}{10}
\subsubsection{}

\begin{align*}
  \overrightarrow{P_1 P_2} \times \overrightarrow{P_1 P_3} & = \begin{vmatrix}
                                                                 \mathbf{i} & \mathbf{j} & \mathbf{k} \\
                                                                 -2         & 2          & -4         \\
                                                                 -3         & 1          & 1
                                                               \end{vmatrix}        \\
                                                           & = 6 \mathbf{i} + 14 \mathbf{j} + 4 \mathbf{k}
\end{align*}

\setcounter{subsubsection}{16}
\subsubsection{}

\begin{enumerate}
  \item

        \begin{align*}
          \mathbf{b} \times \mathbf{c}                     & = \begin{vmatrix}
                                                                 \mathbf{i} & \mathbf{j} & \mathbf{k} \\
                                                                 2          & 1          & 1          \\
                                                                 3          & 1          & 1
                                                               \end{vmatrix}  \\
                                                           & = \mathbf{j} - \mathbf{k}               \\
          \mathbf{a} \times (\mathbf{b} \times \mathbf{c}) & = \begin{vmatrix}
                                                                 \mathbf{i} & \mathbf{j} & \mathbf{k} \\
                                                                 1          & -1         & 2          \\
                                                                 0          & 1          & -1
                                                               \end{vmatrix}  \\
                                                           & = -\mathbf{i} + \mathbf{j} + \mathbf{k}
        \end{align*}
\end{enumerate}

\setcounter{subsubsection}{18}
\subsubsection{}

$2 \mathbf{k}$

\setcounter{subsubsection}{20}
\subsubsection{}

\begin{align*}
  \mathbf{k} \times (2 \mathbf{i} - \mathbf{j}) & = (\mathbf{k} \times 2 \mathbf{i}) - (\mathbf{k} \times \mathbf{j}) \\
                                                & = \mathbf{i} + 2 \mathbf{j}
\end{align*}

\setcounter{subsubsection}{22}
\subsubsection{}

\begin{align*}
  [(2 \mathbf{k}) \times (3 \mathbf{j})] \times (4 \mathbf{j}) & = (-6 \mathbf{i}) \times (4 \mathbf{j}) \\
                                                               & = -24 \mathbf{k}
\end{align*}

\setcounter{subsubsection}{36}
\subsubsection{}

$12 \mathbf{i} - 9 \mathbf{j} + 18 \mathbf{k}$

\setcounter{subsubsection}{52}
\subsubsection{}

\begin{align*}
  \mathbf{b} \times \mathbf{c}                    & = \begin{vmatrix}
                                                        \mathbf{i}  & \mathbf{j} & \mathbf{k}  \\
                                                        -2          & 6          & -6          \\
                                                        \frac{5}{2} & 3          & \frac{1}{2}
                                                      \end{vmatrix}        \\
                                                  & = 21 \mathbf{i} - 14 \mathbf{j} - 21 \mathbf{k} \\
  \mathbf{a} \cdot (\mathbf{b} \times \mathbf{c}) & = 4 \times 21 + 6 \times (-14)                  \\
                                                  & = 0
\end{align*}

They are coplanar.

\subsection{Lines and Planes in 3-Space}

\subsubsection{}

$\mathbf{r} = \langle 1, 2, 1 \rangle + t \langle 2, 3, -3 \rangle$

\setcounter{subsubsection}{6}
\subsubsection{}

\begin{align*}
  x & = 2 + 4 t \\
  y & = 3 - 4 t \\
  z & = 5 + 3 t
\end{align*}

\setcounter{subsubsection}{12}
\subsubsection{}

\begin{align*}
  x & = 1 + 9 t  \\
  y & = 4 + 10 t \\
  z & = -9 + 7 t \\
  \frac{x - 1}{9} = \frac{y - 4}{10} = \frac{z + 9}{7}
\end{align*}

\setcounter{subsubsection}{18}
\subsubsection{}

\begin{align*}
  x & = 4 + 3 t            \\
  y & = 6 + \frac{1}{2} t  \\
  z & = -7 - \frac{3}{2} t \\
  \frac{x - 4}{3} = \frac{y - 6}{1 / 2} = -\frac{z + 7}{3 / 2}
\end{align*}

\setcounter{subsubsection}{22}
\subsubsection{}

\begin{align*}
  x & = 6 + 2 t  \\
  y & = 4 - 3 t  \\
  z & = -2 + 6 t
\end{align*}

\setcounter{subsubsection}{24}
\subsubsection{}

\begin{align*}
  x & = 2 + t \\
  y & = -2    \\
  z & = 15
\end{align*}

\setcounter{subsubsection}{28}
\subsubsection{}

$(0, 5, 15), \,(5, 0, \frac{15}{2}), \,(10, -5, 0)$

\setcounter{subsubsection}{30}
\subsubsection{}

\begin{align*}
  4 + t_x    & = 6 + 2 t_x  \\
  t_x        & = -2         \\ \\
  5 + t_y    & = 11 + 4 t_y \\
  t_y        & = -2         \\ \\
  -1 + 2 t_z & = -3 + t_z   \\
  t_z        & = -2
\end{align*}

$(2, 3, -5)$

\setcounter{subsubsection}{34}
\subsubsection{}

\begin{align*}
  \mathbf{a}     & = \langle -1, 2, -2 \rangle                                                 \\
  ||\mathbf{a}|| & = 3                                                                         \\
  \mathbf{b}     & = \langle 2, 3, -6 \rangle                                                  \\
  ||\mathbf{b}|| & = 7                                                                         \\
  \theta         & = \arccos \frac{\mathbf{a} \cdot \mathbf{b}}{||\mathbf{a}|| ||\mathbf{b}||} \\
                 & \approx \ang{40.37}
\end{align*}

\setcounter{subsubsection}{36}
\subsubsection{}

\begin{align*}
  \mathbf{a}                   & = \langle 1, 1, 1 \rangle                                                                       \\
  \mathbf{b}                   & = \langle -2, 1, -5 \rangle                                                                     \\
  \mathbf{a} \times \mathbf{b} & = \begin{vmatrix}
                                     \hat{\mathbf{i}} & \hat{\mathbf{j}} & \hat{\mathbf{k}} \\
                                     1                & 1                & 1                \\
                                     -2               & 1                & -5
                                   \end{vmatrix} \\
                               & = \langle -6, 3, 3 \rangle                                                                      \\
  x                            & = 4 - 6 t                                                                                       \\
  y                            & = 1 + 3 t                                                                                       \\
  z                            & = 6 + 3 t
\end{align*}

\setcounter{subsubsection}{38}
\subsubsection{}

\begin{align*}
  \langle 2, -3, 4 \rangle \cdot (\mathbf{r} - \langle 5, 1, 3 \rangle) & = 0 \\
  2 (x - 5) - 3 (y - 1) + 4 (z - 3)                                     & = 0 \\
  2 x - 3 y + 4 z - 19                                                  & = 0
\end{align*}

\setcounter{subsubsection}{44}
\subsubsection{}

\begin{align*}
  \mathbf{a}                                                                             & = \langle 3, 5, 2 \rangle              \\
  \mathbf{b}                                                                             & = \langle 2, 3, 1 \rangle              \\
  \mathbf{c}                                                                             & = \langle -1, -1, 4 \rangle            \\
  \mathbf{a} - \mathbf{c}                                                                & = \langle 4, 6, -2 \rangle             \\
  \mathbf{b} - \mathbf{c}                                                                & = \langle 3, 4, -3 \rangle             \\
  (\mathbf{a} - \mathbf{c}) \times (\mathbf{b} - \mathbf{c})                             & = \begin{vmatrix}
                                                                                               \mathbf{i} & \mathbf{j} & \mathbf{k} \\
                                                                                               4          & 6          & -2         \\
                                                                                               3          & 4          & -3
                                                                                             \end{vmatrix} \\
                                                                                         & = \langle -10, 6, -2 \rangle           \\
  \mathbf{n} \cdot (\mathbf{r} - \mathbf{c})                                             & = 0                                    \\
  \langle -10, 6, -2 \rangle \cdot (\langle x, y, z \rangle - \langle -1, -1, 4 \rangle) & = 0                                    \\
  -10 (x + 1) + 6 (y + 1) - 2 (z - 4)                                                    & = 0                                    \\
  -10 x + 6 y - 2 z + 4                                                                  & = 0
\end{align*}

\setcounter{subsubsection}{50}
\subsubsection{}

\begin{align*}
  \langle 1, 1, -4 \rangle \cdot (\mathbf{r} - \langle 2, 3, -5 \rangle) & = 0  \\
  (x - 2) + (y - 3) - 4 (z + 5)                                          & = 0  \\
  x + y - 4 z                                                            & = 25
\end{align*}

\setcounter{subsubsection}{62}
\subsubsection{}

\begin{enumerate}
  \item Not perpendicular

  \item Not perpendicular

  \item Perpendicular

  \item Perpendicular
\end{enumerate}

\setcounter{subsubsection}{64}
\subsubsection{}

\begin{align*}
  5 x - 4 y - 9 t & = 8               \\
  x + 4 y + 3 t   & = 4               \\ \\
  6 x - 6 t       & = 12              \\
  x               & = 2 + t           \\ \\
  y               & = \frac{1}{2} - t \\ \\
  z               & = t
\end{align*}

\setcounter{subsubsection}{68}
\subsubsection{}

\begin{align*}
  2 (1 + 2 t) - 3 (2 - t) + 2 (-3 t) & = -7 \\
  t                                  & = -3 \\
  x                                  & = -5 \\
  y                                  & = 5  \\
  z                                  & = 9
\end{align*}

\setcounter{subsubsection}{72}
\subsubsection{}

\begin{align*}
  x + y - 4 t       & = 2        \\
  2 x - y + t       & = 10       \\ \\
  3 x - 3 t         & = 12       \\
  x                 & = 4 + t    \\ \\
  2 (4 + t) - y + t & = 10       \\
  8 + 2 t - y + t   & = 10       \\
  y                 & = -2 + 3 t \\ \\
  z                 & = t        \\ \\
  x                 & = 5 + t    \\
  y                 & = 6 + 3 t  \\
  z                 & = -12 + t
\end{align*}

\setcounter{subsubsection}{74}
\subsubsection{}

\begin{align*}
  \mathbf{n}                                              & = \begin{vmatrix}
                                                                \mathbf{i} & \mathbf{j} & \mathbf{k} \\
                                                                3          & -1         & 5          \\
                                                                1          & 1          & 1
                                                              \end{vmatrix} \\
                                                          & = \langle -6, 2, 4 \rangle             \\
  \mathbf{n} \cdot (\mathbf{r} - \langle 4, 0, 1 \rangle) & = 0                                    \\
  -6 (x - 4) + 2 y + 4 (z - 1)                            & = 0                                    \\
  -6 x + 2 y + 4 z                                        & = -20                                  \\
  3 x - y - 2 z                                           & = 10
\end{align*}

\subsection{Vector Spaces}

\subsubsection{}

Violates axiom 6

\setcounter{subsubsection}{2}
\subsubsection{}

Violates axiom 10

\setcounter{subsubsection}{4}
\subsubsection{}

Vector space

\setcounter{subsubsection}{6}
\subsubsection{}

Violates axiom 2

\setcounter{subsubsection}{8}
\subsubsection{}

Vector space

\setcounter{subsubsection}{10}
\subsubsection{}

Subspace

\setcounter{subsubsection}{12}
\subsubsection{}

Not a subspace

\setcounter{subsubsection}{14}
\subsubsection{}

Subspace

\setcounter{subsubsection}{16}
\subsubsection{}

Subspace

\setcounter{subsubsection}{18}
\subsubsection{}

Not a subspace

\setcounter{subsubsection}{22}
\subsubsection{}

\begin{enumerate}
  \item

        \begin{align*}
          k_1 \mathbf{u}_1 + k_2 \mathbf{u}_2 + k_3 \mathbf{u}_3                                  & = \mathbf{0} \\
          k_1 \langle 1, 0, 0 \rangle + k_2 \langle 1, 1, 0 \rangle + k_3 \langle 1, 1, 1 \rangle & = \mathbf{0} \\ \\
          k_3                                                                                     & = 0          \\ \\
          k_2 + k_3                                                                               & = 0          \\
          k_2                                                                                     & = 0          \\ \\
          k_1 + k_2 + k_3                                                                         & = 0          \\
          k_1                                                                                     & = 0
        \end{align*}

  \item \[\mathbf{a} = 7 \mathbf{u}_1 - 12 \mathbf{u}_2 + 8 \mathbf{u}_3\]
\end{enumerate}

\setcounter{subsubsection}{24}
\subsubsection{}

Dependent

\setcounter{subsubsection}{26}
\subsubsection{}

Independent

\setcounter{subsubsection}{28}
\subsubsection{}

$f(x)$ is undefined at $x = -3$ and $x = -1$.

\setcounter{subsubsection}{30}
\subsubsection{}

\begin{align*}
  ||x||      & = \sqrt{(x, x)}                                                      \\
             & = \sqrt{\int_0^{2 \pi} x^2 \,dx}                                     \\
             & = \sqrt{\left[ \frac{1}{3} x^3 \right]_0^{2 \pi}}                    \\
             & = \sqrt{\frac{8}{3} \pi^3}                                           \\
  ||\sin x|| & = \sqrt{(\sin x, \sin x)}                                            \\
             & = \sqrt{\int_0^{2 \pi} \sin^2 x \,dx}                                \\
             & = \sqrt{\left[ \frac{x}{2} - \frac{1}{4} \sin 2 x \right]_0^{2 \pi}} \\
             & = \sqrt{\pi}
\end{align*}

\subsection{Gram–Schmidt Orthogonalization Process}

\subsubsection{}

\begin{align*}
  \langle \frac{12}{13}, \frac{5}{13} \rangle \cdot \langle \frac{5}{13}, -\frac{12}{13} \rangle & = 0                                                                                                                                                   \\
  \sqrt{\left( \frac{12}{13} \right)^2 + \left( \frac{5}{13} \right)^2}                          & = 1                                                                                                                                                   \\
  \mathbf{u}                                                                                     & = \left( \langle 4, 2 \rangle \cdot \langle \frac{12}{13}, \frac{5}{13} \rangle \right) \langle \frac{12}{13}, \frac{5}{13} \rangle                   \\
                                                                                                 & \qquad + \left( \langle 4, 2 \rangle \cdot \langle \frac{5}{13}, -\frac{12}{13} \rangle \right) \langle \frac{5}{13}, -\frac{12}{13} \rangle          \\
                                                                                                 & = \left( \frac{58}{13} \right) \langle \frac{12}{13}, \frac{5}{13} \rangle - \left( \frac{4}{13} \right) \langle \frac{5}{13}, -\frac{12}{13} \rangle
\end{align*}

\setcounter{subsubsection}{2}
\subsubsection{}

\begin{align*}
  \langle 1, 0, 1 \rangle \cdot \langle 0, 1, 0 \rangle  & = 0                                                                                                                                                                                          \\
  \langle 1, 0, 1 \rangle \cdot \langle -1, 0, 1 \rangle & = 0                                                                                                                                                                                          \\
  \langle 0, 1, 0 \rangle \cdot \langle -1, 0, 1 \rangle & = 0                                                                                                                                                                                          \\
  B'                                                     & = \{\langle \frac{1}{\sqrt{2}}, 0, \frac{1}{\sqrt{2}} \rangle, \langle 0, 1, 0 \rangle, \langle -\frac{1}{\sqrt{2}}, 0, \frac{1}{\sqrt{2}} \rangle\}                                         \\
  \mathbf{u}                                             & = -\frac{3}{\sqrt{2}} \langle \frac{1}{\sqrt{2}}, 0, \frac{1}{\sqrt{2}} \rangle + 7 \langle 0, 1, 0 \rangle - \frac{23}{\sqrt{2}} \langle -\frac{1}{\sqrt{2}}, 0, \frac{1}{\sqrt{2}} \rangle
\end{align*}

\setcounter{subsubsection}{4}
\subsubsection{}

\begin{enumerate}
  \item

        \begin{align*}
          B            & = \{ \langle -3, 2 \rangle, \langle -1, -1 \rangle \}                                                                                                                        \\
          \mathbf{v}_1 & = \mathbf{u}_1                                                                                                                                                               \\
                       & = \langle -3, 2 \rangle                                                                                                                                                      \\
          \mathbf{v}_2 & = \mathbf{u}_2 - \text{proj}_{\mathbf{v}_1} \mathbf{u}_2                                                                                                                     \\
                       & = \langle -1, -1 \rangle - \left( \frac{\langle -1, -1 \rangle \cdot \langle -3, 2 \rangle}{\langle -3, 2 \rangle \cdot \langle -3, 2 \rangle} \right) \langle -3, 2 \rangle \\
                       & = \langle -1, -1 \rangle - \frac{1}{13} \langle -3, 2 \rangle                                                                                                                \\
                       & = \langle -\frac{10}{13}, -\frac{15}{13} \rangle                                                                                                                             \\
          \mathbf{w}_1 & = \langle -\frac{3}{\sqrt{13}}, \frac{2}{\sqrt{13}} \rangle                                                                                                                  \\
          \mathbf{w}_2 & = \sqrt{\frac{169}{325}} \langle -\frac{10}{13}, -\frac{15}{13} \rangle                                                                                                      \\
                       & = \frac{\sqrt{13}}{5} \langle -\frac{10}{13}, -\frac{15}{13} \rangle                                                                                                         \\
                       & = \langle -\frac{2}{\sqrt{13}}, -\frac{3}{\sqrt{13}} \rangle
        \end{align*}
\end{enumerate}

\setcounter{subsubsection}{8}
\subsubsection{}

\begin{align*}
  B            & = \{ \langle 1, 1, 0 \rangle, \langle 1, 2, 2 \rangle, \langle 2, 2, 1 \rangle \}                                                                                                                                                                       \\
  \mathbf{v}_1 & = \langle 1, 1, 0 \rangle                                                                                                                                                                                                                               \\
  \mathbf{v}_2 & = \mathbf{u}_2 - \text{proj}_{\mathbf{v}_1} \mathbf{u}_2                                                                                                                                                                                                \\
               & = \langle 1, 2, 2 \rangle - \left( \frac{\langle 1, 2, 2 \rangle \cdot \langle 1, 1, 0 \rangle}{\langle 1, 1, 0 \rangle \cdot \langle 1, 1, 0 \rangle} \right) \langle 1, 1, 0 \rangle                                                                  \\
               & = \langle 1, 2, 2 \rangle - \frac{3}{2} \langle 1, 1, 0 \rangle                                                                                                                                                                                         \\
               & = \langle -\frac{1}{2}, \frac{1}{2}, 2 \rangle                                                                                                                                                                                                          \\
  \mathbf{v}_3 & = \mathbf{u}_3 - \text{proj}_{\mathbf{v}_1} \mathbf{u}_3 - \text{proj}_{\mathbf{v}_2} \mathbf{u}_3                                                                                                                                                      \\
               & = \langle 2, 2, 1 \rangle - \left( \frac{\langle 2, 2, 1 \rangle \cdot \langle 1, 1, 0 \rangle}{\langle 1, 1, 0 \rangle \cdot \langle 1, 1, 0 \rangle} \right) \langle 1, 1, 0 \rangle                                                                  \\
               & \qquad - \left( \frac{\langle 2, 2, 1 \rangle \cdot \langle -\frac{1}{2}, \frac{1}{2}, 2 \rangle}{\langle -\frac{1}{2}, \frac{1}{2}, 2 \rangle \cdot \langle -\frac{1}{2}, \frac{1}{2}, 2 \rangle} \right) \langle -\frac{1}{2}, \frac{1}{2}, 2 \rangle \\
               & = \langle 2, 2, 1 \rangle - 2 \langle 1, 1, 0 \rangle - \frac{4}{9} \langle -\frac{1}{2}, \frac{1}{2}, 2 \rangle                                                                                                                                        \\
               & = \langle \frac{2}{9}, -\frac{2}{9}, \frac{1}{9} \rangle                                                                                                                                                                                                \\
  \mathbf{w}_1 & = \langle \frac{1}{\sqrt{2}}, \frac{1}{\sqrt{2}}, 0 \rangle                                                                                                                                                                                             \\
  \mathbf{w}_2 & = \langle -\frac{1}{3 \sqrt{2}}, \frac{1}{3 \sqrt{2}}, \frac{4}{3 \sqrt{2}} \rangle                                                                                                                                                                     \\
  \mathbf{w}_3 & = 3 \langle \frac{2}{9}, -\frac{2}{9}, \frac{1}{9} \rangle                                                                                                                                                                                              \\
               & = \langle \frac{2}{3}, -\frac{2}{3}, \frac{1}{3} \rangle
\end{align*}

\setcounter{subsubsection}{16}
\subsubsection{}

\begin{align*}
  B            & = \{1, x, x^2\}                                                                                                                                                                                                            \\
  \mathbf{v}_1 & = \mathbf{u}_1                                                                                                                                                                                                             \\
               & = 1                                                                                                                                                                                                                        \\
  \mathbf{v}_2 & = \mathbf{u}_2 - \text{proj}_{\mathbf{v}_1} \mathbf{u}_2                                                                                                                                                                   \\
               & = \mathbf{u}_2 - \left( \frac{\mathbf{u}_2 \cdot \mathbf{v}_1}{\mathbf{v}_1 \cdot \mathbf{v}_1} \right) \mathbf{v}_1                                                                                                       \\
               & = x - \frac{\int_{-1}^1 x \,dx}{\int_{-1}^1 dx}                                                                                                                                                                            \\
               & = x - \frac{\left[ \frac{1}{2} x^2 \right]_{-1}^1}{2}                                                                                                                                                                      \\
               & = x                                                                                                                                                                                                                        \\
  \mathbf{v}_3 & = \mathbf{u}_3 - \text{proj}_{\mathbf{v}_1} \mathbf{u}_3 - \text{proj}_{\mathbf{v}_2} \mathbf{u}_3                                                                                                                         \\
               & = \mathbf{u}_3 - \left( \frac{\mathbf{u}_3 \cdot \mathbf{v}_1}{\mathbf{v}_1 \cdot \mathbf{v}_1} \right) \mathbf{v}_1 - \left( \frac{\mathbf{u}_3 \cdot \mathbf{v}_2}{\mathbf{v}_2 \cdot \mathbf{v}_2} \right) \mathbf{v}_2 \\
               & = x^2 - \frac{\int_{-1}^1 x^2 \,dx}{\int_{-1}^1 dx} - \frac{\int_{-1}^1 x^3 \,dx}{\int_{-1}^1 x^2 \,dx} x                                                                                                                  \\
               & = x^2 - \frac{\left[ \frac{1}{3} x^3 \right]_{-1}^1}{2} - \frac{\left[ \frac{1}{4} x^4 \right]_{-1}^1}{\left[ \frac{1}{3} x^3 \right]_{-1}^1} x                                                                            \\
               & = x^2 - \frac{1}{3}
\end{align*}

\setcounter{subsubsection}{18}
\subsubsection{}

\begin{align*}
  ||\mathbf{v}_1||^2 & = \int_{-1}^1 \,dx                                                                  \\
                     & = 2                                                                                 \\
  \mathbf{w}_1       & = \frac{1}{\sqrt{2}}                                                                \\
  ||\mathbf{v}_2||^2 & = \int_{-1}^1 x^2 \,dx                                                              \\
                     & = \left[ \frac{1}{3} x^3 \right]_{-1}^1                                             \\
                     & = \frac{2}{3}                                                                       \\
  \mathbf{w}_2       & = \frac{3}{\sqrt{6}} x                                                              \\
  ||\mathbf{v}_3||^2 & = \int_{-1}^1 \left( x^2 - \frac{1}{3} \right)^2 \,dx                               \\
                     & = \int_{-1}^1 \left( x^4 - \frac{2}{3} x^2 + \frac{1}{9} \right) \,dx               \\
                     & = \left[ \frac{1}{5} x^5 - \frac{2}{9} x^3 + \frac{1}{9} x \right]_{-1}^1           \\
                     & = \frac{1}{5} - \frac{2}{9} + \frac{1}{9} + \frac{1}{5} - \frac{2}{9} + \frac{1}{9} \\
                     & = \frac{2}{5} - \frac{2}{9}                                                         \\
                     & = \frac{8}{45}                                                                      \\
  \mathbf{w}_3       & = \sqrt{\frac{45}{8}} \left( x^2 - \frac{1}{3} \right)                              \\
                     & = \frac{5}{2 \sqrt{10}} \left( 3 x^2 - 1 \right)
\end{align*}

\setcounter{subsubsection}{20}
\subsubsection{}

\begin{align*}
  (\mathbf{p}, \mathbf{w}_1) & = \int_{-1}^1 \frac{1}{\sqrt{2}} (9 x^2 - 6 x + 5) \,dx                                                                                                                                              \\
                             & = \frac{1}{\sqrt{2}} \left[ 3 x^3 - 3 x^2 + 5 x \right]_{-1}^1                                                                                                                                       \\
                             & = \frac{1}{\sqrt{2}} (3 - 3 + 5 + 3 + 3 + 5)                                                                                                                                                         \\
                             & = \frac{16}{\sqrt{2}}                                                                                                                                                                                \\
  (\mathbf{p}, \mathbf{w}_2) & = \int_{-1}^1 \frac{3}{\sqrt{6}} x (9 x^2 - 6 x + 5) \,dx                                                                                                                                            \\
                             & = \frac{3}{\sqrt{6}} \left[ \frac{9}{4} x^4 - 2 x^3 + \frac{5}{2} x^2 \right]_{-1}^1                                                                                                                 \\
                             & = \frac{3}{\sqrt{6}} \left( \frac{9}{4} - 2 + \frac{5}{2} - \frac{9}{4} - 2 - \frac{5}{2} \right)                                                                                                    \\
                             & = \frac{3}{\sqrt{6}} \left( \frac{9}{4} - \frac{8}{4} + \frac{10}{4} - \frac{9}{4} - \frac{8}{4} - \frac{10}{4} \right)                                                                              \\
                             & = -\frac{12}{\sqrt{6}}                                                                                                                                                                               \\
  (\mathbf{p}, \mathbf{w}_3) & = \int_{-1}^1 \frac{5}{2 \sqrt{10}} (3 x^2 - 1) (9 x^2 - 6 x + 5) \,dx                                                                                                                               \\
                             & = \frac{5}{2 \sqrt{10}} \int_{-1}^1 (27 x^4 - 18 x^3 + 6 x^2 + 6 x - 5) \,dx                                                                                                                         \\
                             & = \frac{5}{2 \sqrt{10}} \left[ \frac{27}{5} x^5 - \frac{9}{2} x^4 + 2 x^3 + 3 x^2 - 5 x \right]_{-1}^1                                                                                               \\
                             & = \frac{5}{2 \sqrt{10}} \left( \frac{27}{5} - \frac{9}{2} + 2 + 3 - 5 + \frac{27}{5} + \frac{9}{2} + 2 - 3 - 5 \right)                                                                               \\
                             & = \frac{5}{2 \sqrt{10}} \left( \frac{54}{10} - \frac{45}{10} + \frac{20}{10} + \frac{30}{10} - \frac{50}{10} + \frac{54}{10} + \frac{45}{10} + \frac{20}{10} - \frac{30}{10} - \frac{50}{10} \right) \\
                             & = \frac{5}{2 \sqrt{10}} \frac{48}{10}                                                                                                                                                                \\
                             & = \frac{12}{\sqrt{10}}                                                                                                                                                                               \\
  \mathbf{p}                 & = \frac{16}{\sqrt{2}} \mathbf{w}_1 - \frac{12}{\sqrt{6}} \mathbf{w}_2 + \frac{12}{\sqrt{10}} \mathbf{w}_3                                                                                            \\
\end{align*}

\subsection{Chapter in Review}

\subsubsection{}

True

\setcounter{subsubsection}{2}
\subsubsection{}

\begin{align*}
  \mathbf{u} & = \langle 5, -2, 1 \rangle \\
  \mathbf{v} & = \langle 2, 3, -4 \rangle
\end{align*}

False

\setcounter{subsubsection}{4}
\subsubsection{}

True

\setcounter{subsubsection}{6}
\subsubsection{}

True

\setcounter{subsubsection}{8}
\subsubsection{}

True

\setcounter{subsubsection}{10}
\subsubsection{}

$9 \mathbf{i} + 2 \mathbf{j} + 2 \mathbf{k}$

\setcounter{subsubsection}{12}
\subsubsection{}

\begin{align*}
  (-\mathbf{k}) \times (5 \mathbf{j}) & = \begin{vmatrix}
                                            \mathbf{i} & \mathbf{j} & \mathbf{k} \\
                                            0          & 0          & -1         \\
                                            0          & 5          & 0
                                          \end{vmatrix} \\
                                      & = 5 \mathbf{i}
\end{align*}

\setcounter{subsubsection}{14}
\subsubsection{}

\[||-12 \mathbf{i} + 4 \mathbf{j} + 6 \mathbf{k}|| = \sqrt{12^2 + 4^2 + 6^2} = 14\]

\setcounter{subsubsection}{16}
\subsubsection{}

$\langle -6, 1, -7 \rangle$

\setcounter{subsubsection}{18}
\subsubsection{}

\begin{align*}
  x                                   & = 1 + t    \\
  y                                   & = -2 + 3 t \\
  z                                   & = -1 + 2 t \\ \\
  x + 2 y - z                         & = 13       \\
  (1 + t) + 2 (-2 + 3 t) - (-1 + 2 t) & = 13       \\
  1 + t - 4 + 6 t + 1 - 2 t           & = 13       \\
  -2 + 5 t                            & = 13       \\
  t                                   & = 3        \\ \\
  x                                   & = 4        \\
  y                                   & = 7        \\
  z                                   & = 5
\end{align*}

\setcounter{subsubsection}{20}
\subsubsection{}

\begin{align*}
  \overrightarrow{P_1 P_2} & = \overrightarrow{P_2} - \overrightarrow{P_1}        \\
  \overrightarrow{P_2}     & = \overrightarrow{P_1 P_2} + \overrightarrow{P_1}    \\
                           & = \langle 3, 5, -4 \rangle + \langle 2, 1, 7 \rangle \\
                           & = \langle 5, 6, 3 \rangle
\end{align*}

\setcounter{subsubsection}{22}
\subsubsection{}

$\mathbf{a} \cdot \mathbf{b} = -36 \sqrt{2}$

\setcounter{subsubsection}{24}
\subsubsection{}

$x = 12$, $y = -8$, $z = 6$

\setcounter{subsubsection}{26}
\subsubsection{}

\begin{align*}
  \frac{1}{2} (\mathbf{a} \times \mathbf{b}) & = \frac{1}{2} \begin{vmatrix}
                                                               \mathbf{i} & \mathbf{j} & \mathbf{k} \\
                                                               1          & 3          & -1         \\
                                                               2          & -1         & 2
                                                             \end{vmatrix} \\
                                             & = \frac{1}{2} \langle 5, -4, -7 \rangle            \\
                                             & = \langle \frac{5}{2}, -2, -\frac{7}{2} \rangle    \\
\end{align*}

The area is $\sqrt{\left( \frac{5}{2} \right)^2 + (-2)^2 + \left( -\frac{7}{2} \right)^2} = \frac{3}{2} \sqrt{10}$

\setcounter{subsubsection}{28}
\subsubsection{}

$2$

\setcounter{subsubsection}{30}
\subsubsection{}

\begin{align*}
  \mathbf{a} \times \mathbf{b}               & = \begin{vmatrix}
                                                   \mathbf{i} & \mathbf{j} & \mathbf{k} \\
                                                   1          & 1          & 0          \\
                                                   1          & -2         & 1
                                                 \end{vmatrix}                                            \\
                                             & = \langle 1, -1, -3 \rangle                                                       \\
  ||\mathbf{a} \times \mathbf{b}||           & = \sqrt{11}                                                                       \\
  \text{norm} (\mathbf{a} \times \mathbf{b}) & = \langle \frac{1}{\sqrt{11}}, -\frac{1}{\sqrt{11}}, -\frac{3}{\sqrt{11}} \rangle
\end{align*}

\setcounter{subsubsection}{32}
\subsubsection{}

\[\text{comp}_\mathbf{b} \mathbf{a} = \frac{\mathbf{a} \cdot \mathbf{b}}{||\mathbf{b}||} = \frac{10}{5} = 2\]

\setcounter{subsubsection}{34}
\subsubsection{}

\begin{align*}
  \mathbf{a}                                       & = \langle 1, 2, -2 \rangle                                                                                 \\
  \mathbf{b}                                       & = \langle 4, 3, 0 \rangle                                                                                  \\
  \mathbf{a} + \mathbf{b}                          & = \langle 5, 5, -2 \rangle                                                                                 \\
  \text{proj}_\mathbf{a} (\mathbf{a} + \mathbf{b}) & = \left( \frac{(\mathbf{a} + \mathbf{b}) \cdot \mathbf{a}}{\mathbf{a} \cdot \mathbf{a}} \right) \mathbf{a} \\
                                                   & = \frac{19}{9} \langle 1, 2, -2 \rangle                                                                    \\
                                                   & = \langle \frac{19}{9}, \frac{38}{9}, -\frac{38}{9} \rangle
\end{align*}

\setcounter{subsubsection}{36}
\subsubsection{}

\begin{enumerate}
  \item

  \item A plane with normal $\mathbf{a}$
\end{enumerate}

\setcounter{subsubsection}{38}
\subsubsection{}

\[\frac{x - 7}{4} = \frac{y - 3}{-2} = \frac{z + 5}{6}\]

\setcounter{subsubsection}{40}
\subsubsection{}

\begin{align*}
  \langle -2, 3, 1 \rangle \cdot \langle 2, 1, 1 \rangle & = -4 + 3 + 1                 \\
                                                         & = 0                          \\ \\
  1 - 2 t                                                & = 1 + 2 s                    \\
  t                                                      & = -s                         \\ \\
  3 t                                                    & = -4 + s                     \\
  t                                                      & = -\frac{4}{3} + \frac{s}{3} \\
                                                         & = -\frac{4}{3} - \frac{t}{3} \\
  \frac{4}{3} t                                          & = -\frac{4}{3}               \\
  t                                                      & = -1                         \\ \\
  s                                                      & = 1                          \\ \\
  \langle 3, -3, 0 \rangle
\end{align*}

\setcounter{subsubsection}{42}
\subsubsection{}

\begin{align*}
  \mathbf{u}                                                                           & = \langle 1, 4, -2 \rangle             \\
  \mathbf{v}                                                                           & = \langle 1, 1, 3 \rangle              \\
  \mathbf{n}                                                                           & = \mathbf{u} \times \mathbf{v}         \\
                                                                                       & = \begin{vmatrix}
                                                                                             \mathbf{i} & \mathbf{j} & \mathbf{k} \\
                                                                                             1          & 4          & -2         \\
                                                                                             1          & 1          & 3
                                                                                           \end{vmatrix} \\
                                                                                       & = \langle 14, -5, -3 \rangle           \\ \\
  \mathbf{n} \cdot (\mathbf{r} - \mathbf{v})                                           & = 0                                    \\
  \langle 14, -5, -3 \rangle \cdot (\langle x, y, z \rangle - \langle 1, 1, 3 \rangle) & = 0                                    \\
  14 (x - 1) - 5 (y - 1) - 3 (z - 3)                                                   & = 0                                    \\
  14 x - 5 y - 3 z                                                                     & = 0
\end{align*}

\setcounter{subsubsection}{44}
\subsubsection{}

\begin{align*}
  \mathbf{F}                  & = \langle \frac{10}{\sqrt{2}}, \frac{10}{\sqrt{2}}, 0 \rangle \\
  \mathbf{d}                  & = \langle 3, 3, 0 \rangle                                     \\
  \mathbf{F} \cdot \mathbf{d} & = 30 \sqrt{2} \,\unit{J}
\end{align*}

\setcounter{subsubsection}{46}
\subsubsection{}

\begin{align*}
  \mathbf{F}_1     & = \langle 200, 0, 0 \rangle                                                                  \\
  \mathbf{F}_2     & = \langle \frac{200}{\sqrt{2}}, \frac{200}{\sqrt{2}}, 0 \rangle                              \\
  \mathbf{F}_2     & = \mathbf{F}_1 + \mathbf{F}_3                                                                \\
  \mathbf{F}_3     & = \mathbf{F}_2 - \mathbf{F}_1                                                                \\
                   & = \langle \frac{200}{\sqrt{2}}, \frac{200}{\sqrt{2}}, 0 \rangle - \langle 200, 0, 0 \rangle  \\
                   & = \langle \frac{200}{\sqrt{2}} - 200, \frac{200}{\sqrt{2}}, 0 \rangle                        \\
  ||\mathbf{F}_3|| & = \sqrt{\left( \frac{200}{\sqrt{2}} - 200 \right)^2 + \left( \frac{200}{\sqrt{2}} \right)^2} \\
                   & = \sqrt{\frac{40000}{2} - \frac{80000}{\sqrt{2}} + 40000 + \frac{40000}{2}}                  \\
                   & = 200 \sqrt{2 \left( 1 - \frac{1}{\sqrt{2}} \right)}                                         \\
                   & \approx \qty{153}{lb}
\end{align*}

\section{Matrices}

\subsection{Matrix Algebra}

\subsubsection{}

$2 \times 4$

\setcounter{subsubsection}{2}
\subsubsection{}

$3 \times 3$

\setcounter{subsubsection}{4}
\subsubsection{}

$3 \times 4$

\setcounter{subsubsection}{6}
\subsubsection{}

No

\setcounter{subsubsection}{8}
\subsubsection{}

No

\setcounter{subsubsection}{10}
\subsubsection{}

\begin{align*}
  x       & = y - 2 \\
  3 x - 2 & = y     \\ \\
  2 x - 2 & = 2     \\
  2 x     & = 4     \\
  x       & = 2     \\ \\
  2       & = y - 2 \\
  y       & = 4
\end{align*}

\setcounter{subsubsection}{12}
\subsubsection{}

\begin{align*}
  c_{23} & = 9  \\
  c_{12} & = 12
\end{align*}

\setcounter{subsubsection}{14}
\subsubsection{}

\begin{enumerate}
  \item $\begin{pmatrix}
            2 & 11 \\
            2 & -1
          \end{pmatrix}$

  \item $\begin{pmatrix}
            -6 & 1   \\
            14 & -19
          \end{pmatrix}$

  \item $\begin{pmatrix}
            2  & 28  \\
            12 & -12
          \end{pmatrix}$
\end{enumerate}

\setcounter{subsubsection}{16}
\subsubsection{}

\begin{enumerate}
  \item $\begin{pmatrix}
            -11 & 6   \\
            17  & -22
          \end{pmatrix}$

  \item $\begin{pmatrix}
            -32 & 27 \\
            -4  & -1
          \end{pmatrix}$

  \item $\begin{pmatrix}
            19  & -18 \\
            -30 & 31
          \end{pmatrix}$

  \item $\begin{pmatrix}
            19 & 6  \\
            3  & 22
          \end{pmatrix}$
\end{enumerate}

\setcounter{subsubsection}{20}
\subsubsection{}

\begin{enumerate}
  \item $180$

  \item $\begin{pmatrix}
            4  & 8  & 10 \\
            8  & 16 & 20 \\
            10 & 20 & 25
          \end{pmatrix}$

  \item $\begin{pmatrix}
            6  \\
            12 \\
            -5
          \end{pmatrix}$
\end{enumerate}

\setcounter{subsubsection}{22}
\subsubsection{}

\begin{enumerate}
  \item $\begin{pmatrix}
            7  & 38 \\
            10 & 75
          \end{pmatrix}$

  \item $\begin{pmatrix}
            7  & 38 \\
            10 & 75
          \end{pmatrix}$
\end{enumerate}

\setcounter{subsubsection}{24}
\subsubsection{}

$\begin{pmatrix}
    -14 \\
    1
  \end{pmatrix}$

\setcounter{subsubsection}{26}
\subsubsection{}

$\begin{pmatrix}
    -38 \\
    -2
  \end{pmatrix}$

\setcounter{subsubsection}{28}
\subsubsection{}

$4 \times 5$

\setcounter{subsubsection}{40}
\subsubsection{}

\begin{align*}
  a_{11} x_1 + a_{12} x_2 & = b_1 \\
  a_{21} x_1 + x_{22} x_2 & = b_2
\end{align*}

\setcounter{subsubsection}{42}
\subsubsection{}

\begin{align*}
  \begin{pmatrix}
    x & y
  \end{pmatrix} \begin{pmatrix}
                  a             & \frac{1}{2} b \\
                  \frac{1}{2} b & c
                \end{pmatrix} \begin{pmatrix}
                                x \\
                                y
                              \end{pmatrix} & = \begin{pmatrix}
                                                  x & y
                                                \end{pmatrix} \begin{pmatrix}
                                                                a x + \frac{1}{2} b y \\
                                                                \frac{1}{2} b x + c y
                                                              \end{pmatrix}          \\
                                   & = a x^2 + \frac{1}{2} b x y + \frac{1}{2} b x y + c y^2 \\
                                   & = a x^2 + b x y + c y^2
\end{align*}

\setcounter{subsubsection}{44}
\subsubsection{}

$\langle -1, 1 \rangle$

\setcounter{subsubsection}{46}
\subsubsection{}

$\langle -2, 0 \rangle$

\setcounter{subsubsection}{48}
\subsubsection{}

$\begin{pmatrix}
    1 & 0  \\
    0 & -1
  \end{pmatrix}$

\setcounter{subsubsection}{50}
\subsubsection{}

\begin{enumerate}
  \setcounter{enumi}{1}
  \item

        \begin{align*}
          \begin{pmatrix}
            x_S \\
            y_S \\
            z_S
          \end{pmatrix} & = \begin{pmatrix}
                              \cos \gamma  & \sin \gamma & 0 \\
                              -\sin \gamma & \cos \gamma & 0 \\
                              0            & 0           & 1
                            \end{pmatrix} \begin{pmatrix}
                                            \cos \beta & 0 & -\sin \beta \\
                                            0          & 1 & 0           \\
                                            \sin \beta & 0 & \cos \beta
                                          \end{pmatrix} \\
                          & \qquad \begin{pmatrix}
                                     1 & 0            & 0           \\
                                     0 & \cos \alpha  & \sin \alpha \\
                                     0 & -\sin \alpha & \cos \alpha
                                   \end{pmatrix} \begin{pmatrix}
                                                   x \\
                                                   y \\
                                                   z
                                                 \end{pmatrix}
        \end{align*}
\end{enumerate}

\subsection{Systems of Linear Algebraic Equations}

\subsubsection{}

\begin{align*}
  \begin{amatrix}{2}
    1 & -1 & 11 \\
    4 & 3 & -5
  \end{amatrix} \\
  \begin{amatrix}{2}
    1 & -1 & 11 \\
    0 & 7 & -49
  \end{amatrix} \\
  \begin{amatrix}{2}
    1 & -1 & 11 \\
    0 & 1 & -7
  \end{amatrix} \\
  \begin{amatrix}{2}
    1 & 0 & 4 \\
    0 & 1 & -7
  \end{amatrix} \\
\end{align*}

\setcounter{subsubsection}{4}
\subsubsection{}

\begin{align*}
  \begin{amatrix}{3}
    1 & -1 & -1 & -3 \\
    2 & 3 & 5 & 7 \\
    1 & -2 & 3 & -11
  \end{amatrix}                   \\
  \begin{amatrix}{3}
    1 & -1 & -1 & -3 \\
    0 & 5 & 7 & 13 \\
    0 & -1 & 4 & -8
  \end{amatrix}                   \\
  \begin{amatrix}{3}
    1 & -1 & -1 & -3 \\
    0 & 1 & \frac{7}{5} & \frac{13}{5} \\
    0 & -1 & 4 & -8
  \end{amatrix}  \\
  \begin{amatrix}{3}
    1 & -1 & -1 & -3 \\
    0 & 1 & \frac{7}{5} & \frac{13}{5} \\
    0 & 0 & \frac{27}{5} & -\frac{27}{5}
  \end{amatrix} \\
  \begin{amatrix}{3}
    1 & -1 & -1 & -3 \\
    0 & 1 & \frac{7}{5} & \frac{13}{5} \\
    0 & 0 & 1 & -1
  \end{amatrix}  \\
  \begin{amatrix}{3}
    1 & -1 & 0 & -4 \\
    0 & 1 & 0 & 4 \\
    0 & 0 & 1 & -1
  \end{amatrix}                   \\
  \begin{amatrix}{3}
    1 & 0 & 0 & 0 \\
    0 & 1 & 0 & 4 \\
    0 & 0 & 1 & -1
  \end{amatrix}                   \\
\end{align*}

\subsection{Rank of a Matrix}

\subsubsection{}

\begin{align*}
  \begin{pmatrix}
    3 & -1 \\
    1 & 3  \\
  \end{pmatrix}   \\
  \begin{pmatrix}
    1 & -\frac{1}{3} \\
    1 & 3            \\
  \end{pmatrix} \\
  \begin{pmatrix}
    1 & -\frac{1}{3} \\
    0 & \frac{10}{3} \\
  \end{pmatrix} \\
\end{align*}

Rank 2

\setcounter{subsubsection}{2}
\subsubsection{}

\begin{align*}
  \begin{pmatrix}
    2  & 1            & 3            \\
    6  & 3            & 9            \\
    -1 & -\frac{1}{2} & -\frac{3}{2}
  \end{pmatrix} \\
  \begin{pmatrix}
    1  & \frac{1}{2}  & \frac{3}{2}  \\
    6  & 3            & 9            \\
    -1 & -\frac{1}{2} & -\frac{3}{2}
  \end{pmatrix} \\
  \begin{pmatrix}
    1 & \frac{1}{2} & \frac{3}{2} \\
    0 & 0           & 0           \\
    0 & 0           & 0
  \end{pmatrix}    \\
\end{align*}

Rank 1

\setcounter{subsubsection}{4}
\subsubsection{}

\begin{align*}
  \begin{pmatrix}
    1 & 1 & 1 \\
    1 & 0 & 4 \\
    1 & 4 & 1
  \end{pmatrix} \\
  \begin{pmatrix}
    1 & 1  & 1 \\
    0 & -1 & 3 \\
    0 & 3  & 0
  \end{pmatrix} \\
  \begin{pmatrix}
    1 & 1 & 1  \\
    0 & 1 & -3 \\
    0 & 3 & 0
  \end{pmatrix} \\
  \begin{pmatrix}
    1 & 1 & 1  \\
    0 & 1 & -3 \\
    0 & 0 & 9
  \end{pmatrix} \\
  \begin{pmatrix}
    1 & 1 & 1  \\
    0 & 1 & -3 \\
    0 & 0 & 1
  \end{pmatrix} \\
\end{align*}

Rank 3

\setcounter{subsubsection}{6}
\subsubsection{}

\begin{align*}
  \begin{pmatrix}
    1 & -2 \\
    3 & -6 \\
    7 & -1 \\
    4 & 5
  \end{pmatrix} \\
  \begin{pmatrix}
    1 & -2 \\
    0 & 13 \\
    0 & 13 \\
    0 & 0
  \end{pmatrix} \\
  \begin{pmatrix}
    1 & -2 \\
    0 & 13 \\
    0 & 0  \\
    0 & 0
  \end{pmatrix} \\
\end{align*}

Rank 2

\setcounter{subsubsection}{10}
\subsubsection{}

\begin{align*}
  \begin{pmatrix}
    1 & 2  & 3 \\
    1 & 0  & 1 \\
    1 & -1 & 5
  \end{pmatrix} \\
  \begin{pmatrix}
    1 & 2  & 3  \\
    0 & -2 & -2 \\
    0 & -3 & 2
  \end{pmatrix} \\
  \begin{pmatrix}
    1 & 2  & 3 \\
    0 & 1  & 1 \\
    0 & -3 & 2
  \end{pmatrix} \\
  \begin{pmatrix}
    1 & 2 & 3 \\
    0 & 1 & 1 \\
    0 & 0 & 5
  \end{pmatrix} \\
\end{align*}

Linearly independent

\setcounter{subsubsection}{14}
\subsubsection{}

5

\setcounter{subsubsection}{16}
\subsubsection{}

$\rank(\mathbf{A}) = 2$

\subsection{Determinants}

\subsubsection{}

$9$

\setcounter{subsubsection}{2}
\subsubsection{}

$1$

\setcounter{subsubsection}{4}
\subsubsection{}

\begin{align*}
  M_{33} & = \begin{vmatrix}
               0 & 2 & 0 \\
               1 & 2 & 3 \\
               1 & 1 & 2
             \end{vmatrix}    \\
         & = -2 \begin{vmatrix}
                  1 & 3 \\
                  1 & 2
                \end{vmatrix} \\
         & = 2
\end{align*}

\setcounter{subsubsection}{6}
\subsubsection{}

\begin{align*}
  C_{34} & = (-1)^{3 + 4} \begin{vmatrix}
                            0 & 2 & 4  \\
                            1 & 2 & -2 \\
                            1 & 1 & 1
                          \end{vmatrix}                        \\
         & = -\left( -2 \begin{vmatrix}
                            1 & -2 \\
                            1 & 1
                          \end{vmatrix} + 4 \begin{vmatrix}
                                              1 & 2 \\
                                              1 & 1
                                            \end{vmatrix} \right) \\
         & = 10
\end{align*}

\setcounter{subsubsection}{8}
\subsubsection{}

$-7$

\setcounter{subsubsection}{10}
\subsubsection{}

$17$

\setcounter{subsubsection}{12}
\subsubsection{}

\begin{align*}
  (1 - \lambda)(2 - \lambda) - 6 & = 2 - \lambda - 2 \lambda + \lambda^2 - 6 \\
                                 & = \lambda^2 - 3 \lambda - 4               \\
                                 & = (\lambda + 1) (\lambda - 4)
\end{align*}

\setcounter{subsubsection}{14}
\subsubsection{}

$-48$

\setcounter{subsubsection}{22}
\subsubsection{}

\begin{align*}
  \begin{vmatrix}
    1 & 1 & 1 \\
    x & y & z \\
    2 & 3 & 4
  \end{vmatrix} & = \begin{vmatrix}
                      y & z \\
                      3 & 4
                    \end{vmatrix} - \begin{vmatrix}
                                      x & z \\
                                      2 & 4
                                    \end{vmatrix} + \begin{vmatrix}
                                                      x & y \\
                                                      2 & 3
                                                    \end{vmatrix} \\
                  & = 4 y - 3 z - 4 x + 2 z + 3 x - 2 y            \\
                  & = -x + 2 y - z
\end{align*}

\setcounter{subsubsection}{28}
\subsubsection{}

\begin{align*}
  \begin{vmatrix}
    (-3 - \lambda) & 10            \\
    2              & (5 - \lambda)
  \end{vmatrix}            & = 0                                    \\
  (-3 - \lambda) (5 - \lambda) - 20            & = 0                \\
  -15 + 3 \lambda - 5 \lambda + \lambda^2 - 20 & = 0                \\
  \lambda^2 - 2 \lambda - 35                   & = 0                \\
  (\lambda - 7) (\lambda + 5)                  & = 0                \\
  \lambda                                      & = -5 \text{ or } 7
\end{align*}

\subsection{Properties of Determinants}

\subsubsection{}

8.5.4

\setcounter{subsubsection}{2}
\subsubsection{}

8.5.7

\setcounter{subsubsection}{4}
\subsubsection{}

8.5.5

\setcounter{subsubsection}{6}
\subsubsection{}

8.5.3

\setcounter{subsubsection}{8}
\subsubsection{}

8.5.1

\setcounter{subsubsection}{10}
\subsubsection{}

$-5$

\setcounter{subsubsection}{12}
\subsubsection{}

$-5$

\setcounter{subsubsection}{14}
\subsubsection{}

$5$

\setcounter{subsubsection}{16}
\subsubsection{}

$80$

\setcounter{subsubsection}{18}
\subsubsection{}

$-105$

\setcounter{subsubsection}{24}
\subsubsection{}

\begin{align*}
  \mathbf{A} \mathbf{A}                 & = \mathbf{I}      \\
  \det \mathbf{A} \cdot \det \mathbf{A} & = \det \mathbf{I} \\
  (\det \mathbf{A})^2                   & = 1               \\
  \det \mathbf{A}                       & = \pm 1
\end{align*}

\setcounter{subsubsection}{26}
\subsubsection{}

\begin{align*}
  \begin{vmatrix}
    a & a + 1 & a + 2 \\
    b & b + 1 & b + 2 \\
    c & c + 1 & c + 2
  \end{vmatrix} & = \begin{vmatrix}
                      a & 1 & 2 \\
                      b & 1 & 2 \\
                      c & 1 & 2
                    \end{vmatrix}        \\
                       & = \begin{vmatrix}
                             a & 1 & 1 \\
                             b & 1 & 1 \\
                             c & 1 & 1
                           \end{vmatrix} \\
                       & = 0
\end{align*}

\setcounter{subsubsection}{28}
\subsubsection{}

\begin{align*}
  \begin{vmatrix}
    1 & 1  & 5 \\
    4 & 3  & 6 \\
    0 & -1 & 1
  \end{vmatrix} & = \begin{vmatrix}
                      1 & 1  & 5   \\
                      0 & -1 & -14 \\
                      0 & -1 & 1
                    \end{vmatrix}    \\
                  & = -\begin{vmatrix}
                         1 & 1 & 5  \\
                         0 & 1 & 14 \\
                         0 & 0 & 15
                       \end{vmatrix} \\
                  & = -15
\end{align*}

\setcounter{subsubsection}{36}
\subsubsection{}

\begin{align*}
  \begin{vmatrix}
    1   & 1   & 1   \\
    a   & b   & c   \\
    a^2 & b^2 & c^2
  \end{vmatrix} & = \begin{vmatrix}
                      1 & 1         & 1         \\
                      0 & b - a     & c - a     \\
                      0 & b^2 - a^2 & c^2 - a^2
                    \end{vmatrix}                        \\
                     & = \begin{vmatrix}
                           1 & 1     & 1                          \\
                           0 & b - a & c - a                      \\
                           0 & 0     & c^2 - a^2 - (b + a)(c - a)
                         \end{vmatrix}      \\
                     & = (b - a) (c^2 - a^2 - (b + a) (c - a))       \\
                     & = (b - a) (c^2 - a^2 - b c + a b - a c + a^2) \\
                     & = (b - a) (c^2 + a b - a c - b c)             \\
                     & = (b - a) (c - a) (c - b)
\end{align*}

\setcounter{subsubsection}{38}
\subsubsection{}

\begin{align*}
  a_{21} C_{11} + a_{22} C_{12} + a_{23} C_{13} & = (-1) (4) + (2) (5) + (1) (-6) \\
                                                & = 0                             \\
  a_{13} C_{12} + a_{23} C_{22} + a_{33} C_{32} & = (2) (5) + (1) (-7) + (1) (-3) \\
                                                & = 0
\end{align*}

\setcounter{subsubsection}{40}
\subsubsection{}

\begin{align*}
  \mathbf{A} + \mathbf{B}        & = \begin{pmatrix}
                                       10 & 0  \\
                                       0  & -3
                                     \end{pmatrix} \\
  \det (\mathbf{A} + \mathbf{B}) & = -30            \\
  \det \mathbf{A}                & = 10             \\
  \det \mathbf{B}                & = -31            \\
  -30                            & \ne 10 - 31
\end{align*}

\subsection{Inverse of a Matrix}

\setcounter{subsubsection}{2}
\subsubsection{}

\begin{align*}
  \det \mathbf{A} & = 9                          \\
  \mathbf{A}^{-1} & = \frac{1}{9} \begin{pmatrix}
                                    1  & 1 \\
                                    -4 & 5
                                  \end{pmatrix} \\
                  & = \begin{pmatrix}
                        \frac{1}{9}  & \frac{1}{9} \\
                        -\frac{4}{9} & \frac{5}{9}
                      \end{pmatrix}
\end{align*}

\setcounter{subsubsection}{4}
\subsubsection{}

\begin{align*}
  \det \mathbf{A} & = 12                          \\
  \mathbf{A}^{-1} & = \frac{1}{12} \begin{pmatrix}
                                     2 & 0 \\
                                     3 & 6
                                   \end{pmatrix} \\
                  & = \begin{pmatrix}
                        \frac{1}{6} & 0           \\
                        \frac{1}{4} & \frac{1}{2}
                      \end{pmatrix}
\end{align*}

\setcounter{subsubsection}{6}
\subsubsection{}

\begin{align*}
  \det \mathbf{A} & = (1) \begin{vmatrix}
                            4  & 4 \\
                            -1 & 1
                          \end{vmatrix} - (3) \begin{vmatrix}
                                                2 & 4 \\
                                                1 & 1
                                              \end{vmatrix} + (5) \begin{vmatrix}
                                                                    2 & 4  \\
                                                                    1 & -1
                                                                  \end{vmatrix} \\
                  & = 8 + 6 - 30                                                 \\
                  & = -16                                                        \\
  \mathbf{A}^{-1} & = -\frac{1}{16} \begin{pmatrix}
                                      8  & 2  & -6 \\
                                      -8 & -4 & 4  \\
                                      -8 & 6  & -2
                                    \end{pmatrix}^T                              \\
                  & = \begin{pmatrix}
                        -\frac{1}{2} & \frac{1}{2}  & \frac{1}{2}  \\
                        -\frac{1}{8} & \frac{1}{4}  & -\frac{3}{8} \\
                        \frac{3}{8}  & -\frac{1}{4} & \frac{1}{8}
                      \end{pmatrix}
\end{align*}

\setcounter{subsubsection}{14}
\subsubsection{}

\begin{align*}
  \left( \begin{array}{cc|cc}
           6 & -2 & 1 & 0 \\
           0 & 4  & 0 & 1
         \end{array} \right)                          \\
  \left( \begin{array}{cc|cc}
           1 & -\frac{1}{3} & \frac{1}{6} & 0           \\
           0 & 1            & 0           & \frac{1}{4}
         \end{array} \right) \\
  \left( \begin{array}{cc|cc}
           1 & 0 & \frac{1}{6} & \frac{1}{12} \\
           0 & 1 & 0           & \frac{1}{4}
         \end{array} \right)           \\
\end{align*}

\setcounter{subsubsection}{16}
\subsubsection{}

\begin{align*}
  \left( \begin{array}{cc|cc}
           1 & 3 & 1 & 0 \\
           5 & 3 & 0 & 1
         \end{array} \right)                  \\
  \left( \begin{array}{cc|cc}
           1 & 3   & 1  & 0 \\
           0 & -12 & -5 & 1
         \end{array} \right)                  \\
  \left( \begin{array}{cc|cc}
           1 & 3 & 1            & 0             \\
           0 & 1 & \frac{5}{12} & -\frac{1}{12}
         \end{array} \right) \\
  \left( \begin{array}{cc|cc}
           1 & 0 & -\frac{1}{4} & \frac{1}{4}   \\
           0 & 1 & \frac{5}{12} & -\frac{1}{12}
         \end{array} \right) \\
\end{align*}

\setcounter{subsubsection}{26}
\subsubsection{}

\begin{align*}
  (\mathbf{A} \mathbf{B})^{-1} & = \mathbf{B}^{-1} \mathbf{A}^{-1}           \\
                               & = \begin{pmatrix}
                                     \frac{2}{3}  & \frac{4}{3} \\
                                     -\frac{1}{3} & \frac{5}{2}
                                   \end{pmatrix} \begin{pmatrix}
                                                   \frac{1}{2}  & -\frac{5}{2} \\
                                                   -\frac{1}{2} & \frac{3}{2}
                                                 \end{pmatrix} \\
                               & = \begin{pmatrix}
                                     -\frac{1}{3}   & \frac{1}{3}   \\
                                     -\frac{17}{12} & \frac{55}{12}
                                   \end{pmatrix}
\end{align*}

\setcounter{subsubsection}{28}
\subsubsection{}

\[\begin{pmatrix}
    -2 & 3  \\
    3  & -4
  \end{pmatrix}\]

\setcounter{subsubsection}{30}
\subsubsection{}

\begin{align*}
  \begin{pmatrix}
    4 & -3 \\
    x & -4
  \end{pmatrix} & = \frac{1}{3 x - 16} \begin{pmatrix}
                                         -4 & -x \\
                                         3  & 4
                                       \end{pmatrix}^T  \\
                  & = \frac{1}{3 x - 16} \begin{pmatrix}
                                           -4 & 3 \\
                                           -x & 4
                                         \end{pmatrix} \\
  -1              & = \frac{1}{3 x - 16}                \\
  16 - 3 x        & = 1                                 \\
  3 x             & = 15                                \\
  x               & = 5
\end{align*}

\setcounter{subsubsection}{44}
\subsubsection{}

\begin{align*}
  \begin{pmatrix}
    1 & 1  \\
    2 & -1
  \end{pmatrix} \begin{pmatrix}
                  x_1 \\
                  x_2
                \end{pmatrix} & = \begin{pmatrix}
                                    4 \\
                                    14
                                  \end{pmatrix}                            \\
  \begin{pmatrix}
    x_1 \\
    x_2
  \end{pmatrix}               & = -\frac{1}{3} \begin{pmatrix}
                                                 -1 & -1 \\
                                                 -2 & 1
                                               \end{pmatrix} \begin{pmatrix}
                                                               4 \\
                                                               14
                                                             \end{pmatrix} \\
                                & = -\frac{1}{3} \begin{pmatrix}
                                                   -18 \\
                                                   6
                                                 \end{pmatrix}             \\
                                & = \begin{pmatrix}
                                      6 \\
                                      -2
                                    \end{pmatrix}
\end{align*}

\setcounter{subsubsection}{48}
\subsubsection{}

\begin{align*}
  \begin{pmatrix}
    1 & 0  & 1 \\
    1 & 1  & 1 \\
    5 & -1 & 0
  \end{pmatrix} \begin{pmatrix}
                  x_1 \\
                  x_2 \\
                  x_3
                \end{pmatrix} & = \begin{pmatrix}
                                    -4 \\
                                    0  \\
                                    6
                                  \end{pmatrix}                              \\
  \det \mathbf{A}               & = \begin{vmatrix}
                                      1  & 1 \\
                                      -1 & 0
                                    \end{vmatrix} + \begin{vmatrix}
                                                      1 & 1  \\
                                                      5 & -1
                                                    \end{vmatrix}            \\
                                & = 1 - 6                                     \\
                                & = -5                                        \\
  \begin{pmatrix}
    x_1 \\
    x_2 \\
    x_3
  \end{pmatrix}               & = -\frac{1}{5} \begin{pmatrix}
                                                 1  & 5  & -6 \\
                                                 -1 & -5 & 1  \\
                                                 -1 & 0  & 1
                                               \end{pmatrix}^T \begin{pmatrix}
                                                                 -4 \\
                                                                 0  \\
                                                                 6
                                                               \end{pmatrix} \\
                                & = -\frac{1}{5} \begin{pmatrix}
                                                   1  & -1 & -1 \\
                                                   5  & -5 & 0  \\
                                                   -6 & 1  & 1
                                                 \end{pmatrix} \begin{pmatrix}
                                                                 -4 \\
                                                                 0  \\
                                                                 6
                                                               \end{pmatrix} \\
                                & = -\frac{1}{5} \begin{pmatrix}
                                                   -10 \\
                                                   -20 \\
                                                   30
                                                 \end{pmatrix}               \\
                                & = \begin{pmatrix}
                                      2 \\
                                      4 \\
                                      -6
                                    \end{pmatrix}
\end{align*}

\setcounter{subsubsection}{54}
\subsubsection{}

\begin{align*}
  \det \begin{pmatrix}
         1 & 2  & -1 \\
         4 & -1 & 1  \\
         5 & 1  & -2
       \end{pmatrix} & = (1) \begin{vmatrix}
                               -1 & 1  \\
                               1  & -2
                             \end{vmatrix} - (2) \begin{vmatrix}
                                                   4 & 1  \\
                                                   5 & -2
                                                 \end{vmatrix} + (-1) \begin{vmatrix}
                                                                        4 & -1 \\
                                                                        5 & 1
                                                                      \end{vmatrix} \\
                       & = 1 + 26 - 9                                                \\
                       & = 18
\end{align*}

Only trivial solution

\subsection{Cramer’s Rule}

\subsubsection{}

\begin{align*}
  \mathbf{A}        & = \begin{pmatrix}
                          -3 & 1  \\
                          2  & -4
                        \end{pmatrix} \\
  \mathbf{B}        & = \begin{pmatrix}
                          3 \\
                          -6
                        \end{pmatrix} \\
  \det \mathbf{A}   & = 10             \\
  \det \mathbf{A}_1 & = -6             \\
  \det \mathbf{A}_2 & = 12             \\
  x_1               & = -\frac{3}{5}   \\
  x_2               & = \frac{6}{5}
\end{align*}

\setcounter{subsubsection}{10}
\subsubsection{}

\begin{align*}
  \mathbf{A}        & = \begin{pmatrix}
                          2 - k & k     \\
                          k     & 3 - k
                        \end{pmatrix}           \\
  \mathbf{B}        & = \begin{pmatrix}
                          4 \\
                          3
                        \end{pmatrix}           \\
  \det \mathbf{A}   & = (2 - k) (3 - k) - k^2    \\
                    & = 6 - 5 k                  \\
  \det \mathbf{A}_1 & = 4 (3 - k) - 3 k          \\
                    & = 12 - 7 k                 \\
  \det \mathbf{A}_2 & = 3 (2 - k) - 4 k          \\
                    & = 6 - 7 k                  \\
  x_1               & = \frac{12 - 7 k}{6 - 5 k} \\
  x_2               & = \frac{6 - 7 k}{6 - 5 k}
\end{align*}

The system is inconsistent when $k = \frac{6}{5}$

\setcounter{subsubsection}{12}
\subsubsection{}

\begin{align*}
  \mathbf{A}        & = \begin{pmatrix}
                          \cos \ang{25} & -\cos \ang{15} \\
                          \sin \ang{25} & \sin \ang{15}
                        \end{pmatrix}                            \\
  \mathbf{B}        & = \begin{pmatrix}
                          0 \\
                          300
                        \end{pmatrix}                                            \\
  \det \mathbf{A}   & = \cos \ang{25} \sin \ang{15} + \cos \ang{15} \sin \ang{25} \\
                    & = \sin \ang{40}                                             \\
  \det \mathbf{A}_1 & = 300 \cos \ang{15}                                         \\
  \det \mathbf{A}_2 & = 300 \cos \ang{25}                                         \\
  T_1               & = \frac{300 \cos \ang{15}}{\sin \ang{40}}                   \\
                    & \approx \qty{451}{lb}                                       \\
  T_2               & = \frac{300 \cos \ang{25}}{\sin \ang{40}}                   \\
                    & \approx \qty{423}{lb}
\end{align*}

\subsection{The Eigenvalue Problem}

\subsubsection{}

$\mathbf{K}_3$ with $\lambda = -1$

\setcounter{subsubsection}{2}
\subsubsection{}

$\mathbf{K}_3$ with $\lambda = 0$

\setcounter{subsubsection}{4}
\subsubsection{}

$\mathbf{K}_2$ with $\lambda = 3$

$\mathbf{K}_3$ with $\lambda = 1$

\setcounter{subsubsection}{6}
\subsubsection{}

\begin{align*}
  \det (\mathbf{A} - \lambda \mathbf{I}) & = \begin{vmatrix}
                                               -1 - \lambda & 2           \\
                                               -7           & 8 - \lambda
                                             \end{vmatrix}                \\
                                         & = (-1 - \lambda) (8 - \lambda) + 14         \\
                                         & = -8 + \lambda - 8 \lambda + \lambda^2 + 14 \\
                                         & = \lambda^2 - 7 \lambda + 6                 \\
                                         & = (\lambda - 1) (\lambda - 6)               \\
  \lambda_1                              & = 1                                         \\
  \lambda_2                              & = 6
\end{align*}

\[\left( \begin{array}{cc|c}
      -2 & 2 & 0 \\
      -7 & 7 & 0
    \end{array} \right)\]

$x_1 = x_2$

\[\mathbf{X}_1 = \begin{pmatrix}
    1 \\
    1
  \end{pmatrix}\]

\[\left( \begin{array}{cc|c}
      -7 & 2 & 0 \\
      -7 & 2 & 0
    \end{array} \right)\]

$x_1 = \frac{2}{7} x_2$

\[\mathbf{X}_2 = \begin{pmatrix}
    2 \\
    7
  \end{pmatrix}\]

Nonsingular

\setcounter{subsubsection}{8}
\subsubsection{}

\begin{align*}
  \det (\mathbf{A} - \lambda \mathbf{I}) & = \begin{vmatrix}
                                               -8 - \lambda & -1       \\
                                               16           & -\lambda
                                             \end{vmatrix}      \\
                                         & = -\lambda (-8 - \lambda) + 16 \\
                                         & = 8 \lambda + \lambda^2 + 16   \\
                                         & = (\lambda + 4)^2              \\
  \lambda_1 = \lambda_2                  & = -4
\end{align*}

\[\left( \begin{array}{cc|c}
      -4 & -1 & 0 \\
      16 & 4  & 0
    \end{array} \right)\]

$x_1 = -\frac{1}{4} x_2$

\[\mathbf{X}_1 = \begin{pmatrix}
    -1 \\
    4
  \end{pmatrix}\]

Nonsingular

\setcounter{subsubsection}{10}
\subsubsection{}

\begin{align*}
  \det (\mathbf{A} - \lambda \mathbf{I}) & = \begin{vmatrix}
                                               -1 - \lambda & 2           \\
                                               -5           & 1 - \lambda
                                             \end{vmatrix}              \\
                                         & = (-1 - \lambda) (1 - \lambda) + 10       \\
                                         & = -1 + \lambda - \lambda + \lambda^2 + 10 \\
                                         & = \lambda^2 + 9                           \\
                                         & = (\lambda - 3 i) (\lambda + 3 i)         \\
  \lambda_1                              & = 3 i                                     \\
  \lambda_2                              & = -3 i
\end{align*}

\[\left( \begin{array}{cc|c}
      -1 - 3 i & 2       & 0 \\
      -5       & 1 - 3 i & 0
    \end{array} \right)\]

\[\mathbf{X}_1 = \begin{pmatrix}
    1 - 3 i \\
    5
  \end{pmatrix}\]

\[\mathbf{X}_2 = \begin{pmatrix}
    1 + 3 i \\
    5       \\
  \end{pmatrix}\]

Nonsingular

\setcounter{subsubsection}{22}
\subsubsection{}

\begin{align*}
  \det (\mathbf{A} - \lambda \mathbf{I}) & = \begin{vmatrix}
                                               5 - \lambda & 1           \\
                                               1           & 5 - \lambda
                                             \end{vmatrix}       \\
                                         & = (5 - \lambda)^2 - 1             \\
                                         & = 25 - 10 \lambda + \lambda^2 - 1 \\
                                         & = \lambda^2 - 10 \lambda + 24     \\
                                         & = (\lambda - 4) (\lambda - 6)     \\
  \lambda_1                              & = 4                               \\
  \lambda_2                              & = 6
\end{align*}

\[\begin{pmatrix}
    1 & 1 \\
    1 & 1
  \end{pmatrix}\]

\[\mathbf{X}_1 = \begin{pmatrix}
    1 \\
    -1
  \end{pmatrix}\]

\[\begin{pmatrix}
    -1 & 1  \\
    1  & -1
  \end{pmatrix}\]

\[\mathbf{X_2} = \begin{pmatrix}
    1 \\
    1
  \end{pmatrix}\]

\begin{align*}
  \lambda_1'    & = \frac{1}{4}  \\
  \lambda_2'    & = \frac{1}{6}  \\
  \mathbf{X}_1' & = \mathbf{X}_1 \\
  \mathbf{X}_2' & = \mathbf{X}_2
\end{align*}

\subsection{Powers of Matrices}

\subsubsection{}

\begin{align*}
  \det (\mathbf{A} - \lambda \mathbf{I}) & = (1 - \lambda) (5 - \lambda) + 8         \\
                                         & = 5 - \lambda - 5 \lambda + \lambda^2 + 8 \\
                                         & = \lambda^2 - 6 \lambda + 13              \\
  \mathbf{A}^2                           & = 6 \mathbf{A} - 13 \mathbf{I}            \\
  \begin{pmatrix}
    1 & -2 \\
    4 & 5
  \end{pmatrix} \begin{pmatrix}
                  1 & -2 \\
                  4 & 5
                \end{pmatrix}          & = 6 \begin{pmatrix}
                                               1 & -2 \\
                                               4 & 5
                                             \end{pmatrix} - 13 \begin{pmatrix}
                                                                  1 & 0 \\
                                                                  0 & 1
                                                                \end{pmatrix}       \\
  \begin{pmatrix}
    -7 & -12 \\
    24 & 17
  \end{pmatrix}                        & = \begin{pmatrix}
                                             -7 & -12 \\
                                             24 & 17
                                           \end{pmatrix}
\end{align*}

\setcounter{subsubsection}{2}
\subsubsection{}

\begin{align*}
  \mathbf{A}                             & = \begin{pmatrix}
                                               -1 & 3 \\
                                               2  & 4
                                             \end{pmatrix}                                      \\
  \det (\mathbf{A} - \lambda \mathbf{I}) & = (-1 - \lambda) (4 - \lambda) - 6                    \\
                                         & = -4 + \lambda - 4 \lambda + \lambda^2 - 6            \\
                                         & = \lambda^2 - 3 \lambda - 10                          \\
                                         & = (\lambda - 5) (\lambda + 2)                         \\ \\
  \lambda^m                              & = c_0 + c_1 \lambda                                   \\
  (-2)^m                                 & = c_0 - 2 c_1                                         \\
  (5)^m                                  & = c_0 + 5 c_1                                         \\ \\
  (-2)^m + \frac{2}{5} (5)^m             & = \frac{7}{5} c_0                                     \\
  \frac{5}{7} (-2)^m + \frac{2}{7} (5)^m & = c_0                                                 \\ \\
  (5)^m - (-2)^m                         & = 7 c_1                                               \\
  \frac{1}{7} (5)^m - \frac{1}{7} (-2)^m & = c_1                                                 \\ \\
  \mathbf{A}^m                           & = \frac{1}{7} \begin{pmatrix}
                                                           5^m + 6 (-2)^m     & 3 (5)^m - 3 (-2)^m \\
                                                           2 (5)^m - 2 (-2)^m & 6 (5)^m + (-2)^m
                                                         \end{pmatrix} \\
  \mathbf{A}^3                           & = \frac{1}{7} \begin{pmatrix}
                                                           5^3 + 6 (-2)^3     & 3 (5)^3 - 3 (-2)^3 \\
                                                           2 (5)^3 - 2 (-2)^3 & 6 (5)^3 + (-2)^3
                                                         \end{pmatrix} \\
                                         & = \begin{pmatrix}
                                               11 & 57  \\
                                               38 & 106
                                             \end{pmatrix}
\end{align*}

\setcounter{subsubsection}{4}
\subsubsection{}

\begin{align*}
  \mathbf{A}                              & = \begin{pmatrix}
                                                8 & 5 \\
                                                4 & 0
                                              \end{pmatrix}                                                   \\
  \det (\mathbf{A} - \lambda \mathbf{I})  & = -\lambda (8 - \lambda) - 20                                      \\
                                          & = -8 \lambda + \lambda^2 - 20                                      \\
                                          & = \lambda^2 - 8 \lambda - 20                                       \\
                                          & = (\lambda - 10) (\lambda + 2)                                     \\ \\
  \lambda^m                               & = c_0 + c_1 \lambda                                                \\
  (-2)^m                                  & = c_0 - 2 c_1                                                      \\
  10^m                                    & = c_0 + 10 c_1                                                     \\ \\
  10^m + 5 (-2)^m                         & = 6 c_0                                                            \\
  \frac{1}{6} (10^m + 5 (-2)^m)           & = c_0                                                              \\
  \frac{1}{12} (2 \cdot 10^m + 10 (-2)^m) & = c_0                                                              \\ \\
  10^m - (-2)^m                           & = 12 c_1                                                           \\
  \frac{1}{12} (10^m - (-2)^m)            & = c_1                                                              \\ \\
  \mathbf{A}^m                            & = \frac{1}{12} \begin{pmatrix}
                                                             10 \cdot 10^m + 2 (-2)^m & 5 \cdot 10^m - 5 (-2)^m  \\
                                                             4 \cdot 10^m - 4 (-2)^m  & 2 \cdot 10^m + 10 (-2)^m
                                                           \end{pmatrix} \\
  \mathbf{A}^5                            & = \begin{pmatrix}
                                                83328 & 41680 \\
                                                33344 & 16640
                                              \end{pmatrix}
\end{align*}

\setcounter{subsubsection}{10}
\subsubsection{}

\begin{align*}
  \mathbf{A}                             & = \begin{pmatrix}
                                               7  & 3 \\
                                               -3 & 1
                                             \end{pmatrix}                                                \\
  \det (\mathbf{A} - \lambda \mathbf{I}) & = (7 - \lambda) (1 - \lambda) + 9                               \\
                                         & = 7 - 7 \lambda - \lambda + \lambda^2 + 9                       \\
                                         & = \lambda^2 - 8 \lambda + 16                                    \\
                                         & = (\lambda - 4)^2                                               \\ \\
  \lambda^m                              & = c_0 + c_1 \lambda                                             \\
  m \lambda^{m - 1}                      & = c_1                                                           \\ \\
  4^{m - 1} m                            & = c_1                                                           \\ \\
  4^m                                    & = c_0 + 4^m m                                                   \\
  4^m (1 - m)                            & = c_0                                                           \\ \\
  \mathbf{A}^m                           & = \begin{pmatrix}
                                               4^m (1 - m) + 7 \cdot 4^{m - 1} m & 3 \cdot 4^{m - 1} m       \\
                                               -3 \cdot 4^{m - 1} m              & 4^m (1 - m) + 4^{m - 1} m
                                             \end{pmatrix} \\
                                         & = 4^m \begin{pmatrix}
                                                   1 - m + \frac{7}{4} m & \frac{3}{4} m         \\
                                                   -\frac{3}{4} m        & 1 - m + \frac{1}{4} m
                                                 \end{pmatrix}             \\
                                         & = 4^m \begin{pmatrix}
                                                   1 + \frac{3}{4} m & \frac{3}{4} m     \\
                                                   -\frac{3}{4} m    & 1 - \frac{3}{4} m
                                                 \end{pmatrix}                     \\
  \mathbf{A}^6                           & = 4^6 \begin{pmatrix}
                                                   1 + \frac{3}{4} 6 & \frac{3}{4} 6     \\
                                                   -\frac{3}{4} 6    & 1 - \frac{3}{4} 6
                                                 \end{pmatrix}                     \\
                                         & = 4^5 \begin{pmatrix}
                                                   4 + 18 & 18     \\
                                                   -18    & 4 - 18
                                                 \end{pmatrix}                                           \\
                                         & = 4^5 \begin{pmatrix}
                                                   22  & 18  \\
                                                   -18 & -14
                                                 \end{pmatrix}                                            \\
                                         & = \begin{pmatrix}
                                               22528  & 18432  \\
                                               -18432 & -14336
                                             \end{pmatrix}
\end{align*}

\setcounter{subsubsection}{12}
\subsubsection{}

\begin{enumerate}
  \item

        \begin{align*}
          \mathbf{A}                             & = \begin{pmatrix}
                                                       1 & 1 \\
                                                       3 & 3
                                                     \end{pmatrix}                          \\
          \det (\mathbf{A} - \lambda \mathbf{I}) & = (1 - \lambda) (3 - \lambda) - 3         \\
                                                 & = 3 - \lambda - 3 \lambda + \lambda^2 - 3 \\
                                                 & = \lambda^2 - 4 \lambda                   \\
                                                 & = \lambda (\lambda - 4)                   \\ \\
          \lambda^m                              & = c_1 \lambda                             \\
          4^m                                    & = 4 c_1                                   \\
          c_1                                    & = 4^{m - 1}                               \\ \\
          \mathbf{A}^m                           & = 4^{m - 1} \mathbf{A}
        \end{align*}
\end{enumerate}

\setcounter{subsubsection}{14}
\subsubsection{}

\begin{align*}
  \mathbf{A}                             & = \begin{pmatrix}
                                               2 & -4 \\
                                               1 & 3
                                             \end{pmatrix}                                     \\
  \det (\mathbf{A} - \lambda \mathbf{I}) & = (2 - \lambda) (3 - \lambda) + 4                    \\
                                         & = 6 - 2 \lambda - 3 \lambda + \lambda^2 + 4          \\
                                         & = \lambda^2 - 5 \lambda + 10                         \\
  10 \mathbf{I}                          & = 5 \mathbf{A} - \mathbf{A}^2                        \\
  \mathbf{I}                             & = \frac{1}{2} \mathbf{A} - \frac{1}{10} \mathbf{A}^2 \\
  \mathbf{A}^{-1}                        & = \frac{1}{2} \mathbf{I} - \frac{1}{10} \mathbf{A}   \\
                                         & = \begin{pmatrix}
                                               \frac{3}{10}  & \frac{2}{5} \\
                                               -\frac{1}{10} & \frac{1}{5}
                                             \end{pmatrix}
\end{align*}

\subsection{Orthogonal Matrices}

\setcounter{subsubsection}{4}
\subsubsection{}

Orthogonal

\setcounter{subsubsection}{6}
\subsubsection{}

Orthogonal

\setcounter{subsubsection}{8}
\subsubsection{}

Not orthogonal

\setcounter{subsubsection}{10}
\subsubsection{}

\begin{align*}
  \mathbf{A}                             & = \begin{pmatrix}
                                               1 & 9 \\
                                               9 & 1
                                             \end{pmatrix}                 \\
  \det (\mathbf{A} - \lambda \mathbf{I}) & = (1 - \lambda)^2 - 81           \\
                                         & = \lambda^2 - 2 \lambda + 1 - 81 \\
                                         & = \lambda^2 - 2 \lambda - 80     \\
                                         & = (\lambda - 10) (\lambda + 8)
\end{align*}

\[\begin{pmatrix}
    9 & 9 \\
    9 & 9
  \end{pmatrix}\]

\[\mathbf{X}_1 = \begin{pmatrix}
    1 \\
    -1
  \end{pmatrix}\]

\[\begin{pmatrix}
    -9 & 9  \\
    9  & -9
  \end{pmatrix}\]

\[\mathbf{X}_2 = \begin{pmatrix}
    1 \\
    1
  \end{pmatrix}\]

\[\begin{pmatrix}
    \frac{1}{\sqrt{2}}  & \frac{1}{\sqrt{2}} \\
    -\frac{1}{\sqrt{2}} & \frac{1}{\sqrt{2}}
  \end{pmatrix}\]

\setcounter{subsubsection}{12}
\subsubsection{}

\begin{align*}
  \mathbf{A}                             & = \begin{pmatrix}
                                               1 & 3 \\
                                               3 & 9
                                             \end{pmatrix}                          \\
  \det (\mathbf{A} - \lambda \mathbf{I}) & = (1 - \lambda) (9 - \lambda) - 9         \\
                                         & = 9 - \lambda - 9 \lambda + \lambda^2 - 9 \\
                                         & = \lambda^2 - 10 \lambda                  \\
                                         & = \lambda (\lambda - 10)
\end{align*}

\[\begin{pmatrix}
    1 & 3 \\
    3 & 9
  \end{pmatrix}\]

\[\mathbf{X}_1 = \begin{pmatrix}
    3 \\
    -1
  \end{pmatrix}\]

\[\begin{pmatrix}
    -9 & 3  \\
    3  & -1
  \end{pmatrix}\]

\[\mathbf{X}_2 = \begin{pmatrix}
    1 \\
    3
  \end{pmatrix}\]

\[\begin{pmatrix}
    \frac{3}{\sqrt{10}}  & \frac{1}{\sqrt{10}} \\
    -\frac{1}{\sqrt{10}} & \frac{3}{\sqrt{10}}
  \end{pmatrix}\]

\setcounter{subsubsection}{18}
\subsubsection{}

\begin{align*}
  \frac{3}{5} a + \frac{4}{5} b & = 0              \\
  3 a                           & = -4 b           \\
  a                             & = -\frac{4}{3} b \\
  a                             & = -\frac{4}{5}   \\
  b                             & = \frac{3}{5}
\end{align*}

\setcounter{subsubsection}{20}
\subsubsection{}

\begin{enumerate}
  \setcounter{enumi}{1}
  \item $\lambda_1 = -2, \lambda_2 = -2, \lambda_3 = 4$

  \item

        \begin{align*}
          \mathbf{W}_1   & = \begin{pmatrix}
                               \frac{1}{\sqrt{2}}  \\
                               -\frac{1}{\sqrt{2}} \\
                               0
                             \end{pmatrix}                                           \\
          \mathbf{V}_2   & = \mathbf{K}_2 - (\mathbf{K}_2 \cdot \mathbf{W}_1) \mathbf{W}_1               \\
                         & = \begin{pmatrix}
                               1 \\
                               0 \\
                               -1
                             \end{pmatrix} - \frac{1}{\sqrt{2}} \begin{pmatrix}
                                                                  \frac{1}{\sqrt{2}}  \\
                                                                  -\frac{1}{\sqrt{2}} \\
                                                                  0
                                                                \end{pmatrix}        \\
                         & = \begin{pmatrix}
                               \frac{1}{2} \\
                               \frac{1}{2} \\
                               -1
                             \end{pmatrix}                                                              \\
          |\mathbf{V}_2| & = \sqrt{\left( \frac{1}{2} \right)^2 + \left( \frac{1}{2} \right)^2 + (-1)^2} \\
                         & = \sqrt{\frac{3}{2}}                                                          \\
          \mathbf{W}_2   & = \sqrt{\frac{2}{3}} \begin{pmatrix}
                                                  \frac{1}{2} \\
                                                  \frac{1}{2} \\
                                                  -1
                                                \end{pmatrix}                                           \\
                         & = \begin{pmatrix}
                               \frac{1}{\sqrt{6}} \\
                               \frac{1}{\sqrt{6}} \\
                               -\sqrt{\frac{2}{3}}
                             \end{pmatrix}
        \end{align*}

        \[\begin{pmatrix}
            \frac{1}{\sqrt{2}}  & \frac{1}{\sqrt{6}}  & \frac{1}{\sqrt{3}} \\
            -\frac{1}{\sqrt{2}} & \frac{1}{\sqrt{6}}  & \frac{1}{\sqrt{3}} \\
            0                   & -\sqrt{\frac{2}{3}} & \frac{1}{\sqrt{3}}
          \end{pmatrix}\]
\end{enumerate}

\subsection{Approximation of Eigenvalues}

\subsubsection{}

\begin{align*}
  \mathbf{A}                & = \begin{pmatrix}
                                  1 & 1 \\
                                  2 & 0
                                \end{pmatrix}                                                                     \\
  \mathbf{X}_0              & = \begin{pmatrix}
                                  1 \\
                                  1
                                \end{pmatrix}                                                                     \\
  \mathbf{A}^5 \mathbf{X}_0 & = 32\begin{pmatrix}
                                    1 \\
                                    1
                                  \end{pmatrix}                                                                   \\
  \mathbf{K}_1              & = \begin{pmatrix}
                                  1 \\
                                  1
                                \end{pmatrix}                                                                     \\
  \lambda_1                 & = \frac{\mathbf{A} \mathbf{X}_5 \cdot \mathbf{X}_5}{\mathbf{X}_5 \cdot \mathbf{X}_5} \\
                            & = 2
\end{align*}

\setcounter{subsubsection}{2}
\subsubsection{}

\begin{align*}
  \mathbf{A}   & = \begin{pmatrix}
                     2 & 4  \\
                     3 & 13
                   \end{pmatrix}                                                                     \\
  \mathbf{X}_1 & = \begin{pmatrix}
                     1 \\
                     1
                   \end{pmatrix}                                                                     \\
  \mathbf{X}_2 & = \begin{pmatrix}
                     \frac{3}{8} \\
                     1
                   \end{pmatrix}                                                                     \\
  \mathbf{X}_3 & = \begin{pmatrix}
                     0.3363 \\
                     1
                   \end{pmatrix}                                                                     \\
  \mathbf{X}_4 & = \begin{pmatrix}
                     0.3335 \\
                     1
                   \end{pmatrix}                                                                     \\
  \mathbf{X}_5 & = \begin{pmatrix}
                     0.3333 \\
                     1
                   \end{pmatrix}                                                                     \\
  \mathbf{K}_1 & = \begin{pmatrix}
                     \frac{1}{3} \\
                     1
                   \end{pmatrix}                                                                     \\
  \lambda_1    & = \frac{\mathbf{A} \mathbf{K}_1 \cdot \mathbf{K}_1}{\mathbf{K}_1 \cdot \mathbf{K}_1} \\
               & = 14
\end{align*}

\setcounter{subsubsection}{6}
\subsubsection{}

\begin{align*}
  \mathbf{A}                & = \begin{pmatrix}
                                  3 & 2 \\
                                  2 & 6
                                \end{pmatrix}                                     \\
  \mathbf{X}_1              & = \begin{pmatrix}
                                  1 \\
                                  1
                                \end{pmatrix}                                     \\
  \mathbf{A}^5 \mathbf{X}_1 & = \begin{pmatrix}
                                  0.5008 \\
                                  1
                                \end{pmatrix}                                     \\
  \lambda_1                 & = 7                                                  \\
  \mathbf{K}_1              & = \begin{pmatrix}
                                  \frac{1}{\sqrt{5}} \\
                                  \frac{2}{\sqrt{5}}
                                \end{pmatrix}                    \\
  \mathbf{B}                & = \mathbf{A} - \lambda_1 \mathbf{K}_1 \mathbf{K}_1^T \\
                            & = \begin{pmatrix}
                                  \frac{8}{5}  & -\frac{4}{5} \\
                                  -\frac{4}{5} & \frac{2}{5}
                                \end{pmatrix}                        \\
  \mathbf{B}^5 \mathbf{X}_1 & = \begin{pmatrix}
                                  1 \\
                                  -\frac{1}{2}
                                \end{pmatrix}                                     \\
  \lambda_2                 & = 2
\end{align*}

\setcounter{subsubsection}{10}
\subsubsection{}

\begin{align*}
  \mathbf{A}                       & = \begin{pmatrix}
                                         1 & 1 \\
                                         3 & 4
                                       \end{pmatrix} \\
  \det \mathbf{A}                  & = 1              \\
  \mathbf{A}^{-1}                  & = \begin{pmatrix}
                                         4  & -1 \\
                                         -3 & 1
                                       \end{pmatrix} \\
  \mathbf{X}_1                     & = \begin{pmatrix}
                                         1 \\
                                         1
                                       \end{pmatrix} \\
  (\mathbf{A}^{-1})^5 \mathbf{X}_1 & = \begin{pmatrix}
                                         1 \\
                                         -0.7913
                                       \end{pmatrix} \\
  \lambda_1'                       & \approx 4.78     \\
  \lambda_1                        & \approx 0.21
\end{align*}

\setcounter{subsubsection}{12}
\subsubsection{}

\begin{enumerate}
  \item

        \begin{align*}
          E I \frac{d^2 y}{d x^2} + P y                                        & = 0 \\
          E I \left( \frac{y_{i + 1} - 2 y_i + y_{i - 1}}{h^2} \right) + P y_i & = 0 \\
          E I (y_{i + 1} - 2 y_i + y_{i - 1}) + P h^2 y_i                      & = 0
        \end{align*}

  \item

        \begin{align*}
          E I (y_2 - 2 y_1) + P h^2 y_1       & = 0         \\
          E I (2 y_1 - y_2)                   & = P h^2 y_1 \\ \\
          E I (y_3 - 2 y_2 + y_1) + P h^2 y_2 & = 0         \\
          E I (-y_1 + 2 y_2 - y_3)            & = P h^2 y_2 \\ \\
          E I (-2 y_3 + y_2) + P h^2 y_3      & = 0         \\
          E I (-y_2 + 2 y_3)                  & = P h^2 y_3
        \end{align*}

        \[\begin{pmatrix}
            2  & -1 & 0  \\
            -1 & 2  & -1 \\
            0  & -1 & 2
          \end{pmatrix} \begin{pmatrix}
            y_1 \\
            y_2 \\
            y_3
          \end{pmatrix} = \frac{P L^2}{16 E I} \begin{pmatrix}
            y_1 \\
            y_2 \\
            y_3
          \end{pmatrix}\]

  \item

        \begin{align*}
          \det \mathbf{A} & = 2 \begin{vmatrix}
                                  2  & -1 \\
                                  -1 & 2
                                \end{vmatrix} + \begin{vmatrix}
                                                  -1 & -1 \\
                                                  0  & 2
                                                \end{vmatrix}          \\
                          & = 4                                         \\
          \mathbf{A}^{-1} & = \frac{1}{\det \mathbf{A}} \adj \mathbf{A} \\
                          & = \frac{1}{4} \begin{pmatrix}
                                            3 & 2 & 1 \\
                                            2 & 4 & 2 \\
                                            1 & 2 & 3
                                          \end{pmatrix}^T               \\
                          & = \begin{pmatrix}
                                \frac{3}{4} & \frac{1}{2} & \frac{1}{4} \\
                                \frac{1}{2} & 1           & \frac{1}{2} \\
                                \frac{1}{4} & \frac{1}{2} & \frac{3}{4}
                              \end{pmatrix}
        \end{align*}

  \item

        \begin{align*}
          \mathbf{X}_1                     & = \begin{pmatrix}
                                                 1 \\
                                                 1 \\
                                                 1
                                               \end{pmatrix} \\
          (\mathbf{A}^{-1})^6 \mathbf{X}_1 & = \begin{pmatrix}
                                                 0.7071 \\
                                                 1      \\
                                                 0.7071
                                               \end{pmatrix} \\
          \lambda_1'                       & \approx 1.7071   \\
          \lambda_1                        & \approx 0.59
        \end{align*}

  \item

        \begin{align*}
          \lambda_1 & = \frac{P L^2}{16 E I}         \\
          P         & = \frac{16 E I \lambda_1}{L^2} \\
                    & \approx \frac{9.44 E I}{L^2}
        \end{align*}
\end{enumerate}

\subsection{Diagonalization}

\subsubsection{}

\begin{align*}
  \mathbf{A}                             & = \begin{pmatrix}
                                               2 & 3 \\
                                               1 & 4
                                             \end{pmatrix}                            \\
  \det (\mathbf{A} - \lambda \mathbf{I}) & = (2 - \lambda) (4 - \lambda) - 3           \\
                                         & = 8 - 2 \lambda - 4 \lambda + \lambda^2 - 3 \\
                                         & = \lambda^2 - 6 \lambda + 5                 \\
                                         & = (\lambda - 5) (\lambda - 1)               \\
  \lambda_1                              & = 1                                         \\
  \lambda_2                              & = 5
\end{align*}

\[\begin{pmatrix}
    1 & 3 \\
    1 & 3
  \end{pmatrix}\]

\[\mathbf{X}_1 = \begin{pmatrix}
    -3 \\
    1
  \end{pmatrix}\]

\[\begin{pmatrix}
    -3 & 3  \\
    1  & -1
  \end{pmatrix}\]

\[\mathbf{X}_2 = \begin{pmatrix}
    1 \\
    1
  \end{pmatrix}\]

\begin{align*}
  \mathbf{P}      & = \begin{pmatrix}
                        -3 & 1 \\
                        1  & 1
                      \end{pmatrix}             \\
  \mathbf{P}^{-1} & = \begin{pmatrix}
                        -\frac{1}{4} & \frac{1}{4} \\
                        \frac{1}{4}  & \frac{3}{4}
                      \end{pmatrix} \\
  \mathbf{D}      & = \begin{pmatrix}
                        1 & 0 \\
                        0 & 5
                      \end{pmatrix}
\end{align*}

\setcounter{subsubsection}{2}
\subsubsection{}

Not diagonalisable

\setcounter{subsubsection}{4}
\subsubsection{}

\begin{align*}
  \mathbf{P} & = \begin{pmatrix}
                   13 & 1 \\
                   2  & 1
                 \end{pmatrix} \\
  \mathbf{D} & = \begin{pmatrix}
                   -7 & 0 \\
                   0  & 4
                 \end{pmatrix}
\end{align*}

\setcounter{subsubsection}{6}
\subsubsection{}

\begin{align*}
  \mathbf{P} & = \begin{pmatrix}
                   1 & -1 \\
                   1 & 1
                 \end{pmatrix}            \\
  \mathbf{D} & = \begin{pmatrix}
                   \frac{2}{3} & 0           \\
                   0           & \frac{1}{3}
                 \end{pmatrix}
\end{align*}

\setcounter{subsubsection}{20}
\subsubsection{}

\begin{align*}
  \mathbf{P} & = \begin{pmatrix}
                   1 & -1 \\
                   1 & 1
                 \end{pmatrix} \\
  \mathbf{D} & = \begin{pmatrix}
                   2 & 0 \\
                   0 & 0
                 \end{pmatrix}
\end{align*}

\setcounter{subsubsection}{22}
\subsubsection{}

\begin{align*}
  \mathbf{P} & = \begin{pmatrix}
                   \sqrt{\frac{2}{5}} & -\sqrt{\frac{5}{2}} \\
                   1                  & 1
                 \end{pmatrix} \\
  \mathbf{D} & = \begin{pmatrix}
                   10 & 0 \\
                   0  & 3
                 \end{pmatrix}
\end{align*}

\setcounter{subsubsection}{34}
\subsubsection{}

\begin{align*}
  \mathbf{A} & = \mathbf{P} \mathbf{D} \mathbf{P}^{-1}      \\
             & = \begin{pmatrix}
                   1 & 1 \\
                   2 & 1
                 \end{pmatrix} \begin{pmatrix}
                                 2 & 0 \\
                                 0 & 3
                               \end{pmatrix} \begin{pmatrix}
                                               -1 & 1  \\
                                               2  & -1
                                             \end{pmatrix} \\
             & = \begin{pmatrix}
                   4 & -1 \\
                   2 & 1
                 \end{pmatrix}
\end{align*}

\setcounter{subsubsection}{38}
\subsubsection{}

\begin{align*}
  \mathbf{D}   & = \begin{pmatrix}
                     2 & 0  \\
                     0 & -1
                   \end{pmatrix}                                         \\
  \mathbf{P}   & = \begin{pmatrix}
                     1 & -1 \\
                     1 & 2
                   \end{pmatrix}                                         \\
  \mathbf{A}^5 & = \mathbf{P} \mathbf{D}^5 \mathbf{P}^{-1}                \\
               & = \begin{pmatrix}
                     1 & -1 \\
                     1 & 2
                   \end{pmatrix} \begin{pmatrix}
                                   32 & 0  \\
                                   0  & -1
                                 \end{pmatrix} \begin{pmatrix}
                                                 \frac{2}{3}  & \frac{1}{3} \\
                                                 -\frac{1}{3} & \frac{1}{3}
                                               \end{pmatrix} \\
               & = \begin{pmatrix}
                     21 & 11 \\
                     22 & 10
                   \end{pmatrix}
\end{align*}

\subsection{LU-Factorisation}

\subsubsection{}

\begin{align*}
  \mathbf{A}      & = \mathbf{L U}                           \\
  \begin{pmatrix}
    2 & -2 \\
    1 & 2
  \end{pmatrix} & = \begin{pmatrix}
                      1      & 0 \\
                      l_{21} & 1
                    \end{pmatrix} \begin{pmatrix}
                                    u_{11} & u_{12} \\
                                    0      & u_{22}
                                  \end{pmatrix}            \\
                  & = \begin{pmatrix}
                        u_{11}        & u_{12}                 \\
                        l_{21} u_{11} & l_{21} u_{12} + u_{22}
                      \end{pmatrix} \\
  u_{11}          & = 2                                      \\
  u_{12}          & = -2                                     \\
  l_{21}          & = \frac{1}{2}                            \\
  u_{22}          & = 3                                      \\
  \mathbf{L}      & = \begin{pmatrix}
                        1           & 0 \\
                        \frac{1}{2} & 1
                      \end{pmatrix}                        \\
  \mathbf{U}      & = \begin{pmatrix}
                        2 & -2 \\
                        0 & 3
                      \end{pmatrix}
\end{align*}

\setcounter{subsubsection}{2}
\subsubsection{}

\begin{align*}
  \begin{pmatrix}
    -1 & 4 \\
    2  & 2
  \end{pmatrix} & = \begin{pmatrix}
                      u_{11}        & u_{12}                 \\
                      l_{21} u_{11} & l_{21} u_{12} + u_{22}
                    \end{pmatrix} \\
  u_{11}          & = -1                                   \\
  u_{12}          & = 4                                    \\
  l_{21}          & = -2                                   \\
  u_{22}          & = 10                                   \\
  \mathbf{L}      & = \begin{pmatrix}
                        1  & 0 \\
                        -2 & 1
                      \end{pmatrix}                       \\
  \mathbf{U}      & = \begin{pmatrix}
                        -1 & 4  \\
                        0  & 10
                      \end{pmatrix}
\end{align*}

\setcounter{subsubsection}{10}
\subsubsection{}

\begin{align*}
  \begin{pmatrix}
    3 & 9  \\
    1 & 11
  \end{pmatrix} & \Rightarrow \begin{pmatrix}
                                3 & 9 \\
                                0 & 8
                              \end{pmatrix} \\
  \mathbf{L}      & = \begin{pmatrix}
                        1           & 0 \\
                        \frac{1}{3} & 1
                      \end{pmatrix}        \\
  \mathbf{U}      & = \begin{pmatrix}
                        3 & 9 \\
                        0 & 8
                      \end{pmatrix}
\end{align*}

\setcounter{subsubsection}{12}
\subsubsection{}

\begin{align*}
  \begin{pmatrix}
    -4 & -2 \\
    -1 & -3
  \end{pmatrix} & \Rightarrow \begin{pmatrix}
                                -4 & -2           \\
                                0  & -\frac{5}{2}
                              \end{pmatrix} \\
  \mathbf{L}      & = \begin{pmatrix}
                        1           & 0 \\
                        \frac{1}{4} & 1
                      \end{pmatrix}           \\
  \mathbf{U}      & = \begin{pmatrix}
                        -4 & -2           \\
                        0  & -\frac{5}{2}
                      \end{pmatrix}
\end{align*}

\setcounter{subsubsection}{20}
\subsubsection{}

\begin{align*}
  \mathbf{L}                            & = \begin{pmatrix}
                                              1           & 0 \\
                                              \frac{1}{2} & 1
                                            \end{pmatrix} \\
  \mathbf{U}                            & = \begin{pmatrix}
                                              2 & -2 \\
                                              0 & 3
                                            \end{pmatrix}  \\
  \mathbf{L Y}                          & = \mathbf{B}      \\
  \begin{pmatrix}
    1           & 0 \\
    \frac{1}{2} & 1
  \end{pmatrix} \begin{pmatrix}
                  y_1 \\
                  y_2
                \end{pmatrix}         & = \begin{pmatrix}
                                            1 \\
                                            -2
                                          \end{pmatrix}    \\
  y_1                                   & = 1               \\
  y_2                                   & = -\frac{5}{2}    \\
  \mathbf{U X}                          & = \mathbf{Y}      \\
  \begin{pmatrix}
    2 & -2 \\
    0 & 3
  \end{pmatrix} \begin{pmatrix}
                  x_1 \\
                  x_2
                \end{pmatrix}         & = \begin{pmatrix}
                                            1 \\
                                            -\frac{5}{2}
                                          \end{pmatrix}    \\
  x_2                                   & = -\frac{5}{6}    \\
  2 x_1 - 2 \left( -\frac{5}{6} \right) & = 1               \\
  2 x_1                                 & = -\frac{2}{3}    \\
  x_1                                   & = -\frac{1}{3}    \\
  \mathbf{X}                            & = \begin{pmatrix}
                                              -\frac{1}{3} \\
                                              -\frac{5}{6}
                                            \end{pmatrix}
\end{align*}

\setcounter{subsubsection}{22}
\subsubsection{}

\begin{align*}
  \mathbf{L}                    & = \begin{pmatrix}
                                      1  & 0 \\
                                      -2 & 1
                                    \end{pmatrix} \\
  \mathbf{U}                    & = \begin{pmatrix}
                                      -1 & 4  \\
                                      0  & 10
                                    \end{pmatrix} \\
  \mathbf{L Y}                  & = \mathbf{B}     \\
  \begin{pmatrix}
    1  & 0 \\
    -2 & 1
  \end{pmatrix} \begin{pmatrix}
                  y_1 \\
                  y_2
                \end{pmatrix} & = \begin{pmatrix}
                                    15 \\
                                    5
                                  \end{pmatrix}   \\
  y_1                           & = 15             \\
  y_2                           & = 35             \\
  \mathbf{U X}                  & = \mathbf{Y}     \\
  \begin{pmatrix}
    -1 & 4  \\
    0  & 10
  \end{pmatrix} \begin{pmatrix}
                  x_1 \\
                  x_2
                \end{pmatrix} & = \begin{pmatrix}
                                    15 \\
                                    35
                                  \end{pmatrix}   \\
  x_2                           & = \frac{7}{2}    \\
  -x_1 + 4 \frac{7}{2}          & = 15             \\
  x_1                           & = -1             \\
  \mathbf{X}                    & = \begin{pmatrix}
                                      -1 \\
                                      \frac{7}{2}
                                    \end{pmatrix}
\end{align*}

\setcounter{subsubsection}{30}
\subsubsection{}

\begin{align*}
  \mathbf{L Y}                  & = \mathbf{B}     \\
  \begin{pmatrix}
    1 & 0 & 0 \\
    1 & 1 & 0 \\
    1 & 1 & 1
  \end{pmatrix} \begin{pmatrix}
                  y_1 \\
                  y_2 \\
                  y_3
                \end{pmatrix} & = \begin{pmatrix}
                                    2 \\
                                    4 \\
                                    1
                                  \end{pmatrix}   \\
  y_1                           & = 2              \\
  y_2                           & = 2              \\
  y_3                           & = -3             \\
  \mathbf{U X}                  & = \mathbf{Y}     \\
  \begin{pmatrix}
    1 & 1 & 1 \\
    0 & 1 & 1 \\
    0 & 0 & 1
  \end{pmatrix} \begin{pmatrix}
                  x_1 \\
                  x_2 \\
                  x_3
                \end{pmatrix} & = \begin{pmatrix}
                                    2 \\
                                    2 \\
                                    -3
                                  \end{pmatrix}   \\
  x_3                           & = -3             \\
  x_2                           & = 5              \\
  x_1                           & = 0              \\
  \mathbf{X}                    & = \begin{pmatrix}
                                      0 \\
                                      5 \\
                                      -3
                                    \end{pmatrix}
\end{align*}

\subsection{Cryptography}

\subsubsection{}

\begin{align*}
  \mathbf{B} & = \mathbf{A M}                       \\
             & = \begin{pmatrix}
                   1 & 2 \\
                   1 & 1
                 \end{pmatrix} \begin{pmatrix}
                                 19 & 5 & 14 & 4  & 0 \\
                                 8  & 5 & 12 & 16 & 0
                               \end{pmatrix} \\
             & = \begin{pmatrix}
                   35 & 15 & 38 & 36 & 0 \\
                   27 & 10 & 26 & 20 & 0
                 \end{pmatrix}
\end{align*}

\setcounter{subsubsection}{6}
\subsubsection{}

\begin{align*}
  \mathbf{A}^{-1} & = \begin{pmatrix}
                        2  & -3 \\
                        -5 & 8
                      \end{pmatrix}         \\
  \mathbf{M}      & = \mathbf{A^{-1} B}      \\
                  & = \begin{pmatrix}
                        19 & 20 & 21 & 4  & 25 \\
                        0  & 8  & 1  & 18 & 4
                      \end{pmatrix} \\
                  & = \text{STUDY HARD}
\end{align*}

\setcounter{subsubsection}{10}
\subsubsection{}

\begin{align*}
  \mathbf{A^{-1} B}                    & = \mathbf{M}                    \\
  \begin{pmatrix}
    a_{11} & a_{12} \\
    a_{21} & a_{22}
  \end{pmatrix} \begin{pmatrix}
                  17  & 16  \\
                  -30 & -31
                \end{pmatrix}        & = \begin{pmatrix}
                                           4      & 1      \\
                                           m_{21} & m_{22}
                                         \end{pmatrix}                 \\
  17 a_{11} - 30 a_{12}                & = 4                             \\
  16 a_{11} - 31 a_{12}                & = 1                             \\ \\
  17 a_{11} - \frac{30}{31} 16 a_{11}  & = 4 - \frac{30}{31}             \\
  527 a_{11} - 480 a_{11}              & = 124 - 30                      \\
  47 a_{11}                            & = 94                            \\
  a_{11}                               & = 2                             \\ \\
  -30 a_{12} + \frac{17}{16} 31 a_{12} & = 4 - \frac{17}{16}             \\
  -480 a_{12} + 527 a_{12}             & = 64 - 17                       \\
  47 a_{12}                            & = 47                            \\
  a_{12}                               & = 1                             \\ \\
  \begin{pmatrix}
    2      & 1      \\
    a_{21} & a_{22}
  \end{pmatrix} \begin{pmatrix}
                  5  & 25  \\
                  -6 & -50
                \end{pmatrix}        & = \begin{pmatrix}
                                           m_{11} & m_{12} \\
                                           1      & 25
                                         \end{pmatrix}                 \\
  5 a_{21} - 6 a_{22}                  & = 1                             \\
  25 a_{21} - 50 a_{22}                & = 25                            \\ \\
  5 a_{21} - \frac{6}{50} 25 a_{21}    & = 1 - \frac{6}{50} 25           \\
  250 a_{21} - 150 a_{21}              & = 50 - 150                      \\
  100 a_{21}                           & = -100                          \\
  a_{21}                               & = -1                            \\ \\
  4 a_{22}                             & = -4                            \\
  a_{22}                               & = -1                            \\ \\
  \mathbf{A}^{-1}                      & = \begin{pmatrix}
                                             2  & 1  \\
                                             -1 & -1
                                           \end{pmatrix}                \\
  \mathbf{A^{-1} B}                    & = \text{DAD I NEED MONEY TODAY}
\end{align*}

\end{document}