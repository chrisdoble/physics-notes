\documentclass{article}
\usepackage{amsfonts} % For open face letters
\usepackage{amsmath} % For align*
\usepackage{float} % For the [H] option on figures
\usepackage{graphicx} % For images
\usepackage{siunitx} % For units
\graphicspath{{./images/}}

\renewcommand{\vec}[1]{\boldsymbol{\mathbf{#1}}}
\newcommand{\dvec}[1]{\dot{\vec{#1}}}
\newcommand{\ddvec}[1]{\ddot{\vec{#1}}}
\newcommand{\uvec}[1]{\hat{\vec{#1}}}

\newcommand{\ke}{\frac{1}{4 \pi \epsilon_0}}

\def\rcurs{{\mbox{$\resizebox{.09in}{.08in}{\includegraphics[trim= 1em 0 14em 0,clip]{ScriptR}}$}}}
\def\brcurs{{\mbox{$\resizebox{.09in}{.08in}{\includegraphics[trim= 1em 0 14em 0,clip]{BoldR}}$}}}
\def\hrcurs{{\mbox{$\hat \brcurs$}}}

\title{Introduction to Electrodynamics by David J. Griffiths Notes}
\author{Chris Doble}
\date{December 2023}

\begin{document}

\maketitle

\tableofcontents

\section{Vector Algebra}

\setcounter{subsection}{5}
\subsection{The Theory of Vector Fields}

\subsubsection{The Helmholtz Theorem}

\begin{itemize}
  \item The \textbf{Helmholtz theorem} states that a vector field $\vec{F}$ is uniquely determined if you're given its divergence $\nabla \cdot \vec{F}$, curl $\nabla \times \vec{F}$, and sufficient boundary conditions.
\end{itemize}

\subsubsection{Potentials}

\begin{itemize}
  \item If the curl of a vector field vanishes everywhere, then it can be expressed as the gradient of a \textbf{scalar potential} \[\nabla \times \vec{F} = \vec{0} \Leftrightarrow \vec{F} = -\nabla V.\]

  \item If the divergence of a vector field vanishes everywhere, then it can be expressed as the curl of a \textbf{vector potential} \[\nabla \cdot \vec{F} = 0 \Leftrightarrow \vec{F} = \nabla \times \vec{A}.\]
\end{itemize}

\section{Electrostatics}

\subsection{The Electric Field}

\setcounter{subsubsection}{1}
\subsubsection{Coulomb's Law}

\begin{itemize}
  \item \textbf{Couloumb's law} gives the force between two point charges $q$ and $Q$ \[\vec{F} = \frac{1}{4 \pi \epsilon_0} \frac{q Q}{\rcurs} \hrcurs\] where \[\epsilon_0 = \qty{8.85e-12}{C^2/(N.m^2)}\] is the \textbf{permittivity of free space} and $\brcurs$ is the separation vector between the two charges.
\end{itemize}

\subsubsection{The Electric Field}

\begin{itemize}
  \item The \textbf{electric field} $\vec{E}$ is a vector field that varies from point to point and gives the force per unit charge that would be exerted on a test charge if placed at a particular point.

  \item For a collection of $n$ source charges $q_i$ at displacements $\brcurs_i$ from a test charge, the electric field is \[\vec{E} = \frac{1}{4 \pi \epsilon_0} \sum_{i = 1}^n \frac{q_i}{\rcurs_i^2} \hrcurs.\]
\end{itemize}

\subsubsection{Continuous Charge Distributions}

\begin{itemize}
  \item Couloumb's law for a continuous charge distribution is \[\vec{E} = \ke \int \frac{1}{\rcurs^2} \hrcurs \,d q.\]
\end{itemize}

\subsection{Divergence and Curl of Electrostatic Fields}

\subsubsection{Field Lines, Flux, and Gauss's Law}

\begin{itemize}
  \item \textbf{Gauss's law} states that the electric field flux through a closed surface is proportional to the amount of charge within that surface \[\oint \vec{E} \cdot d \vec{a} = \frac{1}{\epsilon_0} Q_\text{enc}\] or \[\nabla \cdot \vec{E} = \frac{1}{\epsilon_0} \rho.\]
\end{itemize}

\setcounter{subsubsection}{3}
\subsubsection{The Curl of E}

\begin{itemize}
  \item The curl of an electric field is $\vec{0}$ \[\nabla \times \vec{E} = \vec{0}.\]
\end{itemize}

\subsection{Electric Potential}

\subsubsection{Introduction to Potential}

\begin{itemize}
  \item The \textbf{electric potential} at a point $\vec{r}$ is defined as \[V(\vec{r}) = -\int_\mathcal{O}^{\vec{r}} \vec{E} \cdot d \vec{l}\] where $\mathcal{O}$ is an agreed origin.

  \item The potential difference between two points $\vec{a}$ and $\vec{b}$ is \[V(\vec{b}) - V(\vec{a}) = -\int_{\vec{a}}^{\vec{b}} \vec{E} \cdot d \vec{l}.\]

  \item The electric field and potential are also related by the equation \[\vec{E} = -\nabla V.\]
\end{itemize}

\subsubsection{Comments on Potential}

\begin{itemize}
  \item The choice of origin $\mathcal{O}$ in the definiton of vector potential only affects the absolute potential values, not potential differences. Typically it is chosen to be ``at infinity'' unless the charge distribution itself extends to infinity.

  \item Electric potential obeys the superposition principle.

  \item The units of electric potential is $\unit{N.m/C} = \unit{J/C} = \unit{V}$.
\end{itemize}

\subsubsection{Poisson's Equation and Laplace's Equation}

\begin{itemize}
  \item If \[\nabla \cdot \vec{E} = \frac{\rho}{\epsilon_0}\] and \[\vec{E} = -\nabla V\] then \begin{align*}
          \nabla \cdot (-\nabla V) & = \frac{\rho}{\epsilon_0}   \\
          \nabla^2 V               & = -\frac{\rho}{\epsilon_0}.
        \end{align*} This is known as \textbf{Poisson's equation}. In regions where $\rho = 0$ it reduces to \textbf{Laplace's equation} \[\nabla^2 V = 0.\]
\end{itemize}

\subsubsection{The Potential of a Localized Charge Distribution}

\begin{itemize}
  \item The potential of a continuous charge distribution is \[V(\vec{r}) = \ke \int \frac{\rho(\vec{r}')}{\rcurs} \,d \tau'\] where the reference is point is set to infinity.
\end{itemize}

\subsubsection{Boundary Conditions}

\begin{figure}[H]
  \centering
  \includegraphics[scale=0.5]{electrostatic-relations}
\end{figure}

\begin{itemize}
  \item The normal component of the electric field is discontinuous by an amount $\sigma / \epsilon_0$ at any boundary, i.e. \[E_\text{above} - E_\text{below} = \frac{\sigma}{\epsilon_0}.\]

  \item The tangential component of the electric field is always continuous at any boundary.

  \item The electric potential is always continuous at any boundary, however because $\vec{E} = -\nabla V$, the gradient of the electric potential inherits the discontinuity at boundaries with surface charge, i.e. \[\nabla V_\text{above} - \nabla V_\text{below} = -\frac{\sigma}{\epsilon_0} \uvec{n}\] or \[\frac{\partial V_\text{above}}{\partial n} - \frac{\partial V_\text{below}}{\partial n} = -\frac{\sigma}{\epsilon_0}\] where \[\frac{\partial V}{\partial n} = \nabla V \cdot \uvec{n}.\]
\end{itemize}

\end{document}