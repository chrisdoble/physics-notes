\documentclass{article}
\usepackage{amsmath} % For align*
\usepackage{graphicx} % For images
\usepackage{siunitx} % For units
\graphicspath{{./images/}}

\renewcommand{\vec}[1]{\boldsymbol{\mathbf{#1}}}
\newcommand{\dvec}[1]{\dot{\vec{#1}}}
\newcommand{\ddvec}[1]{\ddot{\vec{#1}}}
\newcommand{\uvec}[1]{\hat{\vec{#1}}}

\newcommand{\ke}{\frac{1}{4 \pi \epsilon_0}}

\def\rcurs{{\mbox{$\resizebox{.09in}{.08in}{\includegraphics[trim= 1em 0 14em 0,clip]{ScriptR}}$}}}
\def\brcurs{{\mbox{$\resizebox{.09in}{.08in}{\includegraphics[trim= 1em 0 14em 0,clip]{BoldR}}$}}}
\def\hrcurs{{\mbox{$\hat \brcurs$}}}

\title{Introduction to Electrodynamics by David J. Griffiths Notes}
\author{Chris Doble}
\date{December 2023}

\begin{document}

\maketitle

\tableofcontents

\section{Vector Algebra}

\setcounter{subsection}{5}
\subsection{The Theory of Vector Fields}

\subsubsection{The Helmholtz Theorem}

\begin{itemize}
  \item The \textbf{Helmholtz theorem} states that a vector field $\vec{F}$ is uniquely determined if you're given its divergence $\nabla \cdot \vec{F}$, curl $\nabla \times \vec{F}$, and sufficient boundary conditions.
\end{itemize}

\subsubsection{Potentials}

\begin{itemize}
  \item If the curl of a vector field vanishes everywhere, then it can be expressed as the gradient of a \textbf{scalar potential} \[\nabla \times \vec{F} = \vec{0} \Leftrightarrow \vec{F} = -\nabla V.\]

  \item If the divergence of a vector field vanishes everywhere, then it can be expressed as the curl of a \textbf{vector potential} \[\nabla \cdot \vec{F} = 0 \Leftrightarrow \vec{F} = \nabla \times \vec{A}.\]
\end{itemize}

\section{Electrostatics}

\subsection{The Electric Field}

\setcounter{subsubsection}{1}
\subsubsection{Coulomb's Law}

\begin{itemize}
  \item \textbf{Couloumb's law} gives the force between two point charges $q$ and $Q$ \[\vec{F} = \frac{1}{4 \pi \epsilon_0} \frac{q Q}{\rcurs} \hrcurs\] where \[\epsilon_0 = \qty{8.85e-12}{C^2/(N.m^2)}\] is the \textbf{permittivity of free space} and $\brcurs$ is the separation vector between the two charges.
\end{itemize}

\subsubsection{The Electric Field}

\begin{itemize}
  \item The \textbf{electric field} $\vec{E}$ is a vector field that varies from point to point and gives the force per unit charge that would be exerted on a test charge if placed at a particular point.

  \item For a collection of $n$ source charges $q_i$ at displacements $\brcurs_i$ from a test charge, the electric field is \[\vec{E} = \frac{1}{4 \pi \epsilon_0} \sum_{i = 1}^n \frac{q_i}{\rcurs_i^2} \hrcurs.\]
\end{itemize}

\subsubsection{Continuous Charge Distributions}

\begin{itemize}
  \item Couloumb's law for a continuous charge distribution is \[\vec{E} = \ke \int \frac{1}{\rcurs^2} \hrcurs \,d q.\]
\end{itemize}

\end{document}