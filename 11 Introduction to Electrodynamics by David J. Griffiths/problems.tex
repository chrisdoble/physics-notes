\documentclass{article}
\usepackage{amsfonts} % For open face letters
\usepackage{amsmath} % For align*
\usepackage{graphicx} % For images
\usepackage{enumitem} % For customisable list labels
\graphicspath{{./images/}}

\renewcommand{\vec}[1]{\boldsymbol{\mathbf{#1}}}
\newcommand{\dvec}[1]{\dot{\vec{#1}}}
\newcommand{\ddvec}[1]{\ddot{\vec{#1}}}
\newcommand{\uvec}[1]{\hat{\vec{#1}}}

\newcommand{\ke}{\frac{1}{4 \pi \epsilon_0}}

\def\rcurs{{\mbox{$\resizebox{.09in}{.08in}{\includegraphics[trim= 1em 0 14em 0,clip]{ScriptR}}$}}}
\def\brcurs{{\mbox{$\resizebox{.09in}{.08in}{\includegraphics[trim= 1em 0 14em 0,clip]{BoldR}}$}}}
\def\hrcurs{{\mbox{$\hat \brcurs$}}}

\setlist[enumerate, 1]{label={(\alph*)}}
\setlist[enumerate, 2]{label={(\roman*)}}

\title{Introduction to Electrodynamics by David J. Griffiths Problems}
\author{Chris Doble}
\date{December 2023}

\begin{document}

\maketitle

\tableofcontents

\setcounter{section}{1}
\section{Electrostatics}

\subsection{}

\begin{enumerate}
  \item $\vec{0}$

  \item The same as if only the opposite charge were present — all others are cancelled out.
\end{enumerate}

\subsection{}

\begin{align*}
  \vec{E} & = \ke 2 \frac{q}{\rcurs^2} \cos \theta \uvec{x}      \\
          & = \ke \frac{d q}{[(d / 2)^2 + z^2]^{3 / 2}} \uvec{x}
\end{align*}

\subsection{}

\begin{align*}
  \vec{r}  & = z \uvec{z}                                                                                                                          \\
  \vec{r}' & = x \uvec{x}                                                                                                                          \\
  \brcurs  & = z \uvec{z} - x \uvec{x}                                                                                                             \\
  \rcurs   & = \sqrt{x^2 + z^2}                                                                                                                    \\
  \hrcurs  & = \frac{z \uvec{z} - x \uvec{x}}{\sqrt{x^2 + z^2}}                                                                                    \\
  \vec{E}  & = \ke \int_0^L \frac{\lambda}{x^2 + z^2} \frac{z \uvec{z} - x \uvec{x}}{\sqrt{x^2 + z^2}} \,d x                                       \\
           & = \ke \lambda \left( z \uvec{z} \int_0^L \frac{1}{(x^2 + z^2)^{3 / 2}} \,d x - \uvec{x} \int_0^L \frac{x}{(x^2 + z^2)} \,d x \right)  \\
           & = \ke \lambda \left[ \frac{L}{z \sqrt{L^2 + z^2}} \uvec{z} - \left( \frac{1}{z} - \frac{1}{\sqrt{L^2 + z^2}} \right) \uvec{x} \right] \\
           & = \ke \frac{\lambda}{z} \left[ \left( -1 + \frac{z}{\sqrt{L^2 + z^2}} \right) \uvec{x} + \frac{L}{\sqrt{L^2 + z^2}} \uvec{z} \right]
\end{align*}

\subsection{}

The electric field a distance $z$ above the midpoint of a line segment of length $2 L$ and uniform line charge $\lambda$ is \[\vec{E} = \ke \frac{2 \lambda L}{z \sqrt{z^2 + L^2}} \uvec{z}.\]

Applying this to the four sides of the square, the horizontal components of opposite sides cancel leaving only the vertical component.

\begin{align*}
  \cos \theta & = \frac{z}{\rcurs}                                                                                                      \\
              & = \frac{z}{\sqrt{(a / 2)^2 + z^2}}                                                                                      \\
  \vec{E}     & = 4 \left( \ke \frac{\lambda a}{\sqrt{(a / 2)^2 + z^2} \sqrt{(a / 2)^2 + (a / 2)^2 + z^2}} \uvec{z} \right) \cos \theta \\
              & = \ke \frac{4 a \lambda z}{[(a / 2)^2 + z^2] \sqrt{(a^2 / 2) + z^2}} \uvec{z}
\end{align*}

\subsection{}

\begin{align*}
  \vec{E} & = \ke \int_0^{2 \pi} \frac{\lambda r}{r^2 + z^2} \cos \alpha \,d \theta \,\uvec{z} \\
          & = \ke \frac{2 \pi \lambda r z}{(r^2 + z^2)^{3 / 2}} \uvec{z}
\end{align*}

\subsection{}

\begin{align*}
  \vec{E} & = \ke \int \frac{d q}{\rcurs^2} \cos \theta \uvec{z}                                                          \\
          & = \ke \int_0^{2 \pi} \int_0^R \frac{\sigma}{r^2 + z^2} \frac{z}{\sqrt{r^2 + z^2}} r \,d r \,d \theta \uvec{z} \\
          & = \ke 2 \pi \sigma z \int_0^R \frac{r}{(r^2 + z^2)^{3 / 2}} \,d r \,\uvec{z}                                  \\
          & = \ke 2 \pi \sigma z \left( \frac{1}{z} - \frac{1}{\sqrt{R^2 + z^2}} \right) \uvec{z}
\end{align*}

When $R \rightarrow \infty$ \[\vec{E} = \frac{\sigma}{2 \epsilon_0} \uvec{z}.\]

\subsection{}

\[\vec{E} = \begin{cases}
    \ke \frac{q}{z^2} \uvec{z} & z > R \\
    \vec{0}                    & z < R
  \end{cases}\]

\subsection{}

\[\vec{E} = \begin{cases}
    \ke \frac{q}{z^2} \uvec{z}   & z > R \\
    \ke \frac{q z}{R^3} \uvec{z} & z < R
  \end{cases}\]

\end{document}