\documentclass{article}
\usepackage{amsmath} % For align*
\usepackage{enumitem} % For customisable list labels
\usepackage{graphicx} % For images
\usepackage{siunitx} % For units
\graphicspath{{./images/}}

\title{Vibrations and Waves by George C. King Notes}
\author{Chris Doble}
\date{April 2022}

\begin{document}

\maketitle

\tableofcontents

\section{Simple Harmonic Motion}

\begin{itemize}
  \item The equation of motion for a simple harmonic oscillator is \[\frac{d^2 x}{d t^2} = -\omega^2 x\] where \[\omega^2 = \frac{k}{m}\]

  \item The general solution of the equation of motion for a simple harmonic oscillator is \[x = A \cos (\omega t + \phi)\] or equivalently \[x = a \cos \omega t + b \sin \omega t\]

  \item The angular frequency $\omega$ is determined entirely by properties of the oscillator, e.g. its mass and spring coefficient

  \item The total energy of a harmonic oscillator is \[E = \frac{1}{2} k A^2\]

  \item Nearly all potential wells have a shape that is parabolic when sufficiently close to the equilibrium position, so most oscillating systems will oscillate with SHM when the amplitude of oscillation is small

  \item The vibrations of nuclei in a molecule can be modeled by SHM, but only a discrete set of vibrational energies is possible, namely \[\frac{1}{2} \hbar \omega, \, \frac{3}{2} \hbar \omega, \, \frac{5}{2} \hbar \omega, \, \ldots\] where $\hbar$ is Planck's constant divided by $2 \pi$

  \item The total energy of a system undergoing SHM is always given by an expression of the form \[E = \frac{1}{2} \alpha v^2 + \frac{1}{2} \beta x^2\] where $\alpha$ and $\beta$ are physical constants — if we obtain this equation during the analysis of a system we know we have SHM

  \item The equation of motion for a system described by the energy equation above is \[\frac{d^2 x}{d t^2} = -\frac{\beta}{\alpha} x\]
\end{itemize}

\section{The Damped Harmonic Oscillator}

\begin{itemize}
  \item The equation of motion of a damped harmonic oscillator is

        \begin{align*}
          F = m a                                                     & = -k x - b v \\
          m \frac{d^2 x}{d t^2} + b \frac{d x}{d t} + k x             & = 0          \\
          \frac{d^2 x}{d t^2} + \gamma \frac{d x}{d t} + \omega_0^2 x & = 0
        \end{align*}

        where $\gamma = b / m$ and $\omega_0^2 = k / m$

  \item $\omega_0$ is known as the \textbf{natural frequency of oscillation}, i.e. the oscillation frequency if there were no damping

  \item \textbf{Light damping / underdamped}

        \begin{itemize}
          \item The motion is still oscillatory but the amplitude decreases expontentially

          \item This occurs when $\gamma^2 / 4 < \omega_0^2$

          \item The general solution is \[x = A_0 e^{-\gamma t / 2} \cos (\omega t + \phi)\] where $A_0$ is the initial amplitude

          \item Successive maxima decrease by the same fractional amount \[\frac{A_n}{A_{n + 1}} = e^{\gamma T / 2}\]

          \item The natural logarithm of $A_n / A_{n + 1}$ is called the \textbf{logarithmic decrement} \[\ln \left( \frac{A_n}{A_{n + 1}} \right) = \frac{\gamma T}{2}\]
        \end{itemize}

  \item \textbf{Heavy damping / overdamped}

        \begin{itemize}
          \item The motion is not oscillatory and returns sluggishly to the equilibrium position

          \item This occurs when $\gamma^2 / 4 > \omega_0^2$

          \item The general solution is

                \begin{align*}
                  x & = e^{-\gamma t / 2} [A e^{\alpha t} + B e^{-\alpha t}]           \\
                    & = A e^{(\alpha - \gamma / 2) t} + B e^{-(\alpha + \gamma / 2) t}
                \end{align*}

                where $\alpha = \sqrt{\gamma^2 / 4 - \omega_0^2}$
        \end{itemize}

  \item \textbf{Critical damping}

        \begin{itemize}
          \item The motion is not oscillatory and returns as quickly as possible to the equilibrium position

          \item This occurs when $\gamma^2 / 4 = \omega_0^2$

          \item The general solution is \[x = A e^{-\gamma t / 2} + B t e^{-\gamma t / 2}\]
        \end{itemize}

  \item The total energy of an underdamped system decreases over time \[E = E_0 e^{-\gamma t}\] where $E_0$ is the initial energy of the system

  \item The \textbf{decay time} or \textbf{time constant} of the system $\tau = 1 / \gamma$ is the time it takes for its energy to decrease by a factor of $e$

  \item The \textbf{quality factor} of a harmonic oscillator is a dimensionless value that gives a measure of the degree of damping \[Q = \frac{\omega_0}{\gamma}\] where large values indicate little damping and small values indicate more damping

\item The quality factor can also be used as a measure of fraction of energy lost (i.e. $\Delta E / E$) per cycle $2 \pi / Q$ or per radian $1 / Q$
\end{itemize}

\section{Forced Oscillations}

\begin{itemize}
  \item The equation of motion for an undamped forced harmonic oscillator is \[m \frac{d^2 x}{d t^2} + k x = F_0 \cos \omega t\] the general solution of which is \[x = A(\omega) \cos (\omega t - \delta)\] where \[A(\omega) = \frac{F_0}{k (1 - \omega^2 / \omega_0^2)} \text{ and } \delta = 0\] for $\omega < \omega_0$ and \[A(\omega) = -\frac{F_0}{k (1 - \omega^2 / \omega_0^2)} \text{ and } \delta = \pi\] for $\omega > \omega_0$

  \item From the above it can be seen that:

        \begin{itemize}
          \item $A(\omega) \rightarrow F_0 / k$ as $\omega \rightarrow 0$

          \item $A(\omega) \rightarrow \infty$ as $\omega \rightarrow \omega_0$

          \item $A(\omega) \rightarrow 0$ as $\omega \rightarrow \infty$
        \end{itemize}

  \item The equation of motion for a damped forced harmonic oscillator is \[\frac{d^2 x}{d t^2} + \gamma \frac{d x}{d t} + \omega_0^2 x = \frac{F_0}{m} \cos \omega t\] where $\gamma = b / m$ and $\omega_0^2 = k / m$ the general solution of which is \[x = A(\omega) \cos (\omega t - \delta)\] where \[A(\omega) = \frac{F_0}{m [(\omega_0^2 - \omega^2)^2 + \omega^2 \gamma^2]^{1 / 2}}\] and \[\delta = \arctan \frac{\omega \gamma}{\omega_0^2 - \omega^2}\]

  \item From the above it can be seen that:

        \begin{itemize}
          \item $A(\omega) \rightarrow F_0 / k$ as $\omega \rightarrow 0$

          \item $A(\omega) \rightarrow F_0 \omega_0 / k \gamma$ as $\omega \rightarrow \omega_0$

          \item $A(\omega) \rightarrow 0$ as $\omega \rightarrow \infty$
        \end{itemize}

  \item $A(\omega)$ is maximised when its denominator is minimised, leading to \[\omega_\text{max} = \omega_0 (1 - \gamma^2 / 2 \omega_0^2)^{1 / 2}\] and thus \[A_\text{max} = \frac{F_0 \omega_0 / \gamma}{k (1 - \gamma^2 / 4 \omega_0^2)^{1 / 2}}\]

  \item The power absorbed by a damped oscillator to sustain its motion is exactly equal to the rate at which the energy is dissipated, i.e.

        \begin{align*}
          P(t) & = b v(t) \times v(t)                      \\
               & = b [v(t)]^2                              \\
               & = v [v_0(t)]^2 \sin^2 (\omega t - \delta)
        \end{align*}

  \item The average power absorbed over one cycle is \[\overline{P}(\omega) = \frac{b[v_0(\omega)]^2}{2} = \frac{\omega^2 F_0^2 \gamma}{2 m [(\omega_0^2 - \omega^2)^2 + \omega^2 \gamma^2]}\]

  \item From the above it can be seen that:

        \begin{itemize}
          \item $\overline{P}(\omega) \rightarrow 0$ as $\omega \rightarrow 0$

          \item $\overline{P}(\omega) \rightarrow F_0^2 / 2 m \gamma$ as $\omega \rightarrow \omega_0$

          \item $\overline{P}(\omega) \rightarrow 0$ as $\omega \rightarrow \infty$
        \end{itemize}

  \item The \textbf{power resonance curve} of an oscillating system graphs the average power absorbed by the system over a cycle to the driving frequency

  \item The \textbf{full width at half height} of a power resonance curve is the width of the curve at height $P_\text{max} / 2$, is a measure of the sharpness of the system's response to an applied force, and is equal to $\omega_\text{fwhh} = \gamma = \omega_0 / Q$

  \item From the above it can be seen that \[Q = \frac{\omega_0}{\gamma} = \frac{\omega_0}{\omega_\text{fwhh}}\]

  \item A resonance circuit can be used to amplify AC signals around a particular frequency by the $Q$-factor of the circuit — this makes them useful in radio receivers to tune a specific frequency

  \item When a driving force is first applied to a system, the system will be inclined to oscillate at its natural frequency $\omega_0$. The behaviour of the system is described by the sum of two oscillations, one at frequency $\omega_0$ and the other at $\omega$. Eventually the $\omega_0$ oscillations die out leaving the system in its \textbf{steady state} condition. The initial behaviour is reffered to as its \textbf{transient response}.

  \item The equation of motion for damped forced oscillations is the second-order nonhomogeneous linear differential equation \[\frac{d^2 x}{d t^2} + \gamma \frac{d x}{d t} + \omega_0^2 x = \frac{F_0}{m} \cos \omega t.\] The oscillations at frequency $\omega_0$ present only during the transient response are described by the complementary function of this equation, i.e. a fundamental set of solutions of the associated homogeneous differential equation, and the oscillations at frequency $\omega$ are described by a particular solution of this equation.

  \item If $z = x + y i$, the \textbf{complex conjugate} of $z$ is $z^* = x - y i$

  \item The product of a complex number with its conjugate is $z z^* = z^2 + y^2$

  \item The \textbf{modulus} of a complex number is defined as $|z| = \sqrt{z z^*} = \sqrt{x^2 + y^2}$

  \item Division of complex numbers can be performed like so \[\frac{z_1}{z_2} = \frac{z_1 z_2^*}{z_2 z_2^*} = \frac{(x_1 + i y_1) (x_2 - i y_2)}{x_2^2 + y_2^2} = \frac{(x_1 x_2 + y_1 y_2) + i (x_2 y_1 - x_1 y_2)}{x_2^2 + y_2^2}\]

  \item An \textbf{Argand diagram} is two-dimensional graph where the $x$-axis is used as the real axis and the $y$-axis is used as the imaginary axis

  \item Using \textbf{Euler's formula} \[e^{i x} = \cos x + i \sin x\] a complex number can be represented as \[z = x + i y = r (\cos \theta + i \sin \theta) = z e^{i \theta}\] where $r$ is the modulus $|z|$ and $\theta$ is the angle of $z$ from the positive $x$-axis known as its \textbf{argument}

  \item Multiplication of complex numbers is equivalent to rotation and scaling in the complex plane \[r_1 e^{i \theta} \times r_2 e^{i \phi} = r_1 r_2 e^{i (\theta + \phi)}\]

  \item Phasor diagrams can be represented on the complex plane with phasors as complex numbers $z = A e^{i (\omega t + \phi)}$ and their projection onto the $x$-axis as their real components

  \item Differentiation with respect to time of a complex phasor is equivalent to multiplication by $i \omega$
\end{itemize}

\section{Coupled Oscillators}

\begin{itemize}
  \item Systems of two or more coupled oscillators can oscillate in multiple ways called \textbf{normal modes}, each with its own frequency called the \textbf{normal frequency}

  \item In a normal mode, each oscillator oscillates at the same frequency

  \item Without damping, once a system is in a normal mode it stays there

  \item The equations of motion of a system of coupled oscillators are a system of differential equations and thus the movements of the oscillators are described by a linear combination of the solutions of that system

  \item Those equations of motion are often intertwined and involve multiple variables, e.g. the positions of two pendulums $x_1$ and $x_2$. It's possible to introduce new variables called \textbf{normal coordinates} that result in independent solutions in one variable, e.g. $q_1 = x_a + x_b$ and $q_2 = x_a - x_b$

  \item Energy never flows from one normal mode to another

  \item In general it's difficult to determine the normal modes of the system a priori. A more general approach is to take advantage of the knowledge that in a normal mode all oscillators will oscillate at the same frequency and:

        \begin{enumerate}
          \item assume solutions of the form $A \cos \omega t$, $B \cos \omega t$, etc.,

          \item subtitute them into the equations of motion, and

          \item rearrange to remove the constants $A$, $B$, etc. and solve for $\omega$
        \end{enumerate}

  \item There are as many normal modes as there are degrees of freedom in the system, e.g. two coupled oscillators moving in one dimension have 2 normal modes, three coupled oscillators moving on two dimensions have 6 normal modes, etc.

  \item Coupled oscillators experience large amplitude oscillations when the driving frequency is close to the normal frequency

  \item The motion of driven coupled oscillators may be solved in a similar fashion to their free moving counterparts:

        \begin{enumerate}
          \item Determine the equations of motion for the oscillators

          \item Combine the equations in such a way that the normal coordinates are evident

          \item Conver the equations to use normal coordinates

          \item Solve the resulting second-order nonhomogeneous linear differential equations by assuming solutions of the form $C_1 \cos \omega_1 t$, etc.

          \item Convert the solutions back from normal coordinates
        \end{enumerate}

  \item Oscillations that occur along the line connecting oscillators are called \textbf{longitudinal oscillations}

  \item Oscillations that occur perpendicular to the line connecting oscillators are called \textbf{transverse oscillations}
\end{itemize}

\section{Travelling Waves}

\begin{itemize}
  \item The equation of a wave moving at velocity $v$ in the positive $x$ direction is of the form \[y(x, t) = f(x - v t)\]

  \item The equation of a wave moving at velocity $v$ in the negative $x$ direction is of the form \[y(x, t) = g(x + v t)\]

  \item The general form of any transverse wave motion can be written as \[y = f(x - v t) + g(x + v t)\]

  \item The \textbf{wavelength} $\lambda$ of a wave is the length of one complete pattern of the wave, e.g. between two maxima

  \item When the displacements of a wave lie in a single plane, e.g. the $x$-$y$ plane, it is said to be \textbf{linearly polarised}

  \item The frequency $f$, wavelength $\lambda$, and velocity $v$ of a wave are related by the expression \[f \lambda = v\]

  \item The frequency $f$ and period $T$ of a wave are related by the expression \[f = \frac{1}{T}\]

  \item The \textbf{wavenumber} of a wave \[k = \frac{2 \pi}{\lambda}\] is a measure of radians per unit distance

  \item The \textbf{wave equation} in one dimension is \[\frac{\partial^2 \psi}{\partial t^2} = v^2 \frac{\partial^2 \psi}{\partial x^2}\] and its general solution is \[\psi = f(x - v t) + g(x - v t)\]

  \item An intuition for the wave equation is ``the acceleration experienced by a point on the wave at a particular time is a constant multiple of the curvature of the wave at that point''

  \item The velocity of a wave in a taut string $v$ is \[v = \sqrt{\frac{T}{\mu}}\] where $T$ is the tension in the string and $\mu$ is its mass per unit length

  \item The total kinetic and potential energies contained within one wavelength of a sinusoidal wave \[y = A \sin (k x - \omega t)\] are equal and have the value \[\frac{1}{4} \mu \omega^2 A^2 \lambda\] meaning the total energy is \[\frac{1}{2} \mu \omega^2 A^2 \lambda\]

  \item The power of a sinusoidal wave, i.e. the energy carried by the wave past a point per unit time, is \[P = \frac{1}{2} \mu \omega^2 A^2 v\]

  \item When a wave encounters a discontinuity, some fraction of the wave is transmitted and the remaining fraction is reflected

  \item The \textbf{incident wave} is the original wave

  \item On either side of a discontinuity, the displacement and the gradient must be the same at all times

\item The ratio of the transmitted amplitude to the incident amplitude is \[\frac{A_2}{A_1} = \frac{2 k_1}{k_1 + k_2} = \frac{2 \sqrt{\mu_1}}{\sqrt{\mu_1} + \sqrt{\mu_2}} = T_{12}\] where $k_n$ is the wave number of each medium, $\mu_n$ is the mass per unit length of each medium, and $T_{12}$ is called the \textbf{transmission coefficient of amplitude}

  \item The transmission coefficient of amplitude is a positive value in the range $(0, 2)$, i.e. the transmitted wave is always in phase with the incident wave

  \item The ratio of the reflected amplitude to the incident amplitude is \[\frac{B_1}{A_1} = \frac{k_1 - k_2}{k_1 + k_2} = \frac{\sqrt{\mu_1} - \sqrt{\mu_2}}{\sqrt{\mu_1} + \sqrt{\mu_2}} = R_{12}\] where $R_{12}$ is called the \textbf{reflection coefficient of amplitude}

  \item The reflection coefficient of amplitude is a value in the range $(-1, 1)$, i.e. if $\mu_1 > \mu_2$ the reflected wave will be in phase with the incident wave and if $\mu_1 < \mu_2$ it will be $\ang{90}$ out of phase

  \item The wave equation in two dimensions is \[\frac{\partial^2 \psi}{\partial t^2} = v^2 \left( \frac{\partial^2 \psi}{\partial x^2} + \frac{\partial^2 \psi}{\partial y^2} \right)\]

\item For a sinusoidal wave travelling in two dimensions, the solution to the wave equation is \[z(x, y, t) = A \cos (k_1 x + k_2 y - \omega t)\] which is a planar wave with velocity \[v = \sqrt{\frac{S}{\sigma}} = \frac{\omega}{\sqrt{k_1^2 + k_2^2}} = \frac{\omega}{k}\] and angle from the positive $x$-axis $\phi$ where \[\tan \phi = -\frac{k_1}{k_2}\]

  \item The wave equation for two-dimensional waves of circular symmetry is \[\frac{\partial^2 z}{\partial r^2} + \frac{1}{r} \frac{\partial z}{\partial r} = \frac{1}{v^2} \frac{\partial^2 z}{\partial t^2}\] and its solutions involve Bessel functions but for large $r$ it can be simplified to \[\frac{\partial^2 z}{\partial r^2} = \frac{1}{v^2} \frac{\partial^2 z}{\partial t^2}\] which has the same form as the one-dimensional wave equation with solutions of the form \[z(r, t) = A \cos (k r - \omega t)\]
\end{itemize}

\section{Standing Waves}

\begin{itemize}
  \item In a standing wave, a \textbf{node} is a point whose displacement is $0$ at all times and an \textbf{antinode} is a point taht experiences that maximum displacement

  \item In a standing wave, each individual particle undergoes SHM about its equilibrium position but different particules have different amplitudes

  \item The general equation for a standing wave on a string is \[y_n(x, t) = A_n \sin \left( \frac{n \pi}{L} x \right) \cos \omega_n t\] where \[\omega_n = \frac{n \pi v}{L}\] and $n = 1, \,2, \,3, \,\ldots$ corresponds to a different standing wave pattern or \textbf{mode}

  \item The first normal mode can also be called the \textbf{fundamental mode}

  \item The period of a standing wave is \[T = \frac{2 \pi}{\omega_n} = \frac{2 L}{n v}\]

  \item The wavelength of a standing wave is \[\lambda_n = \frac{2 L}{n}\]

  \item The wavenumber of a standing wave is \[k_n = \frac{n \pi}{L}\]

  \item Using the definition of the wavenumber of a standing wave, the general equation can be rewritten \[y_n(x, t) = A_n \sin k_n x \cos \omega t\]

  \item A standing wave is a superposition of two travelling waves of equal frequency and amplitude travelling in opposite directions

  \item The total energy of a string vibrating in the $n$-th mode is \[E_n = \frac{1}{4} \mu L A_n^2 \omega_n^2\]

  \item Standing waves are the normal modes of a vibrating string

  \item The superposition principle states that if $y_1(x, t)$ and $y_2(x, t)$ are both solutions to the wave equation, then so is $c_1 y_1(x, t) + c_2 y_2(x, t)$

  \item In general, the motion of a vibrating string can be expressed as a superposition of normal modes of the string, i.e. \[y(x, t) = \sum_n y_n(x, t) = \sum_n A_n \sin \left( \frac{n \pi}{L} x \right) \cos \omega_n t\]

  \item Any function $f(x)$ where $f(0) = f(L) = 0$ can be written as a superposition of sine waves \[f(x) = \sum_n A_n \sin \left( \frac{n \pi}{L} x \right)\] where \[A_n = \frac{2}{L} \int_0^L dx \sin \left( \frac{n \pi}{L} x \right) f(x), \,n = 1, \,2, \,\ldots\]

  \item This works because $\sin \frac{m \pi}{L} x$ and $\sin \frac{n \pi}{L} x$ are orthogonal when $m \ne n$, i.e. \[\int_0^L dx \sin \left( \frac{m \pi}{L} x \right) \sin \left( \frac{n \pi}{L} x \right) = 0, \,m \ne n\] but \[\int_0^L dx \sin^2 \left( \frac{n \pi}{L} x \right) = \frac{L}{2}\] so integrating the product of $\sin \left( \frac{n \pi}{L} x \right)$ and $f(x)$ in $A_n$ effectively ``picks out'' the amplitude of the $n$-th normal mode

  \item The total energy of a vibrating string expressed as a superposition of normal modes is given by \[E = \sum_n E_n = \sum_n \frac{1}{4} \mu L A_n^2 \omega_n^2 = \frac{1}{4} \mu L \sum_n A_n^2 \omega_n^2\]
\end{itemize}

\section{Interference and Diffraction of Waves}

\begin{itemize}
  \item If two waves are emitted from the same source, travel different distances, and recombine, the difference in distance travelled $s$ determines the shape of the resulting wave:

        \begin{itemize}
          \item If $s$ is an integral multiple of $\lambda$ the waves are said to be \textbf{in phase} and there is \textbf{constructive interference} and the amplitude of the resulting wave is double the amplitude of the original waves \[s = n \lambda, \,n = 0, \,\pm 1, \,\pm 2, \,\ldots\] or \[\phi = 2 n \pi, \,n = 0, \,\pm 1, \,\pm 2, \,\ldots\]

          \item If $s$ is an odd integral multiple of $\lambda / 2$ the waves are said to be \textbf{out of phase} and there is \textbf{destructive interference} and the amplitude of the resulting wave is $0$ \[s = \left( n + \frac{1}{2} \right) \lambda, \,n = 0, \,\pm 1, \,\pm 2, \,\ldots\] or \[\phi = (2 n + 1) \pi, \,n = 0, \,\pm 1, \,\pm 2, \,\ldots\]

          \item Other values of $s$ will result in behaviour somewhere between these two extremes
        \end{itemize}

  \item \textbf{Huygen's principle} states that each point on a primary wavefront acts as a source of secondary wavelets such that the wavefront at some later time is the envelope of these wavelets, i.e.

        \begin{enumerate}
          \item For each point on a primary wavefront draw an arc of radius $v \Delta t$ in the direction of travel where $v$ is the velocity of the wave

          \item The envelope of the resulting arcs is the shape of the wave at time $t + \Delta t$
        \end{enumerate}

  \item Two waves are said to be \textbf{coherent} if their frequency and waveform are identical

  \item \textbf{Young's double-slit experiment} works via \textbf{division of wavefront} where a single source wave is sent through two narrow slits and the resulting waves interfere with each other on a far away detector

  \item Interference can also occur via \textbf{division of amplitude} where a single source wave is split via e.g. a semi-silvered mirror

  \item The bending or spreading of waves around obstances is called \textbf{diffraction}

  \item If a test point $P$ is sufficiently far from a slit causing diffraction that the secondary wavelets have become plane waves we have \textbf{Fraunhofer diffraction}

  \item If the source of the primary waves or a test point $P$ is too close to a slit causing diffraction that the curvature of the incoming or outgoing wavefronts must be taken into consideration we have \textbf{Fresnel diffraction}
\end{itemize}

\end{document}