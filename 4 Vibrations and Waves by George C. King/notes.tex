\documentclass{article}
\usepackage{amsmath} % For align*
\usepackage{enumitem} % For customisable list labels
\usepackage{graphicx} % For images
\usepackage{siunitx} % For units
\graphicspath{{./images/}}

\title{Vibrations and Waves by George C. King Notes}
\author{Chris Doble}
\date{April 2022}

\begin{document}

\maketitle

\tableofcontents

\section{Simple Harmonic Motion}

\begin{itemize}
  \item The equation of motion for a simple harmonic oscillator is \[\frac{d^2 x}{d t^2} = -\omega^2 x\] where \[\omega^2 = \frac{k}{m}\]

  \item The general solution of the equation of motion for a simple harmonic oscillator is \[x = A \cos (\omega t + \phi)\] or equivalently \[x = a \cos \omega t + b \sin \omega t\]

  \item The angular frequency $\omega$ is determined entirely by properties of the oscillator, e.g. its mass and spring coefficient

  \item The total energy of a harmonic oscillator is \[E = \frac{1}{2} k A^2\]

  \item Nearly all potential wells have a shape that is parabolic when sufficiently close to the equilibrium position, so most oscillating systems will oscillate with SHM when the amplitude of oscillation is small

  \item The vibrations of nuclei in a molecule can be modeled by SHM, but only a discrete set of vibrational energies is possible, namely \[\frac{1}{2} \hbar \omega, \, \frac{3}{2} \hbar \omega, \, \frac{5}{2} \hbar \omega, \, \ldots\] where $\hbar$ is Planck's constant divided by $2 \pi$

  \item The total energy of a system undergoing SHM is always given by an expression of the form \[E = \frac{1}{2} \alpha v^2 + \frac{1}{2} \beta x^2\] where $\alpha$ and $\beta$ are physical constants — if we obtain this equation during the analysis of a system we know we have SHM

  \item The equation of motion for a system described by the energy equation above is \[\frac{d^2 x}{d t^2} = -\frac{\beta}{\alpha} x\]
\end{itemize}

\section{The Damped Harmonic Oscillator}

\begin{itemize}
  \item The equation of motion of a damped harmonic oscillator is

        \begin{align*}
          F = m a                                                     & = -k x - b v \\
          m \frac{d^2 x}{d t^2} + b \frac{d x}{d t} + k x             & = 0          \\
          \frac{d^2 x}{d t^2} + \gamma \frac{d x}{d t} + \omega_0^2 x & = 0
        \end{align*}

        where $\gamma = b / m$ and $\omega_0^2 = k / m$

  \item $\omega_0$ is known as the \textbf{natural frequency of oscillation}, i.e. the oscillation frequency if there were no damping

  \item \textbf{Light damping / underdamped}

        \begin{itemize}
          \item The motion is still oscillatory but the amplitude decreases expontentially

          \item This occurs when $\gamma^2 / 4 < \omega_0^2$

          \item The general solution is \[x = A_0 e^{-\gamma t / 2} \cos (\omega t + \phi)\] where $A_0$ is the initial amplitude

          \item Successive maxima decrease by the same fractional amount \[\frac{A_n}{A_{n + 1}} = e^{\gamma T / 2}\]

          \item The natural logarithm of $A_n / A_{n + 1}$ is called the \textbf{logarithmic decrement} \[\ln \left( \frac{A_n}{A_{n + 1}} \right) = \frac{\gamma T}{2}\]
        \end{itemize}

  \item \textbf{Heavy damping / overdamped}

        \begin{itemize}
          \item The motion is not oscillatory and returns sluggishly to the equilibrium position

          \item This occurs when $\gamma^2 / 4 > \omega_0^2$

          \item The general solution is

                \begin{align*}
                  x & = e^{-\gamma t / 2} [A e^{\alpha t} + B e^{-\alpha t}]           \\
                    & = A e^{(\alpha - \gamma / 2) t} + B e^{-(\alpha + \gamma / 2) t}
                \end{align*}

                where $\alpha = \sqrt{\gamma^2 / 4 - \omega_0^2}$
        \end{itemize}

  \item \textbf{Critical damping}

        \begin{itemize}
          \item The motion is not oscillatory and returns as quickly as possible to the equilibrium position

          \item This occurs when $\gamma^2 / 4 = \omega_0^2$

          \item The general solution is \[x = A e^{-\gamma t / 2} + B t e^{-\gamma t / 2}\]
        \end{itemize}

  \item The total energy of an underdamped system decreases over time \[E = E_0 e^{-\gamma t}\] where $E_0$ is the initial energy of the system

  \item The \textbf{decay time} or \textbf{time constant} of the system $\tau = 1 / \gamma$ is the time it takes for its energy to decrease by a factor of $e$

  \item The \textbf{quality factor} of a harmonic oscillator is a dimensionless value that gives a measure of the degree of damping \[Q = \frac{\omega_0}{\gamma}\] where large values indicate little damping and small values indicate more damping

\item The quality factor can also be used as a measure of fraction of energy lost (i.e. $\Delta E / E$) per cycle $2 \pi / Q$ or per radian $1 / Q$
\end{itemize}

\section{Forced Oscillations}

\begin{itemize}
  \item The equation of motion for an undamped forced harmonic oscillator is \[m \frac{d^2 x}{d t^2} + k x = F_0 \cos \omega t\] the general solution of which is \[x = A(\omega) \cos (\omega t - \delta)\] where \[A(\omega) = \frac{F_0}{k (1 - \omega^2 / \omega_0^2)} \text{ and } \delta = 0\] for $\omega < \omega_0$ and \[A(\omega) = -\frac{F_0}{k (1 - \omega^2 / \omega_0^2)} \text{ and } \delta = \pi\] for $\omega > \omega_0$

  \item From the above it can be seen that:

        \begin{itemize}
          \item $A(\omega) \rightarrow F_0 / k$ as $\omega \rightarrow 0$

          \item $A(\omega) \rightarrow \infty$ as $\omega \rightarrow \omega_0$

          \item $A(\omega) \rightarrow 0$ as $\omega \rightarrow \infty$
        \end{itemize}

  \item The equation of motion for a damped forced harmonic oscillator is \[\frac{d^2 x}{d t^2} + \gamma \frac{d x}{d t} + \omega_0^2 x = \frac{F_0}{m} \cos \omega t\] where $\gamma = b / m$ and $\omega_0^2 = k / m$ the general solution of which is \[x = A(\omega) \cos (\omega t - \delta)\] where \[A(\omega) = \frac{F_0}{m [(\omega_0^2 - \omega^2)^2 + \omega^2 \gamma^2]^{1 / 2}}\] and \[\delta = \arctan \frac{\omega \gamma}{\omega_0^2 - \omega^2}\]

  \item From the above it can be seen that:

        \begin{itemize}
          \item $A(\omega) \rightarrow F_0 / k$ as $\omega \rightarrow 0$

          \item $A(\omega) \rightarrow F_0 \omega_0 / k \gamma$ as $\omega \rightarrow \omega_0$

          \item $A(\omega) \rightarrow 0$ as $\omega \rightarrow \infty$
        \end{itemize}

  \item $A(\omega)$ is maximised when its denominator is minimised, leading to \[\omega_\text{max} = \omega_0 (1 - \gamma^2 / 2 \omega_0^2)^{1 / 2}\] and thus \[A_\text{max} = \frac{F_0 \omega_0 / \gamma}{k (1 - \gamma^2 / 4 \omega_0^2)^{1 / 2}}\]

  \item The power absorbed by a damped oscillator to sustain its motion is exactly equal to the rate at which the energy is dissipated, i.e.

        \begin{align*}
          P(t) & = b v(t) \times v(t)                      \\
               & = b [v(t)]^2                              \\
               & = v [v_0(t)]^2 \sin^2 (\omega t - \delta)
        \end{align*}

  \item The average power absorbed over one cycle is \[\overline{P}(\omega) = \frac{b[v_0(\omega)]^2}{2} = \frac{\omega^2 F_0^2 \gamma}{2 m [(\omega_0^2 - \omega^2)^2 + \omega^2 \gamma^2]}\]

  \item From the above it can be seen that:

        \begin{itemize}
          \item $\overline{P}(\omega) \rightarrow 0$ as $\omega \rightarrow 0$

          \item $\overline{P}(\omega) \rightarrow F_0^2 / 2 m \gamma$ as $\omega \rightarrow \omega_0$

          \item $\overline{P}(\omega) \rightarrow 0$ as $\omega \rightarrow \infty$
        \end{itemize}

  \item The \textbf{power resonance curve} of an oscillating system graphs the average power absorbed by the system over a cycle to the driving frequency

  \item The \textbf{full width at half height} of a power resonance curve is the width of the curve at height $P_\text{max} / 2$, is a measure of the sharpness of the system's response to an applied force, and is equal to $\omega_\text{fwhh} = \gamma = \omega_0 / Q$

  \item From the above it can be seen that \[Q = \frac{\omega_0}{\gamma} = \frac{\omega_0}{\omega_\text{fwhh}}\]

  \item A resonance circuit can be used to amplify AC signals around a particular frequency by the $Q$-factor of the circuit — this makes them useful in radio receivers to tune a specific frequency

  \item When a driving force is first applied to a system, the system will be inclined to oscillate at its natural frequency $\omega_0$. The behaviour of the system is described by the sum of two oscillations, one at frequency $\omega_0$ and the other at $\omega$. Eventually the $\omega_0$ oscillations die out leaving the system in its \textbf{steady state} condition. The initial behaviour is reffered to as its \textbf{transient response}.

  \item The equation of motion for damped forced oscillations is the second-order nonhomogeneous linear differential equation \[\frac{d^2 x}{d t^2} + \gamma \frac{d x}{d t} + \omega_0^2 x = \frac{F_0}{m} \cos \omega t.\] The oscillations at frequency $\omega_0$ present only during the transient response are described by the complementary function of this equation, i.e. a fundamental set of solutions of the associated homogeneous differential equation, and the oscillations at frequency $\omega$ are described by a particular solution of this equation.

  \item If $z = x + y i$, the \textbf{complex conjugate} of $z$ is $z^* = x - y i$

  \item The product of a complex number with its conjugate is $z z^* = z^2 + y^2$

  \item The \textbf{modulus} of a complex number is defined as $|z| = \sqrt{z z^*} = \sqrt{x^2 + y^2}$

  \item Division of complex numbers can be performed like so \[\frac{z_1}{z_2} = \frac{z_1 z_2^*}{z_2 z_2^*} = \frac{(x_1 + i y_1) (x_2 - i y_2)}{x_2^2 + y_2^2} = \frac{(x_1 x_2 + y_1 y_2) + i (x_2 y_1 - x_1 y_2)}{x_2^2 + y_2^2}\]

  \item An \textbf{Argand diagram} is two-dimensional graph where the $x$-axis is used as the real axis and the $y$-axis is used as the imaginary axis

  \item Using \textbf{Euler's formula} \[e^{i x} = \cos x + i \sin x\] a complex number can be represented as \[z = x + i y = r (\cos \theta + i \sin \theta) = z e^{i \theta}\] where $r$ is the modulus $|z|$ and $\theta$ is the angle of $z$ from the positive $x$-axis known as its \textbf{argument}

  \item Multiplication of complex numbers is equivalent to rotation and scaling in the complex plane \[r_1 e^{i \theta} \times r_2 e^{i \phi} = r_1 r_2 e^{i (\theta + \phi)}\]

  \item Phasor diagrams can be represented on the complex plane with phasors as complex numbers $z = A e^{i (\omega t + \phi)}$ and their projection onto the $x$-axis as their real components

  \item Differentiation with respect to time of a complex phasor is equivalent to multiplication by $i \omega$
\end{itemize}

\section{Coupled Oscillators}

\begin{itemize}
  \item Systems of two or more coupled oscillators can oscillate in multiple ways called \textbf{normal modes}, each with its own frequency called the \textbf{normal frequency}

  \item In a normal mode, each oscillator oscillates at the same frequency

  \item Without damping, once a system is in a normal mode it stays there

  \item The equations of motion of a system of coupled oscillators are a system of differential equations and thus the movements of the oscillators are described by a linear combination of the solutions of that system

  \item Those equations of motion are often intertwined and involve multiple variables, e.g. the positions of two pendulums $x_1$ and $x_2$. It's possible to introduce new variables called \textbf{normal coordinates} that result in independent solutions in one variable, e.g. $q_1 = x_a + x_b$ and $q_2 = x_a - x_b$

  \item Energy never flows from one normal mode to another

  \item In general it's difficult to determine the normal modes of the system a priori. A more general approach is to take advantage of the knowledge that in a normal mode all oscillators will oscillate at the same frequency and:

        \begin{enumerate}
          \item assume solutions of the form $A \cos \omega t$, $B \cos \omega t$, etc.,

          \item subtitute them into the equations of motion, and

          \item rearrange to remove the constants $A$, $B$, etc. and solve for $\omega$
        \end{enumerate}

  \item There are as many normal modes as there are degrees of freedom in the system, e.g. two coupled oscillators moving in one dimension have 2 normal modes, three coupled oscillators moving on two dimensions have 6 normal modes, etc.

  \item Coupled oscillators experience large amplitude oscillations when the driving frequency is close to the normal frequency

  \item The motion of driven coupled oscillators may be solved in a similar fashion to their free moving counterparts:

        \begin{enumerate}
          \item Determine the equations of motion for the oscillators

          \item Combine the equations in such a way that the normal coordinates are evident

          \item Conver the equations to use normal coordinates

          \item Solve the resulting second-order nonhomogeneous linear differential equations by assuming solutions of the form $C_1 \cos \omega_1 t$, etc.

          \item Convert the solutions back from normal coordinates
        \end{enumerate}

  \item Oscillations that occur along the line connecting oscillators are called \textbf{longitudinal oscillations}

  \item Oscillations that occur perpendicular to the line connecting oscillators are called \textbf{transverse oscillations}
\end{itemize}

\end{document}