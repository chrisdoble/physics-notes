\documentclass{article}
\usepackage{amsmath} % For align*
\usepackage{enumitem} % For customisable list labels
\usepackage{graphicx} % For images
\usepackage{siunitx} % For units
\graphicspath{{./images/}}

\title{Vibrations and Waves by George C. King Notes}
\author{Chris Doble}
\date{April 2022}

\begin{document}

\maketitle

\tableofcontents

\section{Simple Harmonic Motion}

\begin{itemize}
  \item The equation of motion for a simple harmonic oscillator is \[\frac{d^2 x}{d t^2} = -\omega^2 x\] where \[\omega^2 = \frac{k}{m}\]

  \item The general solution of the equation of motion for a simple harmonic oscillator is \[x = A \cos (\omega t + \phi)\] or equivalently \[x = a \cos \omega t + b \sin \omega t\]

  \item The angular frequency $\omega$ is determined entirely by properties of the oscillator, e.g. its mass and spring coefficient

  \item The total energy of a harmonic oscillator is \[E = \frac{1}{2} k A^2\]

  \item Nearly all potential wells have a shape that is parabolic when sufficiently close to the equilibrium position, so most oscillating systems will oscillate with SHM when the amplitude of oscillation is small

  \item The vibrations of nuclei in a molecule can be modeled by SHM, but only a discrete set of vibrational energies is possible, namely \[\frac{1}{2} \hbar \omega, \, \frac{3}{2} \hbar \omega, \, \frac{5}{2} \hbar \omega, \, \ldots\] where $\hbar$ is Planck's constant divided by $2 \pi$

  \item The total energy of a system undergoing SHM is always given by an expression of the form \[E = \frac{1}{2} \alpha v^2 + \frac{1}{2} \beta x^2\] where $\alpha$ and $\beta$ are physical constants — if we obtain this equation during the analysis of a system we know we have SHM

  \item The equation of motion for a system described by the energy equation above is \[\frac{d^2 x}{d t^2} = -\frac{\beta}{\alpha} x\]
\end{itemize}

\end{document}