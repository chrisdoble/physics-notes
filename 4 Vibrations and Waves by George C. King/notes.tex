\documentclass{article}
\usepackage{amsmath} % For align*
\usepackage{enumitem} % For customisable list labels
\usepackage{graphicx} % For images
\usepackage{siunitx} % For units
\graphicspath{{./images/}}

\title{Vibrations and Waves by George C. King Notes}
\author{Chris Doble}
\date{April 2022}

\begin{document}

\maketitle

\tableofcontents

\section{Simple Harmonic Motion}

\begin{itemize}
  \item The equation of motion for a simple harmonic oscillator is \[\frac{d^2 x}{d t^2} = -\omega^2 x\] where \[\omega^2 = \frac{k}{m}\]

  \item The general solution of the equation of motion for a simple harmonic oscillator is \[x = A \cos (\omega t + \phi)\] or equivalently \[x = a \cos \omega t + b \sin \omega t\]

  \item The angular frequency $\omega$ is determined entirely by properties of the oscillator, e.g. its mass and spring coefficient

  \item The total energy of a harmonic oscillator is \[E = \frac{1}{2} k A^2\]

  \item Nearly all potential wells have a shape that is parabolic when sufficiently close to the equilibrium position, so most oscillating systems will oscillate with SHM when the amplitude of oscillation is small

  \item The vibrations of nuclei in a molecule can be modeled by SHM, but only a discrete set of vibrational energies is possible, namely \[\frac{1}{2} \hbar \omega, \, \frac{3}{2} \hbar \omega, \, \frac{5}{2} \hbar \omega, \, \ldots\] where $\hbar$ is Planck's constant divided by $2 \pi$

  \item The total energy of a system undergoing SHM is always given by an expression of the form \[E = \frac{1}{2} \alpha v^2 + \frac{1}{2} \beta x^2\] where $\alpha$ and $\beta$ are physical constants — if we obtain this equation during the analysis of a system we know we have SHM

  \item The equation of motion for a system described by the energy equation above is \[\frac{d^2 x}{d t^2} = -\frac{\beta}{\alpha} x\]
\end{itemize}

\section{The Damped Harmonic Oscillator}

\begin{itemize}
  \item The equation of motion of a damped harmonic oscillator is

        \begin{align*}
          F = m a                                                     & = -k x - b v \\
          m \frac{d^2 x}{d t^2} + b \frac{d x}{d t} + k x             & = 0          \\
          \frac{d^2 x}{d t^2} + \gamma \frac{d x}{d t} + \omega_0^2 x & = 0
        \end{align*}

        where $\gamma = b / m$ and $\omega_0^2 = k / m$

  \item $\omega_0$ is known as the \textbf{natural frequency of oscillation}, i.e. the oscillation frequency if there were no damping

  \item \textbf{Light damping / underdamped}

        \begin{itemize}
          \item The motion is still oscillatory but the amplitude decreases expontentially

          \item This occurs when $\gamma^2 / 4 < \omega_0^2$

          \item The general solution is \[x = A_0 e^{-\gamma t / 2} \cos (\omega t + \phi)\] where $A_0$ is the initial amplitude

          \item Successive maxima decrease by the same fractional amount \[\frac{A_n}{A_{n + 1}} = e^{\gamma T / 2}\]

          \item The natural logarithm of $A_n / A_{n + 1}$ is called the \textbf{logarithmic decrement} \[\ln \left( \frac{A_n}{A_{n + 1}} \right) = \frac{\gamma T}{2}\]
        \end{itemize}

  \item \textbf{Heavy damping / overdamped}

        \begin{itemize}
          \item The motion is not oscillatory and returns sluggishly to the equilibrium position

          \item This occurs when $\gamma^2 / 4 > \omega_0^2$

          \item The general solution is

                \begin{align*}
                  x & = e^{-\gamma t / 2} [A e^{\alpha t} + B e^{-\alpha t}]           \\
                    & = A e^{(\alpha - \gamma / 2) t} + B e^{-(\alpha + \gamma / 2) t}
                \end{align*}

                where $\alpha = \sqrt{\gamma^2 / 4 - \omega_0^2}$
        \end{itemize}

  \item \textbf{Critical damping}

        \begin{itemize}
          \item The motion is not oscillatory and returns as quickly as possible to the equilibrium position

          \item This occurs when $\gamma^2 / 4 = \omega_0^2$

          \item The general solution is \[x = A e^{-\gamma t / 2} + B t e^{-\gamma t / 2}\]
        \end{itemize}

  \item The total energy of an underdamped system decreases over time \[E = E_0 e^{-\gamma t}\] where $E_0$ is the initial energy of the system

  \item The \textbf{decay time} or \textbf{time constant} of the system $\tau = 1 / \gamma$ is the time it takes for its energy to decrease by a factor of $e$

  \item The \textbf{quality factor} of a harmonic oscillator is a dimensionless value that gives a measure of the degree of damping \[Q = \frac{\omega_0}{\gamma}\] where large values indicate little damping and small values indicate more damping

  \item The quality factor can also be used as a measure of the energy lost per cycle $2 \pi / Q$ or the energy lost per radian $1 / Q$
\end{itemize}

\end{document}