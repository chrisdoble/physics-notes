\documentclass{article}
\usepackage{amsmath} % For align*
\usepackage{amsfonts} % For open face letters
\usepackage{enumitem} % For customisable list labels
\usepackage{graphicx} % For images
\usepackage{siunitx} % For units
\graphicspath{{./images/}}

\setlist[enumerate, 1]{label={(\alph*)}}
\setlist[enumerate, 2]{label={(\roman*)}}

\title{Vibrations and Waves by George C. King Problems}
\author{Chris Doble}
\date{April 2022}

\begin{document}

\maketitle

\tableofcontents

\section{Simple Harmonic Motion}

\subsection{}

\begin{enumerate}
  \item

        \begin{enumerate}
          \item $T = \qty{4}{s}$

          \item $\omega = \frac{\pi}{2} \,\unit{rad/s}$

          \item $\omega^2 = \frac{k}{m} \Rightarrow k = m \omega^2 = \frac{\pi^2}{8} \,\unit{N/m}$
        \end{enumerate}
\end{enumerate}

\subsection{}

\begin{enumerate}
  \item

        \begin{align*}
          x            & = A \cos \omega t           \\
                       & = A \cos 2 \pi f t          \\
          v            & = -2 \pi f A \sin 2 \pi f t \\
          v_\text{max} & = 2 \pi f A                 \\
                       & = \qty{1.38}{m/s}
        \end{align*}

  \item

        \begin{align*}
          a            & = -4 \pi^2 f^2 A \cos 2 \pi f t \\
          a_\text{max} & = 4 \pi^2 f^2 A                 \\
                       & = \qty{3.82e3}{m/s^2}
        \end{align*}
\end{enumerate}

\subsection{}

\begin{align*}
  a_\text{max}  & \le g                          \\
  4 \pi^2 f^2 A & \le g                          \\
  f             & \le \sqrt{\frac{g}{4 \pi^2 A}} \\
                & \le \qty{1.11}{Hz}
\end{align*}

\subsection{}

\begin{enumerate}
  \item \[\frac{U}{E} = \frac{\frac{1}{2} k \left( \frac{1}{2} A \right)^2}{\frac{1}{2} k A^2} = \frac{1}{4} \Rightarrow \frac{K}{E} = \frac{3}{4}\]

  \item

        \begin{enumerate}
          \item The total energy will increase by a factor of $4$

          \item The maximum velocity will increase by a factor of $2$

          \item The maximum acceleration will increase by a factor of $2$ and the period won't change
        \end{enumerate}
\end{enumerate}

\subsection{}

\begin{enumerate}
  \item $E = \frac{1}{2} m v^2 + \frac{1}{2} k x^2 = \qty{0.41}{J}$

  \item

        \begin{align*}
          E      & = \frac{1}{2} k A^2                             \\
          A      & = \sqrt{\frac{2 E}{k}}                          \\
                 & = \qty{4.5}{cm}                                 \\
          \omega & = \sqrt{\frac{k}{m}}                            \\
                 & = \sqrt{\frac{1600}{3}}                         \\
                 & = \frac{40}{\sqrt{3}}                           \\
                 & = \qty{23}{rad/s}                               \\
          x      & = A \cos (\omega t + \phi)                      \\
          \phi   & = \arccos \left( \frac{x}{A} \right) - \omega t \\
                 & = \qty{2.7}{rad}                                \\
          x      & = 0.045 \cos (23 t + 2.7) \,\unit{m}
        \end{align*}
\end{enumerate}

\subsection{}

Using the angular frequency of system (b) $\omega_b$ as the baseline, the angular frequency of system (a) $\omega_a$ is

\begin{align*}
  F = m a  & = -2 k x               \\
  a        & = -\frac{2 k}{m} x     \\
  \omega_a & = \sqrt{\frac{2 k}{m}} \\
           & = \sqrt{2} \omega_b
\end{align*}

and the angular frequency of system (c) $\omega_c$ is

\begin{align*}
  F = m a  & = -\frac{k}{2} x              \\
  a        & = -\frac{k}{2 m} x            \\
  \omega_c & = \sqrt{\frac{k}{2 m}}        \\
           & = \sqrt{\frac{1}{2}} \omega_b
\end{align*}

\subsection{}

\begin{enumerate}
  \item The test tube experiences a bouyancy force of $F = A g \rho x$ so its equation of motion is

        \begin{align*}
          F = m a & = -A g \rho x               \\
          a       & = -\frac{A g \rho}{m} x     \\
          \omega  & = \sqrt{\frac{A g \rho}{m}}
        \end{align*}

  \item The work done by the bouyancy force when moving from equilibrium to $x$ and thus the potential energy is

        \begin{align*}
          U & = \int_0^x A g \rho x' \,dx' \\
            & = \frac{1}{2} A g \rho x^2
        \end{align*}

        so the total energy of the system is \[E = \frac{1}{2} m v^2 + \frac{1}{2} A g \rho x^2\]
\end{enumerate}

\subsection{}

\[\unit{s} \propto \unit{kg}^\alpha \, \unit{m}^\beta \, \left( \unit{m/s^2} \right)^\gamma\] so $\alpha = 0$, $\beta = 1 / 2$, and $\gamma = -1 / 2$ meaning \[T \propto \sqrt{\frac{l}{g}}\]

\subsection{}

\begin{enumerate}
  \item

        \begin{align*}
          x            & = A \cos \sqrt{\frac{g}{l}} t                     \\
          v            & = -\sqrt{\frac{g}{l}} A \sin \sqrt{\frac{g}{l}} t \\
          v_\text{max} & = \sqrt{\frac{g}{l}} A                            \\
                       & = \qty{0.018}{m/s}
        \end{align*}

  \item The pendulum reaches its maximum speed at the bottom of its swing which occurs after a quarter cycle \[\frac{1}{4} T = \frac{1}{4} \frac{2 \pi}{\omega} = \frac{\pi}{2 \sqrt{g / l}} = \qty{0.43}{s}\]
\end{enumerate}

\subsection{}

\begin{align*}
  I \frac{d^2 \theta}{d t^2}                 & = \tau                           \\
  \frac{1}{3} M L^2 \frac{d^2 \theta}{d t^2} & = -k L \sin \theta L \cos \theta \\
  \frac{1}{3} M \frac{d^2 \theta}{d t^2}     & = -k \theta                      \\
  \frac{d^2 \theta}{d t^2}                   & = -\frac{3 k}{M} \theta          \\
  T                                          & = \frac{2 \pi}{\omega}           \\
                                             & = 2 \pi \sqrt{\frac{M}{3 k}}     \\
\end{align*}

\subsection{}

\begin{enumerate}
  \item

        \begin{align*}
          F = -\frac{d U}{d x} & = -\left( \frac{6 a}{x^7} - \frac{12 b}{x^{13}} \right) \\
          0                    & = \frac{12 b}{x^{13}} - \frac{6 a}{x^7}                 \\
                               & = \frac{12 b}{x^6} - 6 a                                \\
          6 a                  & = \frac{12 b}{x^6}                                      \\
          x^6                  & = \frac{2 b}{a}                                         \\
          x                    & = \left( \frac{2 b}{a} \right)^{1 / 6}
        \end{align*}
\end{enumerate}

\subsection{}

\begin{enumerate}
  \item

        \begin{align*}
          K & = \frac{1}{2} M v^2 + \int \,dK                                                            \\
            & = \frac{1}{2} M v^2 + \int_0^L \frac{1}{2} \frac{m}{L} \left( \frac{l}{L} v \right)^2 \,dl \\
            & = \frac{1}{2} M v^2 + \frac{1}{2} \frac{m v^2}{L^3} \int_0^L l^2 \,dl                      \\
            & = \frac{1}{2} M v^2 + \frac{1}{2} \frac{m v^2}{L^3} \frac{1}{3} L^3                        \\
            & = \frac{1}{2} M v^2 + \frac{1}{6} m v^2                                                    \\
            & = \frac{1}{2} (M + m / 3) v^2                                                              \\
          E & = K + U                                                                                    \\
            & = \frac{1}{2} (M + m / 3) v^2 + \frac{1}{2} k x^2
        \end{align*}

  \item

        \[\omega = \sqrt{\frac{k}{M + m / 3}}\]
\end{enumerate}

\subsection{}

\begin{enumerate}
  \item

        \begin{align*}
          K                 & = E - U                      \\
          \frac{1}{2} m v^2 & = U(A) - U(x)                \\
          v                 & = \sqrt{2 [U(A) - U(x)] / m}
        \end{align*}

  \item

        \begin{align*}
          T & = 4 \int_0^A \frac{dx}{v}                                              \\
            & = 4 \int_0^A \sqrt{\frac{m}{2 [U(A) - U(x)]}} \,dx                     \\
            & = 4 \sqrt{\frac{m}{2 U(A)}} \int_0^A \frac{dx}{\sqrt{1 - U(x) / U(A)}}
        \end{align*}

  \item

        \begin{align*}
          T & = 4 \sqrt{\frac{m}{2 \alpha A^n}} \int_0^A \frac{dx}{\sqrt{1 - (x / A)^n}}    \\
            & = 4 \sqrt{\frac{m}{2 \alpha A^n}} \int_0^1 \frac{A \,d \xi}{\sqrt{1 - \xi^n}} \\
            & = c A^{(n / 2) - 1}
        \end{align*}
\end{enumerate}

\section{The Damped Harmonic Oscillator}

\subsection{}

\begin{align*}
  \left( \frac{\gamma}{2} \right)^2 & = \omega_0^2                   \\
  \frac{b}{2 m}                     & = \sqrt{\frac{k}{m}}           \\
  b                                 & = 2 m \sqrt{\frac{k}{m}}       \\
                                    & = 2 m \sqrt{\frac{m g / x}{m}} \\
                                    & = 2 m \sqrt{\frac{g}{x}}       \\
                                    & = \qty{64}{kg / s}
\end{align*}

\subsection{}

\begin{align*}
  \frac{A_{n + 1}}{A_n} & = 0.90                             \\
  e^{-2.5 \gamma / 2}   & = 0.90                             \\
  e^{2.5 \gamma / 2}    & = \frac{1}{0.90}                   \\
  \frac{2.5 \gamma}{2}  & = \ln \frac{1}{0.90}               \\
  \gamma                & = \frac{2}{2.5} \ln \frac{1}{0.90} \\
                        & = \qty{8.43e-2}{s^{-1}}            \\
  F                     & = -b v                             \\
                        & = -(\num{4.21e-2}) v
\end{align*}

\subsection{}

After 10 cycles the amplitude has decreased by a factor of $5 / 3$. The energy of the system is proportional to the amplitude squared, so

\begin{align*}
  E(300)         & = E(0) e^{-300 / \tau}            \\
  e^{300 / \tau} & = \frac{E(0)}{E(300)}             \\
  \tau           & = \frac{300}{\ln [E(0) / E(300)]} \\
                 & = \frac{300}{\ln \frac{25}{9}}    \\
                 & = \qty{294}{s}                    \\
  Q              & = \omega_0 \tau                   \\
                 & = \frac{2 \pi \tau}{T}            \\
                 & = 61.5
\end{align*}

\subsection{}

\begin{align*}
  \frac{E(10 T)}{E_0} & = \frac{E_0 e^{-\gamma 10 T}}{E_0} \\
  \frac{1}{2}         & = e^{-\gamma 10 T}                 \\
  \frac{E(50 T)}{E_0} & = \frac{E_0 e^{-\gamma 50 T}}{E_0} \\
                      & = (e^{-\gamma 10 T})^5             \\
                      & = \left( \frac{1}{2} \right)^5     \\
                      & = \frac{1}{32}
\end{align*}

\subsection{}

\begin{enumerate}
  \item

        \begin{align*}
          Q_{0.01}      & = 310               \\
          \omega_{0.01} & = \qty{3.14}{rad/s} \\
          Q_{0.30}      & = 10.5              \\
          \omega_{0.30} & = \qty{3.14}{rad/s} \\
          Q_{1.00}      & = 3.14              \\
          \omega_{1.00} & = \qty{3.10}{rad/s}
        \end{align*}

        \setcounter{enumi}{2}
  \item

        \begin{align*}
          \gamma^2 / 4 & = \pi^2                                                  \\
          \gamma       & = 2 \pi                                                  \\
          x            & = A e^{-\pi t} + B t e^{-\pi t}                          \\
          A            & = \qty{10}{mm}                                           \\
          v            & = -10 \pi e^{-\pi t} + B e^{-\pi t} - \pi B t e^{-\pi t} \\
          0            & = -10 \pi + B                                            \\
          B            & = 10 \pi                                                 \\
          x            & = 10 e^{-\pi t} + 10 \pi t e^{-\pi t}
        \end{align*}
\end{enumerate}

\subsection{}

\begin{align*}
  \frac{\omega}{\omega_0} & = \frac{\omega_0 \sqrt{1 - 1 / 4 Q^2}}{\omega_0} \\
                          & = \sqrt{1 - 1 / 4 Q^2}                           \\
                          & = 1 - \frac{1 / 4 Q^2}{2} + \cdots               \\
                          & \approx 1 - \frac{Q^2}{8}
\end{align*}

\subsection{}

The amplitude of each pendulum decreases over time by a factor of

\begin{align*}
  \exp \left( -\frac{\gamma t}{2} \right) & = \exp \left( -\frac{b t}{2 m} \right)                              \\
                                          & = \exp \left( -\frac{b t}{2 \cdot \frac{4}{3} \pi r^3 \rho} \right) \\
                                          & = \exp \left( -\frac{3 b t}{8 \pi r^3 \rho} \right)                 \\
                                          & = \exp \left( -\frac{3 b t}{8 \pi r^3} \right)^{1 / \rho}.
\end{align*}

After 10 minutes the amplitude of oscillation of the aluminium pendulum has decreased to half of its initial value

\begin{align*}
  \exp \left( -\frac{225 b}{\pi r^3} \right)^{1 / \rho_a} & = \frac{1}{2}                         \\
  \exp \left( -\frac{225 b}{\pi r^3} \right)              & = \left( \frac{1}{2} \right)^{\rho_a}
\end{align*}

so the brass pendulum's amplitude of oscillation has decreased by a factor of

\begin{align*}
  \exp \left( -\frac{225 b}{\pi r^3} \right)^{1 / \rho_b} & = \left( \frac{1}{2} \right)^{\rho_a / \rho_b} \\
                                                          & = 0.802
\end{align*}

\subsection{}

\begin{enumerate}
  \item

        \begin{align*}
          x & = A \sin \omega t                                                                                                   \\
          v & = \omega A \cos \omega t                                                                                            \\
          a & = -\omega^2 A \sin \omega t                                                                                         \\
          E & = \int_0^T \frac{K e^2 a^2}{c^3} \,dt                                                                               \\
            & = \int_0^T \frac{K e^2 \omega^4 A^2 \sin^2 \omega t}{c^3} \,dt                                                      \\
            & = \frac{K e^2 \omega^4 A^2}{c^3} \int_0^{2 \pi / \omega} \sin^2 \omega t \,dt                                       \\
            & = \frac{K e^2 \omega^4 A^2}{c^3} \left[ \frac{t}{2} - \frac{1}{4 \omega} \sin 2 \omega t \right]_0^{2 \pi / \omega} \\
            & = \frac{K e^2 \omega^3 A^2 \pi}{c^3}
        \end{align*}

  \item

        \begin{align*}
          Q & = \frac{\frac{1}{2} m \omega^2 A^2}{\frac{K e^2 \omega^3 A^2 \pi}{2 \pi c^3}} \\
            & = \frac{c^3 m}{e^2 K \omega}
        \end{align*}

  \item

        \begin{align*}
          \tau & = \frac{1}{\gamma}                                         \\
               & = \frac{Q}{\omega}                                         \\
               & = \frac{c^3 m}{e^2 K \omega^2}                             \\
               & = \frac{c^3 m}{e^2 K \left( 2 \pi (c / \lambda) \right)^2} \\
               & = \frac{\lambda^2 c m}{4 \pi^2 e^2 K}                      \\
               & \approx \qty{1.13e-8}{s}
        \end{align*}
\end{enumerate}

\end{document}