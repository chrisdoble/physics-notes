\documentclass{article}
\usepackage{amsmath} % For align*
\usepackage{amsfonts} % For open face letters
\usepackage{enumitem} % For customisable list labels
\usepackage{graphicx} % For images
\usepackage{siunitx} % For units
\graphicspath{{./images/}}

\renewcommand{\Im}{\operatorname{Im}}
\renewcommand{\Re}{\operatorname{Re}}

\setlist[enumerate, 1]{label={(\alph*)}}
\setlist[enumerate, 2]{label={(\roman*)}}

\title{Vibrations and Waves by George C. King Problems}
\author{Chris Doble}
\date{April 2022}

\begin{document}

\maketitle

\tableofcontents

\section{Simple Harmonic Motion}

\subsection{}

\begin{enumerate}
  \item

        \begin{enumerate}
          \item $T = \qty{4}{s}$

          \item $\omega = \frac{\pi}{2} \,\unit{rad/s}$

          \item $\omega^2 = \frac{k}{m} \Rightarrow k = m \omega^2 = \frac{\pi^2}{8} \,\unit{N/m}$
        \end{enumerate}
\end{enumerate}

\subsection{}

\begin{enumerate}
  \item

        \begin{align*}
          x            & = A \cos \omega t           \\
                       & = A \cos 2 \pi f t          \\
          v            & = -2 \pi f A \sin 2 \pi f t \\
          v_\text{max} & = 2 \pi f A                 \\
                       & = \qty{1.38}{m/s}
        \end{align*}

  \item

        \begin{align*}
          a            & = -4 \pi^2 f^2 A \cos 2 \pi f t \\
          a_\text{max} & = 4 \pi^2 f^2 A                 \\
                       & = \qty{3.82e3}{m/s^2}
        \end{align*}
\end{enumerate}

\subsection{}

\begin{align*}
  a_\text{max}  & \le g                          \\
  4 \pi^2 f^2 A & \le g                          \\
  f             & \le \sqrt{\frac{g}{4 \pi^2 A}} \\
                & \le \qty{1.11}{Hz}
\end{align*}

\subsection{}

\begin{enumerate}
  \item \[\frac{U}{E} = \frac{\frac{1}{2} k \left( \frac{1}{2} A \right)^2}{\frac{1}{2} k A^2} = \frac{1}{4} \Rightarrow \frac{K}{E} = \frac{3}{4}\]

  \item

        \begin{enumerate}
          \item The total energy will increase by a factor of $4$

          \item The maximum velocity will increase by a factor of $2$

          \item The maximum acceleration will increase by a factor of $2$ and the period won't change
        \end{enumerate}
\end{enumerate}

\subsection{}

\begin{enumerate}
  \item $E = \frac{1}{2} m v^2 + \frac{1}{2} k x^2 = \qty{0.41}{J}$

  \item

        \begin{align*}
          E      & = \frac{1}{2} k A^2                             \\
          A      & = \sqrt{\frac{2 E}{k}}                          \\
                 & = \qty{4.5}{cm}                                 \\
          \omega & = \sqrt{\frac{k}{m}}                            \\
                 & = \sqrt{\frac{1600}{3}}                         \\
                 & = \frac{40}{\sqrt{3}}                           \\
                 & = \qty{23}{rad/s}                               \\
          x      & = A \cos (\omega t + \phi)                      \\
          \phi   & = \arccos \left( \frac{x}{A} \right) - \omega t \\
                 & = \qty{2.7}{rad}                                \\
          x      & = 0.045 \cos (23 t + 2.7) \,\unit{m}
        \end{align*}
\end{enumerate}

\subsection{}

Using the angular frequency of system (b) $\omega_b$ as the baseline, the angular frequency of system (a) $\omega_a$ is

\begin{align*}
  F = m a  & = -2 k x               \\
  a        & = -\frac{2 k}{m} x     \\
  \omega_a & = \sqrt{\frac{2 k}{m}} \\
           & = \sqrt{2} \omega_b
\end{align*}

and the angular frequency of system (c) $\omega_c$ is

\begin{align*}
  F = m a  & = -\frac{k}{2} x              \\
  a        & = -\frac{k}{2 m} x            \\
  \omega_c & = \sqrt{\frac{k}{2 m}}        \\
           & = \sqrt{\frac{1}{2}} \omega_b
\end{align*}

\subsection{}

\begin{enumerate}
  \item The test tube experiences a bouyancy force of $F = A g \rho x$ so its equation of motion is

        \begin{align*}
          F = m a & = -A g \rho x               \\
          a       & = -\frac{A g \rho}{m} x     \\
          \omega  & = \sqrt{\frac{A g \rho}{m}}
        \end{align*}

  \item The work done by the bouyancy force when moving from equilibrium to $x$ and thus the potential energy is

        \begin{align*}
          U & = \int_0^x A g \rho x' \,dx' \\
            & = \frac{1}{2} A g \rho x^2
        \end{align*}

        so the total energy of the system is \[E = \frac{1}{2} m v^2 + \frac{1}{2} A g \rho x^2\]
\end{enumerate}

\subsection{}

\[\unit{s} \propto \unit{kg}^\alpha \, \unit{m}^\beta \, \left( \unit{m/s^2} \right)^\gamma\] so $\alpha = 0$, $\beta = 1 / 2$, and $\gamma = -1 / 2$ meaning \[T \propto \sqrt{\frac{l}{g}}\]

\subsection{}

\begin{enumerate}
  \item

        \begin{align*}
          x            & = A \cos \sqrt{\frac{g}{l}} t                     \\
          v            & = -\sqrt{\frac{g}{l}} A \sin \sqrt{\frac{g}{l}} t \\
          v_\text{max} & = \sqrt{\frac{g}{l}} A                            \\
                       & = \qty{0.018}{m/s}
        \end{align*}

  \item The pendulum reaches its maximum speed at the bottom of its swing which occurs after a quarter cycle \[\frac{1}{4} T = \frac{1}{4} \frac{2 \pi}{\omega} = \frac{\pi}{2 \sqrt{g / l}} = \qty{0.43}{s}\]
\end{enumerate}

\subsection{}

\begin{align*}
  I \frac{d^2 \theta}{d t^2}                 & = \tau                           \\
  \frac{1}{3} M L^2 \frac{d^2 \theta}{d t^2} & = -k L \sin \theta L \cos \theta \\
  \frac{1}{3} M \frac{d^2 \theta}{d t^2}     & = -k \theta                      \\
  \frac{d^2 \theta}{d t^2}                   & = -\frac{3 k}{M} \theta          \\
  T                                          & = \frac{2 \pi}{\omega}           \\
                                             & = 2 \pi \sqrt{\frac{M}{3 k}}     \\
\end{align*}

\subsection{}

\begin{enumerate}
  \item

        \begin{align*}
          F = -\frac{d U}{d x} & = -\left( \frac{6 a}{x^7} - \frac{12 b}{x^{13}} \right) \\
          0                    & = \frac{12 b}{x^{13}} - \frac{6 a}{x^7}                 \\
                               & = \frac{12 b}{x^6} - 6 a                                \\
          6 a                  & = \frac{12 b}{x^6}                                      \\
          x^6                  & = \frac{2 b}{a}                                         \\
          x                    & = \left( \frac{2 b}{a} \right)^{1 / 6}
        \end{align*}
\end{enumerate}

\subsection{}

\begin{enumerate}
  \item

        \begin{align*}
          K & = \frac{1}{2} M v^2 + \int \,dK                                                            \\
            & = \frac{1}{2} M v^2 + \int_0^L \frac{1}{2} \frac{m}{L} \left( \frac{l}{L} v \right)^2 \,dl \\
            & = \frac{1}{2} M v^2 + \frac{1}{2} \frac{m v^2}{L^3} \int_0^L l^2 \,dl                      \\
            & = \frac{1}{2} M v^2 + \frac{1}{2} \frac{m v^2}{L^3} \frac{1}{3} L^3                        \\
            & = \frac{1}{2} M v^2 + \frac{1}{6} m v^2                                                    \\
            & = \frac{1}{2} (M + m / 3) v^2                                                              \\
          E & = K + U                                                                                    \\
            & = \frac{1}{2} (M + m / 3) v^2 + \frac{1}{2} k x^2
        \end{align*}

  \item

        \[\omega = \sqrt{\frac{k}{M + m / 3}}\]
\end{enumerate}

\subsection{}

\begin{enumerate}
  \item

        \begin{align*}
          K                 & = E - U                      \\
          \frac{1}{2} m v^2 & = U(A) - U(x)                \\
          v                 & = \sqrt{2 [U(A) - U(x)] / m}
        \end{align*}

  \item

        \begin{align*}
          T & = 4 \int_0^A \frac{dx}{v}                                              \\
            & = 4 \int_0^A \sqrt{\frac{m}{2 [U(A) - U(x)]}} \,dx                     \\
            & = 4 \sqrt{\frac{m}{2 U(A)}} \int_0^A \frac{dx}{\sqrt{1 - U(x) / U(A)}}
        \end{align*}

  \item

        \begin{align*}
          T & = 4 \sqrt{\frac{m}{2 \alpha A^n}} \int_0^A \frac{dx}{\sqrt{1 - (x / A)^n}}    \\
            & = 4 \sqrt{\frac{m}{2 \alpha A^n}} \int_0^1 \frac{A \,d \xi}{\sqrt{1 - \xi^n}} \\
            & = c A^{(n / 2) - 1}
        \end{align*}
\end{enumerate}

\section{The Damped Harmonic Oscillator}

\subsection{}

\begin{align*}
  \left( \frac{\gamma}{2} \right)^2 & = \omega_0^2                   \\
  \frac{b}{2 m}                     & = \sqrt{\frac{k}{m}}           \\
  b                                 & = 2 m \sqrt{\frac{k}{m}}       \\
                                    & = 2 m \sqrt{\frac{m g / x}{m}} \\
                                    & = 2 m \sqrt{\frac{g}{x}}       \\
                                    & = \qty{64}{kg / s}
\end{align*}

\subsection{}

\begin{align*}
  \frac{A_{n + 1}}{A_n} & = 0.90                             \\
  e^{-2.5 \gamma / 2}   & = 0.90                             \\
  e^{2.5 \gamma / 2}    & = \frac{1}{0.90}                   \\
  \frac{2.5 \gamma}{2}  & = \ln \frac{1}{0.90}               \\
  \gamma                & = \frac{2}{2.5} \ln \frac{1}{0.90} \\
                        & = \qty{8.43e-2}{s^{-1}}            \\
  F                     & = -b v                             \\
                        & = -(\num{4.21e-2}) v
\end{align*}

\subsection{}

After 10 cycles the amplitude has decreased by a factor of $5 / 3$. The energy of the system is proportional to the amplitude squared, so

\begin{align*}
  E(300)         & = E(0) e^{-300 / \tau}            \\
  e^{300 / \tau} & = \frac{E(0)}{E(300)}             \\
  \tau           & = \frac{300}{\ln [E(0) / E(300)]} \\
                 & = \frac{300}{\ln \frac{25}{9}}    \\
                 & = \qty{294}{s}                    \\
  Q              & = \omega_0 \tau                   \\
                 & = \frac{2 \pi \tau}{T}            \\
                 & = 61.5
\end{align*}

\subsection{}

\begin{align*}
  \frac{E(10 T)}{E_0} & = \frac{E_0 e^{-\gamma 10 T}}{E_0} \\
  \frac{1}{2}         & = e^{-\gamma 10 T}                 \\
  \frac{E(50 T)}{E_0} & = \frac{E_0 e^{-\gamma 50 T}}{E_0} \\
                      & = (e^{-\gamma 10 T})^5             \\
                      & = \left( \frac{1}{2} \right)^5     \\
                      & = \frac{1}{32}
\end{align*}

\subsection{}

\begin{enumerate}
  \item

        \begin{align*}
          Q_{0.01}      & = 310               \\
          \omega_{0.01} & = \qty{3.14}{rad/s} \\
          Q_{0.30}      & = 10.5              \\
          \omega_{0.30} & = \qty{3.14}{rad/s} \\
          Q_{1.00}      & = 3.14              \\
          \omega_{1.00} & = \qty{3.10}{rad/s}
        \end{align*}

        \setcounter{enumi}{2}
  \item

        \begin{align*}
          \gamma^2 / 4 & = \pi^2                                                  \\
          \gamma       & = 2 \pi                                                  \\
          x            & = A e^{-\pi t} + B t e^{-\pi t}                          \\
          A            & = \qty{10}{mm}                                           \\
          v            & = -10 \pi e^{-\pi t} + B e^{-\pi t} - \pi B t e^{-\pi t} \\
          0            & = -10 \pi + B                                            \\
          B            & = 10 \pi                                                 \\
          x            & = 10 e^{-\pi t} + 10 \pi t e^{-\pi t}
        \end{align*}
\end{enumerate}

\subsection{}

\begin{align*}
  \frac{\omega}{\omega_0} & = \frac{\omega_0 \sqrt{1 - 1 / 4 Q^2}}{\omega_0} \\
                          & = \sqrt{1 - 1 / 4 Q^2}                           \\
                          & = 1 - \frac{1 / 4 Q^2}{2} + \cdots               \\
                          & \approx 1 - \frac{Q^2}{8}
\end{align*}

\subsection{}

The amplitude of each pendulum decreases over time by a factor of

\begin{align*}
  \exp \left( -\frac{\gamma t}{2} \right) & = \exp \left( -\frac{b t}{2 m} \right)                              \\
                                          & = \exp \left( -\frac{b t}{2 \cdot \frac{4}{3} \pi r^3 \rho} \right) \\
                                          & = \exp \left( -\frac{3 b t}{8 \pi r^3 \rho} \right)                 \\
                                          & = \exp \left( -\frac{3 b t}{8 \pi r^3} \right)^{1 / \rho}.
\end{align*}

After 10 minutes the amplitude of oscillation of the aluminium pendulum has decreased to half of its initial value

\begin{align*}
  \exp \left( -\frac{225 b}{\pi r^3} \right)^{1 / \rho_a} & = \frac{1}{2}                         \\
  \exp \left( -\frac{225 b}{\pi r^3} \right)              & = \left( \frac{1}{2} \right)^{\rho_a}
\end{align*}

so the brass pendulum's amplitude of oscillation has decreased by a factor of

\begin{align*}
  \exp \left( -\frac{225 b}{\pi r^3} \right)^{1 / \rho_b} & = \left( \frac{1}{2} \right)^{\rho_a / \rho_b} \\
                                                          & = 0.802
\end{align*}

\subsection{}

\begin{enumerate}
  \item

        \begin{align*}
          x & = A \sin \omega t                                                                                                   \\
          v & = \omega A \cos \omega t                                                                                            \\
          a & = -\omega^2 A \sin \omega t                                                                                         \\
          E & = \int_0^T \frac{K e^2 a^2}{c^3} \,dt                                                                               \\
            & = \int_0^T \frac{K e^2 \omega^4 A^2 \sin^2 \omega t}{c^3} \,dt                                                      \\
            & = \frac{K e^2 \omega^4 A^2}{c^3} \int_0^{2 \pi / \omega} \sin^2 \omega t \,dt                                       \\
            & = \frac{K e^2 \omega^4 A^2}{c^3} \left[ \frac{t}{2} - \frac{1}{4 \omega} \sin 2 \omega t \right]_0^{2 \pi / \omega} \\
            & = \frac{K e^2 \omega^3 A^2 \pi}{c^3}
        \end{align*}

  \item

        \begin{align*}
          Q & = \frac{\frac{1}{2} m \omega^2 A^2}{\frac{K e^2 \omega^3 A^2 \pi}{2 \pi c^3}} \\
            & = \frac{c^3 m}{e^2 K \omega}
        \end{align*}

  \item

        \begin{align*}
          \tau & = \frac{1}{\gamma}                                         \\
               & = \frac{Q}{\omega}                                         \\
               & = \frac{c^3 m}{e^2 K \omega^2}                             \\
               & = \frac{c^3 m}{e^2 K \left( 2 \pi (c / \lambda) \right)^2} \\
               & = \frac{\lambda^2 c m}{4 \pi^2 e^2 K}                      \\
               & \approx \qty{1.13e-8}{s}
        \end{align*}
\end{enumerate}

\section{Forced Oscillations}

\subsection{}

\begin{align*}
  A(\qty{2}{rad / s})      & = \qty{1.3e-2}{m} \\
  \delta(\qty{2}{rad/s})   & = \ang{0.58}      \\
  A(\qty{20}{rad / s})     & = \qty{0.13}{m}   \\
  \delta(\qty{20}{rad/s})  & = \ang{90}        \\
  A(\qty{100}{rad / s})    & = \qty{5.2e-4}{m} \\
  \delta(\qty{100}{rad/s}) & = \ang{179}       \\
\end{align*}

\subsection{}

\begin{align*}
  A(\omega) & = \frac{a \omega_0 / \omega}{\sqrt{(\omega_0 / \omega - \omega / \omega_0)^2 + 1 / Q^2}} \\
            & = \frac{a u}{\sqrt{(u - 1 / u)^2 + 1 / Q^2}}                                             \\
            & = \frac{a}{\sqrt{(1 - 1 / u^2)^2 + 1 / u^2 Q^2}}
\end{align*}

$A(\omega)$ is maximised when the denominator is minimised which occurs when

\begin{align*}
  \frac{d}{d u} ((1 - u^{-2})^2 + Q^{-2} u^{-2}) & = 0                                  \\
  2 (1 - u^{-2}) 2 u^{-3} - 2 Q^{-2} u^{-3}      & = 0                                  \\
  4 (1 - u^{-2}) - 2 Q^{-2}                      & = 0                                  \\
  \frac{4 Q^2 - 2}{Q^2}                          & = \frac{4}{u^2}                      \\
  \frac{Q^2}{4 Q^2 - 2}                          & = \frac{u^2}{4}                      \\
  \frac{4 Q^2}{4 Q^2 - 2}                        & = u^2                                \\
  \frac{1}{1 - 1 / 2 Q^2}                        & = u^2                                \\
  \frac{1}{\sqrt{1 - 1 / 2 Q^2}}                 & = \frac{\omega_0}{\omega_\text{max}} \\
  \omega_\text{max}                              & = \omega_0 \sqrt{1 - 1 / 2 Q^2}
\end{align*}

at which point the amplitude will be

\begin{align*}
  A_\text{max} & = \frac{a \omega_0 / \omega_0 \sqrt{1 - 1 / 2 Q^2}}{\sqrt{\left( \frac{\omega_0}{\omega_0 \sqrt{1 - 1 / 2 Q^2}} - \frac{\omega_0 \sqrt{1 - 1 / 2 Q^2}}{\omega_0} \right)^2 + 1 / Q^2}} \\
               & = \frac{a / \sqrt{1 - 1 / 2 Q^2}}{\sqrt{\left( \frac{1}{\sqrt{1 - 1 / 2 Q^2}} - \sqrt{1 - 1 / 2 Q^2} \right)^2 + 1 / Q^2}}                                                             \\
               & = \frac{a / \sqrt{1 - 1 / 2 Q^2}}{\sqrt{\left( \frac{1 - 1 + 1 / 2 Q^2}{\sqrt{1 - 1 / 2 Q^2}} \right)^2 + 1 / Q^2}}                                                                    \\
               & = \frac{a / \sqrt{1 - 1 / 2 Q^2}}{\sqrt{\frac{1}{4 Q^4 (1 - 1 / 2 Q^2)} + 1 / Q^2}}                                                                                                    \\
               & = \frac{a / \sqrt{1 - 1 / 2 Q^2}}{\sqrt{\frac{1 + 4 Q^2 (1 - 1 / 2 Q^2)}{4 Q^2 (1 - 1 / 2 Q^2)}}}                                                                                      \\
               & = \frac{a}{\sqrt{1 - 1 / 2 Q^2}} \sqrt{\frac{4 Q^2 (1 - 1 / 2 Q^2)}{1 + 4 Q^2 (1 - 1 / 2 Q^2)}}                                                                                        \\
               & = a \sqrt{\frac{4 Q^2}{1 + 4 Q^2 (1 - 1 / 2 Q^2)}}                                                                                                                                     \\
               & = a Q \sqrt{\frac{4}{4 (1 - 1 / 2 Q^2) + 1 / Q^2}}                                                                                                                                     \\
               & = \frac{a Q}{\sqrt{1 - 1 / 2 Q^2 + 1 / 4 Q^2}}                                                                                                                                         \\
               & = \frac{a Q}{\sqrt{1 - 1 / 4 Q^2}}
\end{align*}

\subsection{}

\begin{enumerate}
  \item

        \begin{align*}
          \frac{\omega_0 - \omega_\text{max}}{\omega_0} & = 1 - \sqrt{1 - 1 / 2 Q^2}   \\
                                                        & \approx \qty{0.25}{\percent}
        \end{align*}

  \item

        \begin{align*}
          \frac{A_\text{max} - A_0}{A_0} & = \frac{1}{\sqrt{1 - 1 / 4 Q^2}} - 1 \\
                                         & \approx \qty{0.13}{\percent}
        \end{align*}
\end{enumerate}

\subsection{}

\begin{align*}
  \overline{P}_\text{max} & = \frac{F_0^2}{2 m \gamma}                                                       \\
                          & = \qty{50}{W}                                                                    \\
  \overline{P}(\omega)    & = \frac{F_0^2}{2 m \omega_0 Q [4 (\Delta \omega / \omega_0)^2 + 1 / Q^2]}        \\
                          & = \frac{\overline{P}_\text{max}}{Q^2 [4 (\Delta \omega / \omega_0)^2 + 1 / Q^2]} \\
                          & = \frac{50}{625 [4 (\Delta \omega / 100)^2 + 1 / 625]}                           \\
                          & = \frac{50}{\frac{1}{4} \Delta \omega^2 + 1}
\end{align*}

\subsection{}

\begin{enumerate}
  \item \[\frac{\omega_0}{2 \pi} = \frac{1}{2 \pi \sqrt{L C}} \approx \qty{398}{Hz}\]

  \item \[I = \frac{V_0}{R} = \frac{15}{75} = \qty{0.2}{A}\]
\end{enumerate}

\subsection{}

\[i^i = (e^{i \pi / 2})^i = e^{-\pi / 2} = 0.208\]

\subsection{}

\begin{align*}
  z               & = A e^{i (\omega t + \phi)}                 \\
  x               & = \Re z                                     \\
                  & = A \cos (\omega t + \phi)                  \\
  \frac{d z}{d t} & = i \omega A e^{i (\omega t + \phi)}        \\
  \frac{d x}{ dt} & = \Re \frac{d z}{d t}                       \\
                  & = -\omega A \sin (\omega t + \phi)          \\
                  & = \omega A \cos (\omega t + \phi + \pi / 2)
\end{align*}

$\frac{d x}{d t}$ is in advance of $x$ by $\ang{90}$

\subsection{}

\begin{enumerate}
  \item
        \begin{align*}
          m \frac{d^2 x}{d t^2} + b \frac{d x}{d t} + m g \frac{x - \xi}{l}             & = 0                                   \\
          m \frac{d^2 x}{d t^2} + b \frac{d x}{d t} + m \frac{g}{l} x                   & = m \frac{g}{l} a \cos \omega t       \\
          m \frac{d^2 x}{d t^2} + b \frac{d x}{d t} + m \omega_0^2 x                    & = m \omega_0^2 a \cos \omega t        \\
          \Re \left( m \frac{d^2 z}{d t^2} + b \frac{d z}{d t} + m \omega_0^2 z \right) & = \Re (m \omega_0^2 a e^{i \omega t})
        \end{align*}

  \item

        % z = A e^{i (\omega t - \delta)}

        \begin{align*}
          m \frac{d^2}{d t^2} (A e^{i (\omega t - \delta)}) + b \frac{d}{d t} (A e^{i (\omega t - \delta)}) + m \omega_0^2 A e^{i (\omega t - \delta)} & = m \omega_0^2 a e^{i \omega t} \\
          -m \omega^2 A e^{i (\omega t - \delta)} + i \omega b A e^{i (\omega t - \delta)} + m \omega_0^2 A e^{i (\omega t - \delta)}                  & = m \omega_0^2 a e^{i \omega t} \\
          -m \omega^2 A + i \omega b A + m \omega_0^2 A                                                                                                & = m \omega_0^2 a e^{i \delta}   \\ \\
          -m \omega^2 A + m \omega_0^2 A                                                                                                               & = m \omega_0^2 a \cos \delta    \\
          -\omega^2 A + \omega_0^2 A                                                                                                                   & = \omega_0^2 a \cos \delta      \\
          \frac{\omega_0^2 - \omega^2}{\omega_0^2 a} A                                                                                                 & = \cos \delta                   \\ \\
          \omega b A                                                                                                                                   & = m \omega_0^2 a \sin \delta    \\
          \frac{\omega b}{m \omega_0^2 a} A                                                                                                            & = \sin \delta                   \\ \\
          \frac{\omega b}{m (\omega_0^2 - \omega^2)}                                                                                                   & = \tan \delta                   \\
          \frac{\omega \gamma}{\omega_0^2 - \omega^2}                                                                                                  & = \tan \delta
        \end{align*}

        \begin{align*}
          A & = \frac{\omega_0^2 a \cos \delta}{\omega_0^2 - \omega^2}                                                                        \\
            & = \frac{\omega_0^2 a}{\omega_0^2 - \omega^2} \frac{\omega_0^2 - \omega^2}{\sqrt{(\omega_0^2 - \omega^2)^2 + \omega^2 \gamma^2}} \\
            & = \frac{a \omega_0^2}{\sqrt{(\omega_0^2 - \omega^2)^2 + \omega^2 \gamma^2}}
        \end{align*}
\end{enumerate}

\subsection{}

\begin{enumerate}
  \item

        \begin{align*}
          A(t_{75})               & = A_0 e^{-\gamma t_{75} / 2}                     \\
          \frac{A(t_{75})}{A_0}   & = e^{-\gamma t_{75} / 2}                         \\
          \ln \frac{A_0 / e}{A_0} & = -\frac{\gamma t_{75}}{2}                       \\
          -1                      & = -\frac{\gamma t_{75}}{2}                       \\
          \frac{2}{t_{75}}        & = \gamma                                         \\ \\
          Q                       & = \frac{\omega_0}{\gamma}                        \\
                                  & = \frac{75 \cdot 2 \pi}{t_{75}} \frac{t_{75}}{2} \\
                                  & = 75 \pi
        \end{align*}

  \item

        \begin{align*}
          A(\omega_0) & = a \frac{\omega_0^2}{\sqrt{(\omega_0^2  - \omega_0^2)^2 + \omega_0^2 \gamma^2}} \\
                      & = a \frac{\omega_0^2}{\sqrt{\omega_0^2 \gamma^2}}                                \\
                      & = a \frac{\omega_0}{\gamma}                                                      \\
                      & = a Q                                                                            \\
                      & = (\qty{0.5}{mm}) 75 \pi                                                         \\
                      & = \qty{0.12}{m}
        \end{align*}

  \item

        \begin{align*}
          \frac{a Q}{2}                & = a \frac{\omega_0^2}{\sqrt{(\omega_0^2 - \omega^2)^2 + \omega^2 \gamma^2}}               \\
                                       & \approx a \frac{\omega_0^2}{\sqrt{(2 \omega_0 (-\Delta \omega))^2 + \omega_0^2 \gamma^2}} \\
          \frac{2 \omega_0^2}{Q}       & = \sqrt{4 \omega_0^2 \Delta \omega^2 + \omega_0^2 \gamma^2}                               \\
          \frac{4 \omega_0^4}{Q^2}     & = 4 \omega_0^2 \Delta \omega^2 + \omega_0^2 \gamma^2                                      \\
          4 \omega_0^2 \Delta \omega^2 & = \frac{4 \omega_0^4}{(\omega_0 / \gamma)^2} - \omega_0^2 \gamma^2                        \\
          (2 \Delta \omega)^2          & = 4 \gamma^2 - \gamma^2                                                                   \\
          2 \Delta \omega              & = \gamma \sqrt{3}                                                                         \\
                                       & = \frac{\omega_0}{Q} \sqrt{3}                                                             \\
                                       & = \frac{\sqrt{g / l}}{Q} \sqrt{3}                                                         \\
                                       & = \qty{0.019}{rad/s}
        \end{align*}
\end{enumerate}

\subsection{}

\begin{enumerate}
  \item

        \begin{enumerate}
          \item

                \begin{align*}
                  K & = \frac{1}{2} m v^2                                                 \\
                    & = \frac{1}{2} m \left( -\omega A \sin (\omega t - \delta) \right)^2 \\
                    & = \frac{1}{2} m \omega^2 A^2 \sin^2 (\omega t - \delta)
                \end{align*}

          \item

                \begin{align*}
                  U & = \frac{1}{2} k x^2                            \\
                    & = \frac{1}{2} k (A \cos (\omega t - \delta))^2 \\
                    & = \frac{1}{2} k A^2 \cos^2 (\omega t - \delta)
                \end{align*}

          \item

                \begin{align*}
                  E & = \frac{1}{2} m \omega^2 A^2 \sin^2 (\omega t - \delta) + \frac{1}{2} k A^2 \cos^2 (\omega t - \delta) \\
                    & = \frac{1}{2} m A^2 [\omega^2 \sin^2 (\omega t - \delta) + \omega_0^2 \cos^2 (\omega t - \delta)]
                \end{align*}
        \end{enumerate}

  \item

        \begin{align*}
          0      & = \frac{d E}{d t}                                                                                                                                   \\
                 & = \frac{1}{2} m A^2 [2 \omega^3 \sin (\omega t - \delta) \cos (\omega t - \delta) - 2 \omega_0^3 \cos (\omega t - \delta) \sin (\omega t - \delta)] \\
                 & = \frac{1}{2} m A^2 \sin (2 (\omega t - \delta)) (\omega^3 - \omega_0^3)                                                                            \\
          \omega & = \omega_0                                                                                                                                          \\ \\
          E      & = \frac{1}{2} m A^2 [\omega_0^2 \sin^2 (\omega_0 t - \delta) + \omega_0^2 \cos^2 (\omega_0 t - \delta)]                                             \\
                 & = \frac{1}{2} m A^2 \omega_0^2
        \end{align*}

  \item

        \begin{align*}
          \overline{K}                      & = \frac{1}{T} \int_{t_0}^{t_0 + T} \frac{1}{2} m \omega^2 A^2 \sin^2 (\omega t - \delta) \,dt                                                                      \\
                                            & = \frac{m \omega^2 A^2}{2 T} \int_{t_0}^{t_0 + T} \sin^2 (\omega t - \delta) \,dt                                                                                  \\
                                            & = \frac{m \omega^2 A^2}{4}                                                                                                                                         \\
          \overline{E}                      & = \frac{1}{T} \int_{t_0}^{t_0 + T} \frac{1}{2} m A^2 [\omega^2 \sin^2 (\omega t - \delta) + \omega_0^2 \cos^2 (\omega t - \delta)] \,dt                            \\
                                            & = \frac{m A^2}{2 T} \left( \int_{t_0}^{t_0 + T} \omega^2 \sin^2 (\omega t - \delta) \,dt + \int_{t_0}^{t_0 + T} \omega_0^2 \cos^2 (\omega t - \delta) \,dt \right) \\
                                            & = \frac{m A^2}{4} (\omega^2 + \omega_0^2)                                                                                                                          \\
          \frac{\overline{K}}{\overline{E}} & = \frac{m \omega^2 A^2}{4} \frac{4}{m A^2 (\omega^2 + \omega_0^2)}                                                                                                 \\
                                            & = \frac{\omega^2}{\omega^2 + \omega_0^2}                                                                                                                           \\
                                            & = \frac{1}{1 + (\omega_0 / \omega)^2}
        \end{align*}

  \item

        \begin{align*}
          \overline{E} & = \overline{K} + \overline{U}                                                                                               \\
                       & = \frac{m \omega^2 A^2}{4} + \frac{k A^2}{4}                                                                                \\
                       & = \frac{1}{4} m A^2 (\omega^2 + \omega_0^2)                                                                                 \\
                       & = \frac{1}{4} m (\omega^2 + \omega_0^2) \left( \frac{F_0}{m \sqrt{(\omega_0^2 - \omega^2)^2 + \omega^2 \gamma^2}} \right)^2 \\
                       & = \frac{1}{4} m (\omega^2 + \omega_0^2) \frac{F_0^2}{m^2 [(\omega_0^2 - \omega^2)^2 + \omega^2 \gamma^2]}                   \\
                       & = \frac{F_0^2 (\omega_0^2 + \omega^2)}{4 m [(\omega_0^2 - \omega^2)^2 + \omega^2 b^2 / m^2]}
        \end{align*}
\end{enumerate}

\subsection{}

\begin{enumerate}
  \item

        \begin{align*}
          x & = A \cos \omega t                                             \\
          v & = -\omega A \sin \omega t                                     \\
          W & = \int_0^T b v^2 \,dt                                         \\
            & = b \int_0^{2 \pi / \omega} \omega^2 A^2 \sin^2 \omega t \,dt \\
            & = \pi b \omega A^2
        \end{align*}

  \item

        \begin{align*}
          \frac{W}{E} & = \frac{\pi b \omega A^2}{\frac{1}{2} m \omega^2 A^2} \\
                      & = \frac{2 \pi b}{m \omega}
        \end{align*}

  \item

        \begin{align*}
          \frac{W}{E} & = \frac{2 \pi b}{m \omega_0}    \\
                      & = \frac{2 \pi \gamma}{\omega_0} \\
                      & = \frac{2 \pi}{Q}
        \end{align*}
\end{enumerate}

\subsection{}

Let $T'$ be the number of seconds in 8 days and $T = 2 \pi \sqrt{l / g}$ by the period of the pendulum, then

\begin{align*}
  Q & = 2 \pi \frac{\text{stored energy}}{\text{energy dissipated/cycle}}                                     \\
    & = 2 \pi \frac{1}{2} m_1 \omega^2 A^2 \frac{\text{cycles in 8 days}}{\text{energy dissipated in 8 days}} \\
    & = \pi m_1 \frac{g}{l} A^2 \frac{T' / T}{m_2 g h}                                                        \\
    & = \frac{\pi m_1 A^2 T'}{m_2 l h T}                                                                      \\
    & = \frac{m_1 A^2 \cdot 8 \cdot 24 \cdot 60 \cdot 60}{2 m_2 l h \sqrt{l / g}}                             \\
    & \approx 70
\end{align*}

\section{Coupled Oscillators}

\subsection{}

\begin{enumerate}
  \item

        \begin{align*}
          \omega_1 & = \sqrt{\frac{g}{l}}                 \\
                   & = \qty{5.7}{rad/s}                   \\
          \omega_2 & = \sqrt{\frac{g}{l} + \frac{2 k}{m}} \\
                   & = \qty{6.0}{rad/s}
        \end{align*}

  \item The oscillation of the first pendulum is described by the equation \[x_a = A \cos \frac{(\omega_2 - \omega_1) t}{2} \cos \frac{(\omega_2 + \omega_1) t}{2},\] the amplitude of which temporarily becomes $0$ at \[\frac{(\omega_2 - \omega_1) t}{2} = \frac{\pi}{2} \Rightarrow t = \frac{\pi}{\omega_2 - \omega_1} = \qty{11.6}{s}.\]
\end{enumerate}

\subsection{}

\begin{enumerate}
  \item

        \begin{align*}
          \qty{5.0}{mm} & = \frac{1}{2} (C_1 + C_2) \\
          \qty{5.0}{mm} & = \frac{1}{2} (C_1 - C_2) \\
          \qty{10}{mm}  & = C_1                     \\
          \qty{0.0}{mm} & = C_2
        \end{align*}

  \item

        \begin{enumerate}
          \item

                \begin{align*}
                  \qty{5.0}{mm}  & = \frac{1}{2} (C_1 + C_2) \\
                  \qty{-5.0}{mm} & = \frac{1}{2} (C_1 - C_2) \\
                  \qty{0.0}{mm}  & = C_1                     \\
                  \qty{10}{mm}   & = C_2
                \end{align*}

          \item

                \begin{align*}
                  \qty{10}{mm} & = \frac{1}{2} (C_1 + C_2) \\
                  \qty{0}{mm}  & = \frac{1}{2} (C_1 - C_2) \\
                  \qty{10}{mm} & = C_1                     \\
                  \qty{10}{mm} & = C_2
                \end{align*}

          \item

                \begin{align*}
                  \qty{10}{mm}  & = \frac{1}{2} (C_1 + C_2) \\
                  \qty{5.0}{mm} & = \frac{1}{2} (C_1 - C_2) \\
                  \qty{15}{mm}  & = C_1                     \\
                  \qty{5.0}{mm} & = C_2
                \end{align*}
        \end{enumerate}
\end{enumerate}

\subsection{}

\begin{align*}
  m \frac{d^2 x_a}{d t^2}         & = k x_b - 2 k x_a      \\
  m \frac{d^2 x_b}{d t^2}         & = k x_a - 2 k x_b      \\ \\
  m \frac{d^2 (x_a + x_b)}{d t^2} & = -k (x_a + x_b)       \\
  m \frac{d^2 q_1}{d t^2}         & = -k q_1               \\
  \omega_1                        & = \sqrt{\frac{k}{m}}   \\ \\
  m \frac{d^2 (x_a - x_b)}{d t^2} & = -3 k (x_a - x_b)     \\
  m \frac{d^2 q_2}{d t^2}         & = -3 k q_2             \\
  \omega_2                        & = \sqrt{\frac{3 k}{m}}
\end{align*}

\subsection{}

\begin{enumerate}
  \item

        \begin{align*}
          E_a & = \frac{1}{2} m \omega^2 A^2                                                                                \\
              & = \frac{1}{2} m A^2 \left( \frac{\omega_2 + \omega_1}{2} \right)^2 \cos^2 \frac{(\omega_2 - \omega_1) t}{2} \\
          E_b & = \frac{1}{2} m \omega^2 A^2                                                                                \\
              & = \frac{1}{2} m \left( \frac{\omega_2 + \omega_1}{2} \right)^2 A^2 \sin^2 \frac{(\omega_2 - \omega_1) t}{2}
        \end{align*}

  \item

        \begin{align*}
          \frac{(\omega_2 - \omega_1) T}{2} & = \pi                               \\
          T                                 & = \frac{2 \pi}{\omega_2 - \omega_1} \\
          \omega                            & = \frac{2 \pi}{T}                   \\
                                            & = \omega_2 - \omega_1
        \end{align*}
\end{enumerate}

\subsection{}

\begin{enumerate}
  \item

        \begin{align*}
          m \frac{d^2 x_1}{d t^2}                                     & = -2 m g \frac{x_1}{l} + m g \frac{x_2 - x_1}{l} \\
          \frac{d^2 x_1}{d t^2} + \frac{3 g}{l} x_1 - \frac{g}{l} x_2 & = 0                                              \\ \\
          m \frac{d^2 x_2}{d t^2}                                     & = -m g \frac{x_2 - x_1}{l}                       \\
          \frac{d^2 x_2}{d t^2} - \frac{g}{l} x_1 + \frac{g}{l} x_2   & = 0
        \end{align*}

  \item

        \begin{align*}
          -\omega^2 A \cos \omega t + \frac{3 g}{l} A \cos \omega t - \frac{g}{l} B \cos \omega t   & = 0                                       \\
          A \left( \frac{3 g}{l} - \omega^2 \right)                                                 & = B \left( \frac{g}{l} \right)            \\ \\
          -\omega^2 B \cos \omega t - \frac{g}{l} A \cos \omega t + \frac{g}{l} B \cos \omega t     & = 0                                       \\
          A \left( \frac{g}{l} \right)                                                              & = B \left( \frac{g}{l} - \omega^2 \right) \\ \\
          \frac{3 g / l - \omega^2}{g / l}                                                          & = \frac{g / l}{g / l - \omega^2}          \\
          \left( \frac{3 g}{l} - \omega^2 \right) \left( \frac{g}{l} - \omega^2 \right)             & = \left( \frac{g}{l} \right)^2            \\
          3 \left( \frac{g}{l} \right)^2 - \frac{3 g}{l} \omega^2 - \frac{g}{l} \omega^2 + \omega^4 & = \left( \frac{g}{l} \right)^2            \\
          (\omega^2)^2 - \frac{4 g}{l} \omega^2 + 2 \left( \frac{g}{l} \right)^2                    & = 0                                       \\
        \end{align*}

        \begin{align*}
          \omega^2 & = \frac{\frac{4 g}{l} \pm \sqrt{\frac{16 g^2}{l^2} - 8 \left( \frac{g}{l} \right)^2}}{2} \\
                   & = (2 \pm \sqrt{2}) \frac{g}{l}                                                           \\
          \omega   & = \sqrt{(2 \pm \sqrt{2}) \frac{g}{l}}
        \end{align*}

        \begin{align*}
          \frac{A}{B} & = \frac{g / l}{3 g / l - \omega^2}               \\
                      & = \frac{g}{3 g - l \omega^2}                     \\
                      & = \frac{g}{3 g - l (2 \pm \sqrt{2}) \frac{g}{l}} \\
                      & = \frac{1}{3 - (2 \pm \sqrt{2})}                 \\
                      & = \frac{1}{1 \pm \sqrt{2}}
        \end{align*}

  \item

        \begin{align*}
          T & = \frac{2 \pi}{\omega}                              \\
            & = \frac{2 \pi}{\sqrt{(2 \pm \sqrt{2}) \frac{g}{l}}} \\
            & = \qty{1.09}{s} \text{ or } \qty{2.62}{s}
        \end{align*}
\end{enumerate}

\subsection{}

\begin{enumerate}
  \setcounter{enumi}{1}
  \item

        \begin{align*}
          m \frac{d^2 x_1}{d t^2}                                                    & = k (x_2 - x_1)                  \\
          \frac{d^2 x_1}{d t^2} + \omega_1^2 x_1 - \omega_1^2 x_2                    & = 0                              \\ \\
          M \frac{d^2 x_2}{d t^2}                                                    & = -k (x_2 - x_1) + k (x_3 - x_2) \\
                                                                                     & = k x_1 - 2 k x_2 + k x_3        \\
          \frac{d^2 x_2}{d t^2} - \omega_2^2 x_1 + 2 \omega_2^2 x_2 - \omega_2^2 x_3 & = 0                              \\ \\
          m \frac{d^2 x_3}{d t^2}                                                    & = -k (x_3 - x_2)                 \\
          \frac{d^2 x_3}{d t^2} - \omega_1^2 x_2 + \omega_1^2 x_3                    & = 0
        \end{align*}

  \item

        \begin{align*}
          x_1 & = A \cos \omega t \\
          x_2 & = B \cos \omega t \\
          x_3 & = C \cos \omega t
        \end{align*}

        \begin{align*}
          -\omega^2 A \cos \omega t + \omega_1^2 A \cos \omega t - \omega_1^2 B \cos \omega t                                & = 0                           \\
          A (\omega_1^2 - \omega^2)                                                                                          & = B (\omega_1^2)              \\ \\
          -\omega^2 B \cos \omega t - \omega_2^2 A \cos \omega t + 2 \omega_2^2 B \cos \omega t - \omega_2^2 C \cos \omega t & = 0                           \\
          (A + C) (\omega_2^2)                                                                                               & = B (2 \omega_2^2 - \omega^2) \\ \\
          -\omega^2 C \cos \omega t - \omega_1^2 B \cos \omega t + \omega_1^2 C \cos \omega t                                & = 0                           \\
          C (\omega_1^2 - \omega^2)                                                                                          & = B (\omega_1^2)              \\ \\
          (A + C)(\omega_1^2 - \omega^2)                                                                                     & = B (2 \omega_1^2)
        \end{align*}

        \begin{align*}
          \frac{\omega_2^2}{\omega_1^2 - \omega^2} & = \frac{2 \omega_2^2 - \omega^2}{2 \omega_1^2}                                     \\
          2 \omega_1^2 \omega_2^2                  & = (2 \omega_2^2 - \omega^2)(\omega_1^2 - \omega^2)                                 \\
                                                   & = 2 \omega_1^2 \omega_2^2 - 2 \omega_2^2 \omega^2 - \omega_1^2 \omega^2 + \omega^4 \\
          0                                        & = (\omega^2)^2 - (\omega_1^2 + 2 \omega_2^2) \omega^2                              \\
                                                   & = \omega^2 - \omega_1^2 - 2 \omega_2^2                                             \\
          \omega^2                                 & = \omega_1^2 + 2 \omega_2^2                                                        \\
                                                   & = \frac{k}{m} + 2 \frac{k}{M}                                                      \\
          \omega                                   & = \sqrt{\frac{k (M + 2 m)}{M m}}
        \end{align*}

  \item

        \begin{align*}
          \frac{\sqrt{\frac{k (M + 2 m)}{M m}}}{\sqrt{\frac{k}{m}}} & = \sqrt{\frac{M + 2 m}{M}} \\
                                                                    & = \sqrt{1 + 2 m / M}       \\
                                                                    & = \sqrt{1 + 16 / 6}        \\
                                                                    & \approx 1.91
        \end{align*}
\end{enumerate}

\subsection{}

\begin{enumerate}
  \item Let $x_1$ be the top mass's displacement from equilibrium and $x_2$ be the bottom mass's, then the equations of motion are

        \begin{align*}
          3 m \frac{d^2 x_1}{d t^2}                                       & = -4 k x_1 + k (x_2 - x_1) \\
                                                                          & = -5 k x_1 + k x_2         \\
          \frac{d^2 x_1}{d t^2} + \frac{5 k}{3 m} x_1 - \frac{k}{3 m} x_2 & = 0                        \\ \\
          m \frac{d^2 x_2}{d t^2}                                         & = -k (x_2 - x_1)           \\
                                                                          & = k x_1 - k x_2            \\
          \frac{d^2 x_2}{d t^2} - \frac{k}{m} x_1 + \frac{k}{m} x_2       & = 0
        \end{align*}

        Assuming solutions of the form $x_1 = A \cos \omega t$ and $x_2 = B \cos \omega t$ and substituting into the above gives

        \begin{align*}
          -\omega^2 A \cos \omega t + \frac{5 k}{3 m} A \cos \omega t - \frac{k}{3 m} B \cos \omega t & = 0                                       \\
          A \left( \frac{5 k}{3 m} - \omega^2 \right)                                                 & = B \left( \frac{k}{3 m} \right)          \\ \\
          -\omega^2 B \cos \omega t - \frac{k}{m} A \cos \omega t + \frac{k}{m} B \cos \omega t       & = 0                                       \\
          A \left( \frac{k}{m} \right)                                                                & = B \left( \frac{k}{m} - \omega^2 \right) \\ \\
        \end{align*}

        \begin{align*}
          \frac{\frac{5 k}{3 m} - \omega^2}{\frac{k}{m}}                                                        & = \frac{\frac{k}{3 m}}{\frac{k}{m} - \omega^2} \\
          \left( \frac{5 k}{3 m} - \omega^2 \right) \left( \frac{k}{m} - \omega^2 \right)                       & = \frac{1}{3} \left( \frac{k}{m} \right)^2     \\
          \frac{5}{3} \left( \frac{k}{m} \right)^2 - \frac{5 k}{3 m} \omega^2 - \frac{k}{m} \omega^2 + \omega^4 & = \frac{1}{3} \left( \frac{k}{m} \right)^2     \\
          (\omega^2)^2 - \frac{8}{3} \frac{k}{m} \omega^2 + \frac{4}{3} \left( \frac{k}{m} \right)^2            & = 0
        \end{align*}

        \begin{align*}
          \omega^2 & = \frac{\frac{8}{3} \frac{k}{m} \pm \sqrt{\left( \frac{8}{3} \frac{k}{m} \right)^2 - \frac{16}{3} \left( \frac{k}{m} \right)^2}}{2}  \\
                   & = \frac{\frac{8}{3} \frac{k}{m} \pm \sqrt{\frac{64}{9} \left( \frac{k}{m} \right)^2 - \frac{16}{3} \left( \frac{k}{m} \right)^2}}{2} \\
                   & = \frac{\frac{8}{3} \frac{k}{m} \pm \sqrt{\frac{16}{9} \left( \frac{k}{m} \right)^2}}{2}                                             \\
                   & = \frac{\frac{8}{3} \frac{k}{m} \pm \frac{4}{3} \frac{k}{m}}{2}                                                                      \\
                   & = \left( \frac{4}{3} \pm \frac{2}{3} \right) \frac{k}{m}                                                                             \\
          \omega   & = \sqrt{\frac{2 k}{3 m}} \text{ or } \sqrt{\frac{2 k}{m}}
        \end{align*}

  \item

        \begin{align*}
          A \left( \frac{k}{m} \right) & = B \left( \frac{k}{m} - \omega^2 \right)        \\ \\
          A \left( \frac{k}{m} \right) & = B \left( \frac{k}{m} - \frac{2 k}{3 m} \right) \\
                                       & = B \left( \frac{k}{3 m} \right)                 \\
          A                            & = \frac{1}{3} B                                  \\ \\
          A \left( \frac{k}{m} \right) & = B \left( \frac{k}{m} - \frac{2 k}{m} \right)   \\
                                       & = B \left( -\frac{k}{m} \right)                  \\
          A                            & = -B
        \end{align*}

        So the first normal mode is

        \begin{align*}
          x_1 & = A \cos \sqrt{\frac{2 k}{3 m}} t   \\
          x_2 & = 3 A \cos \sqrt{\frac{2 k}{3 m}} t
        \end{align*}

        where the masses oscillate in phase and the lower mass has an amplitude 3 times greater than the upper mass.

        The second normal mode is

        \begin{align*}
          x_1 & = A \cos \sqrt{\frac{2 k}{m}} t  \\
          x_2 & = -A \cos \sqrt{\frac{2 k}{m}} t
        \end{align*}

        where the masses oscillate $\ang{180}$ out of phase with equal amplitude.
\end{enumerate}

\subsection{}

There are 5 normal modes in the transverse direction.

\subsection{}

\begin{enumerate}
  \item

        \begin{align*}
          M \frac{d^2 x_1}{d t^2} + (k_1 + k_2) x_1 - k_2 x_2 & = F_0 \cos \omega t \\
          m \frac{d^2 x_2}{d t^2} - k_2 x_1 + k_2 x_2         & = 0
        \end{align*}

  \item

        \begin{align*}
          -M \omega^2 A \cos \omega t + (k_1 + k_2) A \cos \omega t - k_2 B \cos \omega t               & = F_0 \cos \omega t                      \\
          A (k_1 + k_2 - M \omega^2)                                                                    & = B k_2 + F_0                            \\ \\
          -m \omega^2 B \cos \omega t - k_2 A \cos \omega t + k_2 B \cos \omega t                       & = 0                                      \\
          A k_2                                                                                         & = B (k_2 - m \omega^2)                   \\ \\
          B                                                                                             & = A \frac{k_2}{k_2 - m \omega^2}         \\
          A (k_1 + k_2 - M \omega^2)                                                                    & = A \frac{k_2^2}{k_2 - m \omega^2} + F_0 \\
          A \left( k_1 + k_2 - M \omega^2 - \frac{k_2^2}{k_2 - m \omega^2} \right)                      & = F_0                                    \\
          A \left( \frac{(k_1 + k_2 - M \omega^2) (k_2 - m \omega^2) - k_2^2}{k_2 - m \omega^2} \right) & = F_0                                    \\
          \frac{F_0 (k_2 - m \omega^2)}{(k_1 + k_2 - M \omega^2) (k_2 - m \omega^2) - k_2^2}            & = A
        \end{align*}

        \begin{align*}
          A                                                                              & = \frac{B k_2 + F_0}{k_1 + k_2 - M \omega^2}                              \\
          B                                                                              & = \frac{B k_2 + F_0}{k_1 + k_2 - M \omega^2} \frac{k_2}{k_2 - m \omega^2} \\
                                                                                         & = \frac{B k_2^2 + F_0 k_2}{(k_1 + k_2 - M \omega^2) (k_2 - m \omega^2)}   \\
          B \left( 1 - \frac{k_2^2}{(k_1 + k_2 - M \omega^2) (k_2 - m \omega^2)} \right) & = \frac{F_0 k_2}{(k_1 + k_2 - M \omega^2) (k_2 - m \omega^2)}             \\
          B [(k_1 + k_2 - M \omega^2) (k_2 - m \omega^2) - k_2^2]                        & = F_0 k_2                                                                 \\
          \frac{F_0 k_2}{(k_1 + k_2 - M \omega^2) (k_2 - m \omega^2) - k_2^2}            & = B
        \end{align*}

  \item

        \begin{align*}
          A & = \frac{F_0 \left( k_2 - m \frac{k_1}{M} \right)}{\left( k_1 + k_2 - M \frac{k_1}{M} \right) \left( k_2 - m \frac{k_1}{M} \right) - k_2^2} \\
            & = \frac{F_0 \left( k_2 - \frac{k_2}{k_1} k_1 \right)}{\left( k_1 + k_2 - k_1 \right) \left( k_2 - \frac{k_2}{k_1} k_1 \right) - k_2^2}     \\
            & = 0
        \end{align*}
\end{enumerate}

\subsection{}

\begin{enumerate}
  \item

        \begin{align*}
          m \frac{d^2 x_1}{d t^2}                           & = -k x_1 + k (x_2 - x_1)         \\
          m \frac{d^2 x_1}{d t^2} + 2 k x_1 - k x_2         & = 0                              \\ \\
          m \frac{d^2 x_2}{d t^2}                           & = -k (x_2 - x_1) + k (x_3 - x_2) \\
          m \frac{d^2 x_2}{d t^2} - k x_1 + 2 k x_2 - k x_3 & = 0                              \\ \\
          m \frac{d^2 x_3}{d t^2}                           & = -k (x_3 - x_2) - k x_3         \\
          m \frac{d^2 x_3}{d t^2} - k x_2 + 2 k x_3         & = 0
        \end{align*}

        \begin{align*}
          -m \omega^2 A \cos \omega t + 2 k A \cos \omega t - k B \cos \omega t                     & = 0                    \\
          A (2k - m \omega^2)                                                                       & = B (k)                \\ \\
          -m \omega^2 B \cos \omega t - k A \cos \omega t + 2 k B \cos \omega t - k C \cos \omega t & = 0                    \\
          (A + C) (k)                                                                               & = B (2 k - m \omega^2) \\ \\
          -m \omega^2 C \cos \omega t - k B \cos \omega t + 2 k C \cos \omega t                     & = 0                    \\
          C (2 k - m \omega^2)                                                                      & = B (k)                \\ \\
          (A + C) (2 k - m \omega^2)                                                                & = B (2 k)
        \end{align*}

        \begin{align*}
          \frac{k}{2 k - m \omega^2}                                             & = \frac{2 k - m \omega^2}{2 k}                                                                        \\
          (2 k - m \omega^2)^2                                                   & = 2 k^2                                                                                               \\
          4 k^2 - 4 k m \omega^2 + m^2 \omega^4                                  & = 2 k ^2                                                                                              \\
          (\omega^2)^2 - 4 \frac{k}{m} \omega^2 + 2 \left( \frac{k}{m} \right)^2 & = 0                                                                                                   \\
          \omega^2                                                               & = \frac{4 \frac{k}{m} \pm \sqrt{\left( 4 \frac{k}{m} \right)^2 - 8 \left( \frac{k}{m} \right)^2}}{2}  \\
                                                                                 & = \frac{4 \frac{k}{m} \pm \sqrt{16 \left( \frac{k}{m} \right)^2 - 8 \left( \frac{k}{m} \right)^2}}{2} \\
          \omega                                                                 & = \sqrt{(2 \pm \sqrt{2}) \frac{k}{m}}
        \end{align*}

  \item

        \begin{enumerate}
          \item For $\omega = \sqrt{\frac{2 k}{m}}$

                \begin{align*}
                  A \left( 2 k - m \frac{2 k}{m} \right) & = B (k) \\
                  0                                      & = B (k)
                \end{align*}

                so $B = 0$ and

                \begin{align*}
                  (A + C) (k) & = 0  \\
                  A           & = -C
                \end{align*}

          \item For $\omega = \sqrt{(2 + \sqrt{2}) \frac{k}{m}}$

                \begin{align*}
                  A \left( 2 k - m (2 + \sqrt{2}) \frac{k}{m} \right) & = B (k)                 \\
                  A (2 k - 2 k - \sqrt{2} k)                          & = B (k)                 \\
                  A                                                   & = -\frac{1}{\sqrt{2}} B
                \end{align*}

                and

                \begin{align*}
                  C \left( 2 k - m (2 + \sqrt{2}) \frac{k}{m} \right) & = B (k)                 \\
                  C (2 k - 2 k - \sqrt{2} k)                          & = B (k)                 \\
                  C                                                   & = -\frac{1}{\sqrt{2}} B \\
                                                                      & = A
                \end{align*}

          \item For $\omega = \sqrt{(2 - \sqrt{2}) \frac{k}{m}}$

                \begin{align*}
                  A \left( 2 k - m (2 - \sqrt{2}) \frac{k}{m} \right) & = B (k)                \\
                  A                                                   & = \frac{1}{\sqrt{2}} B
                \end{align*}

                and

                \begin{align*}
                  C \left(2 k - m (2 - \sqrt{2}) \frac{k}{m} \right) & = B (k)                \\
                  C                                                  & = \frac{1}{\sqrt{2}} B \\
                                                                     & = A
                \end{align*}
        \end{enumerate}
\end{enumerate}

\end{document}