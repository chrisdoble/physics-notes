\documentclass{article}
\usepackage{amsmath} % For align*
\usepackage{amsfonts} % For open face letters
\usepackage{enumitem} % For customisable list labels
\usepackage{graphicx} % For images
\usepackage{siunitx} % For units
\graphicspath{{./images/}}

\renewcommand{\vec}[1]{\boldsymbol{\mathbf{#1}}}
\newcommand{\dvec}[1]{\dot{\vec{#1}}}
\newcommand{\ddvec}[1]{\ddot{\vec{#1}}}
\newcommand{\uvec}[1]{\hat{\vec{#1}}}
\newcommand{\arcsinh}{\operatorname{arcsinh}}
\newcommand{\arctanh}{\operatorname{arctanh}}
\newcommand{\crd}{\operatorname{crd}}
\renewcommand{\Re}{\operatorname{Re}}

\setlist[enumerate, 1]{label={(\alph*)}}
\setlist[enumerate, 2]{label={(\roman*)}}

\title{Classical Mechanics by John R. Taylor Problems}
\author{Chris Doble}
\date{August 2023}

\begin{document}

\maketitle

\tableofcontents

\section{Newton's Laws of Motion}

\subsection{}

\begin{align*}
  \vec{b} + \vec{c}      & = 2 \uvec{x} + \uvec{y} + \uvec{z}     \\
  5 \vec{b} + 2 \vec{x}  & = 7 \uvec{x} + 5 \uvec{y} + 2 \uvec{z} \\
  \vec{b} \cdot \vec{c}  & = 1                                    \\
  \vec{b} \times \vec{c} & = \begin{vmatrix}
                               \uvec{x} & \uvec{y} & \uvec{z} \\
                               1        & 1        & 0        \\
                               1        & 0        & 1
                             \end{vmatrix}       \\
                         & = \uvec{x} - \uvec{y} - \uvec{z}
\end{align*}

\setcounter{subsection}{4}
\subsection{}

\begin{align*}
  \vec{v}_\text{body}                           & = \uvec{x} + \uvec{y} + \uvec{z}          \\
  \vec{v}_\text{face}                           & = \uvec{x} + \uvec{z}                     \\
  \vec{v}_\text{body} \cdot \vec{v}_\text{face} & = v_\text{body} v_\text{face} \cos \theta \\
  2                                             & = \sqrt{6} \cos \theta                    \\
  \cos \theta                                   & = \frac{2}{\sqrt{6}}                      \\
  \theta                                        & = \arccos \frac{2}{\sqrt{6}}              \\
                                                & = \ang{35.26}
\end{align*}

\setcounter{subsection}{10}
\subsection{}

The particle moves counterclockwise in an ellipse of width $2 b$ and height $2 c$. The angular speed is $\omega$.

\setcounter{subsection}{22}
\subsection{}

\begin{align*}
  \vec{v} & = v \cos \theta \frac{\vec{b}}{b} - v \sin \theta \frac{\vec{b} \times \vec{c}}{b c}   \\
          & = \frac{\lambda}{b} \frac{\vec{b}}{b} - \frac{c}{b} \frac{\vec{b} \times \vec{c}}{b c} \\
          & = \frac{\lambda \vec{b} - \vec{b} \times \vec{c}}{b^2}
\end{align*}

\setcounter{subsection}{24}
\subsection{}

\begin{align*}
  \frac{d f}{d t}             & = -3 f       \\
  \frac{1}{f} \frac{d f}{d t} & = -3         \\
  \ln f                       & = -3 t + c   \\
  f                           & = c e^{-3 t}
\end{align*}

One constant.

\setcounter{subsection}{34}
\subsection{}

\begin{align*}
  F_x   & = 0                                             \\
  m a_x & = 0                                             \\
  a_x   & = 0                                             \\
  v_x   & = c_1                                           \\
        & = v_o \cos \theta                               \\
  r_x   & = v_o \cos (\theta) t + c_2                     \\
        & = v_o \cos (\theta) t                           \\ \\
  F_y   & = 0                                             \\
  m a_y & = 0                                             \\
  a_y   & = 0                                             \\
  v_y   & = c_3                                           \\
  v_y   & = 0                                             \\
  r_y   & = c_4                                           \\
  r_y   & = 0                                             \\ \\
  F_z   & = -m g                                          \\
  m a_z & = -m g                                          \\
  a_z   & = -g                                            \\
  v_z   & = -g t + c_5                                    \\
        & = v_o \sin \theta - g t                         \\
  r_z   & = v_o \sin (\theta) t - \frac{1}{2} g t^2 + c_6 \\
        & = v_o \sin (\theta) t - \frac{1}{2} g t^2       \\ \\
  0     & = v_o \sin (\theta) t - \frac{1}{2} g t^2       \\
  t     & = \frac{2 \sin (\theta) v_o}{g}                 \\
  r_x   & = v_o \cos (\theta) t                           \\
        & = \frac{2 \cos (\theta) \sin (\theta) v_o^2}{g} \\
        & = \frac{\sin (2 \theta) v_o^2}{g}
\end{align*}

\setcounter{subsection}{36}
\subsection{}

\begin{enumerate}
  \item

        \begin{align*}
          F   & = -m g \sin \theta                      \\
          m a & = -m g \sin \theta                      \\
          a   & = -g \sin \theta                        \\
          v   & = c_1 - g t \sin \theta                 \\
              & = v_o - g t \sin \theta                 \\
          x   & = v_o t - \frac{1}{2} g t^2 \sin \theta
        \end{align*}

  \item \[t = \frac{2 v_o}{g \sin \theta}\]
\end{enumerate}

\setcounter{subsection}{38}
\subsection{}

\begin{align*}
  F_x   & = -m g \sin \phi                                                                                            \\
  m a_x & = -m g \sin \phi                                                                                            \\
  a_x   & = -g \sin \phi                                                                                              \\
  v_x   & = c_1 - g t \sin \phi                                                                                       \\
        & = v_o \cos \theta - g t \sin \phi                                                                           \\
  r_x   & = v_o t \cos \theta - \frac{1}{2} g t^2 \sin \phi + c_2                                                     \\
        & = v_o t \cos \theta - \frac{1}{2} g t^2 \sin \phi                                                           \\ \\
  F_y   & = -m g \cos \phi                                                                                            \\
  m a_y & = -m g \cos \phi                                                                                            \\
  a_y   & = -g \cos \phi                                                                                              \\
  v_y   & = c_3 - g t \cos \phi                                                                                       \\
        & = v_o \sin \theta - g t \cos \phi                                                                           \\
  r_y   & = v_o t \sin \theta - \frac{1}{2} g t^2 \cos \phi + c_4                                                     \\
        & = v_o t \sin \theta - \frac{1}{2} g t^2 \cos \phi                                                           \\ \\
  0     & = v_o t \sin \theta - \frac{1}{2} g t^2 \cos \phi                                                           \\
  t     & = \frac{2 v_o \sin \theta}{g \cos \phi}                                                                     \\ \\
  r_x   & = \frac{2 v_o^2 \cos \theta \sec \phi \sin \theta}{g} - \frac{2 v_o^2 \sec \phi \sin^2 \theta \tan \phi}{g} \\
        & = \frac{2 v_o^2 \sin \theta (\cos \theta \cos \phi - \sin \theta \sin \phi)}{g \cos^2 \phi}                 \\
        & = \frac{2 v_o^2 \sin \theta \cos (\theta + \phi)}{g \cos^2 \phi}
\end{align*}

\begin{align*}
  \frac{d r_x}{d \theta} & = \frac{2 v_o^2}{g \cos^2 \phi} [\cos \theta \cos (\theta + \phi) - \sin \theta \sin (\theta + \phi)]                                        \\
                         & = \frac{2 v_o^2 \cos (2 \theta + \phi)}{g \cos^2 \phi}                                                                                       \\
  0                      & = \frac{2 v_o^2 \cos (2 \theta + \phi)}{g \cos^2 \phi}                                                                                       \\
                         & = \cos (2 \theta + \phi)                                                                                                                     \\
  2 \theta + \phi        & = \frac{\pi}{2}                                                                                                                              \\
  \theta                 & = \frac{\pi}{4} - \frac{\phi}{2}                                                                                                             \\
  r_{x\text{,max}}       & = \frac{2 v_o^2 \sin \left( \frac{\pi}{4} - \frac{\phi}{2} \right) \cos \left( \frac{\pi}{4} - \frac{\phi}{2} + \phi \right)}{g \cos^2 \phi} \\
                         & = \frac{v_o^2 (1 - \sin \phi)}{g \cos^2 \phi}                                                                                                \\
                         & = \frac{v_o^2}{g (1 + \sin \phi)}
\end{align*}

\setcounter{subsection}{40}
\subsection{}

\begin{align*}
  F & = m a                      \\
  T & = m \frac{v^2}{R}          \\
    & = m \frac{(\omega R)^2}{R} \\
    & = m \omega^2 R
\end{align*}

\setcounter{subsection}{46}
\subsection{}

\begin{enumerate}
  \item

        \begin{align*}
          \rho & = \sqrt{x^2 + y^2}    \\
          \phi & = \arctan \frac{y}{x} \\
          z    & = z
        \end{align*}

        $\rho$ is the distance of $P$ from the $z$-axis.

        The use of $r$ may be unfortunate because it suggests it's the distance of $P$ from the origin.

  \item $\uvec{\rho}$ points away from the $z$-axis, $\uvec{\phi}$ points counter-clockwise around the $z$-axis, and $\uvec{z}$ points in the positive $z$ direction.

        \[\vec{r} = \rho \uvec{\rho} + z \uvec{z} + \sqrt{x^2 + y^2} \uvec{\rho} + z \uvec{z}\]

  \item

        \begin{align*}
          \vec{v} & = \frac{d \vec{r}}{d t}                                                                                                                                                                             \\
                  & = \dot{\rho} \uvec{\rho} + \rho \frac{d \uvec{\rho}}{d t} + \dot{z} \uvec{z} + z \frac{d \uvec{z}}{d t}                                                                                             \\
                  & = \dot{\rho} \uvec{\rho} + \rho \dot{\phi} \uvec{\phi} + \dot{z} \uvec{z}                                                                                                                           \\
          \vec{a} & = \frac{d \vec{v}}{d t}                                                                                                                                                                             \\
                  & = \ddot{\rho} \uvec{\rho} + \dot{\rho} \frac{d \uvec{\rho}}{d t} + \dot{\rho} \dot{\phi} \uvec{\phi} + \rho \ddot{\phi} \uvec{\phi} + \rho \dot{\phi} \frac{d \uvec{\phi}}{d t} + \ddot{z} \uvec{z} \\
                  & = \ddot{\rho} \uvec{\rho} + \dot{\rho} \dot{\phi} \uvec{\phi} + \dot{\rho} \dot{\phi} \uvec{\phi} + \rho \ddot{\phi} \uvec{\phi} - \rho \dot{\phi}^2 \uvec{\rho} + \ddot{z} \uvec{z}                \\
                  & = (\ddot{\rho} - \rho \dot{\phi}^2) \uvec{\rho} + (2 \dot{\rho} \dot{\phi} + \rho \ddot{\phi}) \uvec{\phi} + \ddot{z} \uvec{z}
        \end{align*}
\end{enumerate}

\section{Projectiles and Charged Particles}

\subsection{}

\begin{align*}
  1 & = (\num{1.6e3}) D v         \\
  v & = \frac{1}{(\num{1.6e3}) D} \\
    & = \qty{8.9}{mm/s}
\end{align*}

When $v \gg \qty{1}{cm/s}$ the drag force can be treated as purely quadratic. For a beach ball this becomes $v \gg \qty{1}{mm/s}$.

\setcounter{subsection}{2}
\subsection{}

\begin{enumerate}
  \item

        \begin{align*}
          \frac{f_\text{quad}}{f_\text{lin}} & = \frac{(1 / 4) \rho A v^2}{3 \pi \eta D v}                     \\
                                             & = \frac{\rho \pi \left( \frac{D}{2} \right)^2 v}{12 \pi \eta D} \\
                                             & = \frac{\rho D v}{48 \eta}                                      \\
                                             & = \frac{R}{48}
        \end{align*}

  \item \[R = \frac{D v \rho}{\eta} \approx 0.01\]
\end{enumerate}

\setcounter{subsection}{4}
\subsection{}

\begin{align*}
  v_y(t) & = v_\text{ter} + (v_\text{yo} - v_\text{ter}) e^{-t / \tau}    \\
         & = v_\text{ter} + (2 v_\text{ter} - v_\text{ter}) e^{-t / \tau} \\
         & = v_\text{ter} (1 + e^{-t / \tau})
\end{align*}

The velocity starts at $2 v_\text{ter}$ and asymptotically approaches $v_\text{ter}$.

\setcounter{subsection}{6}
\subsection{}

\begin{align*}
  F                  & = F(v)                                     \\
  m \dot{v}          & = F(v)                                     \\
  m \frac{d v}{F(v)} & = d t                                      \\
  t                  & = \int_{v_\text{o}}^v m \frac{d v'}{F(v')} \\ \\
  F                  & = F(v)                                     \\
  m \dot{v}          & = F_\text{o}                               \\
  v                  & = \frac{F_\text{o}}{m} t + c
\end{align*}

\setcounter{subsection}{10}
\subsection{}

\begin{enumerate}
  \item

        \begin{align*}
          m \dot{v}                   & = -m g - b v                                                             \\
          \dot{v}                     & = -g - k v                                                               \\
          \frac{1}{-g - k v} \dot{v}  & = 1                                                                      \\
          -\frac{1}{k} \ln (-g - k v) & = t + c                                                                  \\
          \ln (-g - k v)              & = c - \frac{t}{\tau}                                                     \\
          -g - k v                    & = A e^{-t / \tau}                                                        \\
          v                           & = \tau (-g - A e^{-t / \tau})                                            \\
                                      & = -v_\text{ter} - \tau A e^{-t / \tau}                                   \\ \\
          v_\text{o}                  & = -v_\text{ter} - \tau A                                                 \\
          A                           & = -k (v_\text{o} + v_\text{ter})                                         \\ \\
          v                           & = -v_\text{ter} + (v_\text{o} + v_\text{ter}) e^{-t / \tau}              \\ \\
          y                           & = -v_\text{ter} t - \tau (v_\text{o} + v_\text{ter}) e^{-t / \tau} + c   \\ \\
          0                           & = -\tau (v_\text{o} + v_\text{ter}) + c                                  \\
          c                           & = \tau (v_\text{o} + v_\text{ter})                                       \\ \\
          y                           & = -v_\text{ter} t + \tau (v_\text{o} + v_\text{ter}) (1 - e^{-t / \tau})
        \end{align*}

  \item

        \begin{align*}
          0               & = -v_\text{ter} + (v_\text{o} + v_\text{ter}) e^{-t / \tau}                                                                                                                                  \\
          e^{-t / \tau}   & = \frac{v_\text{ter}}{v_\text{o} + v_\text{ter}}                                                                                                                                             \\
          -\frac{t}{\tau} & = \ln \frac{v_\text{ter}}{v_\text{o} + v_\text{ter}}                                                                                                                                         \\
          t               & = -\tau \ln \frac{v_\text{ter}}{v_\text{o} + v_\text{ter}}                                                                                                                                   \\ \\
          y_\text{max}    & = -v_\text{ter} \left( -\tau \ln \frac{v_\text{ter}}{v_\text{o} + v_\text{ter}} \right) + \tau (v_\text{o} + v_\text{ter}) \left( 1 - \frac{v_\text{ter}}{v_\text{o} + v_\text{ter}} \right) \\
                          & = \tau v_\text{ter} \ln \frac{v_\text{ter}}{v_\text{o} + v_\text{ter}} + \tau (v_\text{o} + v_\text{ter} - v_\text{ter})                                                                     \\
                          & = \tau \left( v_\text{o} + v_\text{ter} \ln \frac{v_\text{ter}}{v_\text{o} + v_\text{ter}} \right)                                                                                           \\
                          & = \tau \left[ v_\text{o} - v_\text{ter} \ln \left( 1 + \frac{v_\text{o}}{v_\text{ter}} \right) \right]
        \end{align*}

  \item

        \begin{align*}
          y_\text{max} & = \tau \left[ v_\text{o} - v_\text{ter} \ln \left( 1 + \frac{v_\text{o}}{v_\text{ter}} \right) \right]                                              \\
                       & = \tau \left[ v_\text{o} - g \tau \ln \left( 1 + \frac{v_\text{o}}{g \tau} \right) \right]                                                          \\
                       & \approx \tau \left\{ v_\text{o} - g \tau \left[ \frac{v_\text{o}}{g \tau} - \frac{1}{2} \left( \frac{v_\text{o}}{g \tau} \right)^2 \right] \right\} \\
                       & = \tau \left( v_\text{o} - v_\text{o} + \frac{1}{2} \frac{v_\text{o}^2}{g \tau} \right)                                                             \\
                       & = \frac{1}{2} \frac{v_\text{o}^2}{g}
        \end{align*}
\end{enumerate}

\setcounter{subsection}{12}
\subsection{}

\begin{align*}
  v^2                                                                             & = \frac{2}{m} \int_{x_0}^x -k x' \,d x'                             \\
                                                                                  & = -\frac{2 k}{m} \left( \frac{1}{2} x^2 - \frac{1}{2} x_0^2 \right) \\
                                                                                  & = -\frac{k}{m} (x^2 - x_0^2)                                        \\
  v                                                                               & = \sqrt{\frac{k}{m} (x_0^2 - x^2)}                                  \\
                                                                                  & = \omega \sqrt{x_0^2 - x^2}                                         \\ \\
  \int_{x_0}^x \frac{1}{\sqrt{x_0^2 - x'^2}} \,d x'                               & = \int_0^t \omega \,d t                                             \\
  \arctan \frac{x}{\sqrt{x_0^2 - x^2}} - \arctan \frac{x_0}{\sqrt{x_0^2 - x_0^2}} & = \omega t                                                          \\
  \arctan \frac{x}{\sqrt{x_0^2 - x^2}}                                            & = \omega t + \frac{\pi}{2}                                          \\
  \frac{x}{\sqrt{x_0^2 - x^2}}                                                    & = \tan \left( \omega t + \frac{\pi}{2} \right)                      \\
                                                                                  & = -\cot \omega t                                                    \\
  \frac{\sqrt{x_0^2 - x^2}}{x}                                                    & = -\tan \omega t                                                    \\
  \sqrt{x_0^2 - x^2}                                                              & = -x \tan \omega t                                                  \\
  x_0^2 - x^2                                                                     & = x^2 \tan^2 \omega t                                               \\
  x^2                                                                             & = \frac{x_0^2}{1 + \tan^2 \omega t}                                 \\
                                                                                  & = \frac{x_0^2 \cos^2 \omega t}{\cos^2 \omega t + \sin^2 \omega t}   \\
                                                                                  & = x_0^2 \cos^2 \omega t                                             \\
  x                                                                               & = x_0 \cos \omega t
\end{align*}

\setcounter{subsection}{14}
\subsection{}

\begin{align*}
  a_y & = -g                           \\
  v_y & = v_{y0} - g t                 \\
  y   & = v_{y0} t - \frac{1}{2} g t^2 \\
  0   & = v_{y0} t - \frac{1}{2} g t^2 \\
  t   & = \frac{2 v_{y0}}{g}           \\
  x   & = v_{x0} t                     \\
  R   & = \frac{2 v_{x0} v_{y0}}{g}
\end{align*}

\setcounter{subsection}{18}
\subsection{}

\begin{enumerate}
  \item

        \begin{align*}
          x & = v_{x0} t                                                                  \\
          y & = v_{y0} t- \frac{1}{2} g t^2                                               \\
            & = \frac{v_{y0}}{v_{x0}} x - \frac{1}{2} g \left( \frac{x}{v_{x0}} \right)^2
        \end{align*}

  \item

        \begin{align*}
          y & = \frac{v_{y0} + v_\text{ter}}{v_{x0}} x + v_\text{ter} \tau \ln \left( 1 - \frac{x}{v_{x0} \tau} \right)                                                      \\
            & \approx \frac{v_{y0}}{v_{x0}} x + \frac{g \tau}{v_{x0}} x - g \tau^2 \left[ \frac{x}{v_{x0} \tau} + \frac{1}{2} \left( \frac{x}{v_{x0} \tau} \right)^2 \right] \\
            & = \frac{v_{y0}}{v_{x0}} x - \frac{1}{2} g \left( \frac{x}{v_{x0}} \right)^2
        \end{align*}
\end{enumerate}

% Section 2.4

\setcounter{subsection}{22}
\subsection{}

\begin{enumerate}
  \item

        \begin{align*}
          v_\text{ter} & = \sqrt{\frac{m g}{c}}                                                        \\
                       & = \sqrt{\frac{m g}{\gamma D^2}}                                               \\
                       & = \sqrt{\frac{m g}{0.25 D^2}}                                                 \\
                       & = \sqrt{\frac{\rho \frac{4}{3} \pi \left( \frac{D}{2} \right)^3 g}{0.25 D^2}} \\
                       & = \sqrt{\frac{4 \pi \rho D g}{6}}                                             \\
                       & = \qty{22}{m/s}
        \end{align*}

  \item

        \begin{align*}
          m            & = \rho V                                            \\
                       & = \rho \frac{4}{3} \pi \left( \frac{D}{2} \right)^3 \\
                       & = \frac{\pi \rho D^3}{6}                            \\
          D^2          & = \left( \frac{6 m}{\pi \rho} \right)^{2 / 3}       \\
          v_\text{ter} & = \sqrt{\frac{m g}{0.25 D^2}}                       \\
                       & = \sqrt{\frac{m g}{0.25 (6 m / \pi \rho)^{2 / 3}}}  \\
                       & = \qty{140}{m/s}
        \end{align*}

  \item \[v_\text{ter} = \qty{107}{m/s}\]
\end{enumerate}

\setcounter{subsection}{26}
\subsection{}

\begin{align*}
  m \dot{v}                                                                                  & = -m g \sin \theta - c v^2                                                                        \\
  -\frac{\sqrt{m} \arctan \frac{\sqrt{c} v}{\sqrt{g m \sin \theta}}}{\sqrt{c g \sin \theta}} & = t + c_1                                                                                         \\
  \arctan \frac{\sqrt{c} v}{\sqrt{g m \sin \theta}}                                          & = \sqrt{\frac{c g \sin \theta}{m}} (c_1 - t)                                                      \\
  \frac{\sqrt{c} v}{\sqrt{g m \sin \theta}}                                                  & = \tan \left[ \sqrt{\frac{c g \sin \theta}{m}} (c_1 - t) \right]                                  \\
  v                                                                                          & = \sqrt{\frac{g m \sin \theta}{c}} \tan \left[ \sqrt{\frac{c g \sin \theta}{m}} (c_1 - t) \right] \\ \\
  v_0                                                                                        & = \sqrt{\frac{g m \sin \theta}{c}} \tan \left( \sqrt{\frac{c g \sin \theta}{m}} c_1 \right)       \\
  c_1                                                                                        & = \sqrt{\frac{m}{c g \sin \theta}} \arctan \left( \sqrt{\frac{c}{g m \sin \theta}} v_0 \right)
\end{align*}

\begin{align*}
  v                                  & = \sqrt{\frac{g m \sin \theta}{c}} \tan \left[ \arctan \left( \sqrt{\frac{c}{g m \sin \theta}} v_0 \right) - \sqrt{\frac{c g \sin \theta}{m}} t \right] \\ \\
  0                                  & = \sqrt{\frac{g m \sin \theta}{c}} \tan \left[ \arctan \left( \sqrt{\frac{c}{g m \sin \theta}} v_0 \right) - \sqrt{\frac{c g \sin \theta}{m}} t \right] \\
  \sqrt{\frac{c g \sin \theta}{m}} t & = \arctan \left( \sqrt{\frac{c}{g m \sin \theta}} v_0 \right)                                                                                           \\
  t                                  & = \sqrt{\frac{m}{c g \sin \theta}} \arctan \left( \sqrt{\frac{c}{g m \sin \theta}} v_0 \right)
\end{align*}

\setcounter{subsection}{28}
\subsection{}

\begin{align*}
  v(t)  & = v_\text{ter} \tanh \frac{g t}{v_\text{ter}} \\
  v(1)  & = \qty{9.6}{m/s}                              \\
  v(5)  & = \qty{38}{m/s}                               \\
  v(10) & = \qty{48}{m/s}                               \\
  v(20) & = \qty{50}{m/s}                               \\
  v(30) & = \qty{50}{m/s}
\end{align*}

\setcounter{subsection}{30}
\subsection{}

\begin{enumerate}
  \item

        \begin{align*}
          v_\text{ter} & = \sqrt{\frac{m g}{c}}        \\
                       & = \sqrt{\frac{m g}{0.25 D^2}} \\
                       & = \qty{20.2}{m/s}
        \end{align*}

  \item

        \begin{align*}
          y       & = -30 + \frac{v_\text{ter}^2}{g} \ln \left( \cosh \frac{g t}{v_\text{ter}} \right) \\
          0       & = -30 + \frac{v_\text{ter}^2}{g} \ln \left( \cosh \frac{g t}{v_\text{ter}} \right) \\
          t       & = \qty{2.78}{s}                                                                    \\
          v(2.78) & = \qty{17.6}{m/s}
        \end{align*}
\end{enumerate}

\setcounter{subsection}{32}
\subsection{}

\begin{enumerate}
  \setcounter{enumi}{1}
  \item

        \begin{align*}
          \cosh z  & = \frac{e^z + e ^{-z}}{2}                                                                                                                                  \\
                   & = \frac{1}{2} \left[ \left( 1 + z + \frac{z^2}{2} + \frac{z^3}{6} + \cdots \right) + \left( 1 - z + \frac{z^2}{2} - \frac{z^3}{6} + \cdots \right) \right] \\
                   & = 1 + \frac{z^2}{2} + \frac{z^4}{24} + \frac{z^6}{720} + \cdots                                                                                            \\
          \cos i z & = 1 - \frac{(i z)^2}{2} + \frac{(i z)^4}{24} - \frac{(i z)^6}{720} + \cdots                                                                                \\
                   & = 1 + \frac{z^2}{2} + \frac{z^4}{24} + \frac{z^6}{720} + \cdots                                                                                            \\
                   & = \cosh z                                                                                                                                                  \\
          \sinh z  & = -i \sin i z
        \end{align*}

  \item

        \begin{align*}
          \frac{d}{d z} \cosh z & = \frac{d}{d z} \left( \frac{e^z + e^{-z}}{2} \right) \\
                                & = \frac{e^z - e^{-z}}{2}                              \\
                                & = \sinh z                                             \\
          \frac{d}{d z} \sinh z & = \frac{d}{d z} \left( \frac{e^z - e^{-z}}{2} \right) \\
                                & = \frac{e^z + e^{-z}}{2}                              \\
                                & = \cosh z
        \end{align*}

  \item

        \begin{align*}
          \cosh^2 z - \sinh^2 z & = \left( \frac{e^z + e^{-z}}{2} \right)^2 - \left( \frac{e^z - e^{-z}}{2} \right)^2 \\
                                & = \frac{1}{4} (e^{2 z} + 2 + e^{-2 z} - e^{2 z} + 2 - e^{-2 z})                     \\
                                & = 1
        \end{align*}

  \item

        \begin{align*}
          \int \frac{1}{\sqrt{1 + x^2}} \,d x & = \int \frac{\cosh z}{\sqrt{1 + \sinh^2 z}} \,d z \\
                                              & = \int 1 \,d z                                    \\
                                              & = z                                               \\
                                              & = \arcsinh x
        \end{align*}
\end{enumerate}

\setcounter{subsection}{34}
\subsection{}

\begin{enumerate}
  \item

        \begin{align*}
          m \dot{v}                                           & = m g - c v^2                                                                               \\
          \dot{v}                                             & = g \left( 1 - \frac{v^2}{v_\text{ter}^2} \right)                                           \\
          \int_0^v \frac{1}{1 - v'^2 / v_\text{ter}^2} \,d v' & = \int_0^t g \,d t                                                                          \\
          v_\text{ter} \arctanh \frac{v}{v_\text{ter}}        & = g t                                                                                       \\
          v                                                   & = v_\text{ter} \tanh \frac{g t}{v_\text{ter}}                                               \\ \\
          y                                                   & = \int_0^t v_\text{ter} \tanh \frac{g t'}{v_\text{ter}} d t'                                \\
                                                              & = \frac{v_\text{ter}^2}{g} \ln \left[ \cosh \left( \frac{g t}{v_\text{ter}} \right) \right]
        \end{align*}

  \item

        \begin{align*}
          v         & = g \tau \tanh \frac{t}{\tau}                                     \\
          y         & = g \tau^2 \ln \left[ \cosh \left( \frac{t}{\tau} \right) \right] \\
          v(\tau)   & = g \tau \tanh 1                                                  \\
                    & = 0.76 v_\text{ter}                                               \\
          v(2 \tau) & = 0.96 v_\text{ter}                                               \\
          v(3 \tau) & = 0.99 v_\text{ter}
        \end{align*}

  \item

        \begin{align*}
          y & = g \tau^2 \ln \left[ \cosh \left( \frac{t}{\tau} \right) \right]    \\
            & = g \tau^2 \ln \left( \frac{e^{t / \tau} + e^{-t / \tau}}{2} \right) \\
            & = g \tau^2 \ln \left( \frac{e^{t / \tau}}{2} \right)                 \\
            & = g \tau^2 (\ln e^{t / \tau} - \ln 2)                                \\
            & = g \tau t - g \tau^2 \ln 2                                          \\
            & = v_\text{ter} t - g \tau^2 \ln 2
        \end{align*}

  \item

        \begin{align*}
          y & = \frac{(v_\text{ter})^2}{g} \ln \left[ \cosh \left( \frac{g t}{v_\text{ter}} \right) \right]                  \\
            & \approx \frac{(v_\text{ter})^2}{g} \ln \left[ 1 + \frac{1}{2} \left( \frac{g t}{v_\text{ter}} \right)^2\right] \\
            & \approx \frac{(v_\text{ter})^2}{g} \frac{1}{2} \left( \frac{g t}{v_\text{ter}} \right)^2                       \\
            & = \frac{1}{2} g t^2
        \end{align*}
\end{enumerate}

\setcounter{subsection}{38}
\subsection{}

\begin{enumerate}
  \item

        \begin{align*}
          m \dot{v}                                                                                                                       & = -c v^2 - 3    \\
          \int_{v_0}^v \frac{m}{-c v'^2 - 3} \,d v'                                                                                       & = \int_0^t d t' \\
          \frac{m}{\sqrt{3 c}} \left[ \arctan \left( \sqrt{\frac{c}{3}} v_0 \right) - \arctan \left( \sqrt{\frac{c}{3}} v \right) \right] & = t
        \end{align*}

  \item

        \begin{tabular}{c c}
          Speed         & Time          \\
          \hline
          \qty{15}{m/s} & \qty{6.34}{s} \\
          \qty{10}{m/s} & \qty{18.4}{s} \\
          \qty{5}{m/s}  & \qty{48.3}{s} \\
          \qty{0}{m/s}  & \qty{142}{s}
        \end{tabular}
\end{enumerate}

\setcounter{subsection}{40}
\subsection{}

\begin{align*}
  m \dot{v}                                                                              & = -m g - c v^2                                                                  \\
  \dot{v}                                                                                & = -g \left[ 1 + \left( \frac{v}{v_\text{ter}} \right)^2 \right]                 \\
  v \frac{d v}{d y}                                                                      & = -g \left[ 1 + \left( \frac{v}{v_\text{ter}} \right)^2 \right]                 \\
  \int_{v_0}^v \frac{v'}{1 + (v' / v_\text{ter})^2} \,d v'                               & = \int_0^y -g \,d y'                                                            \\
  \frac{1}{2} v_\text{ter}^2 [\ln (v_\text{ter}^2 + v^2) - \ln (v_\text{ter}^2 + v_0^2)] & = -g y                                                                          \\
  \ln \frac{v_\text{ter}^2 + v^2}{v_\text{ter}^2 + v_0^2}                                & = -\frac{2 g y}{v_\text{ter}^2}                                                 \\
  \frac{v_\text{ter}^2 + v^2}{v_\text{ter}^2 + v_0^2}                                    & = e^{-2 g y / v_\text{ter}^2}                                                   \\
  v                                                                                      & = \sqrt{(v_\text{ter}^2 + v_0^2) e^{-2 g y / v_\text{ter}^2} - v_\text{ter}^2}  \\ \\
  0                                                                                      & = \sqrt{(v_\text{ter}^2 + v_0^2) e^{-2 g y / v_\text{ter}^2} - v_\text{ter}^2}  \\
  v_\text{ter}^2                                                                         & = (v_\text{ter}^2 + v_0^2) e^{-2 g y / v_\text{ter}^2}                          \\
  \frac{v_\text{ter}^2}{v_\text{ter}^2 + v_0^2}                                          & = e^{-2 g y / v_\text{ter}^2}                                                   \\
  -\frac{2 g y}{v_\text{ter}^2}                                                          & = \ln \frac{v_\text{ter}^2}{v_\text{ter}^2 + v_0^2}                             \\
  y                                                                                      & = -\frac{v_\text{ter}^2}{2 g} \ln \frac{v_\text{ter}^2}{v_\text{ter}^2 + v_0^2} \\
                                                                                         & = \frac{v_\text{ter}^2}{2 g} \ln \frac{v_\text{ter}^2 + v_0^2}{v_\text{ter}^2}  \\ \\
  y_\text{max}                                                                           & = \qty{17.6}{m}
\end{align*}

\setcounter{subsection}{44}
\subsection{}

\begin{enumerate}
  \item

        \begin{align*}
          z & = r e^{i \theta}                                                                                                      \\
            & = r (\cos \theta + i \sin \theta)                                                                                     \\
            & = \sqrt{x^2 + y^2} \left[ \cos \left( \arctan \frac{y}{x} \right) + i \sin \left( \arctan \frac{y}{x} \right) \right] \\
            & = \sqrt{x^2 + y^2} \left( \frac{1}{\sqrt{1 + \frac{y^2}{x^2}}} + i \frac{y}{x \sqrt{1 + \frac{y^2}{x^2}}} \right)     \\
            & = \sqrt{x^2 + y^2} \left( \frac{x}{\sqrt{x^2 + y^2}} + i \frac{y}{\sqrt{x^2 + y^2}} \right)                           \\
            & = x + i y
        \end{align*}

        $r$ is the distance between $z$ and the origin, $\theta$ is the angle between the positive real axis and $z$.

  \item \[z = \sqrt{3^2 + 4^2} e^{i \arctan 4 / 3} = 5 e^{0.927 i}\]

  \item \[z = 2 \cos -\frac{\pi}{3} + i 2 \sin -\frac{\pi}{3} = 1 - \sqrt{3} i\]
\end{enumerate}

\setcounter{subsection}{46}
\subsection{}

\begin{enumerate}
  \item

        \begin{align*}
          z + w       & = 9 + 4 i                                            \\
          z - w       & = 3 + 12 i                                           \\
          z w         & = (6 + 8 i) (3 - 4 i)                                \\
                      & = 18 - 24 i + 24 i + 32                              \\
                      & = 50                                                 \\
          \frac{z}{w} & = \frac{z w*}{w w*}                                  \\
                      & = \frac{(6 + 8 i) (3 + 4 i)}{(3 - 4 i) (3 + 4 i)}    \\
                      & = \frac{18 + 24 i + 24 i - 32}{9 + 12 i - 12 i + 16} \\
                      & = \frac{-14 + 48 i}{25}                              \\
                      & = -\frac{14}{25} + \frac{48}{25} i
        \end{align*}

  \item

        \begin{align*}
          z + w       & = \left( 8 \cos \frac{\pi}{3} + i 8 \sin \frac{\pi}{3} \right) + \left( 4 \cos \frac{\pi}{6} + i 4 \sin \frac{\pi}{6} \right) \\
                      & = (4 + 2 \sqrt{3}) + i (4 \sqrt{3} + 2)                                                                                       \\
          z - w       & = (4 - 2 \sqrt{3}) + i (4 \sqrt{3} - 2)                                                                                       \\
          z w         & = 32 e^{i \pi / 2}                                                                                                            \\
                      & = 32 i                                                                                                                        \\
          \frac{z}{w} & = 2 e^{i \pi / 6}                                                                                                             \\
                      & = \sqrt{3} + i
        \end{align*}
\end{enumerate}

\setcounter{subsection}{48}
\subsection{}

\begin{enumerate}
  \item

        \begin{align*}
          z^2                             & = (e^{i \theta})^2                                            \\
                                          & = e^{i 2 \theta}                                              \\
                                          & = \cos 2 \theta + i \sin 2 \theta                             \\
          z^2                             & = (\cos \theta + i \sin \theta)^2                             \\
                                          & = \cos^2 \theta + 2 i \cos \theta \sin \theta - \sin^2 \theta \\
                                          & = \cos^2 \theta + i \sin 2 \theta - \sin^2 \theta             \\ \\
          \cos 2 \theta                   & = \cos^2 \theta - \sin^2 \theta                               \\ \\
          \cos 2 \theta + i \sin 2 \theta & = \cos^2 \theta + 2 i \sin \theta \cos \theta - \sin^2 \theta \\
          i \sin 2 \theta                 & = 2 i \sin \theta \cos \theta                                 \\
          \sin 2 \theta                   & = 2 \sin \theta \cos \theta
        \end{align*}

  \item

        \begin{align*}
          z^3                             & = (e^{i \theta})^3                                                                                      \\
                                          & = e^{i 3 \theta}                                                                                        \\
                                          & = \cos 3 \theta + i \sin 3 \theta                                                                       \\
          z^3                             & = (\cos \theta + i \sin \theta)^3                                                                       \\
                                          & = \cos^3 \theta + 3 i \cos^2 \theta \sin \theta - 3 \cos \theta \sin^2 \theta - i \sin^3 \theta         \\ \\
          \cos 3 \theta + i \sin 3 \theta & = \cos^3 \theta + 3 i \cos^2 \theta \sin \theta - 3 \cos \theta \sin^2 \theta - i \sin^3 \theta         \\
                                          & = \cos \theta (\cos^2 \theta - 3 \sin^2 \theta) + i (3 \cos^2 \theta \sin \theta - \sin^3 \theta)       \\
                                          & = \cos \theta (\cos^2 \theta - 3 \sin^2 \theta) + i [3 (1 - \sin^2 \theta) \sin \theta - \sin^3 \theta] \\
                                          & = \cos \theta (\cos^2 \theta - 3 \sin^2 \theta) + i (3 \sin \theta - 4 \sin^3 \theta)                   \\
          \cos 3 \theta                   & = \cos \theta (\cos^2 \theta - 3 \sin^2 \theta)                                                         \\ \\
          \sin 3 \theta                   & = 3 \sin \theta - 4 \sin^3 \theta
        \end{align*}
\end{enumerate}

\setcounter{subsection}{52}
\subsection{}

\begin{align*}
  \vec{B}                       & = B_z \uvec{z}                                                                                                                                              \\
  \vec{E}                       & = E_z \uvec{z}                                                                                                                                              \\
  \vec{v} \times \vec{B}        & = B_z v_y \uvec{x} - B_z v_x \uvec{y}                                                                                                                       \\
  \vec{F}                       & = q (\vec{E} + \vec{v} \times \vec{B})                                                                                                                      \\
                                & = q B_z v_y \uvec{x} - q B_z v_x \uvec{y} + q E_z \uvec{z}                                                                                                  \\ \\
  m \dot{v}_x                   & = q B_z v_y                                                                                                                                                 \\
  \dot{v}_x                     & = \omega v_y                                                                                                                                                \\ \\
  m \dot{v}_y                   & = -q B_z v_x                                                                                                                                                \\
  \dot{v}_y                     & = -\omega v_x                                                                                                                                               \\ \\
  m \dot{v}_z                   & = q E_z                                                                                                                                                     \\
  \dot{v}_z                     & = \frac{q}{m} E_z                                                                                                                                           \\ \\
  \frac{d}{d t} \begin{pmatrix}
                  v_x \\
                  v_y
                \end{pmatrix} & = \begin{pmatrix}
                                    0       & \omega \\
                                    -\omega & 0
                                  \end{pmatrix} \begin{pmatrix}
                                                  v_x \\
                                                  v_y
                                                \end{pmatrix}                                                                                                                                \\
  \begin{pmatrix}
    v_x \\
    v_y
  \end{pmatrix}               & = c_1 \left[ \begin{pmatrix}
                                                 0 \\
                                                 1
                                               \end{pmatrix} \cos \omega t - \begin{pmatrix}
                                                                               -1 \\
                                                                               0
                                                                             \end{pmatrix} \sin \omega t \right] + c_2 \left[ \begin{pmatrix}
                                                                                                                                -1 \\
                                                                                                                                0
                                                                                                                              \end{pmatrix} \cos \omega t + \begin{pmatrix}
                                                                                                                                                              0 \\
                                                                                                                                                              1
                                                                                                                                                            \end{pmatrix} \sin \omega t \right] \\
                                & = \begin{pmatrix}
                                      c_1 \sin \omega t - c_2 \cos \omega t \\
                                      c_1 \cos \omega t + c_2 \sin \omega t
                                    \end{pmatrix}                                                                                                                     \\ \\
  x                             & = -\frac{c_1}{\omega} \cos \omega t - \frac{c_2}{\omega} \sin \omega t + x_0                                                                                \\
  y                             & = \frac{c_1}{\omega} \sin \omega t - \frac{c_2}{\omega} \cos \omega t + y_0                                                                                 \\ \\
  v_z                           & = \frac{q}{m} E_z t + v_{z0}                                                                                                                                \\
  z                             & = \frac{q}{2 m} E_z t^2 + v_{z0} t + z_0
\end{align*}

The particle moves in a helix oriented along the $z$-axis.

\setcounter{subsection}{54}
\subsection{}

\begin{enumerate}
  \item

        \begin{align*}
          \vec{B}   & = B \uvec{z}                                \\
          \vec{E}   & = E \uvec{y}                                \\
          \vec{F}   & = q (\vec{E} + \vec{v} \times \vec{B})      \\
                    & = B q v_y \uvec{x} + q (E - B v_x) \uvec{y} \\
          \dot{v}_x & = \frac{B q}{m} v_y                         \\
                    & = \omega v_y                                \\
          \dot{v}_y & = \frac{E q}{m} - \frac{B q}{m} v_x         \\
                    & = \frac{E q}{m} - \omega v_x                \\
          \dot{v}_z & = 0
        \end{align*}

        The net force has no $\uvec{z}$ component, so the motion stays in the $xy$-plane.

  \item

        \begin{align*}
          0   & = \frac{E q}{m} - \omega v_x \\
          v_x & = \frac{E q}{\omega m}       \\
              & = \frac{E}{B}
        \end{align*}

  \item

        \begin{align*}
          \begin{pmatrix}
            \dot{v}_x \\
            \dot{v}_y
          \end{pmatrix} & = \begin{pmatrix}
                              0       & \omega \\
                              -\omega & 0
                            \end{pmatrix} \begin{pmatrix}
                                            v_x \\
                                            v_y
                                          \end{pmatrix} + \begin{pmatrix}
                                                            0 \\
                                                            \frac{E q}{m}
                                                          \end{pmatrix}          \\
          \mathbf{V}_c    & = \begin{pmatrix}
                                c_1 \sin \omega t - c_2 \cos \omega t \\
                                c_1 \cos \omega t + c_2 \sin \omega t
                              \end{pmatrix}               \\
          \mathbf{V}_p    & = \begin{pmatrix}
                                c_3 \\
                                c_4
                              \end{pmatrix}                                      \\
          \begin{pmatrix}
            0 \\
            0
          \end{pmatrix} & = \begin{pmatrix}
                              0       & \omega \\
                              -\omega & 0
                            \end{pmatrix} \begin{pmatrix}
                                            c_3 \\
                                            c_4
                                          \end{pmatrix} + \begin{pmatrix}
                                                            0 \\
                                                            \frac{E q}{m}
                                                          \end{pmatrix}          \\
                          & = \begin{pmatrix}
                                c_4 \omega \\
                                -c_3 \omega + \frac{E q}{m}
                              \end{pmatrix}                          \\
          c_3             & = \frac{E q}{m \omega}                                \\
                          & = \frac{E}{B}                                         \\
                          & = v_\text{dr}                                         \\
          c_4             & = 0                                                   \\
          \mathbf{V}_p    & = \begin{pmatrix}
                                v_\text{dr} \\
                                0
                              \end{pmatrix}                                      \\
          \mathbf{V}      & = \begin{pmatrix}
                                c_1 \sin \omega t - c_2 \cos \omega t + v_\text{dr} \\
                                c_1 \cos \omega t + c_2 \sin \omega t
                              \end{pmatrix} \\
          \begin{pmatrix}
            v_{x0} \\
            0
          \end{pmatrix} & = \begin{pmatrix}
                              -c_2 + v_\text{dr} \\
                              c_1
                            \end{pmatrix}                                    \\
          c_1             & = 0                                                   \\
          c_2             & = v_\text{dr} - v_{x0}                                \\
          \mathbf{V}      & = \begin{pmatrix}
                                v_\text{dr} + (v_{x0} - v_\text{dr}) \cos \omega t \\
                                -(v_{x0} - v_\text{dr}) \sin \omega t
                              \end{pmatrix}
        \end{align*}

  \item

        \begin{align*}
          x & = v_\text{dr} t + \frac{v_{x0} - v_\text{dr}}{\omega} \sin \omega t + x_0 \\
          y & = \frac{v_{x0} - v_\text{dr}}{\omega} \cos \omega t + y_0                 \\
          z & = z_0
        \end{align*}
\end{enumerate}

\section{Momentum and Angular Momentum}

\setcounter{subsection}{2}
\subsection{}

\begin{align*}
  m v_0     & = \frac{m}{3} (v_1 + v_2 \cos \theta + v_3 \cos \theta)                                 \\
  3 v_0     & = v_0 + \sqrt{2} v_2                                                                    \\
  2 v_0     & = \sqrt{2} v_2                                                                          \\
  v_2       & = \sqrt{2} v_0                                                                          \\
  \vec{v}_2 & = \sqrt{2} v_0 \left( \cos \frac{\pi}{4} \uvec{x} + \sin \frac{\pi}{4} \uvec{y} \right) \\
            & = v_0 (\uvec{x} + \uvec{y})                                                             \\
  \vec{v}_3 & = v_0 (\uvec{x} - \uvec{y})
\end{align*}

\setcounter{subsection}{6}
\subsection{}

\begin{align*}
  v & = v_\text{ex} \ln \frac{m_0}{m} \\
    & = \qty{2079}{m/s}               \\
  F & = \qty{25}{MN}                  \\
  W & = \qty{19.6}{MN}
\end{align*}

The thrust is $1.28$ times the weight on Earth.

\setcounter{subsection}{8}
\subsection{}

\begin{align*}
  -\dot{m} v_\text{ex} & = m_0 g                  \\
  v_\text{ex}          & = -\frac{m_0 g}{\dot{m}} \\
                       & = \qty{2352}{m/s}
\end{align*}

\setcounter{subsection}{10}
\subsection{}

\begin{enumerate}
  \item \[m \dot{v} = -\dot{m} v_\text{ex} + F_\text{ext}\]

  \item

        \begin{align*}
          m \dot{v}              & = -\dot{m} v_\text{ex} - m g                                        \\
          \dot{v}                & = -\frac{v_\text{ex}}{m} \dot{m} - g                                \\
          \int_0^t \dot{v} \,d t & = \int_0^t \left( -\frac{v_\text{ex}}{m} \dot{m} - g \right) \,d t  \\
          \int_0^v \,d v'        & = -v_\text{ex} \int_{m_0}^m \frac{1}{m'} \,d m' - \int_0^t g \,d t' \\
          v                      & = -v_\text{ex} \ln \frac{m}{m_0} - g t                              \\
                                 & = v_\text{ex} \ln \frac{m_0}{m} - g t
        \end{align*}

  \item \[v = \qty{903}{m/s}\]

        It would be $\qty{2079}{m/s}$ without gravity ($2.3$ times larger).

  \item The rocket wouldn't take off until it was light enough (from burning fuel) that its thrust was greater than its weight.
\end{enumerate}

\setcounter{subsection}{12}
\subsection{}

\begin{align*}
  v               & = v_\text{ex} \ln \frac{m_0}{m} - g t                                           \\
                  & = v_\text{ex} \ln \frac{m_0}{m_0 - k t} - g t                                   \\
  y               & = v_\text{ex} t - \frac{m v_\text{ex}}{k} \ln \frac{m_0}{m} - \frac{1}{2} g t^2 \\
  y(\qty{2}{min}) & = \qty{40}{km}
\end{align*}

\setcounter{subsection}{14}
\subsection{}

\begin{align*}
  M       & = m_1 + m_2 + m_3                                                                           \\
          & = m_1 + m_1 + 10 m_1                                                                        \\
          & = 12 m_1                                                                                    \\
  \vec{R} & = \frac{1}{12 m_1} \left[ m_1 \begin{pmatrix}
                                              1 \\
                                              1 \\
                                              0
                                            \end{pmatrix} + m_2 \begin{pmatrix}
                                                                  1  \\
                                                                  -1 \\
                                                                  0
                                                                \end{pmatrix} + m_3 \begin{pmatrix}
                                                                                      0 \\
                                                                                      0 \\
                                                                                      0
                                                                                    \end{pmatrix} \right] \\
          & = \frac{1}{12 m_1} \begin{pmatrix}
                                 2 m_1 \\
                                 0     \\
                                 0
                               \end{pmatrix}                                                           \\
          & = \begin{pmatrix}
                \frac{1}{6} \\
                0           \\
                0
              \end{pmatrix}
\end{align*}

\setcounter{subsection}{16}
\subsection{}

If we let Earth be at the origin, then

\begin{align*}
  R & = \frac{d M_m}{M_e + M_m} \\
    & = \qty{4630}{km}
\end{align*}

The centre of mass is inside Earth.

\setcounter{subsection}{18}
\subsection{}

\begin{enumerate}
  \item No external forces apply during the explosion so the path of the centre of mass would be unchanged.

  \item \qty{100}{m} before the target.

  \item No.
\end{enumerate}

\setcounter{subsection}{20}
\subsection{}

\begin{align*}
  \vec{R} & = \frac{1}{M} \int \vec{r} \,d m                                                                     \\
          & = \frac{2}{\sigma \pi R^2} \int \vec{r} \sigma \,d A                                                 \\
          & = \frac{2}{\pi R^2} \int_0^R \int_0^\pi \vec{r} r \,d \phi \,d r                                     \\
          & = \frac{2}{\pi R^2} \int_0^R \int_0^\pi r (\cos \phi \uvec{x} + \sin \phi \uvec{y}) r \,d \phi \,d r \\
          & = \frac{2}{\pi R^2} \int_0^R r^2 \int_0^\pi (\cos \phi \uvec{x} + \sin \phi \uvec{y}) \,d \phi \,d r \\
          & = \frac{2}{\pi R^2} \int_0^R r^2 [\sin \phi \uvec{x} - \cos \phi \uvec{y}]_0^\pi \,d r               \\
          & = \frac{4}{\pi R^2} \int_0^R r^2 \,d r \,\uvec{y}                                                    \\
          & = \frac{4 R}{3 \pi} \uvec{y}
\end{align*}

\setcounter{subsection}{24}
\subsection{}

\begin{align*}
  L            & = L_0                                     \\
  I \omega     & = I_0 \omega_0                            \\
  m r^2 \omega & = m r_0^2 \omega_0                        \\
  \omega       & = \left( \frac{r_0}{r} \right)^2 \omega_0
\end{align*}

\setcounter{subsection}{28}
\subsection{}

\begin{align*}
  I \omega                                                       & = I_0 \omega_0                                                         \\
  \frac{2}{5} \left( \frac{4}{3} \pi R^3 \rho \right) R^2 \omega & = \frac{2}{5} \left( \frac{4}{3} \pi R_0^3 \rho \right) R_0^2 \omega_0 \\
  \omega                                                         & = \left( \frac{R_0}{R} \right)^5 \omega_0
\end{align*}

If the radius doubles the angular velocity is $\omega_0 / 32$.

\setcounter{subsection}{30}
\subsection{}

\begin{align*}
  I & = \int r^2 \,d m                                               \\
    & = \int r^2 \sigma \,d A                                        \\
    & = \frac{M}{\pi R^2} \int_0^R \int_0^{2 \pi} r^3 \,d \phi \,d r \\
    & = \frac{1}{2} M R^2
\end{align*}

\setcounter{subsection}{32}
\subsection{}

\begin{align*}
  I & = \int r^2 \,d m                                                                  \\
    & = \int r^2 \sigma \,d A                                                           \\
    & = \frac{M}{(2 b)^2} \int_{-b}^b \int_{-b}^b (x^2 + y^2) \,d x \,d y               \\
    & = \frac{M}{4 b^2} \int_{-b}^b \left[ \frac{1}{3} x^3 + x y^2 \right]_{-b}^b \,d y \\
    & = \frac{M}{4 b} \int_{-b}^b \left( \frac{2}{3} b^2 + 2 y^2 \right) \,d y          \\
    & = \frac{M}{4 b} \left[ \frac{2}{3} b^2 y + \frac{2}{3} y^3 \right]_{-b}^b         \\
    & = \frac{2}{3} M b^2
\end{align*}

\setcounter{subsection}{34}
\subsection{}

\begin{enumerate}
  \setcounter{enumi}{1}
  \item

        \begin{align*}
          \Gamma_\text{ext}              & = R M g \sin \gamma           \\ \\
          \dot{L}                        & = \Gamma_\text{ext}           \\
          I \dot{\omega}                 & = R M g \sin \gamma           \\
          \frac{3}{2} M R^2 \dot{\omega} & = R M g \sin \gamma           \\
          \dot{\omega}                   & = \frac{2 g \sin \gamma}{3 R} \\ \\
          \dot{v}                        & = R \dot{\omega}              \\
                                         & = \frac{2}{3} g \sin \gamma
        \end{align*}

  \item

        \begin{align*}
          M \dot{v}                      & = M g \sin \gamma - f                   \\
          f                              & = M (g \sin \gamma - \dot{v})           \\ \\
          \Gamma_\text{ext}              & = R f                                   \\
                                         & = R M (g \sin \gamma - \dot{v})         \\ \\
          \dot{L}                        & = \Gamma_\text{ext}                     \\
          I \dot{\omega}                 & = R M (g \sin \gamma - \dot{v})         \\
          \frac{1}{2} M R^2 \dot{\omega} & = R M (g \sin \gamma - \dot{v})         \\
          \dot{\omega}                   & = \frac{2 (g \sin \gamma - \dot{v})}{R} \\ \\
          \dot{v}                        & = R \dot{\omega}                        \\
                                         & = 2 (g \sin \gamma - \dot{v})           \\
                                         & = \frac{2}{3} g \sin \gamma
        \end{align*}
\end{enumerate}

\setcounter{subsection}{36}
\subsection{}

\begin{enumerate}
  \setcounter{enumi}{1}
  \item

        \begin{align*}
          \sum m_\alpha r_\alpha' & = \sum m_\alpha (r_\alpha - R)             \\
                                  & = \sum m_\alpha r_\alpha - \sum m_\alpha R \\
                                  & = M R - M R                                \\
                                  & = 0
        \end{align*}
\end{enumerate}

\section{Energy}

\setcounter{subsection}{2}
\subsection{}

\begin{enumerate}
  \item

        \begin{align*}
          \int_P^Q \vec{F} \cdot d \vec{r} & = \int_P^O \vec{F} \cdot d \vec{r} + \int_O^Q \vec{F} \cdot d \vec{r} \\
                                           & = 0
        \end{align*}

  \item

        \begin{align*}
          x                                & = 1 - t                         \\
          y                                & = t                             \\
          \vec{r}                          & = (1 - t) \uvec{x} + t \uvec{y} \\
          d \vec{r}                        & = (-\uvec{x} + \uvec{y}) \,d t  \\
          \vec{F} \cdot d \vec{r}          & = (x + y) \,d t                 \\
                                           & = [(1 - t) + 1] \,d t           \\
                                           & = d t                           \\
          \int_P^Q \vec{F} \cdot d \vec{r} & = \int_0^1 \,d t                \\
                                           & = 1
        \end{align*}

  \item

        \begin{align*}
          \vec{r}                          & = \cos \phi \uvec{x} + \sin \phi \uvec{y}             \\
          d \vec{r}                        & = (-\sin \phi \uvec{x} + \cos \phi \uvec{y}) \,d \phi \\
          \vec{F} \cdot d \vec{r}          & = (\sin^2 \phi + \cos^2 \phi) \,d \phi                \\
                                           & = d \phi                                              \\
          \int_P^Q \vec{F} \cdot d \vec{r} & = \int_0^{\pi / 2} d \phi                             \\
                                           & = \frac{\pi}{2}
        \end{align*}
\end{enumerate}

\setcounter{subsection}{6}
\subsection{}

\begin{enumerate}
  \item

        \begin{align*}
          \vec{F}    & = -m \gamma y^2 \uvec{y}                               \\
          W          & = \int_{\vec{r}_1}^{\vec{r}_2} \vec{F} \cdot d \vec{r} \\
                     & = -m \gamma \int_{y_1}^{y_2} y^2 \,d y                 \\
                     & = -\frac{1}{3} m \gamma (y_2^3 - y_1^3)                \\
          U(\vec{r}) & = \frac{1}{3} m \gamma y^3
        \end{align*}

  \item Assuming no friction

        \begin{align*}
          \frac{1}{2} m v^2 & = \frac{1}{3} m \gamma h^3      \\
          v                 & = \sqrt{\frac{2}{3} \gamma h^3}
        \end{align*}
\end{enumerate}

\setcounter{subsection}{8}
\subsection{}

\begin{enumerate}
  \item

        \begin{align*}
          U(x) & = -\int_0^x -k x' \,d x' \\
               & = \frac{1}{2} k x^2
        \end{align*}
\end{enumerate}

\setcounter{subsection}{10}
\subsection{}

\begin{enumerate}
  \item

        \begin{align*}
          \frac{\partial f}{\partial x} & = 0             \\
          \frac{\partial f}{\partial y} & = 2 a y + 2 b z \\
          \frac{\partial f}{\partial z} & = 2 b y + 2 c z
        \end{align*}

  \item

        \begin{align*}
          \frac{\partial g}{\partial x} & = -a y^2 z^3 \sin (a x y^2 z^3)     \\
          \frac{\partial g}{\partial y} & = -2 a x y z^3 \sin (a x y^2 z^3)   \\
          \frac{\partial g}{\partial z} & = -3 a x y^2 z^2 \sin (a x y^2 z^3)
        \end{align*}

  \item

        \begin{align*}
          \frac{\partial h}{\partial x} & = \frac{a x}{r} \\
          \frac{\partial h}{\partial y} & = \frac{a y}{r} \\
          \frac{\partial h}{\partial z} & = \frac{a z}{r}
        \end{align*}
\end{enumerate}

\setcounter{subsection}{12}
\subsection{}

\begin{enumerate}
  \item \[\nabla f = \frac{1}{r^2} (x \uvec{x} + y \uvec{y} + z \uvec{z})\]

  \item \[\nabla f = n r^{n - 2} (x \uvec{x} + y \uvec{y} + z \uvec{z})\]

  \item \[\nabla f = \frac{g'(r)}{r} (x \uvec{x} + y \uvec{y} + z \uvec{z})\]
\end{enumerate}

\setcounter{subsection}{14}
\subsection{}

\begin{align*}
  d f                              & = \nabla f \cdot d \vec{r}           \\
                                   & = (2, 4, 6) \cdot (0.01, 0.03, 0.05) \\
                                   & = 0.44                               \\
  f(1.01, 1.03, 1.05) - f(1, 1, 1) & = 0.4494
\end{align*}

\setcounter{subsection}{18}
\subsection{}

\begin{enumerate}
  \item An ellipse that is two times wider than it is tall.

  \item

        \begin{align*}
          \nabla f                            & = (2 x, 8 y, 0)         \\
          \left. \nabla f \right|_{(1, 1, 1)} & = (2, 8, 0)             \\
          \vec{n}                             & = (1, 4, 0) / \sqrt{17}
        \end{align*}
\end{enumerate}

\setcounter{subsection}{20}
\subsection{}

\begin{align*}
  \vec{F}               & = -\frac{G M m}{r^2} \uvec{r}                                                                                                                                      \\
  \nabla \times \vec{F} & = \begin{vmatrix}
                              \uvec{x}                    & \uvec{y}                    & \uvec{z}                    \\
                              \frac{\partial}{\partial x} & \frac{\partial}{\partial y} & \frac{\partial}{\partial z} \\
                              -\frac{G M m}{r^3} x        & -\frac{G M m}{r^3} y        & -\frac{G M m}{r^3} z        \\
                            \end{vmatrix}                                                                          \\
                        & = \left[ \frac{\partial}{\partial y} \left( -\frac{G M m}{r^3} z \right) - \frac{\partial}{\partial z} \left( -\frac{G M m}{r^3} y \right) \right] \uvec{x}        \\
                        & \qquad - \left[ \frac{\partial}{\partial x} \left( -\frac{G M m}{r^3} z \right) - \frac{\partial}{\partial z} \left( -\frac{G M m}{r^3} x \right) \right] \uvec{y} \\
                        & \qquad + \left[ \frac{\partial}{\partial x} \left( -\frac{G M m}{r^3} y \right) - \frac{\partial}{\partial y} \left( -\frac{G M m}{r^3} x \right) \right] \uvec{z} \\
                        & = -G M m \left[ \left( -\frac{3 y z}{r^5} + \frac{3 y z}{r^5} \right) \uvec{x} + \left( -\frac{3 x z}{r^5} + \frac{3 x z}{r^5} \right) \uvec{y} \right.            \\
                        & \qquad \left. \left( -\frac{3 x y}{r^5} + \frac{3 x y}{r^4} \right) \uvec{z} \right]                                                                               \\
                        & = \vec{0}                                                                                                                                                          \\
  U(\vec{r})            & = -\int_{\vec{\infty}}^{\vec{r}} \vec{F} \cdot d \vec{r}'                                                                                                          \\
                        & = -\int_\infty^r -\frac{G M m}{r'^2} \,d r'                                                                                                                        \\
                        & = G M m \left[ -\frac{1}{r'} \right]_\infty^r                                                                                                                      \\
                        & = G M m \left( -\frac{1}{r} + \frac{1}{\infty} \right)                                                                                                             \\
                        & = -\frac{G M m}{r}
\end{align*}

\setcounter{subsection}{22}
\subsection{}

\begin{enumerate}
  \item

        \begin{align*}
          \nabla \times \vec{F} & = \vec{0}                                                                          \\
          U                     & = -\int_{\vec{0}}^{\vec{r}} \vec{F} \cdot d \vec{r}                                \\
                                & = -\left( \int_0^x k x \,d x + \int_0^y 2 k y \,d y + \int_0^z 3 k z \,d z \right) \\
                                & = -\frac{1}{2} k (x^2 + 2 y^2 + 3 z^2)
        \end{align*}

  \item

        \begin{align*}
          \nabla \times \vec{F} & = \vec{0}                                                 \\
          U                     & = -\int_{\vec{0}}^{\vec{r}} \vec{F} \cdot d \vec{r}       \\
                                & = -\left( \int_0^x k y \,d x + \int_0^y k x \,d y \right) \\
                                & = -k x y
        \end{align*}

  \item \[\nabla \times \vec{F} = 2 k \uvec{z}\] Not conservative
\end{enumerate}

\setcounter{subsection}{28}
\subsection{}

\begin{enumerate}
  \item The mass will oscillate around $x = 0$.

  \item

        \begin{align*}
          t    & = \sqrt{\frac{m}{2}} \int_0^A \frac{d x}{\sqrt{k A^4 - k x^4}} \\
               & = \sqrt{\frac{m}{2 k}} \int_0^A \frac{d x}{\sqrt{A^4 - x^4}}   \\
          \tau & = 4 t
        \end{align*}

        \setcounter{enumi}{3}
  \item $\tau \approx \qty{3.71}{s}$
\end{enumerate}

\setcounter{subsection}{30}
\subsection{}

\begin{enumerate}
  \item \[E = \frac{1}{2} (m_1 + m_2) \dot{x}^2 + (m_2 - m_1) g x\]

  \item

        \begin{align*}
          (m_1 + m_2) \ddot{x} & = (m_1 - m_2) g                                        \\ \\
          c                    & = \frac{1}{2} (m_1 + m_2) \dot{x}^2 + (m_2 - m_1) g x  \\
          0                    & = (m_1 + m_2) \dot{x} \ddot{x} + (m_2 - m_1) g \dot{x} \\
          (m_1 + m_2) \ddot{x} & = (m_1 - m_2) g
        \end{align*}
\end{enumerate}

\setcounter{subsection}{34}
\subsection{}

\begin{enumerate}
  \item \[E = \frac{1}{2} \left( m_1 + m_2 + \frac{I}{R^2} \right) \dot{x}^2 + (m_2 - m_1) g x\]

  \item

        \begin{align*}
          0                                                 & = \left( m_1 + m_2 + \frac{I}{R^2} \right) \dot{x} \ddot{x} + (m_2 - m_1) g \dot{x} \\
          \left( m_1 + m_2 + \frac{I}{R^2} \right) \ddot{x} & = (m_1 - m_2) g                                                                     \\ \\
          m_1 \ddot{x}                                      & = m_1 g - T_2                                                                       \\
          T_2                                               & = m_1 g - m_1 \ddot{x}                                                              \\ \\
          m_2 \ddot{x}                                      & = T_1 - m_2 g                                                                       \\
          T_1                                               & = m_2 \ddot{x} + m_2 g                                                              \\ \\
          \omega                                            & = -\frac{\dot{x}}{R}                                                                \\
          \dot{\omega}                                      & = -\frac{\ddot{x}}{R}                                                               \\ \\
          I \dot{\omega}                                    & = (T_1 - T_2) R                                                                     \\
          -I \frac{\ddot{x}}{R}                             & = \left( m_2 \ddot{x} + m_2 g - m_1 g - m_1 \ddot{x} \right) R                      \\
          \left( m_1 + m_2 + \frac{I}{R^2} \right) \ddot{x} & = (m_1 - m_2) g
        \end{align*}
\end{enumerate}

\setcounter{subsection}{36}
\subsection{}

\begin{enumerate}
  \item \[U(\phi) = M g R (1 - \cos \phi) - m g R \phi\]

  \item

        \begin{align*}
          \frac{d U(\phi)}{d \phi}     & = M g R \sin \phi - m g R \\
                                       & = g R (M \sin \phi - m)   \\
          0                            & = g R (M \sin \phi - m)   \\
          m                            & = M \sin \phi             \\ \\
          \frac{d^2 U(\phi)}{d \phi^2} & = M g R \cos \phi
        \end{align*}

        There is a position of equilibrium at $\phi = \arcsin \frac{m}{M}$. It is stable if $\phi < \frac{\pi}{2}$, i.e. $m < M$.
\end{enumerate}

\setcounter{subsection}{50}
\subsection{}

\begin{align*}
  U(\vec{r}_1, \vec{r}_2, \vec{r}_3, \vec{r}_4) & = U_\text{int} + U_\text{ext}                                                                                                    \\
                                                & = [U_{12}(\vec{r}_1 - \vec{r}_2) + U_{13}(\vec{r}_1 - \vec{r}_3) + U_{14}(\vec{r}_1 - \vec{r}_4) + U_{23}(\vec{r}_2 - \vec{r}_3) \\
                                                & \qquad + U_{24}(\vec{r}_2 - \vec{r}_4) + U_{34}(\vec{r}_3 - \vec{r}_4)] + [U_1(\vec{r}_1) + U_2(\vec{r}_2) + U_3(\vec{r}_3)      \\
                                                & \qquad + U_4(\vec{r}_4)]                                                                                                         \\
  \vec{F}_3                                     & = \vec{F}_{3 \text{,int}} + \vec{F}_{3 \text{,ext}}                                                                              \\
                                                & = [\vec{F}_{13} + \vec{F}_{23} + \vec{F}_{34}] + \vec{F}_{3 \text{,ext}}                                                         \\
                                                & = -\nabla_3 U_{13} - \nabla_3 U_{23} - \nabla_3 U_{34} - \nabla_3 U_{3 \text{,ext}}                                              \\
                                                & = -\nabla_3 U
\end{align*}

\setcounter{subsection}{52}
\subsection{}

\begin{enumerate}
  \item

        \begin{align*}
          F                 & = m a               \\
          \frac{k e^2}{r^2} & = m \frac{v^2}{r}   \\
          v^2               & = \frac{k e^2}{m r} \\
          K                 & = \frac{1}{2} m v^2 \\
                            & = \frac{k e^2}{2 r} \\
          U                 & = -\frac{k e^2}{r}  \\
          K                 & = -\frac{1}{2} U
        \end{align*}

  \item

        \begin{align*}
          E & = K_1 + K_2 + U_{1 2} + U_{1 p} + U_{2 p}                                                                            \\
            & = \frac{1}{2} m v_1^2 + \frac{1}{2} m v_2^2 - k e^2 \left( \frac{1}{r_1} + \frac{1}{r_2} - \frac{1}{r_{1 2}} \right)
        \end{align*}

  \item

        \begin{align*}
          E_\text{before}         & = K_1 + K_2 + U_1 + U_2 + U_{1 2}                                 \\
                                  & = T_2 + \frac{k e^2}{2 r} - \frac{k e^2}{r}                       \\
                                  & = T_2 - \frac{k e^2}{2 r}                                         \\ \\
          E_\text{after}          & = K_1' + K_2' + U_1 + U_2 + U_{1 2}                               \\
                                  & = T_1' + \frac{k e^2}{2 r'} - \frac{k e^2}{r'}                    \\
                                  & = T_1' - \frac{k e^2}{2 r'}                                       \\
          T_2 - \frac{k e^2}{2 r} & = T_1' - \frac{k e^2}{2 r'}                                       \\
          T_1'                    & = T_2 + \frac{k e^2}{2} \left( \frac{1}{r'} - \frac{1}{r} \right)
        \end{align*}
\end{enumerate}

\section{Oscillations}

\setcounter{subsection}{2}
\subsection{}

\begin{align*}
  U(\phi) & = m g l (1 - \cos \phi)                                 \\
          & \approx m g l \left( 1 - 1 + \frac{1}{2} \phi^2 \right) \\
          & = \frac{1}{2} m g l \phi^2                              \\
  k       & = m g l
\end{align*}

\setcounter{subsection}{4}
\subsection{}

\begin{align*}
  x(t)   & = C_1 e^{i \omega t} + C_2 e^{-i \omega t}                                      \\
         & = C_1 (\cos \omega t + i \sin \omega t) + C_2 (\cos \omega t - i \sin \omega t) \\
         & = (C_1 + C_2) \cos \omega t + i (C_1 - C_2) \sin \omega t                       \\
         & = B_1 \cos \omega t + B_2 \sin \omega t                                         \\
  B_1    & = C_1 + C_2                                                                     \\
  B_2    & = i (C_1 - C_2)                                                                 \\ \\
  x(t)   & = A \cos (\omega t - \delta)                                                    \\
  A      & = \sqrt{B_1^2 + B_2^2}                                                          \\
  \delta & = \arctan \frac{B_2}{B_1}                                                       \\ \\
  x(t)   & = C \Re e^{\omega t}                                                            \\
  C      & = A e^{-i \delta}
\end{align*}

\setcounter{subsection}{6}
\subsection{}

\begin{enumerate}
  \item $B_1 = x_0$, $B_2 = \frac{v_0}{\omega}$

  \item

        \begin{align*}
          \omega & = \sqrt{\frac{k}{m}} \\
                 & = \qty{10}{rad/s}    \\
          B_1    & = \qty{3.0}{m}       \\
          B_2    & = \qty{5.0}{m}
        \end{align*}

  \item $x = \qty{0}{m}$ at $t = \qty{0.26}{s}$, $\dot{x} = \qty{0}{m/s}$ at $t = \qty{0.10}{s}$.
\end{enumerate}

\setcounter{subsection}{8}
\subsection{}

\begin{align*}
  \frac{1}{2} k A^2 & = \frac{1}{2} m v^2            \\
  \frac{k}{m}       & = \left( \frac{v}{A} \right)^2 \\
  \tau              & = \frac{2 \pi}{\omega}         \\
                    & = \frac{2 \pi}{\sqrt{k / m}}   \\
                    & = \frac{2 \pi}{v / A}          \\
                    & = \qty{1.05}{s}
\end{align*}

\setcounter{subsection}{10}
\subsection{}

\begin{align*}
  \frac{1}{2} k x_1^2 + \frac{1}{2} m v_1^2 & = \frac{1}{2} k x_2^2 + \frac{1}{2} m v_2^2                                  \\
  k x_1^2 + m v_1^2                         & = k x_2^2 + m v_2^2                                                          \\
  \frac{k}{m} x_1^2 + v_1^2                 & = \frac{k}{m} x_2^2 + v_2^2                                                  \\
  \omega^2 (x_1^2 - x_2^2)                  & = v_2^2 - v_1^2                                                              \\
  \omega                                    & = \sqrt{\frac{v_2^2 - v_1^2}{x_1^2 - x_2^2}}                                 \\ \\
  \frac{1}{2} k A^2                         & = \frac{1}{2} k x_1^2 + \frac{1}{2} m v_1^2                                  \\
  A^2                                       & = x_1^2 + \frac{m}{k} v_1^2                                                  \\
  A                                         & = \sqrt{x_1^2 + \frac{v_1^2}{\omega^2}}                                      \\
                                            & = \sqrt{x_1^2 + v_1^2 \frac{x_1^2 - x_2^2}{v_2^2 - v_1^2}}                   \\
                                            & = \sqrt{\frac{x_1^2 (v_2^2 - v_1^2) + v_1^2 (x_1^2 - x_2^2)}{v_2^2 - v_1^2}} \\
                                            & = \sqrt{\frac{x_2^2 v_1^2 - x_1^2 v_2^2}{v_1^2 - v_2^2}}
\end{align*}

\setcounter{subsection}{12}
\subsection{}

\begin{align*}
  U(r)                & = U_0 \left( \frac{r}{R} + \lambda^2 \frac{R}{r} \right)                                                                                  \\
  \frac{d U(r)}{d r}  & = U_0 \left( \frac{1}{R} - \lambda^2 \frac{R}{r^2} \right)                                                                                \\
  0                   & = \frac{d U(r_0)}{d r}                                                                                                                    \\
                      & = U_0 \left( \frac{1}{R} - \lambda^2 \frac{R}{r_0^2} \right)                                                                              \\
  \frac{1}{R}         & = \lambda^2 \frac{R}{r_0^2}                                                                                                               \\
  r_0                 & = \lambda R                                                                                                                               \\ \\
  U(r_0 + x) - U(r_0) & = U_0 \left( \frac{r_0 + x}{R} + \lambda^2 \frac{R}{r_0 + x} \right) - U_0 \left( \frac{r_0}{R} + \lambda^2 \frac{R}{r_0} \right)         \\
                      & = U_0 \left[ \frac{1}{R} x + \lambda^2 R \left( \frac{1}{r_0 + x} - \frac{1}{r_0} \right) \right]                                         \\
                      & \approx U_0 \left[ \frac{1}{R} x + \lambda^2 R \left( \frac{1}{r_0} - \frac{x}{r_0^2} + \frac{x^2}{r_0^3} - \frac{1}{r_0} \right) \right] \\
                      & = U_0 \left[ \frac{1}{R} x + \lambda^2 R \left( \frac{x^2}{r_0^3} - \frac{x}{r_0^2} \right) \right]                                       \\
                      & = U_0 \left[ \frac{1}{R} x + \lambda^2 R \left( \frac{x^2}{(\lambda R)^3} - \frac{x}{(\lambda R)^2} \right) \right]                       \\
                      & = \frac{U_0 x^2}{\lambda R^2}                                                                                                             \\
                      & = \frac{1}{2} \left( \frac{2 U_0}{\lambda R^2} \right) x^2                                                                                \\ \\
  \omega              & = \sqrt{\frac{k}{m}}                                                                                                                      \\
                      & = \sqrt{\frac{2 U_0}{\lambda m R^2}}
\end{align*}

\setcounter{subsection}{16}
\subsection{}

\begin{enumerate}
  \item

        \begin{align*}
          x(t)                       & = A_x \cos \omega_x t                                                                 \\
          y(t)                       & = A_y \cos (\omega_y t - \delta)                                                      \\
          \frac{\omega_x}{\omega_y}  & = \frac{p}{q}                                                                         \\
          \omega_x \tau              & = 2 \pi p                                                                             \\
          \omega_y \tau              & = 2 \pi q                                                                             \\
          (\omega_x + \omega_y) \tau & = 2 \pi (p + q)                                                                       \\
          \tau                       & = \frac{2 \pi (p + q)}{\omega_x + \omega_y}                                           \\
          x(\tau)                    & = A_x \cos \left( \omega_x \frac{2 \pi (p + q)}{\omega_x + \omega_y} \right)          \\
                                     & = A_x \cos \left( \frac{2 \pi (p + q)}{1 + \omega_y / \omega_x} \right)               \\
                                     & = A_x \cos \left( \frac{2 \pi (p + q)}{1 + q / p} \right)                             \\
                                     & = A_x \cos \left( 2 \pi \frac{p (p + q)}{p + q} \right)                               \\
          y(\tau)                    & = A_y \cos \left( \omega_y \frac{2 \pi (p + q)}{\omega_x + \omega_y} - \delta \right) \\
                                     & = A_y \cos \left( 2 \pi \frac{p + q}{1 + \omega_x / \omega_y} - \delta \right)        \\
                                     & = A_y \cos \left( 2 \pi \frac{q (p + q)}{p + q} - \delta \right)
        \end{align*}
\end{enumerate}

\setcounter{subsection}{22}
\subsection{}

\begin{align*}
  \frac{d E}{d t} & = m \dot{x} \ddot{x} + k x \dot{x} \\
                  & = \dot{x} (m \ddot{x} + k x)       \\
                  & = -b \dot{x}^2
\end{align*}

\setcounter{subsection}{24}
\subsection{}

\begin{enumerate}
  \item \[\tau = \frac{2 \pi}{\omega_1}\]

  \item

        \begin{align*}
          0 & = A e^{-\beta t} \cos \omega_1 t                           \\
            & = \cos \omega_1 t                                          \\
          t & = \frac{\frac{\pi}{2} + n \pi}{\omega_1}, n \in \mathbb{Z} \\
        \end{align*}

        The time between successive zeroes is $\pi / \omega_1$. The period $\tau$ is twice this.

  \item

        \begin{align*}
          e^{-\beta \tau} & = e^{-(\omega_0 / 2) (2 \pi / \omega_1)}                   \\
                          & = e^{-\pi \omega_0 / \omega_1}                             \\
                          & = e^{-\pi \omega_0 / \sqrt{\omega_0^2 - (\omega_0 / 2)^2}} \\
                          & = e^{-\pi \omega_0 / \sqrt{3 \omega_0^2 / 4}}              \\
                          & = e^{-\pi \sqrt{4 / 3}}                                    \\
                          & \approx 0.027
        \end{align*}
\end{enumerate}

\setcounter{subsection}{28}
\subsection{}

\begin{align*}
  \tau_0                                      & = \qty{1}{s}                                                              \\
  \omega_0                                    & = 2 \pi f                                                                 \\
                                              & = \frac{2 \pi}{\tau}                                                      \\
                                              & = 2 \pi \,\unit{rad/s}                                                    \\
  \frac{1}{2}                                 & = e^{-\beta \tau_1}                                                       \\
                                              & = e^{-2 \pi \beta / \omega_1}                                             \\
                                              & = e^{-2 \pi \beta / \sqrt{\omega_0^2 - \beta^2}}                          \\
  \ln \frac{1}{2}                             & = -\frac{2 \pi \beta}{\sqrt{\omega_0^2 - \beta^2}}                        \\
  \sqrt{\omega_0^2 - \beta^2} \ln \frac{1}{2} & = -2 \pi \beta                                                            \\
  (\omega_0^2 - \beta^2) \ln^2 \frac{1}{2}    & = 4 \pi^2 \beta^2                                                         \\
  \omega_0^2 \ln^2 \frac{1}{2}                & = \left( 4 \pi^2 + \ln^2 \frac{1}{2} \right) \beta^2                      \\
  \beta                                       & = \pm \frac{\ln \frac{1}{2}}{\sqrt{4 \pi^2 + \ln^2 \frac{1}{2}}} \omega_0 \\
                                              & \approx 0.11 \omega_0                                                     \\
  \tau_1                                      & = \frac{2 \pi}{\omega_1}                                                  \\
                                              & = \frac{2 \pi}{\sqrt{\omega_0^2 - \beta^2}}                               \\
                                              & = \frac{2 \pi}{\sqrt{\omega_0^2 - 0.0121 \omega_0^2}}                     \\
                                              & \approx \qty{1.006}{s}
\end{align*}

\setcounter{subsection}{42}
\subsection{}

\begin{enumerate}
  \item

        \begin{align*}
          4 m g & = 4 k x                \\
          k     & = \frac{m g}{x}        \\
                & \approx \qty{4e4}{N/m}
        \end{align*}

  \item \[\omega_0 = \sqrt{\frac{2 k}{m}} = \qty{40}{rad/s} \approx \qty{6}{Hz}\]

  \item \[\qty{5}{m/s} \approx \qty{18}{km/h}\]
\end{enumerate}

\section{Calculus of Variations}

\setcounter{subsection}{4}
\subsection{}

\begin{align*}
  |AP|                                                 & = \frac{|AB|}{2} \crd \left( \frac{\pi}{2} - \theta \right)                                                                                                \\
                                                       & = \frac{|AB|}{2} 2 \sin \left( \frac{\frac{\pi}{2} - \theta}{2} \right)                                                                                    \\
                                                       & = |AB| \sin \left( \frac{\pi}{4} - \frac{\theta}{2} \right)                                                                                                \\ \\
  |PB|                                                 & = \frac{|AB|}{2} \crd \left( \frac{\pi}{2} + \theta \right)                                                                                                \\
                                                       & = \frac{|AB|}{2} 2 \sin \left( \frac{\frac{\pi}{2} + \theta}{2} \right)                                                                                    \\
                                                       & = |AB| \sin \left( \frac{\pi}{4} + \frac{\theta}{2} \right)                                                                                                \\ \\
  S                                                    & = |AP| + |PB|                                                                                                                                              \\
                                                       & = |AB| \left[ \sin \left( \frac{\pi}{4} - \frac{\theta}{2} \right) + \sin \left( \frac{\pi}{4} + \frac{\theta}{2} \right) \right]                          \\ \\
  \frac{d S}{d \theta}                                 & = |AB| \left[ -\frac{1}{2} \cos \left( \frac{\pi}{4} - \frac{\theta}{2} \right) + \frac{1}{2} \cos \left( \frac{\pi}{4} + \frac{\theta}{2} \right) \right] \\
                                                       & = \frac{1}{2} |AB| \left[ \cos \left( \frac{\pi}{4} + \frac{\theta}{2} \right) - \cos \left( \frac{\pi}{4} - \frac{\theta}{2} \right) \right]              \\ \\
  \frac{d S}{d \theta}                                 & = 0                                                                                                                                                        \\
  \frac{\pi}{4} + \frac{\theta}{2}                     & = \frac{\pi}{4} - \frac{\theta}{2}                                                                                                                         \\
  \theta                                               & = 0                                                                                                                                                        \\ \\
  \frac{d^2 S}{d \theta^2}                             & = -\frac{1}{4} |AB| \left[ \sin \left( \frac{\pi}{4} - \frac{\theta}{2} \right) + \sin \left( \frac{\pi}{4} + \frac{\theta}{2} \right) \right]             \\
  \left. \frac{d^2 S}{d \theta^2} \right|_{\theta = 0} & = -\frac{\sqrt{2}}{4} |AB|
\end{align*}

$\frac{d S}{d \theta} = 0$ and $\frac{d^2 S}{d \theta^2} < 0$ at $\theta = 0$ so it is a maximum.

\setcounter{subsection}{6}
\subsection{}

\begin{align*}
  S                                                                                  & = \int_P^Q \,dS                                                                   \\
                                                                                     & = \int_P^Q \sqrt{(R \,d \phi)^2 + (d z)^2}                                        \\
                                                                                     & = \int_{z_1}^{z_2} \sqrt{1 + \left( R \frac{d \phi}{d z} \right)^2} \,d z         \\
  f \left( z, \phi, \frac{d \phi}{d z} \right)                                       & = \sqrt{1 + \left( R \frac{d \phi}{d z} \right)^2}                                \\
  \frac{\partial f}{\partial \phi}                                                   & = 0                                                                               \\
  \frac{\partial f}{\partial \phi'}                                                  & = \frac{R^2 \frac{d \phi}{d z}}{\sqrt{1 + \left( R \frac{d \phi}{d z} \right)^2}} \\
  \frac{\partial f}{\partial \phi} - \frac{d}{d z} \frac{\partial f}{\partial \phi'} & = 0                                                                               \\
  \frac{R^2 \frac{d \phi}{d z}}{\sqrt{1 + \left( R \frac{d \phi}{d z} \right)^2}}    & = c_1                                                                             \\
  R^4 \left( \frac{d \phi}{d z} \right)^2                                            & = c_1 \left[ 1 + \left( R \frac{d \phi}{d z} \right)^2 \right]                    \\
  (R^4 - c_1 R^2) \left( \frac{d \phi}{d z} \right)^2                                & = c_1                                                                             \\
  \frac{d \phi}{d z}                                                                 & = \sqrt{\frac{c_1}{R^4 - c_1 R^2}}                                                \\
                                                                                     & = c_2                                                                             \\
  \phi                                                                               & = c_2 z + c_3
\end{align*}

\setcounter{subsection}{8}
\subsection{}

\begin{align*}
  f(x, y, y')                                                                  & = y'^2 + y y' + y^2                                  \\
  \frac{\partial f}{\partial y}                                                & = y' + 2 y                                           \\
  \frac{\partial f}{\partial y'}                                               & = 2 y' + y                                           \\
  \frac{d}{d x} \frac{\partial f}{\partial y'}                                 & = 2 y'' + y'                                         \\
  \frac{\partial f}{\partial y} - \frac{d}{d x} \frac{\partial f}{\partial y'} & = 0                                                  \\
  y' + 2 y - 2 y'' - y'                                                        & = 0                                                  \\
  y''                                                                          & = y                                                  \\
  y                                                                            & = c_1 e^x + c_2 e^{-x}                               \\
  y(0)                                                                         & = 0                                                  \\
                                                                               & = c_1 + c_2                                          \\
  y(1)                                                                         & = 1                                                  \\
                                                                               & = c_1 e + c_2 e^{-1}                                 \\
  1                                                                            & = -c_2 e + c_2 e^{-1}                                \\
                                                                               & = c_2 (e^{-1} - e)                                   \\
  c_2                                                                          & = \frac{1}{e^{-1} - e}                               \\
  c_1                                                                          & = \frac{1}{e - e^{-1}}                               \\
  y                                                                            & = \frac{e^x}{e - e^{-1}} + \frac{e^{-x}}{e^{-1} - e} \\
                                                                               & = \frac{e^x - e^{-x}}{e - e^{-1}}                    \\
                                                                               & = \frac{\sinh x}{\sinh 1}
\end{align*}

\setcounter{subsection}{10}
\subsection{}

\begin{align*}
  f(x, y, y')                                                                  & = \sqrt{x (1 + y'^2)}                    \\
  \frac{\partial f}{\partial y}                                                & = 0                                      \\
  \frac{\partial f}{\partial y'}                                               & = \frac{\sqrt{x} y'}{\sqrt{1 + y'^2}}    \\
  \frac{\partial f}{\partial y} - \frac{d}{d x} \frac{\partial f}{\partial y'} & = 0                                      \\
  \frac{\sqrt{x} y'}{\sqrt{1 + y'^2}}                                          & = c_1                                    \\
  x y'^2                                                                       & = c_1 (1 + y'^2)                         \\
  (x - c_1) y'^2                                                               & = c_1                                    \\
  y'                                                                           & = \sqrt{\frac{c_1}{x - c_1}}             \\
  y                                                                            & = 2 c_1 \sqrt{\frac{x - c_1}{c_1}} + c_2
\end{align*}

\section{Lagrange's Equations}

\subsection{}

\begin{align*}
  K           & = \frac{1}{2} m \vec{v}^2                                   \\
              & = \frac{1}{2} m (\dot{x}^2 + \dot{y}^2 + \dot{z}^2)         \\
  U           & = m g z                                                     \\
  \mathcal{L} & = K - U                                                     \\
              & = \frac{1}{2} m (\dot{x}^2 + \dot{y}^2 + \dot{z}^2) - m g z \\
  F_x         & = \frac{\partial \mathcal{L}}{\partial x}                   \\
              & = 0                                                         \\
  F_y         & = \frac{\partial \mathcal{L}}{\partial y}                   \\
              & = 0                                                         \\
  F_z         & = \frac{\partial \mathcal{L}}{\partial z}                   \\
              & = -m g
\end{align*}

\setcounter{subsection}{2}
\subsection{}

\begin{align*}
  \mathcal{L}                             & = K - U                                                             \\
                                          & = \frac{1}{2} m (\dot{x}^2 + \dot{y}^2) - \frac{1}{2} k (x^2 + y^2) \\ \\
  \frac{\partial \mathcal{L}}{\partial x} & = \frac{d}{d t} \frac{\partial \mathcal{L}}{\partial \dot{x}}       \\
  m \ddot{x}                              & = -k x                                                              \\ \\
  \frac{\partial \mathcal{L}}{\partial y} & = \frac{d}{d t} \frac{\partial \mathcal{L}}{\partial \dot{y}}       \\
  m \ddot{y}                              & = -k y
\end{align*}

The mass is a harmonic oscillator in each dimension.

\setcounter{subsection}{4}
\subsection{}

\begin{align*}
  d f               & = \frac{\partial f}{\partial r} \,d r + \frac{\partial f}{\partial \theta} \,d \theta             \\
  d \vec{r}         & = d r + r \,d \theta                                                                              \\
  d f               & = \nabla f \cdot d \vec{r}                                                                        \\
                    & = (\nabla f)_r \,d r + (\nabla f)_\theta r \,d \theta                                             \\
  (\nabla f)_r      & = \frac{\partial f}{\partial r}                                                                   \\
  (\nabla f)_\theta & = \frac{1}{r} \frac{\partial f}{\partial \theta}                                                  \\
  \nabla f          & = \frac{\partial f}{\partial r} \,d r + \frac{1}{r} \frac{\partial f}{\partial \theta} \,d \theta
\end{align*}

\setcounter{subsection}{6}
\subsection{}

\begin{enumerate}
  \item \[m_i \ddvec{r}_i = -\nabla U_i\]

  \item \[\mathcal{L}({\vec{r}_1, \ldots, \vec{r}_n, \dvec{r}_1, \ldots, \dvec{r}_n}) = \frac{1}{2} m_1 \dot{r}_1^2 + \ldots + \frac{1}{2} m_n \dot{r}_n^2 - U(\vec{r}_1, \ldots, \vec{r}_n)\]
\end{enumerate}

\setcounter{subsection}{8}
\subsection{}

\begin{align*}
  x    & = R \cos \phi         \\
  y    & = R \sin \phi         \\
  \phi & = \arctan \frac{y}{x}
\end{align*}

\setcounter{subsection}{10}
\subsection{}

\begin{align*}
  x    & = A \cos \omega t + l \sin \phi         \\
  y    & = l \cos \phi                           \\
  \phi & = \arctan \frac{x - A \cos \omega t}{y}
\end{align*}

\setcounter{subsection}{14}
\subsection{}

\begin{align*}
  \mathcal{L}                             & = K - U                                                       \\
                                          & = \frac{1}{2} (m_1 + m_2) \dot{x}^2 + m_2 g x                 \\
  \frac{\partial \mathcal{L}}{\partial x} & = \frac{d}{d t} \frac{\partial \mathcal{L}}{\partial \dot{x}} \\
  (m_1 + m_2) \ddot{x}                    & = m_2 g                                                       \\
  \ddot{x}                                & = \frac{m_2}{m_1 + m_2} g
\end{align*}

\setcounter{subsection}{16}
\subsection{}

\begin{align*}
  \mathcal{L}                             & = K - U                                                                                                \\
                                          & = \frac{1}{2} m_1 \dot{x}^2 + \frac{1}{2} I \omega^2 + \frac{1}{2} m_2 \dot{x}^2 - (m_2 g x - m_1 g x) \\
                                          & = \frac{1}{2} \left( m_1 + m_2 + \frac{I}{R^2} \right) \dot{x}^2 + (m_1 - m_2) g x                     \\
  \frac{\partial \mathcal{L}}{\partial x} & = \frac{d}{d t} \frac{\partial \mathcal{L}}{\partial \dot{x}}                                          \\
  (m_1 - m_2) g                           & = \left( m_1 + m_2 + \frac{I}{R^2} \right) \ddot{x}                                                    \\
  \ddot{x}                                & = \frac{(m_1 - m_2) g}{m_1 + m_2 + \frac{I}{R^2}}
\end{align*}

\setcounter{subsection}{20}
\subsection{}

\begin{align*}
  \mathcal{L}                             & = K - U                                                       \\
                                          & = \frac{1}{2} m v^2                                           \\
                                          & = \frac{1}{2} m [\dot{r}^2 + (r \dot{\phi})^2]                \\
                                          & = \frac{1}{2} m (\dot{r}^2 + r^2 \omega^2)                    \\
  \frac{\partial \mathcal{L}}{\partial r} & = \frac{d}{d t} \frac{\partial \mathcal{L}}{\partial \dot{r}} \\
  m \ddot{r}                              & = m r \omega^2                                                \\
  \ddot{r}                                & = r \omega^2                                                  \\
  r                                       & = c_1 e^{\omega t} + c_2 e^{-\omega t}
\end{align*}

If the bead is initially at rest at the origin
\begin{align*}
  0   & = c_1 + c_2               \\
  0   & = c_1 \omega - c_2 \omega \\
      & = c_1 - c_2               \\
  c_1 & = 0                       \\
  c_2 & = 0
\end{align*}
thus the bead stays at the origin.

If it is released from a point $r_0 > 0$
\begin{align*}
  r_0 & = c_1 + c_2                                    \\
  0   & = c_1 - c_2                                    \\
  c_1 & = c_2                                          \\
      & = \frac{r_0}{2}                                \\
  r   & = \frac{r_0}{2} (e^{\omega t} + e^{-\omega t})
\end{align*}
$r$ eventually grows as $\frac{r_0}{2} e^{\omega t}$.

\setcounter{subsection}{22}
\subsection{}

\begin{align*}
  K                                       & = \frac{1}{2} m v^2                                                                                               \\
                                          & = \frac{1}{2} m (\dot{x} + \dot{X})^2                                                                             \\
                                          & = \frac{1}{2} m (\dot{x} - A \omega \sin \omega t)^2                                                              \\
                                          & = \frac{1}{2} m (\dot{x}^2 - 2 \dot{x} A \omega \sin \omega t + A^2 \omega^2 \sin^2 \omega t)                     \\
  U                                       & = \frac{1}{2} k x^2                                                                                               \\
  \mathcal{L}                             & = K - U                                                                                                           \\
                                          & = \frac{1}{2} m (\dot{x}^2 - 2 \dot{x} A \omega \sin \omega t + A^2 \omega^2 \sin^2 \omega t) - \frac{1}{2} k x^2 \\
  \frac{\partial \mathcal{L}}{\partial x} & = \frac{d}{d t} \frac{\partial \mathcal{L}}{\partial \dot{x}}                                                     \\
  -k x                                    & = \frac{d}{d t} \left[ \frac{1}{2} m \left( 2 \dot{x} - 2 A \omega \sin \omega t \right) \right]                  \\
                                          & = m \ddot{x} - A m \omega^2 \cos \omega t                                                                         \\
  \ddot{x} + \frac{k}{m} x                & = A \omega^2 \cos \omega t                                                                                        \\
  \ddot{x} + \omega_0^2 x                 & = B \cos \omega t
\end{align*}

\setcounter{subsection}{26}
\subsection{}

\begin{align*}
  K                                         & = \frac{1}{2} (4 m) \dot{x}_1^2 + \frac{1}{2} (3 m) (\dot{x}_1 - \dot{x}_2)^2 + \frac{1}{2} m (\dot{x}_1 + \dot{x}_2)^2                                   \\
                                            & = 2 m \dot{x}_1^2 + \frac{3}{2} m (\dot{x}_1^2 - 2 \dot{x}_1 \dot{x}_2 + \dot{x}_2^2) + \frac{1}{2} m (\dot{x}_1^2 + 2 \dot{x}_1 \dot{x}_2 + \dot{x}_2^2) \\
                                            & = \frac{1}{2} m (4 \dot{x}_1^2 + 3 \dot{x}_1^2 - 6 \dot{x}_1 \dot{x}_2 + 3 \dot{x}_2^2 + \dot{x}_1^2 + 2 \dot{x}_1 \dot{x}_2 + \dot{x}_2^2)               \\
                                            & = 2 m (2 \dot{x}_1^2 - \dot{x}_1 \dot{x}_2 + \dot{x}_2^2)                                                                                                 \\
  U                                         & = 4 m g x_1 + 3 m g (x_2 - x_1) - m g (x_1 + x_2)                                                                                                         \\
                                            & = m g (4 x_1 + 3 x_2 - 3 x_1 - x_1 - x_2)                                                                                                                 \\
                                            & = 2 m g x_2                                                                                                                                               \\
  \mathcal{L}                               & = K - U                                                                                                                                                   \\
                                            & = 2 m (2 \dot{x}_1^2 - \dot{x}_1 \dot{x}_2 + \dot{x}_2^2) - 2 m g x_2                                                                                     \\ \\
  \frac{\partial \mathcal{L}}{\partial x_1} & = \frac{d}{d t} \frac{\partial \mathcal{L}}{\partial \dot{x}_1}                                                                                           \\
  0                                         & = \frac{d}{d t} (8 m \dot{x}_1 - 2 m \dot{x}_2)                                                                                                           \\
                                            & = 8 m \ddot{x}_1 - 2 m \ddot{x}_2                                                                                                                         \\
  \ddot{x}_1                                & = \frac{1}{4} \ddot{x}_2                                                                                                                                  \\ \\
  \frac{\partial \mathcal{L}}{\partial x_2} & = \frac{d}{d t} \frac{\partial \mathcal{L}}{\partial \dot{x}_2}                                                                                           \\
  -2 m g                                    & = \frac{d}{d t} (-2 m \dot{x}_1 + 4 m \dot{x}_2)                                                                                                          \\
                                            & = -2 m \ddot{x}_1 + 4 m \ddot{x}_2                                                                                                                        \\
  \ddot{x}_2                                & = \frac{1}{2} \ddot{x}_1 - \frac{1}{2} g                                                                                                                  \\ \\
  \ddot{x}_1                                & = \frac{1}{4} \left( \frac{1}{2} \ddot{x}_1 - \frac{1}{2} g \right)                                                                                       \\
                                            & = \frac{1}{8} \ddot{x}_1 - \frac{1}{8} g                                                                                                                  \\
  \frac{7}{8} \ddot{x}_1                    & = -\frac{1}{8} g                                                                                                                                          \\
  \ddot{x}_1                                & = -\frac{1}{7} g
\end{align*}

The acceleration of the mass $4 m$ is $\frac{1}{7} g$ downwards.

\setcounter{subsection}{28}
\subsection{}

\begin{align*}
  x           & = R \cos \omega t + l \sin \phi                                                                                                          \\
  \dot{x}     & = -R \omega \sin \omega t + l \dot{\phi} \cos \phi                                                                                       \\
  y           & = R \sin \omega t - l \cos \phi                                                                                                          \\
  \dot{y}     & = R \omega \cos \omega t + l \dot{\phi} \sin \phi                                                                                        \\
  K           & = \frac{1}{2} m v^2                                                                                                                      \\
              & = \frac{1}{2} m (\dot{x}^2 + \dot{y}^2)                                                                                                  \\
              & = \frac{1}{2} m [R^2 \omega^2 + l^2 \dot{\phi}^2 + 2 R \omega l \dot{\phi} \sin (\phi - \omega t)]                                       \\
  U           & = m g y                                                                                                                                  \\
              & = m g (R \sin \omega t - l \cos \phi)                                                                                                    \\
  \mathcal{L} & = K - U                                                                                                                                  \\
              & = \frac{1}{2} m [R^2 \omega^2 + l^2 \dot{\phi}^2 + 2 R \omega l \dot{\phi} \sin (\phi - \omega t)] - m g (R \sin \omega t - l \cos \phi) \\
\end{align*}

\begin{align*}
  \frac{\partial \mathcal{L}}{\partial \phi}                       & = \frac{d}{d t} \frac{\partial \mathcal{L}}{\partial \dot{\phi}}                \\
  m R \omega l \dot{\phi} \cos (\phi - \omega t) - m g l \sin \phi & = \frac{d}{d t} [m l^2 \dot{\phi} + m R \omega l \sin (\phi - \omega t)]        \\
                                                                   & = m l^2 \ddot{\phi} + m R \omega l \cos (\phi - \omega t) (\dot{\phi} - \omega) \\
  l \ddot{\phi}                                                    & = R \omega^2 \cos (\phi - \omega t) - g \sin \phi
\end{align*}

\setcounter{subsection}{30}
\subsection{}

\begin{enumerate}
  \item

        \begin{align*}
          X           & = x + L \sin \phi                                                                                                                       \\
          \dot{X}     & = \dot{x} + L \dot{\phi} \cos \phi                                                                                                      \\
          y           & = -L \cos \phi                                                                                                                          \\
          \dot{y}     & = L \dot{\phi} \sin \phi                                                                                                                \\ \\
          K           & = \frac{1}{2} m \dot{x}^2 + \frac{1}{2} M v^2                                                                                           \\
                      & = \frac{1}{2} (m + M) \dot{x}^2 + \frac{1}{2} L M \dot{\phi} (2 \dot{x} \cos \phi + L \dot{\phi})                                       \\ \\
          U           & = \frac{1}{2} k x^2 - M g L \cos \phi                                                                                                   \\ \\
          \mathcal{L} & = \frac{1}{2} (m + M) \dot{x}^2 + \frac{1}{2} L M \dot{\phi} (2 \dot{x} \cos \phi + L \dot{\phi}) - \frac{1}{2} k x^2 + M g L \cos \phi
        \end{align*}

        \begin{align*}
          \frac{\partial \mathcal{L}}{\partial x}    & = \frac{d}{d t} \frac{\partial \mathcal{L}}{\partial \dot{x}}             \\
          -k x                                       & = (m + M) \ddot{x} + L M (\ddot{\phi} \cos \phi - \dot{\phi}^2 \sin \phi) \\ \\
          \frac{\partial \mathcal{L}}{\partial \phi} & = \frac{d}{d t} \frac{\partial \mathcal{L}}{\partial \dot{\phi}}          \\
          M (L \ddot{\phi} + \ddot{x} \cos \phi)     & = -M g \sin \phi
        \end{align*}

  \item

        \begin{align*}
          -k x                         & = (m + M) \ddot{x} + L M (\ddot{\phi} - \phi \dot{\phi}^2) \\
                                       & \approx (m + M) \ddot{x} + L M \ddot{\phi}                 \\ \\
          M (L \ddot{\phi} + \ddot{x}) & = -M g \phi
        \end{align*}
\end{enumerate}

\setcounter{subsection}{32}
\subsection{}

\begin{align*}
  X                                       & = -x \cos \omega t                                                                                         \\
  \dot{X}                                 & = -\dot{x} \cos \omega t + \omega x \sin \omega t                                                          \\
  y                                       & = x \sin \omega t                                                                                          \\
  \dot{y}                                 & = \dot{x} \sin \omega t + \omega x \cos \omega t                                                           \\
  v^2                                     & = \dot{X}^2 + \dot{y}^2                                                                                    \\
                                          & = (-\dot{x} \cos \omega t + \omega x \sin \omega t)^2 + (\dot{x} \sin \omega t + \omega x \cos \omega t)^2 \\
                                          & = \dot{x}^2 + \omega^2 x^2                                                                                 \\
  K                                       & = \frac{1}{2} m v^2                                                                                        \\
                                          & = \frac{1}{2} m (\dot{x}^2 + \omega^2 x^2)                                                                 \\
  U                                       & = m g y                                                                                                    \\
                                          & = m g x \sin \omega t                                                                                      \\
  \mathcal{L}                             & = K - U                                                                                                    \\
                                          & = \frac{1}{2} m (\dot{x}^2 + \omega^2 x^2) - m g x \sin \omega t                                           \\
  \frac{\partial \mathcal{L}}{\partial x} & = \frac{d}{d t} \frac{\partial \mathcal{L}}{\partial \dot{x}}                                              \\
  \ddot{x} - \omega^2 x                   & = -g \sin \omega t                                                                                         \\
  x                                       & = x_0 \cosh \omega t + \frac{g}{2 \omega^2} (\sin \omega - \sinh \omega t)
\end{align*}

\setcounter{subsection}{34}
\subsection{}

\begin{align*}
  x           & = R \cos \omega t + R \cos (\omega t + \phi)                                                        \\
  \dot{x}     & = -R [\omega \sin \omega t + (\omega + \dot{\phi}) \sin (\omega t + \phi)]                          \\
  y           & = R [\sin \omega t + \sin (\omega t + \phi)]                                                        \\
  \dot{y}     & = R [\omega \cos \omega t + (\omega + \dot{\phi}) \cos (\omega t + \phi)]                           \\
  v^2         & = R^2 [\omega^2 + (\omega + \dot{\phi})^2 + 2 \omega (\omega + \dot{\phi}) \cos \phi]               \\
  \mathcal{L} & = K - U                                                                                             \\
              & = \frac{1}{2} m v^2                                                                                 \\
              & = \frac{1}{2} m R^2 [\omega^2 + (\omega + \dot{\phi})^2 + 2 \omega (\omega + \dot{\phi}) \cos \phi]
\end{align*}

\begin{align*}
  \frac{\partial \mathcal{L}}{\partial \phi}    & = \frac{d}{d t} \frac{\partial \mathcal{L}}{\partial \dot{\phi}} \\
  -m \omega (\omega + \dot{\phi}) R^2 \sin \phi & = m R^2 (\ddot{\phi} - \omega \dot{\phi} \sin \phi)              \\
  -\omega^2 \sin \phi                           & = \ddot{\phi}
\end{align*}

This is the equation of motion for a simple pendulum. For small $\phi$ the angular frequency is $\omega$.

\setcounter{subsection}{36}
\subsection{}

\begin{enumerate}
  \item

        \begin{align*}
          K           & = \frac{1}{2} m \dot{r}^2 + \frac{1}{2} I \omega^2 + \frac{1}{2} m \dot{r}^2 \\
                      & = m \dot{r}^2 + \frac{1}{2} m r^2 \dot{\phi}^2                               \\
          U           & = m g r                                                                      \\
          \mathcal{L} & = K - U                                                                      \\
                      & = m \dot{r}^2 + \frac{1}{2} m r^2 \dot{\phi}^2 - m g r
        \end{align*}

  \item

        \begin{align*}
          \frac{\partial \mathcal{L}}{\partial r}    & = \frac{d}{d t} \frac{\partial \mathcal{L}}{\partial \dot{r}}    \\
          m r \dot{\phi}^2 - m g                     & = \frac{d}{d t} (2 m \dot{r})                                    \\
                                                     & = 2 m \ddot{r}                                                   \\ \\
          \frac{\partial \mathcal{L}}{\partial \phi} & = \frac{d}{d t} \frac{\partial \mathcal{L}}{\partial \dot{\phi}} \\
          0                                          & = \frac{d}{d t} (m r^2 \dot{\phi})                               \\
          \ell                                       & = m r^2 \dot{\phi}
        \end{align*}

  \item

        \begin{align*}
          \dot{\phi}                                    & = \frac{\ell}{m r^2}                     \\ \\
          m r \left( \frac{\ell}{m r^2} \right)^2 - m g & = 2 m \ddot{r}                           \\
          \frac{\ell^2}{m r^3} - m g                    & = 2 m \ddot{r}                           \\
          m \ddot{r}                                    & = \frac{\ell^2}{2 m r^3} - \frac{m g}{2} \\ \\
          \frac{\ell^2}{2 m r_0^3} - \frac{m g}{2}      & = 0                                      \\
          \frac{2 m r_0^3}{\ell^2}                      & = \frac{2}{m g}                          \\
          r_0                                           & = \sqrt[3]{\frac{\ell^2}{m^2 g}}
        \end{align*}
\end{enumerate}

\setcounter{subsection}{38}
\subsection{}

\begin{enumerate}
  \item

        \begin{align*}
          K           & = \frac{1}{2} m \dot{r}^2 + \frac{1}{2} I_\theta \dot{\theta}^2 + \frac{1}{2} I_\phi \dot{\phi}^2           \\
                      & = \frac{1}{2} m \dot{r}^2 + \frac{1}{2} m r^2 \dot{\theta}^2 + \frac{1}{2} m (r \sin \theta)^2 \dot{\phi}^2 \\
                      & = \frac{1}{2} m [\dot{r}^2 + (r \dot{\theta})^2 + (r \sin \theta \dot{\phi})^2]                             \\
          \mathcal{L} & = K - U                                                                                                     \\
                      & = \frac{1}{2} m [\dot{r}^2 + (r \dot{\theta})^2 + (r \sin \theta \dot{\phi})^2] - U(r)
        \end{align*}

  \item

        \begin{align*}
          \frac{\partial \mathcal{L}}{\partial r}                                  & = \frac{d}{d t} \frac{\partial \mathcal{L}}{\partial \dot{r}}      \\
          m r \dot{\theta}^2 + m r \sin^2 \theta \dot{\phi}^2 - \frac{d U(r)}{d r} & = m \ddot{r}                                                       \\ \\
          \frac{\partial \mathcal{L}}{\partial \theta}                             & = \frac{d}{d t} \frac{\partial \mathcal{L}}{\partial \dot{\theta}} \\
          m r^2 \sin \theta \cos \theta \dot{\phi}^2                               & = \frac{d}{d t} (m r^2 \dot{\theta})                               \\ \\
          \frac{\partial \mathcal{L}}{\partial \phi}                               & = \frac{d}{d t} \frac{\partial \mathcal{L}}{\partial \dot{\phi}}   \\
          0                                                                        & = \frac{d}{d t} (m r^2 \sin^2 \theta \dot{\phi})                   \\
        \end{align*}

  \item The motion remains in the equatorial plane.

  \item The motion remains on that line of longitude.
\end{enumerate}

\setcounter{subsection}{40}
\subsection{}

\begin{align*}
  z                                                          & = k \rho^2                                                                                           \\
  \dot{z}                                                    & = 2 k \rho \dot{\rho}                                                                                \\
  \phi                                                       & = \omega t                                                                                           \\
  \dot{\phi}                                                 & = \omega                                                                                             \\
  K                                                          & = \frac{1}{2} m \dot{\rho}^2 + \frac{1}{2} I_\phi \omega^2 + \frac{1}{2} m \dot{z}^2                 \\
                                                             & = \frac{1}{2} m \dot{\rho}^2 + \frac{1}{2} m \rho^2 \omega^2 + \frac{1}{2} m (2 k \rho \dot{\rho})^2 \\
                                                             & = \frac{1}{2} m [\dot{\rho}^2 + (\rho \omega)^2 + (2 k \rho \dot{\rho})^2]                           \\
  U                                                          & = m g z                                                                                              \\
                                                             & = m g k \rho^2                                                                                       \\
  \mathcal{L}                                                & = K - U                                                                                              \\
                                                             & = \frac{1}{2} m [\dot{\rho}^2 + (\rho \omega)^2 + (2 k \rho \dot{\rho})^2] - m g k \rho^2            \\
  \frac{\partial \mathcal{L}}{\partial \rho}                 & = \frac{d}{d t} \frac{\partial \mathcal{L}}{\partial \dot{\rho}}                                     \\
  m \rho \omega^2 + 4 m k^2 \rho \dot{\rho}^2 - 2 m g k \rho & = \frac{d}{d t} (m \dot{\rho} + 4 m k^2 \rho^2 \dot{\rho})                                           \\
                                                             & = m \ddot{\rho} + 8 m k^2 \rho \dot{\rho}^2 + 4 m k^2 \rho^2 \ddot{\rho}                             \\
  (1 + 4 k^2 \rho^2) \ddot{\rho} + 4 k^2 \rho \dot{\rho}^2   & = (\omega^2 - 2 g k) \rho
\end{align*}

\end{document}