\documentclass{article}
\usepackage{amsfonts} % For \mathbb
\usepackage{amsmath} % For align*
\usepackage{enumitem} % For customisable list labels
\usepackage{graphicx} % For images
\usepackage{siunitx} % For units
\graphicspath{{./images/}}

\renewcommand{\vec}[1]{\boldsymbol{\mathbf{#1}}}
\newcommand{\dvec}[1]{\dot{\vec{#1}}}
\newcommand{\ddvec}[1]{\ddot{\vec{#1}}}
\newcommand{\uvec}[1]{\hat{\vec{#1}}}

\title{Classical Mechanics by John R. Taylor Notes}
\author{Chris Doble}
\date{August 2023}

\begin{document}

\maketitle

\tableofcontents

\section{Newton's Laws of Motion}

\setcounter{subsection}{1}
\subsection{Space and Time}

\begin{itemize}
  \item In cartesian coordinates the basis vectors don't depend on time so their derivatives are $\mathbf{0}$. This means that \begin{align*}
          \frac{d}{d t} (x \uvec{x} + y \uvec{y} + z \uvec{z}) & = \frac{d x}{d t} \uvec{x} + x \frac{d \uvec{x}}{d t} + \frac{d y}{d t} \uvec{y} + y \frac{d \uvec{y}}{d t} + \frac{d z}{d t} \uvec{z} + z \frac{d \uvec{z}}{d t} \\
                                                               & = \frac{d x}{d t} \uvec{x} + \frac{d y}{d t} \uvec{y} + \frac{d z}{d t} \uvec{z}
        \end{align*} as expected. However, in order coordinate systems (e.g. polar, spherical) the basis vectors may depend on time and their derivatives aren't $\vec{0}$.
\end{itemize}

\setcounter{subsection}{3}
\subsection{Newton's First and Second Laws; Inertial Frames}

\begin{itemize}
  \item Newton's second law $\vec{F} = m \vec{a}$ can be restated as $\vec{F} = \dvec{p}$.

  \item An inertial frame is one where Newton's first law holds. Typically this means the frame isn't accelerating or rotating.
\end{itemize}

\subsection{The Third Law and Conservation of Momentum}

\begin{itemize}
  \item Forces that act along the line joining two objects are called \textbf{central forces}.

  \item The \textbf{principle of conservation of momentum} states that if the net external force $\vec{F}_\text{ext}$ on an $N$-particle system is zero, the system's total momentum $\vec{P}$ is constant.
\end{itemize}

\setcounter{subsection}{6}
\subsection{Two-Dimensional Polar Coordinates}

\begin{itemize}
  \item In two-dimensional polar coordinates, the unit vectors $\uvec{r}$ and $\uvec{\phi}$ depend on position and thus time. Their derivatives are \begin{align*}
          \frac{d \uvec{r}}{d t}    & = \dot{\phi} \uvec{\phi} \\
          \frac{d \uvec{\phi}}{d t} & = -\dot{\phi} \uvec{r}.
        \end{align*}

        Consequently, the derivatives of the position vector $\vec{r} = r \uvec{r}$ are \begin{align*}
          \frac{d \vec{r}}{d t} & = \frac{d}{d t} (r \uvec{r})                  \\
                                & = \dot{r} \uvec{r} + r \frac{d \uvec{r}}{d t} \\
                                & = \dot{r} \uvec{r} + r \dot{\phi} \uvec{\phi}
        \end{align*} and \begin{align*}
          \frac{d^2 \vec{r}}{d t^2} & = \frac{d}{d t} (\dot{r} \uvec{r} + r \dot{\phi} \uvec{\phi})                                                                                              \\
                                    & = \ddot{r} \uvec{r} + \dot{r} \frac{d \uvec{r}}{d t} + \dot{r} \dot{\phi} \uvec{\phi} + r \ddot{\phi} \uvec{\phi} + r \dot{\phi} \frac{d \uvec{\phi}}{d t} \\
                                    & = \ddot{r} \uvec{r} + \dot{r} \dot{\phi} \uvec{\phi} + \dot{r} \dot{\phi} \uvec{\phi} + r \ddot{\phi} \uvec{\phi} - r \dot{\phi}^2 \uvec{r}                \\
                                    & = (\ddot{r} - r \dot{\phi}^2) \uvec{r} + (r \ddot{\phi} + 2 \dot{r} \dot{\phi}) \uvec{\phi}.
        \end{align*}

  \item In light of the above, Newton's second law in polar coordinates can be written \begin{align*}
          F_r    & = m (\ddot{r} - r \dot{\phi}^2)             \\
          F_\phi & = m (r \ddot{\phi} + 2 \dot{r} \dot{\phi}).
        \end{align*}
\end{itemize}

\section{Projectiles and Charged Particles}

\subsection{Air Resistance}

\begin{itemize}
  \item Air resistance depends on the speed $v$ of the moving object. For many objects the direction of the air resistance force $\vec{f}$ is opposite to $\vec{v}$, but not always. For example, the air resistance force on an airplane causes lift.

  \item An air resistance force can be described by the equation \[\vec{f} = -f(v) \uvec{v}\] where $\uvec{v} = \vec{v} / |\vec{v}|$ gives the direction and $f(v)$ gives the magnitude.

  \item $f(v)$ can be approximated as \[f(v) = f_\text{lin} + f_\text{quad} = b v + c v^2.\]

  \item The linear term $f_\text{lin}$ arises from the viscous drag of the medium and is generally proportional to the projectile's linear size.

  \item The quadratic term $f_\text{quad}$ arises from the fact that the projectile must accelerate the air with which it is continually colliding and it is proportional to the density of the medium and the cross-sectional area of the projectile.

  \item For a spherical projectile the coefficients $b$ and $c$ above have the form \[b = \beta D \text{ and } c = \gamma D^2\] where $D$ is the diameter of the sphere and the coefficients $\beta$ and $\gamma$ depend on the nature of the medium. In air at STP they have approximate values \[\beta = \qty{1.6e-4}{N.s/m^2}\] and \[\gamma = \qty{0.25}{N.s^2/m^4}.\]

  \item Depending on the natures of the medium and projectile it's often possible to neglect one of the terms in $f(v)$. To determine if this is the case we can calculate their ratio. For example, for a spherical projectile at STP \[\frac{f_\text{quad}}{f_\text{lin}} = \frac{c v^2}{b v} = \frac{\gamma D}{\beta} v = (\qty{1.6e3}{s/m^2}) D v.\] If the ratio is large $f_\text{lin}$ can be ignored. If it's small $f_\text{quad}$ can be ignored.

  \item The \textbf{Reynolds number} can be used to characterise the behaviour of an object in a fluid \[R = \frac{\rho}{\mu} D v\] where $\rho$ is the medium's density, $\mu$ is its viscosity, $D$ is the linear dimension of the projectile (diameter for spherical projectiles), and $v$ is the projectile's speed. The quadratic force $f_\text{quad}$ is dominant when the Reynolds number $R$ is large and the linear force $f_\text{linear}$ is dominant when it is small.
\end{itemize}

\subsection{Linear Air Resistance}

\begin{itemize}
  \item When the quadratic drag force is negligible the equation of motion becomes \begin{align*}
          \vec{F}    & = \vec{W} - \vec{f}      \\
          m \vec{a}  & = m \vec{g} - b \vec{v}  \\
          m \dvec{v} & = m \vec{g} - b \vec{v}.
        \end{align*} This is a first-order differential equation for $\vec{v}$ where the horizontal and vertical components can be separated to \begin{align*}
          m \dot{v}_x & = -b v_x       \\
          m \dot{v}_y & = m g - b v_y,
        \end{align*} each of which is easily solvable.

  \item The \textbf{terminal speed} of an object undergoing freefall and experiencing only linear drag is \[v_\text{ter} = \frac{m g}{b}.\]

  \item The \textbf{characteristic time} \[\tau = \frac{1}{k} = \frac{1}{b / m} = \frac{m}{b}\] is a measure of the importance of air resistance.

        \begin{itemize}
          \item For horizontal motion with drag it's a measure of the time it takes for the projectile to reach $1 / e$ of its initial velocity.

          \item For freefall with drag it's a measure of the time it would take the projectile to reach its terminal velocity if it didn't experience drag \[v_\text{ter} = g \tau.\]

          \item For freefall with drag it can also be used to gauge what percentage of its terminal velocity a projectile will reach after a certain time:

                \begin{center}
                  \begin{tabular}{c c}
                    Time $t$ & Percent of $v_\text{ter}$ \\
                    \hline
                    $0$      & $0$                       \\
                    $\tau$   & $63\%$                    \\
                    $2 \tau$ & $86\%$                    \\
                    $3 \tau$ & $95\%$
                  \end{tabular}
                \end{center}

                From this it can be seen that after $t = 3 \tau$ the projectile has effectively reached its terminal velocity.
        \end{itemize}
\end{itemize}

\setcounter{subsection}{3}
\subsection{Quadratic Air Resistance}

\begin{itemize}
  \item Equations of motion for quadratic air resistance can be solved analytically when the projectile moves in one dimension, but can only be solved numerically when it moves in multiple dimensions.

  \item When a projectile moves in one dimension and only experiences the force of air resistance (i.e. there are no other forces), the equation of motion is \[m \dot{v} = -c v^2.\] Using separation of variables the solution can be found to be \[v(t) = \frac{v_0}{1 + t / \tau}\] where \[\tau = \frac{m}{c v_0}.\]

  \item As in the linear case, $\tau$ is a measure of how long it takes for air resistance to slow down the projectile ($v = v_0 / 2$ at $t = \tau$).

  \item Integrating the equation for $v(t)$ gives \[x(t) = v_0 \tau \ln \left( 1 + \frac{t}{\tau} \right).\]

  \item When a projectile moves in one dimension and experiences the forces of air resistance and weight, the equation of motion (with $y$ down) is \[m \dot{v} = m g - c v^2.\] Using separation of variables the solution can be found to be \[v(t) = v_\text{ter} \tanh \frac{g t}{v_\text{ter}}\] where \[v_\text{ter} = \sqrt{\frac{m g}{c}}.\]

  \item Integrating the equation for $v(t)$ gives \[y = \frac{v_\text{ter}^2}{g} \ln \left( \cosh \frac{g t}{v_\text{ter}} \right).\]
\end{itemize}

\subsection{Motion of a Charge in a Uniform Magnetic Field}

\begin{itemize}
  \item When a particle of charge $q$ moves in a magnetic field $\vec{B} = (0, 0, B_z)$ with velocity $\vec{v} = (v_x, v_y, v_z)$ it experiences a force \[\vec{F} = q \vec{v} \times \vec{B} = q (v_y B, -b_x B, 0).\] This gives the coupled equations of motion \begin{align*}
          m \dot{v}_x & = q B v_y  \\
          m \dot{v}_y & = -q B v_x \\
          m \dot{v}_z & = 0
        \end{align*} or \begin{align*}
          \dot{v}_x & = \omega v_y  \\
          \dot{v}_y & = -\omega v_x \\
          \dot{v}_z & = 0
        \end{align*} where $\omega = q B / m$ is called the \textbf{cyclotron frequency}.

  \item If we define a complex value \[\eta = v_x + i v_y,\] its derivative is \begin{align*}
          \dot{\eta} & = \dot{v}_x + i \dot{v}_y   \\
                     & = \omega v_y - i \omega v_x \\
                     & = -i \omega \eta
        \end{align*} which has the solution \[\eta = A e^{-i \omega t}.\]
\end{itemize}

\section{Momentum and Angular Momentum}

\subsection{Conservation of Momentum}

\begin{itemize}
  \item The \textbf{principle of conservation of momentum} states that if the net external force $\vec{F}_\text{ext}$ on an $N$-particle system is zero, the system's total mechanical momentum $\vec{P} = \sum m_\alpha v_\alpha$ is constant.
\end{itemize}

\subsection{Rockets}

\begin{itemize}
  \item Newton's second law for a rocket is \[m \dot{v} = -\dot{m} v_\text{ex}\] where $\dot{m}$ is the (negative) rate of change of the mass of the rocket and $v_\text{ex}$ is the velocity of the exhaust. The quantity on the right hand side of the equation is called the \textbf{thrust}.

  \item The equation above can be solved by separation of variables giving \[v - v_0 = v_\text{ex} \ln \frac{m_0}{m}\] which is often called the \textbf{rocket equation}.
\end{itemize}

\subsection{The Center of Mass}

\begin{itemize}
  \item The \textbf{centre of mass} of a system is defined to be \[\vec{R} = \frac{1}{M} \sum_{\alpha = 1}^N m_\alpha \vec{r}_\alpha\] where $M$ is the total mass of all particles in the system, $m_\alpha$ is the mass of particle $\alpha$, and $\vec{r}_\alpha$ is the vector from the origin to particle $\alpha$.

  \item The total momentum of a system can be written in terms of its centre of mass \[\vec{P} = \sum_\alpha \vec{p}_\alpha = \sum_\alpha m_\alpha \dvec{r}_\alpha = M \dvec{R}\] i.e. the total momentum of $N$ particles is equivalent to that of a single particle of mass $M$ with velocity equal to that of the centre of mass.

  \item Differentiating the above we find \begin{align*}
          \frac{d}{d t} \vec{P} & = \frac{d}{d t} (M \dvec{R}) \\
          \vec{F}_\text{ext}    & = M \ddvec{R}
        \end{align*} i.e. the centre of mass moves as if it was a single particle of mass $M$ subject to the net external force on the system.

  \item When a body is continuous the expression for its centre of mass becomes an integral \[\vec{R} = \frac{1}{M} \int \vec{r} \,d m = \frac{1}{M} \int \rho \vec{r} \,d V.\]
\end{itemize}

\subsection{Angular Momentum for a Single Particle}

\begin{itemize}
  \item The \textbf{angular momentum} of a particle relative to an origin $O$ is \[\vec{L} = \vec{r} \times \vec{p}\] where $\vec{r}$ is measured relative to $O$.

  \item Taking the derivative of angular momentum gives \begin{align*}
          \frac{d}{d t} \vec{L} & = \frac{d}{d t} (\vec{r} \times \vec{p})            \\
          \dvec{L}              & = \dvec{r} \times \vec{p} + \vec{r} \times \dvec{p} \\
                                & = \vec{r} \times \vec{F}                            \\
                                & = \vec{\tau}.
        \end{align*} In other words, the rate of change in angular momentum about an origin $O$ is equal to the net torque about that origin.

  \item We can simplify some one-particle problems by choosing the origin such that the net torque is $0$ and thus angular momentum is constant.
\end{itemize}

\subsection{Angular Momentum for Several Particles}

\begin{itemize}
  \item The \textbf{total angular momentum} of a system is \[\vec{L} = \sum_{\alpha = 1}^N \vec{L}_\alpha = \sum_{\alpha = 1}^N \vec{r}_\alpha \times \vec{p}_\alpha.\]

  \item Differentiating the above \[\dvec{L} = \sum_\alpha \dvec{L}_\alpha = \sum_\alpha \vec{r}_\alpha \times \vec{F}_\alpha = \vec{\tau}_\text{ext}\] we find that the rate of change of the total angular momentum of the system is equal to the net torque on the system.

  \item The \textbf{principle of conservation of angular momentum} states that if the net external torque on a system is $0$, the system's total angular momentum is constant. This assumes that all internal forces are central and obey Newton's third law.

  \item The principle of conservation of momentum and the result $\dvec{L} = \vec{\tau}_\text{ext}$ also hold if $\vec{L}$ and $\vec{\tau}_\text{ext}$ are measured about the centre of mass, even if the centre of mass is being accelerated and is thus not an interial frame.
\end{itemize}

\section{Energy}

\subsection{Kinetic Energy and Work}

\begin{itemize}
  \item The \textbf{work-kinetic-energy theorem} states that the change in a particle's kinetic energy between two points is equal to the work done by the net force on the particle between those two points \[\Delta K = \int_1^2 \vec{F} \cdot d \vec{r}.\]
\end{itemize}

\subsection{Potential Energy and Conservative Forces}

\begin{itemize}
  \item A force $\vec{F}$ acting on a particle is considered \textbf{conservative} if:

        \begin{itemize}
          \item $\vec{F}$ depends only on the particle's position $\vec{r}$ (and not on its velocity $\vec{v}$, time $t$, or any other variable), and

          \item for any two points $1$ and $2$, the work done by $\vec{F}$ is the same for all paths between $1$ and $2$.
        \end{itemize}

  \item Only conservative forces have associated \textbf{potential energy} functions.

  \item The potential energy function $U(\vec{r})$ of a conservative force $\vec{F}$ is defined as \[U(\vec{r}) = -\int_{\vec{r}_0}^{\vec{r}} \vec{F}(\vec{r}') \,d \vec{r}'\] where $\vec{r}_0$ is an arbitrary point at which $U(\vec{r}_0)$ is defined to be $0$.

  \item The \textbf{principle of conservation of energy} states that if all the forces acting on a particle are conservative, each with its corresponding potential energy function $U_i(\vec{r})$, the \textbf{total mechanical energy} \[E = K + U = K + U_1(\vec{r}) + \cdots + U_n(\vec{r}),\] is constant in time.

  \item If nonconservative forces do work then the total energy of the system changes by that amount \[\Delta E = W_\text{nc}.\]
\end{itemize}

\subsection{Force as the Gradient of Potential Energy}

\begin{itemize}
  \item A conservative force $\vec{F}$ can be expressed as the negative gradient of its potential energy function $U$ \[\vec{F} = -\nabla U.\]
\end{itemize}

\subsection{The Second Condition that F be Conservative}

\begin{itemize}
  \item A force $\vec{F}$ is conservative if $\nabla \times \vec{F} = \vec{0}$.
\end{itemize}

\subsection{Time-Dependent Potential Energy}

\begin{itemize}
  \item If a time-dependent force $\vec{F}(t)$ has the property $\nabla \times \vec{F}(t) = \vec{0}$ it's still possible to define an associated potential energy function $U(\vec{r}, t)$ where $\vec{F}(t) = -\nabla U(t)$ but it's no longer guaranteed that total mechanical energy is conserved over time.
\end{itemize}

\setcounter{subsection}{7}
\subsection{Central Forces}

\begin{itemize}
  \item A central force is conservative if and only if it's spherically symmetric.
\end{itemize}

\section{Oscillations}

\setcounter{subsection}{1}
\subsection{Simple Harmonic Motion}

\begin{itemize}
  \item The equation of motion for a harmonic oscillator \[\ddot{x} = -\frac{k}{m} x = -\omega^2 x\] can be solved in multiple ways:

        \begin{itemize}
          \item the exponential solution \[x = c_1 e^{i \omega t} + c_2 e^{-i \omega t},\]

          \item the sine and cosine solutions \[x = c_1 \cos \omega t + c_2 \sin \omega t,\] and

          \item the phase shifted cosine solution \[x = A \cos (\omega t - \delta)\] where \[A = \sqrt{c_1^2 + c_2^2}\] with $c_1$ and $c_2$ coming from the sine and cosine solutions above and \[\delta = \arctan -\frac{c_1}{c_2}.\]
        \end{itemize}
\end{itemize}

\subsection{Two-Dimensional Oscillators}

\begin{itemize}
  \item An \textbf{isotropic harmonic oscillator} in $n > 1$ dimensional space experiences a restoring force directed towards the equilibrium position and with magnitude $k r$ where $r$ is the object's distance from equilibrium.

  \item In two dimensions an isotropic harmonic oscillator has general solutions \begin{align*}
          x(t) & = A_x \cos \omega t             \\
          y(t) & = A_y \cos (\omega t - \delta).
        \end{align*} It was possible to eliminate the phase from $x(t)$ by redefining the origin of time but in general it isn't possible to also eliminate the phase from $y(t)$.

  \item An \textbf{anisotropic harmonic oscillator} is similar to an isotropic harmonic oscillator but the spring constants are different in different directions.
\end{itemize}

\setcounter{subsection}{6}
\subsection{Fourier Series}

\begin{itemize}
  \item Any periodic function with period $T$ can be represented as a Fourier series \[f(t) = \sum_{n = 0}^\infty [a_n \cos (n \omega t) + b_n \sin (n \omega t)]\] where \begin{align*}
          a_0    & = \frac{1}{T} \int_{-T / 2}^{T / 2} f(t) \,d t                   \\
          a_n    & = \frac{2}{T} \int_{-T / 2}^{T / 2} f(t) \cos (n \omega t) \,d t \\
          b_0    & = 0                                                              \\
          b_n    & = \frac{2}{T} \int_{-T / 2}^{T / 2} f(t) \sin (n \omega t) \,d t \\
          \omega & = \frac{2 \pi}{T}.
        \end{align*}
\end{itemize}

\section{Calculus of Variations}

\begin{itemize}
  \item A \textbf{functional} is a mapping from a space $X$ to the real or complex numbers. When $X$ is the space of functions a functional is a ``function of a function'', i.e. it takes a function as an argument.

  \item The goal of the \textbf{calculus of variations} is to find maxima and minima of functionals, i.e. functions that maximise or minimise the value of the functional. This is analogous to finding real numbers that maximise or minimise a function in single-variable calculus.

  \item A functional of the form \[S = \int_{x_1}^{x_2} f[x, y(x), y'(x)] \,dx\] can be solved using the \textbf{Euler-Lagrange equation} \[\frac{\partial f}{\partial y} - \frac{d}{d x} \frac{\partial f}{\partial y'} = 0.\]

  \item A solution to the Euler-Lagrange equation isn't guaranteed to be a minimum — it could be a maximum or an inflection point, as in single-variable calculus. In general it's difficult to determine the nature of a given solution so other methods (e.g. inspection) must be used.

  \item A functional with multiple functions as arguments, e.g. \[S = \int_{t_1}^{t_2} f[t, x(t), x'(t), y(t), y'(t)] \,dt,\] results in a Euler-Lagrange equation for each function, e.g. \begin{align*}
          \frac{\partial f}{\partial x} - \frac{d}{d t} \frac{\partial f}{\partial x'} & = 0  \\
          \frac{\partial f}{\partial y} - \frac{d}{d t} \frac{\partial f}{\partial y'} & = 0.
        \end{align*} These can then be solved as above.

  \item Under Lagrangian mechanics, the independent variable is time $t$ and the dependent variable(s) depend on the system under consideration. In general they're denoted $q_1, q_2, \ldots, q_n$ and are called \textbf{generalized coordinates}.
\end{itemize}

\section{Lagrange's Equations}

\begin{itemize}
  \item Lagrangian mechanics has two advantages over Newtonian mechanics:

        \begin{itemize}
          \item Lagrange's equations have the same form in all coordinate systems, and

          \item Lagrange's equations omit the forces of constraint (e.g. the normal force that keeps a bead on a wire), simplifying calculations.
        \end{itemize}
\end{itemize}

\subsection{Lagrange's Equations for Unconstrained Motion}

\begin{itemize}
  \item The \textbf{Lagrangian function} or \textbf{Lagrangian} is defined as \[\mathcal{L} = K - U,\] i.e. the kinetic energy minus the potential energy.

  \item \textbf{Hamilton's principle} states that the actual path taken by a particle between points 1 and 2 in a given time interval $t_1$ to $t_2$ is such that the action integral \[S = \int_{t_1}^{t_2} \mathcal{L} \,dt\] is stationary when taken along the actual path, i.e. the actual path is the solution of the Euler-Lagrange equation when applied to the Lagrangian.

  \item A Lagrangian can be written in terms of any \textbf{generalized coordinates} $q_1, q_2, q_3$ providing each position $\vec{r}$ corresponds to a unique value $(q_1, q_2, q_3)$ and vice versa.

  \item The derivative of the Lagrangian with respect to $x$ \[\frac{\partial \mathcal{L}}{\partial x} = \frac{\partial}{\partial x} (K - U) = \frac{\partial}{\partial x} \left( \frac{1}{2} m (\dot{x}^2 + \dot{y}^2) - U(x, y) \right) = -\frac{\partial U(x, y)}{\partial x} = F_x\] is the $x$ component of the force while the derivative with respect to $\dot{x}$ \[\frac{\partial \mathcal{L}}{\partial \dot{x}} = \frac{\partial}{\partial \dot{x}} (K - U) = \frac{\partial}{\partial \dot{x}} \left( \frac{1}{2} m (\dot{x}^2 + \dot{y}^2) - U(x, y) \right) = m \dot{x} = p_x\] is the $x$ component of the momentum. The same applies to the $y$ and $z$ dimensions. When generalized coordinates $q_1, q_2, q_3$ are used the corresponding values behave like forces and momenta and are called \textbf{generalized forces} and \textbf{generalized momenta}, respectively.

  \item Another way of stating the above is \[\frac{\partial \mathcal{L}}{\partial q_i} = (i\text{th component of generalized force})\] and \[\frac{\partial \mathcal{L}}{\partial \dot{q}_i} = (i\text{th component of generalized momentum}).\] Using this terminology, the Euler-Lagrange equation \[\frac{\partial \mathcal{L}}{\partial q_i} = \frac{d}{d t} \frac{\partial \mathcal{L}}{\partial \dot{q}_i}\] takes the form \[(\text{generalized force}) = (\text{rate of change of generalized momentum}).\]

  \item For example, in 2D polar coordinates $(r, \phi)$ the generalized force for the $\phi$ coordinate is the torque on the particle and the generalized momentum is the angular momentum.

  \item Conservation laws can be derived from the Euler-Lagrange equations in generalized coordinates. For example, if the $i$th component of the generalized force is zero \[\frac{\partial \mathcal{L}}{\partial q_i} = 0\] then the rate of change of the $i$th component of the generalized momentum is also zero and thus it doesn't change.

  \item If the relationship between $\vec{r}$ and the generalized coordinates $q_1, q_2, \ldots, q_n$ doesn't involve $t$ the generalized coordinates are called \textbf{natural} and have some additional properties.

  \item The number of \textbf{degrees of freedom} of a system is the number of coordinates that can be independently varied in a small displacement, i.e. the number of independent ``directions'' in which the system can move from any given initial configuration.

  \item When the number of degrees of freedom of an $N$ particle system is less than $3 N$ (or $2 N$ in two dimensions), the system is said to be \textbf{constrained}.

  \item When the number of degrees of freedom of a system matches the number of generalized coordinates required to model the system, it is said to be \textbf{holonomic}.

  \item In order to apply Lagrange's equations to a system its constraints must be holonomic, i.e. they must be expressible in the form \[f(q_1, q_2, \ldots, q_n, t) = 0.\]

  \item The generalized coordinates can be measured relative to a non-inertial reference frame providing the Lagrangian $\mathcal{L} = K - U$ is originally written as inertial.
\end{itemize}

\setcounter{subsection}{5}
\subsection{Generalized Momenta and Ignorable Coordinates}

\begin{itemize}
  \item When the Lagrangian is independent of a coordinate $q_i$, that coordinate is said to be \textbf{ignorable} or \textbf{cyclic}. When choosing coordinates, it is desirable to make as many ignorable as possible.
\end{itemize}

\setcounter{subsection}{7}
\subsection{More about Conservation Laws}

\begin{itemize}
  \item If the Lagrangian is unchanged by spacial translation, the total momentum of the system is conserved.

  \item If  the Lagrangian is unchanged by time translation, the total energy of the system is conserved.
\end{itemize}

\subsection{Lagrange's Equations for Magnetic Forces}

\begin{itemize}
  \item For a given mechanical system with generalized coordinates \\ $q = (q_1, \ldots, q_n)$, a \textbf{Lagrangian} $\mathcal{L}$ is a function $\mathcal{L} (q_1, \ldots, q_n, \dot{q}_1, \ldots, \dot{q}_n, t)$ of the coordinates and velocities, such that the correct equations of motion for the system are the Lagrange equations \[\frac{\partial \mathcal{L}}{\partial q_i} = \frac{d}{d t} \frac{\partial \mathcal{L}}{\partial \dot{q}_i} \text{ for } i = 1, \ldots, n.\]

  \item It's important to note that the above does not define a unique Lagrangian function — any function $\mathcal{L}$ that gives the correct equations of motion is valid and has all the correct properties.

  \item The Lagrangian for a particle of charge $q$ and mass $m$ moving in electric and magnetic fields $\vec{E}$ and $\vec{B}$ is \[\mathcal{L} = \frac{1}{2} m \dvec{r}^2 - q (V - \dvec{r} \cdot \vec{A}).\]
\end{itemize}

\section{Two-Body Central Force Problems}

\subsection{The Problem}

\begin{itemize}
  \item If two objects that experience a conservative central force, their potential energy depends only on the distance between them \[U(\vec{r}_1, \vec{r}_2) = U(|\vec{r}_1 - \vec{r}_2|) = U(r)\] and thus the Lagrangian is \[\mathcal{L} = \frac{1}{2} m_1 \dot{r}_1^2 + \frac{1}{2} m_2 \dot{r}_2^2 - U(r).\]
\end{itemize}

\subsection{CM and Relative Coordinates; Reduced Mass}

\begin{itemize}
  \item It is simplest if the generalized coordinates are chosen to be the position of the centre of mass of the system \[\vec{R} = \frac{m_1 \vec{r}_1 + m_2 \vec{r}_2}{m_1 + m_2} = \frac{m_1 \vec{r}_1 + m_2 \vec{r}_2}{M}\] and the relative position of the two bodies \[\vec{r} = \vec{r}_1 - \vec{r}_2.\]

  \item This results in a kinetic energy \[K = \frac{1}{2} (M \dvec{R}^2 + \mu \dvec{r}^2)\] where \[\mu = \frac{m_1 m_2}{M} = \frac{m_1 m_2}{m_1 + m_2}\] is the \textbf{reduced mass} of the system.

  \item The Lagrangian is then \[\mathcal{L} = K - U = \frac{1}{2} M \dvec{R}^2 + \left[ \frac{1}{2} \mu \dvec{r}^2 - U(r) \right] = \mathcal{L}_\text{cm} + \mathcal{L}_\text{rel}\] where each generalized coordinate only appears in one ``sub-Lagrangian'' and can be solved separately.
\end{itemize}

\subsection{The Equations of Motion}

\begin{itemize}
  \item Because $\mathcal{L}_\text{cm}$ doesn't include $\vec{R}$ the equation of motion for the centre of mass is \[M \ddvec{R} = \vec{0},\] i.e. the centre of mass moves with constant velocity.

  \item The equation of relative motion is \[\mu \ddvec{r} = -\nabla U(r),\] i.e. the two bodies move as if they were a single particle of mass $\mu$ with potential energy $U(r)$.

  \item If we choose to use the inertial centre-of-mass reference frame, $\mathcal{L}_\text{cm} = 0$ and $\mathcal{L} = \mathcal{L}_\text{rel}$ becomes a one-body problem.

  \item The total angular momentum in the centre-of-mass frame is \[\vec{L} = \vec{r} \times \mu \dvec{r}.\] Because the total angular momentum is conserved — including its direction — this means that $\vec{r}$ and $\dvec{r}$ are confined to a plane that we can choose to be the $xy$ plane. The three-dimensional two-body problem has been turned into a two-dimensional one-body problem.

  \item The Lagrangian for this two-dimensional problem in polar coordinates is \[\mathcal{L} = \frac{1}{2} \mu (\dot{r}^2 + r^2 \dot{\phi}^2) - U(r).\] Because this doesn't involve $\phi$ the Lagrange equation corresponding to $\phi$ is \[\frac{\partial \mathcal{L}}{\partial \dot{\phi}} = \mu r^2 \dot{\phi} = \text{const} = \ell\] which is simply a statement of the conservation of angular momentum. The Lagrange equation corresponding to $r$ is \[\mu r \dot{\phi}^2 - \frac{d U}{d r} = \mu \ddot{r}.\]
\end{itemize}

\subsection{The Equivalent One-Dimensional Problem}

\begin{itemize}
  \item Rearranging the $\phi$ equation we find \[\dot{\phi} = \frac{\ell}{\mu r^2}\] where $\ell$ is determined by initial conditions.

  \item The radial equation can be rewritten as \begin{align*}
          \mu \ddot{r} & = -\frac{d U}{d r} + \mu r \dot{\phi}^2 \\
                       & = -\frac{d U}{d r} + F_\text{cf}
        \end{align*} where $F_\text{cf}$ is the fictitious centifugal force \[F_\text{cf} = \mu r \dot{\phi}^2 = \frac{\ell^2}{\mu r^3} = -\frac{d}{d r} \left( \frac{\ell^2}{2 \mu r^2} \right) = -\frac{d U_\text{cf}}{d r}.\]

  \item The radial equation can now be written in terms of the \textbf{effective potential energy} \[\mu \ddot{r} = -\frac{d}{d r} [U(r) + U_\text{cf}(r)] = -\frac{d}{d r} U_\text{eff}(r).\]

  \item The total energy of the one-body system is \[E = \frac{1}{2} \mu \dot{r}^2 + \frac{1}{2} \mu r^2 \dot{\phi}^2 + U(r)\] and this value is conserved.
\end{itemize}

\section{Mechanics in Noninertial Frames}

\subsection{Acceleration without Rotation}

\begin{itemize}
  \item A noninertial frame of reference $\mathcal{S}$ has acceleration $\vec{A}$ relative to an inertial frame of reference $\mathcal{S}_0$. Newton's second law can be used in $\mathcal{S}$ providing we add an extra force-like term called the \textbf{inertial force} \[m \ddvec{r} = \vec{F} - m \vec{A}.\]
\end{itemize}

\setcounter{subsection}{2}
\subsection{The Angular Velocity Vector}

\begin{itemize}
  \item The angular velocity vector $\vec{\omega}$ has direction equal to that of the axis of rotation (using the right-hand rule to disambiguate direction) and magnitude equal to the rate of rotation.

  \item If a body is rotating with angular velocity $\vec{\omega}$ about an axis through $O$, the velocity of a point $P$ (position $\vec{r}$) fixed on the body is \[\vec{v} = \vec{\omega} \times \vec{r}.\]

  \item Suppose there are two frames of reference 2 and 1. Frame 2 is rotating with angular velocity $\vec{\omega}_{21}$ relative to frame 1. A body 3 is rotating with angular velocities $\vec{\omega}_{31}$ and $\vec{\omega}_{32}$ relative to frames 1 and 2, respectively. These angular velocity vectors add such that \[\vec{\omega}_{31} = \vec{\omega}_{32} + \vec{\omega}_{21}.\]
\end{itemize}

\subsection{Time Derivatives in a Rotating Frame}

\begin{itemize}
  \item Given an inertial reference frame $\mathcal{S}_0$ and a noninertial reference frame $\mathcal{S}$ that is rotating with angular velocity $\vec{\Omega}$ relative to $\mathcal{S}_0$, the time derivative of a vector $\vec{Q}$ differs between the two with the relation \[\left( \frac{d \vec{Q}}{d t} \right)_{\mathcal{S}_0} = \left( \frac{d \vec{Q}}{d t} \right)_\mathcal{S} + \vec{\Omega} \times \vec{Q}.\]
\end{itemize}

\subsection{Newton's Second Law in a Rotating Frame}

\begin{itemize}
  \item Newton's second law in a noninertial frame rotating with angular velocity $\vec{\Omega}$ is \begin{align*}
          m \ddvec{r} & = \vec{F} + \vec{F}_\text{cor} + \vec{F}_{cf}                                                      \\
                      & = \vec{F} + 2 m \dvec{r} \times \vec{\Omega} + m (\vec{\Omega} \times \vec{r}) \times \vec{\Omega}
        \end{align*} where $\vec{F}_\text{cor}$ is the \textbf{Coriolis force} and $\vec{F}_\text{cf}$ is the \textbf{centrifugal force}.
\end{itemize}

\section{Rotational Motion of Rigid Bodies}

\subsection{Properties of the Centre of Mass}

\begin{itemize}
  \item The total angular momentum of a system of $N$ particles is \begin{align*}
          \vec{L} & = \vec{L}_\text{O} + \vec{L}_\text{CM}                                           \\
                  & = \vec{R} \times \vec{P} + \sum \vec{r}_\alpha' \times m_\alpha \dvec{r}_\alpha'
        \end{align*} where $\vec{L}_\text{O}$ is the angular momentum of the centre of mass of the system relative to the origin, $\vec{L}_\text{CM}$ is the angular momentum of the particles of the system relative to its centre of mass, $\vec{R}$ is the centre of mass of the system, $\vec{P}$ is the linear momentum of the system, $\vec{r}_\alpha'$ is the position of partical $\alpha$ relative to the centre of mass of the system, $m_\alpha$ is the mass of particle $\alpha$, and $\dvec{r}_\alpha'$ is the velocity of particle $\alpha$ relative to the centre of mass of the system.

  \item The values $\vec{L}_\text{O}$ and $\vec{L}_\text{CM}$ are independently conserved, with \[\dvec{L}_\text{O} = \vec{R} \times \vec{F}^\text{ext} = \vec{\tau}^\text{ext} \text{ (about origin)}\] and \[\dvec{L}_\text{CM} = \sum \vec{r}_\alpha' \times \vec{F}_\alpha^\text{ext} = \vec{\tau}^\text{ext} \text{ (about centre of mass)}.\]

  \item The total kinetic energy of a system of $N$ particles is \begin{align*}
          T & = T_\text{CM} + T_\text{relative to CM}                                  \\
            & = \frac{1}{2} M \dvec{R}^2 + \frac{1}{2} \sum m_\alpha \dvec{r}_\alpha^2
        \end{align*}

  \item The total potential energy of a system of $N$ particles is \[U = U^\text{ext} + U^\text{int}\] where $U^\text{ext}$ is the total potential energy due to external forces and $U^\text{int}$ is the total potential energies for all pairs of particles \[U^\text{int} = \sum_{\alpha < \beta} U_{\alpha \beta} (r_{\alpha \beta}).\] However, in a rigid body the distances between all particles are fixed so $U^\text{int}$ is constant and can be ignored.
\end{itemize}

\subsection{Rotation about a Fixed Axis}

\begin{itemize}
  \item For a system of $N$ particles rotating around the $z$ axis, the angular momentum of the system is is \begin{align*}
          \vec{L} & = \sum_\alpha \vec{\ell}_\alpha                                                                  \\
                  & =  \sum_\alpha m_\alpha \vec{r}_\alpha \times \vec{v}_\alpha                                     \\
                  & = \sum_\alpha m_\alpha \omega (-z_\alpha x_\alpha, -z_\alpha y_\alpha, x_\alpha^2 + y_\alpha^2).
        \end{align*} Thus, $\vec{L}$ may have $\vec{x}$ and/or $\vec{y}$ components and $\vec{L} = I \vec{\omega}$ doesn't hold.

  \item The angular momentum can be expressed as \[\vec{L} = (I_{xz} \omega, I_{yz} \omega, I_{zz} \omega)\] where \begin{align*}
          I_{xz} & = -\sum_\alpha m_\alpha x_\alpha z_\alpha \text{,}     \\
          I_{yz} & = -\sum_\alpha m_\alpha y_\alpha z_\alpha \text{, and} \\
          I_{zz} & = \sum_\alpha m_\alpha (x_\alpha^2 + y_\alpha^2)
        \end{align*} are called the \textbf{products of inertia} of the body. The notation means that e.g. $I_{xz}$ is the $x$ component of the angular momentum for a body rotating around the $z$ axis.
\end{itemize}

\subsection{Rotation about Any Axis; the Inertia Tensor}

\begin{itemize}
  \item Continuing from the above, if a body has angular velocity \[\vec{\omega} = (\omega_x, \omega_y, \omega_z)\] its angular momentum is \begin{align*}
          L_x & = I_{x x} \omega_x + I_{x y} \omega_y + I_{x z} \omega_z \\
          L_y & = I_{y x} \omega_x + I_{y y} \omega_y + I_{y z} \omega_z \\
          L_z & = I_{z x} \omega_x + I_{z y} \omega_y + I_{z z} \omega_z
        \end{align*} where \begin{align*}
          I_{x x} & = \sum_\alpha m_\alpha (y_\alpha^2 + z_\alpha^2), \\
          I_{x y} & = -\sum_\alpha m_\alpha x_\alpha y_\alpha,
        \end{align*} and the other values follow the same pattern.

  \item The above can also be written \[\vec{L} = \vec{I} \vec{\omega}\] where $\vec{I}$ is the \textbf{moment of inertia tensor} \[\vec{I} = \begin{pmatrix}
            I_{x x} & I_{x y} & I_{x z} \\
            I_{y x} & I_{y y} & I_{y z} \\
            I_{z x} & I_{z y} & I_{z z}
          \end{pmatrix}.\]
\end{itemize}

\subsection{Principle Axes of Inertia}

\begin{itemize}
  \item In general, the angular momentum $\vec{L}$ of a spinning object is not parallel to $\vec{\omega}$. However for any rigid body and any point $O$ there are three perpendicular axes through $O$ such that $\vec{I}$ is diagonal when $\vec{\omega}$ is parallel to one of these axes, so is $\vec{L}$. These are called the \textbf{principle axes}.

  \item The three moments of inertia about the three principle axes are called the \textbf{principle moments}.

  \item If we choose the principle axes of a rigid body as our frame of reference rotation calculations are simplified but we're using a noninertial frame of reference.
\end{itemize}

\subsection{Finding the Principle Axes; Eigenvalue Equations}

\begin{itemize}
  \item If $\vec{\omega}$ is parallel to a principle axis, then $\vec{L} = \vec{I} \vec{\omega} = \lambda \vec{\omega}$. This can be converted to an eigenvalue equation \[(\vec{I} - \lambda \vec{1}) \vec{\omega} = 0\] where the eigenvectors are the principle axes. Thus to find the principle axes of a rigid body about a point we must calculate an initial intertia tensor then calculate its eigenvectors. The updated inertia tensor will be \[\vec{I} = \begin{pmatrix}
            \lambda_1 & 0         & 0         \\
            0         & \lambda_2 & 0         \\
            0         & 0         & \lambda_3
          \end{pmatrix}.\]
\end{itemize}

\setcounter{subsection}{6}
\subsection{Euler's Equations}

\begin{itemize}
  \item The axes $x$, $y$, and $z$ are chosen to be in an inertial frame called the \textbf{space frame} while the axes $\vec{e}_1$, $\vec{e}_2$, and $\vec{e}_3$ are chosen to align with the body's principle axes and this is is a non-inertial frame known as the \textbf{body frame}. If the body is rotating around a fixed point that is taken to be the origin of the body frame, otherwise the body's centre of mass is used.

  \item Suppose a body is rotating with angular velocity $\vec{\omega}$ and has principle moments of inertia $\lambda_1$, $\lambda_2$, and $\lambda_3$, then its angular momentum in the body frame is \[\vec{L} = (\lambda_1 \omega_1, \lambda_2 \omega_2, \lambda_3 \omega_3).\] If a torque $\vec{\Gamma}$ acts on the body, then in the space frame \[\left( \frac{d \vec{L}}{d t} \right)_\text{space} = \vec{\Gamma}.\] In the body frame this becomes \begin{align*}
          \left( \frac{d \vec{L}}{d t} \right)_\text{space} & = \left( \frac{d \vec{L}}{d t} \right)_\text{body} + \vec{\omega} \times \vec{L} \\
                                                            & = \dvec{L} + \vec{\omega} \times \vec{L}.
        \end{align*} Equating the previous two equations gives \[\dvec{L} + \vec{\omega} + \vec{L} = \vec{\Gamma}\] or \begin{align*}
          \lambda_1 \dot{\omega}_1 - (\lambda_2 - \lambda_3) \omega_2 \omega_3 & = \Gamma_1  \\
          \lambda_2 \dot{\omega}_2 - (\lambda_3 - \lambda_1) \omega_3 \omega_1 & = \Gamma_2  \\
          \lambda_3 \dot{\omega}_3 - (\lambda_1 - \lambda_2) \omega_1 \omega_2 & = \Gamma_3.
        \end{align*} where all quantities are in the body frame. This is known as \textbf{Euler's equation} and it is the equation of motion for the rotation of the body.
\end{itemize}

\subsection{Euler's Equations with Zero Torque}

\begin{itemize}
  \item If a body has three different principle moments of inertia and it is rotating about one of its principle axes, its angular velocity will not change. Conversely, if it isn't rotating about one of its principle axes, its angular velocity will change.
\end{itemize}

\section{Coupled Oscillators and Normal Modes}

\subsection{Two Masses and Three Springs}

\begin{itemize}
  \item Two carts of masses $m_1$ and $m_2$ are connected to walls via springs of spring constant $k_1$ and $k_3$, respectively. The carts are joined together by another spring of spring constant $k_2$. The equations of motion for the carts are \begin{align*}
          m_1 \ddot{x}_1 & = -(k_1 + k_2) x_1 + k_2 x_2 \\
          m_2 \ddot{x}_2 & = k_2 x_1 - (k_2 + k_3) x_2
        \end{align*} or \[\vec{M} \ddvec{x} = -\vec{K} \vec{x}.\] Assuming a solution of the form \begin{align*}
          \vec{z}(t) & = \begin{pmatrix}
                           \alpha_1 e^{i (\omega t - \delta_1)} \\
                           \alpha_2 e^{i (\omega t - \delta_2)}
                         \end{pmatrix}    \\
                     & = \begin{pmatrix}
                           \alpha_1 e^{-i \delta_1} e^{i \omega t} \\
                           \alpha_2 e^{-i \delta_2} e^{i \omega t}
                         \end{pmatrix} \\
                     & = \begin{pmatrix}
                           \alpha_1 e^{-i \delta_1} \\
                           \alpha_2 e^{-i \delta_2}
                         \end{pmatrix} e^{i \omega t}            \\
                     & = \vec{a} e^{i \omega t}
        \end{align*} and substituting it into the equations of motion gives \begin{align*}
          -\omega^2 \vec{M} \vec{a} e^{i \omega t} & = -\vec{K} \vec{a} e^{i \omega t} \\
          (\vec{K} - \omega^2 \vec{M}) \vec{a}     & = 0
        \end{align*} where the solutions $\omega$ give the \textbf{normal frequencies} of the system.
\end{itemize}

\subsection{Identical Springs and Equal Masses}

\begin{itemize}
  \item Because the equations of motion are linear, a linear combination of solutions is also a solution. The general solution is thus a linear combination of a fundamental set of solutions.

  \item \textbf{Normal coordinates} are alternate coordinates for the system that result in separate differential equations.
\end{itemize}

\setcounter{subsection}{5}
\subsection{Three Coupled Pendulums}

\begin{itemize}
  \item Choosing \textbf{natural units} is the process of choosing a system of units such that uninteresting parameters, e.g. mass $m$ and length $L$, are equal to one unit of mass/length/etc. This means they have the value $1$ and can be dropped from equations.
\end{itemize}

\section{Nonlinear Mechanics and Chaos}

\subsection{Linearity and Nonlinearity}

\begin{itemize}
  \item For a system to exhibit chaos its equations of motion must be nonlinar.

  \item Nonlinearity is essential but not sufficient for chaos, e.g. the equation of motion of a simple pendulum without the small angle approximation.

  \item Nonlinear equations are complicated, in part, because the superposition principle doesn't apply — you can't just take the linear combination of $n$ independent solutions as the general solution.
\end{itemize}

\subsection{The Driven Damped Pendulum DDP}

\begin{itemize}
  \item The equation of motion for the DDP is \begin{align*}
          I \ddot{\phi}     & = \Gamma                                        \\
          m L^2 \ddot{\phi} & = -L b v - m g L \sin \phi + L F(t)             \\
                            & = -b L^2 \dot{\phi} - m g L \sin \phi + L F(t).
        \end{align*}

  \item If we assume $F(t) = F_0 \cos \omega t$ the above becomes \begin{align*}
          \ddot{\phi} + \frac{b}{m} \dot{\phi} + \frac{g}{L} \sin \phi & = \frac{F_0}{m L} \cos \omega t   \\
          \ddot{\phi} + 2 \beta \dot{\phi} + \omega_0^2 \sin \phi      & = \gamma \omega_0^2 \cos \omega t
        \end{align*} where \[\beta = \frac{b}{2 m}\] is the \textbf{damping constant} of the pendulum, \[\omega_0 = \sqrt{\frac{g}{L}}\] is the \textbf{natural frequency} of the pendulum, and \[\gamma = \frac{F_0}{m L \omega_0^2} = \frac{F_0}{m g}\] is the \textbf{drive strength} of the driving force.

  \item The drive strength is the ratio of the amplitude of the driving force to the weight of the pendulum. When $\gamma < 1$ we would expect the driving force to produce small movement and when $\gamma \ge 1$ large movement.
\end{itemize}

\subsection{Some Expected Features of the DDP}

\begin{itemize}
  \item When the initial angle, angular velocity, and driving force are small, the small angle approximation is valid and the system reaches a steady state after the transient response dies out.

  \item When the drive strength is increased but is still less than $1$ the system still exhibits periodic motion, but it is composed of harmonics of the driving force rather than a single cosine.
\end{itemize}

\subsection{The DDP: Approach to Chaos}

\begin{itemize}
  \item As the drive strength is increased over $1$ the transient behaviour becomes wild and less predictable.

  \item Continuing to increase the drive strength results in a steady state period two times the driving period — the system is said to have \textbf{period two}. This isn't possible with harmonics alone so \textbf{subharmonics} (sinusoidal terms with frequency $\omega / n$ rather than $n \omega$) must be present.

  \item Continuing to increase the drive strength results in a steady state with period three.

  \item Changing the initial conditions of the DDP and keeping all other values the same can result in different steady state solutions. This differs from the linear case where there is only one possible steady state solution.

  \item If initial conditions are kept constant and the drive strength is increased we observe a \textbf{period-doubling cascade}, i.e. the period continually doubles as the drive strength increases.

  \item Period-doubling cascades occur across many nonlinear systems and the parameter that determines when the cascades occur (the drive strength above) is called the \textbf{control parameter}.

  \item In many systems, as the control parameter changes there are certain \textbf{threshold values} or \textbf{bifurcation points} at which the period doubles. As the threshold value continues to change the rate at which the period doubles increases. If $\gamma_n$ represents bifurcation point $n$ then the intervals between successive bifurcation points are related by \[\gamma_{n + 1} - \gamma_n \approx \frac{1}{\delta} (\gamma_n - \gamma_{n - 1})\] where $\delta = \num{4.6692016}$ is called the \textbf{Feigenbaum number}.

  \item This implies that the intervals between successive bifucation points shrink as the control parameter increases, approaching a finite limit $\gamma_c$ as $n \rightarrow \infty$.

  \item If $\gamma > \gamma_c$ chaos sets in. This is why the period-doubling cascade is sometimes called a \textbf{route to chaos}.
\end{itemize}

\end{document}