\documentclass{article}
\usepackage{amsfonts} % For \mathbb
\usepackage{amsmath} % For align*
\usepackage{enumitem} % For customisable list labels
\usepackage{graphicx} % For images
\usepackage{siunitx} % For units
\graphicspath{{./images/}}

\renewcommand{\vec}[1]{\boldsymbol{\mathbf{#1}}}
\newcommand{\dvec}[1]{\dot{\vec{#1}}}
\newcommand{\ddvec}[1]{\ddot{\vec{#1}}}
\newcommand{\uvec}[1]{\hat{\vec{#1}}}

\title{Classical Mechanics by John R. Taylor Notes}
\author{Chris Doble}
\date{August 2023}

\begin{document}

\maketitle

\tableofcontents

\section{Newton's Laws of Motion}

\setcounter{subsection}{1}
\subsection{Space and Time}

\begin{itemize}
  \item In cartesian coordinates the basis vectors don't depend on time so their derivatives are $\mathbf{0}$. This means that \begin{align*}
          \frac{d}{d t} (x \uvec{x} + y \uvec{y} + z \uvec{z}) & = \frac{d x}{d t} \uvec{x} + x \frac{d \uvec{x}}{d t} + \frac{d y}{d t} \uvec{y} + y \frac{d \uvec{y}}{d t} + \frac{d z}{d t} \uvec{z} + z \frac{d \uvec{z}}{d t} \\
                                                               & = \frac{d x}{d t} \uvec{x} + \frac{d y}{d t} \uvec{y} + \frac{d z}{d t} \uvec{z}
        \end{align*} as expected. However, in order coordinate systems (e.g. polar, spherical) the basis vectors may depend on time and their derivatives aren't $\vec{0}$.
\end{itemize}

\setcounter{subsection}{3}
\subsection{Newton's First and Second Laws; Inertial Frames}

\begin{itemize}
  \item Newton's second law $\vec{F} = m \vec{a}$ can be restated as $\vec{F} = \dvec{p}$.

  \item An inertial frame is one where Newton's first law holds. Typically this means the frame isn't accelerating or rotating.
\end{itemize}

\subsection{The Third Law and Conservation of Momentum}

\begin{itemize}
  \item Forces that act along the line joining two objects are called \textbf{central forces}.

  \item The \textbf{principle of conservation of momentum} states that if the net external force $\vec{F}_\text{ext}$ on an $N$-particle system is zero, the system's total momentum $\vec{P}$ is constant.
\end{itemize}

\setcounter{subsection}{6}
\subsection{Two-Dimensional Polar Coordinates}

\begin{itemize}
  \item In two-dimensional polar coordinates, the unit vectors $\uvec{r}$ and $\uvec{\phi}$ depend on position and thus time. Their derivatives are \begin{align*}
          \frac{d \uvec{r}}{d t}    & = \dot{\phi} \uvec{\phi} \\
          \frac{d \uvec{\phi}}{d t} & = -\dot{\phi} \uvec{r}.
        \end{align*}

        Consequently, the derivatives of the position vector $\vec{r} = r \uvec{r}$ are \begin{align*}
          \frac{d \vec{r}}{d t} & = \frac{d}{d t} (r \uvec{r})                  \\
                                & = \dot{r} \uvec{r} + r \frac{d \uvec{r}}{d t} \\
                                & = \dot{r} \uvec{r} + r \dot{\phi} \uvec{\phi}
        \end{align*} and \begin{align*}
          \frac{d^2 \vec{r}}{d t^2} & = \frac{d}{d t} (\dot{r} \uvec{r} + r \dot{\phi} \uvec{\phi})                                                                                              \\
                                    & = \ddot{r} \uvec{r} + \dot{r} \frac{d \uvec{r}}{d t} + \dot{r} \dot{\phi} \uvec{\phi} + r \ddot{\phi} \uvec{\phi} + r \dot{\phi} \frac{d \uvec{\phi}}{d t} \\
                                    & = \ddot{r} \uvec{r} + \dot{r} \dot{\phi} \uvec{\phi} + \dot{r} \dot{\phi} \uvec{\phi} + r \ddot{\phi} \uvec{\phi} - r \dot{\phi}^2 \uvec{r}                \\
                                    & = (\ddot{r} - r \dot{\phi}^2) \uvec{r} + (r \ddot{\phi} + 2 \dot{r} \dot{\phi}) \uvec{\phi}.
        \end{align*}

  \item In light of the above, Newton's second law in polar coordinates can be written \begin{align*}
          F_r    & = m (\ddot{r} - r \dot{\phi}^2)             \\
          F_\phi & = m (r \ddot{\phi} + 2 \dot{r} \dot{\phi}).
        \end{align*}
\end{itemize}

\section{Projectiles and Charged Particles}

\subsection{Air Resistance}

\begin{itemize}
  \item Air resistance depends on the speed $v$ of the moving object. For many objects the direction of the air resistance force $\vec{f}$ is opposite to $\vec{v}$, but not always. For example, the air resistance force on an airplane causes lift.

  \item An air resistance force can be described by the equation \[\vec{f} = -f(v) \uvec{v}\] where $\uvec{v} = \vec{v} / |\vec{v}|$ gives the direction and $f(v)$ gives the magnitude.

  \item $f(v)$ can be approximated as \[f(v) = f_\text{lin} + f_\text{quad} = b v + c v^2.\]

  \item The linear term $f_\text{lin}$ arises from the viscous drag of the medium and is generally proportional to the projectile's linear size.

  \item The quadratic term $f_\text{quad}$ arises from the fact that the projectile must accelerate the air with which it is continually colliding and it is proportional to the density of the medium and the cross-sectional area of the projectile.

  \item For a spherical projectile the coefficients $b$ and $c$ above have the form \[b = \beta D \text{ and } c = \gamma D^2\] where $D$ is the diameter of the sphere and the coefficients $\beta$ and $\gamma$ depend on the nature of the medium. In air at STP they have approximate values \[\beta = \qty{1.6e-4}{N.s/m^2}\] and \[\gamma = \qty{0.25}{N.s^2/m^4}.\]

  \item Depending on the natures of the medium and projectile it's often possible to neglect one of the terms in $f(v)$. To determine if this is the case we can calculate their ratio. For example, for a spherical projectile at STP \[\frac{f_\text{quad}}{f_\text{lin}} = \frac{c v^2}{b v} = \frac{\gamma D}{\beta} v = (\qty{1.6e3}{s/m^2}) D v.\] If the ratio is large $f_\text{lin}$ can be ignored. If it's small $f_\text{quad}$ can be ignored.

  \item The \textbf{Reynolds number} can be used to characterise the behaviour of an object in a fluid \[R = \frac{\rho}{\mu} D v\] where $\rho$ is the medium's density, $\mu$ is its viscosity, $D$ is the linear dimension of the projectile (diameter for spherical projectiles), and $v$ is the projectile's speed. The quadratic force $f_\text{quad}$ is dominant when the Reynolds number $R$ is large and the linear force $f_\text{linear}$ is dominant when it is small.
\end{itemize}

\subsection{Linear Air Resistance}

\begin{itemize}
  \item When the quadratic drag force is negligible the equation of motion becomes \begin{align*}
          \vec{F}    & = \vec{W} - \vec{f}      \\
          m \vec{a}  & = m \vec{g} - b \vec{v}  \\
          m \dvec{v} & = m \vec{g} - b \vec{v}.
        \end{align*} This is a first-order differential equation for $\vec{v}$ where the horizontal and vertical components can be separated to \begin{align*}
          m \dot{v}_x & = -b v_x       \\
          m \dot{v}_y & = m g - b v_y,
        \end{align*} each of which is easily solvable.

  \item The \textbf{terminal speed} of an object undergoing freefall and experiencing only linear drag is \[v_\text{ter} = \frac{m g}{b}.\]

  \item The \textbf{characteristic time} \[\tau = \frac{1}{k} = \frac{1}{b / m} = \frac{m}{b}\] is a measure of the importance of air resistance.

        \begin{itemize}
          \item For horizontal motion with drag it's a measure of the time it takes for the projectile to reach $1 / e$ of its initial velocity.

          \item For freefall with drag it's a measure of the time it would take the projectile to reach its terminal velocity if it didn't experience drag \[v_\text{ter} = g \tau.\]

          \item For freefall with drag it can also be used to gauge what percentage of its terminal velocity a projectile will reach after a certain time:

                \begin{center}
                  \begin{tabular}{c c}
                    Time $t$ & Percent of $v_\text{ter}$ \\
                    \hline
                    $0$      & $0$                       \\
                    $\tau$   & $63\%$                    \\
                    $2 \tau$ & $86\%$                    \\
                    $3 \tau$ & $95\%$
                  \end{tabular}
                \end{center}

                From this it can be seen that after $t = 3 \tau$ the projectile has effectively reached its terminal velocity.
        \end{itemize}
\end{itemize}

\setcounter{subsection}{3}
\subsection{Quadratic Air Resistance}

\begin{itemize}
  \item Equations of motion for quadratic air resistance can be solved analytically when the projectile moves in one dimension, but can only be solved numerically when it moves in multiple dimensions.

  \item When a projectile moves in one dimension and only experiences the force of air resistance (i.e. there are no other forces), the equation of motion is \[m \dot{v} = -c v^2.\] Using separation of variables the solution can be found to be \[v(t) = \frac{v_0}{1 + t / \tau}\] where \[\tau = \frac{m}{c v_0}.\]

  \item As in the linear case, $\tau$ is a measure of how long it takes for air resistance to slow down the projectile ($v = v_0 / 2$ at $t = \tau$).

  \item Integrating the equation for $v(t)$ gives \[x(t) = v_0 \tau \ln \left( 1 + \frac{t}{\tau} \right).\]

  \item When a projectile moves in one dimension and experiences the forces of air resistance and weight, the equation of motion (with $y$ down) is \[m \dot{v} = m g - c v^2.\] Using separation of variables the solution can be found to be \[v(t) = v_\text{ter} \tanh \frac{g t}{v_\text{ter}}\] where \[v_\text{ter} = \sqrt{\frac{m g}{c}}.\]

  \item Integrating the equation for $v(t)$ gives \[y = \frac{v_\text{ter}^2}{g} \ln \left( \cosh \frac{g t}{v_\text{ter}} \right).\]
\end{itemize}

\subsection{Motion of a Charge in a Uniform Magnetic Field}

\begin{itemize}
  \item When a particle of charge $q$ moves in a magnetic field $\vec{B} = (0, 0, B_z)$ with velocity $\vec{v} = (v_x, v_y, v_z)$ it experiences a force \[\vec{F} = q \vec{v} \times \vec{B} = q (v_y B, -b_x B, 0).\] This gives the coupled equations of motion \begin{align*}
          m \dot{v}_x & = q B v_y  \\
          m \dot{v}_y & = -q B v_x \\
          m \dot{v}_z & = 0
        \end{align*} or \begin{align*}
          \dot{v}_x & = \omega v_y  \\
          \dot{v}_y & = -\omega v_x \\
          \dot{v}_z & = 0
        \end{align*} where $\omega = q B / m$ is called the \textbf{cyclotron frequency}.

  \item If we define a complex value \[\eta = v_x + i v_y,\] its derivative is \begin{align*}
          \dot{\eta} & = \dot{v}_x + i \dot{v}_y   \\
                     & = \omega v_y - i \omega v_x \\
                     & = -i \omega \eta
        \end{align*} which has the solution \[\eta = A e^{-i \omega t}.\]
\end{itemize}

\section{Momentum and Angular Momentum}

\subsection{Conservation of Momentum}

\begin{itemize}
  \item The \textbf{principle of conservation of momentum} states that if the net external force $\vec{F}_\text{ext}$ on an $N$-particle system is zero, the system's total mechanical momentum $\vec{P} = \sum m_\alpha v_\alpha$ is constant.
\end{itemize}

\subsection{Rockets}

\begin{itemize}
  \item Newton's second law for a rocket is \[m \dot{v} = -\dot{m} v_\text{ex}\] where $\dot{m}$ is the (negative) rate of change of the mass of the rocket and $v_\text{ex}$ is the velocity of the exhaust. The quantity on the right hand side of the equation is called the \textbf{thrust}.

  \item The equation above can be solved by separation of variables giving \[v - v_0 = v_\text{ex} \ln \frac{m_0}{m}\] which is often called the \textbf{rocket equation}.
\end{itemize}

\subsection{The Center of Mass}

\begin{itemize}
  \item The \textbf{centre of mass} of a system is defined to be \[\vec{R} = \frac{1}{M} \sum_{\alpha = 1}^N m_\alpha \vec{r}_\alpha\] where $M$ is the total mass of all particles in the system, $m_\alpha$ is the mass of particle $\alpha$, and $\vec{r}_\alpha$ is the vector from the origin to particle $\alpha$.

  \item The total momentum of a system can be written in terms of its centre of mass \[\vec{P} = \sum_\alpha \vec{p}_\alpha = \sum_\alpha m_\alpha \dvec{r}_\alpha = M \dvec{R}\] i.e. the total momentum of $N$ particles is equivalent to that of a single particle of mass $M$ with velocity equal to that of the centre of mass.

  \item Differentiating the above we find \begin{align*}
          \frac{d}{d t} \vec{P} & = \frac{d}{d t} (M \dvec{R}) \\
          \vec{F}_\text{ext}    & = M \ddvec{R}
        \end{align*} i.e. the centre of mass moves as if it was a single particle of mass $M$ subject to the net external force on the system.

  \item When a body is continuous the expression for its centre of mass becomes an integral \[\vec{R} = \frac{1}{M} \int \vec{r} \,d m = \frac{1}{M} \int \rho \vec{r} \,d V.\]
\end{itemize}

\subsection{Angular Momentum for a Single Particle}

\begin{itemize}
  \item The \textbf{angular momentum} of a particle relative to an origin $O$ is \[\vec{L} = \vec{r} \times \vec{p}\] where $\vec{r}$ is measured relative to $O$.

  \item Taking the derivative of angular momentum gives \begin{align*}
          \frac{d}{d t} \vec{L} & = \frac{d}{d t} (\vec{r} \times \vec{p})            \\
          \dvec{L}              & = \dvec{r} \times \vec{p} + \vec{r} \times \dvec{p} \\
                                & = \vec{r} \times \vec{F}                            \\
                                & = \vec{\tau}.
        \end{align*} In other words, the rate of change in angular momentum about an origin $O$ is equal to the net torque about that origin.

  \item We can simplify some one-particle problems by choosing the origin such that the net torque is $0$ and thus angular momentum is constant.
\end{itemize}

\subsection{Angular Momentum for Several Particles}

\begin{itemize}
  \item The \textbf{total angular momentum} of a system is \[\vec{L} = \sum_{\alpha = 1}^N \vec{L}_\alpha = \sum_{\alpha = 1}^N \vec{r}_\alpha \times \vec{p}_\alpha.\]

  \item Differentiating the above \[\dvec{L} = \sum_\alpha \dvec{L}_\alpha = \sum_\alpha \vec{r}_\alpha \times \vec{F}_\alpha = \vec{\tau}_\text{ext}\] we find that the rate of change of the total angular momentum of the system is equal to the net torque on the system.

  \item The \textbf{principle of conservation of angular momentum} states that if the net external torque on a system is $0$, the system's total angular momentum is constant. This assumes that all internal forces are central and obey Newton's third law.

  \item The principle of conservation of momentum and the result $\dvec{L} = \vec{\tau}_\text{ext}$ also hold if $\vec{L}$ and $\vec{\tau}_\text{ext}$ are measured about the centre of mass, even if the centre of mass is being accelerated and is thus not an interial frame.
\end{itemize}

\section{Energy}

\subsection{Kinetic Energy and Work}

\begin{itemize}
  \item The \textbf{work-kinetic-energy theorem} states that the change in a particle's kinetic energy between two points is equal to the work done by the net force on the particle between those two points \[\Delta K = \int_1^2 \vec{F} \cdot d \vec{r}.\]
\end{itemize}

\subsection{Potential Energy and Conservative Forces}

\begin{itemize}
  \item A force $\vec{F}$ acting on a particle is considered \textbf{conservative} if:

        \begin{itemize}
          \item $\vec{F}$ depends only on the particle's position $\vec{r}$ (and not on its velocity $\vec{v}$, time $t$, or any other variable), and

          \item for any two points $1$ and $2$, the work done by $\vec{F}$ is the same for all paths between $1$ and $2$.
        \end{itemize}

  \item Only conservative forces have associated \textbf{potential energy} functions.

  \item The potential energy function $U(\vec{r})$ of a conservative force $\vec{F}$ is defined as \[U(\vec{r}) = -\int_{\vec{r}_0}^{\vec{r}} \vec{F}(\vec{r}') \,d \vec{r}'\] where $\vec{r}_0$ is an arbitrary point at which $U(\vec{r}_0)$ is defined to be $0$.

  \item The \textbf{principle of conservation of energy} states that if all the forces acting on a particle are conservative, each with its corresponding potential energy function $U_i(\vec{r})$, the \textbf{total mechanical energy} \[E = K + U = K + U_1(\vec{r}) + \cdots + U_n(\vec{r}),\] is constant in time.

  \item If nonconservative forces do work then the total energy of the system changes by that amount \[\Delta E = W_\text{nc}.\]
\end{itemize}

\subsection{Force as the Gradient of Potential Energy}

\begin{itemize}
  \item A conservative force $\vec{F}$ can be expressed as the negative gradient of its potential energy function $U$ \[\vec{F} = -\nabla U.\]
\end{itemize}

\end{document}