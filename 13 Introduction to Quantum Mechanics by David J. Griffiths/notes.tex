\documentclass{article}
\usepackage{amsmath} % For align*
\usepackage{bookmark} % For links
\usepackage{float} % For the [H] option on figures
\usepackage{graphicx} % For images

\graphicspath{{./images/}}
\hypersetup{
  colorlinks=true,
  linkcolor=blue,
  urlcolor=blue
}

\title{Introduction to Quantum Mechanics by David J. Griffiths Notes}
\author{Chris Doble}
\date{March 2024}

\begin{document}

\maketitle

\tableofcontents

\part{Theory}

\section{The Wave Function}

\subsection{The Schrödinger Equation}

\begin{itemize}
  \item The \textbf{Schrödinger equation} \[i \hbar \frac{\partial \Psi}{\partial t} = -\frac{\hbar^2}{2 m} \frac{\partial^2 \Psi}{\partial x^2} + V \Psi\] is to quantum mechanics what Newton's second law is to classical mechanics. Given suitable initial conditions — typically $\Psi(x, 0)$ — the Schrödinger equation determines $\Psi(x, t)$ for all future time.
\end{itemize}

\subsection{The Statistical Interpretation}

\begin{itemize}
  \item The \textbf{Born rule} states that $|\Psi(x, t)|^2$ gives the probability of finding the particle at point $x$ at time $t$ or \[\int_a^b |\Psi(x, t)|^2 \,d x\] gives the probability of finding the particle between $a$ and $b$ at time $t$.

  \item This statistical interpretation introduces indeterminacy to quantum mechanics — we can't predict with certainty the particle's position.

  \item Suppose we measure a particle's position to be $C$. Where was it before we took the measurement? In the past there were three main schools of thought:

        \begin{enumerate}
          \item The \textbf{realist} position believes that the particle was at $C$ but $\Psi$ doesn't give us enough information to determine that — there's another \textbf{hidden variable} that would allow us to.

          \item The \textbf{orthodox} position (also known as the \textbf{Copenhagen interpretation}) believes that the particle didn't have a definite position but the act of measuring it forced it to do so.

          \item The \textbf{agnostic} position believes that it doesn't matter and is potentially unknowable.
        \end{enumerate}

  \item \textbf{Bell's theorem} confirms the orthodox interpretation.

  \item If we take two consecutive measurements of a particle, they will both yield the same result. The first measurement causes the wavefunction to \textbf{collapse} such that it is peaked only at the particle's measured location. If the system is allowed to evolve between the measurements the wavefunction will ``spread out'' but if done in quick succession the result won't change.
\end{itemize}

\end{document}