\documentclass{article}
\usepackage{amsmath} % For align*
\usepackage{amsfonts} % For mathbb
\usepackage{bookmark} % For links
\usepackage{braket} % For bra-ket notation
\usepackage{enumitem} % For customisable list labels
\usepackage{siunitx} % For units

\hypersetup{
  colorlinks=true,
  linkcolor=blue,
  urlcolor=blue
}

\DeclareMathOperator{\erf}{erf}
\newcommand{\ev}[1]{\left< #1 \right>}
\renewcommand{\Im}{\operatorname{Im}}
\renewcommand{\Re}{\operatorname{Re}}
\renewcommand{\vec}[1]{\boldsymbol{\mathbf{#1}}}
\newcommand{\uvec}[1]{\hat{\vec{#1}}}

\setlist[enumerate, 1]{label={(\alph*)}}
\setlist[enumerate, 2]{label={(\roman*)}}

\title{Introduction to Quantum Mechanics by David J. Griffiths Problems}
\author{Chris Doble}
\date{March 2023}

\begin{document}

\maketitle

\tableofcontents

\part{Theory}

\section{The Wave Function}

\subsection{}

\begin{enumerate}
  \item

        \begin{align*}
          \langle j^2 \rangle & = \sum j^2 P(j)                                                                                                         \\
                              & = 14^2 \frac{1}{14} + 15^2 \frac{1}{14} + 16^2 \frac{3}{14} + 22^2 \frac{2}{14} + 24^2 \frac{2}{14} + 25^2 \frac{5}{14} \\
                              & = \frac{3217}{7}                                                                                                        \\
                              & \approx 459.571                                                                                                         \\
          \langle j \rangle^2 & = \left( \sum j P(j) \right)^2                                                                                          \\
                              & = 441
        \end{align*}

  \item

        \begin{align*}
          \Delta j_{14} & = -7                     \\
          \Delta j_{15} & = -6                     \\
          \Delta j_{16} & = -5                     \\
          \Delta j_{22} & = 1                      \\
          \Delta j_{24} & = 3                      \\
          \Delta j_{25} & = 4                      \\
          \sigma^2      & = \sum (\Delta j)^2 P(j) \\
                        & = \frac{130}{7}          \\
                        & \approx 18.571
        \end{align*}

  \item \[\sigma^2 = \sqrt{\langle j^2 \rangle - \langle j \rangle^2} = 18.571\]
\end{enumerate}

\subsection{}

\begin{enumerate}
  \item

        \begin{align*}
          \langle x^2 \rangle & = \int_0^h x^2 \rho(x) \,d x                                    \\
                              & = \int_0^h \frac{x^{3 / 2}}{2 \sqrt{h}} \,d x                   \\
                              & = \frac{1}{2 \sqrt{h}} \left[ \frac{2}{5} x^{5 / 2} \right]_0^h \\
                              & = \frac{h^2}{5}                                                 \\
          \langle x \rangle^2 & = \frac{h^2}{9}                                                 \\
          \sigma              & = \sqrt{\langle x^2 \rangle - \langle x \rangle^2}              \\
                              & = \sqrt{\frac{h^2}{5} - \frac{h^2}{9}}                          \\
                              & = h \sqrt{\frac{4}{45}}                                         \\
                              & = \frac{2}{3 \sqrt{5}} h
        \end{align*}

  \item

        \begin{align*}
          1 - \int_{\langle x \rangle - \sigma}^{\langle x \rangle + \sigma} \rho(x) \,d x & = 1 - \frac{1}{2 \sqrt{h}} [2 \sqrt{x}]_{\langle x \rangle - \sigma}^{\langle x \rangle + \sigma}                                     \\
                                                                                           & = 1 - \frac{1}{\sqrt{h}} \left( \sqrt{\frac{1}{3} h + \frac{2}{3 \sqrt{5}} h} - \sqrt{\frac{1}{3} h - \frac{2}{3 \sqrt{5}} h} \right) \\
                                                                                           & = 1 - \left( \sqrt{\frac{1}{3} + \frac{2}{3 \sqrt{5}}} - \sqrt{\frac{1}{3} - \frac{2}{3 \sqrt{5}}} \right)                            \\
                                                                                           & \approx 0.393
        \end{align*}
\end{enumerate}

\subsection{}

\begin{enumerate}
  \item

        \begin{align*}
          \rho(x) & = A e^{-\lambda (x - a)^2}                             \\
          1       & = \int_{-\infty}^\infty \rho(x) \,d x                  \\
                  & = A \int_{-\infty}^\infty e^{-\lambda (x - a)^2} \,d x \\
                  & = A \sqrt{\frac{\pi}{\lambda}}                         \\
          A       & = \sqrt{\frac{\lambda}{\pi}}
        \end{align*}

  \item

        \begin{align*}
          \langle x \rangle   & = \sqrt{\frac{\lambda}{\pi}} \int_{-\infty}^\infty x e^{-\lambda (x - a)^2} \,d x   \\
                              & = a                                                                                 \\
          \langle x^2 \rangle & = \sqrt{\frac{\lambda}{\pi}} \int_{-\infty}^\infty x^2 e^{-\lambda (x - a)^2} \,d x \\
                              & = a^2 + \frac{1}{2 \lambda}                                                         \\
          \sigma              & = \sqrt{\langle x^2 \rangle - \langle x \rangle^2}                                  \\
                              & = \sqrt{a^2 + \frac{1}{2 \lambda} - a^2}                                            \\
                              & = \frac{1}{\sqrt{2 \lambda}}
        \end{align*}
\end{enumerate}

\subsection{}

\begin{enumerate}
  \item

        \begin{align*}
          1 & = \int_{-\infty}^\infty |\Psi(x, 0)|^2 \,d x                                                                  \\
            & = \left( \frac{A}{a} \right)^2 \int_0^a x^2 \,d x + \left( \frac{A}{b - a} \right)^2 \int_a^b (b - x)^2 \,d x \\
            & = \frac{1}{3} A^2 a + \left( \frac{A}{b - a} \right)^2 \left[ -\frac{1}{3} (b - x)^3 \right]_a^b              \\
            & = \frac{1}{3} A^2 a + \frac{1}{3} A^2 (b - a)                                                                 \\
            & = \frac{1}{3} A^2 b                                                                                           \\
          A & = \sqrt{\frac{3}{b}}
        \end{align*}

        \setcounter{enumi}{2}
  \item $x = a$

  \item

        \begin{align*}
          \int_0^a |\Psi(x, 0)|^2 \,d x & = \frac{3}{a^2 b} \left[ \frac{1}{3} x^3 \right]_0^a \\
                                        & = \frac{a}{b}
        \end{align*}

  \item

        \begin{align*}
          \langle x \rangle & = \int_{-\infty}^\infty x |\Psi(x, 0)|^2 \,d x                                                                                                                             \\
                            & = \frac{3}{a^2 b} \left[ \frac{1}{4} x^4 \right]_0^a + \frac{3}{b (b - a)^2} \int_a^b x (b - x)^2 \,d x                                                                    \\
                            & = \frac{3 a^2}{4 b} + \frac{3}{b (b - a)^2} \int_a^b (b^2 x - 2 b x^2 + x^3) \,d x                                                                                         \\
                            & = \frac{3 a^2}{4 b} + \frac{3}{b (b - a)^2} \left[ \frac{1}{2} b^2 x^2 - \frac{2}{3} b x^3 + \frac{1}{4} x^4 \right]_a^b                                                   \\
                            & = \frac{3 a^2}{4 b} + \frac{3}{b (b - a)^2} \left( \frac{1}{2} b^4 - \frac{2}{3} b^4 + \frac{1}{4} b^4 - \frac{1}{2} a^2 b^2 + \frac{2}{3} a^3 b - \frac{1}{4} a^4 \right) \\
                            & = \frac{3 a^2}{4 b} + \frac{3}{b (b - a)^2} \frac{1}{12} (b - a)^3 (3 a + b)                                                                                               \\
                            & = \frac{3 a^2}{4 b} + \frac{1}{4 b} (3 a b + b^2 - 3 a^2 - a b)                                                                                                            \\
                            & = \frac{1}{2} a + \frac{1}{4} b
        \end{align*}
\end{enumerate}

\subsection{}

\begin{enumerate}
  \item

        \begin{align*}
          \Psi(x, t) & = A e^{-\lambda |x|} e^{-i \omega t}                                  \\
          \Psi(x, 0) & = A e^{-\lambda |x|}                                                  \\
          1          & = A^2 \int_{-\infty}^\infty e^{-2 \lambda |x|} \,d x                  \\
                     & = 2 A^2 \int_0^\infty e^{-2 \lambda x} \,d x                          \\
                     & = 2 A^2 \left[ -\frac{1}{2 \lambda} e^{-2 \lambda x} \right]_0^\infty \\
                     & = \frac{A^2}{\lambda}                                                 \\
          A          & = \sqrt{\lambda}
        \end{align*}

  \item

        \begin{align*}
          \langle x \rangle   & = \int_{-\infty}^\infty x \lambda e^{-2 \lambda |x|} \,d x   \\
                              & = \lambda \int_{-\infty}^\infty x e^{-2 \lambda |x|} \,d x   \\
                              & = 0                                                          \\
          \langle x^2 \rangle & = \int_{-\infty}^\infty x^2 \lambda e^{-2 \lambda |x|} \,d x \\
                              & = 2 \lambda \int_0^\infty x^2 e^{-2 \lambda x} \,d x         \\
                              & = \frac{1}{2 \lambda^2}
        \end{align*}

  \item

        \begin{align*}
          \sigma                                                    & = \sqrt{\langle x^2 \rangle - \langle x \rangle^2}                            \\
                                                                    & = \frac{1}{\sqrt{2} \lambda}                                                  \\
          1 - \int_{-\sigma}^\sigma \lambda e^{-2 \lambda |x|} \,dx & = 1 - 2 \lambda \int_0^\sigma e^{-2 \lambda x} \,d x                          \\
                                                                    & = 1 - 2 \lambda \left[ -\frac{1}{2 \lambda} e^{-2 \lambda x} \right]_0^\sigma \\
                                                                    & = e^{-2 \lambda \sigma}                                                       \\
                                                                    & = e^{-\sqrt{2}}                                                               \\
                                                                    & \approx 0.243
        \end{align*}
\end{enumerate}

\subsection{}

The chain rule requires that you apply it to both $x$ and $|\Psi|^2$ which gives the same result

\begin{align*}
  \frac{d \ev{x}}{d t} & = \frac{d}{d t} \int x |\Psi|^2 \,d x                                                 \\
                       & = \int \frac{d}{d t} (x |\Psi|^2) \,d x                                               \\
                       & = \int \left( 0 \cdot |\Psi|^2 + x \frac{\partial |\Psi|^2}{\partial t} \right) \,d x \\
                       & = \int x \frac{\partial |\Psi|^2}{\partial t} \,d x
\end{align*}

\setcounter{subsection}{7}
\subsection{}

\begin{align*}
  i \hbar \frac{\partial}{\partial t} \left( e^{-i V_0 t / \hbar} \Psi \right)                                                  & = -\frac{\hbar^2}{2 m} \frac{\partial^2}{\partial x^2} \left( e^{-i V_0 t / \hbar} \Psi \right) + (V + V_0) \left( e^{-i V_0 t / \hbar} \Psi \right) \\
  i \hbar \left( -\frac{i V_0}{\hbar} e^{-i V_0 t / \hbar} \Psi + e^{-i V_0 t / \hbar} \frac{\partial \Psi}{\partial t} \right) & = -\frac{\hbar^2}{2 m} e^{-i V_0 t / \hbar} \frac{\partial^2 \Psi}{\partial x^2} + V e^{-i V_0 t / \hbar} \Psi + V_0 e^{-i V_0 t / \hbar} \Psi       \\
  V_0 \Psi + i \hbar \frac{\partial \Psi}{\partial t}                                                                           & = -\frac{\hbar^2}{2 m} \frac{\partial^2 \Psi}{\partial x^2} + V \Psi + V_0 \Psi                                                                      \\
  i \hbar \frac{\partial \Psi}{\partial t}                                                                                      & = -\frac{\hbar^2}{2 m} \frac{\partial^2 \Psi}{\partial x^2} + V \Psi                                                                                 \\
\end{align*}

\begin{align*}
  \ev{Q(x, p)} & = \int \left( e^{-i V_0 t / \hbar} \Psi \right)^* \left[ Q(x, -i \hbar \partial / \partial x) \right] e^{-i V_0 t / \hbar} \Psi \,d x \\
               & = \int e^{i V_0 t / \hbar} \Psi^* \left[ Q(x, -i \hbar \partial / \partial x) \right] e^{-i V_0 t / \hbar} \Psi \,d x                 \\
               & = \int \Psi^* [Q(x, -i \hbar \partial / \partial x)] \Psi \,d x
\end{align*}

No effect on the expectation value.

\subsection{}

\begin{enumerate}
  \item

        \begin{align*}
          \Psi(x, t) & = A e^{-a [(m x^2 / \hbar) + i t]}                         \\
          1          & = A^2 \int_{-\infty}^\infty e^{-2 a (m x^2 / \hbar)} \,d x \\
                     & = A^2 \int_{-\infty}^\infty e^{-2 a (m x^2 / \hbar)} \,d x \\
                     & = A^2 \sqrt{\frac{\pi \hbar}{2 a m}}                       \\
          A^2        & = \sqrt{\frac{2 a m}{\pi \hbar}}                           \\
          A          & = \left( \frac{2 a m}{\pi \hbar} \right)^{1 / 4}
        \end{align*}

  \item

        \begin{align*}
          \Psi                                 & = A e^{-a [(m x^2 / \hbar) + i t]}                                                                    \\
          \frac{\partial \Psi}{\partial t}     & = -i a \Psi                                                                                           \\
          \frac{\partial \Psi}{\partial x}     & = -\frac{2 a m x}{\hbar} \Psi                                                                         \\
          \frac{\partial^2 \Psi}{\partial x^2} & = -\frac{2 a m}{\hbar} \left( \Psi + x \frac{\partial \Psi}{\partial x} \right)                       \\
                                               & = -\frac{2 a m}{\hbar} \left( 1 - \frac{2 a m x^2}{\hbar} \right) \Psi                                \\
          V \Psi                               & = i \hbar \frac{\partial \Psi}{\partial t} + \frac{\hbar^2}{2 m} \frac{\partial^2 \Psi}{\partial x^2} \\
                                               & = a \hbar \Psi - a \hbar \left( 1 - \frac{2 a m x^2}{\hbar} \right) \Psi                              \\
          V                                    & = a \hbar - a \hbar + 2 a^2 m x^2                                                                     \\
                                               & = 2 a^2 m x^2
        \end{align*}

  \item

        \begin{align*}
          \ev{x}   & = A^2 \int_{-\infty}^\infty e^{-2 a (m x^2 / \hbar)} x \,d x                                                                                                                            \\
                   & = 0                                                                                                                                                                                     \\
          \ev{x^2} & = A^2 \int_{-\infty}^\infty e^{-2 a (m x^2 / \hbar)} x^2 \,d x                                                                                                                          \\
                   & = 2 A^2 \int_0^\infty e^{-2 a (m x^2 / \hbar)} x^2 \,d x                                                                                                                                \\
                   & = \frac{\hbar}{4 a m}                                                                                                                                                                   \\
          \ev{p}   & = \int_{-\infty}^\infty \Psi^* \left[ -i \hbar \frac{\partial}{\partial x} \right] \Psi \,d x                                                                                           \\
                   & = -i \hbar \int_{-\infty}^\infty A e^{-a [(m x^2 / \hbar) - i t]} \left( -\frac{2 a m x}{\hbar} A e^{-a [(m x^2 / \hbar) + i t]} \right) \,d x                                          \\
                   & = 2 i A^2 a m \int_{-\infty}^\infty x e^{-2 a m x^2 / \hbar} \,d x                                                                                                                      \\
                   & = 0                                                                                                                                                                                     \\
          \ev{p^2} & = \int_{-\infty}^\infty \Psi^* \left[ -\hbar^2 \frac{\partial^2}{\partial x^2} \right] \Psi \,d x                                                                                       \\
                   & = -\hbar^2 \int_{-\infty}^\infty A e^{-a [(m x^2 / \hbar) - i t]} \left[ -\frac{2 a m}{\hbar} \left( 1 - \frac{2 a m x^2}{\hbar} \right) A e^{-a [(m x^2 / \hbar) + i t]} \right] \,d x \\
                   & = 2 A^2 a m \hbar \int_{-\infty}^\infty e^{-2 a m x^2 / \hbar} \left( 1 - \frac{2 a m x^2}{\hbar} \right) \,d x                                                                         \\
                   & = a m \hbar
        \end{align*}

  \item

        \begin{align*}
          \sigma_x          & = \sqrt{\ev{x^2} - \ev{x}^2} \\
                            & = \sqrt{\frac{\hbar}{4 a m}} \\
          \sigma_p          & = \sqrt{a m \hbar}           \\
          \sigma_x \sigma_p & = \sqrt{\frac{1}{4} \hbar^2} \\
                            & = \frac{1}{2} \hbar          \\
                            & \ge \frac{1}{2} \hbar
        \end{align*}

        Yes, this is consistent with Heisenberg's uncertainty principle.
\end{enumerate}

\subsection{}

\begin{enumerate}
  \item

        \begin{align*}
          P(0) & = 0            \\
          P(1) & = \frac{2}{25} \\
               & = 0.08         \\
          P(2) & = \frac{3}{25} \\
               & = 0.12         \\
          P(3) & = \frac{1}{5}  \\
               & = 0.2          \\
          P(4) & = \frac{3}{25} \\
               & = 0.12         \\
          P(5) & = \frac{3}{25} \\
               & = 0.2          \\
          P(6) & = \frac{3}{25} \\
               & = 0.2          \\
          P(7) & = \frac{1}{25} \\
               & = 0.04         \\
          P(8) & = \frac{2}{25} \\
               & = 0.08         \\
          P(9) & = \frac{3}{25} \\
               & = 0.12
        \end{align*}

  \item The most probable digit is $3$, the median digit is $4$, and the average value is $\frac{118}{25} = 4.72$.

  \item $\sigma = 2.474$
\end{enumerate}

\setcounter{subsection}{13}
\subsection{}

\begin{enumerate}
  \item

        \begin{align*}
          P_{a b}(t)            & = \int_a^b |\Psi(x, t)|^2 \,d x                                                                                                                                                  \\
          \frac{d P_{a b}}{d t} & = \frac{d}{d t} \int_a^b |\Psi(x, t)|^2 \,d x                                                                                                                                    \\
                                & = \int_a^b \frac{d}{d t} \left( |\Psi(x, t)|^2 \right) \,d x                                                                                                                     \\
                                & = \int_a^b \frac{\partial}{\partial x} \left[ \frac{i \hbar}{2 m} \left( \Psi^* \frac{\partial \Psi}{\partial x} - \frac{\partial \Psi^*}{\partial x} \Psi \right) \right] \,d x \\
                                & = J(a, t) - J(b, t)
        \end{align*}

        The units are $\unit{s^{-1}}$.

  \item

        \begin{align*}
          \Psi(x, t)                         & = A e^{-a [(m x^2 / \hbar) + i t]}                                                                                                         \\
          \frac{\partial \Psi}{\partial x}   & = -\frac{2 a m x}{\hbar} \Psi                                                                                                              \\
          \Psi^*(x, t)                       & = A e^{-a [(m x^2 / \hbar) - i t]}                                                                                                         \\
          \frac{\partial \Psi^*}{\partial x} & = -\frac{2 a m x}{\hbar} \Psi^*                                                                                                            \\
          J(x, t)                            & = \frac{i \hbar}{2 m} \left( \Psi \frac{\partial \Psi^*}{\partial x} - \Psi^* \frac{\partial \Psi}{\partial x} \right)                     \\
                                             & = \frac{i \hbar}{2 m} \left[ \Psi \left( -\frac{2 a m x}{\hbar} \Psi^* \right) - \Psi^* \left( -\frac{2 a m x}{\hbar} \Psi \right) \right] \\
                                             & = 0
        \end{align*}
\end{enumerate}

\subsection{}

\begin{align*}
  \frac{d}{d t} \int_{-\infty}^\infty \Psi_1^* \Psi_2 \,d x & = \int_{-\infty}^\infty \left( \frac{\partial \Psi_1^*}{\partial t} \Psi_2 + \Psi_1^* \frac{\partial \Psi_2}{\partial t} \right) \,d x                                                       \\
                                                            & = \int_{-\infty}^\infty \left[ \left( -i \frac{\hbar}{2 m} \frac{\partial^2 \Psi_1^*}{\partial x^2} + i \frac{V}{\hbar} \Psi_1^* \right) \Psi_2 \right.                                      \\
                                                            & \qquad \left. + \Psi_1^* \left( i \frac{\hbar}{2 m} \frac{\partial^2 \Psi_2}{\partial x^2} - i \frac{V}{\hbar} \Psi_2 \right) \right] \,d x                                                  \\
                                                            & = i \frac{\hbar}{2 m} \int_{-\infty}^\infty \left( \Psi_1^* \frac{\partial^2 \Psi_2}{\partial x^2} - \frac{\partial^2 \Psi_1^*}{\partial x^2} \Psi_2 \right) \,d x                           \\
                                                            & = i \frac{\hbar}{2 m} \left[ \left. \Psi_1^* \frac{\partial \Psi_2}{\partial x} \right|_{-\infty}^\infty - \int_{-\infty}^\infty \frac{\partial}{\partial x} (\Psi_1^* \Psi_2) \,d x \right. \\
                                                            & \qquad \left. \left. \frac{\partial \Psi_1^*}{\partial x} \Psi_2 \right|_{-\infty}^\infty - \int_{-\infty}^\infty \frac{\partial}{\partial x} (\Psi_1^* \Psi_2) \,d x \right]                \\
                                                            & = 0
\end{align*}

\subsection{}

\begin{enumerate}
  \item

        \begin{align*}
          1 & = \int_{-a}^a A^2 (a^2 - x^2)^2 \,d x \\
            & = A^2 \int_0^a (a^2 - x^2)^2 \,d x    \\
            & = \frac{16}{15} A^2 a^5               \\
          A & = \sqrt{\frac{15}{16 a^5}}
        \end{align*}

  \item

        \begin{align*}
          \ev{x} & = \int_{-a}^a x A (a^2 - x^2) \,d x \\
                 & = 0
        \end{align*}

  \item

        \begin{align*}
          \ev{p} & = \int_{-a}^a \Psi^* \left( -i \hbar \frac{\partial}{\partial x} \right) \Psi \,d x \\
                 & = 2 i A^2 \hbar \int_{-a}^a x (a^2 - x^2) \,d x                                     \\
                 & = 0
        \end{align*}

  \item

        \begin{align*}
          \ev{x^2} & = \int_{-a}^a \Psi^* x^2 \Psi \,d x       \\
                   & = A^2 \int_{-a}^a x^2 (a^2 - x^2)^2 \,d x \\
                   & = A^2 \frac{16}{105} a^7                  \\
                   & = \frac{a^2}{7}
        \end{align*}

  \item

        \begin{align*}
          \ev{p^2} & = \int_{-a}^a \Psi^* \left( -\hbar^2 \frac{\partial^2}{\partial x^2} \right) \Psi \,d x \\
                   & = -\hbar^2 \int_{-a}^a A (a^2 - x^2) (-2 A) \,d x                                       \\
                   & = 4 A^2 \hbar^2 \int_0^a (a^2 - x^2) \,d x                                              \\
                   & = 4 A^2 \hbar^2 \left[ a^2 x - \frac{1}{3} x^3 \right]_0^a                              \\
                   & = 4 A^2 \hbar^2 \left( a^3 - \frac{1}{3} a^3 \right)                                    \\
                   & = \frac{8}{3} A^2 a^3 \hbar^2                                                           \\
                   & = \frac{8}{3} \frac{15}{16 a^5} a^3 \hbar^2                                             \\
                   & = \frac{5}{2} \frac{\hbar^2}{a^2}
        \end{align*}

  \item

        \begin{align*}
          \sigma_x & = \sqrt{\ev{x^2} - \ev{x}^2} \\
                   & = \sqrt{\frac{a^2}{7}}       \\
                   & = \frac{a}{\sqrt{7}}
        \end{align*}

  \item

        \begin{align*}
          \sigma_p & = \sqrt{\ev{p^2} - \ev{p}^2}         \\
                   & = \sqrt{\frac{5}{2}} \frac{\hbar}{a}
        \end{align*}

  \item

        \begin{align*}
          \sigma_x \sigma_p & = \sqrt{\frac{5}{14}} \hbar \\
                            & \ge \frac{1}{2} \hbar
        \end{align*}
\end{enumerate}

\setcounter{subsection}{17}
\subsection{}

\begin{enumerate}
  \item

        \begin{align*}
          % Electron
          \frac{h}{\sqrt{3 m k_B T}} & > d                       \\
          \frac{\sqrt{3 m k_B T}}{h} & < \frac{1}{d}             \\
          T_\text{electron}          & < \frac{h^2}{3 d^2 m k_B} \\
                                     & < \qty{1.3e5}{K}          \\
          % Nuclei
          T_\text{nuclei}            & < \qty{2.5}{K}
        \end{align*}

  \item

        \begin{align*}
          P V                        & = N k_B T                                                        \\
          \frac{V}{N}                & = \frac{k_B T}{P}                                                \\
          d                          & = \left( \frac{k_B T}{P} \right)^{1 / 3}                         \\
          \frac{h}{\sqrt{3 m k_B t}} & > \left( \frac{k_B T}{P} \right)^{1 / 3}                         \\
          T                          & < \frac{1}{k_B} \left( \frac{h^2}{3 m} \right)^{3 / 5} P^{2 / 5}
        \end{align*}
\end{enumerate}

\section{Time-Independent Schrödinger Equation}

\subsection{}

\begin{enumerate}
  \item

        \begin{align*}
          \int_{-\infty}^\infty |\Psi|^2 \,d x & = \int_{-\infty}^\infty \Psi^* \Psi \,d x                                                                    \\
                                               & = \int_{-\infty}^\infty \psi^* e^{i (E_0 - i \Gamma) t / \hbar} \psi e^{-i (E_0 + i \Gamma) t / \hbar} \,d x \\
                                               & = e^{2 \Gamma t / \hbar} \int_{-\infty}^\infty |\psi|^2 \,d x
        \end{align*}

        In order for this to equal $1$ for all $t$, $\Gamma$ must be $0$.

  \item If $\psi(x)$ is a complex solution to the time-independent Schrödinger equation then so is $\psi^*(x)$ and $\psi(x) + \psi^*(x)$ which is real.
\end{enumerate}

\subsection{}

If $\psi$ and its second derivative always have the same sign, $\psi$ will increase or decrease without bound forever. This means there is no non-zero choice of constant $A$ such that \[\int_{-\infty}^\infty |A \Psi|^2 \,d x = 1\] and thus the equation can't be normalised.

The classical analog of this is statements is that the potential energy of a system can't exceed its total energy.

\subsection{}

The time-independent Schrödinger equation in an infinite square well is \[-\frac{\hbar^2}{2 m} \frac{d^2 \psi}{d x^2} = E \psi.\]

If $E = 0$ then $\psi = A x + B$ which isn't normalisable.

If $E < 0$ then $\psi = A e^{k t} + B e^{-k t}$ where $k \in \mathbb{R}$ which also isn't normalisable.

\subsection{}

\begin{align*}
  \Psi_n(x, t) & = \sqrt{\frac{2}{a}} \sin \left( \frac{n \pi}{a} x \right) e^{-i (n^2 \pi^2 \hbar / 2 m a^2) t}                           \\
  \ev{x}       & = \int_0^a \Psi_n^* x \Psi_n \,d x                                                                                        \\
               & = \frac{2}{a} \int_0^a x \sin^2 \left( \frac{n \pi}{a} x \right) \,d x                                                    \\
               & = \frac{a}{2}                                                                                                             \\
  \ev{x^2}     & = \int_0^a \Psi_n^* x^2 \Psi_n \,d x                                                                                      \\
               & = \frac{2}{a} \int_0^a x^2 \sin^2 \left( \frac{n \pi}{a} x \right) \,d x                                                  \\
               & = a^2 \left( \frac{1}{3} - \frac{1}{2 n^2 \pi^2} \right)                                                                  \\
  \ev{p}       & = \int_0^a \Psi_n^* \left( -i \hbar \frac{\partial}{\partial x} \right) \Psi_n \,d x                                      \\
               & = -i \frac{2 \hbar n \pi}{a^2} \int_0^a \sin \left( \frac{n \pi}{a} x \right) \cos \left( \frac{n \pi}{a} x \right) \,d x \\
               & = 0                                                                                                                       \\
  \ev{p^2}     & = \int_0^a \Psi_n^* \left( -\hbar^2 \frac{\partial^2}{\partial x^2} \right) \Psi_n \,d x                                  \\
               & = \frac{2 \hbar^2 n^2 \pi^2}{a^3} \int_0^a \sin^2 \left( \frac{n \pi}{a} x \right) \,d x                                  \\
               & = \left( \frac{n \pi \hbar}{a} \right)^2                                                                                  \\
  \sigma_x     & = \sqrt{\ev{x^2} - \ev{x}^2}                                                                                              \\
               & = \frac{a}{2} \sqrt{\frac{1}{3} - \frac{2}{n^2 \pi^2}}                                                                    \\
  \sigma_p     & = \sqrt{\ev{p^2} - \ev{p}^2}                                                                                              \\
               & = \frac{n \pi \hbar}{a}
\end{align*}

\subsection{}

\begin{enumerate}
  \item

        \begin{align*}
          1 & = \int_0^a A^2 (\psi_1 + \psi_2)^2 \,d x                                                                                                       \\
            & = A^2 \int_0^a (\psi_1^2 + 2 \psi_1 \psi_2 + \psi_2^2) \,d x                                                                                   \\
            & = \frac{2 A^2}{a} \left[ \int_0^a \sin^2 \left( \frac{\pi}{a} x \right) \,d x + \int_0^a \sin^2 \left( \frac{2 \pi}{a} x \right) \,d x \right] \\
            & = 2 A^2                                                                                                                                        \\
          A & = \frac{1}{\sqrt{2}}
        \end{align*}

  \item

        \begin{align*}
          \Psi(x, t)     & = \frac{1}{\sqrt{2}} \left[ \sqrt{\frac{2}{a}} \sin \left( \frac{\pi}{a} x \right) e^{-i \omega t} + \sqrt{\frac{2}{a}} \sin \left( \frac{2 \pi}{a} x \right) e^{-4 i \omega t} \right] \\
          |\Psi(x, t)|^2 & = \Psi^* \Psi                                                                                                                                                                           \\
                         & = \frac{1}{a} \left[ \sin \left( \frac{\pi}{a} x \right) e^{i \omega t} + \sin \left( \frac{2 \pi}{a} x \right) e^{4 i \omega t} \right]                                                \\
                         & \qquad \left[ \sin \left( \frac{\pi}{a} x \right) e^{-i \omega t} + \sin \left( \frac{2 \pi}{a} x \right) e^{-4 i \omega t} \right]                                                     \\
                         & = \frac{1}{a} \left[ \sin^2 \left( \frac{\pi}{a} x \right) + \sin \left( \frac{\pi}{a} x \right) \sin \left( \frac{2 \pi}{a} x \right) e^{-3 i \omega t} \right.                        \\
                         & \qquad \left. + \sin \left( \frac{\pi}{a} x \right) \sin \left( \frac{2 \pi}{a} x \right) e^{3 i \omega t} + \sin^2 \left( \frac{2 \pi}{a} x \right) \right]                            \\
                         & = \frac{1}{a} \left[ \sin^2 \left( \frac{\pi}{a} x \right) + \sin^2 \left( \frac{2 \pi}{a} x \right) \right.                                                                            \\
                         & \qquad \left. + 2 \sin \left( \frac{\pi}{a} x \right) \sin \left( \frac{2 \pi}{a} x \right) \cos (3 \omega t) \right]
        \end{align*}

  \item

        \begin{align*}
          \ev{x} & = \int_0^a \Psi^* x \Psi \,d x                                        \\
                 & = \int_0^a x |\Psi|^2 \,d x                                           \\
                 & = \frac{a}{2} \left[ 1 - \frac{32}{9 \pi^2} \cos (3 \omega t) \right]
        \end{align*}

  \item

        \begin{align*}
          \ev{p} & = m \frac{d \ev{x}}{d t}                          \\
                 & = \frac{16 a m \omega}{3 \pi^2} \sin (3 \omega t) \\
                 & = \frac{8 \hbar}{3 a} \sin (3 \omega t)
        \end{align*}

  \item You can get $E_1$ or $E_2$ and the probability of getting each is $1 / 2$.

        $H = \frac{1}{2} (E_1 + E_2)$ is the mean of the two possible energy values.
\end{enumerate}

\subsection{}

\begin{align*}
  \Psi(x, 0) & = A [\psi_1 + e^{i \phi} \psi_2]                                                                                                                                       \\
  1          & = \int_0^a |\Psi|^2 \,d x                                                                                                                                              \\
             & = \int_0^a \Psi^* \Psi \,d x                                                                                                                                           \\
             & = A^2 \int_0^a (\psi_1 + e^{-i \phi} \psi_2) (\psi_1 + e^{i \phi} \psi_2) \,d x                                                                                        \\
             & = A^2 \int_0^a (\psi_1^2 + e^{i \phi} \psi_1 \psi_2 + e^{-i \phi} \psi_1 \psi_2 + \psi_2^2) \,d x                                                                      \\
             & = \frac{2 A^2}{a} \int_0^a \left[ \sin^2 \left( \frac{\pi}{a} x \right) + e^{i \phi} \sin \left( \frac{\pi}{a} x \right) \sin \left( \frac{2 \pi}{a} x \right) \right. \\
             & \qquad \left. e^{-i \phi} \sin \left( \frac{\pi}{a} x \right) \sin \left( \frac{2 \pi}{a} x \right) + \sin^2 \left( \frac{2 \pi}{a} x \right) \right] \,d x            \\
             & = \frac{2 A^2}{a} \int_0^a \left[ \sin^2 \left( \frac{\pi}{a} x \right) + \sin \left( \frac{\pi}{a} x \right) \sin \left( \frac{2 \pi}{a} x \right) \cos \phi \right.  \\
             & \qquad \left. + \sin^2 \left( \frac{2 \pi}{a} x \right) \right] \,d x                                                                                                  \\
             & = 2 A^2                                                                                                                                                                \\
  A          & = \frac{1}{\sqrt{2}}                                                                                                                                                   \\
  \Psi(x, t) & = \frac{1}{\sqrt{a}} \left[ \sin \left( \frac{\pi}{a} x \right) e^{-i \omega t} + \sin \left( \frac{2 \pi}{a} x \right) e^{i (\phi - 4 \omega t)} \right]
\end{align*}

\begin{align*}
  |\Psi|^2 & = \Psi^* \Psi                                                                                                                                                            \\
           & = \frac{1}{a} \left[ \sin \left( \frac{\pi}{a} x \right) e^{i \omega t} + \sin \left( \frac{2 \pi}{a} x \right) e^{-i (\phi - 4 \omega t)} \right]                       \\
           & \qquad \left[ \sin \left( \frac{\pi}{a} x \right) e^{-i \omega t} + \sin \left( \frac{2 \pi}{a} x \right) e^{i (\phi - 4 \omega t)} \right]                              \\
           & = \frac{1}{a} \left[ \sin^2 \left( \frac{\pi}{a} x \right) + \sin \left( \frac{\pi}{a} x \right) \sin \left( \frac{2 \pi}{a} x \right) e^{i (\phi - 3 \omega t)} \right. \\
           & \qquad \left. \sin \left( \frac{\pi}{a} x \right) \sin \left( \frac{2 \pi}{a} x \right) e^{-i (\phi - 3 \omega t)} + \sin^2 \left( \frac{2 \pi}{a} x \right) \right]     \\
           & = \frac{1}{a} \left[ \sin^2 \left( \frac{\pi}{a} x \right) + \sin^2 \left( \frac{2 \pi}{a} x \right) \right.                                                             \\
           & \qquad \left. + 2 \sin \left( \frac{\pi}{a} x \right) \sin \left( \frac{2 \pi}{a} x \right) \cos (\phi - 3 \omega t) \right]                                             \\
  \ev{x}   & = \int_0^a \Psi^* x \Psi \,d x                                                                                                                                           \\
           & = \int_0^a x |\Psi|^2 \,d x                                                                                                                                              \\
           & = \frac{a}{2} \left[ 1 - \frac{32}{9 \pi^2} \cos (3 \omega t - \phi) \right]
\end{align*}

\subsection{}

\begin{enumerate}
  \item

        \begin{align*}
          1 & = \int_0^a |\Psi|^2 \,d x                                                                                         \\
            & = A^2 \left[ \int_0^{a / 2} x^2 \,d x + \int_{a / 2}^a (a - x)^2 \,d x \right]                                    \\
            & = A^2 \left\{ \frac{1}{3} \left[ \frac{a}{2} \right]^3 + \left[ -\frac{1}{3} (a - x)^3 \right]_{a / 2}^a \right\} \\
            & = A^2 \left( \frac{a^3}{24} + \frac{a^3}{24} \right)                                                              \\
            & = \frac{A^2 a^3}{12}                                                                                              \\
          A & = \frac{2 \sqrt{3}}{\sqrt{a^3}}
        \end{align*}

  \item

        \begin{align*}
          c_n        & = \sqrt{\frac{2}{a}} \int_0^a \sin \left( \frac{n \pi}{a} x \right) \Psi(x, 0) \,d x                                                                                                        \\
                     & = \sqrt{\frac{2}{a}} \left[ \int_0^{a / 2} \sin \left( \frac{n \pi}{a} x \right) A x \,d x + \int_{a / 2}^a \sin \left( \frac{n \pi}{a} x \right) A (a - x) \,d x \right]                   \\
                     & = \frac{2 \sqrt{6}}{a^2} \left[ \int_0^{a / 2} x \sin \left( \frac{n \pi}{a} x \right) \,d x + \int_{a / 2}^a (a - x) \sin \left( \frac{n \pi}{a} x \right) \,d x \right]                   \\
                     & = \frac{8 \sqrt{6}}{n^2 \pi^2} \sin^2 \left( \frac{n \pi}{4} \right) \sin \left( \frac{n \pi}{2} \right)                                                                                    \\
                     & = \begin{cases}
                           0                                               & n \text{ even} \\
                           (-1)^{(n - 1) / 2} \frac{4 \sqrt{6}}{n^2 \pi^2} & n \text{ odd}
                         \end{cases}                                                                                                           \\
          \Psi(x, t) & = \frac{4 \sqrt{6}}{\pi^2} \sqrt{\frac{2}{a}} \sum_{n = 1, 3, 5, \ldots}^\infty (-1)^{(n - 1) / 2} \frac{1}{n^2} \sin \left( \frac{n \pi}{a} x \right) e^{-i (n^2 \pi^2 \hbar / 2 m a^2) t} \\
        \end{align*}

  \item

        \begin{align*}
          |c_1|^2 & = \left( \frac{4 \sqrt{6}}{\pi^2} \right)^2 \\
                  & \approx 0.985
        \end{align*}

  \item

        \begin{align*}
          E_n    & = \frac{n^2 \pi^2 \hbar^2}{2 m a^2}                                                                                   \\
          \ev{H} & = \sum_{n = 0}^\infty |c_{2 n + 1}|^2 E_{2 n + 1}                                                                     \\
                 & = \sum_{n = 0}^\infty \left( \frac{4 \sqrt{6}}{(2 n + 1)^2 \pi^2} \right)^2 \frac{(2 n + 1)^2 \pi^2 \hbar^2}{2 m a^2} \\
                 & = \sum_{n = 0}^\infty \frac{48 \hbar^2}{(2 n + 1)^2 m a^2 \pi^2}                                                      \\
                 & = \frac{48 \hbar^2}{m a^2 \pi^2} \sum_{n = 0}^\infty \frac{1}{(2 n + 1)^2}                                            \\
                 & = \frac{6 \hbar^2}{m a^2}
        \end{align*}
\end{enumerate}

\subsection{}

\begin{align*}
  1       & = \int_0^{a / 2} |\Psi|^2 \,d x                                          \\
          & = A^2 \int_0^{a / 2} \,d x                                               \\
          & = \frac{a A^2}{2}                                                        \\
  A       & = \sqrt{\frac{2}{a}}                                                     \\
  c_n     & = \frac{2}{a} \int_0^{a / 2} \sin \left( \frac{n \pi}{a} x \right) \,d x \\
  |c_1|^2 & = \left( \frac{2}{\pi} \right)^2                                         \\
          & \approx 0.405
\end{align*}

\subsection{}

\begin{align*}
  \Psi(x, 0) & = A x (a - x)                                                                                                \\
  \ev{H}     & = \int_0^a \Psi(x, 0)^* \hat{H} \Psi(x, 0) \,d x                                                             \\
             & = \int_0^a \Psi(x, 0)^* \left( -\frac{\hbar^2}{2 m} \frac{\partial^2}{\partial x^2} \right) \Psi(x, 0) \,d x \\
             & = \frac{A^2 \hbar^2}{m} \int_0^a x (a - x) \,d x                                                             \\
             & = \frac{30 \hbar^2}{m a^5} \frac{a^3}{6}                                                                     \\
             & = \frac{5 \hbar^2}{m a^2}
\end{align*}

\subsection{}

\begin{enumerate}
  \item

        \begin{align*}
          \psi_2(x) & = \frac{1}{\sqrt{2!}} (\hat{a}_+) \psi_1                                                                                                                                                                                                                            \\
                    & = \frac{1}{\sqrt{2}} \frac{1}{\sqrt{2 \hbar m \omega}} \left( -\hbar \frac{d}{d x} + m \omega x \right) \left( \frac{m \omega}{\pi \hbar} \right)^{1 / 4} \sqrt{\frac{2 m \omega}{\hbar}} x e^{-\frac{m \omega}{2 \hbar} x^2}                                       \\
                    & = \frac{1}{\sqrt{2} \hbar} \left( \frac{m \omega}{\pi \hbar} \right)^{1 / 4} \left( -\hbar \frac{d}{d x} + m \omega x \right) x e^{-\frac{m \omega}{2 \hbar} x^2}                                                                                                   \\
                    & = \frac{1}{\sqrt{2} \hbar} \left( \frac{m \omega}{\pi \hbar} \right)^{1 / 4} \left[ -\hbar \left( e^{-\frac{m \omega}{2 \hbar} x^2} - \frac{m \omega}{\hbar} x^2 e^{-\frac{m \omega}{2 \hbar} x^2} \right) + m \omega x^2 e^{-\frac{m \omega}{2 \hbar} x^2} \right] \\
                    & = \frac{1}{\sqrt{2}} \left( \frac{m \omega}{\pi \hbar} \right)^{1 / 4} \left( \frac{2 m \omega}{\hbar} x^2 - 1 \right) e^{-\frac{m \omega}{2 \hbar} x^2}
        \end{align*}
\end{enumerate}

\subsection{}

\begin{enumerate}
  \item

        % \psi_0
        \begin{align*}
          \ev{x}   & = \int_{-\infty}^\infty \psi_0^* x \psi_0 \,d x                                                                                                                                                                                                                  \\
                   & = \alpha^2 \int_{-\infty}^\infty x e^{-\frac{m \omega}{\hbar} x^2} \,d x                                                                                                                                                                                         \\
                   & = 0                                                                                                                                                                                                                                                              \\
          \ev{p}   & = m \frac{d \ev{x}}{d t}                                                                                                                                                                                                                                         \\
                   & = 0                                                                                                                                                                                                                                                              \\
          \ev{x^2} & = \int_{-\infty}^\infty \psi_0^* x^2 \psi_0 \,d x                                                                                                                                                                                                                \\
                   & = \alpha^2 \int_{-\infty}^\infty x^2 e^{-\frac{m \omega}{\hbar} x^2} \,d x                                                                                                                                                                                       \\
                   & = \frac{\hbar}{2 m \omega}                                                                                                                                                                                                                                       \\
          \ev{p^2} & = \int_{-\infty}^\infty \psi_0^* \left( -\hbar^2 \frac{d^2}{d x^2} \right) \psi_0 \,d x                                                                                                                                                                          \\
                   & = -\hbar^2 \left( \frac{m \omega}{\pi \hbar} \right)^{1 / 2} \int_{-\infty}^\infty e^{-\frac{m \omega}{2 \hbar} x^2} \frac{d}{d x} \left( -\frac{m \omega}{\hbar} x e^{-\frac{m \omega}{2 \hbar} x^2} \right) \,d x                                              \\
                   & = \hbar^2 \left( \frac{m \omega}{\pi \hbar} \right)^{1 / 2} \frac{m \omega}{\hbar} \int_{-\infty}^\infty e^{-\frac{m \omega}{2 \hbar} x^2} \left( e^{-\frac{m \omega}{2 \hbar} x^2} - \frac{m \omega}{\hbar} x^2 e^{-\frac{m \omega}{2 \hbar} x^2} \right) \,d x \\
                   & = \hbar^2 \left( \frac{m \omega}{\pi \hbar} \right)^{1 / 2} \frac{m \omega}{\hbar} \int_{-\infty}^\infty \left( 1 - \frac{m \omega}{\hbar} x^2 \right) e^{-\frac{m \omega}{\hbar} x^2} \,d x                                                                     \\
                   & = \hbar^2 \left( \frac{m \omega}{\pi \hbar} \right)^{1 / 2} \frac{m \omega}{\hbar} \frac{\hbar \sqrt{\pi}}{2 \sqrt{\hbar m \omega}}                                                                                                                              \\
                   & = \frac{1}{2} m \hbar \omega
        \end{align*}

        % \psi_1
        \begin{align*}
          \psi_1(x) & = \left( \frac{m \omega}{\pi \hbar} \right)^{1 / 4} \sqrt{\frac{2 m \omega}{\hbar}} x e^{-\frac{m \omega}{2 \hbar} x^2}                                                                                                                                                             \\
          \ev{x}    & = 0                                                                                                                                                                                                                                                                                 \\
          \ev{p}    & = m \frac{d \ev{x}}{d t}                                                                                                                                                                                                                                                            \\
                    & = 0                                                                                                                                                                                                                                                                                 \\
          \ev{x^2}  & = \int_{-\infty}^\infty \psi_1^* x^2 \psi_1 \,d x                                                                                                                                                                                                                                   \\
                    & = \left( \frac{m \omega}{\pi \hbar} \right)^{1 / 2} \frac{2 m \omega}{\hbar} \int_{-\infty}^\infty x^4 e^{-\frac{m \omega}{\hbar} x^2} \,d x                                                                                                                                        \\
                    & = \left( \frac{m \omega}{\pi \hbar} \right)^{1 / 2} \frac{2 m \omega}{\hbar} \frac{3}{4} \sqrt{\pi} \left( \frac{\hbar}{m \omega} \right)^{5 / 2}                                                                                                                                   \\
                    & = \frac{3}{2} \frac{\hbar}{m \omega}                                                                                                                                                                                                                                                \\
          \ev{p^2}  & = \int_{-\infty}^\infty \psi_1^* \left( -\hbar^2 \frac{d^2}{d x^2} \right) \psi_1 \,d x                                                                                                                                                                                             \\
                    & = -\hbar^2 \left( \frac{m \omega}{\pi \hbar} \right)^{1 / 2} \frac{2 m \omega}{\hbar} \int_{-\infty}^\infty x e^{-\frac{m \omega}{2 \hbar} x^2} \frac{d}{d x} \left( e^{-\frac{m \omega}{2 \hbar} x^2} - \frac{m \omega}{\hbar} x^2 e^{-\frac{m \omega}{2 \hbar} x^2} \right) \,d x \\
                    & = -\hbar^2 \left( \frac{m \omega}{\pi \hbar} \right)^{1 / 2} \frac{2 m \omega}{\hbar} \int_{-\infty}^\infty x e^{-\frac{m \omega}{2 \hbar} x^2} \left[ -\frac{m \omega}{\hbar} x e^{-\frac{m \omega}{2 \hbar} x^2} \right.                                                          \\
                    & \qquad \left. - \frac{2 m \omega}{\hbar} x e^{-\frac{m \omega}{2 \hbar} x^2} + \left( \frac{m \omega}{\hbar} \right)^2 x^3 e^{-\frac{m \omega}{2 \hbar} x^2} \right] \,d x                                                                                                          \\
                    & = 2 \hbar^2 \left( \frac{m \omega}{\pi \hbar} \right)^{1 / 2} \left( \frac{m \omega}{\hbar} \right)^2 \int_{-\infty}^\infty x^2 e^{-\frac{m \omega}{\hbar} x^2} \left( 3 - \frac{m \omega}{\hbar} x^2 \right) \,d x                                                                 \\
                    & = 2 \hbar^2 \left( \frac{m \omega}{\pi \hbar} \right)^{1 / 2} \left( \frac{m \omega}{\hbar} \right)^2 \frac{3}{4} \sqrt{\pi} \left( \frac{h}{m \omega} \right)^{3 / 2}                                                                                                              \\
                    & = \frac{3}{2} m \hbar \omega
        \end{align*}

  \item

        \begin{align*}
          % \psi_0
          \sigma_x          & = \sqrt{\ev{x^2} - \ev{x}^2}        \\
                            & = \sqrt{\frac{\hbar}{2 m \omega}}   \\
          \sigma_p          & = \sqrt{\ev{p^2} - \ev{p}^2}        \\
                            & = \sqrt{\frac{m \hbar \omega}{2}}   \\
          \sigma_x \sigma_p & = \frac{\hbar}{2}                   \\
          % \psi_1
          \sigma_x          & = \sqrt{\frac{3 \hbar}{2 m \omega}} \\
          \sigma_p          & = \sqrt{\frac{3 m \hbar \omega}{2}} \\
          \sigma_x \sigma_p & = \frac{3}{2} \hbar
        \end{align*}

  \item

        \begin{align*}
          % \psi_0
          \ev{T} & = \frac{\ev{p^2}}{2 m}            \\
                 & = \frac{\hbar \omega}{4}          \\
          \ev{V} & = \frac{1}{2} m \omega^2 \ev{x^2} \\
                 & = \frac{1}{4} \hbar \omega        \\
          % \psi_1
          \ev{T} & = \frac{\ev{p^2}}{2 m}            \\
                 & = \frac{3}{4} \hbar \omega        \\
          \ev{V} & = \frac{1}{2} m \omega^2 \ev{x^2} \\
                 & = \frac{3}{4} \hbar \omega
        \end{align*}
\end{enumerate}

\subsection{}

\begin{align*}
  \ev{x}   & = \int_{-\infty}^\infty \psi_n^* x \psi_n \,d x                                                                                                \\
           & = \sqrt{\frac{\hbar}{2 m \omega}} \int_{-\infty}^\infty \psi_n^* (\hat{a}_+ + \hat{a}_-) \psi_n \,d x                                          \\
           & = \sqrt{\frac{\hbar}{2 m \omega}} \int_{-\infty}^\infty \psi_n^* (\sqrt{n + 1} \psi_{n + 1} + \sqrt{n} \psi_{n - 1}) \,d x                     \\
           & = 0                                                                                                                                            \\
  \ev{p}   & = \int_{-\infty}^\infty \psi_n^* p \psi_n \,d x                                                                                                \\
           & = i \sqrt{\frac{\hbar m \omega}{2}} \int_{-\infty}^\infty \psi_n^* (\hat{a}_+ - \hat{a}_-) \psi_n \,d x                                        \\
           & = i \sqrt{\frac{\hbar m \omega}{2}} \int_{-\infty}^\infty \psi_n^* (\sqrt{n + 1} \psi_{n + 1} - \sqrt{n} \psi_{n - 1}) \,d x                   \\
           & = 0                                                                                                                                            \\
  \ev{x^2} & = \int_{-\infty}^\infty \psi_n^* x^2 \psi_n \,d x                                                                                              \\
           & = \frac{\hbar}{2 m \omega} \int_{-\infty}^\infty \psi_n^* (\hat{a}_+^2 + \hat{a}_+ \hat{a}_- + \hat{a}_- \hat{a}_+ + \hat{a}_-^2) \psi_n \,d x \\
           & = \frac{\hbar}{2 m \omega} (2 n + 1) \int_{-\infty}^\infty |\psi_n|^2 \,d x                                                                    \\
           & = \frac{\hbar}{m \omega} \left( n + \frac{1}{2} \right)                                                                                        \\
  \ev{p^2} & = \int_{-\infty}^\infty \psi_n^* p^2 \psi_n \,d x                                                                                              \\
           & = -\frac{\hbar m \omega}{2} \int_{-\infty}^\infty \psi_n^* (\hat{a}_+^2 - \hat{a}_+ \hat{a}_- - \hat{a}_- \hat{a}- + \hat{a}_-^2) \psi_n \,d x \\
           & = \frac{\hbar m \omega}{2} (2 n + 1) \int_{-\infty}^\infty |\psi_n|^2 \,d x                                                                    \\
           & = \hbar m \omega \left( n + \frac{1}{2} \right)                                                                                                \\
  \ev{T}   & = \ev{\frac{p^2}{2 m}}                                                                                                                         \\
           & = \frac{1}{2} \hbar \omega \left( n + \frac{1}{2} \right)
\end{align*}

\begin{align*}
  \sigma_x          & = \sqrt{\ev{x^2} - \ev{x}^2}                                   \\
                    & = \sqrt{\frac{\hbar}{m \omega} \left( n + \frac{1}{2} \right)} \\
  \sigma_p          & = \sqrt{\hbar m \omega \left( n + \frac{1}{2} \right)}         \\
  \sigma_x \sigma_p & = (2 n + 1) \frac{\hbar}{2}                                    \\
                    & \ge \frac{\hbar}{2}
\end{align*}

\subsection{}

\begin{enumerate}
  \item

        \begin{align*}
          \Psi(x, 0) & = A [3 \psi_0(x) + 4 \psi_1(x)]                                                             \\
          1          & = \int_{-\infty}^\infty |\Psi(x, 0)|^2 \,d x                                                \\
                     & = A^2 \int_{-\infty}^\infty [9 \psi_0(x)^2 + 24 \psi_0(x) \psi_1(x) + 16 \psi_1(x)^2] \,d x \\
                     & = 25 A^2                                                                                    \\
          A          & = \frac{1}{5}
        \end{align*}

  \item

        \begin{align*}
          \Psi(x, t)     & = \frac{1}{5} [3 \psi_0(x) e^{-i \omega t / 2} + 4 \psi_1(x) e^{-3 i \omega t / 2}]                                                                      \\
          |\Psi(x, t)|^2 & = \Psi(x, t)^* \Psi(x, t)                                                                                                                                \\
                         & = \frac{1}{25} [3 \psi_0(x) e^{i \omega t / 2} + 4 \psi_1(x) e^{3 i \omega t / 2}] [3 \psi_0(x) e^{-i \omega t / 2} + 4 \psi_1(x) e^{-3 i \omega t / 2}] \\
                         & = \frac{1}{25} [9 \psi_0(x)^2 + 12 \psi_0(x) \psi_1(x) e^{-i \omega t} + 12 \psi_0(x) \psi_1(x) e^{i \omega t} + 16 \psi_1(x)^2]                         \\
                         & = \frac{1}{25} [9 \psi_0(x)^2 + 16 \psi_1(x)^2 + 24 \psi_0(x) \psi_1(x) \cos \omega t]
        \end{align*}

  \item

        \begin{align*}
          \ev{x}                              & = \int_{-\infty}^\infty \Psi^* x \Psi \,d x                                                                                                                                            \\
                                              & = \frac{1}{25} \int_{-\infty}^\infty x (9 \psi_0^2 + 16 \psi_1^2 + 24 \psi_0 \psi_1 \cos \omega t) \,d x                                                                               \\
                                              & = \frac{24}{25} \int_{-\infty}^\infty x \psi_0 \psi_1 \cos (\omega t) \,d x                                                                                                            \\
                                              & = \frac{24}{25} \left( \frac{m \omega}{\pi \hbar} \right)^{1 / 2} \sqrt{\frac{2 m \omega}{\hbar}} \cos (\omega t) \int_{-\infty}^\infty x^2 e^{-\frac{m \omega}{\hbar} x^2} \,d x      \\
                                              & = \frac{24}{25} \left( \frac{m \omega}{\pi \hbar} \right)^{1 / 2} \sqrt{\frac{2 m \omega}{\hbar}} \cos (\omega t) \frac{1}{2} \sqrt{\pi} \left( \frac{\hbar}{m \omega} \right)^{3 / 2} \\
                                              & = \frac{24}{25} \sqrt{\frac{\hbar}{2 m \omega}} \cos (\omega t)                                                                                                                        \\
          \ev{p}                              & = m \frac{d \ev{x}}{d t}                                                                                                                                                               \\
                                              & = -\frac{24}{25} \sqrt{\frac{\hbar m \omega}{2}} \sin (\omega t)                                                                                                                       \\
          \frac{d \ev{p}}{d t}                & = -\frac{24}{25} \sqrt{\frac{\hbar m \omega}{2}} \omega \cos (\omega t)                                                                                                                \\
          V                                   & = \frac{1}{2} m \omega^2 x^2                                                                                                                                                           \\
          \frac{\partial V}{d \partial}       & = m \omega^2 x                                                                                                                                                                         \\
          \ev{-\frac{\partial V}{\partial x}} & = -m \omega^2 \ev{x}                                                                                                                                                                   \\
                                              & = -\frac{24}{25} \sqrt{\frac{\hbar m \omega}{2}} \omega \cos (\omega t)                                                                                                                \\
                                              & = \frac{d \ev{p}}{d t}
        \end{align*}

  \item

        \begin{align*}
          E_0    & = \frac{\hbar \omega}{2}   \\
          P(E_0) & = \frac{9}{25}             \\
          E_1    & = \frac{3 \hbar \omega}{2} \\
          P(E_1) & = \frac{16}{25}
        \end{align*}
\end{enumerate}

\subsection{}

\begin{align*}
  1 - \left( \frac{m \omega}{\pi \hbar} \right)^{1 / 2} \int_{-\sqrt{\hbar / m \omega}}^{\sqrt{\hbar / m \omega}} e^{-m \omega x^2 / \hbar} \,d x & = 1 - \left( \frac{m \omega}{\pi \hbar} \right)^{1 / 2} \sqrt{\frac{\pi \hbar}{m \omega}} \erf 1 \\
                                                                                                                                                  & = 0.157
\end{align*}

\subsection{}

\begin{align*}
  a_{j + 2} & = \frac{-2 (n - j)}{(j + 1) (j + 2)} a_j                          \\
  % H_5
  a_3       & = -\frac{4}{3} a_1                                                \\
  a_5       & = \frac{4}{15} a_1                                                \\
  H_5(\xi)  & = a_1 \left( \xi - \frac{4}{3} \xi^3 + \frac{4}{15} \xi^5 \right) \\
            & = \frac{1}{120} a_1 (120 \xi - 160 \xi^3 + 32 \xi^5)              \\
            & = 32 \xi^5 - 160 \xi^3 + 120 \xi                                  \\
  % H_6
  a_2       & = -6 a_0                                                          \\
  a_4       & = \frac{-8}{12} a_2                                               \\
            & = 4 a_0                                                           \\
  a_6       & = \frac{-4}{30} a_4                                               \\
            & = -\frac{8}{15} a_0                                               \\
  H_6(\xi)  & = a_0 \left( 1 - 6 \xi^2 + 4 \xi^4 - \frac{8}{15} \xi^6 \right)   \\
            & = \frac{1}{120} a_0 (120 - 720 \xi^2 + 480 \xi^4 - 64 \xi^6)      \\
            & = 64 \xi^6 - 480 \xi^4 + 720 \xi^2 - 120
\end{align*}

\setcounter{subsection}{16}
\subsection{}

\begin{align*}
  A e^{i k x} + B e^{-i k x} & = A [\cos (k x) + i \sin (k x)] + B [\cos (k x) - i \sin (k x)] \\
                             & = (A + B) \cos (k x) + i (A - B) \sin (k x)                     \\
  C                          & = A + B                                                         \\
  D                          & = i (A - B)                                                     \\
  -i D                       & = A - B                                                         \\
  A                          & = \frac{C - i D}{2}                                             \\
  B                          & = \frac{C + i D}{2}
\end{align*}

\subsection{}

\begin{align*}
  \Psi_k(x, t) & = A e^{i \left( k x - \frac{\hbar k^2}{2 m} t \right)}                                                                 \\
  J(x, t)      & = \frac{i \hbar}{2 m} \left( \Psi \frac{\partial \Psi^*}{\partial x} - \Psi^* \frac{\partial \Psi}{\partial x} \right) \\
               & = \frac{\hbar k |A|^2}{m}
\end{align*}

The probability travels in the same direction as the wave.

\setcounter{subsection}{19}
\subsection{}

\begin{enumerate}
  \item

        \begin{align*}
          \Psi(x, 0) & = A e^{-a |x|}                                   \\
          1          & = \int_{-\infty}^\infty \Psi^* \Psi \,d x        \\
                     & = |A|^2 \int_{-\infty}^\infty e^{-2 a |x|} \,d x \\
                     & = 2 |A|^2 \int_0^\infty e^{-2 a x} \,d x         \\
                     & = \frac{|A|^2}{a}                                \\
          A          & = \sqrt{a}
        \end{align*}

  \item

        \begin{align*}
          \phi(k) & = \sqrt{\frac{a}{2 \pi}} \int_{-\infty}^\infty e^{-a |x| - i k x} \,d x                     \\
                  & = \sqrt{\frac{a}{2 \pi}} \int_{-\infty}^\infty e^{-a |x|} [\cos (k x) - i \sin (k x)] \,d x \\
                  & = \sqrt{\frac{a}{2 \pi}} \int_{-\infty}^\infty e^{-a |x|} \cos (k x) \,d x                  \\
                  & = \sqrt{\frac{a}{2 \pi}} 2 \int_0^\infty e^{-a x} \cos (k x) \,d x                          \\
                  & = \sqrt{\frac{a}{2 \pi}} \frac{2 a}{a^2 + k^2}
        \end{align*}

  \item \[\Psi(x, t) = \frac{a^{3 / 2}}{\pi} \int_{-\infty}^\infty \frac{1}{a^2 + k^2} e^{i \left( k x - \frac{\hbar k^2}{2 m} t \right)} \,d k\]
\end{enumerate}

\subsection{}

\begin{enumerate}
  \item

        \begin{align*}
          \Psi(x, 0) & = A e^{-a x^2}                                  \\
          1          & = A^2 \int_{-\infty}^\infty  e^{-2 a x^2} \,d x \\
                     & = \sqrt{\frac{\pi}{2 a}} A^2                    \\
          A          & = \left( \frac{2 a}{\pi} \right)^{1 / 4}
        \end{align*}

  \item

        \begin{align*}
          \phi(k)    & = \frac{1}{\sqrt{2 \pi}} \left( \frac{2 a}{\pi} \right)^{1 / 4} \int_{-\infty}^\infty e^{-(a x^2 + i k x)} \,d x                                       \\
                     & = \frac{1}{(2 \pi a)^{1 / 4}} e^{-k^2 / 4 a}                                                                                                           \\
          \Psi(x, t) & = \frac{1}{\sqrt{2 \pi}} \frac{1}{(2 \pi a)^{1 / 4}} \int_{-\infty}^\infty e^{-\frac{k^2}{4 a} + i \left( k x - \frac{\hbar k^2}{2 m} t \right)} \,d k \\
          \Psi(x, t) & = \left( \frac{2 a}{\pi} \right)^{1 / 4} \frac{1}{\gamma} e^{-a x^2 / \gamma^2}
        \end{align*}

  \item

        \begin{align*}
          |\Psi(x, t)|^2 & = \Psi^* \Psi                                                                                                                \\
                         & = \left( \frac{2 a}{\pi} \right)^{1 / 2} \frac{1}{\gamma^*} e^{-a x^2 / (\gamma^*)^2} \frac{1}{\gamma} e^{-a x^2 / \gamma^2} \\
                         & = \left( \frac{2 a}{\pi} \right)^{1 / 2} \frac{1}{\sqrt{1 - 2 i \hbar a t / m}} e^{-a x^2 / (1 - 2 i \hbar a t / m)}         \\
                         & \qquad \frac{1}{\sqrt{1 + 2 i \hbar a t / m}} e^{-a x^2 / (1 + 2 i \hbar a t / m)}                                           \\
                         & = \left( \frac{2 a}{\pi} \right)^{1 / 2} \frac{1}{\sqrt{1 + (2 \hbar a t / m)^2}} e^{-2 a x^2 / [1 + (2 a \hbar t / m)^2]}   \\
                         & = \sqrt{\frac{2}{\pi}} w e^{-2 w^2 x^2}
        \end{align*}

        As $t$ increases $|\Psi|^2$ flattens out and broadens.

  \item

        \begin{align*}
          \ev{x}   & = \int_{-\infty}^\infty \Psi^* x \Psi \,d x                             \\
                   & = 0                                                                     \\
          \ev{p}   & = m \frac{d \ev{x}}{d t}                                                \\
                   & = 0                                                                     \\
          \ev{x^2} & = \int_{-\infty}^\infty \Psi^* x^2 \Psi \,d x                           \\
                   & = \sqrt{\frac{2}{\pi}} w \int_{-\infty}^\infty x^2 e^{-2 w^2 x^2} \,d x \\
                   & = \frac{1}{4 w^2}
        \end{align*}
\end{enumerate}

\subsection{}

\begin{enumerate}
  \item $-25$

  \item $1$

  \item $0$
\end{enumerate}

\setcounter{subsection}{25}
\subsection{}

\begin{align*}
  F(k) & = \frac{1}{\sqrt{2 \pi}} \int_{-\infty}^\infty \delta(x) e^{-i k x} \,d x \\
       & = \frac{1}{\sqrt{2 \pi}}                                                  \\
  f(x) & = \frac{1}{2 \pi} \int_{-\infty}^\infty e^{i k x} \,d k
\end{align*}

\setcounter{subsection}{28}
\subsection{}

\begin{align*}
  \psi(x)                 & = \begin{cases}
                                F e^{-\kappa x} & x > a     \\
                                C \sin (l x)    & 0 < x < a \\
                                -\psi(-x)       & x < 0
                              \end{cases}   \\
  F e^{-\kappa a}         & = C \sin (l a)                  \\
  -\kappa F e^{-\kappa a} & = l C \cos (l a)                \\
  -\frac{1}{\kappa}       & = \frac{1}{l} \tan (l a)        \\
  \tan z                  & = -\frac{l}{\kappa}             \\
                          & = -\frac{l a}{\kappa a}         \\
                          & = -\frac{z}{\sqrt{z_0^2 - z^2}}
\end{align*}

For large $z_0$ the intersections occur just below $z_n = n \pi$ so \begin{align*}
  z       & = l a                                        \\
  n \pi   & \approx \frac{\sqrt{2 m (E + V_0)}}{\hbar} a \\
  E + V_0 & \approx \frac{n^2 \pi^2 \hbar^2}{2 m a^2}.
\end{align*}

As $z_0$ decreases there are fewer and fewer bound states. When $z_0 < \pi / 2$ there are no odd bound states.

\subsection{}

\begin{align*}
  \psi(x) & = \begin{cases}
                F e^{-\kappa x} & x > a     \\
                D \cos (l x)    & 0 < x < a \\
                \psi(-x)        & x < 0
              \end{cases}                                                                      \\
  1       & = \int_{-\infty}^\infty |\psi|^2 \,d x                                                             \\
          & = 2 \left( |D|^2 \int_0^a \cos^2 (l x) \,d x + |F|^2 \int_a^\infty e^{-2 \kappa x} \,d x \right)   \\
          & = 2 \left[ |D|^2 \frac{2 a l + \sin (2 a l)}{4 l} + \frac{|F|^2}{2 \kappa} e^{-2 \kappa a} \right] \\
          & = |D|^2 \left[ a + \frac{\sin (2 a l)}{2 l} + \frac{\cos^2 (a l)}{\kappa} \right]                  \\
          & = |D|^2 \left[ a + \frac{2 \sin (a l) \cos (a l)}{2 l} + \frac{\cos^3 (a l)}{l \sin (a l)} \right] \\
          & = |D|^2 \left\{ a + \frac{\cos (a l)}{l \sin (a l)} [\sin^2 (a l) + \cos^2 (a l)] \right\}         \\
          & = |D|^2 \left[ a + \frac{\cos (a l)}{l \sin (a l)} \right]                                         \\
          & = |D|^2 \left[ a + \frac{1}{l \tan (a l)} \right]                                                  \\
          & = |D|^2 \left[ a + \frac{1}{\kappa} \right]                                                        \\
  D       & = \frac{1}{\sqrt{a + 1 / \kappa}}
\end{align*}

\begin{align*}
  1                                  & = \left\{ \frac{1}{a + 1 / \kappa} \left[ a + \frac{\sin (2 a l)}{2 l} \right] + |F|^2 \frac{e^{-2 \kappa a}}{\kappa} \right\} \\
  \frac{(F e^{-\kappa a})^2}{\kappa} & = 1 - \frac{1}{a + 1 / \kappa} \left[ a + \frac{\sin (2 a l)}{2 l} \right]                                                     \\
  (F e^{-\kappa a})^2                & = \kappa - \frac{\kappa}{a + 1 / \kappa} \left[ a + \frac{\sin (2 a l)}{2 l} \right]                                           \\
                                     & = \frac{\kappa a + 1 - \kappa a - \kappa \sin (a l) \cos (a l) / l}{a + 1 / \kappa}                                            \\
                                     & = \frac{1 - \sin^2 (a l)}{a + 1 / \kappa}                                                                                      \\
  F                                  & = \frac{e^{\kappa a} \cos (a l)}{\sqrt{a + 1 / \kappa}}
\end{align*}

\subsection{}

\begin{align*}
  1                          & = 2 a V_0                            \\
  V_0                        & = \frac{1}{2 a}                      \\
  z_0                        & = \frac{a}{\hbar} \sqrt{2 m V_0}     \\
                             & = \frac{a}{\hbar} \sqrt{\frac{m}{a}} \\
                             & = \frac{\sqrt{a m}}{\hbar}           \\
  \lim_{a \rightarrow 0} z_0 & = 0                                  \\
\end{align*}

\setcounter{subsection}{33}
\subsection{}

\begin{enumerate}
  \item

        \begin{align*}
          V(x)                                                   & = \begin{cases}
                                                                       0   & x \le 0 \\
                                                                       V_0 & x > 0
                                                                     \end{cases}                                                                                                       \\
          % x < 0
          -\frac{\hbar^2}{2 m} \frac{d^2 \psi}{d x^2}            & = E \psi                                                                                                              \\
          \frac{d^2 \psi}{d x^2}                                 & = -\frac{2 m E}{\hbar^2} \psi                                                                                         \\
                                                                 & = -k^2 \psi                                                                                                           \\
          k                                                      & = \frac{\sqrt{2 m E}}{\hbar}                                                                                          \\
          \psi                                                   & = A e^{i k x} + B e^{-i k x}                                                                                          \\
          % x > 0
          -\frac{\hbar^2}{2 m} \frac{d^2 \psi}{d x^2} + V_0 \psi & = E \psi                                                                                                              \\
          \frac{d^2 \psi}{d x^2}                                 & = -\frac{2 m (E - V_0)}{\hbar^2} \psi                                                                                 \\
                                                                 & = \kappa^2 \psi                                                                                                       \\
          \kappa                                                 & = \frac{\sqrt{2 m (V_0 - E)}}{\hbar}                                                                                  \\
          \psi                                                   & = F e^{-\kappa x}                                                                                                     \\
          % Boundary condition 1
          A + B                                                  & = F                                                                                                                   \\
          i k (A - B)                                            & = -\kappa F                                                                                                           \\
          % Calculate B in terms of A
          F                                                      & = -i \frac{k}{\kappa} (A - B)                                                                                         \\
          A + B                                                  & = -i \frac{k}{\kappa} (A - B)                                                                                         \\
          \left( 1 - i \frac{k}{\kappa} \right) B                & = -\left( 1 + i \frac{k}{\kappa} \right) A                                                                            \\
          B                                                      & = -\frac{1 + i k / \kappa}{1 - i k / \kappa} A                                                                        \\
          % Calculate reflection coefficient
          R                                                      & = \frac{|B|^2}{|A|^2}                                                                                                 \\
                                                                 & = \left( -\frac{1 + i k / \kappa}{1 - i k / \kappa} \right) \left( -\frac{1 - i k / \kappa}{1 + i k / \kappa} \right) \\
                                                                 & = 1
        \end{align*}

  \item

        \begin{align*}
          \psi(x)                          & = \begin{cases}
                                                 A e^{i k x} + B e^{-i k x} & x < 0 \\
                                                 F e^{i l x}                & x > 0
                                               \end{cases}                    \\
          k                                & = \frac{\sqrt{2 m E}}{\hbar}                            \\
          l                                & = \frac{\sqrt{2 m (E - V_0)}}{\hbar}                    \\
          % Boundary conditions
          A + B                            & = F                                                     \\
          i k (A - B)                      & = i l F                                                 \\
          % Calculate B in terms of A
          F                                & = \frac{k}{l} (A - B)                                   \\
          A + B                            & = \frac{k}{l} (A - B)                                   \\
          \left( \frac{k}{l} + 1 \right) B & = \left( \frac{k}{l} - 1 \right) A                      \\
          B                                & = \frac{k / l - 1}{k / l + 1} A                         \\
          % Calculate reflection coefficient
          R                                & = \frac{|B|^2}{|A|^2}                                   \\
                                           & = \left( \frac{k / l - 1}{k / l + 1} \right)^2          \\
                                           & = \left( \frac{k - l}{k + l} \right)^2                  \\
                                           & = \frac{(k - l)^4}{(k^2 - l^2)^2}                       \\
          k^2 - l^2                        & = \frac{2 m E}{\hbar^2} - \frac{2 m (E - V_0)}{\hbar^2} \\
                                           & = \frac{2 m}{\hbar^2} V_0                               \\
          k - l                            & = \frac{\sqrt{2 m}}{\hbar} (\sqrt{E} - \sqrt{E - V_0})  \\
          R                                & = \frac{(\sqrt{E} - \sqrt{E - V_0})^4}{V_0^2}
        \end{align*}

        \setcounter{enumi}{3}
  \item

        \begin{align*}
          % Express F in terms of A
          B                                & = F - A                                                                 \\
          F                                & = \frac{k}{l} (A - F + A)                                               \\
          \left( 1 + \frac{k}{l} \right) F & = \frac{2 k}{l} A                                                       \\
          F                                & = \frac{2 k}{k + l} A                                                   \\
          \frac{l}{k}                      & = \sqrt{\frac{E - V_0}{E}}                                              \\
          T                                & = \left| \frac{F}{A} \right|^2 \frac{l}{k}                              \\
                                           & = \left( \frac{2 k}{k + l} \right)^2 \frac{l}{k}                        \\
                                           & = \frac{4 k l}{(k + l)^2}                                               \\
                                           & = \frac{4 k l (k - l)^2}{(k^2 - l^2)^2}                                 \\
                                           & = \frac{4 \sqrt{E} \sqrt{E - V_0} (\sqrt{E} - \sqrt{E - V_0})^2}{V_0^2} \\
          T + R                            & = \frac{4 k l}{(k + l)^2} + \frac{(k - l)^2}{(k + l)^2}                 \\
                                           & = \frac{4 k l + k^2 - 2 k l + l^2}{(k + l)^2}                           \\
                                           & = \frac{k^2 + 2 k l + l^2}{(k + l)^2}                                   \\
                                           & = \frac{(k + l)^2}{(k + l)^2}                                           \\
                                           & = 1
        \end{align*}
\end{enumerate}

\subsection{}

\begin{enumerate}
  \item

        \begin{align*}
          V(x)                                                   & = \begin{cases}
                                                                       0    & x < 0 \\
                                                                       -V_0 & x > 0
                                                                     \end{cases}                                                                 \\
          % x < 0
          -\frac{\hbar^2}{2 m} \frac{d^2 \psi}{d x^2}            & = E \psi                                                                       \\
          \frac{d^2 \psi}{d x^2}                                 & = -\frac{2 m E}{\hbar^2} \psi                                                  \\
                                                                 & = -k^2 \psi                                                                    \\
          k                                                      & = \frac{\sqrt{2 m E}}{\hbar}                                                   \\
          \psi                                                   & = A e^{i k x} + B e^{-i k x}                                                   \\
          % x > 0
          -\frac{\hbar^2}{2 m} \frac{d^2 \psi}{d x^2} - V_0 \psi & = E \psi                                                                       \\
          \frac{d^2 \psi}{d x^2}                                 & = -\frac{2 m}{\hbar^2} (E + V_0) \psi                                          \\
                                                                 & = -l^2 \psi                                                                    \\
          l                                                      & = \frac{\sqrt{2 m (E + V_0)}}{\hbar}                                           \\
          \psi                                                   & = F e^{i l x}                                                                  \\
          % Boundary conditions
          A + B                                                  & = F                                                                            \\
          i k (A - B)                                            & = i l F                                                                        \\
          % Calculate B in terms of A
          k (A - B)                                              & = l (A + B)                                                                    \\
          (k + l) B                                              & = (k - l) A                                                                    \\
          B                                                      & = \frac{k - l}{k + l} A                                                        \\
          R                                                      & = \left| \frac{B}{A} \right|^2                                                 \\
                                                                 & = \left( \frac{k - l}{k + l} \right)^2                                         \\
                                                                 & = \left( \frac{\sqrt{E} - \sqrt{E + V_0}}{\sqrt{E} + \sqrt{E + V_0}} \right)^2 \\
                                                                 & = \frac{1}{9}
        \end{align*}

        \setcounter{enumi}{2}
  \item

        \[T = 1 - R = \frac{8}{9}\]
\end{enumerate}

\subsection{}

\begin{align*}
  V(x)                                        & = \begin{cases}
                                                    0      & |x| < a \\
                                                    \infty & |x| > a
                                                  \end{cases}                                                                  \\
  % |x| < 1
  -\frac{\hbar^2}{2 m} \frac{d^2 \psi}{d x^2} & = E \psi                                                                            \\
  \frac{d^2 \psi}{d x^2}                      & = -\frac{2 m E}{\hbar^2} \psi                                                       \\
                                              & = -k^2 \psi                                                                         \\
  k                                           & = \frac{\sqrt{2 m E}}{\hbar}                                                        \\
  \psi                                        & = A \sin k x + B \cos k x                                                           \\
  % Boundary conditions
  0                                           & = -A \sin k a + B \cos k a                                                          \\
  0                                           & = A \sin k a + B \cos k a                                                           \\
  % Add
  B \cos k a                                  & = 0                                                                                 \\
  k                                           & = \frac{n \pi}{2 a}, \,n = 1, 3, 5, \ldots                                          \\
  E                                           & = \frac{n^2 \pi^2 \hbar^2}{2 m (2 a)^2}                                             \\
  \psi                                        & = B \cos \left( \frac{n \pi}{2 a} x \right), \,n = 1, 3, 5, \ldots                  \\
  1                                           & = |B|^2\int_{-a}^a \cos^2 \left( \frac{n \pi}{2 a} x \right) \,d x                  \\
  B                                           & = \frac{1}{\sqrt{a}}                                                                \\
  \psi                                        & = \frac{1}{\sqrt{a}} \cos \left( \frac{n \pi}{2 a} x \right), \,n = 1, 3, 5, \ldots \\
  % Subtract
  A \sin k a                                  & = 0                                                                                 \\
  k                                           & = \frac{n \pi}{2 a}, \,n = 2, 4, 6, \ldots                                          \\
  E                                           & = \frac{n^2 \pi^2 \hbar^2}{2 m (2 a)^2}                                             \\
  \psi                                        & = A \sin \left( \frac{n \pi}{2 a} x \right), \,n = 2, 4, 6, \ldots                  \\
  1                                           & = |A| \int_{-a}^a \sin^2 \left( \frac{n \pi}{2 a} x \right) \,d x                   \\
  A                                           & = \frac{1}{\sqrt{a}}                                                                \\
  \psi                                        & = \frac{1}{\sqrt{a}} \sin \left( \frac{n \pi}{2 a} x \right), \,n = 2, 4, 6, \ldots
\end{align*}

\subsection{}

\begin{align*}
  \Psi(x, 0) & = A \sin^3 \left( \frac{\pi}{a} x \right)                                                                                                                          \\
             & = A \left[ \frac{3}{4} \sin \left( \frac{\pi}{a} x \right) - \frac{1}{4} \sin \left( \frac{3 \pi}{a} x \right) \right]                                             \\
             & = A \sqrt{\frac{a}{2}} \left[ \frac{3}{4} \psi_1(x) - \frac{1}{4} \psi_3(x) \right]                                                                                \\
  1          & = |A|^2 \frac{a}{2} \int_0^a \left[ \frac{3}{4} \psi_1(x) - \frac{1}{4} \psi_3(x) \right]^2 \,d x                                                                  \\
             & = |A|^2 \frac{a}{2} \int_0^a \left[ \frac{9}{16} \psi_1(x)^2 - \frac{3}{8} \psi_1(x) \psi_3(x) + \frac{1}{16} \psi_3(x)^2 \right] \,d x                            \\
             & = \frac{5}{16} a |A|^2                                                                                                                                             \\
  A          & = \frac{4}{\sqrt{5 a}}                                                                                                                                             \\
  \Psi(x, 0) & = \frac{1}{\sqrt{10}} [3 \psi_1(x) - \psi_3(x)]                                                                                                                    \\
  \Psi(x, t) & = \frac{1}{\sqrt{10}} [3 \psi_1(x) e^{-i E_1 t / \hbar} - \psi_3(x) e^{-i E_3 t / \hbar}]                                                                          \\
  \ev{x}     & = \int_0^a \Psi^* x \Psi \,d x                                                                                                                                     \\
             & = \frac{1}{10} \int_0^a x \left( 9 \psi_1^2 + \psi_3^2 - 3 \psi_1 \psi_3 e^{-i (E_3 - E_1) t / \hbar} - 3 \psi_1 \psi_3 e^{-i (E_1 - E_3) t / \hbar} \right) \,d x \\
             & = \frac{1}{10} \int_0^a x \left[ 9 \psi_1^2 + \psi_3^2 - 6 \psi_1 \psi_3 \cos \left( \frac{E_3 - E_1}{\hbar} t \right) \right] \,d x                               \\
             & = \frac{1}{10} \left[ 9 \ev{x}_1 + \ev{x}_3 - 6 \cos \left( \frac{E_3 - E_1}{\hbar} t \right) \int_0^a x \psi_1 \psi_3 \,d x \right]                               \\
             & = \frac{1}{10} \left[ \frac{9}{2} a + \frac{1}{2} a \right]                                                                                                        \\
             & = \frac{a}{2}                                                                                                                                                      \\
  P(E_1)     & = \frac{9}{10}                                                                                                                                                     \\
  P(E_3)     & = \frac{1}{10}                                                                                                                                                     \\
  \ev{E}     & = E_1 P(E_1) + E_3 P(E_3)                                                                                                                                          \\
             & = \frac{9 \pi^2 \hbar^2}{20 m a^2} + \frac{9 \pi^2 \hbar^2}{20 m a^2}                                                                                              \\
             & = \frac{9 \pi^2 \hbar^2}{10 m a^2}
\end{align*}

\subsection{}

\begin{enumerate}
  \item

        \begin{align*}
          \Psi(x, t) & = \sum_{n = 1}^\infty c_n \psi_n(x) e^{-i E_n t / \hbar}      \\
          \Psi(x, 0) & = \sum_{n = 1}^\infty c_n \psi_n(x)                           \\
          E_n T      & = \frac{n^2 \pi^2 \hbar^2}{2 m a^2} \frac{4 m a^2}{\pi \hbar} \\
                     & = 2 \pi n^2 \hbar                                             \\
          \Psi(x, T) & = \sum_{n = 1}^\infty c_n \psi_n(x) e^{-i E_n T / \hbar}      \\
                     & = \sum_{n = 1}^\infty c_n \psi_n(x) e^{-2 \pi i n^2}          \\
                     & = \sum_{n = 1}^\infty c_n \psi_n(x)                           \\
                     & = \Psi(x, 0)
        \end{align*}

  \item

        \begin{align*}
          E & = \frac{1}{2} m v^2      \\
          v & = \sqrt{\frac{2 E}{m}}   \\
          T & = \frac{2 a}{v}          \\
            & = a \sqrt{\frac{2 m}{E}}
        \end{align*}

  \item

        \begin{align*}
          \frac{4 m a^2}{\pi \hbar}        & = a \sqrt{\frac{2 m}{E}}        \\
          \frac{16 m^2 a^2}{\pi^2 \hbar^2} & = \frac{2 m}{E}                 \\
          E                                & = \frac{\pi^2 \hbar^2}{8 m a^2} \\
                                           & = \frac{E_1}{4}
        \end{align*}
\end{enumerate}

\subsection{}

\begin{enumerate}
  \item

        \begin{align*}
          \Psi(x, 0)               & = \begin{cases}
                                         \frac{2 \sqrt{3}}{a \sqrt{a}} x       & 0 \le x \le a / 2 \\
                                         \frac{2 \sqrt{3}}{a \sqrt{a}} (a - x) & a / 2 \le x \le a
\end{cases} \\
          \frac{d}{d x} \Psi(x, 0) & = \frac{2 \sqrt{3}}{a \sqrt{a}} \left[ 1 - 2 \theta \left( x - \frac{a}{2} \right) \right]
        \end{align*}

  \item \[\frac{d^2}{d x^2} \Psi(x, 0) = -\frac{4 \sqrt{3}}{a \sqrt{a}} \delta \left( x - \frac{a}{2} \right)\]

  \item

        \begin{align*}
          \hat{H} \Psi(x, 0) & = \left[ -\frac{\hbar^2}{2 m} \frac{\partial^2}{\partial x^2} + V(x) \right] \Psi(x, 0)                                                                \\
                             & = \frac{\hbar^2}{2 m} \frac{4 \sqrt{3}}{a \sqrt{a}} \delta \left( x - \frac{a}{2} \right) + V(x) \Psi(x, 0)                                            \\
          \ev{H}             & = \int \Psi(x, 0)^* \hat{H} \Psi(x, 0) \,d x                                                                                                           \\
                             & = \int_0^a \Psi(x, 0)^* \left[ \frac{\hbar^2}{2 m} \frac{4 \sqrt{3}}{a \sqrt{a}} \delta \left( x - \frac{a}{2} \right) + V(x) \Psi(x, 0) \right] \,d x \\
                             & = \Psi \left( \frac{a}{2}, 0 \right)^* \frac{\hbar^2}{2 m} \frac{4 \sqrt{3}}{a \sqrt{a}} + \int_0^a \Psi(x, 0)^* V(x) \Psi(x, 0) \,d x                 \\
                             & = \frac{6 \hbar^2}{m a^2}
        \end{align*}
\end{enumerate}

\subsection{}

\begin{enumerate}
  \item

        \begin{align*}
          V(x)       & = \frac{1}{2} m \omega^2 x^2                                                                                                                                                          \\
          \xi        & = \sqrt{\frac{m \omega}{\hbar}} x                                                                                                                                                     \\
          \psi_n(x)  & = \left( \frac{m \omega}{\pi \hbar} \right)^{1 / 4} \frac{1}{\sqrt{2^n n!}} H_n(\xi) e^{-\xi^2 / 2}                                                                                   \\
          \Psi(x, 0) & = A \left( 1 - 2 \sqrt{\frac{m \omega}{\hbar}} x \right)^2 e^{-\frac{m \omega}{2 \hbar} x^2}                                                                                          \\
                     & = A \left( 1 - 4 \sqrt{\frac{m \omega}{\hbar}} x + \frac{4 m \omega}{\hbar} x^2 \right) e^{-\frac{m \omega}{2 \hbar} x^2}                                                             \\
                     & = A \left( \frac{\pi \hbar}{m \omega} \right)^{1 / 4} \left[ 3 \psi_0(x) - 2 \sqrt{2} \psi_1(x) + 2 \sqrt{2} \psi_2(x) \right]                                                        \\
          1          & = A^2 \sqrt{\frac{\pi \hbar}{m \omega}} \int_{-\infty}^\infty (3 \psi_0 - 2 \sqrt{2} \psi_1 + 2 \sqrt{2} \psi_2)^2 \,d x                                                              \\
                     & = A^2 \sqrt{\frac{\pi \hbar}{m \omega}} \int_{-\infty}^\infty (9 \psi_0^2 - 12 \sqrt{2} \psi_0 \psi_1 + 12 \sqrt{2} \psi_0 \psi_2 + 8 \psi_1^2 - 16 \psi_1 \psi_2 + 8 \psi_2^2) \,d x \\
                     & = 25 A^2 \sqrt{\frac{\pi \hbar}{m \omega}}                                                                                                                                            \\
          A          & = \frac{1}{5} \left( \frac{m \omega}{\pi \hbar} \right)^{1 / 4}                                                                                                                       \\
          \Psi(x, 0) & = \frac{3}{5} \psi_0(x) - \frac{2 \sqrt{2}}{5} \psi_1(x) + \frac{2 \sqrt{2}}{5} \psi_2(x)                                                                                             \\
        \end{align*}

  \item

        \begin{align*}
          E_0    & = \frac{\hbar \omega}{2}                                                                                              \\
          P(E_0) & = \frac{9}{25}                                                                                                        \\
          E_1    & = \frac{3 \hbar \omega}{2}                                                                                            \\
          P(E_1) & = \frac{8}{25}                                                                                                        \\
          E_2    & = \frac{5 \hbar \omega}{2}                                                                                            \\
          P(E_2) & = \frac{8}{25}                                                                                                        \\
          \ev{E} & = \frac{\hbar \omega}{2} \frac{9}{25} + \frac{3 \hbar \omega}{2} \frac{8}{25} + \frac{5 \hbar \omega}{2} \frac{8}{25} \\
                 & = \frac{73}{50} \hbar \omega
        \end{align*}

  \item

        \begin{align*}
          \xi               & = \sqrt{\frac{m \omega}{\hbar}} x                                                                                                                                                       \\
          \psi_n(x)         & = \left( \frac{m \omega}{\pi \hbar} \right)^{1 / 4} \frac{1}{\sqrt{2^n n!}} H_n(\xi) e^{-\xi^2 / 2}                                                                                     \\
          \Psi(x, T)        & = B \left( 1 + 2 \sqrt{\frac{m \omega}{\hbar} x} \right)^2 e^{-\frac{m \omega}{2 \hbar} x^2}                                                                                            \\
                            & = B \left( 1 + 4 \sqrt{\frac{m \omega}{\hbar}} x + 4 \frac{m \omega}{\hbar} x^2 \right) e^{-\frac{m \omega}{2 \hbar} x^2}                                                               \\
                            & = B \left( \frac{\pi \hbar}{m \omega} \right)^{1 / 4} \left[ 3 \psi_0(x) + 2 \sqrt{2} \psi_1(x) + 2 \sqrt{2} \psi_2(x) \right]                                                          \\
          1                 & = |B|^2 \sqrt{\frac{\pi \hbar}{m \omega}} \int_{-\infty}^\infty  [3 \psi_0 + 2 \sqrt{2} \psi_1 + 2 \sqrt{2} \psi_2]^2 \,d x                                                             \\
                            & = |B|^2 \sqrt{\frac{\pi \hbar}{m \omega}} \int_{-\infty}^\infty [9 \psi_0^2 + 12 \sqrt{2} \psi_0 \psi_1 + 12 \sqrt{2} \psi_0 \psi_2 + 8 \psi_1^2 + 16 \psi_1 \psi_2 + 8 \psi_2^2] \,d x \\
                            & = 25 |B|^2 \sqrt{\frac{\pi \hbar}{m \omega}}                                                                                                                                            \\
          B                 & = \frac{1}{5} \left( \frac{m \omega}{\pi \hbar} \right)^{1 / 4}                                                                                                                         \\
          \Psi(x, T)        & = \frac{3}{5} \psi_0(x) + \frac{2 \sqrt{2}}{5} \psi_1(x) + \frac{2 \sqrt{2}}{5} \psi_2(x)                                                                                               \\
          \Psi(x, t)        & = \frac{3}{5} \psi_0(x) e^{-i \omega t / 2} - \frac{2 \sqrt{2}}{5} \psi_1(x) e^{-3 i \omega t / 2} + \frac{2 \sqrt{2}}{5} \psi_2(x) e^{-5 i \omega t / 2}                               \\
                            & = e^{-i \omega t / 2} \left[ \frac{3}{5} \psi_0(x) - \frac{2 \sqrt{2}}{5} \psi_1(x) e^{-i \omega t} + \frac{2 \sqrt{2}}{5} e^{-2 i \omega t} \right]                                    \\
          e^{-i \omega T}   & = -1                                                                                                                                                                                    \\
          e^{-2 i \omega T} & = 1                                                                                                                                                                                     \\
          T                 & = \frac{\pi}{\omega}
        \end{align*}
\end{enumerate}

\subsection{}

The argument for calculating the allowed energies and wavefunctions is the same, except there is a boundary condition $\psi(0) = 0$. This leaves only $\psi_n(x)$ for odd $n$.

\setcounter{subsection}{42}
\subsection{}

\begin{align*}
  % Odd
  k                                        & = \frac{\sqrt{2 m E}}{\hbar}                                         \\
  \psi(x)                                  & = \begin{cases}
                                                 -A \sin k x + B \cos k x & -a < x \le 0 \\
                                                 A \sin k x + B \cos k x  & 0 \le x < a
                                               \end{cases}                            \\
  \Delta \left( \frac{d \psi}{d x} \right) & = 2 A k                                                              \\
  \Delta \left( \frac{d \psi}{d x} \right) & = \frac{2 m \alpha}{\hbar^2} \psi(0)                                 \\
  2 A k                                    & = \frac{2 m \alpha}{\hbar^2} B                                       \\
  B                                        & = \frac{\hbar^2 k}{m \alpha} A                                       \\
  \psi(x)                                  & = A \left( \sin k x + \frac{\hbar^2 k}{m \alpha} \cos k x \right)    \\
  0                                        & = A \sin k a + \frac{\hbar^2 k}{m \alpha} A \cos k a                 \\
  \tan k a                                 & = -\frac{\hbar^2 k}{m \alpha}                                        \\
  k a                                      & \approx \frac{n \pi}{2}, \,n = 1, 3, 5, \ldots                       \\
  E                                        & \approx \frac{n^2 \pi^2 \hbar^2}{2 m (2 a)^2}, \,n = 1, 3, 5, \ldots \\
  % Even
  \psi(x)                                  & = \begin{cases}
                                                 A \sin k x - B \cos k x & -a < x \le 0 \\
                                                 A \sin k x + B \cos k x & 0 \le x < a
                                               \end{cases}                             \\
  -B                                       & = B                                                                  \\
  B                                        & = 0                                                                  \\
  \psi(x)                                  & = A \sin k x                                                         \\
  0                                        & = A \sin k a                                                         \\
  k a                                      & = \frac{n \pi}{2}, \,n = 2, 4, 6, \ldots                             \\
  \psi(x)                                  & = A \sin \left( \frac{n \pi}{2 a} x \right)                          \\
  E                                        & = \frac{n^2 \pi^2 \hbar^2}{2 m (2 a)^2}, \,n = 2, 4, 6, \ldots
\end{align*}

\subsection{}

\begin{align*}
  -\frac{\hbar^2}{2 m} \frac{d^2 \psi_1}{d x^2} \psi_2 + V \psi_1 \psi_2                 & = E \psi_1 \psi_2                       \\
  -\frac{\hbar^2}{2 m} \frac{d^2 \psi_2}{d x^2} \psi_1 + V \psi_1 \psi_2                 & = E \psi_1 \psi_2                       \\
  \psi_2 \frac{d^2 \psi_1}{d x^2} - \psi_1 \frac{d^2 \psi_2}{d x^2}                      & = 0                                     \\
  \frac{d}{d x} \left( \psi_2 \frac{d \psi_1}{d x} - \psi_1 \frac{d \psi_2}{d x} \right) & = 0                                     \\
  \psi_2 \frac{d \psi_1}{d x} - \psi_1 \frac{d \psi_2}{d x}                              & = c                                     \\
  c                                                                                      & = 0                                     \\
  \psi_2 \frac{d \psi_1}{d x}                                                            & = \psi_1 \frac{d \psi_2}{d x}           \\
  \frac{1}{\psi_1} \frac{d \psi_1}{d x}                                                  & = \frac{1}{\psi_2} \frac{d \psi_2}{d x} \\
  \ln \psi_1                                                                             & = \ln \psi_2 + c                        \\
  \psi_1                                                                                 & = A \psi_2
\end{align*}

\subsection{}

\begin{enumerate}
  \item

        \begin{align*}
          -\frac{\hbar^2}{2 m} \frac{d^2 \psi_n}{d x^2} \psi_m + V(x) \psi_n \psi_m              & = E_n \psi_n \psi_m                             \\
          -\frac{\hbar^2}{2 m} \frac{d^2 \psi_m}{d x^2} \psi_n + V(x) \psi_n \psi_m              & = E_m \psi_n \psi_m                             \\
          \frac{d^2 \psi_m}{d x^2} \psi_n - \frac{d^2 \psi_n}{d x^2} \psi_m                      & = \frac{2 m}{\hbar^2} (E_n - E_m) \psi_n \psi_m \\
          \frac{d}{d x} \left( \frac{d \psi_m}{d x} \psi_n - \frac{d \psi_n}{d x} \psi_m \right) & = \frac{2 m}{\hbar^2} (E_n - E_m) \psi_n \psi_m
        \end{align*}

  \item

        \begin{align*}
          \int_{x_1}^{x_2} \frac{d}{d x} (\psi_m' \psi_n - \psi_n' \psi_m) \,d x & = \frac{2 m}{\hbar^2} (E_n - E_m) \int_{x_1}^{x_2} \psi_n \psi_m \,d x \\
          \psi_m'(x_2) \psi_n(x_2) - \psi_m'(x_1) \psi_n(x_1)                    & = \frac{2 m}{\hbar^2} (E_n - E_m) \int_{x_1}^{x_2} \psi_n \psi_m \,d x
        \end{align*}
\end{enumerate}

\setcounter{subsection}{52}
\subsection{}

\begin{enumerate}
  \item

        \[\frac{1}{1 - i \beta} \begin{pmatrix}
            i \beta & 1       \\
            1       & i \beta
          \end{pmatrix}\]

  \item

        \[\frac{e^{-2 i k a}}{\cos (2 l a) - i \frac{(k^2 + l^2)}{2 k l} \sin (2 l a)} \begin{pmatrix}
            i \frac{\sin (2 l a)}{2 k l} (l^2 - k^2) & 1                                        \\
            1                                        & i \frac{\sin (2 l a)}{2 k l} (l^2 - k^2) \\
          \end{pmatrix}\]
\end{enumerate}

\section{Formalism}

\subsection{}

\begin{enumerate}
  \item

        \begin{align*}
          \left| \int_a^b (f^* + g^*) (f + g) \,d x \right| & = \left| \int_a^b (f^* f + f^* g + g^* f + g^* g) \,d x \right|                                                             \\
                                                            & \le \int_a^b |f|^2 \,d x + \left| \int_a^b f^* g \,d x \right| + \left| \int_a^b g^* f \,d x \right| + \int_a^b |g|^2 \,d x \\
                                                            & \le \int_a^b |f|^2 \,d x + 2 \sqrt{\int_a^b |f|^2 \,d x \int_a^b |g|^2 \,d x} + \int_a^b |g|^2 \,d x                        \\
        \end{align*}

        The set of all normalised functions isn't a vector space because e.g. multiplying a function by a constant also multiplies its integral by that constant meaning it's no longer a member of the vector space.

  \item

        \begin{align*}
          \braket{\beta | \alpha} & = \int_a^b \beta^* \alpha \,d x                  \\
                                  & = \left( \int_a^b \alpha^* \beta \,d x \right)^* \\
                                  & = \braket{\alpha | \beta}^*                      \\
          \braket{a | a}          & = \int_a^b |a|^2 \,d x                           \\
                                  & \ge 0
        \end{align*}

        If $\braket{\alpha | \alpha} = 0$ that implies $|\alpha|^2 = 0$ everywhere in the interval and thus $\ket{\alpha} = \ket{0}$.

        \begin{align*}
          \bra{\alpha} (b \ket{\beta} + c \ket{\gamma}) & = \int_{x_1}^{x_2} \alpha^* (b \beta) \,d x + \int_{x_1}^{x_2} \alpha^* (c \gamma) \,d x \\
                                                        & = b \int_{x_1}^{x_2} \alpha^* \beta \,d x + c \int_{x_1}^{x_2} \alpha^* \gamma \,d x     \\
                                                        & = b \braket{\alpha | \beta} + c \braket{\alpha | \gamma}
        \end{align*}
\end{enumerate}

\subsection{}

\begin{enumerate}
  \item

        \[\int_0^1 x^{2 \nu} \,d x = \frac{1}{2 \nu + 1} \left[ x^{2 \nu + 1} \right]_0^1\]

        The integral is defined for $\nu > -1 / 2$. For the case $\nu = -1 / 2$

        \[\int_0^1 x^{-1} \,d x = [\ln x]_0^1 = \ln 1 - \ln 0 = 0 - \infty.\] So $f(x) = x^\nu$ is in Hilbert space for $\nu > -1 / 2$.

  \item

        \begin{align*}
          \int_0^1 x \,d x      & = \frac{1}{2} \\
          \int_0^1 x^3 \,d x    & = \frac{1}{4} \\
          \int_0^1 x^{-1} \,d x & = [\ln x]_0^1 \\
                                & = 0 - \infty
        \end{align*}

        $f(x)$ and $x f(x)$ are in Hilbert space, but not $(d / d x) f(x)$.
\end{enumerate}

\setcounter{subsection}{3}
\subsection{}

\begin{enumerate}
  \item

        \begin{align*}
          \braket{f | (\hat{Q} + \hat{R}) f} & = \braket{f | \hat{Q} f} + \braket{f | \hat{R} f} \\
                                             & = \braket{\hat{Q} f | f} + \braket{\hat{R} f | f} \\
                                             & = \braket{(\hat{Q} + \hat{R}) f | f}
        \end{align*}

  \item

        \begin{align*}
          \braket{f | \alpha \hat{Q} g} & = \alpha \braket{f | \hat{Q} g}   \\
                                        & = \alpha \braket{\hat{Q} f | g}   \\
          \braket{\alpha \hat{Q} f | g} & = \alpha^* \braket{\hat{Q} f | g} \\
          \alpha                        & = \alpha^*
        \end{align*}

        $\alpha$ is real.

  \item

        \begin{align*}
          \braket{f | \hat{Q} \hat{R} g} & = \braket{\hat{Q} f | \hat{R} g} \\
                                         & = \braket{\hat{R} \hat{Q} f | g}
        \end{align*}

        The product of the operators is hermitian when $\hat{Q} \hat{R} = \hat{R} \hat{Q}$ i.e. $[\hat{Q}, \hat{R}] = 0$.

  \item

        \begin{align*}
          \braket{\Psi | \hat{x} \Psi} & = \int \Psi^* \hat{x} \Psi \,d x                                                                                                                                                 \\
                                       & = \int \Psi^* \hat{x}^* \Psi \,d x                                                                                                                                               \\
                                       & = \int (\hat{x} \Psi)^* \Psi \,d x                                                                                                                                               \\
                                       & = \braket{\hat{x} \Psi | \Psi}                                                                                                                                                   \\
          \braket{\Psi | \hat{H} \Psi} & = \int \Psi^* \hat{H} \Psi \,d x                                                                                                                                                 \\
                                       & = \int \Psi^* \left[ -\frac{\hbar^2}{2 m} \frac{d^2}{d x^2} + V(x) \right] \Psi \,d x                                                                                            \\
                                       & = -\frac{\hbar^2}{2 m} \int \Psi^* \frac{d^2 \Psi}{d x^2} \,d x + \int \Psi^* V(x) \Psi \,d x                                                                                    \\
                                       & = -\frac{\hbar^2}{2 m} \left[ \left. \Psi^* \frac{d \Psi}{d x} \right|_{-\infty}^\infty - \int \frac{d \Psi^*}{d x} \frac{d \Psi}{d x} \,d x \right] + \braket{V(x) \Psi | \Psi} \\
                                       & = \frac{\hbar^2}{2 m} \left[ \left. \frac{d \Psi^*}{d x} \Psi \right|_{-\infty}^\infty - \int \frac{d^2 \Psi^*}{d x} \Psi \,d x \right] + \braket{V(x) \Psi | \Psi}              \\
                                       & = -\frac{\hbar^2}{2 m} \int \frac{d^2}{d x^2} \Psi^* \Psi \,d x + \braket{V(x) \Psi | \Psi}                                                                                      \\
                                       & = \Braket{-\frac{\hbar^2}{2 m} \frac{d^2}{d x^2} \Psi | \Psi} + \braket{V(x) \Psi | \Psi}                                                                                        \\
                                       & = \braket{\hat{H} \Psi | \Psi}
        \end{align*}
\end{enumerate}

\subsection{}

\begin{enumerate}
  \item

        \begin{align*}
          x^\dagger                            & = x              \\
          i^\dagger                            & = -i             \\
          \left( \frac{d}{d x} \right)^\dagger & = -\frac{d}{d x}
        \end{align*}

  \item

        \begin{align*}
          \braket{f | \hat{Q} \hat{R} g} & = \int f^\dagger \hat{Q} \hat{R} g \,d x                   \\
                                         & = \int (\hat{Q}^\dagger f)^\dagger \hat{R} g \,d x         \\
                                         & = \int (\hat{R}^\dagger \hat{Q}^\dagger f)^\dagger g \,d x \\
                                         & = \braket{\hat{R}^\dagger \hat{Q}^\dagger f | g}
        \end{align*}

  \item

        \begin{align*}
          \hat{a}_+              & = \frac{1}{\sqrt{2 \hbar m \omega}} (-i \hat{p} + m \omega x)                               \\
          \braket{f | \hat{a} g} & = \Braket{f | \frac{1}{\sqrt{2 \hbar m \omega}} (-i \hat{p} + m \omega x) g}                \\
                                 & = \frac{1}{\sqrt{2 \hbar m \omega}} \braket{f | (-i \hat{p} + m \omega x) g}                \\
                                 & = \frac{1}{\sqrt{2 \hbar m \omega}} (\braket{f | -i \hat{p} g} + \braket{f | m \omega x g}) \\
                                 & = \frac{1}{\sqrt{2 \hbar m \omega}} (\braket{f | -i \hat{p} g} + \braket{m \omega x f | g}) \\
                                 & = \frac{1}{\sqrt{2 \hbar m \omega}} (\braket{i \hat{p} f | g} + \braket{m \omega x f | g})  \\
                                 & = \Braket{\frac{1}{\sqrt{2 \hbar m \omega}} (i \hat{p} + m \omega x) f | g}                 \\
                                 & = \braket{\hat{a}_- f | g}
        \end{align*}
\end{enumerate}

\subsection{}

\begin{align*}
  \braket{f | \hat{Q} g} & = \int_0^{2 \pi} f^* \frac{d^2 g}{d \phi^2} \,d \phi                                                                                \\
                         & = \left. f^* \frac{d g}{d \phi} \right|_0^{2 \pi} - \int_0^{2 \pi} \left( \frac{d f}{d \phi} \right)^* \frac{d g}{d \phi} \,d \phi  \\
                         & = \left. -\left( \frac{d f}{d \phi} \right)^* g \right|_0^{2 \pi} + \int_0^{2 \pi} \left( \frac{d^2 f}{d \phi} \right)^* g \,d \phi \\
                         & = \braket{\hat{Q} f | g}
\end{align*}

Yes, the operator is hermitian.

\begin{align*}
  \hat{Q} f                    & = q f                                                                             \\
  \frac{d^2 f}{d \phi^2}       & = q f                                                                             \\
  \frac{d^2 f}{d \phi^2} - q f & = 0                                                                               \\
  f                            & = A e^{\sqrt{q} \phi} + B e^{-\sqrt{q} \phi}                                      \\
  f(\phi + 2 \pi)              & = A e^{\sqrt{q} (\phi + 2 \pi)} + B e^{\sqrt{q} (\phi + 2 \pi)}                   \\
                               & = A e^{\sqrt{q} \phi} e^{2 \pi \sqrt{q}} + B e^{\sqrt{q} \phi} e^{2 \pi \sqrt{q}} \\
  2 \pi \sqrt{q}               & = 1                                                                               \\
  q                            & = -n^2, \,n = 0, 1, 2, \ldots
\end{align*}

The eigenfunctions are $f = A e^{\pm \sqrt{q} \phi}$ and the eigenvalues are $q = 0, 1, 2, \ldots$ The spectrum is degenerate as there are two eigenfunctions associated with each eigenvalue $q > 0$.

\subsection{}

\begin{enumerate}
  \item

        \begin{align*}
          h         & = a f + b g                     \\
          \hat{Q} h & = \hat{Q} (a f + b g)           \\
                    & = \hat{Q} (a f) + \hat{Q} (b g) \\
                    & = a \hat{Q} f + b \hat{Q} g     \\
                    & = a q f + b q g                 \\
                    & = q (a f + b g)                 \\
                    & = q h
        \end{align*}

  \item

        \begin{align*}
          \frac{d^2}{d x^2} e^x    & = e^x          \\
          \frac{d^2}{d x^2} e^{-x} & = e^{-x}       \\
          f                        & = e^x + e^{-x} \\
          g                        & = e^x - e^{-x}
        \end{align*}
\end{enumerate}

\subsection{}

\begin{enumerate}
  \item

        \begin{align*}
          \hat{Q}                                                  & = i \frac{d}{d \phi}                               \\
          \hat{Q} f                                                & = q f                                              \\
          i \frac{d f}{d \phi}                                     & = q f                                              \\
          \frac{d f}{d \phi} + i q f                               & = 0                                                \\
          f                                                        & = A e^{-i q \phi}                                  \\
          e^{-2 \pi i q}                                           & = 1                                                \\
          q                                                        & = 0, \pm 1, \pm 2, \ldots                          \\
          \int_0^{2 \pi} A e^{-i q \phi} B e^{-i q' \phi} \,d \phi & = A B \int_0^{2 \pi} e^{-i (q + q') \phi} \,d \phi \\
                                                                   & = 0
        \end{align*}

  \item

        \begin{align*}
          \hat{Q}                                                                  & = \frac{d^2}{d \phi^2}                                                              \\
          \hat{Q} f                                                                & = q f                                                                               \\
          \frac{d^2 f}{d \phi^2} - q f                                             & = 0                                                                                 \\
          f                                                                        & = A e^{\pm \sqrt{q} \phi}                                                           \\
          q                                                                        & = -n^2, \,n = 0, 1, 2, \ldots                                                       \\
          \int_0^{2 \pi} A e^{\pm \sqrt{q} \phi} B e^{\pm \sqrt{q'} \phi} \,d \phi & = A B \int_0^{2 \pi} e^{\pm i n \phi} e^{\pm i n' \phi} \,d \phi                    \\
                                                                                   & = A B \int_0^{2 \pi} e^{i (\pm n \pm n') \phi} \,d \phi                             \\
                                                                                   & = A B \left[ \frac{1}{i (\pm n \pm n')} e^{i (\pm n \pm n') \phi} \right]_0^{2 \pi} \\
                                                                                   & = A B \frac{1}{i (\pm n \pm n')} [e^{i (\pm n \pm n') 2 \pi} - 1]                   \\
                                                                                   & = 0
        \end{align*}
\end{enumerate}

\subsection{}

\begin{enumerate}
  \item Infinite square well

  \item Delta function barrier

  \item Delta function well
\end{enumerate}

\setcounter{subsection}{10}
\subsection{}

\begin{align*}
  \Psi_0(x, t)                                                               & = \left( \frac{m \omega}{\pi \hbar} \right)^{1 / 4} e^{-m \omega x^2 / 2 \hbar} e^{-i \omega t / 2}                                                                             \\
  \Phi_0(p, t)                                                               & = \frac{1}{\sqrt{2 \pi \hbar}} \int_{-\infty}^\infty e^{-i p x / \hbar} \left( \frac{m \omega}{\pi \hbar} \right)^{1 / 4} e^{-m \omega x^2 / 2 \hbar} e^{-i \omega t / 2} \,d x \\
                                                                             & = \frac{1}{\sqrt{2 \pi \hbar}} \left( \frac{m \omega}{\pi \hbar} \right)^{1 / 4} e^{-i \omega t / 2} \int_{-\infty}^\infty e^{-i p x / \hbar} e^{-m \omega x^2 / 2 \hbar} \,d x \\
                                                                             & = \frac{1}{\sqrt{2 \pi \hbar}} \left( \frac{m \omega}{\pi \hbar} \right)^{1 / 4} e^{-i \omega t / 2} \sqrt{\frac{2 \pi \hbar}{m \omega}} e^{-p^2 / 2 \hbar m \omega}            \\
                                                                             & = \frac{1}{(\pi \hbar m \omega)^{1 / 4}} e^{-p^2 / 2 \hbar m \omega} e^{-i \omega t / 2}                                                                                        \\
  \frac{p^2}{2 m}                                                            & = \frac{\hbar \omega}{2}                                                                                                                                                        \\
  p                                                                          & = \pm \sqrt{\hbar m \omega}                                                                                                                                                     \\
  1 - \int_{-\sqrt{\hbar m \omega}}^{\sqrt{\hbar m \omega}} |\Phi_0|^2 \,d p & = 1 - \frac{1}{(\pi \hbar m \omega)^{1 / 4}} \int_{-\sqrt{\hbar m \omega}}^{\sqrt{\hbar m \omega}} e^{-p^2 / \hbar m \omega} \,d p                                              \\
                                                                             & = 1 - \frac{1}{(\pi \hbar m \omega)^{1 / 2}} \sqrt{\pi \hbar m \omega} \erf 1                                                                                                   \\
                                                                             & = 0.16
\end{align*}

\subsection{}

\begin{align*}
  \Psi(x, t)     & = \frac{1}{\sqrt{2 \pi}} \int_{-\infty}^\infty \phi(k) e^{i \left( k x - \frac{\hbar k^2}{2 m} t \right)} \,d k                                                                                            \\
  \Phi(p, t)     & = \frac{1}{\sqrt{2 \pi \hbar}} \int_{-\infty}^\infty e^{-i p x / \hbar} \left( \frac{1}{\sqrt{2 \pi}} \int_{-\infty}^\infty \phi(k) e^{i \left( k x - \frac{\hbar k^2}{2 m} t \right)} \,d k \right) \,d x \\
                 & = \frac{1}{2 \pi \sqrt{\hbar}} \int_{-\infty}^\infty \phi(k) e^{-i \frac{\hbar k^2}{2 m} t} \left( \int_{-\infty}^\infty e^{i (k - p / \hbar) x} \,d x \right) \,d k                                       \\
                 & = \frac{1}{2 \pi \sqrt{\hbar}} \int_{-\infty}^\infty \phi(k) e^{-i \frac{\hbar k^2}{2 m} t} 2 \pi \delta(k - p / \hbar) \,d k                                                                              \\
                 & = \frac{1}{\sqrt{\hbar}} \int_{-\infty}^\infty \delta(k - p / \hbar) \phi(k) e^{-i \frac{\hbar k^2}{2 m} t} \,d k                                                                                          \\
                 & = \frac{1}{\sqrt{\hbar}} \phi(p / \hbar) e^{-i \frac{p^2}{2 \hbar m} t}                                                                                                                                    \\
  |\Phi(p, t)|^2 & = \frac{1}{\hbar} |\phi(p / \hbar)|^2
\end{align*}

\setcounter{subsection}{13}
\subsection{}

\begin{enumerate}
  \item

        \begin{align*}
          [\hat{A} + \hat{B}, \hat{C}]                            & = (\hat{A} + \hat{B}) \hat{C} - \hat{C} (\hat{A} + \hat{B})                                             \\
                                                                  & = \hat{A} \hat{C} + \hat{B} \hat{C} - \hat{C} \hat{A} - \hat{C} \hat{B}                                 \\
                                                                  & = \hat{A} \hat{C} - \hat{C} \hat{A} + \hat{B} \hat{C} - \hat{C} \hat{B}                                 \\
                                                                  & = [\hat{A}, \hat{C}] + [\hat{B}, \hat{C}]                                                               \\
          \hat{A} [\hat{B}, \hat{C}] + [\hat{A}, \hat{C}] \hat{B} & = \hat{A} \hat{B} \hat{C} - \hat{A} \hat{C} \hat{B} + \hat{A} \hat{C} \hat{B} - \hat{C} \hat{A} \hat{B} \\
                                                                  & = \hat{A} \hat{B} \hat{C} - \hat{C} \hat{A} \hat{B}                                                     \\
                                                                  & = [\hat{A} \hat{B}, \hat{C}]
        \end{align*}

  \item

        \begin{align*}
          [x^n, \hat{p}] & = \left[ x^n, -i \hbar \frac{d}{d x} \right]                                            \\
                         & = x^n \left( -i \hbar \frac{d}{d x} \right) - \left( -i \hbar \frac{d}{d x} \right) x^n \\
                         & = x^n \left( -i \hbar \frac{d}{d x} \right) (1) + i \hbar n x^{n - 1}                   \\
                         & = i \hbar n x^{n - 1}
        \end{align*}

  \item

        \begin{align*}
          [f(x), \hat{p}] g(x) & = f(x) \left( -i \hbar \frac{d}{d x} \right) g(x) - \left( -i \hbar \frac{d}{d x} \right) [f(x) g(x)] \\
                               & = -i \hbar f(x) \frac{d g}{d x} + i \hbar \left[ \frac{d f}{d x} g(x) + f(x) \frac{d g}{d x} \right]  \\
                               & = i \hbar \frac{d f}{d x} g(x)                                                                        \\
          [f(x), \hat{p}]      & = i \hbar \frac{d f}{d x}
        \end{align*}

  \item

        \begin{align*}
          [\hat{H}, \hat{a}_-] g & = \hat{H} \hat{a}_- g - \hat{a}_- \hat{H} g                                                                                                                         \\
                                 & = \hbar \omega \left( \hat{a}_- \hat{a}_+ - \frac{1}{2} \right) \hat{a}_- g - \hat{a}_- \hbar \omega \left( \hat{a}_- \hat{a}_+ - \frac{1}{2} \right) g             \\
                                 & = \hbar \omega \hat{a}_- \hat{a}_+ \hat{a}_- g - \frac{1}{2} \hbar \omega \hat{a}_- g - \hbar \omega \hat{a}_-^2 \hat{a}_+ g + \frac{1}{2} \hbar \omega \hat{a}_- g \\
                                 & = \hbar \omega \hat{a}_- \hat{a}_+ \hat{a}_- g - \hbar \omega \hat{a}_-^2 \hat{a}_+ g                                                                               \\
                                 & = \hbar \omega \hat{a}_- (\hat{a}_+ \hat{a}_- - \hat{a}_- \hat{a}_+) g                                                                                              \\
                                 & = -\hbar \omega \hat{a}_- g                                                                                                                                         \\
          [\hat{H}, \hat{a}_-]   & = -\hbar \omega \hat{a}_-                                                                                                                                           \\
          [\hat{H}, \hat{a}_+] g & = \hat{H} \hat{a}_+ g - \hat{a}_+ \hat{H} g                                                                                                                         \\
                                 & = \hbar \omega \left( \hat{a}_- \hat{a}_+ - \frac{1}{2} \right) \hat{a}_+ g - \hat{a}_+ \hbar \omega \left( \hat{a}_- \hat{a}_+ - \frac{1}{2} \right) g             \\
                                 & = \hbar \omega \hat{a}_- \hat{a}_+^2 g - \frac{1}{2} \hbar \omega \hat{a}_+ g - \hbar \omega \hat{a}_+ \hat{a}_- \hat{a}_+ g + \frac{1}{2} \hbar \omega \hat{a}_+ g \\
                                 & = \hbar \omega \hat{a}_- \hat{a}_+^2 g - \hbar \omega \hat{a}_+ \hat{a}_- \hat{a}_+ g                                                                               \\
                                 & = \hbar \omega (\hat{a}_- \hat{a}_+ - \hat{a}_+ \hat{a}_-) \hat{a}_+ g                                                                                              \\
                                 & = \hbar \omega \hat{a}_+ g                                                                                                                                          \\
          [\hat{H}, \hat{a}_+]   & = \hbar \omega \hat{a}_+
        \end{align*}
\end{enumerate}

\subsection{}

\begin{align*}
  \left[ x, \frac{p^2}{2 m} + V \right] g & = x \left( \frac{p^2}{2 m} + V \right) g - \left( \frac{p^2}{2 m} + V \right) x g                                                                \\
                                          & = x \frac{p^2}{2 m} g + x V g - \frac{p^2}{2 m} x g - V x g                                                                                      \\
                                          & = \frac{1}{2 m} (x p^2 g - p^2 x g)                                                                                                              \\
                                          & = \frac{1}{2 m} \left[ -\hbar^2 x \frac{d^2 g}{d x^2} + \hbar^2 \frac{d}{d x} \left( g + x \frac{d g}{d x} \right) \right]                       \\
                                          & = \frac{1}{2 m} \left[ -\hbar^2 x \frac{d^2 g}{d x^2} + \hbar^2 \left( \frac{d g}{d x} + \frac{d g}{d x} + x \frac{d^2 g}{d x^2} \right) \right] \\
                                          & = \frac{\hbar^2}{m} \frac{d g}{d x}                                                                                                              \\
  \left[ x, \frac{p^2}{2 m} + V \right]   & = \frac{\hbar^2}{m} \frac{d}{d x}                                                                                                                \\
                                          & = -\frac{\hbar}{i m} \braket{p}                                                                                                                  \\
  \sigma_x^2 \sigma_H^2                   & \ge \left( \frac{1}{2 i} \Braket{\left[ x, \frac{p^2}{2 m} + V \right]} \right)^2                                                                \\
                                          & = \frac{\hbar^2}{4 m^2} |\braket{p}|^2                                                                                                           \\
  \sigma_x \sigma_H                       & \ge \frac{\hbar}{2 m} |\braket{p}|
\end{align*}

This doesn't tell us much because for stationary states $\sigma_H = 0$ and $\braket{p} = 0$ so this says $0 \ge 0$.

\setcounter{subsection}{16}
\subsection{}

\begin{align*}
  \left( -i \hbar \frac{d}{d x} - \braket{p} \right) \Psi                                                         & = i a (x - \braket{x}) \Psi                                                    \\
  -i \hbar \frac{d \Psi}{d x} - \braket{p} \Psi                                                                   & = i a (x - \braket{x}) \Psi                                                    \\
  \frac{d \Psi}{d x} + \frac{\braket{p} + i a (x - \braket{x})}{i \hbar} \Psi                                     & = 0                                                                            \\
  \frac{d \Psi}{d x} + \left[ \frac{a}{\hbar} (x - \braket{x}) - i \frac{\braket{p}}{\hbar} \right] \Psi          & = 0                                                                            \\
  \frac{d \Psi}{d x} + \frac{a}{\hbar} x \Psi - \frac{a}{\hbar} \braket{x} \Psi - i \frac{\braket{p}}{\hbar} \Psi & = 0                                                                            \\
  \Psi                                                                                                            & = A e^{-a x^2 / 2 \hbar} e^{a \braket{x} x / \hbar} e^{i \braket{p} x / \hbar} \\
                                                                                                                  & = B e^{-a (x - \braket{x})^2 / 2 \hbar} e^{i \braket{p} x / \hbar}
\end{align*}

\subsection{}

\begin{enumerate}
  \item

        \begin{align*}
          Q                        & = 1 \\
          \hat{Q}                  & = 1 \\
          [\hat{H}, \hat{Q}]       & = 0 \\
          \frac{d}{d t} \braket{Q} & = 0
        \end{align*}

  \item

        \begin{align*}
          Q                        & = H                  \\
          \hat{Q}                  & = \hat{H}            \\
          [\hat{H}, \hat{Q}]       & = [\hat{H}, \hat{H}] \\
                                   & = 0                  \\
          \frac{d}{d t} \braket{H} & = 0
        \end{align*}

  \item

        \begin{align*}
          Q                        & = x                                             \\
          \hat{Q}                  & = x                                             \\
          [\hat{H}, \hat{Q}]       & = [\hat{H}, x]                                  \\
                                   & = -i \frac{\hbar}{m} \hat{p}                    \\
          \frac{d}{d t} \braket{x} & = \frac{i}{\hbar} \Braket{-i \frac{\hbar}{m} p} \\
                                   & = \frac{\braket{p}}{m}
        \end{align*}

  \item

        \begin{align*}
          Q                        & = p                                                              \\
          \hat{Q}                  & = \hat{p}                                                        \\
          [\hat{H}, \hat{Q}]       & = [\hat{H}, \hat{p}]                                             \\
                                   & = i \hbar \frac{\partial V}{\partial x}                          \\
          \frac{d}{d t} \braket{p} & = \frac{i}{\hbar} \Braket{i \hbar \frac{\partial V}{\partial x}} \\
                                   & = -\Braket{\frac{\partial V}{\partial x}}
        \end{align*}
\end{enumerate}

\subsection{}

\begin{enumerate}
  \item

        \begin{align*}
          \frac{d^2}{d t^2} \braket{x} & = \frac{d}{d x} \left( \frac{\braket{p}}{m} \right)   \\
                                       & = -\frac{1}{m} \Braket{\frac{\partial V}{\partial x}} \\
                                       & = 0
        \end{align*}

  \item

        \begin{align*}
          \frac{d^2}{d t^2} \braket{x}                       & = \frac{d}{d x} \left( \frac{\braket{p}}{m} \right)   \\
                                                             & = -\frac{1}{m} \Braket{\frac{\partial V}{\partial x}} \\
                                                             & = -\omega^2 \braket{x}                                \\
          \frac{d^2}{d t^2} \braket{x} + \omega^2 \braket{x} & = 0                                                   \\
          \braket{x}                                         & = A \sin \omega t + B \cos \omega t
        \end{align*}
\end{enumerate}

\subsection{}

\begin{align*}
  \Psi           & = \frac{1}{\sqrt{2}} (\psi_1 e^{-i E_1 t/ \hbar} + \psi_2 e^{-i E_2 t / \hbar})                                                                                    \\
  \hat{H} \Psi   & = \frac{1}{\sqrt{2}} (E_1 \psi_1 e^{-i E_1 t / \hbar} + E_2 \psi_2 e^{-i E_2 t / \hbar})                                                                           \\
  \hat{H}^2 \Psi & = \frac{1}{\sqrt{2}} (E_1^2 \psi_1 e^{-i E_1 t / \hbar} + E_2^2 \psi_2 e^{-i E_2 t / \hbar})                                                                       \\
  \braket{H^2}   & = \braket{\Psi | \hat{H}^2 \Psi}                                                                                                                                   \\
                 & = \frac{1}{2} \int_0^a (\psi_1^* e^{i E_1 t / \hbar} + \psi_2^* e^{i E_2 t / \hbar}) (E_1^2 \psi_1 e^{-i E_1 t / \hbar} + E_2^2 \psi_2 e^{-i E_2 t / \hbar}) \,d x \\
                 & = \frac{1}{2} \int_0^a (E_1^2 |\psi_1|^2 + E_2^2 \psi_1^* \psi_2^* e^{i (E_1 - E_2) t / \hbar} + E_1^2 \psi_2^* \psi_1 e^{i (E_2 - E_1) t / \hbar}                 \\
                 & \qquad + E_2^2 |\psi_2|^2) \,d x                                                                                                                                   \\
                 & = \frac{1}{2} (E_1^2 + E_2^2)                                                                                                                                      \\
  \braket{H}     & = \frac{1}{2} (E_1 + E_2)                                                                                                                                          \\
  \sigma_H^2     & = \braket{H^2} - \braket{H}^2                                                                                                                                      \\
                 & = \frac{1}{2} (E_1^2 + E_2^2) - \frac{1}{4} (E_1 + E_2)^2                                                                                                          \\
                 & = \frac{1}{4} (E_2 - E_1)^2
\end{align*}

\begin{align*}
  \omega                   & = \frac{\pi^2 \hbar}{2 m a^2}                                                                                                                              \\
  \braket{x}               & = \frac{a}{2} \left[ 1 - \frac{32}{9 \pi^2} \cos (3 \omega t) \right]                                                                                      \\
  \braket{x^2}             & = \frac{1}{2} \int_0^a (\psi_1^* e^{i E_1 t / \hbar} + \psi_2^* e^{i E_2 t / \hbar}) x^2 (\psi_1 e^{-i E_1 t / \hbar} + \psi_2 e^{-i E_2 t / \hbar}) \,d x \\
                           & = \frac{1}{2} \int_0^a x^2 (|\psi_1|^2 + \psi_1^* \psi_2 e^{i (E_1 - E_2) t / \hbar} + \psi_2^* \psi_1 e^{i (E_2 - E_1) t / \hbar} + |\psi_2|^2) \,d x     \\
                           & = \frac{1}{a} \int_0^a x^2 \left[ \sin^2 \left( \frac{\pi}{a} x \right) + \sin^2 \left( \frac{2 \pi}{a} x \right) \right.                                  \\
                           & \qquad \left. + 2 \sin \left( \frac{\pi}{a} x \right) \sin \left( \frac{2 \pi}{a} x \right) \cos (3 \omega t) \right] \,d x                                \\
                           & = \frac{a^2}{144 \pi^2} [-45 + 48 \pi^2 - 256 \cos (3 \omega t)]                                                                                           \\
  \sigma_x^2               & = \braket{x^2} - \braket{x}^2                                                                                                                              \\
                           & = \frac{a^2}{4} \left[ \frac{1}{3} - \frac{5}{4 \pi^2} - \left( \frac{32}{9 \pi^2} \right)^2 \cos^2 (3 \omega t) \right]                                   \\
  \frac{d \braket{x}}{d t} & = \frac{16 a \omega}{3 \pi^2} \sin (3 \omega t)                                                                                                            \\
                           & = \frac{8 \hbar}{3 m a} \sin (3 \omega t)
\end{align*}

\begin{align*}
  \sigma_H^2 \sigma_x^2                                                                                                                                                               & \ge \left( \frac{\hbar}{2} \right)^2 \left| \frac{d \braket{x}}{d t} \right|^2                \\
  \frac{1}{4} \left( \frac{3 \pi^2 \hbar^2}{2 m a^2} \right)^2 \frac{a^2}{4} \left[ \frac{1}{3} - \frac{5}{4 \pi^2} - \left( \frac{32}{9 \pi^2} \right)^2 \cos^2 (3 \omega t) \right] & \ge \left( \frac{\hbar}{2} \right)^2 \left[ \frac{8 \hbar}{3 m a} \sin (3 \omega t) \right]^2 \\
  \left( \frac{3}{4} \right)^2 \left[ \frac{1}{3} - \frac{5}{4 \pi^2} - \left( \frac{32}{9 \pi^2} \right)^2 \cos^2 (3 \omega t) \right]                                               & \ge \left( \frac{8}{3 \pi^2} \right)^2 \sin^2 (3 \omega t)                                    \\
  \frac{1}{3} - \frac{5}{4 \pi^2}                                                                                                                                                     & \ge \left( \frac{32}{9 \pi^2} \right)^2
\end{align*}

\setcounter{subsection}{22}
\subsection{}

\begin{align*}
  \hat{P}^2 \ket{\beta} & = \hat{P} (\hat{P} \ket{\beta})                                   \\
                        & = \hat{P} (\braket{\alpha | \beta} \ket{\alpha})                  \\
                        & = \braket{\alpha | \beta} \hat{P} \ket{\alpha}                    \\
                        & = \braket{\alpha | \beta} (\braket{\alpha | \alpha} \ket{\alpha}) \\
                        & = \braket{\alpha | \beta} \ket{\alpha}                            \\
                        & = \hat{P} \ket{\beta}
\end{align*}

So, $\hat{P}^2 = \hat{P}$.

\begin{align*}
  \hat{P} \ket{\beta}                  & = \lambda \ket{\beta} \\
  \braket{\alpha | \beta} \ket{\alpha} & = \lambda \ket{\beta}
\end{align*}

If $\ket{\beta}$ is a constant multiple of $\ket{\alpha}$, then

\begin{align*}
  \braket{\alpha | c \alpha} \ket{\alpha} & = \lambda c \ket{\alpha} \\
  c \braket{\alpha | \alpha} \ket{\alpha} & = \lambda c \ket{\alpha} \\
  c \ket{\alpha}                          & = \lambda c \ket{\alpha}
\end{align*}

thus $\lambda = 1$.

If $\ket{\beta}$ is orthogonal to $\ket{\alpha}$, then

\begin{align*}
  \braket{\alpha | \beta} \ket{\alpha} & = \lambda \ket{\beta} \\
  0                                    & = \lambda \ket{\beta}
\end{align*}

thus $\lambda = 0$.

\subsection{}

\begin{align*}
  \hat{Q}   & = \hat{Q}^\dagger                      \\
  Q_{m n}   & = \braket{e_m | \hat{Q} | e_n}         \\
  Q_{n m}   & = \braket{e_n | \hat{Q} | e_m}         \\
  Q_{n m}^* & = \braket{e_n | \hat{Q} | e_m}^*       \\
            & = \braket{e_m | \hat{Q}^\dagger | e_n} \\
            & = \braket{e_m | \hat{Q} | e_n}         \\
            & = Q_{m n}
\end{align*}

\subsection{}

\begin{align*}
  \hat{H} \ket{\alpha}                                                                                                 & = \lambda \ket{\alpha}                                                \\
  \epsilon [(\braket{1 | \alpha} + \braket{2 | \alpha}) \ket{1} + (\braket{1 | \alpha} - \braket{2 | \alpha}) \ket{2}] & = \lambda (\braket{1 | \alpha} \ket{1} + \braket{2 | \alpha} \ket{2})
\end{align*}

From this we get two equations

\begin{align*}
  \epsilon (\braket{1 | \alpha} + \braket{2 | \alpha})  & = \lambda \braket{1 | \alpha} \\
  \epsilon (\braket{1 | \alpha}) - \braket{2 | \alpha}) & = \lambda \braket{2 | \alpha}
\end{align*}

If we let $\ket{\alpha} = \begin{pmatrix}
    a \\
    b
  \end{pmatrix}$ then this becomes

\begin{align*}
  \epsilon (a + b) & = \lambda a \\
  \epsilon (a - b) & = \lambda b
\end{align*}

or in matrix form

\begin{align*}
  \epsilon \begin{pmatrix}
             1 & 1  \\
             1 & -1
           \end{pmatrix} \begin{pmatrix}
                           a \\
                           b
                         \end{pmatrix} & = \lambda \begin{pmatrix}
                                                     a \\
                                                     b
                                                   \end{pmatrix}.
\end{align*}

The eigenvalues of this matrix are $\lambda = \pm \sqrt{2} \epsilon$ and the eigenvectors are $\ket{1} + (\sqrt{2} \pm 1) \ket{2}$.

\subsection{}

\begin{enumerate}
  \item

        \begin{align*}
          \bra{\alpha} & = -i \bra{1} - 2 \bra{2} + i \bra{3} \\
          \bra{\beta}  & = -i \bra{1} + 2 \bra{3}
        \end{align*}

  \item

        \begin{align*}
          \braket{\alpha | \beta} & = (-i \bra{1} - 2 \bra{2} + i \bra{3}) (i \ket{1} + 2 \ket{3})                                                      \\
                                  & = \braket{1 | 1} - 2 i \braket{1 | 3} - 2 i \braket{2 | 1} - 4 \braket{2 | 3} - \braket{3 | 1} + 2 i \braket{3 | 3} \\
                                  & = 1 + 2 i                                                                                                           \\
          \braket{\beta | \alpha} & = (-i \bra{1} + 2 \bra{3}) (i \ket{1} - 2 \ket{2} - i \ket{3})                                                      \\
                                  & = \braket{1 | 1} + 2 i \braket{1 | 2} - \braket{1 | 3} + 2 i \braket{3 | 1} - 4 \braket{3 | 2} - 2 i \braket{3 | 3} \\
                                  & = 1 - 2 i                                                                                                           \\
                                  & = \braket{\alpha | \beta}^*
        \end{align*}

  \item

        \begin{align*}
          \hat{A}         & = \ket{\alpha} \bra{\beta}              \\
          \hat{A} \ket{1} & = \ket{\alpha} \braket{\beta | 1}       \\
                          & = \ket{\alpha} \braket{1 | \beta}^*     \\
                          & = -i \ket{\alpha}                       \\
                          & = \ket{1} + 2 i \ket{2} - \ket{3}       \\
          A_{1 1}         & = \braket{1 | \hat{A} | 1}              \\
                          & = 1                                     \\
          A_{2 1}         & = 2 i                                   \\
          A_{3 1}         & = -1                                    \\
          \hat{A} \ket{2} & = \ket{\alpha} \braket{\beta | 2}       \\
                          & = \ket{\alpha} \braket{2 | \beta}^*     \\
                          & = 0                                     \\
          A_{1 2}         & = 0                                     \\
          A_{2 2}         & = 0                                     \\
          A_{3 2}         & = 0                                     \\
          \hat{A} \ket{3} & = \ket{\alpha} \braket{\beta | 3}       \\
                          & = \ket{\alpha} \braket{3 | \beta}^*     \\
                          & = 2 \ket{\alpha}                        \\
                          & = 2 i \ket{1} - 4 \ket{2} - 2 i \ket{3} \\
          A_{1 3}         & = 2 i                                   \\
          A_{2 3}         & = -4                                    \\
          A_{3 3}         & = -2 i                                  \\
          A               & = \begin{pmatrix}
                                1   & 0 & 2 i  \\
                                2 i & 0 & -4   \\
                                -1  & 0 & -2 i
                              \end{pmatrix}
        \end{align*}

        It's not hermitian.
\end{enumerate}

\subsection{}

\begin{enumerate}
  \item

        \begin{align*}
          \hat{Q} \ket{\alpha} & = \hat{Q} \sum_{n = 1}^\infty \braket{e_n | \alpha} \ket{e_n}                 \\
                               & = \hat{Q} \left( \sum_{n = 1}^\infty \ket{e_n} \bra{e_n} \right) \ket{\alpha} \\
                               & = \left( \sum_{n = 1}^\infty \hat{Q} \ket{e_n} \bra{e_n} \right) \ket{\alpha} \\
                               & = \left( \sum_{n = 1}^\infty q_n \ket{e_n} \bra{e_n} \right) \ket{\alpha}     \\
          \hat{Q}              & = \sum_{n = 1}^\infty q_n \ket{e_n} \bra{e_n}
        \end{align*}

  \item

        \begin{align*}
          \hat{Q}     & = \sum_{n = 1}^\infty q_n \ket{e_n} \bra{e_n}                                                                           \\
          \hat{Q}^2   & = \left( \sum_{n = 1}^\infty q_n \ket{e_n} \bra{e_n} \right) \left( \sum_{l = 1}^\infty q_l \ket{e_l} \bra{e_l} \right) \\
                      & = \sum_{n = 1}^\infty \sum_{l = 0}^\infty q_n q_l \ket{e_n} \braket{e_n | e_l} \bra{e_l}                                \\
                      & = \sum_{n = 1}^\infty q_n^2 \ket{e_n} \bra{e_n}                                                                         \\
          e^{\hat{Q}} & = \sum_{n = 1}^\infty e^{q_n} \ket{e_n} \bra{e_n}                                                                       \\
                      & = \sum_{n = 1}^\infty \left( \sum_{k = 0}^\infty \frac{q_n^k}{k!} \right) \ket{e_n} \bra{e_n}                           \\
                      & = \sum_{k = 0}^\infty \frac{1}{k!} \left( \sum_{n = 1}^\infty q_n^k \ket{e_n} \bra{e_n} \right)                         \\
                      & = \sum_{k = 0}^\infty \frac{1}{k!} \hat{Q}^k                                                                            \\
                      & = 1 + \hat{Q} + \frac{1}{2} \hat{Q}^2 + \frac{1}{3!} \hat{Q}^3 + \cdots
        \end{align*}
\end{enumerate}

\subsection{}

\begin{enumerate}
  \item

        \begin{align*}
          \sin \hat{D}       & = \hat{D} - \frac{D^3}{3!} + \frac{\hat{D}^5}{5!} - \frac{\hat{D}^7}{7!} + \cdots \\
          (\sin \hat{D}) x^5 & = 5 x^4 - 10 x^2 + 1
        \end{align*}

  \item

        \begin{align*}
          \frac{1}{1 - \hat{D} / 2}        & = 1 + \frac{\hat{D}}{2} + \left( \frac{\hat{D}}{2} \right)^2 + \left( \frac{\hat{D}}{2} \right)^3 + \cdots                                 \\
                                           & = 1 + \frac{\hat{D}}{2} + \frac{\hat{D}^2}{4} + \frac{\hat{D}^3}{8} + \cdots                                                               \\
          \frac{1}{1 - \hat{D} / 2} \cos x & = \cos x - \frac{1}{2} \sin x - \frac{1}{4} \cos x + \frac{1}{8} \sin x + \cdots                                                           \\
                                           & = \left( -\frac{1}{2} + \frac{1}{8} - \frac{1}{32} + \cdots \right) \sin x + \left( 1 - \frac{1}{4} + \frac{1}{16} - \cdots \right) \cos x \\
                                           & = -\frac{2}{5} \sin x + \frac{4}{5} \cos x
        \end{align*}
\end{enumerate}

\setcounter{subsection}{29}
\subsection{}

\begin{align*}
  c_n(t) & = \braket{n | S(t)}                            \\
         & = \Braket{n | \int d x \Ket{x} \Bra{x} | S(t)} \\
         & = \int \braket{n | x} \braket{x | S(t)} \,d x  \\
         & = \int \braket{x | n}^* \Psi(x, t) \,d x       \\
         & = \int \psi_n(x)^* \Psi(x, t) \,d x
\end{align*}

\subsection{}

\begin{align*}
  \ket{e_1}           & = 1                                                        \\
  \braket{e_1 | e_1}  & = \int_{-1}^1 \,d x                                        \\
                      & = 2                                                        \\
  \ket{e_1'}          & = \frac{1}{\sqrt{2}}                                       \\
  \ket{e_2}           & = x                                                        \\
  \braket{e_1' | e_2} & = \int_{-1}^1 \frac{1}{\sqrt{2}} x \,d x                   \\
                      & = \frac{1}{\sqrt{2}} \left[ \frac{1}{2} x^2 \right]_{-1}^1 \\
                      & = 0                                                        \\
  \braket{e_2 | e_2}  & = \int_{-1}^1 x^2 \,d x                                    \\
                      & = \left[ \frac{1}{3} x^3 \right]_{-1}^1                    \\
                      & = \frac{2}{3}                                              \\
  \ket{e_2'}          & = \sqrt{\frac{3}{2}} x                                     \\
  \ket{e_3}           & = x^2                                                      \\
  \braket{e_1' | e_3} & = \int_{-1}^1 \frac{1}{\sqrt{2}} x^2 \,d x                 \\
                      & = \frac{1}{\sqrt{2}} \left[ \frac{1}{3} x^3 \right]_{-1}^1 \\
                      & = \frac{\sqrt{2}}{3}                                       \\
  \braket{e_2' | e_3} & = \int_{-1}^1 \sqrt{\frac{3}{2}} x^3 \,d x                 \\
                      & = 0                                                        \\
  \ket{e_3'}          & = x^2 - \frac{\sqrt{2}}{3} \ket{e_1'}                      \\
                      & = x^2 - \frac{1}{3}
\end{align*}

\begin{align*}
  \braket{e_3' | e_3'} & = \int_{-1}^1 \left( x^2 - \frac{1}{3} \right)^2 \,d x                                  \\
                       & = \int_{-1}^1 \left( x^4 - \frac{2}{3} x^2 + \frac{1}{9} \right) \,d x                  \\
                       & = \left[ \frac{1}{5} x^5 - \frac{2}{9} x^3 + \frac{1}{9} x \right]_{-1}^1               \\
                       & = \frac{2}{5} - \frac{4}{9} + \frac{2}{9}                                               \\
                       & = \frac{18}{45} - \frac{20}{45} + \frac{10}{45}                                         \\
                       & = \frac{8}{45}                                                                          \\
  \ket{e_3''}          & = \sqrt{\frac{45}{8}} \left( x^2 - \frac{1}{3} \right)                                  \\
                       & = \sqrt{\frac{5}{2}} \left( \frac{3}{2} x^2 - \frac{1}{2} \right)                       \\
  \ket{e_4}            & = x^3                                                                                   \\
  \braket{e_1' | e_4}  & = \int_{-1}^1 \frac{1}{\sqrt{2}} x^3 \,d x                                              \\
                       & = 0                                                                                     \\
  \braket{e_2' | e_4}  & = \int_{-1}^1 \sqrt{\frac{3}{2}} x^4 \,d x                                              \\
                       & = \sqrt{\frac{3}{2}} \left[ \frac{1}{5} x^5 \right]                                     \\
                       & = \frac{\sqrt{6}}{5}                                                                    \\
  \braket{e_3'' | e_4} & = \int_{-1}^1 \sqrt{\frac{5}{2}} \left( \frac{3}{2} x^2 - \frac{1}{2} \right) x^3 \,d x \\
                       & = \sqrt{\frac{5}{2}} \int_{-1}^1 \left( \frac{3}{2} x^5 - \frac{1}{2} x^3 \right) \,d x \\
                       & = 0                                                                                     \\
  \ket{e_4'}           & = \ket{e_4} - \frac{\sqrt{6}}{5} \ket{e_2'}                                             \\
                       & = x^3 - \frac{3}{5} x
\end{align*}

\begin{align*}
  \braket{e_4' | e_4'} & = \int_{-1}^1 \left( x^3 - \frac{3}{5} x \right)^2 \,d x                      \\
                       & = \int_{-1}^1 \left( x^6 - \frac{6}{5} x^4 + \frac{9}{25} x^2 \right) \,d x   \\
                       & = \left[ \frac{1}{7} x^7 - \frac{6}{25} x^5 + \frac{3}{25} x^3 \right]_{-1}^1 \\
                       & = \frac{2}{7} - \frac{12}{25} + \frac{6}{25}                                  \\
                       & = \frac{50}{175} - \frac{84}{175} + \frac{42}{175}                            \\
                       & = \frac{8}{175}                                                               \\
  \ket{e_4''}          & = \sqrt{\frac{175}{8}} \left( x^3 - \frac{3}{5} x \right)                     \\
                       & = \sqrt{\frac{7}{2}} \left( \frac{5}{2} x^3 - \frac{3}{2} x \right)
\end{align*}

\subsection{}

\begin{enumerate}
  \item

        \begin{align*}
          \braket{\hat{Q}} & = \braket{\Psi | \hat{Q} | \Psi}           \\
                           & = \braket{\Psi | \hat{Q}^\dagger | \Psi}^* \\
                           & = -\braket{\Psi | \hat{Q} | \Psi}^*        \\
                           & = -\braket{\hat{Q}}^*
        \end{align*}

  \item

        \begin{align*}
          \hat{Q} \ket{\psi}                     & = \lambda \ket{\psi}               \\
          \braket{\psi | \hat{Q} | \psi}         & = \braket{\psi | \lambda | \psi}   \\
                                                 & = \lambda \braket{\psi | \psi}     \\
          \braket{\psi | \hat{Q} | \psi}^*       & = \lambda^* \braket{\psi | \psi}^* \\
          \braket{\psi | \hat{Q}^\dagger | \psi} & = \lambda^* \braket{\psi | \psi}   \\
          -\braket{\psi | \hat{Q} | \psi}        & = \lambda^* \braket{\psi | \psi}   \\
          \braket{\psi | \hat{Q} | \psi}         & = -\lambda^* \braket{\psi | \psi}  \\
          \hat{Q} \ket{\psi}                     & = -\lambda^* \ket{\psi}
        \end{align*}

  \item

        \begin{align*}
          \hat{Q} \ket{f}                  & = \lambda_1 \ket{f}           \\
          \hat{Q} \ket{g}                  & = \lambda_2 \ket{g}           \\
          \braket{g | \hat{Q} | f}         & = \braket{g | \lambda_1 | f}  \\
                                           & = \lambda_1 \braket{g | f}    \\
          \braket{f | \hat{Q}^\dagger | g} & = \lambda_1^* \braket{f | g}  \\
          -\braket{f | \hat{Q} | g}        & = \lambda_1^* \braket{f | g}  \\
          \braket{f | \lambda_2 | g}       & = -\lambda_1^* \braket{f | g} \\
          \lambda_2 \braket{f | g}         & = -\lambda_1^* \braket{f | g} \\
          \lambda_2                        & = -\lambda_1^*                \\
          \lambda_2                        & = \lambda_1                   \\
          \braket{f | g}                   & = 0
        \end{align*}

  \item

        \begin{align*}
          % Hermitian
          [\hat{A}, \hat{B}]         & = \hat{A} \hat{B} - \hat{B} \hat{A}                                 \\
          [\hat{A}, \hat{B}]^\dagger & = (\hat{A} \hat{B} - \hat{B} \hat{A})^\dagger                       \\
                                     & = \hat{A}^\dagger \hat{B}^\dagger - \hat{B}^\dagger \hat{A}^\dagger \\
                                     & = -(\hat{A} \hat{B} - \hat{B} \hat{A})                              \\
                                     & = -[\hat{A}, \hat{B}]                                               \\
          % Anti-hermitian
          [\hat{A}, \hat{B}]^\dagger & = \hat{A}^\dagger \hat{B}^\dagger - \hat{B}^\dagger \hat{A}^\dagger \\
                                     & = -(\hat{A} \hat{B} - \hat{B} \hat{A})                              \\
                                     & = -[\hat{A}, \hat{B}]
        \end{align*}

  \item

        \begin{align*}
          \hat{Q} \ket{q_n} & = \lambda_n \ket{q_n}                   \\
                            & = (x + i y) \ket{q_n}                   \\
                            & = x \ket{q_n} + i y \ket{q_n}           \\
                            & = \hat{X} \ket{q_n} + \hat{Y} \ket{q_n} \\
                            & = (\hat{X} + \hat{Y}) \ket{q_n}
        \end{align*}
\end{enumerate}

\subsection{}

\begin{enumerate}
  \item $\psi_1$

  \item $b_1$ and $b_2$ with $P(b_1) = \frac{9}{25}$ and $P(b_2) = \frac{16}{25}$.

  \item

        \begin{align*}
          \phi_1 & = \frac{3}{5} \psi_1 + \frac{4}{5} \psi_2                                   \\
          \phi_2 & = \frac{4}{5} \psi_1 - \frac{3}{5} \psi_2                                   \\
          P(a_1) & = P(b_1) \left( \frac{3}{5} \right)^2 + P(b_2) \left( \frac{4}{5} \right)^2 \\
                 & = \left( \frac{9}{25} \right)^2 + \left( \frac{16}{25} \right)^2            \\
                 & = \frac{337}{625}                                                           \\
                 & \approx 53.9\%
        \end{align*}
\end{enumerate}

\subsection{}

\begin{enumerate}
  \item

        \begin{align*}
          \Phi_n(p, t) & = \frac{1}{\sqrt{2 \pi \hbar}} \int_0^a e^{-i p x / \hbar} \sqrt{\frac{2}{a}} \sin \left( \frac{n \pi}{a} x \right) e^{-i E_n t / \hbar} \,d x \\
                       & = \frac{1}{\sqrt{\pi \hbar a}} e^{-i E_n t / \hbar} \int_0^a e^{-i p x / \hbar} \sin \left( \frac{n \pi}{a} x \right) \,d x                    \\
                       & = \sqrt{\frac{a \pi}{\hbar}} \frac{n e^{-i E_n t / \hbar}}{(n \pi)^2 - (a p / \hbar)^2} [1 - (-1)^n e^{-i p a / \hbar}]
        \end{align*}

  \item

        \begin{align*}
          |\Phi_n(p, t)|^2 & = \frac{a \pi}{\hbar} \frac{4 n^2}{[(n \pi)^2 - (a p / \hbar)^2]^2} \begin{cases}
                                                                                                   \cos^2 \left( \frac{a}{2 \hbar} p \right) & n \text{ odd}  \\
                                                                                                   \sin^2 \left( \frac{a}{2 \hbar} p \right) & n \text{ even}
                                                                                                 \end{cases}
        \end{align*}
\end{enumerate}

\subsection{}

\begin{align*}
  \Psi(x, 0) & = \begin{cases}
                   \frac{1}{\sqrt{2 n \lambda}} e^{i 2 \pi x / \lambda} & -n \lambda < x < n \lambda \\
                   0                                                    & \text{otherwise}
\end{cases}                                                                                         \\
  \Phi(p, 0) & = \int_{-\infty}^\infty \frac{1}{\sqrt{2 \pi \hbar}} e^{-i p x / \hbar} \Psi(x, 0) \,d x                                                                                          \\
             & = \frac{1}{2 \sqrt{\pi \hbar n \lambda}} \int_{-n \lambda}^{n \lambda} e^{i (2 \pi / \lambda - p / \hbar) x} \,d x                                                                \\
             & = \frac{1}{2 \sqrt{\pi \hbar n \lambda}} \frac{1}{i (2 \pi / \lambda - p / \hbar)} \left[ e^{i (2 \pi / \lambda - p / \hbar) x} \right]_{-n \lambda}^{n \lambda}                  \\
             & = \frac{1}{2 \sqrt{\pi \hbar n \lambda}} \frac{1}{i (2 \pi / \lambda - p / \hbar)} e^{i (2 \pi / \lambda - p / \hbar) n \lambda} - e^{-i (2 \pi / \lambda - p / \hbar) n \lambda} \\
             & = \frac{1}{\sqrt{\pi \hbar n \lambda}} \frac{1}{2 \pi / \lambda - p / \hbar} \sin \left[ \left( \frac{2 \pi}{\lambda} - \frac{p}{\hbar} \right) n \lambda \right]                 \\
             & = \sqrt{\frac{\lambda \hbar}{n \pi}} \frac{1}{\lambda p - 2 \pi \hbar} \sin \left( \frac{n \lambda}{\hbar} p \right)                                                              \\
  w_x        & = 2 n \lambda                                                                                                                                                                     \\
  w_p        & = \frac{2 \pi \hbar}{n \lambda}
\end{align*}

As $n \rightarrow \infty$, $w_x \rightarrow \infty$ and $w_p \rightarrow 0$.

\begin{align*}
  w_x w_p & = 2 n \lambda \frac{2 \pi \hbar}{n \lambda} \\
          & = 4 \pi \hbar                               \\
          & \ge \frac{\hbar}{2}
\end{align*}

\subsection{}

\begin{enumerate}
  \item

        \begin{align*}
          1 & = \int_{-\infty}^\infty \left( \frac{A}{x^2 + a^2} \right)^2 \,d x \\
            & = |A|^2 \int_{-\infty}^\infty \frac{1}{(x^2 + a^2)^2} \,d x        \\
            & = \frac{\pi |A|^2}{2 a^3}                                          \\
          A & = a \sqrt{\frac{2 a}{\pi}}
        \end{align*}

  \item

        \begin{align*}
          \braket{x}   & = \braket{\Psi | x | \Psi}                                                \\
                       & = \int_{-\infty}^\infty \Psi^* x \Psi \,d x                               \\
                       & = \frac{a^3}{\pi} \int_{-\infty}^\infty \frac{2 x}{(x^2 + a^2)^2} \,d x   \\
          u            & = x^2 + a^2                                                               \\
          d u          & = 2 x \,d x                                                               \\
          \braket{x}   & = \frac{a^3}{\pi} \int_{-\infty}^\infty \frac{1}{u^2}                     \\
                       & = \frac{a^3}{\pi} \left[ -\frac{1}{u} \right]_{-\infty}^\infty            \\
                       & = 0                                                                       \\
          \braket{x^2} & = \braket{\Psi | x^2 \Psi}                                                \\
                       & = \int_{-\infty}^\infty \Psi^* x^2 \Psi \,d x                             \\
                       & = \frac{2 a^3}{\pi} \int_{-\infty}^\infty \frac{x^2}{(x^2 + a^2)^2} \,d x \\
                       & = a^2                                                                     \\
          \sigma_x     & = \sqrt{\braket{x^2} - \braket{x}^2}                                      \\
                       & = a
        \end{align*}

  \item

        \begin{align*}
          \Phi(x, 0)                                                       & = \int_{-\infty}^\infty \frac{1}{\sqrt{2 \pi \hbar}} e^{-i p x / \hbar} a \sqrt{\frac{2 a}{\pi}} \frac{1}{x^2 + a^2} \,d x \\
                                                                           & = \frac{a \sqrt{a}}{\pi \sqrt{\hbar}} \int_{-\infty}^\infty e^{-i p x / \hbar} \frac{1}{x^2 + a^2} \,d x                   \\
                                                                           & = \sqrt{\frac{a}{\hbar}} e^{-|p| a / \hbar}                                                                                \\
          \frac{a}{\hbar} \int_{-\infty}^\infty e^{-2 |p| a / \hbar} \,d p & = 1
        \end{align*}

  \item

        \begin{align*}
          \braket{p}   & = \braket{\Phi | p | \Phi}                                             \\
                       & = \int_{-\infty}^\infty \Phi^* p \Psi \,d p                            \\
                       & = \frac{a}{\hbar} \int_{-\infty}^\infty p e^{-2 |p| a / \hbar} \,d p   \\
                       & = 0                                                                    \\
          \braket{p^2} & = \frac{a}{\hbar} \int_{-\infty}^\infty p^2 e^{-2 |p| a / \hbar} \,d p \\
                       & = \frac{\hbar^2}{2 a^2}                                                \\
          \sigma_p     & = \sqrt{\braket{p^2} - \braket{p}^2}                                   \\
                       & = \frac{\hbar}{\sqrt{2} a}
        \end{align*}

  \item

        \begin{align*}
          \sigma_x \sigma_p & = \frac{\hbar}{\sqrt{2}} \\
                            & \ge \frac{\hbar}{2}
        \end{align*}
\end{enumerate}

\subsection{}

\begin{align*}
  [\hat{H}, x p]             & = x [\hat{H}, p] + [\hat{H}, x] p                                            \\
                             & = i \hbar x \frac{d V}{d x} - \frac{i \hbar p^2}{m}                          \\
  \frac{d}{d t} \braket{x p} & = \frac{i}{\hbar} \braket{[\hat{H}, x p]}                                    \\
                             & = \frac{i}{\hbar} \Braket{i \hbar x \frac{d V}{d x} - \frac{i \hbar p^2}{m}} \\
                             & = \Braket{\frac{p^2}{m}} - \Braket{x \frac{d V}{d x}}                        \\
                             & = 2 \braket{T} - \Braket{x \frac{d V}{d x}}
\end{align*}

The left hand side is $0$ because expectation values are constant in stationary states.

\begin{align*}
  V                 & = \frac{1}{2} m \omega^2 x^2 \\
  \frac{d V}{d x}   & = m \omega^2 x               \\
  x \frac{d V}{d x} & = m \omega^2 x               \\
                    & = 2 V                        \\
  2 \braket{T}      & = \Braket{x \frac{d V}{d x}} \\
                    & = \braket{2 V}               \\
                    & = 2 \braket{V}               \\
  \braket{T}        & = \braket{V}
\end{align*}

\subsection{}

\begin{align*}
  \Psi(x, 0)             & = \frac{1}{\sqrt{2}} (\psi_1 + \psi_2)                                                                                                                                              \\
  \Psi(x, t)             & = \frac{1}{\sqrt{2}} (\psi_1 e^{-i E_1 t / \hbar} + \psi_2 e^{-i E_2 t / \hbar})                                                                                                    \\
  0                      & = \int_{-\infty}^\infty \Psi(x, 0) \Psi(x, t) \,d x                                                                                                                                 \\
                         & = \frac{1}{2} \int_{-\infty}^\infty (\psi_1 + \psi_2) (\psi_1 e^{-i E_1 t / \hbar} + \psi_2 e^{-i E_2 t / \hbar}) \,d x                                                             \\
                         & = \frac{1}{2} \int_{-\infty}^\infty (\psi_1^2 e^{-i E_1 t / \hbar} + \psi_1 \psi_2 e^{-i E_2 t / \hbar} + \psi_1 \psi_2 e^{-i E_1 t / \hbar} + \psi_2^2 e^{-i E_2 t / \hbar}) \,d x \\
                         & = \frac{1}{2} (e^{-i E_1 t / \hbar} + e^{-i E_2 t / \hbar})                                                                                                                         \\
                         & = e^{-i E_1 t / \hbar} + e^{-i E_2 t / \hbar}                                                                                                                                       \\
  e^{-i E_1 t / \hbar}   & = -e^{-i E_2 t / \hbar}                                                                                                                                                             \\
                         & = e^{\pi i} e^{-i E_2 t / \hbar}                                                                                                                                                    \\
  -\frac{i E_1 t}{\hbar} & = i \left( \pi - \frac{E_2 t}{\hbar} \right)                                                                                                                                        \\
  -\frac{E_1 t}{\hbar}   & = \pi - \frac{E_2 t}{\hbar}                                                                                                                                                         \\
  t                      & = \frac{\pi \hbar}{E_2 - E_1}                                                                                                                                                       \\
  \Delta t               & = \frac{t}{\pi}                                                                                                                                                                     \\
                         & = \frac{\hbar}{E_2 - E_1}                                                                                                                                                           \\
  \Delta E               & = \frac{1}{2} (E_2 - E_1)                                                                                                                                                           \\
  \Delta E \Delta t      & = \frac{\hbar}{2}
\end{align*}

\setcounter{subsection}{40}
\subsection{}

\begin{align*}
  \Psi                         & = \frac{e^{i \theta_0}}{\sqrt{2}} \psi_0 e^{-i \omega t / 2} + \frac{e^{i \theta_1}}{\sqrt{2}} \psi_1 e^{-3 i \omega t / 2}                                                     \\
  \braket{p}                   & = \braket{\Psi | p | \Psi}                                                                                                                                                      \\
                               & = \frac{1}{2} \braket{\psi_0 | p | \psi_0} + \frac{1}{2} e^{i (\theta_1 - \theta_0)} e^{-i \omega t} \braket{\psi_0 | p | \psi_1}                                               \\
                               & \qquad + \frac{1}{2} e^{i (\theta_0 - \theta_1)} e^{i \omega t} \braket{\psi_1 | p | \psi_0} + \frac{1}{2} \braket{\psi_1 | p | \psi_1}                                         \\
  \braket{\psi_0 | p | \psi_0} & = \frac{d}{d t} \braket{\psi_0 | x | \psi_0}                                                                                                                                    \\
                               & = 0                                                                                                                                                                             \\
  \braket{\psi_1 | p | \psi_1} & = \frac{d}{d t} \braket{\psi_1 | x | \psi_1}                                                                                                                                    \\
                               & = 0                                                                                                                                                                             \\
  \braket{p}                   & = \frac{1}{2} e^{i (\theta_1 - \theta_0 - \omega t)} \braket{\psi_0 | p | \psi_1} + \frac{1}{2} e^{-i (\theta_1 - \theta_0 - \omega t)} \braket{\psi_1 | p | \psi_0}            \\
  \braket{\psi_0 | p | \psi_1} & = -i \sqrt{\frac{\hbar m \omega}{2}}                                                                                                                                            \\
  \braket{\psi_1 | p | \psi_0} & = i \sqrt{\frac{\hbar m \omega}{2}}                                                                                                                                             \\
  \braket{p}                   & = -\frac{1}{2} i e^{i (\theta_1 - \theta_0 - \omega t)} \sqrt{\frac{\hbar m \omega}{2}} + \frac{1}{2} i e^{-i (\theta_1 - \theta_0 - \omega t)} \sqrt{\frac{\hbar m \omega}{2}} \\
                               & = -\frac{1}{2} i \sqrt{\frac{\hbar m \omega}{2}} (e^{i (\theta_1 - \theta_0 - \omega t)} - e^{-i (\theta_1 - \theta_0 - \omega t)})                                             \\
                               & = \sqrt{\frac{\hbar m \omega}{2}} \sin (\theta_1 - \theta_0 - \omega t)
\end{align*}

The largest possible value of $\braket{p}$ is $\sqrt{\hbar m \omega / 2}$. If it takes on this value at $t = 0$ then \[\Psi(x, t) = \frac{1}{\sqrt{2}} e^{-i \omega t / 2} (\psi_0 + i \psi_1 e^{-i \omega t}).\]

\setcounter{subsection}{42}
\subsection{}

\begin{enumerate}
  \item

        \begin{align*}
          |z|^2                 & = \Re(z)^2 + \Im(z)^2                                                                                                                                                 \\
                                & = \left( \frac{z + z^*}{2} \right)^2 + \left( \frac{z - z^*}{2 i} \right)^2                                                                                           \\
          \sigma_A^2 \sigma_B^2 & \ge \left( \frac{\braket{f | g} + \braket{g | f}}{2} \right)^2 + \left( \frac{\braket{f | g} - \braket{g | f}}{2 i} \right)^2                                         \\
                                & = \left( \frac{\braket{\hat{A} \hat{B}} + \braket{\hat{B} \hat{A}} - 2 \braket{A} \braket{B}}{2} \right)^2 + \left( \frac{\braket{[\hat{A}, \hat{B}]}}{2 i} \right)^2 \\
                                & = \frac{1}{4} [\braket{-i [\hat{A}, \hat{B}]}^2 + (\braket{\hat{A} \hat{B}} + \braket{\hat{B} \hat{A}} - 2 \braket{A} \braket{B})^2]                                  \\
                                & = \frac{1}{4} (\braket{C}^2 + \braket{D}^2)
        \end{align*}

  \item

        \begin{align*}
          B                     & = A                                                   \\
          \hat{C}               & = -i [\hat{A}, \hat{B}]                               \\
                                & = 0                                                   \\
          \hat{D}               & = 2 (\hat{A}^2 - \braket{A}^2)                        \\
          \braket{D}            & = \braket{\Psi | D | \Psi}                            \\
                                & = \braket{\Psi | 2 (\hat{A}^2 - \braket{A}^2) | \Psi} \\
                                & = 2 (\braket{\Psi | \hat{A}^2 | \Psi} - \braket{A}^2) \\
                                & = 2 (\braket{A^2} - \braket{A}^2)                     \\
                                & = 2 \sigma_A^2                                        \\
          \sigma_A^2 \sigma_B^2 & = \frac{1}{4} (\braket{C}^2 + \braket{D}^2)           \\
          \sigma_A^4            & \ge \sigma_A^4
        \end{align*}
\end{enumerate}

\subsection{}

\begin{enumerate}
  \item

        The eigenvalues of $\vec{H}$ are $a - b$, $a + b$, and $c$ and the associated eigenvectors are $\begin{pmatrix}
            -1 \\
            0  \\
            1
          \end{pmatrix}$, $\begin{pmatrix}
            1 \\
            0 \\
            1
          \end{pmatrix}$, and $\begin{pmatrix}
            0 \\
            1 \\
            0
          \end{pmatrix}$, respectively.

        \[\ket{S(t)} = \begin{pmatrix}
            0 \\
            1 \\
            0
          \end{pmatrix} e^{-i c t / \hbar}\]

  \item

        \begin{align*}
          \begin{pmatrix}
            1 \\
            0 \\
            0
          \end{pmatrix} & = \frac{1}{2} \left[ \begin{pmatrix}
                                                   1 \\
                                                   0 \\
                                                   1
                                                 \end{pmatrix} - \begin{pmatrix}
                                                                   -1 \\
                                                                   0  \\
                                                                   1
                                                                 \end{pmatrix} \right]                                                     \\
          \ket{S(t)}      & = \frac{1}{2} \left[ \begin{pmatrix}
                                                     1 \\
                                                     0 \\
                                                     1
                                                   \end{pmatrix} e^{-i (a + b) t / \hbar} - \begin{pmatrix}
                                                                                              -1 \\
                                                                                              0  \\
                                                                                              1
                                                                                            \end{pmatrix} e^{-i (a - b) t / \hbar} \right] \\
                          & = \frac{1}{2} e^{-i a t / \hbar} \begin{pmatrix}
                                                               e^{-i b t / \hbar} + e^{i b t / \hbar} \\
                                                               0                                      \\
                                                               e^{-i b t / \hbar} - e^{i b t / \hbar}
                                                             \end{pmatrix}                                      \\
                          & = e^{-i a t / \hbar} \begin{pmatrix}
                                                   \cos (b t / \hbar) \\
                                                   0                  \\
                                                   -i \sin (b t / \hbar)
                                                 \end{pmatrix}
        \end{align*}
\end{enumerate}

\subsection{}

\begin{align*}
  \braket{n | \hat{x} | S(t)} & = \Braket{n | \hat{x} \sum_{n' = 0}^\infty \Ket{n'} \Bra{n'} | S(t)}                                                          \\
                              & = \sum_{n' = 0}^\infty \braket{n | \hat{x} | n'} \braket{n' | S(t)}                                                           \\
                              & = \sqrt{\frac{\hbar}{2 m \omega}} \sum_{n' = 0}^\infty (\sqrt{n'} \delta_{n, n' - 1} + \sqrt{n} \delta_{n', n - 1}) c_{n'}(t) \\
                              & = \sqrt{\frac{\hbar}{2 m \omega}} [\sqrt{n + 1} c_{n + 1}(t) + \sqrt{n} c_{n - 1}(t)]
\end{align*}

\subsection{}

\begin{enumerate}
  \item

        \begin{align*}
          h_1       & = \hbar \omega                      \\
          h_2       & = 2 \hbar \omega                    \\
          h_3       & = 2 \hbar \omega                    \\
          \ket{h_1} & = \begin{pmatrix}
                          1 \\
                          0 \\
                          0
                        \end{pmatrix}                    \\
          \ket{h_2} & = \begin{pmatrix}
                          0 \\
                          1 \\
                          0
                        \end{pmatrix}                    \\
          \ket{h_3} & = \begin{pmatrix}
                          0 \\
                          0 \\
                          1
                        \end{pmatrix}                    \\
          a_1       & = 2 \lambda                         \\
          a_2       & = \lambda                           \\
          a_3       & = -\lambda                          \\
          \ket{a_1} & = \begin{pmatrix}
                          0 \\
                          0 \\
                          1
                        \end{pmatrix}                    \\
          \ket{a_2} & = \frac{1}{\sqrt{2}} \begin{pmatrix}
                                             1 \\
                                             1 \\
                                             0
                                           \end{pmatrix} \\
          \ket{a_3} & = \frac{1}{\sqrt{2}} \begin{pmatrix}
                                             -1 \\
                                             1  \\
                                             0
                                           \end{pmatrix} \\
          b_1       & = 2 \mu                             \\
          b_2       & = \mu                               \\
          b_3       & = -\mu                              \\
          \ket{b_1} & = \begin{pmatrix}
                          1 \\
                          0 \\
                          0
                        \end{pmatrix}                    \\
          \ket{b_2} & = \frac{1}{\sqrt{2}} \begin{pmatrix}
                                             0 \\
                                             1 \\
                                             1
                                           \end{pmatrix} \\
          \ket{b_3} & = \frac{1}{\sqrt{2}} \begin{pmatrix}
                                             0  \\
                                             -1 \\
                                             1
                                           \end{pmatrix}
        \end{align*}

  \item

        \begin{align*}
          \braket{H} & = \braket{S(0) | H | S(0)}                                \\
                     & = \begin{pmatrix}
                           c_1^* & c_2^* & c_3^*
                         \end{pmatrix} \hbar \omega \begin{pmatrix}
                                                      1 & 0 & 0 \\
                                                      0 & 2 & 0 \\
                                                      0 & 0 & 2
                                                    \end{pmatrix} \begin{pmatrix}
                                                                    c_1 \\
                                                                    c_2 \\
                                                                    c_3
                                                                  \end{pmatrix} \\
                     & = \hbar \omega \begin{pmatrix}
                                        c_1^* & c_2^* & c_3^*
                                      \end{pmatrix} \begin{pmatrix}
                                                      c_1   \\
                                                      2 c_2 \\
                                                      2 c_3
                                                    \end{pmatrix}               \\
                     & = \hbar \omega (|c_1|^2 + 2 |c_2|^2 + 2 |c_3|^2)          \\
          \braket{A} & = \lambda (c_1^* c_2 + c_2^* c_1 + 2 |c_3|^2)             \\
          \braket{B} & = \mu (2 |c_1|^2 + c_2^* c_3 + c_3^* c_2)
        \end{align*}

  \item

        \[\ket{S(t)} = c_1 \ket{h_1} e^{-i \omega t} + c_2 \ket{h_2} e^{-2 i \omega t} + c_3 \ket{h_3} e^{-2 i \omega t}\]

        You could measure $H$ as $\hbar \omega$ with probability $|c_1|^2$ or $2 \hbar \omega$ with probability $|c_2|^2 + |c_3|^2$.
\end{enumerate}

\section{Quantum Mechanics in Three Dimensions}

\subsection{}

\begin{enumerate}
  \item

        \begin{align*}
          [r_i, r_j] & = 0                                                                                                                   \\
          [p_i, p_j] & = 0                                                                                                                   \\
          [x, p_x] f & = x \left( -i \hbar \frac{\partial}{\partial x} \right) f - \left( -i \hbar \frac{\partial}{\partial x} \right) (x f) \\
                     & = -i \hbar x \frac{\partial f}{\partial x} + i \hbar \left( f + x \frac{\partial f}{\partial x} \right)               \\
                     & = i \hbar f                                                                                                           \\
          [p_x, x]   & = \left( -i \hbar \frac{\partial}{\partial x} \right) (x f) - x \left( -i \hbar \frac{\partial}{\partial x} \right) f \\
                     & = -i \hbar \left( f + x \frac{\partial f}{\partial x} \right) + i \hbar x \frac{\partial f}{\partial x}               \\
                     & = -i \hbar f                                                                                                          \\
          [x, p_y]   & = x \left( -i \hbar \frac{\partial}{\partial y} \right) f - \left( -i \hbar \frac{\partial}{\partial y} \right) (x f) \\
                     & = -i \hbar x \frac{\partial f}{\partial y} + i \hbar x \frac{\partial f}{\partial y}                                  \\
                     & = 0                                                                                                                   \\
          [r_i, p_j] & = i \hbar \delta_{i j}                                                                                                \\
          [p_i, r_j] & = -i \hbar \delta_{i j}
        \end{align*}

  \item

        \begin{align*}
          [\hat{H}, x] f                 & = \left( -\frac{\hbar^2}{2 m} \nabla^2 + V \right) (x f) - x \left( -\frac{\hbar^2}{2 m} \nabla^2 + V \right) f                                                                                                                                                                \\
                                         & = -\frac{\hbar^2}{2 m} \left[ \frac{\partial}{\partial x} \left( f + x \frac{\partial f}{\partial x} \right) + \frac{\partial}{\partial y} \left( x \frac{\partial f}{\partial y} \right) + \frac{\partial}{\partial x} \left( z \frac{\partial f}{\partial z} \right) \right] \\
                                         & \qquad + V x f + \frac{\hbar^2}{2 m} x \nabla^2 f - V x f                                                                                                                                                                                                                      \\
                                         & = -\frac{\hbar^2}{2 m} \left( 2 \frac{\partial f}{\partial x} + x \nabla^2 f \right) + \frac{\hbar^2}{2 m} x \nabla^2 f                                                                                                                                                        \\
                                         & = -\frac{\hbar^2}{m} \frac{\partial f}{\partial x}                                                                                                                                                                                                                             \\
          [\hat{H}, x]                   & = -\frac{\hbar^2}{m} \frac{\partial}{\partial x}                                                                                                                                                                                                                               \\
          [\hat{H}, \vec{r}]             & = -\frac{\hbar^2}{m} \nabla                                                                                                                                                                                                                                                    \\
          \frac{d}{d t} \braket{\vec{r}} & = \frac{i}{\hbar} \braket{[\hat{H}, \vec{r}]}                                                                                                                                                                                                                                  \\
                                         & = \frac{1}{m} \braket{-i \hbar \nabla}                                                                                                                                                                                                                                         \\
                                         & = \frac{1}{m} \braket{\vec{p}}                                                                                                                                                                                                                                                 \\
          [\hat{H}, \hat{p}_x] f         & = \left( -\frac{\hbar^2}{2 m} \nabla^2 + V \right) \left( -i \hbar \frac{\partial}{\partial x} \right) f - \left( -i \hbar \frac{\partial}{\partial x} \right) \left( -\frac{\hbar^2}{2 m} \nabla^2 + V \right) f                                                              \\
                                         & = -i \hbar \left( -\frac{\hbar^2}{2 m} \nabla^2 + V \right) \frac{\partial f}{\partial x} + i \hbar \left( \frac{\partial}{\partial x} \right) \left( -\frac{\hbar^2}{2 m} \nabla^2 f + V f \right)                                                                            \\
                                         & = i \frac{\hbar^3}{2 m} \nabla^2 \frac{\partial f}{\partial x} - i \hbar V \frac{\partial f}{\partial x} - i \frac{\hbar^3}{2 m} \nabla^2 \frac{\partial f}{\partial x} + i \hbar \left( \frac{\partial V}{\partial x} f + V \frac{\partial f}{\partial x} \right)             \\
                                         & = i \hbar \frac{\partial V}{\partial x} f                                                                                                                                                                                                                                      \\
          [\hat{H}, \hat{p}_x]           & = i \hbar \frac{\partial V}{\partial x}                                                                                                                                                                                                                                        \\
          [\hat{H}, \vec{p}]             & = i \hbar \nabla V                                                                                                                                                                                                                                                             \\
          \frac{d}{d t} \braket{\vec{p}} & = \frac{i}{\hbar} \braket{[\hat{H}, \vec{p}]}                                                                                                                                                                                                                                  \\
                                         & = \braket{-\nabla V}
        \end{align*}

  \item

        \begin{align*}
          \sigma_{r_i} \sigma_{p_j} & \ge \frac{1}{2 i} \braket{[r_i, p_j]} \\
                                    & = \frac{1}{2 i} i \hbar \delta_{i j}  \\
                                    & = \frac{\hbar}{2} \delta_{i j}
        \end{align*}
\end{enumerate}

\subsection{}

\begin{enumerate}
  \item

        \begin{align*}
          \psi(\vec{r})                                                                                                                                             & = X(x) Y(y) Z(z)         \\
          -\frac{\hbar^2}{2 m} \nabla^2 \psi                                                                                                                        & = E \psi                 \\
          -\frac{\hbar^2}{2 m} \left( \frac{\partial^2 X}{\partial x^2} Y Z + X \frac{\partial^2 Y}{\partial y^2} Z + X Y \frac{\partial^2 Z}{\partial z^2} \right) & = E X Y Z                \\
          \frac{X''}{X} + \frac{Y''}{Y} + \frac{Z''}{Z}                                                                                                             & = -\frac{2 m}{\hbar^2} E
        \end{align*}

        The terms on the left hand side are each functions of a different variable, so they must be constant. Starting with the $X$ term:

        \begin{align*}
          \frac{X''}{X}  & = -\alpha \\
          X'' + \alpha X & = 0
        \end{align*}

        If $\alpha < 0$

        \[X = A_x e^{\sqrt{-\alpha} x} + B_x e^{-\sqrt{-\alpha x}}\]

        If $\alpha = 0$

        \[X = A_x x + B_x\]

        If $\alpha > 0$

        \[X = A_x \sin (\sqrt{\alpha} x) + B_x \cos (\sqrt{\alpha} x)\]

        Boundary conditions require that $\alpha > 0$. $X(0) = 0$ so $B_x = 0$. $X(a) = 0$ so $\sqrt{\alpha} = n_x \pi / a \Rightarrow \alpha = n_x^2 \pi^2 / a^2$, \[X = A_x \sin \left( \frac{n_x \pi}{a} x \right).\]

        Repeating the above for $Y$ and $Z$ finds \begin{align*}
          Y & = A_y \sin \left( \frac{n_y \pi}{a} y \right) \\
          Z & = A_z \sin \left( \frac{n_z \pi}{a} z \right)
        \end{align*} so \[\psi(\vec{r}) = A_x A_y A_z \sin \left( \frac{n_x \pi}{a} x \right) \sin \left( \frac{n_y \pi}{a} y \right) \sin \left( \frac{n_z \pi}{a} z \right).\]

        Assuming $A_x = A_y = A_z = A$ and normalising finds \begin{align*}
          1   & = \int_0^a \int_0^a \int_0^a A^6 \sin^2 \left( \frac{n_x \pi}{a} x \right) \sin^2 \left( \frac{n_y \pi}{a} y \right) \sin^2 \left( \frac{n_z \pi}{a} z \right) \,d^3 \vec{r} \\
              & = A^6 \frac{a^3}{8}                                                                                                                                                          \\
          A^6 & = \frac{8}{a^3}                                                                                                                                                              \\
              & = \left( \frac{2}{a} \right)^3                                                                                                                                               \\
          A   & = \sqrt{\frac{2}{a}}
        \end{align*} so \[\psi(\vec{r}) = \left( \frac{2}{a} \right)^{3 / 2} \sin \left( \frac{n_x \pi}{a} x \right) \sin \left( \frac{n_y \pi}{a} y \right) \sin \left( \frac{n_z \pi}{a} z \right).\]

        Finally \begin{align*}
          -\frac{2 m}{\hbar^2} E & = -\alpha - \beta - \gamma                                                                 \\
                                 & = -\frac{\pi^2 n_x^2}{a^2} - \frac{\pi^2 n_y^2}{a^2} - \frac{\pi^2 n_z^2}{a^2}             \\
          E                      & = \frac{\pi^2 \hbar^2}{2 m a^2} (n_x^2 + n_y^2 + n_z^2), \,n_x, n_y, n_z = 1, 2, 3, \ldots
        \end{align*}

  \item

  \begin{align*}
    E_1 &= 3 \frac{\pi^2 \hbar^2}{2 m a^2}, \,n = 1 \\
    E_2 &= 6 \frac{\pi^2 \hbar^2}{2 m a^2}, \,n = 3 \\
    E_3 &= 9 \frac{\pi^2 \hbar^2}{2 m a^2}, \,n = 3 \\
    E_4 &= 11 \frac{\pi^2 \hbar^2}{2 m a^2}, \,n = 3 \\
    E_5 &= 12 \frac{\pi^2 \hbar^2}{2 m a^2}, \,n = 1 \\
    E_6 &= 14 \frac{\pi^2 \hbar^2}{2 m a^2}, \,n = 6 \\
  \end{align*}

  \item

  \begin{align*}
    E_{14} &= 27 \frac{\pi^2 \hbar^2}{2 m a^2} \\
    d &= 4
  \end{align*}
\end{enumerate}

\subsection{}

\begin{enumerate}
  \item 

  \begin{align*}
    \phi(r, \theta, \phi) &= A e^{-r / a} \\
    \frac{\partial \psi}{\partial r} &= -\frac{A}{a} e^{-r / a} \\
    \frac{\partial}{\partial r} \left( r^2 \frac{\partial \psi}{\partial r} \right) &= \frac{\partial}{\partial r} \left( -\frac{A}{a} r^2 e^{-r / a} \right) \\
    &= -\frac{A}{a} \left( 2 r e^{-r / a} - \frac{1}{a} r^2 e^{-r / a} \right) \\
    &= \frac{A}{a^2} r^2 \left( 1 - \frac{2 a}{r} \right) e^{-r / a} \\
    \frac{\partial \psi}{\partial \theta} &= 0 \\
    \frac{\partial^2 \psi}{\partial \phi^2} &= 0 \\
    -\frac{\hbar^2}{2 m} \frac{A}{a^2} \left( 1 - \frac{2 a}{r} \right) e^{-r / a} + V(r) A e^{-r / a} &= E A e^{-r / a} \\
    -\frac{\hbar^2}{2 m a^2} \left( 1 - \frac{2 a}{r} \right) + V(r) &= E \\
    -\frac{\hbar^2}{2 m a^2} &= E \\
    V(r) &= -\frac{\hbar^2}{m a r}
  \end{align*}

  \item

  \begin{align*}
    \phi(r, \theta, \phi) &= A e^{-r^2 / a^2} \\
    \frac{\partial \psi}{\partial r} &= -\frac{2}{a^2} r A e^{-r^2 / a^2} \\
    \frac{\partial}{\partial r} \left( r^2 \frac{\partial \psi}{\partial r} \right) &= \frac{\partial}{\partial r} \left( -\frac{2}{a^2} r^3 A e^{-r^2 / a^2} \right) \\
    &= -\frac{2 A}{a^2} \left( 3 r^2 e^{-r^2 / a^2} - \frac{2}{a^2} r^4 e^{-r^2 / a^2} \right) \\
    &= \frac{2 A}{a^2} r^2 \left( \frac{2}{a^2} r^2 - 3 \right) e^{-r^2 / a^2}
  \end{align*}

  \begin{align*}
    -\frac{\hbar^2}{2 m} \frac{2 A}{a^2} \left( \frac{2}{a^2} r^2 - 3 \right) e^{-r^2 / a^2} + V(r) A e^{-r^2 / a^2} &= E A e^{-r^2 / a^2} \\
    -\frac{\hbar^2}{2 m} \frac{2}{a^2} \left( \frac{2}{a^2} r^2 - 3 \right) + V(r) &= E \\
    \frac{\hbar^2}{2 m} \frac{2}{a^2} 3 &= E \\
    \frac{3 \hbar^2}{m a^2} &= E \\
    \frac{3 \hbar^2}{m a^2} + \frac{\hbar^2}{2 m} \frac{2}{a^2} \left( \frac{2}{a^2} r^2 - 3 \right) &= V(r) \\
    \frac{2 \hbar^2}{m a^4} r^2 &= V(r) \\
  \end{align*}
\end{enumerate}

\subsection{}

\begin{align*}
  Y_0^0                           & = \frac{1}{2 \sqrt{\pi}}                                                                                                                              \\
  Y_2^1                           & = \frac{1}{2} \sqrt{\frac{5}{6 \pi}} e^{i \phi} P_2^1(\cos \theta)                                                                                    \\
                                  & = \frac{1}{2} \sqrt{\frac{5}{6 \pi}} e^{i \phi} (-1)^1 (1 - \cos^2 \theta)^{1 / 2} \left( \frac{d}{d x} \right) P_2(\cos \theta)                      \\
                                  & = -\frac{1}{2} \sqrt{\frac{5}{6 \pi}} e^{i \phi} \sin \theta \left( \frac{d \theta}{d x} \frac{d}{d \theta} \right) \frac{1}{2} (3 \cos^2 \theta - 1) \\
                                  & = -\frac{1}{4} \sqrt{\frac{5}{6 \pi}} e^{i \phi} \sin \theta \left( -\frac{1}{\sin \theta} \right) (-6 \cos \theta \sin \theta)                       \\
                                  & = -\frac{1}{2} \sqrt{\frac{15}{2 \pi}} e^{i \phi} \sin \theta \cos \theta                                                                             \\
  \int |Y_0^0|^2 \,d \Omega       & = \frac{1}{4 \pi} \int_0^\pi \int_0^{2 \pi} \sin \theta \,d \theta \,d \phi                                                                           \\
                                  & = 1                                                                                                                                                   \\
  \int |Y_2^1|^2 \,d \Omega       & = \frac{15}{8 \pi} \int_0^\pi \int_0^{2 \pi} \sin^3 \theta \cos^2 \theta \,d \theta \,d \phi                                                          \\
                                  & = 1                                                                                                                                                   \\
  \int (Y_0^0)^* Y_2^1 \,d \Omega & = -\frac{1}{2 \sqrt{\pi}} \frac{1}{2} \sqrt{\frac{15}{2 \pi}} \int_0^\pi \int_0^{2 \pi} e^{i \phi} \sin^2 \theta \cos \theta \,d \theta \,d \phi      \\
                                  & = 0
\end{align*}

\subsection{}

\begin{align*}
  \Theta(\theta) & = A \ln \left( \tan \frac{\theta}{2} \right)                                                                                                                                                           \\
  0              & = \sin \theta \frac{d}{d \theta} \left( \sin \theta \frac{d \Theta}{d \theta} \right)                                                                                                                  \\
                 & = \frac{1}{2} A \sin \theta \frac{d}{d \theta} \left( \sin \theta \csc \frac{\theta}{2} \sec \frac{\theta}{2} \right)                                                                                  \\
                 & = \frac{1}{2} A \sin \theta \left( \cos \theta \csc \frac{\theta}{2} \sec \frac{\theta}{2} - \frac{1}{2} \csc^2 \frac{\theta}{2} \sin \theta + \frac{1}{2} \sec^2 \frac{\theta}{2} \sin \theta \right) \\
                 & = 0
\end{align*}

It's not a valid physical solution because it blows up at $\theta = 0$ and $\theta = \pi$.

\subsection{}

\begin{align*}
  Y_\ell^{-m} & = \sqrt{\frac{(2 \ell + 1)}{4 \pi} \frac{(\ell + m)!}{(\ell - m)!}} e^{-i m \phi} P_\ell^{-m} (\cos \theta)                                    \\
              & = \sqrt{\frac{(2 \ell + 1)}{4 \pi} \frac{(\ell + m)!}{(\ell - m)!}} e^{-i m \phi} (-1)^m \frac{(\ell - m)!}{(\ell + m)!} P_\ell^m(\cos \theta) \\
              & = (-1)^m \sqrt{\frac{(2 \ell + 1)}{4 \pi} \frac{(\ell - m)!}{(\ell + m)!}} e^{-i m \phi} P_\ell^m(\cos \theta)                                 \\
              & = (-1)^m (Y_\ell^m)^*
\end{align*}

\subsection{}

\begin{align*}
  Y_3^2(\theta, \phi)       & = \sqrt{\frac{7}{480 \pi}} e^{2 i \phi} P_3^2 (\cos \theta)                                                                                                \\
                            & = \frac{1}{4} \sqrt{\frac{105}{2 \pi}} e^{2 i \phi} \sin^2 \theta \cos \theta                                                                              \\
  Y_\ell^\ell(\theta, \phi) & = \sqrt{\frac{2 \ell + 1}{4 \pi (2 \ell)!}} e^{i \ell \phi} P_\ell^\ell(\cos \theta)                                                                       \\
  P_\ell^\ell(x)            & = (-1)^\ell (1 - x^2)^{\ell / 2} \left( \frac{d}{d x} \right)^\ell P_\ell (x)                                                                              \\
                            & = (-1)^\ell (1 - x^2)^{\ell / 2} \left( \frac{d}{d x} \right)^\ell \left[ \frac{1}{2^\ell \ell!} \left( \frac{d}{d x} \right)^\ell (x^2 - 1)^\ell  \right] \\
                            & = (-1)^\ell \frac{1}{2^\ell \ell!} (1 - x^2)^{\ell / 2} \left( \frac{d}{d x} \right)^{2 \ell} (x^2 - 1)^\ell                                               \\
                            & = (-1)^\ell \frac{(2 \ell)!}{2^\ell \ell!} (1 - x^2)^{\ell / 2}                                                                                            \\
  Y_\ell^\ell(\theta, \phi) & = \sqrt{\frac{2 \ell + 1}{4 \pi (2 \ell)!}} e^{i \ell \phi} (-1)^\ell \frac{(2 \ell)!}{2^\ell \ell!} (1 - \cos^2 \theta)^{\ell / 2}                                  \\
                            & = \frac{1}{\ell!} \sqrt{\frac{(2 \ell + 1)!}{4 \pi}} \left( -\frac{1}{2} e^{i \phi} \sin \theta \right)^\ell
\end{align*}

\setcounter{subsection}{8}
\subsection{}

\begin{enumerate}
  \item

        \begin{align*}
          n_1(x) & = -(-x) \left( \frac{1}{x} \frac{d}{d x} \right) \frac{\cos x}{x}                                                                \\
                 & = -\frac{\sin x}{x} - \frac{\cos x}{x^2}                                                                                         \\
          n_2(x) & = -(-x)^2 \left( \frac{1}{x} \frac{d}{d x} \right)^2 \frac{\cos x}{x}                                                            \\
                 & = -x^2 \left( \frac{1}{x} \frac{d}{d x} \right) \left[ \frac{1}{x} \frac{d}{d x} \left( \frac{\cos x}{x} \right) \right]         \\
                 & = -x^2 \left( \frac{1}{x} \frac{d}{d x} \right) \left[ \frac{1}{x} \left( -\frac{\sin x}{x} - \frac{\cos x}{x^2} \right) \right] \\
                 & = x \frac{d}{d x} \left( \frac{\sin x}{x^2} + \frac{\cos x}{x^3} \right)                                                         \\
                 & = x \left( \frac{\cos x}{x^2} - \frac{2 \sin x}{x^3} - \frac{\sin x}{x^3} - \frac{3 \cos x}{x^4} \right)                         \\
                 & = \left( \frac{1}{x} - \frac{3}{x^3} \right) \cos x - \frac{3 \sin x}{x^2}
        \end{align*}

  \item

        \begin{align*}
          n_1(x) & \approx -\frac{x}{x} - \frac{1}{x^2}                                 \\
                 & = -1 - \frac{1}{x^2}                                                 \\
          n_2(x) & \approx \left( \frac{1}{x} - \frac{3}{x^3} \right) - \frac{3}{x^2} x \\
                 & = -\frac{2}{x} - \frac{3}{x^3}
        \end{align*}
\end{enumerate}

\subsection{}

\begin{enumerate}
  \item

        \begin{align*}
          -\frac{\hbar^2}{2 m} \frac{d^2 u}{d r^2} + \frac{\hbar^2}{2 m} \frac{2}{r^2} u & = E u                                                                    \\
          \frac{d^2 u}{d r^2} - \frac{2}{r^2} u                                          & = -\frac{2 m E}{\hbar^2} u                                               \\
          \frac{d^2 u}{d r^2}                                                            & = \left( \frac{2}{r^2} - k^2 \right) u                                   \\
          j_1(k r)                                                                       & = \frac{\sin (k r)}{(k r)^2} - \frac{\cos (k r)}{k r}                    \\
          u                                                                              & = A r j_1(k r)                                                           \\
                                                                                         & = A r \left[ \frac{\sin (k r)}{(k r)^2} - \frac{\cos (k r)}{k r} \right]
        \end{align*}

        \begin{align*}
          \frac{d u}{d r}     & = A \left[ \frac{\sin (k r)}{(k r)^2} - \frac{\cos (k r)}{k r} \right]                                                                        \\
                              & \qquad + A r \left[ \frac{k \cos (k r)}{(k r)^2} - \frac{2 \sin (k r)}{k^2 r^3} + \frac{k \sin (k r)}{k r} + \frac{\cos (k r)}{k r^2} \right] \\
                              & = \frac{A}{k r} \left[ \frac{\sin (k r)}{k r} - \cos (k r) \right]                                                                            \\
                              & \qquad + \frac{A}{k} \left[ \frac{2 \cos (k r)}{r} - \frac{2 \sin (k r)}{k r^2} + k \sin (k r) \right]                                        \\
          \frac{d^2 u}{d r^2} & = \frac{A (-2 + k^2 r^2) (k r \cos (k r) - \sin (k r))}{k^2 r^3}                                                                              \\
                              & = \frac{2 - k^2 r^2}{r^2} A r \left[ \frac{\sin (k r)}{(k r)^2} - \frac{\cos (k r)}{k r} \right]                                              \\
                              & = \left( \frac{2}{r^2} - k^2 \right) u
        \end{align*}

  \item

        \begin{align*}
          0                                  & = \frac{\sin x}{x^2} - \frac{\cos x}{x}                                              \\
                                             & = \frac{\sin x}{x} - \cos x                                                          \\
          \cos x                             & = \frac{\sin x}{x}                                                                   \\
          x                                  & = \tan x                                                                             \\
          x                                  & \approx \left( N + \frac{1}{2} \right) \pi, \,n \in \mathbb{Z}                       \\
          k a                                & = x                                                                                  \\
          \frac{\sqrt{2 m E_{N 1}}}{\hbar} a & \approx \left( N + \frac{1}{2} \right) \pi                                           \\
          E_{N 1}                            & = \frac{\hbar^2 \pi^2}{2 m a^2} \left( N + \frac{1}{2} \right)^2, \,N \in \mathbb{Z}
        \end{align*}
\end{enumerate}

\setcounter{subsection}{11}
\subsection{}

\begin{align*}
  R_{n \ell}(r) & = \frac{1}{r} \left( \frac{r}{a n} \right)^{\ell + 1} e^{-r / a n} v(r / a n)                                                 \\
  R_{3 0}       & = e^{-r / 3 a} \frac{c_0}{3 a} \left[ 1 - 2 \left( \frac{r}{3 a} \right) - \frac{1}{3} \left( \frac{r}{3 a} \right)^2 \right] \\
  R_{3 1}       & = \frac{1}{r} \left( \frac{r}{3 a} \right)^2 c_0 \left[ 1 - \frac{1}{2} \left( \frac{r}{3 a} \right) \right]                  \\
  R_{3 2}       & = \frac{1}{r} \left( \frac{r}{3 a} \right)^3 c_0
\end{align*}

\subsection{}

\begin{enumerate}
  \item

        \begin{align*}
          R_{2 0}(r) & = \frac{c_0}{2 a} \left( 1 - \frac{r}{2 a} \right) e^{-r / 2 a}                             \\
          1          & = \int_0^\infty |R_{2 0}|^2 r^2 \,d r                                                       \\
                     & = \frac{c_0^2}{4 a^2} \int_0^\infty \left( 1 - \frac{r}{2 a} \right)^2 e^{-r / a} r^2 \,d r \\
                     & = \frac{a c_0^2}{2}                                                                         \\
          c_0        & = \sqrt{\frac{2}{a}}                                                                        \\
          R_{2 0}(r) & = \frac{1}{\sqrt{2}} a^{-3 / 2} \left( 1 - \frac{r}{2 a} \right) e^{-r / 2 a}               \\
          \psi_{200} & = R_{20} Y_0^0                                                                              \\
                     & = \frac{1}{\sqrt{2 \pi a}} \frac{1}{2 a} \left( 1 - \frac{r}{2 a} \right) e^{-r / 2 a}
        \end{align*}

  \item

        \begin{align*}
          R_{2 1}(r)      & = \frac{c_0}{4 a^2} r e^{-r / 2 a}                                                     \\
          1               & = \frac{c_0^2}{16 a^4} \int_0^\infty r^4 e^{-r / a} \,d r                              \\
                          & = \frac{24}{16} a c_0^2                                                                \\
          c_0             & = \sqrt{\frac{2}{3 a}}                                                                 \\
          R_{2 1}(r)      & = \sqrt{\frac{2}{3 a}} \frac{1}{4 a^2} r e^{-r / 2 a}                                  \\
          \psi_{21 \pm 1} & = R_{21} Y_1^1                                                                         \\
                          & = \mp \frac{1}{\sqrt{\pi a}} \frac{1}{8 a^2} r e^{-r / 2 a} \sin \theta e^{\pm i \phi} \\
          \psi_{210}      & = \frac{1}{\sqrt{2 \pi a}} \frac{1}{4 a^2} r e^{-r / 2 a} \cos \theta
        \end{align*}
\end{enumerate}

\subsection{}

\begin{enumerate}
  \item

        \begin{align*}
          L_q & = \frac{e^x}{q!} \left( \frac{d}{d x} \right)^q (e^{-x} x^q)                 \\
          L_0 & = 1                                                                          \\
          L_1 & = e^x \frac{d}{d x} (x e^{-x})                                               \\
              & = e^x (e^{-x} - x e^{-x})                                                    \\
              & = 1 - x                                                                      \\
          L_2 & = \frac{1}{2} e^x \left( \frac{d}{d x} \right)^2 (x^2 e^{-x})                \\
              & = \frac{1}{2} e^x \frac{d}{d x} (2 x e^{-x} - x^2 e^{-x})                    \\
              & = \frac{1}{2} e^x (2 e^{-x} - 2 x e^{-x} - 2 x e^{-x} + x^2 e^{-x})          \\
              & = 1 - 2 x + \frac{1}{2} x^2                                                  \\
          L_3 & = \frac{1}{6} e^x \left( \frac{d}{d x} \right)^3 (x^3 e^{-x})                \\
              & = \frac{1}{6} e^x \left( \frac{d}{d x} \right)^2 (3 x^2 e^{-x} - x^3 e^{-x}) \\
              & = \frac{1}{6} e^x \frac{d}{d x} (6 x e^{-x} - 6 x^2 e^{-x} + x^3 e^{-x})     \\
              & = \frac{1}{6} e^x (6 e^{-x} - 18 x e^{-x} + 9 x^2 e^{-x} - x^3 e^{-x})       \\
              & = 1 - 3 x + \frac{3}{2} x^2 - \frac{1}{6} x^3
        \end{align*}
\end{enumerate}

\subsection{}

\begin{enumerate}
  \item

        \begin{align*}
          \psi_{100}   & = \frac{1}{\sqrt{\pi a^3}} e^{-r / a}                                                                              \\
          \braket{r}   & = \int \psi_{100}^* r \psi_{100}                                                                                   \\
                       & = \frac{1}{\pi a^3} \int_0^\infty \int_0^\pi \int_0^{2 \pi} r^3 e^{-2 r / a} \sin \theta \,d r \,d \theta \,d \phi \\
                       & = \frac{4}{a^3} \int_0^\infty r^3 e^{-2 r / a} \,d r                                                               \\
                       & = \frac{3}{2} a                                                                                                    \\
          \braket{r^2} & = \frac{4}{a^3} \int_0^\infty r^4 e^{-2 r / a} \,d r                                                               \\
                       & = 3 a^2
        \end{align*}

  \item

        \begin{align*}
          \braket{x}   & = 0   \\
          \braket{x^2} & = a^2
        \end{align*}

  \item

        \begin{align*}
          \psi_{211}   & = R_{21} Y_1^1                                                                                                                    \\
                       & = -\frac{1}{\sqrt{\pi a}} \frac{1}{8 a^2} r e^{-r / 2 a} \sin \theta e^{i \phi}                                                   \\
          \braket{x^2} & = \int \psi_{211}^* x^2 \psi_{211}                                                                                                \\
                       & = \frac{1}{64 \pi a^5} \int_0^\infty \int_0^\pi \int_0^{2 \pi} r^6 e^{-r / a} \sin^5 \theta \cos^2 \phi \,d r \,d \theta \,d \phi \\
                       & = 12 a^2
        \end{align*}
\end{enumerate}

\subsection{}

\begin{align*}
  P(r)         & = \int |\psi_{100}|^2 r^2 \sin \theta \,d \theta \,d \phi                      \\
               & = \frac{4}{a^3} r^2 e^{-2 r / a}                                               \\
  P'(r)        & = \frac{4}{a^3} \left( 2 r e^{-2 r / a} - \frac{2}{a} r^2 e^{-2 r / a} \right) \\
               & = \frac{8}{a^3} r \left( 1 - \frac{r}{a} \right) e^{-2 r / a}                  \\
  r_\text{max} & = a
\end{align*}

\setcounter{subsection}{17}
\subsection{}

\begin{enumerate}
  \item

        \begin{align*}
          \Psi(\vec{r}, t) & = \frac{1}{\sqrt{2}} (\psi_{211} + \psi_{21-1}) e^{-i E_2 t / \hbar}                                                                                                                        \\
                           & = \frac{1}{\sqrt{2}} R_{21} (Y_1^1 + Y_1^{-1}) e^{-i E_1 t / 4 \hbar}                                                                                                                       \\
                           & = \frac{1}{4 a^2} \frac{1}{\sqrt{3 a}} r e^{-r / 2 a} \left( -\sqrt{\frac{3}{8 \pi}} \sin \theta e^{i \phi} + \sqrt{\frac{3}{8 \pi}} \sin \theta e^{-i \phi} \right) e^{-i E_1 t / 4 \hbar} \\
                           & = -i \frac{1}{4 a^2} \frac{1}{\sqrt{2 \pi a}} r e^{-r / 2 a} \sin \theta \sin \phi \,e^{-i E_1 t / 4 \hbar}                                                                                 \\
        \end{align*}

  \item

        \begin{align*}
          \braket{V} & = \int \Psi^* \left( -\frac{e^2}{4 \pi \epsilon_0} \frac{1}{r} \right) \Psi                                                                                                                               \\
                     & = -\frac{e^2}{4 \pi \epsilon_0} \frac{1}{16 a^4} \frac{1}{2 \pi a} \int_0^\infty \int_0^\pi \int_0^{2 \pi} r^2 e^{-r / a} \sin^2 \theta \sin^2 \phi \frac{1}{r} r^2 \sin \theta \,d r \,d \theta \,d \phi \\
                     & = -\frac{e^2}{128 \pi^2 \epsilon_0 a^5} \int_0^\infty \int_0^\pi \int_0^{2 \pi} r^3 e^{-r / a} \sin^3 \theta \sin^2 \phi \,d r \,d \theta \,d \phi                                                        \\
                     & = -\frac{e^2}{4 \pi \epsilon_0} \frac{1}{4 a}                                                                                                                                                             \\
                     & = \qty{-6.8}{eV}
        \end{align*}
\end{enumerate}

\subsection{}

\begin{align*}
  E_1(Z)            & = -\left[ \frac{m_e}{2 \hbar^2} \left( \frac{Z e^2}{4 \pi \epsilon_0} \right)^2 \right] \\
                    & = Z^2 E_1(1)                                                                            \\
                    & = (\qty{-13.6}{eV}) Z^2                                                                 \\
  E_n(Z)            & = \frac{E_1(Z)}{n^2}                                                                    \\
                    & = \frac{(\qty{-13.6}{eV}) Z^2}{n^2}                                                     \\
                    & = Z^2 E_n(1)                                                                            \\
  a(Z)              & = \frac{a}{Z}                                                                           \\
  \mathcal{R}(Z)    & = Z^2 R(1)                                                                              \\
  \frac{1}{\lambda} & = Z^2 R                                                                                 \\
  \lambda           & = \frac{1}{Z^2 R}                                                                       \\
  \lambda_{Z = 2}   & = \frac{1}{4 R}                                                                         \\
                    & \approx \qty{2.28e-8}{m}                                                                \\
                    & = \qty{22.8}{nm}
\end{align*}

\subsection{}

\begin{enumerate}
  \item \[V = -\frac{G M m}{r}\]

  \item \[a_g = \frac{\hbar^2}{G M m^2}\]

  \item

        \begin{align*}
          E_c                         & = \frac{1}{2} m v^2 - \frac{G M m}{r_0}                       \\
          \frac{G M m}{r_0^2}         & = \frac{m v^2}{r_0}                                           \\
          \frac{1}{2} m v^2           & = \frac{G M m}{2 r_0}                                         \\
          E_c                         & = -\frac{G M m}{2 r_0}                                        \\
          E_n                         & = -\left[ \frac{m}{2 \hbar^2} (G M m)^2 \right] \frac{1}{n^2} \\
                                      & = -\frac{G^2 M^2 m^3}{2 \hbar^2} \frac{1}{n^2}                \\
          -\frac{G M m}{2 r_0}        & = -\frac{G^2 M^2 m^3}{2 \hbar^2} \frac{1}{n^2}                \\
          \frac{\hbar^2}{G M m^2 r_0} & = \frac{1}{n^2}                                               \\
          \frac{a_g}{r_0}             & = \frac{1}{n^2}                                               \\
          n                           & = \sqrt{\frac{r_0}{a_g}}
        \end{align*}
\end{enumerate}

\subsection{}

\begin{align*}
  \braket{f_\ell^m | L_+ L_- f_\ell^m} & = \braket{f_\ell^m | (L^2 - L_z^2 + \hbar L_z) f_\ell^m}                                                                                   \\
                                       & = \braket{f_\ell^m | L^2 f_\ell^m} - \braket{f_\ell^m | L_z^2 f_\ell^m} + \braket{f_\ell^m | \hbar L_z f_\ell^m}                           \\
                                       & = \braket{f_\ell^m | \hbar^2 \ell (\ell + 1) f_\ell^m} - \braket{f_\ell^m | \hbar^2 m^2 f_\ell^m} + \braket{f_\ell^m | \hbar^2 m f_\ell m} \\
                                       & = \hbar^2 \ell (\ell + 1) \braket{f_\ell^m | f_\ell^m} - \hbar^2 m^2 \braket{f_\ell^m | f_\ell^m} + \hbar^2 m \braket{f_\ell^m | f_\ell^m} \\
                                       & = \hbar^2 \ell (\ell + 1) - \hbar^2 m^2 + \hbar^2 m                                                                                        \\
                                       & = \hbar^2 [\ell (\ell + 1) - m (m - 1)]                                                                                                    \\
  \braket{f_\ell^m | L_+ L_- f_\ell^m} & = \braket{L_- f_\ell^m | L_- f_\ell^m}                                                                                                     \\
                                       & = \braket{B_\ell^m f_\ell^{m - 1} | B_\ell^m f_\ell^{m - 1}}                                                                               \\
                                       & = |B_\ell^m|^2                                                                                                                             \\
  |B_\ell^m|^2                         & = \hbar^2 [\ell (\ell + 1) - m (m - 1)]                                                                                                    \\
  B_\ell^m                             & = \hbar \sqrt{\ell (\ell + 1) - m (m - 1)}
\end{align*}

The same argument for $\braket{f_\ell^m | L_- L_+ f_\ell^m}$ finds $A_\ell^m$.

\subsection{}

\begin{enumerate}
  \item

        \begin{align*}
          [r_i, p_j] & = i \hbar \delta_{i j}                                      \\
          [r_i, r_j] & = 0                                                         \\
          [p_i, p_j] & = 0                                                         \\
          [L_z, x]   & = [x p_y - y p_x, x]                                        \\
                     & = [x p_y, x] - [y p_x, x]                                   \\
                     & = x [p_y, x] + [x, x] p_y - y [p_x, x] - [y, x] p_x         \\
                     & = i \hbar y                                                 \\
          [L_z, y]   & = [x p_y - y p_x, y]                                        \\
                     & = [x p_y, y] - [y p_x, y]                                   \\
                     & = x [p_y, y] + [x, y] p_y - y [p_x, y] - [y, y] p_x         \\
                     & = -i \hbar x                                                \\
          [L_z, z]   & = [x p_y - y p_x, z]                                        \\
                     & = [x p_y, z] - [y p_x, z]                                   \\
                     & = x [p_y, z] + [x, z] p_y - y [p_x, z] - [p_x, z] y         \\
                     & = 0                                                         \\
          [L_z, p_x] & = [x p_y - y p_x, p_x]                                      \\
                     & = [x p_y, p_x] - [y p_x, p_x]                               \\
                     & = x [p_y, p_x] + [x, p_x] p_y - y [p_x, p_x] - [y, p_x] p_x \\
                     & = i \hbar p_y                                               \\
          [L_z, p_y] & = [x p_y - y p_x, p_y]                                      \\
                     & = [x p_y, p_y] - [y p_x, p_y]                               \\
                     & = x [p_y, p_y] + [x, p_y] p_y - y [p_x, p_y] - [y, p_y] p_x \\
                     & = -i \hbar p_x                                              \\
          [L_z, p_z] & = [x p_y - y p_x, p_z]                                      \\
                     & = [x p_y, p_z] - [y p_x, p_z]                               \\
                     & = x [p_y, p_z] + [x, p_z] p_y - y [p_x, p_z] - [y, p_z] p_x \\
                     & = 0
        \end{align*}

  \item

        \begin{align*}
          [L_z, L_x] & = [L_z, y p_z - z p_y]                                      \\
                     & = [L_z, y p_z] - [L_z, z p_y]                               \\
                     & = y [L_z, p_z] + [L_z, y] p_z - z [L_z, p_y] - [L_z, z] p_y \\
                     & = -i \hbar x p_z + i \hbar z p_x                            \\
                     & = i \hbar L_y
        \end{align*}

  \item

        \begin{align*}
          [L_z, r^2] & = [L_z, x^2 + y^2 + z^2]                                                                              \\
                     & = [L_z, x^2] + [L_z, y^2] + [L_z, z^2]                                                                \\
                     & = x [L_z, x] + [L_z, x] x + y [L_z, y] + [L_z, y] y + z [L_z, z] + [L_z, z] z                         \\
                     & = 2 i \hbar x y - 2 i \hbar x y                                                                       \\
                     & = 0                                                                                                   \\
          [L_z, p^2] & = [L_z, p_x^2 + p_y^2 + p_z^2]                                                                        \\
                     & = [L_z, p_x^2] + [L_z, p_y^2] + [L_z, p_z^2]                                                          \\
                     & = p_x [L_z, p_x] + [L_z, p_x] p_x + p_y [L_z, p_y] + [L_z, p_y] p_y + p_z [L_z, p_z] + [L_z, p_z] p_z \\
                     & = 2 i \hbar p_x p_y - 2 i \hbar p_x p_y                                                               \\
                     & = 0
        \end{align*}
\end{enumerate}

\setcounter{subsection}{23}
\subsection{}

\begin{enumerate}
  \item

        \begin{align*}
          L_+ L_- f & = \hbar e^{i \phi} \left( \frac{\partial}{\partial \theta} + i \cot \theta \frac{\partial}{\partial \phi} \right) \left[ -\hbar e^{-i \phi} \left( \frac{\partial}{\partial \theta} - i \cot \theta \frac{\partial}{\partial \phi} \right) \right] f                                                  \\
                    & = -\hbar^2 e^{i \phi} \left( \frac{\partial}{\partial \theta} + i \cot \theta \frac{\partial}{\partial \phi} \right) \left[ e^{-i \phi} \left( \frac{\partial f}{\partial \theta} - i \cot \theta \frac{\partial f}{\partial \phi} \right) \right]                                                    \\
                    & = -\hbar^2 e^{i \phi} \left[ e^{-i \phi} \left( \frac{\partial^2 f}{\partial \theta^2} + i \csc^2 \theta \frac{\partial f}{\partial \phi} - i \cot \theta \frac{\partial^2 f}{\partial \phi \partial \theta} \right) \right.                                                                          \\
                    & \qquad \left. + \cot \theta e^{-i \phi} \left( \frac{\partial f}{\partial \theta} - i \cot \theta \frac{\partial f}{\partial \phi} \right) + i \cot \theta e^{-i \phi} \left( \frac{\partial^2 f}{\partial \phi \partial \theta} - i \cot \theta \frac{\partial^2 f}{\partial \phi^2} \right) \right] \\
                    & = -\hbar^2 \left( \frac{\partial^2 f}{\partial \theta^2} + i \csc^2 \theta \frac{\partial f}{\partial \phi} + \cot \theta \frac{\partial f}{\partial \theta} - i \cot^2 \theta \frac{\partial f}{\partial \phi} + \cot^2 \theta \frac{\partial^2 f}{\partial \phi^2} \right)                          \\
                    & = -\hbar^2 \left( \frac{\partial^2 f}{\partial \theta^2} + i \frac{\partial f}{\partial \phi} + \cot \theta \frac{\partial f}{\partial \theta} + \cot^2 \theta \frac{\partial^2 f}{\partial \phi^2} \right)                                                                                           \\
          L_+ L_-   & = -\hbar^2 \left( \frac{\partial^2}{\partial \theta^2} + \cot \theta \frac{\partial}{\partial \theta} + \cot^2 \theta \frac{\partial^2}{\partial \phi^2} + i \frac{\partial}{\partial \phi} \right)
        \end{align*}

  \item

        \begin{align*}
          L^2 & = L_+ L_- + L_z^2 - \hbar L_z                                                                                                                                                                                                                                                               \\
              & = -\hbar^2 \left( \frac{\partial^2}{\partial \theta^2} + \cot \theta \frac{\partial}{\partial \theta} + \cot^2 \theta \frac{\partial^2}{\partial \phi^2} + i \frac{\partial}{\partial \phi} \right) - \hbar^2 \frac{\partial^2}{\partial \phi^2} + i \hbar^2 \frac{\partial}{\partial \phi} \\
              & = -\hbar^2 \left( \frac{\partial^2}{\partial \theta^2} + \cot \theta \frac{\partial}{\partial \theta} + \cot^2 \theta \frac{\partial^2}{\partial \phi^2} + i \frac{\partial}{\partial \phi} + \frac{\partial^2}{\partial \phi^2} - i \frac{\partial}{\partial \phi} \right)                 \\
              & = -\hbar^2 \left[ \frac{1}{\sin \theta} \frac{\partial}{\partial \theta} \left( \sin \theta \frac{\partial}{\partial \theta} \right) + \cot^2 \theta \frac{\partial^2}{\partial \phi^2} + \frac{\partial^2}{\partial \phi^2} \right]                                                        \\
              & = -\hbar^2 \left[ \frac{1}{\sin \theta} \frac{\partial}{\partial \theta} \left( \sin \theta \frac{\partial}{\partial \theta} \right) + \frac{1}{\sin^2 \theta} \frac{\partial^2}{\partial \phi^2} \right]
        \end{align*}
\end{enumerate}

\subsection{}

\begin{enumerate}
  \item \[L_+ Y_\ell^\ell = 0\]

  \item

        \begin{align*}
          L_z Y_\ell^\ell                                                  & = \hbar \ell Y_\ell^\ell                                                                                                                         \\
          -i \hbar \frac{\partial Y_\ell^\ell}{\partial \phi}              & = \hbar \ell Y_\ell^\ell                                                                                                                         \\
          \frac{\partial Y_\ell^\ell}{\partial \phi}                       & = i \ell Y_\ell^\ell                                                                                                                             \\
          Y_\ell^\ell                                                      & = e^{i \ell \phi} \Theta(\theta)                                                                                                                 \\
          0                                                                & = L_+ Y_\ell^\ell                                                                                                                                \\
                                                                           & = \hbar e^{i \phi} \left( \frac{\partial}{\partial \theta} + i \cot \theta \frac{\partial}{\partial \phi} \right) e^{i \ell \phi} \Theta(\theta) \\
                                                                           & = \hbar e^{i \phi} \left[ e^{i \ell \phi} \frac{\partial \Theta}{\partial \theta} - \ell \cot \theta e^{i \ell \phi} \Theta(\theta) \right]      \\
                                                                           & = \frac{\partial \Theta}{\partial \theta} - \ell \cot \theta \Theta(\theta)                                                                      \\
          \frac{1}{\Theta(\theta)} \frac{\partial \Theta}{\partial \theta} & = \ell \cot \theta                                                                                                                               \\
          \int \frac{1}{\Theta(\theta)} \,d \Theta                         & = \ell \int \frac{\cos \theta}{\sin \theta} \,d \theta                                                                                           \\
          \ln \Theta                                                       & = \ell \ln \sin \theta + c_1                                                                                                                     \\
                                                                           & = \ln (\sin^\ell \theta) + c_1                                                                                                                   \\
          \ln \frac{\Theta}{\sin^\ell \theta}                              & = c_1                                                                                                                                            \\
          \frac{\Theta}{\sin^\ell \theta}                                  & = c_2                                                                                                                                            \\
          \Theta                                                           & = c_2 \sin^\ell \theta                                                                                                                           \\
          Y_\ell^\ell                                                      & = c e^{i \ell \phi} \sin^\ell \theta
        \end{align*}
\end{enumerate}

\subsection{}

\begin{align*}
  Y_2^1(\theta, \phi) & = -\sqrt{\frac{15}{8 \pi}} \sin \theta \cos \theta e^{i \phi}                                                                                                                                    \\
                      & = -\frac{1}{2} \sqrt{\frac{15}{8 \pi}} \sin (2 \theta) e^{i \phi}                                                                                                                                \\
  L_+ Y_2^1           & = \hbar e^{i \phi} \left( \frac{\partial}{\partial \theta} + i \cot \theta \frac{\partial}{\partial \phi} \right) \left[ -\frac{1}{2} \sqrt{\frac{15}{8 \pi}} \sin (2 \theta) e^{i \phi} \right] \\
                      & = -\frac{1}{2} \sqrt{\frac{15}{8 \pi}} \hbar e^{i \phi} [2 \cos(2 \theta) e^{i \phi} - \cot \theta \sin (2 \theta) e^{i \phi}]                                                                   \\
                      & = \sqrt{\frac{15}{8 \pi}} \hbar (e^{i \phi} \sin \theta)^2                                                                                                                                       \\
                      & = \hbar \sqrt{(2 - 1) (2 + 1 + 1)} Y_2^1                                                                                                                                                         \\
                      & = 2 \hbar Y_2^1                                                                                                                                                                                  \\
  Y_2^1               & = \frac{1}{4} \sqrt{\frac{15}{2 \pi}} (e^{i \phi} \sin \theta)^2
\end{align*}

\setcounter{subsection}{27}
\subsection{}

\begin{align*}
  r                 & = \frac{e^2}{4 \pi \epsilon_0 m c^2} \\
                    & = \qty{0.0796}{m}                    \\
  \frac{1}{2} \hbar & = I \omega                           \\
                    & = \frac{2}{5} m r^2 \frac{v}{r}      \\
  v                 & = \frac{5 \hbar}{4 m r}              \\
                    & = \qty{5.14e10}{m/s}
\end{align*}

This is $100$ times the speed of light, so no it doesn't make sense.

\subsection{}

\begin{enumerate}
  \item

        \begin{align*}
          [S_x, S_y] & = S_x S_y - S_y S_x              \\
                     & = \begin{pmatrix}
                           i \hbar^2 / 2 & 0              \\
                           0             & -i \hbar^2 / 2
                         \end{pmatrix} \\
                     & = i \hbar S_z                    \\
          [S_y, S_z] & = i \hbar S_x                    \\
          [S_z, S_x] & = i \hbar S_y
        \end{align*}

  \item

        \begin{align*}
          \sigma_x \sigma_x & = \begin{pmatrix}
                                  1 & 0 \\
                                  0 & 1
                                \end{pmatrix}              \\
                            & = \delta_{x x}                \\
          \delta_x \delta_y & = \begin{pmatrix}
                                  i & 0  \\
                                  0 & -i
                                \end{pmatrix}              \\
                            & = i \epsilon_{x y z} \sigma_z \\
          \delta_x \delta_z & = \begin{pmatrix}
                                  0 & -1 \\
                                  1 & 0
                                \end{pmatrix}              \\
                            & = i \epsilon_{x z y} \sigma_y \\
          \delta_y \delta_x & = \begin{pmatrix}
                                  -i & 0 \\
                                  0  & 1
                                \end{pmatrix}              \\
                            & = i \epsilon_{y x z} \sigma_z \\
          \delta_y \delta_y & = \begin{pmatrix}
                                  1 & 0 \\
                                  0 & 1
                                \end{pmatrix}              \\
                            & = \delta_{y y}                \\
          \delta_y \delta_z & = \begin{pmatrix}
                                  0 & i \\
                                  i & 0
                                \end{pmatrix}              \\
                            & = i \epsilon_{y z x} \sigma_x \\
          \delta_z \delta_x & = \begin{pmatrix}
                                  0  & 1 \\
                                  -1 & 0
                                \end{pmatrix}              \\
                            & = i \epsilon_{z x y} \sigma_y \\
          \delta_z \delta_y & = \begin{pmatrix}
                                  0  & -i \\
                                  -i & 0
                                \end{pmatrix}              \\
                            & = i \epsilon_{z y x} \sigma_x \\
          \delta_z \delta_z & = \begin{pmatrix}
                                  1 & 0 \\
                                  0 & 1
                                \end{pmatrix}              \\
                            & = \delta_{z z}
        \end{align*}
\end{enumerate}

\subsection{}

\begin{enumerate}
  \item

        \begin{align*}
          1 & = \chi^\dagger \chi                  \\
            & = |A|^2 \begin{pmatrix}
                        -3 i & 4
                      \end{pmatrix} \begin{pmatrix}
                                      3 i \\
                                      4
                                    \end{pmatrix} \\
            & = 25 |A|^2                           \\
          A & = \pm \frac{1}{5}
        \end{align*}

  \item

        \begin{align*}
          \chi                             & = \frac{1}{5} \begin{pmatrix}
                                                             3 i \\
                                                             4
                                                           \end{pmatrix}                                  \\
          % S_x
          P \left( \frac{\hbar}{2} \right) & = \frac{1}{2} \left| \frac{1}{5} (3 i + 4) \right|^2          \\
                                           & = \frac{1}{50} (-3 i + 4) (3 i + 4)                           \\
                                           & = \frac{1}{2}                                                 \\
          \braket{S_x}                     & = 0                                                           \\
          % S_y
          \braket{S_y}                     & = \chi^\dagger S_y \chi                                       \\
                                           & = \frac{\hbar}{50} \begin{pmatrix}
                                                                  -3 i & 4
                                                                \end{pmatrix} \begin{pmatrix}
                                                                                0 & -i \\
                                                                                i & 0
                                                                              \end{pmatrix} \begin{pmatrix}
                                                                                              3 i \\
                                                                                              4
                                                                                            \end{pmatrix} \\
                                           & = \frac{\hbar}{50} \begin{pmatrix}
                                                                  -3 i & 4
                                                                \end{pmatrix} \begin{pmatrix}
                                                                                0  & -4 i \\
                                                                                -3 & 0
                                                                              \end{pmatrix}               \\
                                           & = -\frac{12}{25} \hbar                                        \\
          % S_z
          \braket{S_z}                     & = \frac{\hbar}{50} \begin{pmatrix}
                                                                  -3 i & 4
                                                                \end{pmatrix} \begin{pmatrix}
                                                                                1 & 0  \\
                                                                                0 & -1
                                                                              \end{pmatrix} \begin{pmatrix}
                                                                                              3 i \\
                                                                                              4
                                                                                            \end{pmatrix} \\
                                           & = \frac{\hbar}{50} \begin{pmatrix}
                                                                  -3 i & 4
                                                                \end{pmatrix} \begin{pmatrix}
                                                                                3 i \\
                                                                                -4
                                                                              \end{pmatrix}               \\
                                           & = -\frac{7}{50} \hbar
        \end{align*}

  \item

        \begin{align*}
          \braket{S_x^2} & = \frac{\hbar^2}{100} \begin{pmatrix}
                                                   -3 i & 4
                                                 \end{pmatrix} \begin{pmatrix}
                                                                 1 & 0 \\
                                                                 0 & 1
                                                               \end{pmatrix} \begin{pmatrix}
                                                                               3 i \\
                                                                               4
                                                                             \end{pmatrix} \\
                         & = \frac{\hbar^2}{4}                                              \\
          \sigma_{S_x}   & = \sqrt{\braket{S_x^2} - \braket{S_x}^2}                         \\
                         & = \frac{\hbar}{2}                                                \\
          \braket{S_y^2} & = \frac{\hbar^2}{4}                                              \\
          \sigma_{S_y}   & = \sqrt{\frac{1}{4} \hbar^2 - \frac{144}{625} \hbar^2}           \\
                         & = \frac{7}{50} \hbar                                             \\
          \sigma_{S_z}   & = \frac{12}{25} \hbar
        \end{align*}

  \item

        \begin{align*}
          \sigma_{S_x} \sigma_{S_y} & = \frac{7}{100} \hbar^2            \\
                                    & = \frac{\hbar}{2} |\braket{S_z}|   \\
                                    & \ge \frac{\hbar}{2} |\braket{S_z}| \\
          \sigma_{S_y} \sigma_{S_z} & = \frac{42}{625} \hbar^2           \\
                                    & \ge \frac{\hbar}{2} |\braket{S_x}| \\
                                    & = 0                                \\
          \sigma_{S_z} \sigma_{S_x} & = \frac{6}{25} \hbar^2             \\
                                    & = \frac{\hbar}{2} |\braket{S_y}|   \\
                                    & \ge \frac{\hbar}{2} |\braket{S_z}|
        \end{align*}
\end{enumerate}

\subsection{}

\begin{align*}
  \chi                                             & = \begin{pmatrix}
                                                         a \\
                                                         b
                                                       \end{pmatrix}                      \\
  \braket{S_x}                                     & = \frac{\hbar}{2} (a^* b + a b^*)     \\
  \braket{S_y}                                     & = i \frac{\hbar}{2} (a b^* - a^* b)   \\
  \braket{S_z}                                     & = \frac{\hbar}{2} (|a|^2 - |b|^2)     \\
  \braket{S_x^2}                                   & = \frac{\hbar^2}{4} (|a|^2 + |b|^2)   \\
  \braket{S_y^2}                                   & = \frac{\hbar^2}{4} (|a|^2 + |b|^2)   \\
  \braket{S_z^2}                                   & = \frac{\hbar^2}{4} (|a|^2 + |b|^2)   \\
  \braket{S^2}                                     & = 3 \frac{\hbar^2}{4} (|a|^2 + |b|^2) \\
  \braket{S_x^2} + \braket{S_y^2} + \braket{S_z^2} & = 3 \frac{\hbar^2}{4} (|a|^2 + |b|^2) \\
                                                   & = \braket{S^2}
\end{align*}

\subsection{}

\begin{enumerate}
  \item

        \begin{align*}
          0                               & = \begin{vmatrix}
                                                -\lambda    & -i \hbar / 2 \\
                                                i \hbar / 2 & -\lambda
                                              \end{vmatrix}                     \\
                                          & = \lambda^2 - \frac{\hbar^2}{4}                  \\
          \lambda                         & = \pm \frac{\hbar}{2}                            \\
          \frac{\hbar}{2} \begin{pmatrix}
                            0 & -i \\
                            i & 0
                          \end{pmatrix} \begin{pmatrix}
                                          \alpha \\
                                          \beta
                                        \end{pmatrix} & = \pm \frac{\hbar}{2} \begin{pmatrix}
                                                                                \alpha \\
                                                                                \beta
                                                                              \end{pmatrix} \\
          \begin{pmatrix}
            -i \beta \\
            i \alpha
          \end{pmatrix}                 & = \pm \begin{pmatrix}
                                                  \alpha \\
                                                  \beta
                                                \end{pmatrix}                               \\
          \beta                           & = \pm i \alpha                                   \\
          \chi_+^{(y)}                    & = \frac{1}{\sqrt{2}} \begin{pmatrix}
                                                                   1 \\
                                                                   i
                                                                 \end{pmatrix}              \\
          \chi_-^{(y)}                    & = \frac{1}{\sqrt{2}} \begin{pmatrix}
                                                                   1 \\
                                                                   -i
                                                                 \end{pmatrix}
        \end{align*}

  \item

        \begin{align*}
          \chi & = \begin{pmatrix}
                     a \\
                     b
                   \end{pmatrix}                                                                                              \\
               & = \left( \frac{a - i b}{\sqrt{2}} \right) \chi_+^{(x)} + \left( \frac{a + i b}{\sqrt{2}} \right) \chi_-^{(y)}
        \end{align*}

        You could get $\frac{\hbar}{2}$ with probability $\frac{|a - i b|^2}{2}$ or $-\frac{\hbar}{2}$ with probability $\frac{|a + i b|^2}{2}$.

        \begin{align*}
          \frac{1}{2} |a - i b|^2 + \frac{1}{2} |a + i b|^2 & = \frac{1}{2} (a + i b) (a - i b) + \frac{1}{2} (a - i b) (a + i b) \\
                                                            & = \frac{1}{2} (a^2 + b^2) + \frac{1}{2} (a^2 + b^2)                 \\
                                                            & = 1
        \end{align*}

  \item $\hbar^2 / 2$ with probability $1$.
\end{enumerate}

\subsection{}

\begin{align*}
  S_r & = S \cdot \uvec{r}                                                                              \\
      & = S_x \sin \theta \cos \phi + S_y \sin \theta \sin \phi + S_z \cos \theta                       \\
      & = \frac{\hbar}{2} \left[ \begin{pmatrix}
                                     0                     & \sin \theta \cos \phi \\
                                     \sin \theta \cos \phi & 0
                                   \end{pmatrix} + \begin{pmatrix}
                                                     0                       & -i \sin \theta \sin \phi \\
                                                     i \sin \theta \sin \phi & 0
                                                   \end{pmatrix} \right.     \\
      & \qquad \left. + \begin{pmatrix}
                            \cos \theta & 0            \\
                            0           & -\cos \theta
                          \end{pmatrix} \right]                                                      \\
      & = \frac{\hbar}{2} \begin{pmatrix}
                            \cos \theta                           & \sin \theta (\cos \phi - i \sin \phi) \\
                            \sin \theta (\cos \phi + i \sin \phi) & -\cos \theta
                          \end{pmatrix} \\
      & = \frac{\hbar}{2} \begin{pmatrix}
                            \cos \theta            & e^{-i \phi} \sin \theta \\
                            e^{i \phi} \sin \theta & -\cos \theta
                          \end{pmatrix}
\end{align*}

\begin{align*}
  \begin{vmatrix}
    \frac{\hbar}{2} \cos \theta - \lambda  & \frac{\hbar}{2} e^{-i \phi} \sin \theta \\
    \frac{\hbar}{2} e^{i \phi} \sin \theta & -\frac{\hbar}{2} \cos \theta - \lambda
  \end{vmatrix} & = \left( \frac{\hbar}{2} \cos \theta - \lambda \right) \left( -\frac{\hbar}{2} \cos \theta - \lambda \right)                                                                                     \\
                                                                                      & \qquad - \left( \frac{\hbar}{2} e^{-i \phi} \sin \theta \right) \left( \frac{\hbar}{2} e^{i \phi} \sin \theta \right)      \\
                                                                                      & = -\frac{\hbar^2}{4} \cos^2 \theta - \lambda \frac{\hbar}{2} \cos \theta + \lambda \frac{\hbar}{2} \cos \theta + \lambda^2 \\
                                                                                      & \qquad -\frac{\hbar^2}{4} \sin^2 \theta                                                                                    \\
                                                                                      & = \lambda^2 - \frac{\hbar^2}{4} \\
                                                                                      \lambda & = \pm \frac{\hbar}{2}
\end{align*}

\setcounter{subsection}{34}
\subsection{}

\begin{enumerate}
  \item

        \begin{align*}
          \chi(t)               & = \begin{pmatrix}
                                      \cos (\alpha / 2) e^{i \gamma B_0 t / 2} \\
                                      \sin (\alpha / 2) e^{-i \gamma B_0 t / 2}
                                    \end{pmatrix}                                                           \\
          \frac{1}{2} |a + b|^2 & = \frac{1}{2} [\cos (\alpha / 2) e^{-i \gamma B_0 t / 2} + \sin (\alpha / 2) e^{i \gamma B_0 t / 2}] \\
                                & \qquad [\cos (\alpha / 2) e^{i \gamma B_0 t / 2} + \sin (\alpha / 2) e^{-i \gamma B_0 t / 2}]        \\
                                & = \frac{1}{2} [\cos^2 (\alpha / 2) + \cos (\alpha / 2) \sin (\alpha / 2) e^{-i \gamma B_0 t}         \\
                                & \qquad + \cos (\alpha / 2) \sin (\alpha / 2) e^{i \gamma B_0 t} + \sin^2 (\alpha / 2)]               \\
                                & = \frac{1}{2} [1 + \sin \alpha \cos (\gamma B_0 t)]
        \end{align*}

  \item

        \begin{align*}
          \frac{1}{2} |a - i b|^2 & = \frac{1}{2} [\cos (\alpha / 2) e^{-i \gamma B_0 t / 2} + i \sin (\alpha / 2) e^{i \gamma B_0 t / 2}] \\
                                  & \qquad [\cos (\alpha / 2) e^{i \gamma B_0 t / 2} - i \sin (\alpha / 2) e^{-i \gamma B_0 t / 2}]        \\
                                  & = \frac{1}{2} [\cos^2 (\alpha / 2) - i \cos (\alpha / 2) \sin (\alpha / 2) e^{-i \gamma B_0 t}         \\
                                  & \qquad + i \cos (\alpha / 2) \sin (\alpha / 2) e^{i \gamma B_0 t} + \sin^2 (\alpha / 2)]               \\
                                  & = \frac{1}{2} [1 - \sin \alpha \sin (\gamma B_0 t)]
        \end{align*}

  \item \[|a|^2 = \cos^2 \frac{\alpha}{2}\]
\end{enumerate}

\subsection{}

\begin{enumerate}
  \item

        \begin{align*}
          \vec{B} & = B_0 \cos (\omega t) \hat{k}                                \\
          H       & = -\gamma \vec{B} \cdot \vec{S}                              \\
                  & = -\gamma B_0 \cos (\omega t) S_z                            \\
                  & = -\frac{\gamma B_0 \hbar}{2} \cos (\omega t) \begin{pmatrix}
                                                                    1 & 0  \\
                                                                    0 & -1
                                                                  \end{pmatrix}
        \end{align*}

  \item

        \begin{align*}
          i \hbar \frac{\partial \chi}{\partial t} & = H \chi                                                                \\
          i \hbar \begin{pmatrix}
                    a'(t) \\
                    b'(t)
                  \end{pmatrix}                  & = -\frac{\gamma B_0 \hbar}{2} \cos (\omega t) \begin{pmatrix}
                                                                                                   a(t) \\
                                                                                                   -b(t)
                                                                                                 \end{pmatrix}              \\
          i \hbar a'(t)                            & = -\frac{\gamma B_0 \hbar}{2} \cos (\omega t) a(t)                      \\
          \frac{1}{a(t)} a'(t)                     & = i \frac{\gamma B_0}{2} \cos (\omega t)                                \\
          \ln [a(t)]                               & = i \frac{\gamma B_0}{2 \omega} \sin (\omega t) + c_1                   \\
          a(t)                                     & = A e^{i \gamma B_0 \sin (\omega t) / 2 \omega}                         \\
          i \hbar b'(t)                            & = \frac{\gamma B_0 \hbar}{2} \cos (\omega t) b(t)                       \\
          \frac{1}{b(t)} b'(t)                     & = -i \frac{\gamma B_0}{2} \cos (\omega t)                               \\
          \ln [b(t)]                               & = -i \frac{\gamma B_0}{2 \omega} \sin (\omega t) + c_2                  \\
          b(t)                                     & = B e^{-i \gamma B_0 \sin (\omega t) / 2 \omega}                        \\
          \chi                                     & = \begin{pmatrix}
                                                         A e^{i \gamma B_0 \sin (\omega t) / 2 \omega} \\
                                                         B e^{-i \gamma B_0 \sin (\omega t) / 2 \omega}
                                                       \end{pmatrix}                  \\
          % At t = 0
          1                                        & = \frac{1}{2} |A + B|^2                                                 \\
                                                   & = \frac{1}{2} (A^* + B^*) (A + B)                                       \\
                                                   & = \frac{1}{2} (|A|^2 + A^* B + B^* A + |B|^2)                           \\
          0                                        & = \frac{1}{2} |A - B|^2                                                 \\
                                                   & = \frac{1}{2} (A^* - B^*) (A - B)                                       \\
                                                   & = \frac{1}{2} (|A|^2 - A^* B - B^* A + |B|^2)                           \\
          \chi                                     & = \frac{1}{\sqrt{2}} \begin{pmatrix}
                                                                            e^{i \gamma B_0 \sin (\omega t) / 2 \omega} \\
                                                                            e^{-i \gamma B_0 \sin (\omega t) / 2 \omega}
                                                                          \end{pmatrix}
        \end{align*}

  \item

        \begin{align*}
          \frac{A - B}{\sqrt{2}} & = \frac{1}{2} [e^{i \gamma B_0 \sin (\omega t) / 2 \omega} - e^{-i \gamma B_0 \sin (\omega t) / 2 \omega}] \\
                                 & = i \sin \left[ \frac{\gamma B_0}{2 \omega} \sin (\omega t) \right]                                        \\
          \frac{1}{2} |A - B|^2  & = \sin^2 \left[ \frac{\gamma B_0}{2 \omega} \sin (\omega t) \right]
        \end{align*}

  \item \[|B_0| = \frac{\omega \pi}{\gamma}\]
\end{enumerate}

\subsection{}

\begin{enumerate}
  \item

        \begin{align*}
          S_- \ket{1 0} & = S_- \left[ \frac{1}{\sqrt{2}} (\ket{\uparrow \downarrow} + \ket{\downarrow \uparrow}) \right]                                                                                                              \\
                        & = \frac{1}{\sqrt{2}} [(S_-^{(1)} \ket{\uparrow}) \ket{\downarrow} + \ket{\uparrow} (S_-^{(2)} \ket{\downarrow}) + (S_-^{(1)} \ket{\downarrow}) \ket{\uparrow} + \ket{\downarrow} (S_-^{(2)} \ket{\uparrow})] \\
                        & = \frac{1}{\sqrt{2}} [\hbar \ket{\downarrow \downarrow} + \hbar \ket{\downarrow \downarrow}]                                                                                                                 \\
                        & = \sqrt{2} \hbar \ket{1 \,{-1}}
        \end{align*}

  \item

        \begin{align*}
          S_+ \ket{0 0} & = S_+ \left[ \frac{1}{\sqrt{2}} (\ket{\uparrow \downarrow} - \ket{\downarrow \uparrow}) \right]                                                                                                              \\
                        & = \frac{1}{\sqrt{2}} [(S_+^{(1)} \ket{\uparrow}) \ket{\downarrow} + \ket{\uparrow} (S_+^{(2)} \ket{\downarrow}) - (S_+^{(1)} \ket{\downarrow}) \ket{\uparrow} - \ket{\downarrow} (S_+^{(2)} \ket{\uparrow})] \\
                        & = \frac{1}{\sqrt{2}} (\hbar \ket{\uparrow \uparrow} - \hbar \ket{\uparrow \uparrow})                                                                                                                         \\
                        & = 0                                                                                                                                                                                                          \\
          S_- \ket{0 0} & = S_- \left[ \frac{1}{\sqrt{2}} (\ket{\uparrow \downarrow} - \ket{\downarrow \uparrow}) \right]                                                                                                              \\
                        & = \frac{1}{\sqrt{2}} [(S_-^{(1)} \ket{\uparrow}) \ket{\downarrow} + \ket{\uparrow} (S_-^{(2)} \ket{\downarrow}) - (S_-^{(1)} \ket{\downarrow}) \ket{\uparrow} - \ket{\downarrow} (S_-^{(2)} \ket{\uparrow})] \\
                        & = \frac{1}{\sqrt{2}} (\hbar \ket{\downarrow \downarrow} - \hbar \ket{\downarrow \downarrow})                                                                                                                 \\
                        & = 0
        \end{align*}

  \item

        \begin{align*}
          S^{(1)} \cdot S^{(2)} \ket{\uparrow \uparrow}     & = (S_x^{(1)} \ket{\uparrow}) (S_x^{(2)} \ket{\uparrow}) + (S_y^{(1)} \ket{\uparrow}) (S_y^{(2)} \ket{\uparrow})                                                                                         \\
                                                            & \qquad + (S_z^{(1)} \ket{\uparrow}) (S_z^{(2)} \ket{\uparrow})                                                                                                                                          \\
                                                            & = \left( \frac{\hbar}{2} \ket{\downarrow} \right) \left( \frac{\hbar}{2} \ket{\downarrow} \right) + \left( \frac{i \hbar}{2} \ket{\downarrow} \right) \left( \frac{i \hbar}{2} \ket{\downarrow} \right) \\
                                                            & \qquad + \left( \frac{\hbar}{2} \ket{\uparrow} \right) \left( \frac{\hbar}{2} \ket{\uparrow} \right)                                                                                                    \\
                                                            & = \frac{\hbar^2}{4} \ket{\downarrow \downarrow} - \frac{\hbar^2}{4} \ket{\downarrow \downarrow} + \frac{\hbar^2}{4} \ket{\uparrow \uparrow}                                                             \\
          S^2 \ket{\uparrow \uparrow}                       & = \left( \frac{3 \hbar^2}{4} + \frac{3 \hbar^2}{4} + 2 \frac{\hbar^2}{4} \right) \ket{\uparrow \uparrow}                                                                                                \\
                                                            & = 2 \hbar^2 \ket{\uparrow \uparrow}                                                                                                                                                                     \\
          S^{(1)} \cdot S^{(2)} \ket{\downarrow \downarrow} & = (S_x^{(1)} \ket{\downarrow}) (S_x^{(2)} \ket{\downarrow}) + (S_y^{(1)} \ket{\downarrow}) (S_y^{(2)} \ket{\downarrow})                                                                                 \\
                                                            & \qquad + (S_z^{(1)} \ket{\downarrow}) (S_z^{(2)} \ket{\downarrow})                                                                                                                                      \\
                                                            & = \left( \frac{\hbar}{2} \ket{\uparrow} \right) \left( \frac{\hbar}{2} \ket{\uparrow} \right) + \left( -\frac{i \hbar}{2} \ket{\uparrow} \right) \left( -\frac{i \hbar}{2} \ket{\uparrow} \right)       \\
                                                            & \qquad \left( -\frac{\hbar}{2} \ket{\downarrow} \right) \left( -\frac{\hbar}{2} \ket{\downarrow} \right)                                                                                                \\
                                                            & = \frac{\hbar^2}{4} \ket{\uparrow \uparrow} - \frac{\hbar^2}{4} \ket{\uparrow \uparrow} + \frac{\hbar^2}{4} \ket{\downarrow \downarrow}                                                                 \\
          S^2 \ket{\downarrow \downarrow}                   & = \left( \frac{3 \hbar^2}{4} + \frac{3 \hbar^2}{4} + 2 \frac{\hbar^2}{4} \right) \ket{\downarrow \downarrow}                                                                                            \\
                                                            & = 2 \hbar^2 \ket{\downarrow \downarrow}
        \end{align*}
\end{enumerate}

\subsection{}

\begin{enumerate}
  \item $3 / 2, 1 / 2$

  \item $0, 1$
\end{enumerate}

\setcounter{subsection}{39}
\subsection{}

\begin{enumerate}
  \item \[\ket{3 1} = \frac{1}{\sqrt{15}} \ket{2 2} \ket{1 -1} + \sqrt{\frac{8}{15}} \ket{2 1} \ket{1 0} + \sqrt{\frac{6}{15}} \ket{2 0} \ket{1 1}\]

        $2 \hbar$ with probability $1 / 15$, $\hbar$ with probability $8 / 15$, and $0$ with probability $6 / 15$.
\end{enumerate}

\end{document}