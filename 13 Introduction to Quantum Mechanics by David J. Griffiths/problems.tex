\documentclass{article}
\usepackage{amsmath} % For align*
\usepackage{amsfonts} % For mathbb
\usepackage{bookmark} % For links
\usepackage{enumitem} % For customisable list labels
\usepackage{siunitx} % For units

\hypersetup{
  colorlinks=true,
  linkcolor=blue,
  urlcolor=blue
}

\DeclareMathOperator{\erf}{erf}
\newcommand{\ev}[1]{\left< #1 \right>}

\setlist[enumerate, 1]{label={(\alph*)}}
\setlist[enumerate, 2]{label={(\roman*)}}

\title{Introduction to Quantum Mechanics by David J. Griffiths Problems}
\author{Chris Doble}
\date{March 2023}

\begin{document}

\maketitle

\tableofcontents

\part{Theory}

\section{The Wave Function}

\subsection{}

\begin{enumerate}
  \item

        \begin{align*}
          \langle j^2 \rangle & = \sum j^2 P(j)                                                                                                         \\
                              & = 14^2 \frac{1}{14} + 15^2 \frac{1}{14} + 16^2 \frac{3}{14} + 22^2 \frac{2}{14} + 24^2 \frac{2}{14} + 25^2 \frac{5}{14} \\
                              & = \frac{3217}{7}                                                                                                        \\
                              & \approx 459.571                                                                                                         \\
          \langle j \rangle^2 & = \left( \sum j P(j) \right)^2                                                                                          \\
                              & = 441
        \end{align*}

  \item

        \begin{align*}
          \Delta j_{14} & = -7                     \\
          \Delta j_{15} & = -6                     \\
          \Delta j_{16} & = -5                     \\
          \Delta j_{22} & = 1                      \\
          \Delta j_{24} & = 3                      \\
          \Delta j_{25} & = 4                      \\
          \sigma^2      & = \sum (\Delta j)^2 P(j) \\
                        & = \frac{130}{7}          \\
                        & \approx 18.571
        \end{align*}

  \item \[\sigma^2 = \sqrt{\langle j^2 \rangle - \langle j \rangle^2} = 18.571\]
\end{enumerate}

\subsection{}

\begin{enumerate}
  \item

        \begin{align*}
          \langle x^2 \rangle & = \int_0^h x^2 \rho(x) \,d x                                    \\
                              & = \int_0^h \frac{x^{3 / 2}}{2 \sqrt{h}} \,d x                   \\
                              & = \frac{1}{2 \sqrt{h}} \left[ \frac{2}{5} x^{5 / 2} \right]_0^h \\
                              & = \frac{h^2}{5}                                                 \\
          \langle x \rangle^2 & = \frac{h^2}{9}                                                 \\
          \sigma              & = \sqrt{\langle x^2 \rangle - \langle x \rangle^2}              \\
                              & = \sqrt{\frac{h^2}{5} - \frac{h^2}{9}}                          \\
                              & = h \sqrt{\frac{4}{45}}                                         \\
                              & = \frac{2}{3 \sqrt{5}} h
        \end{align*}

  \item

        \begin{align*}
          1 - \int_{\langle x \rangle - \sigma}^{\langle x \rangle + \sigma} \rho(x) \,d x & = 1 - \frac{1}{2 \sqrt{h}} [2 \sqrt{x}]_{\langle x \rangle - \sigma}^{\langle x \rangle + \sigma}                                     \\
                                                                                           & = 1 - \frac{1}{\sqrt{h}} \left( \sqrt{\frac{1}{3} h + \frac{2}{3 \sqrt{5}} h} - \sqrt{\frac{1}{3} h - \frac{2}{3 \sqrt{5}} h} \right) \\
                                                                                           & = 1 - \left( \sqrt{\frac{1}{3} + \frac{2}{3 \sqrt{5}}} - \sqrt{\frac{1}{3} - \frac{2}{3 \sqrt{5}}} \right)                            \\
                                                                                           & \approx 0.393
        \end{align*}
\end{enumerate}

\subsection{}

\begin{enumerate}
  \item

        \begin{align*}
          \rho(x) & = A e^{-\lambda (x - a)^2}                             \\
          1       & = \int_{-\infty}^\infty \rho(x) \,d x                  \\
                  & = A \int_{-\infty}^\infty e^{-\lambda (x - a)^2} \,d x \\
                  & = A \sqrt{\frac{\pi}{\lambda}}                         \\
          A       & = \sqrt{\frac{\lambda}{\pi}}
        \end{align*}

  \item

        \begin{align*}
          \langle x \rangle   & = \sqrt{\frac{\lambda}{\pi}} \int_{-\infty}^\infty x e^{-\lambda (x - a)^2} \,d x   \\
                              & = a                                                                                 \\
          \langle x^2 \rangle & = \sqrt{\frac{\lambda}{\pi}} \int_{-\infty}^\infty x^2 e^{-\lambda (x - a)^2} \,d x \\
                              & = a^2 + \frac{1}{2 \lambda}                                                         \\
          \sigma              & = \sqrt{\langle x^2 \rangle - \langle x \rangle^2}                                  \\
                              & = \sqrt{a^2 + \frac{1}{2 \lambda} - a^2}                                            \\
                              & = \frac{1}{\sqrt{2 \lambda}}
        \end{align*}
\end{enumerate}

\subsection{}

\begin{enumerate}
  \item

        \begin{align*}
          1 & = \int_{-\infty}^\infty |\Psi(x, 0)|^2 \,d x                                                                  \\
            & = \left( \frac{A}{a} \right)^2 \int_0^a x^2 \,d x + \left( \frac{A}{b - a} \right)^2 \int_a^b (b - x)^2 \,d x \\
            & = \frac{1}{3} A^2 a + \left( \frac{A}{b - a} \right)^2 \left[ -\frac{1}{3} (b - x)^3 \right]_a^b              \\
            & = \frac{1}{3} A^2 a + \frac{1}{3} A^2 (b - a)                                                                 \\
            & = \frac{1}{3} A^2 b                                                                                           \\
          A & = \sqrt{\frac{3}{b}}
        \end{align*}

        \setcounter{enumi}{2}
  \item $x = a$

  \item

        \begin{align*}
          \int_0^a |\Psi(x, 0)|^2 \,d x & = \frac{3}{a^2 b} \left[ \frac{1}{3} x^3 \right]_0^a \\
                                        & = \frac{a}{b}
        \end{align*}

  \item

        \begin{align*}
          \langle x \rangle & = \int_{-\infty}^\infty x |\Psi(x, 0)|^2 \,d x                                                                                                                             \\
                            & = \frac{3}{a^2 b} \left[ \frac{1}{4} x^4 \right]_0^a + \frac{3}{b (b - a)^2} \int_a^b x (b - x)^2 \,d x                                                                    \\
                            & = \frac{3 a^2}{4 b} + \frac{3}{b (b - a)^2} \int_a^b (b^2 x - 2 b x^2 + x^3) \,d x                                                                                         \\
                            & = \frac{3 a^2}{4 b} + \frac{3}{b (b - a)^2} \left[ \frac{1}{2} b^2 x^2 - \frac{2}{3} b x^3 + \frac{1}{4} x^4 \right]_a^b                                                   \\
                            & = \frac{3 a^2}{4 b} + \frac{3}{b (b - a)^2} \left( \frac{1}{2} b^4 - \frac{2}{3} b^4 + \frac{1}{4} b^4 - \frac{1}{2} a^2 b^2 + \frac{2}{3} a^3 b - \frac{1}{4} a^4 \right) \\
                            & = \frac{3 a^2}{4 b} + \frac{3}{b (b - a)^2} \frac{1}{12} (b - a)^3 (3 a + b)                                                                                               \\
                            & = \frac{3 a^2}{4 b} + \frac{1}{4 b} (3 a b + b^2 - 3 a^2 - a b)                                                                                                            \\
                            & = \frac{1}{2} a + \frac{1}{4} b
        \end{align*}
\end{enumerate}

\subsection{}

\begin{enumerate}
  \item

        \begin{align*}
          \Psi(x, t) & = A e^{-\lambda |x|} e^{-i \omega t}                                  \\
          \Psi(x, 0) & = A e^{-\lambda |x|}                                                  \\
          1          & = A^2 \int_{-\infty}^\infty e^{-2 \lambda |x|} \,d x                  \\
                     & = 2 A^2 \int_0^\infty e^{-2 \lambda x} \,d x                          \\
                     & = 2 A^2 \left[ -\frac{1}{2 \lambda} e^{-2 \lambda x} \right]_0^\infty \\
                     & = \frac{A^2}{\lambda}                                                 \\
          A          & = \sqrt{\lambda}
        \end{align*}

  \item

        \begin{align*}
          \langle x \rangle   & = \int_{-\infty}^\infty x \lambda e^{-2 \lambda |x|} \,d x   \\
                              & = \lambda \int_{-\infty}^\infty x e^{-2 \lambda |x|} \,d x   \\
                              & = 0                                                          \\
          \langle x^2 \rangle & = \int_{-\infty}^\infty x^2 \lambda e^{-2 \lambda |x|} \,d x \\
                              & = 2 \lambda \int_0^\infty x^2 e^{-2 \lambda x} \,d x         \\
                              & = \frac{1}{2 \lambda^2}
        \end{align*}

  \item

        \begin{align*}
          \sigma                                                    & = \sqrt{\langle x^2 \rangle - \langle x \rangle^2}                            \\
                                                                    & = \frac{1}{\sqrt{2} \lambda}                                                  \\
          1 - \int_{-\sigma}^\sigma \lambda e^{-2 \lambda |x|} \,dx & = 1 - 2 \lambda \int_0^\sigma e^{-2 \lambda x} \,d x                          \\
                                                                    & = 1 - 2 \lambda \left[ -\frac{1}{2 \lambda} e^{-2 \lambda x} \right]_0^\sigma \\
                                                                    & = e^{-2 \lambda \sigma}                                                       \\
                                                                    & = e^{-\sqrt{2}}                                                               \\
                                                                    & \approx 0.243
        \end{align*}
\end{enumerate}

\subsection{}

The chain rule requires that you apply it to both $x$ and $|\Psi|^2$ which gives the same result

\begin{align*}
  \frac{d \ev{x}}{d t} & = \frac{d}{d t} \int x |\Psi|^2 \,d x                                                 \\
                       & = \int \frac{d}{d t} (x |\Psi|^2) \,d x                                               \\
                       & = \int \left( 0 \cdot |\Psi|^2 + x \frac{\partial |\Psi|^2}{\partial t} \right) \,d x \\
                       & = \int x \frac{\partial |\Psi|^2}{\partial t} \,d x
\end{align*}

\setcounter{subsection}{7}
\subsection{}

\begin{align*}
  i \hbar \frac{\partial}{\partial t} \left( e^{-i V_0 t / \hbar} \Psi \right)                                                  & = -\frac{\hbar^2}{2 m} \frac{\partial^2}{\partial x^2} \left( e^{-i V_0 t / \hbar} \Psi \right) + (V + V_0) \left( e^{-i V_0 t / \hbar} \Psi \right) \\
  i \hbar \left( -\frac{i V_0}{\hbar} e^{-i V_0 t / \hbar} \Psi + e^{-i V_0 t / \hbar} \frac{\partial \Psi}{\partial t} \right) & = -\frac{\hbar^2}{2 m} e^{-i V_0 t / \hbar} \frac{\partial^2 \Psi}{\partial x^2} + V e^{-i V_0 t / \hbar} \Psi + V_0 e^{-i V_0 t / \hbar} \Psi       \\
  V_0 \Psi + i \hbar \frac{\partial \Psi}{\partial t}                                                                           & = -\frac{\hbar^2}{2 m} \frac{\partial^2 \Psi}{\partial x^2} + V \Psi + V_0 \Psi                                                                      \\
  i \hbar \frac{\partial \Psi}{\partial t}                                                                                      & = -\frac{\hbar^2}{2 m} \frac{\partial^2 \Psi}{\partial x^2} + V \Psi                                                                                 \\
\end{align*}

\begin{align*}
  \ev{Q(x, p)} & = \int \left( e^{-i V_0 t / \hbar} \Psi \right)^* \left[ Q(x, -i \hbar \partial / \partial x) \right] e^{-i V_0 t / \hbar} \Psi \,d x \\
               & = \int e^{i V_0 t / \hbar} \Psi^* \left[ Q(x, -i \hbar \partial / \partial x) \right] e^{-i V_0 t / \hbar} \Psi \,d x                 \\
               & = \int \Psi^* [Q(x, -i \hbar \partial / \partial x)] \Psi \,d x
\end{align*}

No effect on the expectation value.

\subsection{}

\begin{enumerate}
  \item

        \begin{align*}
          \Psi(x, t) & = A e^{-a [(m x^2 / \hbar) + i t]}                         \\
          1          & = A^2 \int_{-\infty}^\infty e^{-2 a (m x^2 / \hbar)} \,d x \\
                     & = A^2 \int_{-\infty}^\infty e^{-2 a (m x^2 / \hbar)} \,d x \\
                     & = A^2 \sqrt{\frac{\pi \hbar}{2 a m}}                       \\
          A^2        & = \sqrt{\frac{2 a m}{\pi \hbar}}                           \\
          A          & = \left( \frac{2 a m}{\pi \hbar} \right)^{1 / 4}
        \end{align*}

  \item

        \begin{align*}
          \Psi                                 & = A e^{-a [(m x^2 / \hbar) + i t]}                                                                    \\
          \frac{\partial \Psi}{\partial t}     & = -i a \Psi                                                                                           \\
          \frac{\partial \Psi}{\partial x}     & = -\frac{2 a m x}{\hbar} \Psi                                                                         \\
          \frac{\partial^2 \Psi}{\partial x^2} & = -\frac{2 a m}{\hbar} \left( \Psi + x \frac{\partial \Psi}{\partial x} \right)                       \\
                                               & = -\frac{2 a m}{\hbar} \left( 1 - \frac{2 a m x^2}{\hbar} \right) \Psi                                \\
          V \Psi                               & = i \hbar \frac{\partial \Psi}{\partial t} + \frac{\hbar^2}{2 m} \frac{\partial^2 \Psi}{\partial x^2} \\
                                               & = a \hbar \Psi - a \hbar \left( 1 - \frac{2 a m x^2}{\hbar} \right) \Psi                              \\
          V                                    & = a \hbar - a \hbar + 2 a^2 m x^2                                                                     \\
                                               & = 2 a^2 m x^2
        \end{align*}

  \item

        \begin{align*}
          \ev{x}   & = A^2 \int_{-\infty}^\infty e^{-2 a (m x^2 / \hbar)} x \,d x                                                                                                                            \\
                   & = 0                                                                                                                                                                                     \\
          \ev{x^2} & = A^2 \int_{-\infty}^\infty e^{-2 a (m x^2 / \hbar)} x^2 \,d x                                                                                                                          \\
                   & = 2 A^2 \int_0^\infty e^{-2 a (m x^2 / \hbar)} x^2 \,d x                                                                                                                                \\
                   & = \frac{\hbar}{4 a m}                                                                                                                                                                   \\
          \ev{p}   & = \int_{-\infty}^\infty \Psi^* \left[ -i \hbar \frac{\partial}{\partial x} \right] \Psi \,d x                                                                                           \\
                   & = -i \hbar \int_{-\infty}^\infty A e^{-a [(m x^2 / \hbar) - i t]} \left( -\frac{2 a m x}{\hbar} A e^{-a [(m x^2 / \hbar) + i t]} \right) \,d x                                          \\
                   & = 2 i A^2 a m \int_{-\infty}^\infty x e^{-2 a m x^2 / \hbar} \,d x                                                                                                                      \\
                   & = 0                                                                                                                                                                                     \\
          \ev{p^2} & = \int_{-\infty}^\infty \Psi^* \left[ -\hbar^2 \frac{\partial^2}{\partial x^2} \right] \Psi \,d x                                                                                       \\
                   & = -\hbar^2 \int_{-\infty}^\infty A e^{-a [(m x^2 / \hbar) - i t]} \left[ -\frac{2 a m}{\hbar} \left( 1 - \frac{2 a m x^2}{\hbar} \right) A e^{-a [(m x^2 / \hbar) + i t]} \right] \,d x \\
                   & = 2 A^2 a m \hbar \int_{-\infty}^\infty e^{-2 a m x^2 / \hbar} \left( 1 - \frac{2 a m x^2}{\hbar} \right) \,d x                                                                         \\
                   & = a m \hbar
        \end{align*}

  \item

        \begin{align*}
          \sigma_x          & = \sqrt{\ev{x^2} - \ev{x}^2} \\
                            & = \sqrt{\frac{\hbar}{4 a m}} \\
          \sigma_p          & = \sqrt{a m \hbar}           \\
          \sigma_x \sigma_p & = \sqrt{\frac{1}{4} \hbar^2} \\
                            & = \frac{1}{2} \hbar          \\
                            & \ge \frac{1}{2} \hbar
        \end{align*}

        Yes, this is consistent with Heisenberg's uncertainty principle.
\end{enumerate}

\subsection{}

\begin{enumerate}
  \item

        \begin{align*}
          P(0) & = 0            \\
          P(1) & = \frac{2}{25} \\
               & = 0.08         \\
          P(2) & = \frac{3}{25} \\
               & = 0.12         \\
          P(3) & = \frac{1}{5}  \\
               & = 0.2          \\
          P(4) & = \frac{3}{25} \\
               & = 0.12         \\
          P(5) & = \frac{3}{25} \\
               & = 0.2          \\
          P(6) & = \frac{3}{25} \\
               & = 0.2          \\
          P(7) & = \frac{1}{25} \\
               & = 0.04         \\
          P(8) & = \frac{2}{25} \\
               & = 0.08         \\
          P(9) & = \frac{3}{25} \\
               & = 0.12
        \end{align*}

  \item The most probable digit is $3$, the median digit is $4$, and the average value is $\frac{118}{25} = 4.72$.

  \item $\sigma = 2.474$
\end{enumerate}

\setcounter{subsection}{13}
\subsection{}

\begin{enumerate}
  \item

        \begin{align*}
          P_{a b}(t)            & = \int_a^b |\Psi(x, t)|^2 \,d x                                                                                                                                                  \\
          \frac{d P_{a b}}{d t} & = \frac{d}{d t} \int_a^b |\Psi(x, t)|^2 \,d x                                                                                                                                    \\
                                & = \int_a^b \frac{d}{d t} \left( |\Psi(x, t)|^2 \right) \,d x                                                                                                                     \\
                                & = \int_a^b \frac{\partial}{\partial x} \left[ \frac{i \hbar}{2 m} \left( \Psi^* \frac{\partial \Psi}{\partial x} - \frac{\partial \Psi^*}{\partial x} \Psi \right) \right] \,d x \\
                                & = J(a, t) - J(b, t)
        \end{align*}

        The units are $\unit{s^{-1}}$.

  \item

        \begin{align*}
          \Psi(x, t)                         & = A e^{-a [(m x^2 / \hbar) + i t]}                                                                                                         \\
          \frac{\partial \Psi}{\partial x}   & = -\frac{2 a m x}{\hbar} \Psi                                                                                                              \\
          \Psi^*(x, t)                       & = A e^{-a [(m x^2 / \hbar) - i t]}                                                                                                         \\
          \frac{\partial \Psi^*}{\partial x} & = -\frac{2 a m x}{\hbar} \Psi^*                                                                                                            \\
          J(x, t)                            & = \frac{i \hbar}{2 m} \left( \Psi \frac{\partial \Psi^*}{\partial x} - \Psi^* \frac{\partial \Psi}{\partial x} \right)                     \\
                                             & = \frac{i \hbar}{2 m} \left[ \Psi \left( -\frac{2 a m x}{\hbar} \Psi^* \right) - \Psi^* \left( -\frac{2 a m x}{\hbar} \Psi \right) \right] \\
                                             & = 0
        \end{align*}
\end{enumerate}

\subsection{}

\begin{align*}
  \frac{d}{d t} \int_{-\infty}^\infty \Psi_1^* \Psi_2 \,d x & = \int_{-\infty}^\infty \left( \frac{\partial \Psi_1^*}{\partial t} \Psi_2 + \Psi_1^* \frac{\partial \Psi_2}{\partial t} \right) \,d x                                                       \\
                                                            & = \int_{-\infty}^\infty \left[ \left( -i \frac{\hbar}{2 m} \frac{\partial^2 \Psi_1^*}{\partial x^2} + i \frac{V}{\hbar} \Psi_1^* \right) \Psi_2 \right.                                      \\
                                                            & \qquad \left. + \Psi_1^* \left( i \frac{\hbar}{2 m} \frac{\partial^2 \Psi_2}{\partial x^2} - i \frac{V}{\hbar} \Psi_2 \right) \right] \,d x                                                  \\
                                                            & = i \frac{\hbar}{2 m} \int_{-\infty}^\infty \left( \Psi_1^* \frac{\partial^2 \Psi_2}{\partial x^2} - \frac{\partial^2 \Psi_1^*}{\partial x^2} \Psi_2 \right) \,d x                           \\
                                                            & = i \frac{\hbar}{2 m} \left[ \left. \Psi_1^* \frac{\partial \Psi_2}{\partial x} \right|_{-\infty}^\infty - \int_{-\infty}^\infty \frac{\partial}{\partial x} (\Psi_1^* \Psi_2) \,d x \right. \\
                                                            & \qquad \left. \left. \frac{\partial \Psi_1^*}{\partial x} \Psi_2 \right|_{-\infty}^\infty - \int_{-\infty}^\infty \frac{\partial}{\partial x} (\Psi_1^* \Psi_2) \,d x \right]                \\
                                                            & = 0
\end{align*}

\subsection{}

\begin{enumerate}
  \item

        \begin{align*}
          1 & = \int_{-a}^a A^2 (a^2 - x^2)^2 \,d x \\
            & = A^2 \int_0^a (a^2 - x^2)^2 \,d x    \\
            & = \frac{16}{15} A^2 a^5               \\
          A & = \sqrt{\frac{15}{16 a^5}}
        \end{align*}

  \item

        \begin{align*}
          \ev{x} & = \int_{-a}^a x A (a^2 - x^2) \,d x \\
                 & = 0
        \end{align*}

  \item

        \begin{align*}
          \ev{p} & = \int_{-a}^a \Psi^* \left( -i \hbar \frac{\partial}{\partial x} \right) \Psi \,d x \\
                 & = 2 i A^2 \hbar \int_{-a}^a x (a^2 - x^2) \,d x                                     \\
                 & = 0
        \end{align*}

  \item

        \begin{align*}
          \ev{x^2} & = \int_{-a}^a \Psi^* x^2 \Psi \,d x       \\
                   & = A^2 \int_{-a}^a x^2 (a^2 - x^2)^2 \,d x \\
                   & = A^2 \frac{16}{105} a^7                  \\
                   & = \frac{a^2}{7}
        \end{align*}

  \item

        \begin{align*}
          \ev{p^2} & = \int_{-a}^a \Psi^* \left( -\hbar^2 \frac{\partial^2}{\partial x^2} \right) \Psi \,d x \\
                   & = -\hbar^2 \int_{-a}^a A (a^2 - x^2) (-2 A) \,d x                                       \\
                   & = 4 A^2 \hbar^2 \int_0^a (a^2 - x^2) \,d x                                              \\
                   & = 4 A^2 \hbar^2 \left[ a^2 x - \frac{1}{3} x^3 \right]_0^a                              \\
                   & = 4 A^2 \hbar^2 \left( a^3 - \frac{1}{3} a^3 \right)                                    \\
                   & = \frac{8}{3} A^2 a^3 \hbar^2                                                           \\
                   & = \frac{8}{3} \frac{15}{16 a^5} a^3 \hbar^2                                             \\
                   & = \frac{5}{2} \frac{\hbar^2}{a^2}
        \end{align*}

  \item

        \begin{align*}
          \sigma_x & = \sqrt{\ev{x^2} - \ev{x}^2} \\
                   & = \sqrt{\frac{a^2}{7}}       \\
                   & = \frac{a}{\sqrt{7}}
        \end{align*}

  \item

        \begin{align*}
          \sigma_p & = \sqrt{\ev{p^2} - \ev{p}^2}         \\
                   & = \sqrt{\frac{5}{2}} \frac{\hbar}{a}
        \end{align*}

  \item

        \begin{align*}
          \sigma_x \sigma_p & = \sqrt{\frac{5}{14}} \hbar \\
                            & \ge \frac{1}{2} \hbar
        \end{align*}
\end{enumerate}

\setcounter{subsection}{17}
\subsection{}

\begin{enumerate}
  \item

        \begin{align*}
          % Electron
          \frac{h}{\sqrt{3 m k_B T}} & > d                       \\
          \frac{\sqrt{3 m k_B T}}{h} & < \frac{1}{d}             \\
          T_\text{electron}          & < \frac{h^2}{3 d^2 m k_B} \\
                                     & < \qty{1.3e5}{K}          \\
          % Nuclei
          T_\text{nuclei}            & < \qty{2.5}{K}
        \end{align*}

  \item

        \begin{align*}
          P V                        & = N k_B T                                                        \\
          \frac{V}{N}                & = \frac{k_B T}{P}                                                \\
          d                          & = \left( \frac{k_B T}{P} \right)^{1 / 3}                         \\
          \frac{h}{\sqrt{3 m k_B t}} & > \left( \frac{k_B T}{P} \right)^{1 / 3}                         \\
          T                          & < \frac{1}{k_B} \left( \frac{h^2}{3 m} \right)^{3 / 5} P^{2 / 5}
        \end{align*}
\end{enumerate}

\section{Time-Independent Schrödinger Equation}

\subsection{}

\begin{enumerate}
  \item

        \begin{align*}
          \int_{-\infty}^\infty |\Psi|^2 \,d x & = \int_{-\infty}^\infty \Psi^* \Psi \,d x                                                                    \\
                                               & = \int_{-\infty}^\infty \psi^* e^{i (E_0 - i \Gamma) t / \hbar} \psi e^{-i (E_0 + i \Gamma) t / \hbar} \,d x \\
                                               & = e^{2 \Gamma t / \hbar} \int_{-\infty}^\infty |\psi|^2 \,d x
        \end{align*}

        In order for this to equal $1$ for all $t$, $\Gamma$ must be $0$.

  \item If $\psi(x)$ is a complex solution to the time-independent Schrödinger equation then so is $\psi^*(x)$ and $\psi(x) + \psi^*(x)$ which is real.
\end{enumerate}

\subsection{}

If $\psi$ and its second derivative always have the same sign, $\psi$ will increase or decrease without bound forever. This means there is no non-zero choice of constant $A$ such that \[\int_{-\infty}^\infty |A \Psi|^2 \,d x = 1\] and thus the equation can't be normalised.

The classical analog of this is statements is that the potential energy of a system can't exceed its total energy.

\subsection{}

The time-independent Schrödinger equation in an infinite square well is \[-\frac{\hbar^2}{2 m} \frac{d^2 \psi}{d x^2} = E \psi.\]

If $E = 0$ then $\psi = A x + B$ which isn't normalisable.

If $E < 0$ then $\psi = A e^{k t} + B e^{-k t}$ where $k \in \mathbb{R}$ which also isn't normalisable.

\subsection{}

\begin{align*}
  \Psi_n(x, t) & = \sqrt{\frac{2}{a}} \sin \left( \frac{n \pi}{a} x \right) e^{-i (n^2 \pi^2 \hbar / 2 m a^2) t}                           \\
  \ev{x}       & = \int_0^a \Psi_n^* x \Psi_n \,d x                                                                                        \\
               & = \frac{2}{a} \int_0^a x \sin^2 \left( \frac{n \pi}{a} x \right) \,d x                                                    \\
               & = \frac{a}{2}                                                                                                             \\
  \ev{x^2}     & = \int_0^a \Psi_n^* x^2 \Psi_n \,d x                                                                                      \\
               & = \frac{2}{a} \int_0^a x^2 \sin^2 \left( \frac{n \pi}{a} x \right) \,d x                                                  \\
               & = a^2 \left( \frac{1}{3} - \frac{1}{2 n^2 \pi^2} \right)                                                                  \\
  \ev{p}       & = \int_0^a \Psi_n^* \left( -i \hbar \frac{\partial}{\partial x} \right) \Psi_n \,d x                                      \\
               & = -i \frac{2 \hbar n \pi}{a^2} \int_0^a \sin \left( \frac{n \pi}{a} x \right) \cos \left( \frac{n \pi}{a} x \right) \,d x \\
               & = 0                                                                                                                       \\
  \ev{p^2}     & = \int_0^a \Psi_n^* \left( -\hbar^2 \frac{\partial^2}{\partial x^2} \right) \Psi_n \,d x                                  \\
               & = \frac{2 \hbar^2 n^2 \pi^2}{a^3} \int_0^a \sin^2 \left( \frac{n \pi}{a} x \right) \,d x                                  \\
               & = \left( \frac{n \pi \hbar}{a} \right)^2                                                                                  \\
  \sigma_x     & = \sqrt{\ev{x^2} - \ev{x}^2}                                                                                              \\
               & = \frac{a}{2} \sqrt{\frac{1}{3} - \frac{2}{n^2 \pi^2}}                                                                    \\
  \sigma_p     & = \sqrt{\ev{p^2} - \ev{p}^2}                                                                                              \\
               & = \frac{n \pi \hbar}{a}
\end{align*}

\subsection{}

\begin{enumerate}
  \item

        \begin{align*}
          1 & = \int_0^a A^2 (\psi_1 + \psi_2)^2 \,d x                                                                                                       \\
            & = A^2 \int_0^a (\psi_1^2 + 2 \psi_1 \psi_2 + \psi_2^2) \,d x                                                                                   \\
            & = \frac{2 A^2}{a} \left[ \int_0^a \sin^2 \left( \frac{\pi}{a} x \right) \,d x + \int_0^a \sin^2 \left( \frac{2 \pi}{a} x \right) \,d x \right] \\
            & = 2 A^2                                                                                                                                        \\
          A & = \frac{1}{\sqrt{2}}
        \end{align*}

  \item

        \begin{align*}
          \Psi(x, t)     & = \frac{1}{\sqrt{2}} \left[ \sqrt{\frac{2}{a}} \sin \left( \frac{\pi}{a} x \right) e^{-i \omega t} + \sqrt{\frac{2}{a}} \sin \left( \frac{2 \pi}{a} x \right) e^{-4 i \omega t} \right] \\
          |\Psi(x, t)|^2 & = \Psi^* \Psi                                                                                                                                                                           \\
                         & = \frac{1}{a} \left[ \sin \left( \frac{\pi}{a} x \right) e^{i \omega t} + \sin \left( \frac{2 \pi}{a} x \right) e^{4 i \omega t} \right]                                                \\
                         & \qquad \left[ \sin \left( \frac{\pi}{a} x \right) e^{-i \omega t} + \sin \left( \frac{2 \pi}{a} x \right) e^{-4 i \omega t} \right]                                                     \\
                         & = \frac{1}{a} \left[ \sin^2 \left( \frac{\pi}{a} x \right) + \sin \left( \frac{\pi}{a} x \right) \sin \left( \frac{2 \pi}{a} x \right) e^{-3 i \omega t} \right.                        \\
                         & \qquad \left. + \sin \left( \frac{\pi}{a} x \right) \sin \left( \frac{2 \pi}{a} x \right) e^{3 i \omega t} + \sin^2 \left( \frac{2 \pi}{a} x \right) \right]                            \\
                         & = \frac{1}{a} \left[ \sin^2 \left( \frac{\pi}{a} x \right) + \sin^2 \left( \frac{2 \pi}{a} x \right) \right.                                                                            \\
                         & \qquad \left. + 2 \sin \left( \frac{\pi}{a} x \right) \sin \left( \frac{2 \pi}{a} x \right) \cos (3 \omega t) \right]
        \end{align*}

  \item

        \begin{align*}
          \ev{x} & = \int_0^a \Psi^* x \Psi \,d x                                        \\
                 & = \int_0^a x |\Psi|^2 \,d x                                           \\
                 & = \frac{a}{2} \left[ 1 - \frac{32}{9 \pi^2} \cos (3 \omega t) \right]
        \end{align*}

  \item

        \begin{align*}
          \ev{p} & = m \frac{d \ev{x}}{d t}                          \\
                 & = \frac{16 a m \omega}{3 \pi^2} \sin (3 \omega t) \\
                 & = \frac{8 \hbar}{3 a} \sin (3 \omega t)
        \end{align*}

  \item You can get $E_1$ or $E_2$ and the probability of getting each is $1 / 2$.

        $H = \frac{1}{2} (E_1 + E_2)$ is the mean of the two possible energy values.
\end{enumerate}

\subsection{}

\begin{align*}
  \Psi(x, 0) & = A [\psi_1 + e^{i \phi} \psi_2]                                                                                                                                       \\
  1          & = \int_0^a |\Psi|^2 \,d x                                                                                                                                              \\
             & = \int_0^a \Psi^* \Psi \,d x                                                                                                                                           \\
             & = A^2 \int_0^a (\psi_1 + e^{-i \phi} \psi_2) (\psi_1 + e^{i \phi} \psi_2) \,d x                                                                                        \\
             & = A^2 \int_0^a (\psi_1^2 + e^{i \phi} \psi_1 \psi_2 + e^{-i \phi} \psi_1 \psi_2 + \psi_2^2) \,d x                                                                      \\
             & = \frac{2 A^2}{a} \int_0^a \left[ \sin^2 \left( \frac{\pi}{a} x \right) + e^{i \phi} \sin \left( \frac{\pi}{a} x \right) \sin \left( \frac{2 \pi}{a} x \right) \right. \\
             & \qquad \left. e^{-i \phi} \sin \left( \frac{\pi}{a} x \right) \sin \left( \frac{2 \pi}{a} x \right) + \sin^2 \left( \frac{2 \pi}{a} x \right) \right] \,d x            \\
             & = \frac{2 A^2}{a} \int_0^a \left[ \sin^2 \left( \frac{\pi}{a} x \right) + \sin \left( \frac{\pi}{a} x \right) \sin \left( \frac{2 \pi}{a} x \right) \cos \phi \right.  \\
             & \qquad \left. + \sin^2 \left( \frac{2 \pi}{a} x \right) \right] \,d x                                                                                                  \\
             & = 2 A^2                                                                                                                                                                \\
  A          & = \frac{1}{\sqrt{2}}                                                                                                                                                   \\
  \Psi(x, t) & = \frac{1}{\sqrt{a}} \left[ \sin \left( \frac{\pi}{a} x \right) e^{-i \omega t} + \sin \left( \frac{2 \pi}{a} x \right) e^{i (\phi - 4 \omega t)} \right]
\end{align*}

\begin{align*}
  |\Psi|^2 & = \Psi^* \Psi                                                                                                                                                            \\
           & = \frac{1}{a} \left[ \sin \left( \frac{\pi}{a} x \right) e^{i \omega t} + \sin \left( \frac{2 \pi}{a} x \right) e^{-i (\phi - 4 \omega t)} \right]                       \\
           & \qquad \left[ \sin \left( \frac{\pi}{a} x \right) e^{-i \omega t} + \sin \left( \frac{2 \pi}{a} x \right) e^{i (\phi - 4 \omega t)} \right]                              \\
           & = \frac{1}{a} \left[ \sin^2 \left( \frac{\pi}{a} x \right) + \sin \left( \frac{\pi}{a} x \right) \sin \left( \frac{2 \pi}{a} x \right) e^{i (\phi - 3 \omega t)} \right. \\
           & \qquad \left. \sin \left( \frac{\pi}{a} x \right) \sin \left( \frac{2 \pi}{a} x \right) e^{-i (\phi - 3 \omega t)} + \sin^2 \left( \frac{2 \pi}{a} x \right) \right]     \\
           & = \frac{1}{a} \left[ \sin^2 \left( \frac{\pi}{a} x \right) + \sin^2 \left( \frac{2 \pi}{a} x \right) \right.                                                             \\
           & \qquad \left. + 2 \sin \left( \frac{\pi}{a} x \right) \sin \left( \frac{2 \pi}{a} x \right) \cos (\phi - 3 \omega t) \right]                                             \\
  \ev{x}   & = \int_0^a \Psi^* x \Psi \,d x                                                                                                                                           \\
           & = \int_0^a x |\Psi|^2 \,d x                                                                                                                                              \\
           & = \frac{a}{2} \left[ 1 - \frac{32}{9 \pi^2} \cos (3 \omega t - \phi) \right]
\end{align*}

\subsection{}

\begin{enumerate}
  \item

        \begin{align*}
          1 & = \int_0^a |\Psi|^2 \,d x                                                                                         \\
            & = A^2 \left[ \int_0^{a / 2} x^2 \,d x + \int_{a / 2}^a (a - x)^2 \,d x \right]                                    \\
            & = A^2 \left\{ \frac{1}{3} \left[ \frac{a}{2} \right]^3 + \left[ -\frac{1}{3} (a - x)^3 \right]_{a / 2}^a \right\} \\
            & = A^2 \left( \frac{a^3}{24} + \frac{a^3}{24} \right)                                                              \\
            & = \frac{A^2 a^3}{12}                                                                                              \\
          A & = \frac{2 \sqrt{3}}{\sqrt{a^3}}
        \end{align*}

  \item

        \begin{align*}
          c_n        & = \sqrt{\frac{2}{a}} \int_0^a \sin \left( \frac{n \pi}{a} x \right) \Psi(x, 0) \,d x                                                                                                        \\
                     & = \sqrt{\frac{2}{a}} \left[ \int_0^{a / 2} \sin \left( \frac{n \pi}{a} x \right) A x \,d x + \int_{a / 2}^a \sin \left( \frac{n \pi}{a} x \right) A (a - x) \,d x \right]                   \\
                     & = \frac{2 \sqrt{6}}{a^2} \left[ \int_0^{a / 2} x \sin \left( \frac{n \pi}{a} x \right) \,d x + \int_{a / 2}^a (a - x) \sin \left( \frac{n \pi}{a} x \right) \,d x \right]                   \\
                     & = \frac{8 \sqrt{6}}{n^2 \pi^2} \sin^2 \left( \frac{n \pi}{4} \right) \sin \left( \frac{n \pi}{2} \right)                                                                                    \\
                     & = \begin{cases}
                           0                                               & n \text{ even} \\
                           (-1)^{(n - 1) / 2} \frac{4 \sqrt{6}}{n^2 \pi^2} & n \text{ odd}
                         \end{cases}                                                                                                           \\
          \Psi(x, t) & = \frac{4 \sqrt{6}}{\pi^2} \sqrt{\frac{2}{a}} \sum_{n = 1, 3, 5, \ldots}^\infty (-1)^{(n - 1) / 2} \frac{1}{n^2} \sin \left( \frac{n \pi}{a} x \right) e^{-i (n^2 \pi^2 \hbar / 2 m a^2) t} \\
        \end{align*}

  \item

        \begin{align*}
          |c_1|^2 & = \left( \frac{4 \sqrt{6}}{\pi^2} \right)^2 \\
                  & \approx 0.985
        \end{align*}

  \item

        \begin{align*}
          E_n    & = \frac{n^2 \pi^2 \hbar^2}{2 m a^2}                                                                                   \\
          \ev{H} & = \sum_{n = 0}^\infty |c_{2 n + 1}|^2 E_{2 n + 1}                                                                     \\
                 & = \sum_{n = 0}^\infty \left( \frac{4 \sqrt{6}}{(2 n + 1)^2 \pi^2} \right)^2 \frac{(2 n + 1)^2 \pi^2 \hbar^2}{2 m a^2} \\
                 & = \sum_{n = 0}^\infty \frac{48 \hbar^2}{(2 n + 1)^2 m a^2 \pi^2}                                                      \\
                 & = \frac{48 \hbar^2}{m a^2 \pi^2} \sum_{n = 0}^\infty \frac{1}{(2 n + 1)^2}                                            \\
                 & = \frac{6 \hbar^2}{m a^2}
        \end{align*}
\end{enumerate}

\subsection{}

\begin{align*}
  1       & = \int_0^{a / 2} |\Psi|^2 \,d x                                          \\
          & = A^2 \int_0^{a / 2} \,d x                                               \\
          & = \frac{a A^2}{2}                                                        \\
  A       & = \sqrt{\frac{2}{a}}                                                     \\
  c_n     & = \frac{2}{a} \int_0^{a / 2} \sin \left( \frac{n \pi}{a} x \right) \,d x \\
  |c_1|^2 & = \left( \frac{2}{\pi} \right)^2                                         \\
          & \approx 0.405
\end{align*}

\subsection{}

\begin{align*}
  \Psi(x, 0) & = A x (a - x)                                                                                                \\
  \ev{H}     & = \int_0^a \Psi(x, 0)^* \hat{H} \Psi(x, 0) \,d x                                                             \\
             & = \int_0^a \Psi(x, 0)^* \left( -\frac{\hbar^2}{2 m} \frac{\partial^2}{\partial x^2} \right) \Psi(x, 0) \,d x \\
             & = \frac{A^2 \hbar^2}{m} \int_0^a x (a - x) \,d x                                                             \\
             & = \frac{30 \hbar^2}{m a^5} \frac{a^3}{6}                                                                     \\
             & = \frac{5 \hbar^2}{m a^2}
\end{align*}

\subsection{}

\begin{enumerate}
  \item

        \begin{align*}
          \psi_2(x) & = \frac{1}{\sqrt{2!}} (\hat{a}_+) \psi_1                                                                                                                                                                                                                            \\
                    & = \frac{1}{\sqrt{2}} \frac{1}{\sqrt{2 \hbar m \omega}} \left( -\hbar \frac{d}{d x} + m \omega x \right) \left( \frac{m \omega}{\pi \hbar} \right)^{1 / 4} \sqrt{\frac{2 m \omega}{\hbar}} x e^{-\frac{m \omega}{2 \hbar} x^2}                                       \\
                    & = \frac{1}{\sqrt{2} \hbar} \left( \frac{m \omega}{\pi \hbar} \right)^{1 / 4} \left( -\hbar \frac{d}{d x} + m \omega x \right) x e^{-\frac{m \omega}{2 \hbar} x^2}                                                                                                   \\
                    & = \frac{1}{\sqrt{2} \hbar} \left( \frac{m \omega}{\pi \hbar} \right)^{1 / 4} \left[ -\hbar \left( e^{-\frac{m \omega}{2 \hbar} x^2} - \frac{m \omega}{\hbar} x^2 e^{-\frac{m \omega}{2 \hbar} x^2} \right) + m \omega x^2 e^{-\frac{m \omega}{2 \hbar} x^2} \right] \\
                    & = \frac{1}{\sqrt{2}} \left( \frac{m \omega}{\pi \hbar} \right)^{1 / 4} \left( \frac{2 m \omega}{\hbar} x^2 - 1 \right) e^{-\frac{m \omega}{2 \hbar} x^2}
        \end{align*}
\end{enumerate}

\subsection{}

\begin{enumerate}
  \item

        % \psi_0
        \begin{align*}
          \ev{x}   & = \int_{-\infty}^\infty \psi_0^* x \psi_0 \,d x                                                                                                                                                                                                                  \\
                   & = \alpha^2 \int_{-\infty}^\infty x e^{-\frac{m \omega}{\hbar} x^2} \,d x                                                                                                                                                                                         \\
                   & = 0                                                                                                                                                                                                                                                              \\
          \ev{p}   & = m \frac{d \ev{x}}{d t}                                                                                                                                                                                                                                         \\
                   & = 0                                                                                                                                                                                                                                                              \\
          \ev{x^2} & = \int_{-\infty}^\infty \psi_0^* x^2 \psi_0 \,d x                                                                                                                                                                                                                \\
                   & = \alpha^2 \int_{-\infty}^\infty x^2 e^{-\frac{m \omega}{\hbar} x^2} \,d x                                                                                                                                                                                       \\
                   & = \frac{\hbar}{2 m \omega}                                                                                                                                                                                                                                       \\
          \ev{p^2} & = \int_{-\infty}^\infty \psi_0^* \left( -\hbar^2 \frac{d^2}{d x^2} \right) \psi_0 \,d x                                                                                                                                                                          \\
                   & = -\hbar^2 \left( \frac{m \omega}{\pi \hbar} \right)^{1 / 2} \int_{-\infty}^\infty e^{-\frac{m \omega}{2 \hbar} x^2} \frac{d}{d x} \left( -\frac{m \omega}{\hbar} x e^{-\frac{m \omega}{2 \hbar} x^2} \right) \,d x                                              \\
                   & = \hbar^2 \left( \frac{m \omega}{\pi \hbar} \right)^{1 / 2} \frac{m \omega}{\hbar} \int_{-\infty}^\infty e^{-\frac{m \omega}{2 \hbar} x^2} \left( e^{-\frac{m \omega}{2 \hbar} x^2} - \frac{m \omega}{\hbar} x^2 e^{-\frac{m \omega}{2 \hbar} x^2} \right) \,d x \\
                   & = \hbar^2 \left( \frac{m \omega}{\pi \hbar} \right)^{1 / 2} \frac{m \omega}{\hbar} \int_{-\infty}^\infty \left( 1 - \frac{m \omega}{\hbar} x^2 \right) e^{-\frac{m \omega}{\hbar} x^2} \,d x                                                                     \\
                   & = \hbar^2 \left( \frac{m \omega}{\pi \hbar} \right)^{1 / 2} \frac{m \omega}{\hbar} \frac{\hbar \sqrt{\pi}}{2 \sqrt{\hbar m \omega}}                                                                                                                              \\
                   & = \frac{1}{2} m \hbar \omega
        \end{align*}

        % \psi_1
        \begin{align*}
          \psi_1(x) & = \left( \frac{m \omega}{\pi \hbar} \right)^{1 / 4} \sqrt{\frac{2 m \omega}{\hbar}} x e^{-\frac{m \omega}{2 \hbar} x^2}                                                                                                                                                             \\
          \ev{x}    & = 0                                                                                                                                                                                                                                                                                 \\
          \ev{p}    & = m \frac{d \ev{x}}{d t}                                                                                                                                                                                                                                                            \\
                    & = 0                                                                                                                                                                                                                                                                                 \\
          \ev{x^2}  & = \int_{-\infty}^\infty \psi_1^* x^2 \psi_1 \,d x                                                                                                                                                                                                                                   \\
                    & = \left( \frac{m \omega}{\pi \hbar} \right)^{1 / 2} \frac{2 m \omega}{\hbar} \int_{-\infty}^\infty x^4 e^{-\frac{m \omega}{\hbar} x^2} \,d x                                                                                                                                        \\
                    & = \left( \frac{m \omega}{\pi \hbar} \right)^{1 / 2} \frac{2 m \omega}{\hbar} \frac{3}{4} \sqrt{\pi} \left( \frac{\hbar}{m \omega} \right)^{5 / 2}                                                                                                                                   \\
                    & = \frac{3}{2} \frac{\hbar}{m \omega}                                                                                                                                                                                                                                                \\
          \ev{p^2}  & = \int_{-\infty}^\infty \psi_1^* \left( -\hbar^2 \frac{d^2}{d x^2} \right) \psi_1 \,d x                                                                                                                                                                                             \\
                    & = -\hbar^2 \left( \frac{m \omega}{\pi \hbar} \right)^{1 / 2} \frac{2 m \omega}{\hbar} \int_{-\infty}^\infty x e^{-\frac{m \omega}{2 \hbar} x^2} \frac{d}{d x} \left( e^{-\frac{m \omega}{2 \hbar} x^2} - \frac{m \omega}{\hbar} x^2 e^{-\frac{m \omega}{2 \hbar} x^2} \right) \,d x \\
                    & = -\hbar^2 \left( \frac{m \omega}{\pi \hbar} \right)^{1 / 2} \frac{2 m \omega}{\hbar} \int_{-\infty}^\infty x e^{-\frac{m \omega}{2 \hbar} x^2} \left[ -\frac{m \omega}{\hbar} x e^{-\frac{m \omega}{2 \hbar} x^2} \right.                                                          \\
                    & \qquad \left. - \frac{2 m \omega}{\hbar} x e^{-\frac{m \omega}{2 \hbar} x^2} + \left( \frac{m \omega}{\hbar} \right)^2 x^3 e^{-\frac{m \omega}{2 \hbar} x^2} \right] \,d x                                                                                                          \\
                    & = 2 \hbar^2 \left( \frac{m \omega}{\pi \hbar} \right)^{1 / 2} \left( \frac{m \omega}{\hbar} \right)^2 \int_{-\infty}^\infty x^2 e^{-\frac{m \omega}{\hbar} x^2} \left( 3 - \frac{m \omega}{\hbar} x^2 \right) \,d x                                                                 \\
                    & = 2 \hbar^2 \left( \frac{m \omega}{\pi \hbar} \right)^{1 / 2} \left( \frac{m \omega}{\hbar} \right)^2 \frac{3}{4} \sqrt{\pi} \left( \frac{h}{m \omega} \right)^{3 / 2}                                                                                                              \\
                    & = \frac{3}{2} m \hbar \omega
        \end{align*}

  \item

        \begin{align*}
          % \psi_0
          \sigma_x          & = \sqrt{\ev{x^2} - \ev{x}^2}        \\
                            & = \sqrt{\frac{\hbar}{2 m \omega}}   \\
          \sigma_p          & = \sqrt{\ev{p^2} - \ev{p}^2}        \\
                            & = \sqrt{\frac{m \hbar \omega}{2}}   \\
          \sigma_x \sigma_p & = \frac{\hbar}{2}                   \\
          % \psi_1
          \sigma_x          & = \sqrt{\frac{3 \hbar}{2 m \omega}} \\
          \sigma_p          & = \sqrt{\frac{3 m \hbar \omega}{2}} \\
          \sigma_x \sigma_p & = \frac{3}{2} \hbar
        \end{align*}

  \item

        \begin{align*}
          % \psi_0
          \ev{T} & = \frac{\ev{p^2}}{2 m}            \\
                 & = \frac{\hbar \omega}{4}          \\
          \ev{V} & = \frac{1}{2} m \omega^2 \ev{x^2} \\
                 & = \frac{1}{4} \hbar \omega        \\
          % \psi_1
          \ev{T} & = \frac{\ev{p^2}}{2 m}            \\
                 & = \frac{3}{4} \hbar \omega        \\
          \ev{V} & = \frac{1}{2} m \omega^2 \ev{x^2} \\
                 & = \frac{3}{4} \hbar \omega
        \end{align*}
\end{enumerate}

\subsection{}

\begin{align*}
  \ev{x}   & = \int_{-\infty}^\infty \psi_n^* x \psi_n \,d x                                                                                                \\
           & = \sqrt{\frac{\hbar}{2 m \omega}} \int_{-\infty}^\infty \psi_n^* (\hat{a}_+ + \hat{a}_-) \psi_n \,d x                                          \\
           & = \sqrt{\frac{\hbar}{2 m \omega}} \int_{-\infty}^\infty \psi_n^* (\sqrt{n + 1} \psi_{n + 1} + \sqrt{n} \psi_{n - 1}) \,d x                     \\
           & = 0                                                                                                                                            \\
  \ev{p}   & = \int_{-\infty}^\infty \psi_n^* p \psi_n \,d x                                                                                                \\
           & = i \sqrt{\frac{\hbar m \omega}{2}} \int_{-\infty}^\infty \psi_n^* (\hat{a}_+ - \hat{a}_-) \psi_n \,d x                                        \\
           & = i \sqrt{\frac{\hbar m \omega}{2}} \int_{-\infty}^\infty \psi_n^* (\sqrt{n + 1} \psi_{n + 1} - \sqrt{n} \psi_{n - 1}) \,d x                   \\
           & = 0                                                                                                                                            \\
  \ev{x^2} & = \int_{-\infty}^\infty \psi_n^* x^2 \psi_n \,d x                                                                                              \\
           & = \frac{\hbar}{2 m \omega} \int_{-\infty}^\infty \psi_n^* (\hat{a}_+^2 + \hat{a}_+ \hat{a}_- + \hat{a}_- \hat{a}_+ + \hat{a}_-^2) \psi_n \,d x \\
           & = \frac{\hbar}{2 m \omega} (2 n + 1) \int_{-\infty}^\infty |\psi_n|^2 \,d x                                                                    \\
           & = \frac{\hbar}{m \omega} \left( n + \frac{1}{2} \right)                                                                                        \\
  \ev{p^2} & = \int_{-\infty}^\infty \psi_n^* p^2 \psi_n \,d x                                                                                              \\
           & = -\frac{\hbar m \omega}{2} \int_{-\infty}^\infty \psi_n^* (\hat{a}_+^2 - \hat{a}_+ \hat{a}_- - \hat{a}_- \hat{a}- + \hat{a}_-^2) \psi_n \,d x \\
           & = \frac{\hbar m \omega}{2} (2 n + 1) \int_{-\infty}^\infty |\psi_n|^2 \,d x                                                                    \\
           & = \hbar m \omega \left( n + \frac{1}{2} \right)                                                                                                \\
  \ev{T}   & = \ev{\frac{p^2}{2 m}}                                                                                                                         \\
           & = \frac{1}{2} \hbar \omega \left( n + \frac{1}{2} \right)
\end{align*}

\begin{align*}
  \sigma_x          & = \sqrt{\ev{x^2} - \ev{x}^2}                                   \\
                    & = \sqrt{\frac{\hbar}{m \omega} \left( n + \frac{1}{2} \right)} \\
  \sigma_p          & = \sqrt{\hbar m \omega \left( n + \frac{1}{2} \right)}         \\
  \sigma_x \sigma_p & = (2 n + 1) \frac{\hbar}{2}                                    \\
                    & \ge \frac{\hbar}{2}
\end{align*}

\subsection{}

\begin{enumerate}
  \item

        \begin{align*}
          \Psi(x, 0) & = A [3 \psi_0(x) + 4 \psi_1(x)]                                                             \\
          1          & = \int_{-\infty}^\infty |\Psi(x, 0)|^2 \,d x                                                \\
                     & = A^2 \int_{-\infty}^\infty [9 \psi_0(x)^2 + 24 \psi_0(x) \psi_1(x) + 16 \psi_1(x)^2] \,d x \\
                     & = 25 A^2                                                                                    \\
          A          & = \frac{1}{5}
        \end{align*}

  \item

        \begin{align*}
          \Psi(x, t)     & = \frac{1}{5} [3 \psi_0(x) e^{-i \omega t / 2} + 4 \psi_1(x) e^{-3 i \omega t / 2}]                                                                      \\
          |\Psi(x, t)|^2 & = \Psi(x, t)^* \Psi(x, t)                                                                                                                                \\
                         & = \frac{1}{25} [3 \psi_0(x) e^{i \omega t / 2} + 4 \psi_1(x) e^{3 i \omega t / 2}] [3 \psi_0(x) e^{-i \omega t / 2} + 4 \psi_1(x) e^{-3 i \omega t / 2}] \\
                         & = \frac{1}{25} [9 \psi_0(x)^2 + 12 \psi_0(x) \psi_1(x) e^{-i \omega t} + 12 \psi_0(x) \psi_1(x) e^{i \omega t} + 16 \psi_1(x)^2]                         \\
                         & = \frac{1}{25} [9 \psi_0(x)^2 + 16 \psi_1(x)^2 + 24 \psi_0(x) \psi_1(x) \cos \omega t]
        \end{align*}

  \item

        \begin{align*}
          \ev{x}                              & = \int_{-\infty}^\infty \Psi^* x \Psi \,d x                                                                                                                                            \\
                                              & = \frac{1}{25} \int_{-\infty}^\infty x (9 \psi_0^2 + 16 \psi_1^2 + 24 \psi_0 \psi_1 \cos \omega t) \,d x                                                                               \\
                                              & = \frac{24}{25} \int_{-\infty}^\infty x \psi_0 \psi_1 \cos (\omega t) \,d x                                                                                                            \\
                                              & = \frac{24}{25} \left( \frac{m \omega}{\pi \hbar} \right)^{1 / 2} \sqrt{\frac{2 m \omega}{\hbar}} \cos (\omega t) \int_{-\infty}^\infty x^2 e^{-\frac{m \omega}{\hbar} x^2} \,d x      \\
                                              & = \frac{24}{25} \left( \frac{m \omega}{\pi \hbar} \right)^{1 / 2} \sqrt{\frac{2 m \omega}{\hbar}} \cos (\omega t) \frac{1}{2} \sqrt{\pi} \left( \frac{\hbar}{m \omega} \right)^{3 / 2} \\
                                              & = \frac{24}{25} \sqrt{\frac{\hbar}{2 m \omega}} \cos (\omega t)                                                                                                                        \\
          \ev{p}                              & = m \frac{d \ev{x}}{d t}                                                                                                                                                               \\
                                              & = -\frac{24}{25} \sqrt{\frac{\hbar m \omega}{2}} \sin (\omega t)                                                                                                                       \\
          \frac{d \ev{p}}{d t}                & = -\frac{24}{25} \sqrt{\frac{\hbar m \omega}{2}} \omega \cos (\omega t)                                                                                                                \\
          V                                   & = \frac{1}{2} m \omega^2 x^2                                                                                                                                                           \\
          \frac{\partial V}{d \partial}       & = m \omega^2 x                                                                                                                                                                         \\
          \ev{-\frac{\partial V}{\partial x}} & = -m \omega^2 \ev{x}                                                                                                                                                                   \\
                                              & = -\frac{24}{25} \sqrt{\frac{\hbar m \omega}{2}} \omega \cos (\omega t)                                                                                                                \\
                                              & = \frac{d \ev{p}}{d t}
        \end{align*}

  \item

        \begin{align*}
          E_0    & = \frac{\hbar \omega}{2}   \\
          P(E_0) & = \frac{9}{25}             \\
          E_1    & = \frac{3 \hbar \omega}{2} \\
          P(E_1) & = \frac{16}{25}
        \end{align*}
\end{enumerate}

\subsection{}

\begin{align*}
  1 - \left( \frac{m \omega}{\pi \hbar} \right)^{1 / 2} \int_{-\sqrt{\hbar / m \omega}}^{\sqrt{\hbar / m \omega}} e^{-m \omega x^2 / \hbar} \,d x & = 1 - \left( \frac{m \omega}{\pi \hbar} \right)^{1 / 2} \sqrt{\frac{\pi \hbar}{m \omega}} \erf 1 \\
                                                                                                                                                  & = 0.157
\end{align*}

\subsection{}

\begin{align*}
  a_{j + 2} & = \frac{-2 (n - j)}{(j + 1) (j + 2)} a_j                          \\
  % H_5
  a_3       & = -\frac{4}{3} a_1                                                \\
  a_5       & = \frac{4}{15} a_1                                                \\
  H_5(\xi)  & = a_1 \left( \xi - \frac{4}{3} \xi^3 + \frac{4}{15} \xi^5 \right) \\
            & = \frac{1}{120} a_1 (120 \xi - 160 \xi^3 + 32 \xi^5)              \\
            & = 32 \xi^5 - 160 \xi^3 + 120 \xi                                  \\
  % H_6
  a_2       & = -6 a_0                                                          \\
  a_4       & = \frac{-8}{12} a_2                                               \\
            & = 4 a_0                                                           \\
  a_6       & = \frac{-4}{30} a_4                                               \\
            & = -\frac{8}{15} a_0                                               \\
  H_6(\xi)  & = a_0 \left( 1 - 6 \xi^2 + 4 \xi^4 - \frac{8}{15} \xi^6 \right)   \\
            & = \frac{1}{120} a_0 (120 - 720 \xi^2 + 480 \xi^4 - 64 \xi^6)      \\
            & = 64 \xi^6 - 480 \xi^4 + 720 \xi^2 - 120
\end{align*}

\setcounter{subsection}{16}
\subsection{}

\begin{align*}
  A e^{i k x} + B e^{-i k x} & = A [\cos (k x) + i \sin (k x)] + B [\cos (k x) - i \sin (k x)] \\
                             & = (A + B) \cos (k x) + i (A - B) \sin (k x)                     \\
  C                          & = A + B                                                         \\
  D                          & = i (A - B)                                                     \\
  -i D                       & = A - B                                                         \\
  A                          & = \frac{C - i D}{2}                                             \\
  B                          & = \frac{C + i D}{2}
\end{align*}

\subsection{}

\begin{align*}
  \Psi_k(x, t) & = A e^{i \left( k x - \frac{\hbar k^2}{2 m} t \right)}                                                                 \\
  J(x, t)      & = \frac{i \hbar}{2 m} \left( \Psi \frac{\partial \Psi^*}{\partial x} - \Psi^* \frac{\partial \Psi}{\partial x} \right) \\
               & = \frac{\hbar k |A|^2}{m}
\end{align*}

The probability travels in the same direction as the wave.

\setcounter{subsection}{19}
\subsection{}

\begin{enumerate}
  \item

        \begin{align*}
          \Psi(x, 0) & = A e^{-a |x|}                                   \\
          1          & = \int_{-\infty}^\infty \Psi^* \Psi \,d x        \\
                     & = |A|^2 \int_{-\infty}^\infty e^{-2 a |x|} \,d x \\
                     & = 2 |A|^2 \int_0^\infty e^{-2 a x} \,d x         \\
                     & = \frac{|A|^2}{a}                                \\
          A          & = \sqrt{a}
        \end{align*}

  \item

        \begin{align*}
          \phi(k) & = \sqrt{\frac{a}{2 \pi}} \int_{-\infty}^\infty e^{-a |x| - i k x} \,d x                     \\
                  & = \sqrt{\frac{a}{2 \pi}} \int_{-\infty}^\infty e^{-a |x|} [\cos (k x) - i \sin (k x)] \,d x \\
                  & = \sqrt{\frac{a}{2 \pi}} \int_{-\infty}^\infty e^{-a |x|} \cos (k x) \,d x                  \\
                  & = \sqrt{\frac{a}{2 \pi}} 2 \int_0^\infty e^{-a x} \cos (k x) \,d x                          \\
                  & = \sqrt{\frac{a}{2 \pi}} \frac{2 a}{a^2 + k^2}
        \end{align*}

  \item \[\Psi(x, t) = \frac{a^{3 / 2}}{\pi} \int_{-\infty}^\infty \frac{1}{a^2 + k^2} e^{i \left( k x - \frac{\hbar k^2}{2 m} t \right)} \,d k\]
\end{enumerate}

\subsection{}

\begin{enumerate}
  \item

        \begin{align*}
          \Psi(x, 0) & = A e^{-a x^2}                                  \\
          1          & = A^2 \int_{-\infty}^\infty  e^{-2 a x^2} \,d x \\
                     & = \sqrt{\frac{\pi}{2 a}} A^2                    \\
          A          & = \left( \frac{2 a}{\pi} \right)^{1 / 4}
        \end{align*}

  \item

        \begin{align*}
          \phi(k)    & = \frac{1}{\sqrt{2 \pi}} \left( \frac{2 a}{\pi} \right)^{1 / 4} \int_{-\infty}^\infty e^{-(a x^2 + i k x)} \,d x                                       \\
                     & = \frac{1}{(2 \pi a)^{1 / 4}} e^{-k^2 / 4 a}                                                                                                           \\
          \Psi(x, t) & = \frac{1}{\sqrt{2 \pi}} \frac{1}{(2 \pi a)^{1 / 4}} \int_{-\infty}^\infty e^{-\frac{k^2}{4 a} + i \left( k x - \frac{\hbar k^2}{2 m} t \right)} \,d k \\
          \Psi(x, t) & = \left( \frac{2 a}{\pi} \right)^{1 / 4} \frac{1}{\gamma} e^{-a x^2 / \gamma^2}
        \end{align*}

  \item

        \begin{align*}
          |\Psi(x, t)|^2 & = \Psi^* \Psi                                                                                                                \\
                         & = \left( \frac{2 a}{\pi} \right)^{1 / 2} \frac{1}{\gamma^*} e^{-a x^2 / (\gamma^*)^2} \frac{1}{\gamma} e^{-a x^2 / \gamma^2} \\
                         & = \left( \frac{2 a}{\pi} \right)^{1 / 2} \frac{1}{\sqrt{1 - 2 i \hbar a t / m}} e^{-a x^2 / (1 - 2 i \hbar a t / m)}         \\
                         & \qquad \frac{1}{\sqrt{1 + 2 i \hbar a t / m}} e^{-a x^2 / (1 + 2 i \hbar a t / m)}                                           \\
                         & = \left( \frac{2 a}{\pi} \right)^{1 / 2} \frac{1}{\sqrt{1 + (2 \hbar a t / m)^2}} e^{-2 a x^2 / [1 + (2 a \hbar t / m)^2]}   \\
                         & = \sqrt{\frac{2}{\pi}} w e^{-2 w^2 x^2}
        \end{align*}

        As $t$ increases $|\Psi|^2$ flattens out and broadens.

  \item

        \begin{align*}
          \ev{x}   & = \int_{-\infty}^\infty \Psi^* x \Psi \,d x                             \\
                   & = 0                                                                     \\
          \ev{p}   & = m \frac{d \ev{x}}{d t}                                                \\
                   & = 0                                                                     \\
          \ev{x^2} & = \int_{-\infty}^\infty \Psi^* x^2 \Psi \,d x                           \\
                   & = \sqrt{\frac{2}{\pi}} w \int_{-\infty}^\infty x^2 e^{-2 w^2 x^2} \,d x \\
                   & = \frac{1}{4 w^2}
        \end{align*}
\end{enumerate}

\end{document}