\documentclass{article}
\usepackage{amsmath} % For align*
\usepackage{bookmark} % For links
\usepackage{enumitem} % For customisable list labels
\usepackage{siunitx} % For units

\hypersetup{
  colorlinks=true,
  linkcolor=blue,
  urlcolor=blue
}

\newcommand{\ev}[1]{\langle #1 \rangle}

\setlist[enumerate, 1]{label={(\alph*)}}
\setlist[enumerate, 2]{label={(\roman*)}}

\title{Introduction to Quantum Mechanics by David J. Griffiths Problems}
\author{Chris Doble}
\date{March 2023}

\begin{document}

\maketitle

\tableofcontents

\part{Theory}

\section{The Wave Function}

\subsection{}

\begin{enumerate}
  \item

        \begin{align*}
          \langle j^2 \rangle & = \sum j^2 P(j)                                                                                                         \\
                              & = 14^2 \frac{1}{14} + 15^2 \frac{1}{14} + 16^2 \frac{3}{14} + 22^2 \frac{2}{14} + 24^2 \frac{2}{14} + 25^2 \frac{5}{14} \\
                              & = \frac{3217}{7}                                                                                                        \\
                              & \approx 459.571                                                                                                         \\
          \langle j \rangle^2 & = \left( \sum j P(j) \right)^2                                                                                          \\
                              & = 441
        \end{align*}

  \item

        \begin{align*}
          \Delta j_{14} & = -7                     \\
          \Delta j_{15} & = -6                     \\
          \Delta j_{16} & = -5                     \\
          \Delta j_{22} & = 1                      \\
          \Delta j_{24} & = 3                      \\
          \Delta j_{25} & = 4                      \\
          \sigma^2      & = \sum (\Delta j)^2 P(j) \\
                        & = \frac{130}{7}          \\
                        & \approx 18.571
        \end{align*}

  \item \[\sigma^2 = \sqrt{\langle j^2 \rangle - \langle j \rangle^2} = 18.571\]
\end{enumerate}

\subsection{}

\begin{enumerate}
  \item

        \begin{align*}
          \langle x^2 \rangle & = \int_0^h x^2 \rho(x) \,d x                                    \\
                              & = \int_0^h \frac{x^{3 / 2}}{2 \sqrt{h}} \,d x                   \\
                              & = \frac{1}{2 \sqrt{h}} \left[ \frac{2}{5} x^{5 / 2} \right]_0^h \\
                              & = \frac{h^2}{5}                                                 \\
          \langle x \rangle^2 & = \frac{h^2}{9}                                                 \\
          \sigma              & = \sqrt{\langle x^2 \rangle - \langle x \rangle^2}              \\
                              & = \sqrt{\frac{h^2}{5} - \frac{h^2}{9}}                          \\
                              & = h \sqrt{\frac{4}{45}}                                         \\
                              & = \frac{2}{3 \sqrt{5}} h
        \end{align*}

  \item

        \begin{align*}
          1 - \int_{\langle x \rangle - \sigma}^{\langle x \rangle + \sigma} \rho(x) \,d x & = 1 - \frac{1}{2 \sqrt{h}} [2 \sqrt{x}]_{\langle x \rangle - \sigma}^{\langle x \rangle + \sigma}                                     \\
                                                                                           & = 1 - \frac{1}{\sqrt{h}} \left( \sqrt{\frac{1}{3} h + \frac{2}{3 \sqrt{5}} h} - \sqrt{\frac{1}{3} h - \frac{2}{3 \sqrt{5}} h} \right) \\
                                                                                           & = 1 - \left( \sqrt{\frac{1}{3} + \frac{2}{3 \sqrt{5}}} - \sqrt{\frac{1}{3} - \frac{2}{3 \sqrt{5}}} \right)                            \\
                                                                                           & \approx 0.393
        \end{align*}
\end{enumerate}

\subsection{}

\begin{enumerate}
  \item

        \begin{align*}
          \rho(x) & = A e^{-\lambda (x - a)^2}                             \\
          1       & = \int_{-\infty}^\infty \rho(x) \,d x                  \\
                  & = A \int_{-\infty}^\infty e^{-\lambda (x - a)^2} \,d x \\
                  & = A \sqrt{\frac{\pi}{\lambda}}                         \\
          A       & = \sqrt{\frac{\lambda}{\pi}}
        \end{align*}

  \item

        \begin{align*}
          \langle x \rangle   & = \sqrt{\frac{\lambda}{\pi}} \int_{-\infty}^\infty x e^{-\lambda (x - a)^2} \,d x   \\
                              & = a                                                                                 \\
          \langle x^2 \rangle & = \sqrt{\frac{\lambda}{\pi}} \int_{-\infty}^\infty x^2 e^{-\lambda (x - a)^2} \,d x \\
                              & = a^2 + \frac{1}{2 \lambda}                                                         \\
          \sigma              & = \sqrt{\langle x^2 \rangle - \langle x \rangle^2}                                  \\
                              & = \sqrt{a^2 + \frac{1}{2 \lambda} - a^2}                                            \\
                              & = \frac{1}{\sqrt{2 \lambda}}
        \end{align*}
\end{enumerate}

\subsection{}

\begin{enumerate}
  \item

        \begin{align*}
          1 & = \int_{-\infty}^\infty |\Psi(x, 0)|^2 \,d x                                                                  \\
            & = \left( \frac{A}{a} \right)^2 \int_0^a x^2 \,d x + \left( \frac{A}{b - a} \right)^2 \int_a^b (b - x)^2 \,d x \\
            & = \frac{1}{3} A^2 a + \left( \frac{A}{b - a} \right)^2 \left[ -\frac{1}{3} (b - x)^3 \right]_a^b              \\
            & = \frac{1}{3} A^2 a + \frac{1}{3} A^2 (b - a)                                                                 \\
            & = \frac{1}{3} A^2 b                                                                                           \\
          A & = \sqrt{\frac{3}{b}}
        \end{align*}

        \setcounter{enumi}{2}
  \item $x = a$

  \item

        \begin{align*}
          \int_0^a |\Psi(x, 0)|^2 \,d x & = \frac{3}{a^2 b} \left[ \frac{1}{3} x^3 \right]_0^a \\
                                        & = \frac{a}{b}
        \end{align*}

  \item

        \begin{align*}
          \langle x \rangle & = \int_{-\infty}^\infty x |\Psi(x, 0)|^2 \,d x                                                                                                                             \\
                            & = \frac{3}{a^2 b} \left[ \frac{1}{4} x^4 \right]_0^a + \frac{3}{b (b - a)^2} \int_a^b x (b - x)^2 \,d x                                                                    \\
                            & = \frac{3 a^2}{4 b} + \frac{3}{b (b - a)^2} \int_a^b (b^2 x - 2 b x^2 + x^3) \,d x                                                                                         \\
                            & = \frac{3 a^2}{4 b} + \frac{3}{b (b - a)^2} \left[ \frac{1}{2} b^2 x^2 - \frac{2}{3} b x^3 + \frac{1}{4} x^4 \right]_a^b                                                   \\
                            & = \frac{3 a^2}{4 b} + \frac{3}{b (b - a)^2} \left( \frac{1}{2} b^4 - \frac{2}{3} b^4 + \frac{1}{4} b^4 - \frac{1}{2} a^2 b^2 + \frac{2}{3} a^3 b - \frac{1}{4} a^4 \right) \\
                            & = \frac{3 a^2}{4 b} + \frac{3}{b (b - a)^2} \frac{1}{12} (b - a)^3 (3 a + b)                                                                                               \\
                            & = \frac{3 a^2}{4 b} + \frac{1}{4 b} (3 a b + b^2 - 3 a^2 - a b)                                                                                                            \\
                            & = \frac{1}{2} a + \frac{1}{4} b
        \end{align*}
\end{enumerate}

\subsection{}

\begin{enumerate}
  \item

        \begin{align*}
          \Psi(x, t) & = A e^{-\lambda |x|} e^{-i \omega t}                                  \\
          \Psi(x, 0) & = A e^{-\lambda |x|}                                                  \\
          1          & = A^2 \int_{-\infty}^\infty e^{-2 \lambda |x|} \,d x                  \\
                     & = 2 A^2 \int_0^\infty e^{-2 \lambda x} \,d x                          \\
                     & = 2 A^2 \left[ -\frac{1}{2 \lambda} e^{-2 \lambda x} \right]_0^\infty \\
                     & = \frac{A^2}{\lambda}                                                 \\
          A          & = \sqrt{\lambda}
        \end{align*}

  \item

        \begin{align*}
          \langle x \rangle   & = \int_{-\infty}^\infty x \lambda e^{-2 \lambda |x|} \,d x   \\
                              & = \lambda \int_{-\infty}^\infty x e^{-2 \lambda |x|} \,d x   \\
                              & = 0                                                          \\
          \langle x^2 \rangle & = \int_{-\infty}^\infty x^2 \lambda e^{-2 \lambda |x|} \,d x \\
                              & = 2 \lambda \int_0^\infty x^2 e^{-2 \lambda x} \,d x         \\
                              & = \frac{1}{2 \lambda^2}
        \end{align*}

  \item

        \begin{align*}
          \sigma                                                    & = \sqrt{\langle x^2 \rangle - \langle x \rangle^2}                            \\
                                                                    & = \frac{1}{\sqrt{2} \lambda}                                                  \\
          1 - \int_{-\sigma}^\sigma \lambda e^{-2 \lambda |x|} \,dx & = 1 - 2 \lambda \int_0^\sigma e^{-2 \lambda x} \,d x                          \\
                                                                    & = 1 - 2 \lambda \left[ -\frac{1}{2 \lambda} e^{-2 \lambda x} \right]_0^\sigma \\
                                                                    & = e^{-2 \lambda \sigma}                                                       \\
                                                                    & = e^{-\sqrt{2}}                                                               \\
                                                                    & \approx 0.243
        \end{align*}
\end{enumerate}

\subsection{}

The chain rule requires that you apply it to both $x$ and $|\Psi|^2$ which gives the same result

\begin{align*}
  \frac{d \ev{x}}{d t} & = \frac{d}{d t} \int x |\Psi|^2 \,d x                                                 \\
                       & = \int \frac{d}{d t} (x |\Psi|^2) \,d x                                               \\
                       & = \int \left( 0 \cdot |\Psi|^2 + x \frac{\partial |\Psi|^2}{\partial t} \right) \,d x \\
                       & = \int x \frac{\partial |\Psi|^2}{\partial t} \,d x
\end{align*}

\setcounter{subsection}{7}
\subsection{}

\begin{align*}
  i \hbar \frac{\partial}{\partial t} \left( e^{-i V_0 t / \hbar} \Psi \right)                                                  & = -\frac{\hbar^2}{2 m} \frac{\partial^2}{\partial x^2} \left( e^{-i V_0 t / \hbar} \Psi \right) + (V + V_0) \left( e^{-i V_0 t / \hbar} \Psi \right) \\
  i \hbar \left( -\frac{i V_0}{\hbar} e^{-i V_0 t / \hbar} \Psi + e^{-i V_0 t / \hbar} \frac{\partial \Psi}{\partial t} \right) & = -\frac{\hbar^2}{2 m} e^{-i V_0 t / \hbar} \frac{\partial^2 \Psi}{\partial x^2} + V e^{-i V_0 t / \hbar} \Psi + V_0 e^{-i V_0 t / \hbar} \Psi       \\
  V_0 \Psi + i \hbar \frac{\partial \Psi}{\partial t}                                                                           & = -\frac{\hbar^2}{2 m} \frac{\partial^2 \Psi}{\partial x^2} + V \Psi + V_0 \Psi                                                                      \\
  i \hbar \frac{\partial \Psi}{\partial t}                                                                                      & = -\frac{\hbar^2}{2 m} \frac{\partial^2 \Psi}{\partial x^2} + V \Psi                                                                                 \\
\end{align*}

\begin{align*}
  \ev{Q(x, p)} & = \int \left( e^{-i V_0 t / \hbar} \Psi \right)^* \left[ Q(x, -i \hbar \partial / \partial x) \right] e^{-i V_0 t / \hbar} \Psi \,d x \\
               & = \int e^{i V_0 t / \hbar} \Psi^* \left[ Q(x, -i \hbar \partial / \partial x) \right] e^{-i V_0 t / \hbar} \Psi \,d x                 \\
               & = \int \Psi^* [Q(x, -i \hbar \partial / \partial x)] \Psi \,d x
\end{align*}

No effect on the expectation value.

\end{document}