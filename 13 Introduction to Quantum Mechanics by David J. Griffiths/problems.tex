\documentclass{article}
\usepackage{amsmath} % For align*
\usepackage{bookmark} % For links
\usepackage{enumitem} % For customisable list labels
\usepackage{siunitx} % For units

\hypersetup{
  colorlinks=true,
  linkcolor=blue,
  urlcolor=blue
}

\newcommand{\ev}[1]{\langle #1 \rangle}

\setlist[enumerate, 1]{label={(\alph*)}}
\setlist[enumerate, 2]{label={(\roman*)}}

\title{Introduction to Quantum Mechanics by David J. Griffiths Problems}
\author{Chris Doble}
\date{March 2023}

\begin{document}

\maketitle

\tableofcontents

\part{Theory}

\section{The Wave Function}

\subsection{}

\begin{enumerate}
  \item

        \begin{align*}
          \langle j^2 \rangle & = \sum j^2 P(j)                                                                                                         \\
                              & = 14^2 \frac{1}{14} + 15^2 \frac{1}{14} + 16^2 \frac{3}{14} + 22^2 \frac{2}{14} + 24^2 \frac{2}{14} + 25^2 \frac{5}{14} \\
                              & = \frac{3217}{7}                                                                                                        \\
                              & \approx 459.571                                                                                                         \\
          \langle j \rangle^2 & = \left( \sum j P(j) \right)^2                                                                                          \\
                              & = 441
        \end{align*}

  \item

        \begin{align*}
          \Delta j_{14} & = -7                     \\
          \Delta j_{15} & = -6                     \\
          \Delta j_{16} & = -5                     \\
          \Delta j_{22} & = 1                      \\
          \Delta j_{24} & = 3                      \\
          \Delta j_{25} & = 4                      \\
          \sigma^2      & = \sum (\Delta j)^2 P(j) \\
                        & = \frac{130}{7}          \\
                        & \approx 18.571
        \end{align*}

  \item \[\sigma^2 = \sqrt{\langle j^2 \rangle - \langle j \rangle^2} = 18.571\]
\end{enumerate}

\subsection{}

\begin{enumerate}
  \item

        \begin{align*}
          \langle x^2 \rangle & = \int_0^h x^2 \rho(x) \,d x                                    \\
                              & = \int_0^h \frac{x^{3 / 2}}{2 \sqrt{h}} \,d x                   \\
                              & = \frac{1}{2 \sqrt{h}} \left[ \frac{2}{5} x^{5 / 2} \right]_0^h \\
                              & = \frac{h^2}{5}                                                 \\
          \langle x \rangle^2 & = \frac{h^2}{9}                                                 \\
          \sigma              & = \sqrt{\langle x^2 \rangle - \langle x \rangle^2}              \\
                              & = \sqrt{\frac{h^2}{5} - \frac{h^2}{9}}                          \\
                              & = h \sqrt{\frac{4}{45}}                                         \\
                              & = \frac{2}{3 \sqrt{5}} h
        \end{align*}

  \item

        \begin{align*}
          1 - \int_{\langle x \rangle - \sigma}^{\langle x \rangle + \sigma} \rho(x) \,d x & = 1 - \frac{1}{2 \sqrt{h}} [2 \sqrt{x}]_{\langle x \rangle - \sigma}^{\langle x \rangle + \sigma}                                     \\
                                                                                           & = 1 - \frac{1}{\sqrt{h}} \left( \sqrt{\frac{1}{3} h + \frac{2}{3 \sqrt{5}} h} - \sqrt{\frac{1}{3} h - \frac{2}{3 \sqrt{5}} h} \right) \\
                                                                                           & = 1 - \left( \sqrt{\frac{1}{3} + \frac{2}{3 \sqrt{5}}} - \sqrt{\frac{1}{3} - \frac{2}{3 \sqrt{5}}} \right)                            \\
                                                                                           & \approx 0.393
        \end{align*}
\end{enumerate}

\subsection{}

\begin{enumerate}
  \item

        \begin{align*}
          \rho(x) & = A e^{-\lambda (x - a)^2}                             \\
          1       & = \int_{-\infty}^\infty \rho(x) \,d x                  \\
                  & = A \int_{-\infty}^\infty e^{-\lambda (x - a)^2} \,d x \\
                  & = A \sqrt{\frac{\pi}{\lambda}}                         \\
          A       & = \sqrt{\frac{\lambda}{\pi}}
        \end{align*}

  \item

        \begin{align*}
          \langle x \rangle   & = \sqrt{\frac{\lambda}{\pi}} \int_{-\infty}^\infty x e^{-\lambda (x - a)^2} \,d x   \\
                              & = a                                                                                 \\
          \langle x^2 \rangle & = \sqrt{\frac{\lambda}{\pi}} \int_{-\infty}^\infty x^2 e^{-\lambda (x - a)^2} \,d x \\
                              & = a^2 + \frac{1}{2 \lambda}                                                         \\
          \sigma              & = \sqrt{\langle x^2 \rangle - \langle x \rangle^2}                                  \\
                              & = \sqrt{a^2 + \frac{1}{2 \lambda} - a^2}                                            \\
                              & = \frac{1}{\sqrt{2 \lambda}}
        \end{align*}
\end{enumerate}

\subsection{}

\begin{enumerate}
  \item

        \begin{align*}
          1 & = \int_{-\infty}^\infty |\Psi(x, 0)|^2 \,d x                                                                  \\
            & = \left( \frac{A}{a} \right)^2 \int_0^a x^2 \,d x + \left( \frac{A}{b - a} \right)^2 \int_a^b (b - x)^2 \,d x \\
            & = \frac{1}{3} A^2 a + \left( \frac{A}{b - a} \right)^2 \left[ -\frac{1}{3} (b - x)^3 \right]_a^b              \\
            & = \frac{1}{3} A^2 a + \frac{1}{3} A^2 (b - a)                                                                 \\
            & = \frac{1}{3} A^2 b                                                                                           \\
          A & = \sqrt{\frac{3}{b}}
        \end{align*}

        \setcounter{enumi}{2}
  \item $x = a$

  \item

        \begin{align*}
          \int_0^a |\Psi(x, 0)|^2 \,d x & = \frac{3}{a^2 b} \left[ \frac{1}{3} x^3 \right]_0^a \\
                                        & = \frac{a}{b}
        \end{align*}

  \item

        \begin{align*}
          \langle x \rangle & = \int_{-\infty}^\infty x |\Psi(x, 0)|^2 \,d x                                                                                                                             \\
                            & = \frac{3}{a^2 b} \left[ \frac{1}{4} x^4 \right]_0^a + \frac{3}{b (b - a)^2} \int_a^b x (b - x)^2 \,d x                                                                    \\
                            & = \frac{3 a^2}{4 b} + \frac{3}{b (b - a)^2} \int_a^b (b^2 x - 2 b x^2 + x^3) \,d x                                                                                         \\
                            & = \frac{3 a^2}{4 b} + \frac{3}{b (b - a)^2} \left[ \frac{1}{2} b^2 x^2 - \frac{2}{3} b x^3 + \frac{1}{4} x^4 \right]_a^b                                                   \\
                            & = \frac{3 a^2}{4 b} + \frac{3}{b (b - a)^2} \left( \frac{1}{2} b^4 - \frac{2}{3} b^4 + \frac{1}{4} b^4 - \frac{1}{2} a^2 b^2 + \frac{2}{3} a^3 b - \frac{1}{4} a^4 \right) \\
                            & = \frac{3 a^2}{4 b} + \frac{3}{b (b - a)^2} \frac{1}{12} (b - a)^3 (3 a + b)                                                                                               \\
                            & = \frac{3 a^2}{4 b} + \frac{1}{4 b} (3 a b + b^2 - 3 a^2 - a b)                                                                                                            \\
                            & = \frac{1}{2} a + \frac{1}{4} b
        \end{align*}
\end{enumerate}

\subsection{}

\begin{enumerate}
  \item

        \begin{align*}
          \Psi(x, t) & = A e^{-\lambda |x|} e^{-i \omega t}                                  \\
          \Psi(x, 0) & = A e^{-\lambda |x|}                                                  \\
          1          & = A^2 \int_{-\infty}^\infty e^{-2 \lambda |x|} \,d x                  \\
                     & = 2 A^2 \int_0^\infty e^{-2 \lambda x} \,d x                          \\
                     & = 2 A^2 \left[ -\frac{1}{2 \lambda} e^{-2 \lambda x} \right]_0^\infty \\
                     & = \frac{A^2}{\lambda}                                                 \\
          A          & = \sqrt{\lambda}
        \end{align*}

  \item

        \begin{align*}
          \langle x \rangle   & = \int_{-\infty}^\infty x \lambda e^{-2 \lambda |x|} \,d x   \\
                              & = \lambda \int_{-\infty}^\infty x e^{-2 \lambda |x|} \,d x   \\
                              & = 0                                                          \\
          \langle x^2 \rangle & = \int_{-\infty}^\infty x^2 \lambda e^{-2 \lambda |x|} \,d x \\
                              & = 2 \lambda \int_0^\infty x^2 e^{-2 \lambda x} \,d x         \\
                              & = \frac{1}{2 \lambda^2}
        \end{align*}

  \item

        \begin{align*}
          \sigma                                                    & = \sqrt{\langle x^2 \rangle - \langle x \rangle^2}                            \\
                                                                    & = \frac{1}{\sqrt{2} \lambda}                                                  \\
          1 - \int_{-\sigma}^\sigma \lambda e^{-2 \lambda |x|} \,dx & = 1 - 2 \lambda \int_0^\sigma e^{-2 \lambda x} \,d x                          \\
                                                                    & = 1 - 2 \lambda \left[ -\frac{1}{2 \lambda} e^{-2 \lambda x} \right]_0^\sigma \\
                                                                    & = e^{-2 \lambda \sigma}                                                       \\
                                                                    & = e^{-\sqrt{2}}                                                               \\
                                                                    & \approx 0.243
        \end{align*}
\end{enumerate}

\subsection{}

The chain rule requires that you apply it to both $x$ and $|\Psi|^2$ which gives the same result

\begin{align*}
  \frac{d \ev{x}}{d t} & = \frac{d}{d t} \int x |\Psi|^2 \,d x                                                 \\
                       & = \int \frac{d}{d t} (x |\Psi|^2) \,d x                                               \\
                       & = \int \left( 0 \cdot |\Psi|^2 + x \frac{\partial |\Psi|^2}{\partial t} \right) \,d x \\
                       & = \int x \frac{\partial |\Psi|^2}{\partial t} \,d x
\end{align*}

\setcounter{subsection}{7}
\subsection{}

\begin{align*}
  i \hbar \frac{\partial}{\partial t} \left( e^{-i V_0 t / \hbar} \Psi \right)                                                  & = -\frac{\hbar^2}{2 m} \frac{\partial^2}{\partial x^2} \left( e^{-i V_0 t / \hbar} \Psi \right) + (V + V_0) \left( e^{-i V_0 t / \hbar} \Psi \right) \\
  i \hbar \left( -\frac{i V_0}{\hbar} e^{-i V_0 t / \hbar} \Psi + e^{-i V_0 t / \hbar} \frac{\partial \Psi}{\partial t} \right) & = -\frac{\hbar^2}{2 m} e^{-i V_0 t / \hbar} \frac{\partial^2 \Psi}{\partial x^2} + V e^{-i V_0 t / \hbar} \Psi + V_0 e^{-i V_0 t / \hbar} \Psi       \\
  V_0 \Psi + i \hbar \frac{\partial \Psi}{\partial t}                                                                           & = -\frac{\hbar^2}{2 m} \frac{\partial^2 \Psi}{\partial x^2} + V \Psi + V_0 \Psi                                                                      \\
  i \hbar \frac{\partial \Psi}{\partial t}                                                                                      & = -\frac{\hbar^2}{2 m} \frac{\partial^2 \Psi}{\partial x^2} + V \Psi                                                                                 \\
\end{align*}

\begin{align*}
  \ev{Q(x, p)} & = \int \left( e^{-i V_0 t / \hbar} \Psi \right)^* \left[ Q(x, -i \hbar \partial / \partial x) \right] e^{-i V_0 t / \hbar} \Psi \,d x \\
               & = \int e^{i V_0 t / \hbar} \Psi^* \left[ Q(x, -i \hbar \partial / \partial x) \right] e^{-i V_0 t / \hbar} \Psi \,d x                 \\
               & = \int \Psi^* [Q(x, -i \hbar \partial / \partial x)] \Psi \,d x
\end{align*}

No effect on the expectation value.

\subsection{}

\begin{enumerate}
  \item

        \begin{align*}
          \Psi(x, t) & = A e^{-a [(m x^2 / \hbar) + i t]}                         \\
          1          & = A^2 \int_{-\infty}^\infty e^{-2 a (m x^2 / \hbar)} \,d x \\
                     & = A^2 \int_{-\infty}^\infty e^{-2 a (m x^2 / \hbar)} \,d x \\
                     & = A^2 \sqrt{\frac{\pi \hbar}{2 a m}}                       \\
          A^2        & = \sqrt{\frac{2 a m}{\pi \hbar}}                           \\
          A          & = \left( \frac{2 a m}{\pi \hbar} \right)^{1 / 4}
        \end{align*}

  \item

        \begin{align*}
          \Psi                                 & = A e^{-a [(m x^2 / \hbar) + i t]}                                                                    \\
          \frac{\partial \Psi}{\partial t}     & = -i a \Psi                                                                                           \\
          \frac{\partial \Psi}{\partial x}     & = -\frac{2 a m x}{\hbar} \Psi                                                                         \\
          \frac{\partial^2 \Psi}{\partial x^2} & = -\frac{2 a m}{\hbar} \left( \Psi + x \frac{\partial \Psi}{\partial x} \right)                       \\
                                               & = -\frac{2 a m}{\hbar} \left( 1 - \frac{2 a m x^2}{\hbar} \right) \Psi                                \\
          V \Psi                               & = i \hbar \frac{\partial \Psi}{\partial t} + \frac{\hbar^2}{2 m} \frac{\partial^2 \Psi}{\partial x^2} \\
                                               & = a \hbar \Psi - a \hbar \left( 1 - \frac{2 a m x^2}{\hbar} \right) \Psi                              \\
          V                                    & = a \hbar - a \hbar + 2 a^2 m x^2                                                                     \\
                                               & = 2 a^2 m x^2
        \end{align*}

  \item

        \begin{align*}
          \ev{x}   & = A^2 \int_{-\infty}^\infty e^{-2 a (m x^2 / \hbar)} x \,d x                                                                                                                            \\
                   & = 0                                                                                                                                                                                     \\
          \ev{x^2} & = A^2 \int_{-\infty}^\infty e^{-2 a (m x^2 / \hbar)} x^2 \,d x                                                                                                                          \\
                   & = 2 A^2 \int_0^\infty e^{-2 a (m x^2 / \hbar)} x^2 \,d x                                                                                                                                \\
                   & = \frac{\hbar}{4 a m}                                                                                                                                                                   \\
          \ev{p}   & = \int_{-\infty}^\infty \Psi^* \left[ -i \hbar \frac{\partial}{\partial x} \right] \Psi \,d x                                                                                           \\
                   & = -i \hbar \int_{-\infty}^\infty A e^{-a [(m x^2 / \hbar) - i t]} \left( -\frac{2 a m x}{\hbar} A e^{-a [(m x^2 / \hbar) + i t]} \right) \,d x                                          \\
                   & = 2 i A^2 a m \int_{-\infty}^\infty x e^{-2 a m x^2 / \hbar} \,d x                                                                                                                      \\
                   & = 0                                                                                                                                                                                     \\
          \ev{p^2} & = \int_{-\infty}^\infty \Psi^* \left[ -\hbar^2 \frac{\partial^2}{\partial x^2} \right] \Psi \,d x                                                                                       \\
                   & = -\hbar^2 \int_{-\infty}^\infty A e^{-a [(m x^2 / \hbar) - i t]} \left[ -\frac{2 a m}{\hbar} \left( 1 - \frac{2 a m x^2}{\hbar} \right) A e^{-a [(m x^2 / \hbar) + i t]} \right] \,d x \\
                   & = 2 A^2 a m \hbar \int_{-\infty}^\infty e^{-2 a m x^2 / \hbar} \left( 1 - \frac{2 a m x^2}{\hbar} \right) \,d x                                                                         \\
                   & = a m \hbar
        \end{align*}

  \item

        \begin{align*}
          \sigma_x          & = \sqrt{\ev{x^2} - \ev{x}^2} \\
                            & = \sqrt{\frac{\hbar}{4 a m}} \\
          \sigma_p          & = \sqrt{a m \hbar}           \\
          \sigma_x \sigma_p & = \sqrt{\frac{1}{4} \hbar^2} \\
                            & = \frac{1}{2} \hbar          \\
                            & \ge \frac{1}{2} \hbar
        \end{align*}

        Yes, this is consistent with Heisenberg's uncertainty principle.
\end{enumerate}

\subsection{}

\begin{enumerate}
  \item

        \begin{align*}
          P(0) & = 0            \\
          P(1) & = \frac{2}{25} \\
               & = 0.08         \\
          P(2) & = \frac{3}{25} \\
               & = 0.12         \\
          P(3) & = \frac{1}{5}  \\
               & = 0.2          \\
          P(4) & = \frac{3}{25} \\
               & = 0.12         \\
          P(5) & = \frac{3}{25} \\
               & = 0.2          \\
          P(6) & = \frac{3}{25} \\
               & = 0.2          \\
          P(7) & = \frac{1}{25} \\
               & = 0.04         \\
          P(8) & = \frac{2}{25} \\
               & = 0.08         \\
          P(9) & = \frac{3}{25} \\
               & = 0.12
        \end{align*}

  \item The most probable digit is $3$, the median digit is $4$, and the average value is $\frac{118}{25} = 4.72$.

  \item $\sigma = 2.474$
\end{enumerate}

\setcounter{subsection}{13}
\subsection{}

\begin{enumerate}
  \item

        \begin{align*}
          P_{a b}(t)            & = \int_a^b |\Psi(x, t)|^2 \,d x                                                                                                                                                  \\
          \frac{d P_{a b}}{d t} & = \frac{d}{d t} \int_a^b |\Psi(x, t)|^2 \,d x                                                                                                                                    \\
                                & = \int_a^b \frac{d}{d t} \left( |\Psi(x, t)|^2 \right) \,d x                                                                                                                     \\
                                & = \int_a^b \frac{\partial}{\partial x} \left[ \frac{i \hbar}{2 m} \left( \Psi^* \frac{\partial \Psi}{\partial x} - \frac{\partial \Psi^*}{\partial x} \Psi \right) \right] \,d x \\
                                & = J(a, t) - J(b, t)
        \end{align*}

        The units are $\unit{s^{-1}}$.

  \item

        \begin{align*}
          \Psi(x, t)                         & = A e^{-a [(m x^2 / \hbar) + i t]}                                                                                                         \\
          \frac{\partial \Psi}{\partial x}   & = -\frac{2 a m x}{\hbar} \Psi                                                                                                              \\
          \Psi^*(x, t)                       & = A e^{-a [(m x^2 / \hbar) - i t]}                                                                                                         \\
          \frac{\partial \Psi^*}{\partial x} & = -\frac{2 a m x}{\hbar} \Psi^*                                                                                                            \\
          J(x, t)                            & = \frac{i \hbar}{2 m} \left( \Psi \frac{\partial \Psi^*}{\partial x} - \Psi^* \frac{\partial \Psi}{\partial x} \right)                     \\
                                             & = \frac{i \hbar}{2 m} \left[ \Psi \left( -\frac{2 a m x}{\hbar} \Psi^* \right) - \Psi^* \left( -\frac{2 a m x}{\hbar} \Psi \right) \right] \\
                                             & = 0
        \end{align*}
\end{enumerate}

\subsection{}

\begin{align*}
  \frac{d}{d t} \int_{-\infty}^\infty \Psi_1^* \Psi_2 \,d x & = \int_{-\infty}^\infty \left( \frac{\partial \Psi_1^*}{\partial t} \Psi_2 + \Psi_1^* \frac{\partial \Psi_2}{\partial t} \right) \,d x                                                       \\
                                                            & = \int_{-\infty}^\infty \left[ \left( -i \frac{\hbar}{2 m} \frac{\partial^2 \Psi_1^*}{\partial x^2} + i \frac{V}{\hbar} \Psi_1^* \right) \Psi_2 \right.                                      \\
                                                            & \qquad \left. + \Psi_1^* \left( i \frac{\hbar}{2 m} \frac{\partial^2 \Psi_2}{\partial x^2} - i \frac{V}{\hbar} \Psi_2 \right) \right] \,d x                                                  \\
                                                            & = i \frac{\hbar}{2 m} \int_{-\infty}^\infty \left( \Psi_1^* \frac{\partial^2 \Psi_2}{\partial x^2} - \frac{\partial^2 \Psi_1^*}{\partial x^2} \Psi_2 \right) \,d x                           \\
                                                            & = i \frac{\hbar}{2 m} \left[ \left. \Psi_1^* \frac{\partial \Psi_2}{\partial x} \right|_{-\infty}^\infty - \int_{-\infty}^\infty \frac{\partial}{\partial x} (\Psi_1^* \Psi_2) \,d x \right. \\
                                                            & \qquad \left. \left. \frac{\partial \Psi_1^*}{\partial x} \Psi_2 \right|_{-\infty}^\infty - \int_{-\infty}^\infty \frac{\partial}{\partial x} (\Psi_1^* \Psi_2) \,d x \right]                \\
                                                            & = 0
\end{align*}

\subsection{}

\begin{enumerate}
  \item

        \begin{align*}
          1 & = \int_{-a}^a A^2 (a^2 - x^2)^2 \,d x \\
            & = A^2 \int_0^a (a^2 - x^2)^2 \,d x    \\
            & = \frac{16}{15} A^2 a^5               \\
          A & = \sqrt{\frac{15}{16 a^5}}
        \end{align*}

  \item

        \begin{align*}
          \ev{x} & = \int_{-a}^a x A (a^2 - x^2) \,d x \\
                 & = 0
        \end{align*}

  \item

        \begin{align*}
          \ev{p} & = \int_{-a}^a \Psi^* \left( -i \hbar \frac{\partial}{\partial x} \right) \Psi \,d x \\
                 & = 2 i A^2 \hbar \int_{-a}^a x (a^2 - x^2) \,d x                                     \\
                 & = 0
        \end{align*}

  \item

        \begin{align*}
          \ev{x^2} & = \int_{-a}^a \Psi^* x^2 \Psi \,d x       \\
                   & = A^2 \int_{-a}^a x^2 (a^2 - x^2)^2 \,d x \\
                   & = A^2 \frac{16}{105} a^7                  \\
                   & = \frac{a^2}{7}
        \end{align*}

  \item

        \begin{align*}
          \ev{p^2} & = \int_{-a}^a \Psi^* \left( -\hbar^2 \frac{\partial^2}{\partial x^2} \right) \Psi \,d x \\
                   & = -\hbar^2 \int_{-a}^a A (a^2 - x^2) (-2 A) \,d x                                       \\
                   & = 4 A^2 \hbar^2 \int_0^a (a^2 - x^2) \,d x                                              \\
                   & = 4 A^2 \hbar^2 \left[ a^2 x - \frac{1}{3} x^3 \right]_0^a                              \\
                   & = 4 A^2 \hbar^2 \left( a^3 - \frac{1}{3} a^3 \right)                                    \\
                   & = \frac{8}{3} A^2 a^3 \hbar^2                                                           \\
                   & = \frac{8}{3} \frac{15}{16 a^5} a^3 \hbar^2                                             \\
                   & = \frac{5}{2} \frac{\hbar^2}{a^2}
        \end{align*}

  \item

        \begin{align*}
          \sigma_x & = \sqrt{\ev{x^2} - \ev{x}^2} \\
                   & = \sqrt{\frac{a^2}{7}}       \\
                   & = \frac{a}{\sqrt{7}}
        \end{align*}

  \item

        \begin{align*}
          \sigma_p & = \sqrt{\ev{p^2} - \ev{p}^2}         \\
                   & = \sqrt{\frac{5}{2}} \frac{\hbar}{a}
        \end{align*}

  \item

        \begin{align*}
          \sigma_x \sigma_p & = \sqrt{\frac{5}{14}} \hbar \\
                            & \ge \frac{1}{2} \hbar
        \end{align*}
\end{enumerate}

\setcounter{subsection}{17}
\subsection{}

\begin{enumerate}
  \item

        \begin{align*}
          % Electron
          \frac{h}{\sqrt{3 m k_B T}} & > d                       \\
          \frac{\sqrt{3 m k_B T}}{h} & < \frac{1}{d}             \\
          T_\text{electron}          & < \frac{h^2}{3 d^2 m k_B} \\
                                     & < \qty{1.3e5}{K}          \\
          % Nuclei
          T_\text{nuclei}            & < \qty{2.5}{K}
        \end{align*}

  \item

        \begin{align*}
          P V                        & = N k_B T                                                        \\
          \frac{V}{N}                & = \frac{k_B T}{P}                                                \\
          d                          & = \left( \frac{k_B T}{P} \right)^{1 / 3}                         \\
          \frac{h}{\sqrt{3 m k_B t}} & > \left( \frac{k_B T}{P} \right)^{1 / 3}                         \\
          T                          & < \frac{1}{k_B} \left( \frac{h^2}{3 m} \right)^{3 / 5} P^{2 / 5}
        \end{align*}
\end{enumerate}

\section{Time-Independent Schrödinger Equation}

\subsection{}

\begin{enumerate}
  \item

        \begin{align*}
          \int_{-\infty}^\infty |\Psi|^2 \,d x & = \int_{-\infty}^\infty \Psi^* \Psi \,d x                                                                    \\
                                               & = \int_{-\infty}^\infty \psi^* e^{i (E_0 - i \Gamma) t / \hbar} \psi e^{-i (E_0 + i \Gamma) t / \hbar} \,d x \\
                                               & = e^{2 \Gamma t / \hbar} \int_{-\infty}^\infty |\psi|^2 \,d x
        \end{align*}

        In order for this to equal $1$ for all $t$, $\Gamma$ must be $0$.

  \item If $\psi(x)$ is a complex solution to the time-independent Schrödinger equation then so is $\psi^*(x)$ and $\psi(x) + \psi^*(x)$ which is real.
\end{enumerate}

\subsection{}

If $\psi$ and its second derivative always have the same sign, $\psi$ will increase or decrease without bound forever. This means there is no non-zero choice of constant $A$ such that \[\int_{-\infty}^\infty |A \Psi|^2 \,d x = 1\] and thus the equation can't be normalised.

The classical analog of this is statements is that the potential energy of a system can't exceed its total energy.

\end{document}