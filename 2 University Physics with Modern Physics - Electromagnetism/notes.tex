\documentclass{article}
\usepackage{amsmath}
\usepackage{siunitx}

\title{University Physics with Modern Physics Electromagnetism Notes}
\author{Chris Doble}
\date{December 2022}

% The Coulomb constant
\newcommand{\ke}{\frac{1}{4 \pi \epsilon_0}}

\begin{document}

\maketitle

\tableofcontents

\setcounter{section}{20}
\section{Electric Charge and Electric Field}

\subsection{Electric Charge}

\begin{itemize}
  \item Electrons have a much smaller mass than neutrons and protons

  \item Neutrons and protons have a very similar mass

  \item Electrons and protons have the same magnitude of charge

  \item The number of protons in an atom determins its \textbf{atomic number}

  \item If an electron is added to a neutral atom it becomes a \textbf{negative ion}, if one is removed it becomes a \textbf{positive ion} — this is called \textbf{ionisation}

  \item The \textbf{principle of conservation of charge} states that the algebraic sum of all the electric charges in any closed system is constant

  \item The electron or proton's magnitude of charge is a natural unit of charge — every observable amount of electric charge is an integer multiple of this
\end{itemize}

\subsection{Conductors, Insulators, and Incuded Charges}

\begin{itemize}
  \item \textbf{Conductors} pemit easy movement of charge, \textbf{insulators} do not

  \item Holding a charged object near an uncharged object causes free electrons in the latter to move away/towards the former, resulting in a net charge on either side — this is called \textbf{induced charge}
\end{itemize}

\subsection{Coulomb's Law}

\begin{itemize}
  \item The SI unit of charge is called one \textbf{coulomb} (1 C) and is defined such that $1.602176634\times10^{-19}$ C is equal to the charge of an electron or proton

  \item \textbf{Coulomb's law} describes the electric force between two point charges \[F = \frac{1}{4\pi\epsilon_0}\frac{|q_1q_2|}{r^2}\] where the \textbf{electric constant} $\epsilon_0 = 8.854 \times 10^{-12}\,\textrm{C}^2/\textrm{N}\cdot \textrm{m}^2$, $q_1$ and $q_2$ are the magnitudes of the charges, and $r$ is the distance between them

  \item The electric force is always directed along the line between the two charges, attracting opposite charges and repelling like charges

  \item $\frac{1}{4\pi\epsilon_0}$ can be approximated as $9.0 \times 10^9\,\textrm{N}\cdot\textrm{m}^2/\textrm{C}^2$

  \item The principle of superposition of forces also applies to electric charges
\end{itemize}

\subsection{Electric Field and Electric Forces}

\begin{itemize}
  \item The electric force on a charged object is exerted by the electric field created by other charged objects

  \item We can determine if there is an electric field at a point by placing a test charge $q_0$ there and seeing if it experiences an electric force — the electric field at that point (the electric force per unit charge) is then given by \[\mathbf{E} = \frac{\mathbf{F}}{q_0}\]

  \item Rearranging, the force experienced by a charge $q_0$ at a point is given by \[\mathbf{F} = q_0\mathbf{E}\]

  \item When considering an electric field produced by a point charge, the location of the point charge is called the \textbf{source point} and the location at which we're trying to determine the field is called the \textbf{field point}

  \item The electric field produced by a point charge is given by \[\mathbf{E} = \frac{1}{4\pi\epsilon_0} \frac{q}{r^2}\hat{\mathbf{r}}\] where $q$ is the charge of the point charge, $r$ is the distance between the source and field points, and $\hat{\mathbf{r}}$ is the unit vector from the source to the field point

  \item Unlike Coulomb's law this equation doesn't use the absolute value of $q$ meaning that the electric fields of positive charges point away from the charge, while those of negative charges point towards them

  \item In electrostatics, the electric field inside the material of a conductor (but not holes within the material) is $\mathbf{0}$
\end{itemize}

\subsection{Electric-Field Calculations}

\begin{itemize}
  \item The \textbf{principle of superposition of electric fields} states that the total electric field at a point $P$ is the vector sum of the fields at $P$ due to each point charge in the charge distribution \[\mathbf{E} = \mathbf{E}_1 + \mathbf{E}_2 + \cdots\]

  \item For a line charge distribution the \textbf{linear charge density} is represented by $\lambda$ (the charge per unit length, measured in $\textrm{C}/\textrm{m}$)

  \item For a surface charge distribution the \textbf{surface charge density} is represented by $\sigma$ (the charge per unit area, measured in $\textrm{C}/\textrm{m}^2$)

  \item For a volume charge distribution the \textbf{volume charge density} is represented by $\rho$ (the charge per unit volume, measured in $\textrm{C}/\textrm{m}^3$)

  \item The electric field of an infinitely long line charge along the $y$-axis is \[E = \frac{\lambda}{2\pi\epsilon_0 r}\]
\end{itemize}

\subsection{Electric Field Lines}

\begin{itemize}
  \item An \textbf{electric field line} is a line drawn through space such that its tangent at any point is in the direction of the electric field vector at  that point

  \item Fewer lines are drawn in areas where the electric field is weak and more lines are drawn in areas where it's strong
\end{itemize}

\subsection{Electric Dipoles}

\begin{itemize}
  \item An \textbf{electric dipole} is a pair of point charges of equal magnitude $q$ and opposite sign separated by a distance $d$

  \item The net force on an electric dipole in a uniform electric field is $\mathbf{0}$

  \item The \textbf{electric dipole moment} $\mathbf{p}$ of an electric dipole is a vector directed from the negative charge to the positive charge with magnitude $qd$

  \item The net torque on an electric dipole in a uniform electric field is $\mathbf{p} \times \mathbf{E}$ or $qEd\sin\phi$ where $\phi$ is the angle between the electric dipole and the electric field

  \item The potential energy of an electric dipole in a uniform electric field is \[U = -\mathbf{p} \cdot \mathbf{E}\]
\end{itemize}

\section{Gauss's Law}

\subsection{Calculating Electric Flux}

\begin{itemize}
  \item The electric flux of a uniform electric field through a flat surface $A$ is \[\Phi_E = \mathbf{E} \cdot \mathbf{A}\] where $\mathbf{A}$ is normal to $A$ and has a magnitude equal to its area

  \item The electric flux of a nonuniform electric field through a curved surface $A$ is \[\Phi_E = \int \mathbf{E} \cdot \mathbf{dA}\]
\end{itemize}

\subsection{Gauss's Law}

\begin{itemize}
  \item Gauss's law states that the total electric flux through a closed surface is equal to the total electric charge enclosed by the surface divided by $\epsilon_0$ \[\Phi_E = \oint \mathbf{E} \cdot \mathbf{dA} = \frac{Q_\textrm{enc}}{\epsilon_0}\]
\end{itemize}

\subsection{Applications of Gauss's Law}

\begin{itemize}
  \item Gauss's law can be used in two ways:

        \begin{itemize}
          \item If we know the charge distribution and it has enough symmetry to let us evaluate the integral in Gauss's law, we can find the field

          \item If we know the field, we can use Gauss's law to find the charge distribution
        \end{itemize}

  \item Under electrostatics, excess charge always lies of the surface of a conductor

  \item The electric field of an infinite line charge is \[\mathbf{E} = \ke \frac{2 \lambda}{r} \hat{\mathbf{r}}\]
\end{itemize}

\subsection{Charges on Conductors}

\begin{itemize}
  \item If there is excess charge at rest on a conductor, all of that charge must lie on the surface of the conductor and the electric field inside the conductor must be zero. If there is a cavity inside the conductor, the net charge on the cavity walls equals the amount of charge enclosed by the cavity

  \item Charges outside a conductor have no effect on the interior of the conductor, even if it has a cavity inside — this is why Faraday cages work

  \item At the surface of a conductor, the component of the electric field that is perpendicular to the surface is \[E_\perp = \frac{\sigma}{\epsilon_0}\]
\end{itemize}

\section{Electric Potential}

\subsection{Electric Potential Energy}

\begin{itemize}
  \item The electric potential energy of two point charges is \[U = \ke \frac{q_1 q_2}{r}\]

  \item The electric potential energy of a point charge $q_0$ and a collection of charges $q_1$, $q_2$, etc. is \[U = \frac{q_0}{4 \pi \epsilon_0} \left( \frac{q_1}{r_1} + \frac{q_2}{r_2} + \cdots \right) = \frac{q_0}{4 \pi \epsilon_0} \sum_i \frac{q_i}{r_i}\]

  \item For every electric field due to a static charge distribution, the force exterted by that field is conservative

  \item The total electric potential energy of a collection of charges $q_1$, $q_2$, etc. is \[U = \ke \sum_{i < j} \frac{q_i q_j}{r_{ij}}\] where $r_{ij}$ is the distance between $q_i$ and $q_j$
\end{itemize}

\subsection{Electric Potential}

\begin{itemize}
  \item \textbf{Potential} is potential energy per unit charge

  \item The unit of potential is the \textbf{volt}, equal to 1 joule per coulomb

  \item The potential difference between two points $V_{ab} = V_a - V_b$ is called the potential of $a$ with respect to $b$ and equals the amount of work done by the electric force when a unit ($\qty{1}{C}$) of charge moves from $a$ to $b$

  \item The electric potential due to a point charge is \[V = \ke \frac{q}{r}\]

  \item The electric potential due to a collection of point charges is \[V = \ke \sum_i \frac{q_i}{r_i}\]

  \item The electric potential due to a continuous charge distribution is \[V = \ke \int \frac{dq}{r}\]

  \item The electric potential difference between two points is given by \[V_a - V_b = \int_a^b \mathbf{E} \cdot d\mathbf{l} = \int_a^b E \cos \phi \,dl\]

  \item Positive charges tend to "fall" from high- to low-potential regions while negative charges do the opposite

  \item When a particle with charge $e = \qty{1.602e-19}{C}$ moves between two points with a potential difference of $\qty{1}{V} = \qty{1}{J/C}$ the change in energy is $U_a - U_b = qV_{ab} = (\qty{1.602e-19}{C})(\qty{1}{J/C}) = \qty{1.602e-19}{J}$ which is called 1 \textbf{electron volt}
\end{itemize}

\setcounter{subsection}{3}
\subsection{Equipotential Surfaces}

\begin{itemize}
  \item An \textbf{equipotential surface} is a three-dimensional surface on which the electric potential is the same at every point

  \item Because electric potential energy doesn't change as a test charge moves over an equipotential surface, the electric field can do no work and thus \textbf{field lines and equipotential surfaces are always perpendicular}

  \item When all charges are at rest, the surface of a conductor is an equipotential surface

  \item When all charges are at rest, the entire solid volume of a conductor is at the same potential
\end{itemize}

\subsection{Potential Gradient}

\begin{itemize}
  \item The relationship between $\mathbf{E}$ and $V$ is given by \[\mathbf{E} = -\nabla V = -\left( \frac{\partial V}{\partial x} \hat{\mathbf{i}} + \frac{\partial V}{\partial y} \hat{\mathbf{j}} + \frac{\partial V}{\partial z} \hat{\mathbf{k}} \right)\]

  \item If $E$ has a radial component $E_r$ with respect to an axis or a point and $r$ is the distance from that axis or point, then \[E_r = -\frac{\partial V}{\partial r}\]
\end{itemize}

\end{document}