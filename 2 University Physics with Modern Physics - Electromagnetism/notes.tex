\documentclass{article}
\usepackage{amsmath}

\title{University Physics with Modern Physics Electromagnetism Notes}
\author{Chris Doble}
\date{December 2022}

\begin{document}

\maketitle

\tableofcontents

\setcounter{section}{20}
\section{Electric Charge and Electric Field}

\subsection{Electric Charge}

\begin{itemize}
  \item Electrons have a much smaller mass than neutrons and protons

  \item Neutrons and protons have a very similar mass

  \item Electrons and protons have the same magnitude of charge

  \item The number of protons in an atom determins its \textbf{atomic number}

  \item If an electron is added to a neutral atom it becomes a \textbf{negative ion}, if one is removed it becomes a \textbf{positive ion} — this is called \textbf{ionisation}

  \item The \textbf{principle of conservation of charge} states that the algebraic sum of all the electric charges in any closed system is constant

  \item The electron or proton's magnitude of charge is a natural unit of charge — every observable amount of electric charge is an integer multiple of this
\end{itemize}

\end{document}