\documentclass{article}
\usepackage{amsmath}
\usepackage{enumerate}

\title{University Physics with Modern Physics Electromagnetism Problems}
\author{Chris Doble}
\date{December 2022}

\begin{document}

\maketitle

\tableofcontents

\setcounter{section}{20}
\section{Electric Charge and Electric Field}

\setcounter{subsection}{2}
\subsection{Coulomb's Law}

\subsubsection{Example 21.1}

The magnitude of electric repulsion between two $\alpha$ particles is given by \[F_e = \frac{1}{4\pi\epsilon_0}\frac{q^2}{r^2}\] and the magnitude of gravitational attraction is given by \[F_g = \frac{Gm^2}{r^2}\]. The ratio of the two values is

\begin{align*}
  \frac{F_e}{F_g} & = \frac{1}{4\pi\epsilon_0}\frac{q^2}{r^2}\frac{r^2}{Gm^2} \\
                  & = \frac{1}{4\pi\epsilon_0}\frac{q^2}{Gm^2}                \\
                  & = 3.1\times10^{35}
\end{align*}

showing that the electric repulsion is significantly stronger than the gravitational attraction.

\subsubsection{Example 21.2}

\begin{enumerate}[a)]
  \item The magnitude of the force that $q_1$ exerts on $q_2$ is
        \begin{align*}
          F & = \frac{1}{4\pi\epsilon_0}\frac{q_1q_2}{r^2}                                 \\
            & = (9.0 \times 10^9)\frac{|(25 \times 10^{-9})(-75 \times 10^{-9})|}{0.030^2} \\
            & = 1.9 \times 10^{-2}\,\textrm{N}.
        \end{align*}
        Since $q_1$ and $q_2$ have opposite charge, the force is attractive (from $q_2$ to $q_1$).

  \item The magnitude of the force that $q_2$ exerts on $q_1$ is the same as in part a, but the direction is reversed (from $q_1$ to $q_2$).
\end{enumerate}

\subsubsection{Example 21.3}

By the principle of superposition of forces, the net force exerted on $q_3$ is equal to the vector sum of the forces exerted on it by $q_1$ and $q_2$ separately.

Both $q_1$ and $q_3$ have positive charge so they repel each other. $q_1$ is to the right of $q_3$ so $q_3$ experiences a force to the left of magnitude

\begin{align*}
  F_\textrm{1 on 3} & = \frac{1}{4\pi\epsilon_0}\frac{|q_1q_3|}{r^2}                                \\
                    & = (9.0 \times 10^9)\frac{|(1.0 \times 10^{-9})(5.0 \times 10^{-9})|}{0.020^2} \\
                    & = 1.1 \times 10^{-4}\,\textrm{N}.
\end{align*}

However $q_2$ has a negative charge so it attracts $q_3$. It is also to the right of $q_3$ so $q_3$ experiences a force to the right of magnitude

\begin{align*}
  F_\textrm{2 on 3} & = \frac{1}{4\pi\epsilon_0}\frac{|q_2q_3|}{r^2}                                \\
                    & = (9.0 \times 10^9)\frac{|(-3.0 \times 10^{-9})(5.0 \times 10^{-9})}{0.040^2} \\
                    & = 8.4 \times 10^{-5}\,\textrm{N}.
\end{align*}

The net force experienced by $q_3$ is therefore

\begin{align*}
  F & = -F_\textrm{1 on 3} + F_\textrm{2 on 3}   \\
    & = -1.1 \times 10^{-4} + 8.4 \times 10^{-5} \\
    & = -2.6 \times 10^{-5}\,\textrm{N}.
\end{align*}

\subsubsection{Example 21.4}

Since $q_1$ and $q_2$ are of equal charge and are symmetric about the x axis on which $Q$ lies, the vertical components of their forces cancel leaving only the horizontal.

The horizontal component of $q_1$'s force on $Q$ is given by

\begin{align*}
  F_\textrm{1 on Q, x} & = \frac{1}{4\pi\epsilon_0} \frac{q_1Q}{r_\textrm{1,Q}^2} \cos\alpha                                             \\
                       & = (9.0 \times 10^9) \frac{(2.0 \times 10^{-6})(4.0 \times 10^{-6})}{\sqrt{0.30^2 + 0.40^2}^2} \frac{0.40}{0.50} \\
                       & = 0.23\,\textrm{N}.
\end{align*}

Again, since $q_1$ and $q_2$ are of equal charge and symmetric about the x axis, $F_\textrm{1 on Q, x} = F_\textrm{2 on Q, x}$ and the total force experienced by $Q$ is in the positive x direction of magnitude \[F = 2F_\textrm{1 on Q, x} = 0.46\,\textrm{N}.\]

\subsection{Example 21.5}

The magnitude of the electric field vector is given by

\begin{align*}
  E & = \frac{1}{4\pi\epsilon_0} \frac{|q|}{r^2}             \\
    & = (9.0 \times 10^9) \frac{|4.0 \times 10^{-9}|}{2.0^2} \\
    & = 9.0\,\textrm{N}/\textrm{C}.
\end{align*}

\subsection{Example 21.6}

The magnitude of the electric field vector is given by

\begin{align*}
  E & = \frac{1}{4\pi\epsilon_0} \frac{|q|}{r^2}                      \\
    & = (9.0 \times 10^9) \frac{|-8.0 \times 10^{-9}|}{1.2^2 + 1.6^2} \\
    & = 18\,\textrm{N}/\textrm{C}
\end{align*}

and it is directed towards the origin. If $\theta$ is the angle between the positive x axis and $\hat{\mathbf{r}}$ then the component form of $\mathbf{E}$ is

\begin{align*}
  E & = -E\left(\cos\theta\hat{\mathbf{i}} + \sin\theta\hat{\mathbf{j}}\right)                      \\
    & = -E\left(\frac{x}{r}\hat{\mathbf{i}} + \frac{-y}{r}\hat{\mathbf{j}}\right)                   \\
    & = \frac{-18}{\sqrt{1.2^2 + 1.6^2}}\left(1.2\hat{\mathbf{i}} + 1.6\hat{\mathbf{j}}\right)      \\
    & = (-11\,\textrm{N}/\textrm{C})\hat{\mathbf{i}} - (14\,\textrm{N}/\textrm{C})\hat{\mathbf{j}}.
\end{align*}

\subsection{Example 21.7}

\begin{enumerate}[a)]
  \item Electrons have a negative charge and the electric field is directed upwards, so the electron will move downwards. The magnitude of its acceleration is

        \begin{align*}
          a & = \frac{F}{m}                                                           \\
            & = \frac{eE}{m}                                                          \\
            & = \frac{(1.60 \times 10^{-19})(1.00 \times 10^4)}{9.11 \times 10^{-31}} \\
            & = 1.76 \times 10^{15}\,\textrm{m}/\textrm{s}^2.
        \end{align*}

  \item Its acceleration is constant between the plates, so its final speed is

        \begin{align*}
          v^2 & = v_0^2 + 2a(x - x_0)                      \\
              & = 2ax                                      \\
          v   & = \sqrt{2ax}                               \\
              & = \sqrt{2(1.76 \times 10^{15})(0.01)}      \\
              & = 5.9 \times 10^6\,\textrm{m}/\textrm{s}^2
        \end{align*}

        and thus its final kinetic energy is

        \begin{align*}
          K & = \frac{1}{2}mv^2                                      \\
            & = \frac{1}{2}(9.11 \times 10^{-31})(5.9 \times 10^6)^2 \\
            & = 1.6 \times 10^{-17}\,\textrm{J}.
        \end{align*}

  \item We can find the time it takes for the electron to travel this distance by rearranging the kinematic equation \[v = v_0 + at\] to

        \begin{align*}
          t & = \frac{v - v_0}{a}                           \\
            & = \frac{5.9 \times 10^6}{1.76 \times 10^{15}} \\
            & = 3.4 \times 10^{-9}\,\textrm{s}.
        \end{align*}
\end{enumerate}

\end{document}