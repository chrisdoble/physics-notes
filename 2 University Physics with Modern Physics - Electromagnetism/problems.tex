\documentclass{article}
\usepackage{amsmath} % For align*
\usepackage{enumerate} % For customisable list labels
\usepackage{siunitx} % For units

\title{University Physics with Modern Physics Electromagnetism Problems}
\author{Chris Doble}
\date{December 2022}

\begin{document}

\maketitle

\tableofcontents

\setcounter{section}{20}
\section{Electric Charge and Electric Field}

\setcounter{subsection}{2}
\subsection{Coulomb's Law}

\subsubsection{Example 21.1}

The magnitude of electric repulsion between two $\alpha$ particles is given by \[F_e = \frac{1}{4\pi\epsilon_0}\frac{q^2}{r^2}\] and the magnitude of gravitational attraction is given by \[F_g = \frac{Gm^2}{r^2}\]. The ratio of the two values is

\begin{align*}
  \frac{F_e}{F_g} & = \frac{1}{4\pi\epsilon_0}\frac{q^2}{r^2}\frac{r^2}{Gm^2} \\
                  & = \frac{1}{4\pi\epsilon_0}\frac{q^2}{Gm^2}                \\
                  & = 3.1\times10^{35}
\end{align*}

showing that the electric repulsion is significantly stronger than the gravitational attraction.

\subsubsection{Example 21.2}

\begin{enumerate}[a)]
  \item The magnitude of the force that $q_1$ exerts on $q_2$ is
        \begin{align*}
          F & = \frac{1}{4\pi\epsilon_0}\frac{q_1q_2}{r^2}                                 \\
            & = (9.0 \times 10^9)\frac{|(25 \times 10^{-9})(-75 \times 10^{-9})|}{0.030^2} \\
            & = 1.9 \times 10^{-2}\,\textrm{N}.
        \end{align*}
        Since $q_1$ and $q_2$ have opposite charge, the force is attractive (from $q_2$ to $q_1$).

  \item The magnitude of the force that $q_2$ exerts on $q_1$ is the same as in part a, but the direction is reversed (from $q_1$ to $q_2$).
\end{enumerate}

\subsubsection{Example 21.3}

By the principle of superposition of forces, the net force exerted on $q_3$ is equal to the vector sum of the forces exerted on it by $q_1$ and $q_2$ separately.

Both $q_1$ and $q_3$ have positive charge so they repel each other. $q_1$ is to the right of $q_3$ so $q_3$ experiences a force to the left of magnitude

\begin{align*}
  F_\textrm{1 on 3} & = \frac{1}{4\pi\epsilon_0}\frac{|q_1q_3|}{r^2}                                \\
                    & = (9.0 \times 10^9)\frac{|(1.0 \times 10^{-9})(5.0 \times 10^{-9})|}{0.020^2} \\
                    & = 1.1 \times 10^{-4}\,\textrm{N}.
\end{align*}

However $q_2$ has a negative charge so it attracts $q_3$. It is also to the right of $q_3$ so $q_3$ experiences a force to the right of magnitude

\begin{align*}
  F_\textrm{2 on 3} & = \frac{1}{4\pi\epsilon_0}\frac{|q_2q_3|}{r^2}                                \\
                    & = (9.0 \times 10^9)\frac{|(-3.0 \times 10^{-9})(5.0 \times 10^{-9})}{0.040^2} \\
                    & = 8.4 \times 10^{-5}\,\textrm{N}.
\end{align*}

The net force experienced by $q_3$ is therefore

\begin{align*}
  F & = -F_\textrm{1 on 3} + F_\textrm{2 on 3}   \\
    & = -1.1 \times 10^{-4} + 8.4 \times 10^{-5} \\
    & = -2.6 \times 10^{-5}\,\textrm{N}.
\end{align*}

\subsubsection{Example 21.4}

Since $q_1$ and $q_2$ are of equal charge and are symmetric about the x axis on which $Q$ lies, the vertical components of their forces cancel leaving only the horizontal.

The horizontal component of $q_1$'s force on $Q$ is given by

\begin{align*}
  F_\textrm{1 on Q, x} & = \frac{1}{4\pi\epsilon_0} \frac{q_1Q}{r_\textrm{1,Q}^2} \cos\alpha                                             \\
                       & = (9.0 \times 10^9) \frac{(2.0 \times 10^{-6})(4.0 \times 10^{-6})}{\sqrt{0.30^2 + 0.40^2}^2} \frac{0.40}{0.50} \\
                       & = 0.23\,\textrm{N}.
\end{align*}

Again, since $q_1$ and $q_2$ are of equal charge and symmetric about the x axis, $F_\textrm{1 on Q, x} = F_\textrm{2 on Q, x}$ and the total force experienced by $Q$ is in the positive x direction of magnitude \[F = 2F_\textrm{1 on Q, x} = 0.46\,\textrm{N}.\]

\subsection{Electric Field and Electric Forces}

\subsubsection{Example 21.5}

The magnitude of the electric field vector is given by

\begin{align*}
  E & = \frac{1}{4\pi\epsilon_0} \frac{|q|}{r^2}             \\
    & = (9.0 \times 10^9) \frac{|4.0 \times 10^{-9}|}{2.0^2} \\
    & = 9.0\,\textrm{N}/\textrm{C}.
\end{align*}

\subsubsection{Example 21.6}

The magnitude of the electric field vector is given by

\begin{align*}
  E & = \frac{1}{4\pi\epsilon_0} \frac{|q|}{r^2}                      \\
    & = (9.0 \times 10^9) \frac{|-8.0 \times 10^{-9}|}{1.2^2 + 1.6^2} \\
    & = 18\,\textrm{N}/\textrm{C}
\end{align*}

and it is directed towards the origin. If $\theta$ is the angle between the positive x axis and $\hat{\mathbf{r}}$ then the component form of $\mathbf{E}$ is

\begin{align*}
  E & = -E\left(\cos\theta\hat{\mathbf{i}} + \sin\theta\hat{\mathbf{j}}\right)                      \\
    & = -E\left(\frac{x}{r}\hat{\mathbf{i}} + \frac{-y}{r}\hat{\mathbf{j}}\right)                   \\
    & = \frac{-18}{\sqrt{1.2^2 + 1.6^2}}\left(1.2\hat{\mathbf{i}} + 1.6\hat{\mathbf{j}}\right)      \\
    & = (-11\,\textrm{N}/\textrm{C})\hat{\mathbf{i}} - (14\,\textrm{N}/\textrm{C})\hat{\mathbf{j}}.
\end{align*}

\subsubsection{Example 21.7}

\begin{enumerate}[a)]
  \item Electrons have a negative charge and the electric field is directed upwards, so the electron will move downwards. The magnitude of its acceleration is

        \begin{align*}
          a & = \frac{F}{m}                                                           \\
            & = \frac{eE}{m}                                                          \\
            & = \frac{(1.60 \times 10^{-19})(1.00 \times 10^4)}{9.11 \times 10^{-31}} \\
            & = 1.76 \times 10^{15}\,\textrm{m}/\textrm{s}^2.
        \end{align*}

  \item Its acceleration is constant between the plates, so its final speed is

        \begin{align*}
          v^2 & = v_0^2 + 2a(x - x_0)                      \\
              & = 2ax                                      \\
          v   & = \sqrt{2ax}                               \\
              & = \sqrt{2(1.76 \times 10^{15})(0.01)}      \\
              & = 5.9 \times 10^6\,\textrm{m}/\textrm{s}^2
        \end{align*}

        and thus its final kinetic energy is

        \begin{align*}
          K & = \frac{1}{2}mv^2                                      \\
            & = \frac{1}{2}(9.11 \times 10^{-31})(5.9 \times 10^6)^2 \\
            & = 1.6 \times 10^{-17}\,\textrm{J}.
        \end{align*}

  \item We can find the time it takes for the electron to travel this distance by rearranging the kinematic equation \[v = v_0 + at\] to

        \begin{align*}
          t & = \frac{v - v_0}{a}                           \\
            & = \frac{5.9 \times 10^6}{1.76 \times 10^{15}} \\
            & = 3.4 \times 10^{-9}\,\textrm{s}.
        \end{align*}
\end{enumerate}

\subsection{Electric-Field Calculations}

\subsubsection{Example 21.8}

\begin{enumerate}[a)]
  \item At point $a$ the electric field caused by $q_1$ points to the right and has magnitude

        \begin{align*}
          E_1 & = \frac{1}{4\pi\epsilon_0} \frac{|q_1|}{r^2}            \\
              & = (9.0 \times 10^9) \frac{12 \times 10^{-9}}{(0.060)^2} \\
              & = 3.0 \times 10^4\,\textrm{N}/\textrm{C}.
        \end{align*}

        The electric field caused by $q_2$ also points to the right and it has magnitude

        \begin{align*}
          E_2 & = \frac{1}{4\pi\epsilon_0} \frac{|q_2|}{r^2}               \\
              & = (9.0 \times 10^9) \frac{|-12 \times 10^{-9}|}{(0.040)^2} \\
              & = 6.8 \times 10^4\,\textrm{N}/\textrm{C}.
        \end{align*}

        Thus the total field points to the right and has magnitude \[E = E_1 + E_2 = 9.8 \times 10^4\,\textrm{N}/\textrm{C}.\]

  \item At point $b$ the electric field caused by $q_1$ points to the left and has magnitude

        \begin{align*}
          E_1 & = \frac{1}{4\pi\epsilon_0} \frac{|q_1|}{r^2}            \\
              & = (9.0 \times 10^9) \frac{12 \times 10^{-9}}{(0.040)^2} \\
              & = 6.8 \times 10^4\,\textrm{N}/\textrm{C}.
        \end{align*}

        The electric field caused by $q_2$ points to the right and has magnitude

        \begin{align*}
          E_2 & = \frac{1}{4\pi\epsilon_0} \frac{|q_2|}{r^2}               \\
              & = (9.0 \times 10^9) \frac{|-12 \times 10^{-9}|}{(0.140)^2} \\
              & = 0.55 \times 10^4\,\textrm{N}/\textrm{C}.
        \end{align*}

        Thus the total electric field points to the left and has magnitude \[E = E_1 - E_2 = 6.3 \times 10^4\,\textrm{N}/\textrm{C}.\]

  \item At point $c$ the electric field caused by $q_1$ points from $q_1$ to $c$ and has magnitude

        \begin{align*}
          E_1 & = \frac{1}{4\pi\epsilon_0} \frac{|q_1|}{r^2}            \\
              & = (9.0 \times 10^9) \frac{|12 \times 10^{-9}|}{0.130^2} \\
              & = 6.4 \times 10^3\,\textrm{N}/\textrm{C}.
        \end{align*}

        The electric field caused by $q_2$ points from $c$ to $q_2$ and has magnitude

        \begin{align*}
          E_2 & = \frac{1}{4\pi\epsilon_0} \frac{|q_2|}{r^2}             \\
              & = (9.0 \times 10^9) \frac{|-12 \times 10^{-9}|}{0.130^2} \\
              & = 6.4 \times 10^3\,\textrm{N}/\textrm{C}                 \\
              & = E_1.
        \end{align*}

        The vertical components of $\mathbf{E_1}$ and $\mathbf{E_2}$ cancel, leaving only a horizontal component pointing to the right of magnitude

        \begin{align*}
          E & = 2E_1\cos\alpha                          \\
            & = 2(6.4 \times 10^3)\frac{0.050}{0.130}   \\
            & = 4.9 \times 10^3\,\textrm{N}/\textrm{C}.
        \end{align*}
\end{enumerate}

\subsubsection{Example 21.9}

By symmetry, each point on the ring has a corresponding point on the opposite side. The components of their electric fields perpendicular to the axis of the ring cancel, leaving only a component parallel to the axis of the ring. Thus the total magnetic field at $P$ is parallel to the axis of the ring and can be calculated as

\begin{align*}
  E & = \int_0^{2\pi} \frac{1}{4\pi\epsilon_0} \frac{\lambda}{r^2} \cos \alpha\,d\theta \\
    & = \frac{1}{4\pi\epsilon_0} \frac{Qx}{2\pi(a^2 + x^2)^{3/2}} \int_0^{2\pi} d\theta \\
    & = \frac{1}{4\pi\epsilon_0} \frac{Qx}{(a^2 + x^2)^{3/2}}.
\end{align*}

\subsubsection{Example 21.10}

By symmetry, each point on the line has a corresponding point on the opposite side of the $x$-axis. The $y$ components of their electric fields cancel, leaving only the $x$ components. Thus the total magnetic field at $P$ only has an $x$ component and can be calculated as

\begin{align*}
  E & = \int_{-a}^a \frac{1}{4\pi\epsilon_0} \frac{\lambda}{r^2} \cos\alpha\,dy                                        \\
    & = \frac{1}{4\pi\epsilon_0} \frac{Qx}{2a} \int_{-a}^a \frac{1}{(x^2 + y^2)^{3/2}}\,dy                             \\
    & = \frac{1}{4\pi\epsilon_0} \frac{Qx}{2a} \left[\frac{y}{x^2\sqrt{x^2+y^2}}\right]_{-a}^a                         \\
    & = \frac{1}{4\pi\epsilon_0} \frac{Q}{2ax} \left(\frac{a}{\sqrt{x^2 + a^2}} + \frac{a}{\sqrt{x^2 + (-a)^2}}\right) \\
    & = \frac{1}{4\pi\epsilon_0} \frac{Q}{x\sqrt{x^2 + a^2}}.
\end{align*}

\subsubsection{Example 21.11}

By symmetry, each point on the disk has a corresponding point 180° rotation around the $x$-axis. The $y$ and $z$ components of their electric fields cancel, leaving only the $x$ components. Thus the total magnetic field at $P$ only has an $x$ component and can be calculated as

\begin{align*}
  E & = \int_0^R \int_0^{2\pi} \frac{1}{4\pi\epsilon_0} \frac{\sigma}{r^2} s \cos\alpha\,d\theta\,ds                       \\
    & = \frac{\sigma}{4\pi\epsilon_0} \int_0^R \int_0^{2\pi} \frac{s}{s^2 + x^2} \frac{x}{\sqrt{s^2 + x^2}} \,d\theta \,ds \\
    & = \frac{\sigma x}{2\epsilon_0} \int_0^R \frac{s}{(s^2 + x^2)^{3/2}} \,ds                                             \\
    & = \frac{\sigma x}{2\epsilon_0} \left[-\frac{1}{\sqrt{s^2 + x^2}}\right]_0^R                                          \\
    & = \frac{\sigma x}{2\epsilon_0} \left(-\frac{1}{\sqrt{R^2 + x^2}} + \frac{1}{x}\right)                                \\
    & = \frac{\sigma}{2\epsilon_0} \left(1 - \frac{1}{\sqrt{(R/x)^2 + 1}}\right).
\end{align*}

\subsubsection{Example 21.12}

From Example 21.11 we know that the electric field produced by an infinite plane sheet of charge is \[E= \frac{\sigma}{2\epsilon_0}.\] Therefore the electric field outside the sheets is $\mathbf{0}$ and between the sheets is $\sigma/\epsilon_0$ towards the negative sheet.

\setcounter{subsection}{6}
\subsection{Electric Dipoles}

\subsubsection{Example 21.13}

\begin{enumerate}[a)]
  \item The electric field is uniform so the net force exerted on the dipole is $\mathbf{0}$

  \item The electric dipole moment is directed from the negative charge to the positive charge and has magnitude \[p = qd = (1.6 \times 10^{-19})(0.125 \times 10^{-9}) = 2.0 \times 10^{-29}\,\textrm{C}\cdot\textrm{m}\]

  \item The torque aligns the electric dipole moment with the electric field so it is directed out of the page and has magnitude \[\tau = qEd\sin\phi = (1.6 \times 10^{-19})(5.0 \times 10^5)(0.125 \times 10^{-9})\sin 35 = 5.7 \times 10^{-24}\,\textrm{N}\cdot\textrm{m}\]

  \item The potential energy of an electric dipole in a uniform electric field is given by \[U = -qdE\cos\phi = (2.0 \times 10^{-29})(5.0 \times 10^5)\cos 35 = 8.2 \times 10^{-24}\,\textrm{J}\]
\end{enumerate}

\subsubsection{Example 21.14}

As $P$ is on the $y$-axis, the electric fields of the electric dipole's point charges have no $x$ component and thus the net electric field is directed along the $y$-axis.

By the principle of superposition of electric fields, the magnitude of the electric field at $P$ is

\begin{align*}
  E & = E_- + E_+                                                                                                                         \\
    & = \frac{1}{4\pi\epsilon_0} \frac{-q}{(y - (-d/2))^2} + \frac{1}{4\pi\epsilon_0} \frac{q}{(y - d/2)^2}                               \\
    & = \frac{1}{4\pi\epsilon_0}q\left(\frac{1}{(y-d/2)^2} - \frac{1}{(y + d/2)^2}\right)                                                 \\
    & = \frac{1}{4\pi\epsilon_0} \frac{q}{y^2} \left( \left( 1 - \frac{d}{2y} \right)^{-2} - \left( 1 + \frac{d}{2y} \right)^{-2} \right) \\
    & \approx \frac{1}{4\pi\epsilon_0} \frac{q}{y^2} \left( 1 + \frac{d}{y} - 1 + \frac{d}{y} \right)                                     \\
    & = \frac{qd}{2\pi\epsilon_0y^3}                                                                                                      \\
    & = \frac{p}{2\pi\epsilon_0y^3}.
\end{align*}

\subsection{Guided Practice}

\subsubsection{VP21.4.1}

$q_1$ attracts $q_3$ to the left with magnitude

\begin{align*}
  F_1 & = \frac{1}{4\pi\epsilon_0} \frac{|q_1q_3|}{r^2}                                   \\
      & = (9.0 \times 10^9) \frac{|(4.00 \times 10^{-9})(-2.00 \times 10^{-9})}{0.0400^2} \\
      & = 4.5 \times 10^{-5} \,\textrm{N}.
\end{align*}

$q_2$ repels $q_3$ to the left with magnitude

\begin{align*}
  F_2 & = \frac{1}{4\pi\epsilon_0} \frac{|q_2 q_3|}{r^2}                                              \\
      & = (9.0 \times 10^9) \frac{|(-1.20 \times 10^{-9})(-2.00 \times 10^{-9}|}{(0.0600 - 0.0400)^2} \\
      & = 5.4 \times 10^{-5} \,\textrm{N}.
\end{align*}

By the principle of superposition of forces, the net force on $q_3$ is \[\mathbf{F} = (-F_1 - F_2)\hat{\mathbf{i}} = (-9.9 \times 10^{-5} \,\textrm{N}) \hat{\mathbf{i}}.\]

\subsubsection{VP21.4.2}

\begin{enumerate}[a)]
  \item $q_1$ repels $q_2$ in the positive $x$ direction with magnitude

        \begin{align*}
          F_1 & = \frac{1}{4\pi\epsilon_0} \frac{|q_1 q_2|}{r^2}                                                       \\
              & = \frac{1}{4 \pi (8.854 \times 10^{-12})} \frac{|(3.60 \times 10^{-9})(2.00 \times 10^{-9})}{0.0400^2} \\
              & = 40.4 \,\mu\textrm{N}.
        \end{align*}

  \item By the superposition of forces

        \begin{align*}
          F   & = F_1 + F_2            \\
          F_2 & = F - F_1              \\
              & = 54.0 - 40.4          \\
              & = 13.6 \,\mu\textrm{N}
        \end{align*}

        in the positive $x$ direction.

  \item $q_2$ repels $q_3$ so it must also have a positive charge of magnitude

        \begin{align*}
          F   & = \frac{1}{4\pi\epsilon_0} \frac{q_2 q_3}{r^2}                                           \\
          q_2 & = \frac{4\pi\epsilon_0 F r^2}{q_3}                                                       \\
              & = \frac{4\pi(8.854 \times 10^{-12})(1.36 \times 10^{-5})(0.0800)^2}{2.00 \times 10^{-9}} \\
              & = 4.84 \times 10^{-9} \,\textrm{C}.
        \end{align*}
\end{enumerate}

\subsubsection{VP21.4.3}

By symmetry the $x$ components of $q_1$ and $q_2$'s electric fields cancel leaving only their $y$ components which are directed in the negative $y$ direction and equal. $q_3$ is negative and thus experiences a net force in the positive $y$ direction of magnitude

\begin{align*}
  F & = 2 \frac{1}{4\pi\epsilon_0} \frac{q_1q_3}{r^2} \sin \alpha                                                                                            \\
    & = 2 \frac{1}{4\pi (8.854 \times 10^{-12})} \frac{(6.00 \times 10^{-9})(2.50 \times 10^{-9})}{0.150^2 + 0.200^2} \frac{0.200}{\sqrt{0.150^2 + 0.200^2}} \\
    & = 3.45 \times 10^{-6} \,\textrm{N}.
\end{align*}

\subsubsection{VP21.4.4}

The magnitude of the electric force exerted by $q_1$ on $q_3$ is

\begin{align*}
  F_1 & = \frac{1}{4 \pi \epsilon_0} \frac{|q_1 q_3|}{r^2}                                                                \\
      & = \frac{1}{4 \pi (8.854 \times 10^{-12})} \frac{|(4.00 \times 10^{-9}) (-1.50 \times 10^{-9})}{0.250^2 + 0.200^2} \\
      & = 5.26 \times 10^{-7} \,\textrm{N}.
\end{align*}

They have opposite charges so the force is directed from $q_3$ to $q_1$. In component form the force is

\begin{align*}
  \mathbf{F_1} & = -F_1\cos\alpha\hat{\mathbf{i}} + F_1\sin\alpha\hat{\mathbf{j}}                                                     \\
               & = F_1 \left( -\frac{x}{r}\hat{\mathbf{i}} + \frac{y}{r}\hat{\mathbf{j}} \right)                                      \\
               & = \frac{5.26 \times 10^{-7}}{\sqrt{0.250^2 + 0.200^2}} \left( -0.250\hat{\mathbf{i}} + 0.200\hat{\mathbf{j}} \right) \\
               & = (-4.11 \times 10^{-7}\,\textrm{N}) \hat{\mathbf{i}} + (3.29 \times 10^{-7}\,\textrm{N}) \hat{\mathbf{j}}.
\end{align*}

The magnitude of the electric force exerted by $q_2$ on $q_3$ is

\begin{align*}
  F_2 & = \frac{1}{4\pi\epsilon_0} \frac{|q_2 q_3|}{r^2}                                                          \\
      & = \frac{1}{4 \pi (8.854 \times 10^{-12})} \frac{|(-4.00 \times 10^{-9}) (-1.50 \times 10^{-9})|}{0.250^2} \\
      & = 8.63 \times 10^{-7} \,\textrm{N}.
\end{align*}

The have like charges so the force is directed from $q_2$ to $q_3$, i.e. along the positive $x$-axis. In component form the force is \[\mathbf{F_2} = (8.64 \times 10^{-7} \,\textrm{N})\hat{\mathbf{i}}.\]

Thus the net force experienced by $q_3$ is

\begin{align*}
  \mathbf{F} & = \mathbf{F_1} + \mathbf{F_2}                                                                                \\
             & = (4.53 \times 10^{-7} \,\textrm{N}) \hat{\mathbf{i}} + (3.29 \times 10^{-7} \,\textrm{N}) \hat{\mathbf{j}}.
\end{align*}

\subsubsection{VP21.10.1}

\begin{enumerate}[a)]
  \item The source points and field point all lie on the $y$-axis, so the source points' electric fields have no $x$ components. $q_1$ is positive and the field point is below it, so its contribution is negative. $q_2$ is negative and the field point is above it, so its contribution is also negative. Thus the $y$ component of the net electric field is

        \begin{align*}
          E_y & = \frac{1}{4 \pi \epsilon_0} \left( -\frac{q_1}{(y_1 - y)^2} + \frac{q_2}{(y_2 - y)^2} \right)                                                     \\
              & = \frac{1}{4 \pi (8.854 \times 10^{-12})} \left( \frac{4.00 \times 10^{-9}}{(0.200 - 0.100)^2} - \frac{5.00 \times 10^{-9}}{(0 - 0.100)^2} \right) \\
              & = -8.09 \times 10^3 \,\textrm{N}/\textrm{C}.
        \end{align*}

  \item The source points and field point all lie on the $y$-axis, so the source points' electric fields have no $x$ components. $q_1$ is positive and the field point is above it, so its contribution is positive. $q_2$ is negative and the field point is above it, so its contribution is also negative. Thus the $y$ component of the net electric field is

        \begin{align*}
          E_y & = \frac{1}{4 \pi \epsilon_0} \left( \frac{q_1}{(y_1 - y)^2} + \frac{q_2}{(y_2 - y)^2} \right)                                                      \\
              & = \frac{1}{4 \pi (8.854 \times 10^{-12})} \left( \frac{4.00 \times 10^{-9}}{(0.200 - 0.400)^2} - \frac{5.00 \times 10^{-9}}{(0 - 0.400)^2} \right) \\
              & = 618 \,\textrm{N}/\textrm{C}.
        \end{align*}

  \item The electric field of $q_1$ has magnitude

        \begin{align*}
          E_1 & = \frac{1}{4 \pi \epsilon_0} \frac{q_1}{r^2}                                            \\
              & = \frac{1}{4 \pi (8.854 \times 10^{-12})} \frac{4.00 \times 10^{-9}}{0.200^2 + 0.200^2} \\
              & = 449 \,\textrm{N}/\textrm{C}.
        \end{align*}

        It is directed from $q_1$ to the field point and thus in component form is

        \begin{align*}
          \mathbf{E_1} & = E_1(\cos\phi\hat{\mathbf{i}} - \sin\phi\hat{\mathbf{j}})                                       \\
                       & = \frac{449}{\sqrt{0.200^2 + 0.200^2}}(0.200\hat{\mathbf{i}} - 0.200\hat{\mathbf{j}})            \\
                       & = (317 \,\textrm{N}/\textrm{C})\hat{\mathbf{i}} - (317 \,\textrm{N}/\textrm{C})\hat{\mathbf{j}}.
        \end{align*}

        $q_2$ and the field point both lie on the $x$-axis, and thus its electric field has no $y$ component. In component form it is

        \begin{align*}
          \mathbf{E_2} & = \frac{1}{4 \pi \epsilon_0} \frac{q_2}{r^2} \hat{\mathbf{i}}                                   \\
                       & = \frac{1}{4 \pi (8.854 \times 10^{-12})} \frac{-5.00 \times 10^{-9}}{0.200^2} \hat{\mathbf{i}} \\
                       & = (-1.12 \times 10^{3} \,\textrm{N}/\textrm{C})\hat{\mathbf{i}}.
        \end{align*}

        The total electric field is thus

        \begin{align*}
          \mathbf{E} & = \mathbf{E_1} + \mathbf{E_2}                                                                                                \\
                     & = (-8.03 \times 10^2 \,\textrm{N}/\textrm{C})\hat{\mathbf{i}} + (-3.17 \times 10^2 \,\textrm{N}/\textrm{C})\hat{\mathbf{j}}.
        \end{align*}
\end{enumerate}

\subsubsection{VP21.10.2}

\begin{enumerate}[a)]
  \item Both source points and the field point are on the $x$-axis, so the electric fields at $P$ have no $y$ components.

        $q_1$ is positive and $P$ is to the right of $q_1$, so its electric field points to the right and has magnitude

        \begin{align*}
          E_1 & = \frac{1}{4 \pi \epsilon_0} \frac{q_1}{r^2}                                   \\
              & = \frac{1}{4 \pi (8.854 \times 10^{-12})} \frac{1.80 \times 10^{-9}}{0.0200^2} \\
              & = 4.04 \times 10^4 \,\textrm{N}/\textrm{C}.
        \end{align*}

  \item The magnitude and direction of the electric field that $q_2$ causes at $P$ can be calculated as

        \begin{align*}
          E   & = E_1 + E_2                                 \\
          E_2 & = E - E_1                                   \\
              & = 6.75 \times 10^4 - 4.04 \times 10^4       \\
              & = 2.71 \times 10^4 \,\textrm{N}/\textrm{C}.
        \end{align*}

  \item $E_2$ is positive at $P$, so $q_2$ must be negative. Its value is

        \begin{align*}
          E_2 & = -\frac{1}{4 \pi \epsilon_0} \frac{q_2}{r^2}                  \\
          q_2 & = -4 \pi \epsilon_0 E_2 r^2                                    \\
              & = -4 \pi (8.854 \times 10^{-12}) (2.71 \times 10^4) (0.0200)^2 \\
              & = -1.21 \times 10^{-9} \,\textrm{C}.
        \end{align*}
\end{enumerate}

\subsubsection{VP21.10.3}

\newcommand{\ke}{\frac{1}{4 \pi \epsilon_0}}

\begin{enumerate}[a)]
  \item From Example 21.9 we know that the electric field of a charged ring of radius $a$ at a distance $x$ along the ring's axis is directed away from the ring along its axis and has magnitude \[E = \frac{1}{4 \pi \epsilon_0} \frac{Qx}{(x^2 + a^2)^{3/2}}.\]

        By the principle of superposition of electric fields, the electric field of the hydrogen atom is

        \begin{align*}
          E & = \ke \left( \frac{e}{a^2} - \frac{ea}{(a^2 + a^2)^{3/2}} \right) \\
            & = \ke e \left( \frac{1}{a^2} - \frac{a}{2\sqrt{2}a^3} \right)     \\
            & = \ke \frac{e}{a^2} \left( 1 - \frac{1}{2\sqrt{2}} \right).
        \end{align*}

  \item $1 - 1/(2\sqrt{2}) \approx 0.65$ so the field points away from the proton.
\end{enumerate}

\subsubsection{VP21.10.4}

\begin{enumerate}[a)]
  \item The charge per unit length is \[\lambda = \frac{Q}{L}\] so the charge contained in a segment of length $dx$ is \[\lambda\,dx = \frac{Q}{L}\,dx.\]

  \item The field and source points both lie on the $x$-axis, so the differential electric field has no $y$ component. The $x$ component is \[dE_x = -\ke \frac{\lambda \, dx}{x^2} = -\ke \frac{Q}{Lx^2} \, dx.\]

  \item The total electric field at the origin is

        \begin{align*}
          E & = \int dE_x                                                   \\
            & = \int_L^{2L} -\ke \frac{Q}{Lx^2} \, dx                       \\
            & = -\ke \frac{Q}{L} \left[ -\frac{1}{x} \right]_L^{2L}         \\
            & = -\ke \frac{Q}{L} \left( -\frac{1}{2L} + \frac{1}{L} \right) \\
            & = -\ke \frac{Q}{2L^2}.
        \end{align*}
\end{enumerate}

\subsubsection{VP21.14.1}

\begin{enumerate}[a)]
  \item The magnitude of the torque is given by

        \begin{align*}
          \tau & = p E \sin \theta                                 \\
               & = (\num{6.13e-30}) (\num{3.00e5}) \sin \ang{50.0} \\
               & = \qty{1.41e-24}{N.m}.
        \end{align*}

  \item The potential energy is given by

        \begin{align*}
          U & = -p E \cos \theta                                 \\
            & = -(\num{6.13e-30}) (\num{3.00e5}) \cos \ang{50.0} \\
            & = \qty{-1.18e-24}{J}.
        \end{align*}
\end{enumerate}

\subsubsection{VP21.14.2}

To find the magnitude of the charges we can rearrange the torque equation

\begin{align*}
  \tau & = p E \sin \theta                                                      \\
       & = q d E \sin \theta                                                    \\
  q    & = \frac{\tau}{d E \sin \theta}                                         \\
       & = \frac{\num{6.60e-26}}{(\num{1.10e-10}) (\num{8.50e4}) \sin \ang{90}} \\
       & = \qty{7.06e-21}{C}.
\end{align*}

\subsubsection{VP21.14.3}

When the dipole moment is parallel to the field its potential energy is $-p E$ and when it is antiparallel its potential energy is $p E$. Thus, the work required to perform the rotation is $2 p E$ and

\begin{align*}
  W & = 2 p E                                   \\
  p & = \frac{W}{2 E}                           \\
    & = \frac{\num{4.60e-25}}{2 (\num{1.20e5})} \\
    & = \qty{1.92e-30}{C.m}.
\end{align*}

\subsubsection{VP21.14.4}

\newcommand{\kev}{\num{8.988e9}}

\begin{enumerate}[a)]
  \item Rearranging the equation for the dipole moment gives

        \begin{align*}
          p & = q d                                   \\
          d & = \frac{p}{q}                           \\
            & = \frac{\num{3.50e-29}}{\num{1.60e-19}} \\
            & = \qty{2.19e-10}{m}.
        \end{align*}

  \item From Example 21.14, the electric field of the molecule along its axis is \[E_y = \ke \frac{2p}{y^3}.\]

        Rearranging for $y$ and substituting in the desired field strength gives

        \begin{align*}
          y & = \left( \ke \frac{2 p}{E_y} \right)^{1/3}                                             \\
            & = \left( (\kev) \frac{2 (\num{1.60e-19}) (\num{2.19e-10})}{\num{8.00e4}} \right)^{1/3} \\
            & = \qty{1.99e-8}{m}.
        \end{align*}
\end{enumerate}

\subsubsection{Bridging Problem}

By symmetry, each point on the semicircle has a corresponding point on the opposite side of the $y$-axis. The $x$ components of their electric fields cancel, leaving only the $y$ components. Thus, the total electric field at $P$ points in the negative $y$ direction and has magnitude

\begin{align*}
  E & = \int_0^\pi \ke \frac{\lambda a \, d\theta}{a^2} \sin \theta \\
    & = \ke \frac{Q}{\pi a^2} \int_0^\pi \sin \theta \, d\theta     \\
    & = \ke \frac{Q}{\pi a^2} \left[ -\cos \theta \right]_0^\pi     \\
    & = \ke \frac{2Q}{\pi a^2}.
\end{align*}

\section{Gauss's Law}

\setcounter{subsection}{1}
\subsection{Calculating Electric Flux}

\subsubsection{Example 22.1}

\begin{enumerate}[a)]
  \item The electric flux is given by

        \begin{align*}
          \Phi_E & = A E \cos \theta                          \\
                 & = \pi (0.10)^2 (\num{2.0e3}) \cos \ang{30} \\
                 & = \qty{54}{N.m^2/C}.
        \end{align*}

  \item $\Phi_E = 0$

  \item $\Phi_E = A E = \qty{63}{N.m^2/C}$
\end{enumerate}

\subsubsection{Example 22.2}

\begin{enumerate}[a)]
  \item Four of the six sides are parallel to $E$ and thus contribute no flux. The remaining two surfaces contribute $E L^2$ and $-E L^2$ which add to $0$ and also contribute no flux. The total flux is $0$.

  \item Two of the six sides are parallel to $E$ and thus contribute no flux. Two of the remaining four sides contribute $E L^2 \cos \theta$ and $-E L^2 \cos \theta$ which add to $0$ and also contribute no flux. The final two sides contribute $E L^2 \sin \theta$ and $-E L^2 \sin \theta$ which also add to $0$ and contribute no flux. The total flux is $0$.
\end{enumerate}

\subsubsection{Example 22.3}

At all points the electric field is normal to the sphere and its magnitude is \[E = \ke \frac{q}{r^2} = (\kev) \frac{\num{3.0e-6}}{0.20^2} = \qty{6.74e5}{N/C}.\]

The total flux is thus \[\Phi_E = AE = 4 \pi r^2 E = 4 \pi (0.20)^2 (\num{6.74e5}) = \qty{3.4e5}{N.m^2/C}.\]

\setcounter{subsection}{3}
\subsection{Applications of Gauss's Law}

\subsubsection{Example 22.5}

Under electrostatics the electric field inside a conductor is always $\mathbf{0}$.

Using a Gaussian sphere of radius $r > R$ centred on the conducting sphere we can use Gauss's law to calculate the electric field because $\mathbf{E}$ will be perpendicular to $\mathbf{dA}$ and have a constant magnitude

\begin{align*}
  \oint \mathbf{E} \cdot \mathbf{dA} & = \frac{Q_\textrm{enc}}{\epsilon_0}                \\
  E \oint dA                         & = \frac{Q_\textrm{enc}}{\epsilon_0}                \\
  E                                  & = \ke \frac{Q_\textrm{enc}}{r^2}                   \\
  \mathbf{E}                         & = \ke \frac{Q_\textrm{enc}}{r^2} \hat{\mathbf{r}}.
\end{align*}

\subsubsection{Example 22.6}

By symmetry the electric field must only have a radial component. Using a Gaussian cylinder of length $L$ and radius $r$ centred on the line $E$ will be perpendicular to $dA$ and have constant magnitude so we can use Gauss's law to derive the electric field

\begin{align*}
  \oint \mathbf{E} \cdot \mathbf{dA} & = \frac{Q_\textrm{enc}}{\epsilon_0}         \\
  E \oint dA                         & = \frac{L \lambda}{\epsilon_0}              \\
  E                                  & = \frac{L \lambda}{2 \pi r L \epsilon_0}    \\
  \mathbf{E}                         & = \ke \frac{2 \lambda}{r} \hat{\mathbf{r}}.
\end{align*}

\subsubsection{Example 22.7}

By symmetry, the electric field must be directed away from the sheet. Using a Gaussian cube of side $L$, $E$ will be perpendicular to $dA$ on the far sides of the box and have constant magnitude so we can use Gauss's law to dervice the electric field

\begin{align*}
  \oint \mathbf{E} \cdot \mathbf{dA} & = \frac{Q_\textrm{enc}}{\epsilon_0} \\
  2 E L^2                            & = \frac{\sigma L^2}{\epsilon_0}     \\
  E                                  & = \frac{\sigma}{2 \epsilon_0}.
\end{align*}

\subsubsection{Example 22.8}

First we position a Gaussian cylinder such that one end is within the plate and the other is between the plates. The end that is within the plate experiences no flux because there are no electric fields within conductors. We can use Gauss's law to find the electric field at the other end and thus between the plates

\begin{align*}
  \oint \mathbf{E} \cdot \mathbf{dA} & = \frac{Q_\textrm{enc}}{\epsilon_0} \\
  E A                                & = \frac{\sigma A}{\epsilon_0}       \\
  E                                  & = \frac{\sigma}{\epsilon_0}.
\end{align*}

\subsubsection{Example 22.9}

From previous examples we know that for $r > R$ the sphere's electric field is equivalent to that of a point charge \[E = \ke \frac{Q}{r^2}.\]

The sphere is insulating so for $r < R$ its electric field isn't necessarily $\mathbf{0}$. However by symmetry its electric field must only have a radial component so we can use Gauss's law

\begin{align*}
  \oint \mathbf{E} \cdot \mathbf{dA} & = \frac{Q_\textrm{enc}}{\epsilon_0}                                \\
  E \oint dA                         & = \frac{V \rho}{\epsilon_0}                                        \\
  E (4 \pi r^2)                      & = \frac{4}{3} \pi r^3 \frac{Q}{(4/3) \pi R^3} \frac{1}{\epsilon_0} \\
  \mathbf{E}                         & = \ke \frac{Q r}{R^3} \hat{\mathbf{r}}.
\end{align*}

\subsubsection{Example 22.10}

For $r > R$ the electric field is as if all the charge were concentrated at the centre of the sphere. We can rearrange the equation for the electric field of a point charge to find the magnitude of the charge

\begin{align*}
  E & = \ke \frac{q}{r^2}      \\
  q & = 4 \pi \epsilon_0 E r^2 \\
    & = \qty{-1.80e-9}{C}.
\end{align*}

\subsection{Charges on Conductors}

\subsubsection{Example 22.12}

\begin{enumerate}[a)]
  \item Rearranging the equation for the electric field at the surface of a conductor gives

        \begin{align*}
          E_\perp & = \frac{\sigma}{\epsilon_0} \\
          \sigma  & = E_\perp \epsilon_0        \\
                  & = (-150) (\num{8.85e-12})   \\
                  & = -\qty{1.33e-9}{C/m^2}.
        \end{align*}

  \item Multiplying $\sigma$ by the surface area of the Earth gives

        \begin{align*}
          Q & = 4 \pi R^2 \sigma                        \\
            & = 4 \pi (\num{6.38e6})^2 (\qty{-1.33e-9}) \\
            & = \qty{-6.80e5}{C}.
        \end{align*}
\end{enumerate}

\subsection{Guided Practice}

\subsubsection{VP22.4.4}

\begin{enumerate}[a)]
  \item $q_1$ is within the Gaussian surface, so \[\Phi_E = \frac{Q_\textrm{enc}}{\epsilon_0} = \frac{\num{3.00e-9}}{8.854e-12} = \qty{339}{N.m^2/C}.\]

  \item $q_2$ is also within the Guassian surface, so \[\Phi_E = \frac{(\num{3.00e-9}) + (\num{-8.00e-9})}{\num{8.854e-12}} = \qty{-565}{N.m^2/C}.\]

  \item $q_3$ is outside the Guassian surface so the flux is unchanged from part b.
\end{enumerate}

\subsubsection{VP22.10.4}

From Examples 21.12 and 22.8 we know that the electric field between the plates of a capacitor is \[E = \frac{\sigma}{\epsilon_0}.\]

Rearranging the equation for electric force we find

\begin{align*}
  F                  & = qE                                                      \\
                     & = q\frac{\sigma}{\epsilon_0}                              \\
  \Rightarrow \sigma & = \frac{\epsilon_0 F}{q}                                  \\
                     & = \frac{(\num{8.854e-12}) (\num{22.0e-6})}{\num{3.60e-9}} \\
                     & = \qty{5.41e-8}{C/m^2}.
\end{align*}

\subsubsection{VP22.12.1}

The inner surface of the block must have a total charge of $\qty{-3.00}{nC}$ to balance the charge within the cavity, leaving $\qty{-5.00}{nC}$ for the outer surface.

\subsubsection{VP22.12.4}

\begin{enumerate}[a)]
  \item The inner surface must have a total charge of $\qty{-4.00}{nC}$ to balance charge within the cavity, leaving $\qty{-2.00}{nC}$ for the outer surface.

  \item Found in part a: $\qty{-2.00}{nC}$.

  \item $E = \qty{2.50e3}{N/C}$

  \item $E = \qty{1.65e2}{N/C}$
\end{enumerate}

\subsubsection{Bridging Problem}

\begin{enumerate}[a)]
  \item The sphere will include the proton's charge ($+Q$) plus a portion of the electron's charge.

        \begin{align*}
          q & = Q + \int_0^r \rho \, dV                                          \\
            & = Q + \int_0^r -\frac{Q}{\pi a_0^3} e^{-2r'/a_0} 4 \pi r'^2 \, dr' \\
            & = Q - \frac{4Q}{a_0^3} \int_0^r e^{-2r'/a_0} r'^2 \, dr'.          \\
        \end{align*}

        Using integration by parts

        \begin{align*}
          \int e^{-2 r / a_0} r^2 \, dr                    & = -\frac{a_0}{2} e^{-2 r / a_0} r^2 - \int -a_0 e^{-2 r / a_0} r \, dr                                                            \\
                                                           & = -\frac{a_0}{2} e^{-2 r / a_0} r^2 - \left( \frac{a_0^2}{2} e^{-2 r / a_0} r - \int \frac{a_0^2}{2} e^{-2 r / a_0} \, dr \right) \\
                                                           & = -\frac{a_0}{2} e^{-2 r / a_0} r^2 - \frac{a_0^2}{2} e^{-2 r / a_0} r - \frac{a_0^3}{4} e^{-2 r / a_0} + c                       \\
                                                           & = -\frac{a_0}{4} e^{-2 r / a_0} \left( 2 r^2 + 2 a_0 r + a_0^2 \right) + c                                                        \\
          \Rightarrow \int_0^r e^{-2 r' / a_0} r'^2 \, dr' & = \frac{1}{4} a_0 \left( a_0^2 - e^{-2 r / a_0} \left( 2 r^2 + 2 a_0 r + a_0^2 \right) \right)
        \end{align*}

        Therefore

        \begin{align*}
          q & = Q - \frac{4 Q}{a_0^3} \frac{1}{4} a_0 \left( a_0^2 - e^{-2 r / a_0} \left( 2 r^2 + 2 a_0 r + a_0^2 \right) \right) \\
            & = Q \left( 1 - 1 + \frac{e^{-2 r / a_0}}{a_0^2} \left( 2 r^2 + 2 a_0 r + a_0^2 \right) \right)                       \\
            & = Q e^{-2 r / a_0} \left( 2 \left( \frac{r}{a_0} \right)^2 + 2 \frac{r}{a_0} + 1 \right).
        \end{align*}

  \item By Gauss's law and symmetry we know that the electric field will only have a radial component and will have magnitude \[E = \ke \frac{q}{r^2}.\]
\end{enumerate}

\subsection{Exercises}

\subsubsection{22.2}

The electric field is constant so we can calculate the electric flux as

\begin{align*}
  \Phi_E & = \mathbf{A} \cdot \mathbf {E}       \\
         & = A E \cos \theta                    \\
         & = (0.400)(0.600)(90.0) \cos \ang{70} \\
         & = \qty{7.4}{N.m^2/C}.
\end{align*}

\subsubsection{22.9}

\begin{enumerate}[a)]
  \item Assuming the plastic sphere has no charge, the electric field is $\mathbf{0}$.

  \item By Gauss's law and symmetry the electric field is as if all charge were concentrated at the centre of the sphere, giving $E = \qty{-1.22e8}{N/C}$.

  \item Using the same logic as in part b, $E = \qty{-3.64e7}{N/C}$.
\end{enumerate}

\subsubsection{22.13}

The electric field of an infinite line charge is \[\mathbf{E} = \ke \frac{2 \lambda}{r} \hat{\mathbf{r}}.\]

The force experienced by the section is

\begin{align*}
  F & = \int dF                                                   \\
    & = \int E \, dq                                              \\
    & = \int_0^L \ke \frac{2 \lambda}{r} \lambda \, dx            \\
    & = \ke \frac{2 \lambda^2}{r} \int_0^L dx                     \\
    & = (\num{8.988e9}) \frac{2 (\num{5.20e-6})^2}{0.300}(0.0500) \\
    & = \qty{0.0810}{N}.
\end{align*}

\subsubsection{22.23}

\begin{enumerate}[a)]
  \item The electric field of an infinite plane is \[E = \frac{\sigma}{2 \epsilon_0}.\]

        The point charge and sheet have opposite charges so are attracted to each other. The work done on the electron by the electric field of the sheet is \[W = F d = q \frac{\sigma}{2 \epsilon_0} d = (\num{1.60e-19}) \frac{\num{2.90e-12}}{2 (\num{8.85e-12})} (0.250) = \qty{6.56e-21}{J}.\]

  \item By the work-energy theorem, we can use the kinetic energy of the electron to calculate its velocity

        \begin{align*}
          W             & = \frac{1}{2} m v^2                                \\
          \Rightarrow v & = \sqrt{\frac{2 W}{m}}                             \\
                        & = \sqrt{\frac{2 (\num{6.56e-21})}{\num{9.11e-31}}} \\
                        & = \qty{1.20e5}{m/s}.
        \end{align*}
\end{enumerate}

\subsubsection{22.34}

The electric field is non-uniform so to calculate the electric flux we must perform an integral

\begin{align*}
  \Phi_E & = \int E \, dA                           \\
         & = \int_0^L \int_0^L kx \, dx \, dy       \\
         & = k L \left[ \frac{1}{2} x^2 \right]_0^L \\
         & = \frac{1}{2} k L^3                      \\
         & = \frac{1}{2} (964) (0.350)^3            \\
         & = \qty{20.7}{N.m^2/C}.
\end{align*}

\subsubsection{22.45}

\begin{enumerate}[a)]
  \item The charge contained within a sphere of radius $r$ is given by

        \begin{align*}
          q & = \int_a^r \rho \, dV                              \\
            & = \int_a^r \frac{\alpha}{r} 4 \pi r'^2 \, dr'      \\
            & = 4 \pi \alpha \int_a^r r' \, dr'                  \\
            & = 4 \pi \alpha \left[ \frac{1}{2} r'^2 \right]_a^r \\
            & = 2 \pi \alpha (r^2 - a^2).
        \end{align*}

        By symmetry and Gauss's law, the electric field at radius $r$ is the same as if all the charge were concentrated at the centre of the sphere so

        \begin{align*}
          E & = \ke \frac{q}{r^2}                         \\
            & = \ke \frac{2 \pi \alpha (r^2 - a^2)}{r^2}.
        \end{align*}

  \item The net electric field must be equal at $r=a$ and $r=b$

        \begin{align*}
          \ke \frac{1}{a^2} (2 \pi \alpha (a^2 - a^2) + q) & = \ke \frac{1}{b^2} (2 \pi \alpha (b^2 - a^2) + q) \\
          q                                                & = \frac{a^2}{b^2} (2 \pi \alpha (b^2 - a^2) + q)   \\
          q \left(1 - \frac{a^2}{b^2}\right)               & = \frac{a^2}{b^2} 2 \pi \alpha (b^2 - a^2)         \\
          q                                                & = 2 \pi \alpha a^2.
        \end{align*}

  \item The net electric field between $r=a$ and $r=b$ will be

        \begin{align*}
          E & = \ke \frac{1}{r^2} (2 \pi \alpha (r^2 - a^2) + 2 \pi \alpha a^2) \\
            & = \frac{\alpha}{2 \epsilon_0}.
        \end{align*}
\end{enumerate}

\subsubsection{22.61}

\begin{enumerate}[a)]
  \item By Gauss's law and symmetry, the electric field of the sphere is the same as if the charge were concentrated at its centre. Thus the electric flux through the round side of the cylinder is

        \begin{align*}
          \Phi_a & = \int \mathbf{E} \cdot \mathbf{dA}                                                                                              \\
                 & = \int_{-L / 2}^{L / 2} \ke \frac{Q}{r'^2} \cos \theta 2 \pi R \, dz                                                             \\
                 & = \frac{Q R^2}{2 \epsilon_0} \int_{-L / 2}^{L / 2} (R^2 + z^2)^{-3/2} \, dz                                                      \\
                 & = \frac{Q R^2}{2 \epsilon_0} \left[ \frac{z}{R^2 \sqrt{R^2 + z^2}} \right]_{-L / 2}^{L / 2}                                      \\
                 & = \frac{Q R^2}{2 \epsilon_0} \left( \frac{L / 2}{R^2 \sqrt{R^2 + (L / 2)^2}} + \frac{L / 2}{R^2 \sqrt{R^2 + (-L / 2)^2}} \right) \\
                 & = \frac{Q L}{2 \epsilon_0 \sqrt{R^2 + (L / 2)^2}}.
        \end{align*}

  \item The electric flux through the top of the cylinder is

        \begin{align*}
          \Phi_b & = \int \mathbf{E} \cdot \mathbf{dA}                                               \\
                 & = \int_0^R \ke \frac{Q}{r'^2} \cos \theta 2 \pi x \, dx                           \\
                 & = \frac{Q}{2 \epsilon_0} \int_0^R \frac{(L / 2) x}{((L / 2)^2 + x^2)^{3/2}} \, dx \\
                 & = \frac{Q L}{4 \epsilon_0} \int_0^R x ((L / 2)^2 + x^2)^{-3/2} \, dx
        \end{align*}

        Let $u = (L / 2)^2 + x^2$ so $du = 2 x \, dx$ and thus

        \begin{align*}
          \Phi_b & = \frac{Q L}{8 \epsilon_0} \int_{(L / 2)^2}^{(L / 2)^2 + R^2} u^{-3/2} \, du                           \\
                 & = \frac{Q L}{8 \epsilon_0} \left[ -2 u^{-1/2} \right]_{(L / 2)^2}^{(L / 2)^2 + R^2}                    \\
                 & = \frac{Q L}{4 \epsilon_0} \left( \frac{1}{\sqrt{(L / 2)^2}} -\frac{1}{\sqrt{(L / 2)^2 + R^2}} \right) \\
                 & = \frac{Q L}{4 \epsilon_0} \left( \frac{2}{L} - \frac{1}{\sqrt{(L / 2)^2 + R^2}} \right).
        \end{align*}

  \item The same as in part b.
\end{enumerate}

\section{Electric Potential}

\subsection{Electric Potential Energy}

\subsubsection{Example 23.1}

\begin{enumerate}[a)]
  \item The work done on the positron by the $\alpha$ particle's electric field is equal to the negative of the change in electric potential between them

        \begin{align*}
          W & = -\Delta U                                              \\
            & = -(U_b - U_a)                                           \\
            & = \ke q q_0 \left( \frac{1}{r_a} - \frac{1}{r_b} \right) \\
            & = \qty{2.30e-18}{J}.
        \end{align*}

        By the work-energy theorem this is equal to the positron's change in kinetic energy, so its final speed is

        \begin{align*}
          K_a + W                 & = K_b                          \\
          \frac{1}{2} m v_a^2 + W & = \frac{1}{2} m v_b^2          \\
          \Rightarrow v_b         & = \sqrt{v_a^2 + \frac{2 W}{m}} \\
                                  & = \qty{3.75e6}{m/s}.
        \end{align*}

  \item When the positron is very far from the $\alpha$ particle all of their electric potential energy has been converted into kinetic energy

        \begin{align*}
          K_a + U_a                                   & = K_c                                                           \\
          \frac{1}{2} m v_a^2 + \ke \frac{q q_0}{r_a} & = \frac{1}{2} m v_c^2                                           \\
          \Rightarrow v_c                             & = \sqrt{v_a^2 + \frac{1}{2 \pi \epsilon_0} \frac{q q_0}{m r_a}} \\
                                                      & = \qty{4.37e6}{m/s}.
        \end{align*}

  \item The electron's speed would decrease until it's stationary, then it would accelerate towards the $\alpha$ particle.
\end{enumerate}

\subsubsection{Example 23.2}

\begin{enumerate}[a)]
  \item The work that must be done to bring $q_3$ in from infinity is equal to its electric potential energy with the system at its final position

        \begin{align*}
          W & = U                                                                             \\
            & = \frac{q_3}{4 \pi \epsilon_0} \left( \frac{q_1}{r_1} + \frac{q_2}{r_2} \right) \\
            & = \frac{e}{4 \pi \epsilon_0} \left( \frac{-e}{2a} + \frac{e}{a} \right)         \\
            & = \frac{e^2}{8 \pi \epsilon_0 a}
        \end{align*}

  \item The total potential energy is equal to

        \begin{align*}
          E & = \ke \sum_{i < j} \frac{q_i q_j}{r_{ij}}                            \\
            & = \ke \left( \frac{-e^2}{a} - \frac{e^2}{2a} + \frac{e^2}{a} \right) \\
            & = -\frac{e^2}{8 \pi \epsilon_0 a}
        \end{align*}
\end{enumerate}

\subsection{Electric Potential}

\subsubsection{Exaple 23.3}

\begin{enumerate}[a)]
  \item The force on the proton is \[F = eE = (\num{1.602e-19})(\num{1.5e7}) = \qty{2.4e-12}{N}\]

  \item The work done on it by the field is \[W = dF = (\num{0.50})(\num{2.4e-12}) = \qty{1.2e-12}{J} = \qty{7.5}{MeV}\]

  \item The potential difference between the two points is \[V_a - V_b = dE = (0.50)(\num{1.5e7}) = \qty{7.5e6}{V}\]
\end{enumerate}

\subsubsection{Example 23.4}

\begin{enumerate}[a)]
  \item The electric potential at $a$ is

        \begin{align*}
          V & = \ke \left( \frac{q_1}{r_1} + \frac{q_2}{r_2} \right)                                 \\
            & = (\num{8.988e9}) \left( \frac{\num{12e-9}}{0.060} - \frac{\num{12e-9}}{0.040} \right) \\
            & = \qty{-9.0e2}{V}
        \end{align*}

  \item The electric potential at $b$ is

        \begin{align*}
          V & = \ke \left( \frac{q_1}{r_1} + \frac{q_2}{r_2} \right)                                 \\
            & = (\num{8.988e9}) \left( \frac{\num{12e-9}}{0.040} - \frac{\num{12e-9}}{0.140} \right) \\
            & = \qty{1.9e3}{V}
        \end{align*}

  \item $c$ is equidistant from both charges so the electric potential is $0$
\end{enumerate}

\subsubsection{Example 23.5}

\begin{enumerate}[a)]
  \item $U = qV = (\num{4.0e-9})(\num{-9.0e2}) = \qty{-3.6e-6}{J}$

  \item $U = qV = (\num{4.0e-9})(\num{1.9e3}) = \qty{7.6e-6}{J}$

  \item $U = 0$
\end{enumerate}

\subsubsection{Example 23.6}

\begin{align*}
  V_a - V_b           & = \int_a^b \mathbf{E} \cdot d\mathbf{l}       \\
  \Rightarrow V_r - 0 & = \int_r^\infty \ke \frac{q}{r'^2} \,dr'      \\
  V_r                 & = \ke q \left[ -\frac{1}{r'} \right]_r^\infty \\
                      & = \ke \frac{q}{r}
\end{align*}

\subsubsection{Example 23.7}

The potential at $a$ is

\begin{align*}
  V_a & = \ke \left( \frac{q_1}{r_1} + \frac{q_2}{r_2} \right)                                  \\
      & = (\num{8.988e9})\left( \frac{\num{3.0e-9}}{0.010} - \frac{\num{3.0e-9}}{0.020} \right) \\
      & = \qty{1.3e3}{V}.
\end{align*}

The potential at $b$ is

\begin{align*}
  V_b & = \ke \left( \frac{q_1}{r_1} + \frac{q_2}{r_2} \right)                                  \\
      & = (\num{8.988e9})\left( \frac{\num{3.0e-9}}{0.020} - \frac{\num{3.0e-9}}{0.010} \right) \\
      & = \qty{-1.3e3}{V}.
\end{align*}

So the potential difference is \[V_a - V_b = \qty{2.6e3}{V}\] the work done on the dust particle as it moves from $a$ to $b$ is \[q_0 (V_a - V_b) = (\num{2.0e-9})(\num{2.6e3}) = \qty{5.3e-6}{J}\] and its speed at $b$ is

\begin{align*}
  K             & = \frac{1}{2} m v^2                            \\
  \Rightarrow v & = \sqrt{\frac{2 K}{m}}                         \\
                & = \sqrt{\frac{2 (\num{5.3e-6})}{\num{5.0e-9}}} \\
                & = \qty{46}{m/s}.
\end{align*}

\subsection{Calculating Electric Potential}

\subsubsection{Example 23.8}

By Gauss's law and symmetry, the electric field outside the sphere is the same as if all of its charge were concentrated at its centre. So, for $r \ge R$ \[V_\textrm{out} = \ke \frac{q}{r}.\]

There is no charge inside the sphere so the electric potential is equal to that of the surface of the sphere \[V_\textrm{in} = \ke \frac{q}{R}.\]

\subsubsection{Example 23.10}

The line's electric field points radially away from it with magnitude \[E = \frac{1}{2 \pi \epsilon_0} \frac{\lambda}{r}.\]

Setting the electric potential to be $0$ at $r = r_0$ we find that

\begin{align*}
  V & = \int_r^{r_0} \frac{1}{2 \pi \epsilon_0} \frac{\lambda}{r'} \,dr' \\
    & = \frac{\lambda}{2 \pi \epsilon_0} \left[ \ln r' \right]_r^{r_0}   \\
    & = \frac{\lambda}{2 \pi \epsilon_0} \ln{\frac{r_0}{r}}.
\end{align*}

\subsubsection{Example 23.11}

The electric potential is

\begin{align*}
  V & = \int \ke \frac{dq}{r}                  \\
    & = \ke \frac{1}{\sqrt{a^2 + x^2}} \int dq \\
    & = \ke \frac{Q}{\sqrt{a^2 + x^2}}.
\end{align*}

\subsubsection{Example 23.12}

The electric potential is

\begin{align*}
  V & = \int \ke \frac{dq}{r}                                                                  \\
    & = \ke \int_{-a}^a \frac{1}{\sqrt{x^2 + y^2}} \frac{Q}{2 a} \,dy                          \\
    & = \ke \frac{Q}{2 a} \left[ \ln \left( y + \sqrt{x^2 + y^2} \right) \right]_{-a}^a        \\
    & = \ke \frac{Q}{2 a} \ln \left( \frac{\sqrt{a^2 + x^2} + a}{\sqrt{a^2 + x^2} - a} \right)
\end{align*}

\subsection{Potential Gradient}

\subsubsection{23.13}

The electric field is

\begin{align*}
  \mathbf{E} & = -\frac{\partial V}{\partial r} \hat{\mathbf{r}} \\
             & = \ke \frac{q}{r^2} \hat{\mathbf{r}}.
\end{align*}

\subsubsection{Example 23.14}

The electric field is

\begin{align*}
  \mathbf{E} & = -\nabla V                                                                                                                                                        \\
             & = -\left( \frac{\partial V}{\partial x} \hat{\mathbf{i}} + \frac{\partial V}{\partial y} \hat{\mathbf{j}} + \frac{\partial V}{\partial z} \hat{\mathbf{k}} \right) \\
             & = -\frac{\partial}{\partial x} \left( \ke \frac{Q}{\sqrt{x^2 + a^2}} \right) \hat{\mathbf{i}}                                                                      \\
             & = -\ke Q \left( -\frac{1}{2} \frac{2 x}{\left( x^2 + a^2 \right)^{3/2}} \right) \hat{\mathbf{i}}                                                                   \\
             & = \ke \frac{Q x}{\left( x^2 + a^2 \right)^{3/2}} \hat{\mathbf{i}}.
\end{align*}

\subsection{Guided Practice}

\subsubsection{VP23.2.1}

\begin{enumerate}[a)]
  \item The work required to move the electron is equal to the change in the system's potential energy

        \begin{align*}
          W & = -\ke (82 e^2) \left( \frac{1}{r_2} - \frac{1}{r_1} \right)                                                   \\
            & = -(\num{8.988e9}) (82) (\num{-1.60e-19})^2 \left( \frac{1}{\num{5.00e-10}} - \frac{1}{\num{1.00e-10}} \right) \\
            & = \qty{1.51e-16}{J}
        \end{align*}

  \item The electron will be attracted to the lead nucleus and some of its electric potential energy will be converted to kinetic energy

        \begin{align*}
          K_1 + U_1                  & = K_2 + U_2                                                                                               \\
          0 - \ke \frac{82 e^2}{r_1} & = K_2 - \ke \frac{82 e^2}{r_2}                                                                            \\
          \Rightarrow K_2            & = \ke 82 e^2 \left( \frac{1}{r_2} - \frac{1}{r_1} \right)                                                 \\
                                     & = (\num{8.988e9})82(\num{-1.60e-19})^2 \left( \frac{1}{\num{8.00e-12}} - \frac{1}{\num{5.00e-10}} \right) \\
                                     & = \qty{2.32e-15}{J}
        \end{align*}
\end{enumerate}

\subsubsection{VP23.2.4}

\begin{enumerate}[a)]
  \item The potential energy of the charges is equal to the work done to bring in the second charge, so

        \begin{align*}
          W               & = \ke \frac{q_1 q_2}{r}                                                \\
          \Rightarrow q_2 & = \frac{4 \pi \epsilon_0 r W}{q_1}                                     \\
                          & = \frac{4 \pi (\num{8.854e-12})(0.0400)(\num{8.10e-6})}{\num{5.00e-9}} \\
                          & = \qty{7.21e-9}{C}
        \end{align*}

  \item The work required to bring in the third charge is

        \begin{align*}
          W & = \ke q_3 \left( \frac{q_1}{r_1} + \frac{q_2}{r_2} \right)                                                                       \\
            & = (\num{8.988e9}) (\num{2.00e-9}) \left( \frac{\num{5.00e-9}}{0.0300} + \frac{\num{7.21e-9}}{\sqrt{0.0300^2 + 0.0400^2}} \right) \\
            & = \qty{5.59e-6}{J}.
        \end{align*}
\end{enumerate}

\subsubsection{VP23.7.1}

\begin{enumerate}[a)]
  \item The electric field is uniform and both points lie on the $x$-axis so \[V_0 - V_P = d E = (0.0500) (\num{5.00e2}) = \qty{25.0}{V}\]
\end{enumerate}

\subsubsection{VP23.7.2}

\begin{enumerate}[a)]
  \item The electric potential is \[V = 2 \ke \frac{q}{r} = 2 (\num{8.988e9}) \frac{\num{6.0e-9}}{0.030} = \qty{3.6e3}{V}\]

        \setcounter{enumi}{3}
  \item The electric field is 0 at the origin
\end{enumerate}

\subsubsection{VP23.7.4}

\begin{enumerate}[a)]
  \item Let the electric potential at the surface of the sphere be $0$, then the potential difference between the centre of the sphere and the surface is

        \begin{align*}
          V_\textrm{centre} - V_\textrm{surface} & = \int \mathbf{E} \cdot d\mathbf{l}                                 \\
          \Rightarrow V_\textrm{centre}          & = \int_0^R \ke \frac{Q r}{R^3} \,dr                                 \\
                                                 & = \frac{Q}{4 \pi \epsilon_0 R^3} \left[ \frac{1}{2} r^2 \right]_0^R \\
                                                 & = \frac{Q}{8 \pi \epsilon_0 R}
        \end{align*}
\end{enumerate}

\subsubsection{VP23.12.1}

\begin{enumerate}[a)]
  \item By Gauss's law and symmetry, the electric field (and thus the electric potential) are the same as if the sphere's charge were concentrated at its centre. Thus the work required to move a test charge from $5R$ to $3R$ is

        \begin{align*}
          W & = \ke q q_0 \left( \frac{1}{3 R} - \frac{1}{5 R} \right) \\
            & = \ke \frac{2 q q_0}{15 R}                               \\
            & = \frac{q q_0}{30 \pi \epsilon_0 R}
        \end{align*}
\end{enumerate}

\subsubsection{VP23.12.2}

\begin{enumerate}[a)]
  \item The electric field is constant, so the electric potential decreases linearly over the distance between the two plates and thus \[V = \frac{d'}{d} V_{ab} = \frac{3.00}{4.50} 24.0 = \qty{16.0}{V}\]

  \item $U = q V = (\num{2.00e-9}) (16.0) = \qty{3.20e-8}{J}$

  \item It will move towards the lower plate. When it reaches the plate its speed will be

        \begin{align*}
          U             & = K                                              \\
                        & = \frac{1}{2} m v^2                              \\
          \Rightarrow v & = \sqrt{\frac{2 U}{m}}                           \\
                        & = \sqrt{\frac{2 (\num{3.20e-8})}{\num{5.00e-9}}} \\
                        & = \qty{3.58}{m/s}
        \end{align*}
\end{enumerate}

\subsubsection{VP23.12.4}

\begin{enumerate}[a)]
  \item $dq = \frac{Q}{L} \,dx$

  \item The electric potential at the origin is

        \begin{align*}
          V & = \int \ke \frac{dq}{r}                       \\
            & = \ke \int_L^{2 L} \frac{Q}{L x} \,dx         \\
            & = \ke \frac{Q}{L} \left[ \ln x \right]_L^{2L} \\
            & = \ke \frac{Q}{L} \ln 2
        \end{align*}
\end{enumerate}

\subsubsection{Bridging Problem}

The electric potential at $x = L$ due to the rod is

\begin{align*}
  V & = \int \ke \frac{dq}{r}                                                                 \\
    & = \ke \int_{-a}^a \frac{Q}{2 a} \frac{1}{L - x} \,dx                                    \\
    & = \ke \frac{Q}{2 a} \left[ -\ln \left( L - x \right) \right]_{-a}^a                     \\
    & = \ke \frac{Q}{2 a} \left( -\ln \left( L - a \right) + \ln \left( L + a \right) \right) \\
    & = \ke \frac{Q}{2 a} \ln \left( \frac{L + a}{L - a} \right).
\end{align*}

The work required to bring in a charge $q$ from infinity to $x = L$ is

\begin{align*}
  W & = q V                                                        \\
    & = \ke \frac{qQ}{2 a} \ln \left( \frac{L + a}{L - a} \right).
\end{align*}

\subsection{Exercises}

\subsubsection{23.1}

The electric potential energy at the first position is

\begin{align*}
  U & = \ke \frac{q_1 q_2}{r_1}                                        \\
    & = (\num{8.988e9}) \frac{(\num{2.40e-6}) (\num{-4.30e-6})}{0.150} \\
    & = \qty{-0.618}{J}.
\end{align*}

The electric potential energy at the second position is

\begin{align*}
  U & = \ke \frac{q_1 q_2}{r_2}                                                     \\
    & = (\num{8.988e9}) \frac{(\num{2.40e-6}) (\num{-4.30e-6})}{\sqrt{2 (0.250)^2}} \\
    & = \qty{-0.262}{J}.
\end{align*}

The work done by the electric force on $q_2$ is \[W = U_1 - U2 = (\num{-0.618}) - (\num{0.262}) = \qty{-0.356}{J}.\]

\subsubsection{23.3}

The total energy of the system (and thus the work required to assemble it) is

\begin{align*}
  E & = \ke \sum_{i < j} \frac{q_i q_j}{r_{ij}}                     \\
    & = \ke 3 \frac{e^2}{r}                                         \\
    & = (\num{8.988e9}) 3 \frac{(\num{1.60e-19})^2}{\num{2.00e-15}} \\
    & = \qty{3.45e-13}{J}.
\end{align*}

\subsubsection{23.5}

\begin{enumerate}[a)]
  \item When the spheres are $\qty{0.400}{m}$ apart some of the second sphere's initial kinetic energy will have been converted into electric potential energy, slowing it down. Using the principle of conservation of energy we find

        \begin{align*}
          K_1 + U_1                                     & = K_2 + U_2                                                                                      \\
          \frac{1}{2} m v_1^2 + \ke \frac{q_1 q_2}{r_1} & = \frac{1}{2} m v_2^2 + \ke \frac{q_1 q_2}{r_2}                                                  \\
          v_2                                           & = \sqrt{v_1^2 + \frac{q_1 q_2}{2 \pi \epsilon_0 m} \left( \frac{1}{r_1} - \frac{1}{r_2} \right)} \\
                                                        & = \qty{12.5}{m/s}
        \end{align*}

  \item When the sphere's are closest, all of the second sphere's initial kinetic energy will have been converted into electric potential energy. Again, using the principle of conservation of energy we find

        \begin{align*}
          K_1 + U_1                                                         & = U_2                                                        \\
          \frac{1}{2} m v_1^2 + \ke \frac{q_1 q_2}{r_1}                     & = \ke \frac{q_1 q_2}{r_2}                                    \\
          m v_1^2 + \frac{q_1 q_2}{2 \pi \epsilon_0 r_1}                    & = \frac{q_1 q_2}{2 \pi \epsilon_0 r_2}                       \\
          r_2 \left( 2 \pi \epsilon_0 m v_1^2 + \frac{q_1 q_2}{r_1} \right) & = q_1 q_2                                                    \\
          r_2                                                               & = \frac{q_1 q_2 r_1}{2 \pi \epsilon_0 m r_1 v_1^2 + q_1 q_2} \\
                                                                            & = \qty{0.323}{m}
        \end{align*}
\end{enumerate}

\subsubsection{23.7}

The protons will exert the maximum electric force on each other when thye're closest. This will happen when all of their initial kinetic energy has been converted into electric potential energy. Assuming their electric potential energy is $0$ at the start and using the principle of conservation of energy we find that

\begin{align*}
  2 K_1                 & = U_2                                  \\
  2 \frac{1}{2} m v_1^2 & = \ke \frac{e^2}{r_2}                  \\
  r_2                   & = \frac{e^2}{4 \pi \epsilon_0 m v_1^2} \\
                        & = \qty{3.44e-12}{m}
\end{align*}

at which point the electric force between them will be

\begin{align*}
  F & = \ke \frac{e^2}{r_2^2} \\
    & = \qty{1.94e-5}{N}
\end{align*}

\subsubsection{23.9}

\begin{enumerate}[a)]
  \item The protons will reach their maximum speed when all of their initial electric potential energy has been converted into kinetic energy. Using the principle of conservation of energy, the maximum speed they will reach is

        \begin{align*}
          U_1                 & = 2 K_2                                     \\
          \ke \frac{e^2}{r_1} & = 2 \frac{1}{2} m v_2^2                     \\
          v_2                 & = \sqrt{\frac{e^2}{4 \pi \epsilon_0 m r_1}} \\
                              & = \qty{1.36e4}{m/s}
        \end{align*}

  \item The protons will experience their maximum acceleration when they are closest, i.e. at the start. The maximum acceleration will be

        \begin{align*}
          a & = \frac{F}{m}           \\
            & = \ke \frac{e^2}{m r^2} \\
            & = \qty{2.45e17}{m/s^2}
        \end{align*}
\end{enumerate}

\subsubsection{23.11}

The potential difference between two points is equal to the work done by the electric force while moving between them, so

\begin{align*}
  V_a - V_b & = \int_a^b \mathbf{E} \cdot d\mathbf{l}                           \\
            & = \int_{y_a}^{y_b} (\alpha + \frac{\beta}{y^2}) \,dy              \\
            & = \left[ \alpha y - \frac{\beta}{y} \right]_{y_a}^{y_b}           \\
            & = \alpha y_b - \frac{\beta}{y_b} - \alpha y_a + \frac{\beta}{y_a} \\
            & = \qty{89.3}{V}
\end{align*}

\subsubsection{23.13}

The particle has negative charge and thus moves from areas of low to high electric potential. Point $B$ has greater electric potential than point $A$ so the particle has lost electric potential energy. As the electric force is the only force acting on the particle, some of its electric potential energy has been converted into kinetic energy and thus it is moving faster. Its final speed is

\begin{align*}
  K_A + U_A                   & = K_B + U_B                                \\
  \frac{1}{2} m v_A^2 + q V_A & = \frac{1}{2} m v_B^2 + q V_B              \\
  v_B                         & = \sqrt{v_A^2 + \frac{2 q}{m} (V_A - V_B)} \\
                              & = \qty{7.42}{m/s}
\end{align*}

\subsubsection{23.15}

\begin{enumerate}[a)]
  \item $W = \qty{0}{J}$

  \item $W = d E q = \qty{7.50e-4}{J}$

  \item $W = -d \cos \theta E q = \qty{-2.06e-3}{J}$
\end{enumerate}

\subsubsection{23.17}

\begin{enumerate}[a)]
  \item The electric potential at point $a$ is

        \begin{align*}
          V & = \ke \left( \frac{q_1}{r} + \frac{q_2}{r} \right)                                                                       \\
            & = (\num{8.988e9}) \left( \frac{\num{2.00e-6}}{\sqrt{3^2 + 3^2} / 2} - \frac{\num{2.00e-6}}{\sqrt{3^2 + 3^2} / 2} \right) \\
            & = 0
        \end{align*}

  \item The electric potential at $b$ is

        \begin{align*}
          V & = \ke \left( \frac{q_1}{r_1} + \frac{q_2}{r_2} \right)                                                           \\
            & = (\num{8.988e9}) \left( \frac{\num{2.00e-6}}{\sqrt{0.0300^2 + 0.0300^2}} - \frac{\num{2.00e-6}}{0.0300} \right) \\
            & = \qty{-176}{kV}
        \end{align*}

  \item The work done on $q_3$ is \[W = q_3 (V_a - V_b) = \qty{-0.88}{J}\]
\end{enumerate}

\subsubsection{32.19}

\begin{enumerate}[a)]
  \item The potential at point $A$ is \[V = \ke \left( \frac{q_1}{r} + \frac{q_2}{r} \right) = \qty{-737}{V}\]

  \item The potential at point $B$ is \[V = \ke \left( \frac{q_1}{r} + \frac{q_2}{r} \right) = \qty{-704}{V}\]

  \item The work done by the electric field on the charge is \[W = q (V_B - V_A) = \qty{8.25e-8}{J}\]
\end{enumerate}

\subsubsection{23.21}

\begin{enumerate}[a)]
  \item Negative

  \item The electric field is uniform, so

        \begin{align*}
          V_B - V_A       & = (y_B - y_A) E_y             \\
          \Rightarrow E_y & = \frac{V_B - V_A}{y_B - y_A} \\
                          & = \qty{171}{N/C}
        \end{align*}

  \item The electric field only does work when moving along the $y$-axis, so the $x$ coordinate can be ignored and \[V_B - V_C = d E_y = \qty{17.1}{V}\]
\end{enumerate}

\subsubsection{23.23}

\begin{enumerate}[a)]
  \item $b$

  \item $E = \frac{V_B - V_A}{x_B - x_A} = \qty{800}{N/C}$

  \item $W = q E d = \qty{-4.80e-5}{J}$
\end{enumerate}

\subsubsection{23.25}

\begin{enumerate}[a)]
  \item The work done on the sphere is equal to its change in electric potential energy

        \begin{align*}
          W & = U_1 - U_2                                                                                                   \\
            & = 0 - \ke Q \left( \frac{q_1}{r_1} + \frac{q_2}{r_2} \right)                                                  \\
            & = -(\num{8.988e9}) (\num{-0.200e-6}) \left( \frac{\num{6.00e-6}}{0.400} + \frac{\num{6.00e-6}}{0.400} \right) \\
            & = \qty{5.39e-2}{J}
        \end{align*}

  \item By the work-energy theorem, the work done on the sphere equals its change in kinetic energy. It was released from rest, so

        \begin{align*}
          K                 & = W                    \\
          \frac{1}{2} m v^2 & = W                    \\
          v                 & = \sqrt{\frac{2 W}{m}} \\
                            & = \qty{3.00}{m/s}
        \end{align*}
\end{enumerate}

\subsubsection{23.27}

\begin{enumerate}[a)]
  \item

        \begin{enumerate}[i)]
          \item

                \begin{align*}
                  V & = \ke \left( \frac{q_1}{R_1} + \frac{q_2}{R_2} \right)                                        \\
                    & = (\num{8.988e9}) \left( \frac{\num{6.00e-9}}{0.0300} + \frac{\num{-9.00e-9}}{0.0500} \right) \\
                    & = \qty{180}{V}
                \end{align*}

          \item

                \begin{align*}
                  V & = \ke \left( \frac{q_1}{r} + \frac{q_2}{R_2} \right)                                          \\
                    & = (\num{8.988e9}) \left( \frac{\num{6.00e-9}}{0.0400} + \frac{\num{-9.00e-9}}{0.0500} \right) \\
                    & = \qty{-270}{V}
                \end{align*}

          \item

                \begin{align*}
                  V & = \ke \left( \frac{q_1}{r} + \frac{q_2}{R_2} \right)                                          \\
                    & = (\num{8.988e9}) \left( \frac{\num{6.00e-9}}{0.0600} + \frac{\num{-9.00e-9}}{0.0600} \right) \\
                    & = \qty{-450}{V}
                \end{align*}
        \end{enumerate}

  \item The potential difference is

        \begin{align*}
          V_1 - V_2 & = \ke \left( \frac{q_1}{R_1} + \frac{q_2}{R_2} - \frac{q_1}{R_2} - \frac{q_2}{R_2} \right) \\
                    & = \ke q_1 \left( \frac{1}{R_1} - \frac{1}{R_2} \right)                                     \\
                    & = \qty{719}{V}
        \end{align*}

        so the surface of the inner shell has a higher electric potential
\end{enumerate}

\subsubsection{23.29}

\begin{enumerate}[a)]
  \item The electric force pulls the electron towards the ring, causing it to accelerate. It will speed up, move towards the ring, and eventually pass through its centre. Once on the other side the electric force will change direction and pull the electron back towards the ring. It will oscillate forever.

  \item From Example 23.11 we know that the electric potential along the axis of a charged ring is given by \[V = \ke \frac{Q}{\sqrt{x^2 + a^2}}.\]

        The electric potential at the electron's initial position is

        \begin{align*}
          V_1 & = \ke \frac{Q}{\sqrt{x^2 + a^2}}                            \\
              & = (\num{8.988e9}) \frac{\num{24e-9}}{\sqrt{0.3^2 + 0.15^2}} \\
              & = \qty{643}{V}.
        \end{align*}

        The electric potential at the centre of the ring is

        \begin{align*}
          V_2 & = \ke \frac{Q}{a}                          \\
              & = (\num{8.988e9}) \frac{\num{24e-9}}{0.15} \\
              & = \qty{1440}{V}.
        \end{align*}

        By the work-energy principle, all of the work done by the electric force on the electron will be converted to kinetic energy so its speed at the centre of the ring is

        \begin{align*}
          W             & = K                                \\
          e (V_1 - V_2) & = \frac{1}{2} m v^2                \\
          v             & = \sqrt{\frac{2 e (V_1 - V_2)}{m}} \\
                        & = \qty{1.67e7}{m/s}.
        \end{align*}
\end{enumerate}

\subsubsection{23.31}

By Gauss's law both spheres can be treated as point charges. By the principle of conservation of energy, the small sphere will be closest to the large sphere when all of its kinetic energy has been converted into electric potential energy

\begin{align*}
  K_1                 & = U_2                                    \\
  \frac{1}{2} m v_1^2 & = \ke \frac{qQ}{r}                       \\
  v_1                 & = \sqrt{\frac{qQ}{2 \pi \epsilon_0 m r}} \\
                      & = \qty{150}{m/s}.
\end{align*}

\subsubsection{23.33}

\begin{enumerate}[a)]
  \item By Gauss's law the cylinder's electric potential is equivalent to that of an infinite line charge with the same linear charge density \[V = \frac{\lambda}{2 \pi \epsilon_0} \ln \frac{r_0}{r}.\]

        Setting $r_0 = 0.1000$ the potential difference is

        \begin{align*}
          V & = \frac{\num{8.50e-6}}{2 \pi (\num{8.854e-12})} \ln \frac{0.10000}{0.0600} \\
            & = \qty{7.80e4}{V}
        \end{align*}

  \item There is no charge inside the cylinder so it is an equipotential and the potential difference is $0$
\end{enumerate}

\subsubsection{23.35}

The electric potential of the line of charge is \[V = \frac{\lambda}{2 \pi \epsilon_0} \ln \frac{r_0}{r}.\]

Setting $r_0 = 0.0450$ the potential difference between the sphere's initial and final positions is

\begin{align*}
  V_1 - V_2 & = \frac{\lambda}{2 \pi \epsilon_0} \left( \ln \frac{r_0}{r_1} - \ln \frac{r_0}{r_2} \right) \\
            & = \frac{\lambda}{2 \pi \epsilon_0} \ln \frac{r_2}{r_1}                                      \\
            & = \qty{5.92e4}{V}
\end{align*}

By the work-energy theorem, all of the work done by the electric force on the sphere is converted into kinetic energy so

\begin{align*}
  K & = W              \\
    & = q (V_1 - V_2)  \\
    & = \qty{0.474}{J}
\end{align*}

\subsubsection{23.37}

\begin{enumerate}[a)]
  \item $E = V / d = \qty{8e3}{N/C}$

  \item $F = q E = \qty{1.92e-5}{N}$

  \item $W = F d = \qty{8.64e-7}{J}$

  \item $\Delta U = -q \Delta V = \qty{-8.64e-7}{J}$
\end{enumerate}

\subsubsection{23.39}

We can find the total charge of the sphere by rearranging the equation for the magnitude of the electric field at the surface of the sphere

\begin{align*}
  E             & = \ke \frac{q}{R^2}      \\
  \Rightarrow q & = 4 \pi \epsilon_0 E R^2 \\
                & = \qty{-1.69e-8}{C}.
\end{align*}

The sphere is an equipotential so the electric potential inside is equal to that of the surface

\begin{align*}
  V & = \ke \frac{q}{R} \\
    & = \qty{-760}{V}
\end{align*}

\subsubsection{23.41}

\begin{enumerate}[a)]
  \item

        \begin{enumerate}[i)]
          \item $V = \ke q \left( \frac{1}{r_a} - \frac{1}{r_b} \right)$

          \item $V = \ke q \left( \frac{1}{r} - \frac{1}{r_b} \right)$

          \item $V = 0$
        \end{enumerate}

  \item From above, the potential at the surface of the outer sphere is $0$ so the potential difference is equal to the potential at the surface of the inner sphere \[V = \ke q \left( \frac{1}{r_a} - \frac{1}{r_b} \right).\]

  \item The electric field between the spheres has magnitude

        \begin{align*}
          E & = -\frac{\partial V}{\partial r}                                                               \\
            & = -\frac{\partial}{\partial r} \left( \ke q \left( \frac{1}{r} - \frac{1}{r_b} \right) \right) \\
            & = \ke \frac{q}{r^2}                                                                            \\
            & = \frac{V_{ab}}{1/r_a - 1/r_b} \frac{1}{r^2}
        \end{align*}

  \item The electric field outside the larger sphere has magnitude $0$
\end{enumerate}

\subsection{Problems}

\subsubsection{23.47}

\begin{enumerate}[a)]
  \item The potential energy of the system is

        \begin{align*}
          U & = \ke \sum_{i < j} \frac{q_i q_j}{r_{ij}}                                                     \\
            & = \ke \left( \frac{q_1 q_2}{r_{12}} + \frac{q_1 q_3}{r_{13}} + \frac{q_2 q_3}{r_{23}} \right) \\
            & = \qty{-3.60e-7}{J}
        \end{align*}

  \item If the potential energy of the system is $0$ then

        \begin{align*}
          0 & = \ke \left( \frac{q_1 q_2}{r_2} + \frac{q_1 q_3}{r_3} + \frac{q_2 q_3}{r_2 - r_3} \right) \\
            & = (r_2 - r_3) r_3 \frac{q_1 q_2}{r_2} + (r_2 - r_3) q_1 q_3 + r_3 q_2 q_3                  \\
            & = q_1 q_2 r_3 - \frac{q_1 q_2}{r_2} r_3^2 + q_1 q_3 r_2 - q_1 q_3 r_3 + q_2 q_3 r_3        \\
            & = -\frac{q_1 q_2}{r_2} r_3^2 + (q_1 q_2 - q_1 q_3 + q_2 q_3) r_3 + q_1 q_3 r_2.
        \end{align*}

        Let

        \begin{align*}
          a & = -\frac{q_1 q_2}{r_2}                            \\
            & = -\frac{(\num{4.00e-9}) (\num{-3.00e-9})}{0.200} \\
            & = \num{6.00e-17}
        \end{align*}

        \begin{align*}
          b & = q_1 q_2 - q_1 q_3 + q_2 q_3                                                                          \\
            & = (\num{4.00e-9}) (\num{-3.00e-9}) - (\num{4.00e-9})(\num{2.00e-9}) + (\num{-3.00e-9}) (\num{2.00e-9}) \\
            & = \num{-2.60e-17}
        \end{align*}

        \begin{align*}
          c & = q_1 q_3 r_2                             \\
            & = (\num{4.00e-9}) (\num{2.00e-9}) (0.200) \\
            & = \num{1.60e-18}
        \end{align*}

        then

        \begin{align*}
          r_3 & = \frac{-b \pm \sqrt{b^2 - 4 a c}}{2 a}                                                                          \\
              & = \frac{\num{2.60e-17} \pm \sqrt{(\num{-2.60e-17})^2 - 4 (\num{6.00e-17}) (\num{1.60e-18})}}{2 (\num{6.00e-17})} \\
              & = \qty{0.074}{m} \textrm{ or } \qty{0.359}{m}
        \end{align*}

        $r_3 = \qty{0.359}{m}$ doesn't lie between $q_1$ and $q_2$ so it can be discarded and the answer is $r_3 = \qty{0.074}{m}$.
\end{enumerate}

\subsubsection{23.49}

The potential of an infinite line charge is \[V = \frac{\lambda}{2 \pi \epsilon_0} \ln \frac{r_0}{r}.\]

Setting $r_0 = 0.300$, the potential of the line at point $A$ is $V_{A\textrm{,line}} = 0$, at point $B$ is $V_{B\textrm{,line}} = \qty{3.64e4}{V}$, and thus the potential difference is $V_{AB\textrm{,line}} = V_{A\textrm{,line}} - V_{B\textrm{,line}} = \qty{-3.64e4}{V}$.

The electric field of an infinite sheet of charge is directed away from the sheet and has magnitude \[E = \frac{\sigma}{2 \epsilon_0}.\]

The potential difference due to the sheet is equal to the work per unit charge done by its electric field between points $A$ and $B$ so $V_{AB\textrm{,sheet}} = E d = \qty{2.26e5}{V}$.

Thus the total potential difference is $V_{AB} = V_{AB\textrm{,line}} + V_{AB\textrm{,sheet}} = \qty{1.90e5}{V}$ meaning point $A$ is at higher potential.

\subsubsection{23.51}

In order for the alpha particle to reach a distance $\num{2.0e-14}{m}$ from the surface of the gold nucleus its potential energy at that point must equal the energy it gained from the potential difference

\begin{align*}
  U             & = (2 e) V                                           \\
  \Rightarrow V & = \frac{U}{2 e}                                     \\
                & = \ke \frac{79 e}{r}                                \\
                & = (\kev) \frac{79 (\num{1.60e-19})}{\num{2.73e-14}} \\
                & = \qty{4.2e6}{V}.
\end{align*}

\subsubsection{23.55}

\begin{enumerate}[a)]
  \item

        \begin{align*}
          V             & = C x^{4/3}                 \\
          \Rightarrow C & = \frac{V}{x^{4/3}}         \\
                        & = \frac{240}{(0.013)^{4/3}} \\
                        & = \qty{7.85e4}{V/m^{4/3}}
        \end{align*}

  \item
        \begin{align*}
          \mathbf{E} & = -\nabla V                                \\
                     & = -\frac{4}{3} C x^{1/3} \hat{\mathbf{i}}  \\
                     & = -(\num{1.05e5}) x^{1/3} \hat{\mathbf{i}}
        \end{align*}

  \item
        \begin{align*}
          \mathbf{F} & = e \mathbf{E}                                                          \\
                     & = (\num{1.60e-19}) (\num{1.05e5}) (\num{0.0065})^{1/3} \hat{\mathbf{i}} \\
                     & = (\qty{3.14e-15}{N}) \hat{\mathbf{i}}
        \end{align*}
\end{enumerate}

\subsubsection{23.59}

When the sphere is hanging at an angle $\theta$ it experiences net forces

\begin{align*}
  F_x & = q E - T \sin \theta = 0 \\
  F_y & = T \cos \theta - m g = 0
\end{align*}

Rearranging $F_y$ \[T = \frac{m g}{cos \theta}.\]

Substituting this into $F_x$ \[E = \frac{m g \tan \theta}{q}.\]

The electric field between the two plates is constant so their potential difference is \[V = E d = \frac{d m g \tan \theta}{q}.\]

Thus, in order for the sphere to hang at $\ang{30.0}$ the potential difference between the two plates must be

\begin{align*}
  V & = \frac{(0.0500) (0.00150) (9.81) \tan \ang{30.0}}{\num{8.90e-6}} \\
    & = \qty{47.7}{V}.
\end{align*}

\subsubsection{23.67}

Consider a sphere of radius $r$ and charge density \[\rho = \frac{Q}{\frac{4}{3} \pi R^3} = \frac{3 Q}{4 \pi R^3}.\] The charge of the sphere is \[q = \frac{4}{3} \pi r^3 \rho.\] Adding a shell of charge density $\rho$ and thickness $dr$ requires energy

\begin{align*}
  dE & = \int \ke \frac{q \,dq}{r}                                                                                      \\
     & = \int_0^{2 \pi} \int_0^\pi \ke \frac{\frac{4}{3} \pi r^3 \rho r \,d\theta \,r \sin \theta \,d\phi \,dr \rho}{r} \\
     & = \ke \frac{8}{3} \pi^2 r^4 \rho^2 \,dr \int_0^\pi \sin \theta \,d\theta                                         \\
     & = \ke \frac{16}{3} \pi^2 r^4 \rho^2 \,dr
\end{align*}

so assembling a sphere of radius $R$ requires energy

\begin{align*}
  E & = \int dE                                                                                          \\
    & = \int_0^R \ke \frac{16}{3} \pi^2 r^4 \rho^2 \,dr                                                  \\
    & = \ke \frac{16}{3} \pi^2 \left( \frac{3 Q}{4 \pi R^3} \right)^2 \left[ \frac{1}{5} r^5 \right]_0^R \\
    & = \frac{3}{5} \left( \ke \frac{Q^2}{R} \right).
\end{align*}

\subsubsection{23.73}

\begin{enumerate}[a)]
  \item $U = \ke \frac{e^2}{d} + \frac{1}{2} k d^2$

  \item

        \begin{align*}
          \left. \frac{dU}{dd} \right|_{d = d_0} & = 0                           \\
          -\ke \frac{e^2}{d_0^2} + k d_0         & = 0                           \\
          d_0                                    & = \sqrt[3]{\ke \frac{e^2}{k}}
        \end{align*}

  \item

        \begin{align*}
          k & = \ke \frac{e^2}{d_0^3}                                         \\
            & = (\num{8.988e9}) \frac{(\num{1.60e-19})^2}{(\num{1.00e-15})^3} \\
            & = \qty{2.30e17}{N/m}
        \end{align*}

  \item

        \begin{align*}
          U & = \ke \frac{e^2}{d_0} + \frac{1}{2} k d_0^2                                                                  \\
            & = (\num{8.988e9}) \frac{(\num{1.60e-19})^2}{\num{1.00e-15}} + \frac{1}{2} (\num{2.30e17}) (\num{1.00e-15})^2 \\
            & = \qty{3.45e-13}{J}                                                                                          \\
            & = \qty{2.16e6}{eV}
        \end{align*}

  \item

        \begin{align*}
          2 K                                & = U                                            \\
          2 \left( \frac{1}{2} m v^2 \right) & = U                                            \\
          v                                  & = \sqrt{\frac{U}{m}}                           \\
                                             & = \sqrt{\frac{\num{3.45e-13}}{\num{1.67e-27}}} \\
                                             & = \qty{1.44e7}{m/s}
        \end{align*}
\end{enumerate}

\subsubsection{23.79}

\begin{enumerate}[a)]
  \item The oil drop experiences an electric force in the positive $y$ direction and a weight force in the negative $y$ direction. If it is at rest then

        \begin{align*}
          0             & = q E - m g                                       \\
                        & = q \frac{V_{AB}}{d} - \frac{4}{3} \pi r^3 \rho g \\
          \Rightarrow q & = \frac{4 \pi}{3} \frac{\rho r^3 g d}{V_{AB}}
        \end{align*}

  \item At terminal velocity the viscous force equals the drop's weight

        \begin{align*}
          0             & = 6 \pi \eta r v_t - m g                        \\
                        & = 6 \pi \eta r v_t - \frac{4}{3} \pi r^3 \rho g \\
          \Rightarrow r & = \sqrt{\frac{9}{2} \frac{\eta v_t}{\rho g}}
        \end{align*}

        Substituting this into the equation for $q$

        \begin{align*}
          q & = \frac{4 \pi}{3} \frac{\rho g d}{V_{AB}} \left( \sqrt{\frac{9}{2} \frac{\eta v_t}{\rho g}} \right)^3 \\
            & = 18 \pi \frac{d}{V_{AB}} \sqrt{\frac{\eta^3 v_t^3}{2 \rho g}}
        \end{align*}

  \item

        \begin{align*}
          q_1 & = \qty{4.79e-19}{C} \\
          q_2 & = \qty{1.59e-19}{C} \\
          q_3 & = \qty{8.09e-19}{C} \\
          q_4 & = \qty{3.23e-19}{C}
        \end{align*}

  \item

        \begin{align*}
          n_1 & = 3 \\
          n_2 & = 1 \\
          n_3 & = 5 \\
          n_4 & = 2
        \end{align*}

  \item $e \approx \qty{1.60e-19}{C}$
\end{enumerate}

\section{Capacitance and Dielectrics}

\subsection{Capacitors and Capacitance}

\subsubsection{Example 24.1}

\begin{align*}
  C             & = \epsilon_0 \frac{A}{d}                \\
  \Rightarrow A & = \frac{C d}{\epsilon_0}                \\
                & = \frac{(1.0) (0.001)}{\num{8.854e-12}} \\
                & = \qty{1.13e8}{m^2}
\end{align*}

\subsubsection{Example 24.2}

\begin{enumerate}[(a)]
  \item

        \begin{align*}
          C & = \epsilon_0 \frac{A}{d}                                          \\
            & = (\qty{8.854e-12}{F/m}) \frac{\qty{2.00}{m^2}}{\qty{0.00500}{m}} \\
            & = \qty{3.54e-9}{F}
        \end{align*}

  \item

        \begin{align*}
          C             & = \frac{Q}{V}                          \\
          \Rightarrow Q & = C V                                  \\
                        & = (\qty{3.54e-9}{F}) (\qty{10.0e3}{V}) \\
                        & = \qty{3.54e-5}{C}
        \end{align*}

  \item

        \begin{align*}
          E & = \frac{Q}{\epsilon_0 A}                                            \\
            & = \frac{\qty{3.54e-5}{C}}{(\qty{8.854e-12}{F/m}) (\qty{2.00}{m^2})} \\
            & = \qty{2.00e6}{N/C}
        \end{align*}
\end{enumerate}

\subsubsection{Example 24.3}

By Gauss's law and symmetry the electric field between the shells is directed radially outward and has magnitude

\begin{align*}
  \oint \mathbf{E} \cdot d\mathbf{A} & = \frac{Q}{\epsilon_0} \\
  E 4 \pi r^2                        & = \frac{Q}{\epsilon_0} \\
  E                                  & = \ke \frac{Q}{r^2}.
\end{align*}

The potential difference between the shells is

\begin{align*}
  V_a - V_b & = \int_a^b \mathbf{E} \cdot d\mathbf{l}              \\
            & = \int_{r_a}^{r_b} \ke \frac{Q}{r^2} \,dr            \\
            & = \ke Q \left[ -\frac{1}{r} \right]_{r_a}^{r_b}      \\
            & = \ke Q \left( \frac{1}{r_a} - \frac{1}{r_b} \right) \\
            & = \ke Q \frac{r_b - r_a}{r_a r_b}.
\end{align*}

The capacitance is thus

\begin{align*}
  C & = \frac{Q}{V}                                 \\
    & = 4 \pi \epsilon_0 \frac{r_a r_b}{r_b - r_a}.
\end{align*}

\subsubsection{Example 24.4}

By Gauss's law and symmetry, the electric field between the cylinders is directed radially outward and has magnitude

\begin{align*}
  \oint \mathbf{E} \cdot d\mathbf{A} & = \frac{Q}{\epsilon_0}                          \\
  E 2 \pi r L                        & = \frac{\lambda L}{\epsilon_0}                  \\
  E                                  & = \frac{1}{2 \pi \epsilon_0} \frac{\lambda}{r}.
\end{align*}

The potential difference between the cylinders is thus

\begin{align*}
  V_a - V_b & = \int_a^b \mathbf{E} \cdot d\mathbf{l}                               \\
            & = \int_{r_a}^{r_b} \frac{1}{2 \pi \epsilon_0} \frac{\lambda}{r} \,dr  \\
            & = \frac{1}{2 \pi \epsilon_0} \lambda \left[ \ln r \right]_{r_a}^{r_b} \\
            & = \frac{1}{2 \pi \epsilon_0} \lambda \ln \frac{r_b}{r_a}
\end{align*}

and the capacitance per unit length is

\begin{align*}
  C & = \frac{Q}{V}                                                            \\
    & = \frac{\lambda}{\frac{1}{2 \pi \epsilon_0} \lambda \ln \frac{r_b}{r_a}} \\
    & = \frac{2 \pi \epsilon_0}{\ln (r_b / r_a)}.
\end{align*}

\subsection{Capacitors in Series and Parallel}

\subsubsection{Example 24.5}

\begin{enumerate}[(a)]
  \item

        \begin{align*}
          C_\textrm{eq} & = \frac{1}{(1/C_1) + (1/C_2)} = \qty{2.0e-6}{F} \\
          Q_1           & = Q_2 = \qty{3.6e-5}{C}                         \\
          V_1           & = \qty{6.0}{V}                                  \\
          V_2           & = \qty{12}{V}
        \end{align*}

  \item

        \begin{align*}
          C_\textrm{eq} & = C_1 + C_2 = \qty{9.0e-6}{F} \\
          Q_1           & = \qty{1.08e-4}{C}            \\
          Q_2           & = \qty{5.4e-5}{C}             \\
          V_1           & = V_2 = \qty{18}{V}
        \end{align*}
\end{enumerate}

\subsubsection{Example 24.6}

The two capacitors on the right are equivalent to a single capacitor of capacitance \[C = \frac{1}{(1 / (\qty{12}{\mu F})) + (1 / (\qty{6}{\mu F}))} = \qty{4}{\mu F}.\]

The three capacitors in the middle are then equivalent to a single capacitor of capacitance \[C = \qty{3}{\mu F} + \qty{11}{\mu F} + \qty{4}{\mu F} = \qty{18}{\mu F}.\]

Finally the two remaining capacitors are equivalent to a single capacitor of capacitance \[C = \frac{1}{(1 / (\qty{18}{\mu F})) + (1 / (\qty{9}{\mu F}))} = \qty{6}{\mu F}.\]

\subsection{Energy Storage in Capacitors and Electric-Field Energy}

\subsubsection{Example 24.7}

\begin{enumerate}[(a)]
  \item \[Q = C V = (\qty{8.0}{\mu F}) (\qty{120}{V}) = \qty{9.6e-4}{C}\]

  \item \[U = \frac{1}{2} Q V = \frac{1}{2} (\qty{8.0}{\mu F}) (\qty{120}{V}) = \qty{5.8e-2}{J}\]
\end{enumerate}

\subsubsection{Example 24.8}

\begin{enumerate}[(a)]
  \item

        \begin{align*}
          U             & = u V                             \\
                        & = \frac{1}{2} \epsilon_0 E^2 V    \\
          \Rightarrow E & = \sqrt{\frac{2 U}{\epsilon_0 V}} \\
                        & = \qty{4.75e5}{N/C}
        \end{align*}

  \item $100$ times the amount of energy
\end{enumerate}

\subsubsection{Example 24.9}

\begin{enumerate}[(a)]
  \item \[U = \frac{Q^2}{2 C} = \frac{1}{2} Q^2 \ke \frac{r_b - r_a}{r_a r_b}\]

  \item The electric field energy density is \[u = \frac{1}{2} \epsilon_0 E^2 = \frac{1}{2} \epsilon_0 \left( \ke \frac{Q}{r^2} \right)^2 = \frac{Q^2}{32 \pi^2 \epsilon_0 r^4}.\] The electric potential energy stored in the capacitor is thus

        \begin{align*}
          U & = \int u \,dV                                                          \\
            & = \int_{r_a}^{r_b} \frac{Q^2}{32 \pi^2 \epsilon_0 r^4} 4 \pi r^2 \,dr  \\
            & = \frac{Q^2}{8 \pi \epsilon_0} \left[ -\frac{1}{r} \right]_{r_a}^{r_b} \\
            & = \frac{1}{2} Q^2 \ke \frac{r_b - r_a}{r_a r_b}
        \end{align*}
\end{enumerate}

\subsection{Dielectrics}

\subsubsection{Example 24.10}

\begin{enumerate}[(a)]
  \item $C_0 = \epsilon_0 \frac{A}{d} = \qty{1.77e-10}{F}$

  \item $C_0 = \frac{Q}{V} \Rightarrow Q = C_0 V = \qty{5.31e-7}{C}$

  \item $C = \frac{Q}{V'} = \qty{5.31e-10}{F}$

  \item $K = \frac{C}{C_0} = 3.00$

  \item $\epsilon = K \epsilon_0 = \qty{2.66e-11}{C^2 / N m^2}$

  \item $Q_i = Q \left( 1 - \frac{1}{K} \right) = C_0 V \left( 1 - \frac{1}{K} \right) = \qty{3.54e-7}{C}$

  \item $E_0 = \frac{\sigma}{\epsilon_0} = \frac{C_0 V}{A \epsilon_0} = \qty{3.00e5}{N/C}$

  \item $E = \frac{E_0}{K} = \qty{1.00e5}{N/C}$
\end{enumerate}

\subsubsection{Example 24.11}

\begin{align*}
  U_\textrm{before} & = \frac{1}{2} C V^2 = \qty{7.97e-4}{J}              \\
  u_\textrm{before} & = \frac{1}{2} \epsilon_0 E^2 = \qty{3.98e-1}{J/m^2} \\
  U_\textrm{after}  & = \frac{1}{2} C V^2 = \qty{2.66e-4}{J}              \\
  u_\textrm{after}  & = \frac{1}{2} \epsilon E^2 = \qty{1.33e-1}{J/m^3}
\end{align*}

\subsection{Gauss's Law in Dielectrics}

\subsubsection{Example 24.12}

By Gauss's law and symmetry, the electric field between the shells is

\begin{align*}
  \oint K \mathbf{E} \cdot d\mathbf{A} & = \frac{Q}{\epsilon_0}          \\
  K E 4 \pi r^2                        & = \frac{Q}{\epsilon_0}          \\
  E                                    & = \frac{Q}{4 \pi \epsilon r^2}.
\end{align*}

The potential difference between the shells is

\begin{align*}
  V & = \int_a^b \mathbf{E} \cdot d\mathbf{l}                                 \\
    & = \int_{r_a}^{r_b} \frac{Q}{4 \pi \epsilon r^2} \,dr                    \\
    & = \frac{Q}{4 \pi \epsilon} \left( \frac{1}{r_a} - \frac{1}{r_b} \right) \\
    & = \frac{Q (r_b - r_a)}{4 \pi \epsilon r_a r_b}
\end{align*}

and thus the capacitance is

\begin{align*}
  C & = \frac{Q}{V}                              \\
    & = \frac{4 \pi \epsilon r_a r_b}{r_b - r_a} \\
    & = K C_0
\end{align*}

\subsection{Guided Practice}

\subsubsection{VP24.4.1}

\begin{enumerate}[(a)]
  \item The capacitance is given by \[C = \epsilon_0 \frac{A}{d} = \qty{6.58e-9}{F}\]

  \item The potential difference is given by \[C = \frac{Q}{V} \Rightarrow V = \frac{Q}{C} = \frac{\sigma A}{C} = \qty{5.85e3}{V}\]
\end{enumerate}

\subsubsection{VP24.4.4}

\begin{enumerate}[(a)]
  \item

        \begin{align*}
          \frac{C}{L}     & = \frac{2 \pi \epsilon_0}{\ln (r_b / r_a)} \\
          \frac{r_b}{r_a} & = e^{\frac{2 \pi \epsilon_0}{C / L}}       \\
                          & = 2.24
        \end{align*}

  \item

        \begin{align*}
          \frac{C}{L} & = \frac{Q}{L V}       \\
          V           & = \frac{Q / L}{C / L} \\
                      & = \qty{125}{V}
        \end{align*}
\end{enumerate}

\subsubsection{VP24.9.1}

\begin{enumerate}[(a)]
  \item \[C_\textrm{eq} = \frac{1}{(1/C_1) + (1/C_2) + (1/C_3)} = \qty{0.625}{\mu F}\]

  \item \[C_\textrm{eq} = C_1 + C_2 + C_2 = \qty{8.50}{\mu F}\]

  \item

        \begin{align*}
          C_{12}        & = C_1 + C_2 = \qty{3.50}{\mu F}                          \\
          C_\textrm{eq} & = \frac{1}{(1 / C_{12}) + (1 / C_3)} = \qty{2.06}{\mu F}
        \end{align*}

  \item
        \begin{align*}
          C_{12}        & = \frac{1}{(1 / C_1) + (1 / C_2)} = \qty{0.714}{\mu F} \\
          C_\textrm{eq} & = C_{12} + C_3 = \qty{5.71}{\mu F}
        \end{align*}
\end{enumerate}

\subsubsection{VP24.9.4}

\begin{enumerate}[(a)]
  \item $\sigma = Q / A = \qty{5.10e-6}{C/m^2}$

  \item $E = \sigma / \epsilon_0 = \qty{5.76e5}{N/C}$

  \item $u = \frac{1}{2} \epsilon_0 E^2 = \qty{1.47}{J/m^3}$

  \item $U = u V = u A d = \qty{5.04e-3}{J}$
\end{enumerate}

\subsubsection{VP24.11.1}

\begin{enumerate}[(a)]
  \item $C = \epsilon_0 A / d = \qty{1.36}{n F}$

  \item $C = Q / V \Rightarrow Q = C V = \qty{5.44}{\mu C}$

  \item $K = V_0 / V = 1.60$

  \item $C = C_0 K = \qty{2.18}{n F}$

  \item $Q_i = Q (1 - 1 / K) = \qty{2.04e-6}{C}$
\end{enumerate}

\subsubsection{24.11.4}

\begin{enumerate}[(a)]
  \item $U = \frac{1}{2} C V^2 = \qty{6.30}{J}$

  \item $U = \frac{1}{2} K C V^2 = \qty{18.0}{J}$
\end{enumerate}

\subsection{Bridging Problem}

\begin{enumerate}[(a)]
  \item The electric field inside a conductor is $0$ so the energy density is also $0$

  \item By Gauss's law and symmetry the electric field at a distance $r$ from the centre of the sphere points radially outwards and has magnitude \[E = \ke \frac{Q}{r^2}\] so the energy density is \[u = \frac{1}{2} \epsilon_0 E^2 = \frac{Q^2}{32 \pi^2 \epsilon_0 r^4}\]

  \item The electric field energy of a shell of thickness $dr$ is \[dU = 4 \pi r^2 u \,dr = \frac{Q^2}{8 \pi \epsilon_0 r^2} \,dr\] so the electric field energy of all space is

        \begin{align*}
          U & = \int dU                                                           \\
            & = \int_R^\infty \frac{Q^2}{8 \pi \epsilon_0 r^2} \,dr               \\
            & = \frac{Q^2}{8 \pi \epsilon_0} \left[ -\frac{1}{r} \right]_R^\infty \\
            & = \frac{Q^2}{8 \pi \epsilon_0 R}
        \end{align*}

  \item $Q^2 / 8 \pi \epsilon_0 R$

  \item $U = Q^2 / 2C \Rightarrow C = Q^2 / 2 U = 4 \pi \epsilon_0 R$
\end{enumerate}

\subsection{Exercises}

\subsubsection{24.1}

\begin{enumerate}[(a)]
  \item $V = E d = \qty{10}{kV}$

  \item $C = Q / V = \epsilon_0 A / d \Rightarrow A = d Q / \epsilon_0 V = \qty{2.26e-3}{m^3}$

  \item $C = Q / V = \qty{8.00}{pF}$
\end{enumerate}

\subsubsection{24.11}

\begin{enumerate}[(a)]
  \item $C = Q / V = \qty{15.0}{pF}$

  \item

        \begin{align*}
          C                             & = 4 \pi \epsilon_0 \frac{r_a r_b}{r_b - r_a} \\
          \frac{r_b - r_a}{r_a r_b}     & = \frac{4 \pi \epsilon_0}{C}                 \\
          \frac{1}{r_a} - \frac{1}{r_b} & = \frac{4 \pi \epsilon_0}{C}                 \\
          \frac{1}{r_a}                 & = \frac{4 \pi \epsilon_0}{C} + \frac{1}{r_b} \\
                                        & = \frac{4 \pi \epsilon_0 r_b + C}{C r_b}     \\
          r_a                           & = \frac{C r_b}{4 \pi \epsilon_0 r_b + C}     \\
                                        & = \qty{3.08}{cm}
        \end{align*}

  \item

        \begin{align*}
          E & = \frac{\sigma}{\epsilon_0}        \\
            & = \frac{Q}{4 \pi r_a^2 \epsilon_0} \\
            & = \qty{3.13e4}{N/C}
        \end{align*}
\end{enumerate}

\subsubsection{24.13}

\begin{align*}
  V_2                   & = k V                 \\
                        & = k (V_1 + V_2)       \\
  V_2 (1 - k)           & = k V_1               \\
  \frac{Q}{C_2} (1 - k) & = k \frac{Q}{C_1}     \\
  C_1                   & = \frac{C_2 k}{1 - k} \\
                        & = \qty{5.57}{\mu F}
\end{align*}

\subsubsection{24.15}

\begin{enumerate}[(a)]
  \item In series

  \item $5000$ cells
\end{enumerate}

\subsubsection{24.21}

\begin{enumerate}[(a)]
  \item The equivalent capacitance for the three capacitors in the middle is \[C_\text{middle} = \frac{1}{(1 / \qty{18.0}{nF}) + (1 / \qty{30.0}{nF}) + (1 / \qty{10.0}{nF})} = \qty{5.29}{nF}.\]

        The equivalent capacitance for the remaining three capacitors is \[C_\textrm{equiv} = \qty{7.5}{nF} + \qty{5.29}{nF} + \qty{6.5}{nF} = \qty{19.3}{nF}.\]

  \item $C = Q / V \Rightarrow Q = C V = \qty{483}{nC}$

  \item $C = Q / V \Rightarrow Q = C V = \qty{163}{nC}$

  \item $\qty{25}{V}$
\end{enumerate}

\subsubsection{24.23}

\[C = \epsilon_0 \frac{A}{d} \Rightarrow A = \frac{C d}{\epsilon_0} = \qty{3.28e3}{m^2}\]

\[u = \frac{\frac{1}{2} C V^2}{A d} = \qty{2.83e-2}{J/m^3}\]

\subsubsection{24.31}

\begin{enumerate}[(a)]
  \item

        \begin{align*}
          U_\textrm{before} & = \frac{1}{2} C V^2 = \qty{3.60}{mJ}   \\
          U_\textrm{after}  & = \frac{1}{2} K C V^2 = \qty{13.5}{mJ}
        \end{align*}

  \item The energy increased by $\qty{9.9}{mJ}$
\end{enumerate}

\subsubsection{24.33}

\begin{enumerate}[(a)]
  \item \[E = E_0 / K \Rightarrow K = E_0 / E = 1.28\] and \[\sigma_i = \sigma (1 - 1 / K) = \epsilon_0 E_0 (1 - 1 / K) = \qty{6.20e-7}{C/m^2}\]

  \item $K = 1.28$
\end{enumerate}

\subsubsection{24.35}

\begin{align*}
  Q & = C_\textrm{equiv} V                                                                      \\
    & = \frac{1}{\frac{1}{K C_1} + \frac{1}{C_2}} V                                             \\
    & = \frac{K C_1 C_2}{K C_1 + C_2} V                                                         \\
    & = \frac{K \frac{Q_0}{V / 2} \frac{Q_0}{V / 2}}{K \frac{Q_0}{V / 2} + \frac{Q_0}{V / 2}} V \\
    & = \frac{K \frac{Q_0}{V / 2}}{K + 1} V                                                     \\
    & = \frac{2 K Q_0}{V (K + 1)} V                                                             \\
    & = \frac{2 K Q_0}{K + 1}
\end{align*}

The charge on $C_1$ increases

\subsubsection{24.37}

At maximum potential difference the charge on the capacitor will be \[Q = C V = \qty{6.88}{\mu C}.\]

The surface charge density will be $\sigma = Q / A$ and the magnitude of the electric field will be $E = \sigma / \epsilon$. If $E$ is to be less than the dielectric strength of the material then

\begin{align*}
  E_m           & >= E                          \\
                & >= \frac{\sigma}{\epsilon}    \\
                & >= \frac{Q}{A K \epsilon_0}   \\
  \Rightarrow A & >= \frac{Q}{E_m K \epsilon_0} \\
                & >= \qty{1.35e-2}{m^2}
\end{align*}

\subsubsection{24.39}

\begin{enumerate}[(a)]
  \item

        \begin{align*}
          Q_\textrm{after} - Q_\textrm{before} & = C_\textrm{after} V - C_\textrm{before} V    \\
                                               & = (K C_\textrm{before} - C_\textrm{before}) V \\
                                               & = (K - 1) C_\textrm{before} V                 \\
                                               & = \qty{6.3}{\mu C}
        \end{align*}

  \item

        \begin{align*}
          Q_i & = Q (1 - 1 / K)     \\
              & = K C V (1 - 1 / K) \\
              & = \qty{6.3}{\mu C}
        \end{align*}

  \item No change
\end{enumerate}

\subsubsection{24.41}

\begin{enumerate}[(a)]
  \item $U = C V^2 / 2 \Rightarrow V = \sqrt{2 U / C} = \qty{10.1}{V}$

  \item $U = K U_0 \Rightarrow K = U / U_0 = 2.25$
\end{enumerate}

\subsubsection{24.43}

\begin{enumerate}[(a)]
  \item

        \begin{align*}
          \oint K \mathbf{E} \cdot d\mathbf{A} & = \frac{Q_\textrm{encl-free}}{\epsilon_0} \\
          K E A                                & = \frac{\sigma A}{\epsilon_0}             \\
          E                                    & = \frac{Q}{A K \epsilon_0}
        \end{align*}

  \item \[V = E d = \frac{Q d}{A K \epsilon_0}\]

  \item \[C = \frac{Q}{V} = K \frac{A \epsilon_0}{d} = K C_0\]
\end{enumerate}

\subsection{Problems}

\subsubsection{24.45}

\begin{enumerate}[(a)]
  \item At 100\% efficiency the energy required is \[E = P t = \qty{400}{J}\] so at 95\% efficiency the energy required is \[E / 0.95 = \qty{421}{J}\]

  \item

        \begin{align*}
          U                 & = \frac{E}{0.95}               \\
          \frac{1}{2} C V^2 & = \frac{E}{0.95}               \\
          C                 & = \frac{E}{0.95} \frac{2}{V^2} \\
                            & = \qty{5.39e-2}{F}
        \end{align*}
\end{enumerate}

\subsubsection{24.47}

\begin{enumerate}[(a)]
  \item $C = \epsilon_0 A / d = \qty{0.531}{pF}$

  \item

        \begin{align*}
          C - C_0                                                    & = \Delta C                                             \\
          \epsilon_0 \frac{A}{d + \Delta d} - \epsilon_0 \frac{A}{d} & = \Delta C                                             \\
          \frac{A}{d + \Delta d}                                     & = \frac{\Delta C}{\epsilon_0} + \frac{A}{d}            \\
                                                                     & = \frac{\epsilon_0 A + \Delta C d}{\epsilon_0 d}       \\
          \Delta d                                                   & = \frac{\epsilon_0 A d}{\epsilon_0 A + \Delta C d} - d \\
                                                                     & = \qty{-0.224}{mm}
        \end{align*}
\end{enumerate}

\subsubsection{24.49}

\begin{enumerate}[(a)]
  \item $Q = C V = \qty{16.0}{mC}$

  \item The capacitors are connected in parallel so the initial charge will remaing and their potential differences will be equal. The equivalent capacitance is \[C = C_1 + C_2 = \qty{30}{\mu F}\] so the potential difference across both capacitors is \[V = Q / C = \qty{533}{V}\]

  \item $U = C V^2 / 2 = \qty{4.26}{J}$

  \item $\Delta U = U_0 - U = \qty{2.14}{J}$
\end{enumerate}

\subsubsection{24.55}

The potential difference across $C_1$ is \[V_1 = Q_1 / C_1 = \qty{50}{V}.\] It is in parallel with $C_2$ so \[Q_{12} = Q_1 + Q_2\] and \[V_2 = V_1.\]

The equivalent capacitor for $C_1$ and $C_2$ is in series with $C_3$ so \[Q_{12} = Q_3\] meaning \[Q_2 = Q_{12} - Q_1 = Q_3 - Q_1 = \qty{300}{\mu C}\] and thus \[C_2 = Q_2 / V_2 = \qty{6.00}{\mu F}.\]

Finally, \[V = V_1 + V_3 \Rightarrow V_3 = V - V_1\] so \[C_3 = Q_3 / V_3 = Q_3 / (V - V_1) = \qty{4.50}{\mu C}.\]

\subsubsection{24.57}

\begin{enumerate}[(a)]
  \item Because the capacitors are connected in series they all have the same charge which is equal to that of the equivalent capacitor. Its capacitance is \[C = \frac{1}{(1 / C_1) + (1 / C_2) + (1 / C_3)} = \qty{2.10}{\mu F}\] so its charge is \[Q = C V = \qty{76}{\mu C}\]

  \item $U = C V^2 / 2 = \qty{1.4}{mJ}$

  \item The capacitance of the equivalent capacitor is \[C = C_1 + C_2 + C_3 = \qty{21}{\mu F}\] and its charge is \[Q = Q_1 + Q_2 + Q_3 = 3 Q_0 = \qty{0.23}{mJ}\] so its potential difference (and that of each individual capacitor) is \[V = Q / C = \qty{11}{V}\]

  \item $U = C V^2 / 2 = \qty{1.3}{mJ}$
\end{enumerate}

\subsubsection{24.63}

\begin{enumerate}[(a)]
  \item The electric field of an infinite charged sheet is \[E = \frac{\sigma}{2 \epsilon_0} = \frac{Q}{2 A \epsilon_0}.\] When the dielectric is present this becomes \[E = \frac{Q}{2 A \epsilon} = \frac{Q}{2 A K \epsilon_0}\] and thus the other sheet experiences an attractive force of \[F = \frac{Q^2}{2 A K \epsilon_0}\]

  \item

        \begin{align*}
          d & = d_0 - \Delta d                                     \\
            & = d_0 - \frac{F}{A} \frac{d_0}{Y}                    \\
            & = d_0 - \frac{Q^2}{2 A^2 K \epsilon_0} \frac{d_0}{Y} \\
            & = d_0 - \frac{d_0}{2 A^2 K \epsilon_0 Y} Q^2
        \end{align*}

  \item \[s = \frac{d_0}{2 A^2 K \epsilon_0 Y} = \qty{7.53e7}{m/C^2}\]

  \item

        \begin{align*}
          V & = \frac{Q}{C}                            \\
            & = \frac{Q d}{A \epsilon}                 \\
            & = \frac{Q (d_0 - s Q^2)}{A K \epsilon_0} \\
            & = \qty{95.7}{kV}
        \end{align*}

  \item

        \begin{align*}
          V' & = \frac{2 Q (d_0 - s (2 Q)^2)}{A K \epsilon_0} \\
             & = \qty{133}{kV}
        \end{align*}
\end{enumerate}

\subsubsection{24.65}

\begin{enumerate}[(a)]
  \item The capacitance is \[C = \epsilon_0 \frac{A}{d} = \epsilon_0 \frac{\pi r^2}{d}\] so the charge on each plate is \[Q = C V = \epsilon_0 \frac{\pi r^2}{d} V\] and the tension in the rope is

        \begin{align*}
          T & = \frac{Q^2}{2 A \epsilon_0}                          \\
            & = \frac{(\epsilon_0 \frac{A}{d} V)^2}{2 A \epsilon_0} \\
            & = \frac{\epsilon_0 A V^2}{2 d^2}                      \\
            & = \frac{\epsilon_0 \pi r^2 V^2}{2 d^2}
        \end{align*}

  \item

        \begin{align*}
          W & = \int_d^{2 d} \frac{\pi \epsilon_0 r^2 V^2}{2 z^2} \,dz              \\
            & = \frac{\pi \epsilon_0 r^2 V^2}{2} \left[ -\frac{1}{z} \right]_d^{2d} \\
            & = \frac{\pi \epsilon_0 r^2 V^2}{4 d}
        \end{align*}

  \item The energy in the electric field is equal to the energy in the capacitor so \[E_\textrm{before} = \frac{1}{2} C V^2 = \frac{\pi \epsilon_0 r^2 V^2}{2 d}\]

  \item \[E_\textrm{after} = \frac{1}{2} C V^2 = \frac{\pi \epsilon_0 r^2 V^2}{4 d}\]

  \item \[\Delta E = E_\textrm{after} - E_\textrm{before} = -\frac{\pi \epsilon_0 r^2 V^2}{4 d}\]
\end{enumerate}

\subsubsection{24.71}

\begin{enumerate}[(a)]
  \item If the dielectric constant at point $x$ is \[K(x) = 1 + (K - 1) \frac{x}{d} = \frac{d + (K - 1) x}{d}\] then the reciprocal of the capacitance is

        \begin{align*}
          \frac{1}{C} & = \int_0^d \frac{dx}{K(x) \epsilon_0 A}                                       \\
                      & = \int_0^d \frac{d}{(d + (K - 1) x) \epsilon_0 A} \,dx                        \\
                      & = \frac{d}{\epsilon_0 A} \int_0^d \frac{1}{d + (K - 1) x} \,dx                \\
                      & = \frac{d}{\epsilon_0 A} \left[ \frac{\ln (d + (K - 1) x)}{K - 1} \right]_0^d \\
                      & = \frac{d}{\epsilon_0 A (K - 1)} (\ln (d + (K - 1) d) - \ln (d))              \\
                      & = \frac{d \ln K}{\epsilon_0 A (K - 1)}
        \end{align*}

        and the capacitance is \[C = \frac{K - 1}{\ln K} \epsilon_0 \frac{A}{d}\]

  \item

        \begin{align*}
          \lim_{K \rightarrow 1} C & = \lim_{K \rightarrow 1} \frac{K - 1}{\ln K} \epsilon_0 \frac{A}{d}   \\
                                   & = \epsilon_0 \frac{A}{d} \lim_{K \rightarrow 1} \frac{1}{\frac{1}{K}} \\
                                   & = \epsilon_0 \frac{A}{d}
        \end{align*}
\end{enumerate}

\section{Current, Resistance, and Electromotive \\ Force}

\subsection{Current}

\subsubsection{Example 25.1}

\begin{enumerate}[(a)]
  \item $J = I / A = \qty{2.04e6}{A/m^2}$

  \item $v_d = J / n |q| = \qty{0.15}{mm/s}$
\end{enumerate}

\setcounter{subsection}{2}
\subsection{Resistance}

\subsubsection{Example 25.2}

\begin{enumerate}[(a)]
  \item $E = \rho J = \rho \frac{I}{A} = \qty{3.52e-2}{N/C}$

  \item $V = E d = \qty{1.67}{V}$

  \item $R = \rho L / A = \qty{1.05}{\Omega}$
\end{enumerate}

\subsubsection{25.3}

\begin{align*}
  R(T)   & = R_0 [1 + \alpha (T - T_0)] \\
  R(0)   & = \qty{0.967}{\Omega}        \\
  R(100) & = \qty{1.38}{\Omega}
\end{align*}

\subsection{Electromotive Force and Circuits}

\subsubsection{Example 25.5}

\begin{align*}
  I R & = \mathcal{E} - I r         \\
  I   & = \frac{\mathcal{E}}{R + r} \\
      & = \qty{2}{A}
\end{align*}

\[V_{ab} = \mathcal{E} - I r = \qty{8}{V}\]

\subsubsection{Example 25.7}

\[\mathcal{E} = I r \Rightarrow I = \frac{\mathcal{E}}{r} = \qty{6}{A}\]

\[V = \mathcal{E} - I r = \mathcal{E} - \mathcal{E} = 0\]

\subsection{Energy and Power in Electric Circuits}

\subsubsection{Example 25.8}

Chemical energy is converted to electrical energy at a rate of \[\mathcal{E} I = \qty{24}{W}\]

Energy is dissipated in the battery due to internal resistance at a rate of \[I^2 r = \qty{8}{W}\]

Energy is dissipated in the resistor at a rate of \[I^2 R = \qty{16}{W}\]

The battery's net power output is \[\mathcal{E} I - I^2 r = \qty{16}{W}\]

\subsubsection{Example 25.9}

The current changes to \[I = \frac{\mathcal{E}}{r + R} = \qty{1.2}{A}\] so energy is dissipated in the resistor at a rate of \[P = I^2 R = \qty{12}{W}\]

\subsubsection{Example 25.10}

Chemical energy is converted to electrical energy at a rate of \[\mathcal{E} I = \qty{72}{W}\]

Energy is dissipated in the battery due to internal resistance at a rate of \[I^2 r = \qty{72}{W}\]

The net power output of the battery is \[\mathcal{E} I - I^2 r = \qty{0}{W}\]

\subsection{Theory of Metallic Conduction}

\subsubsection{Example 25.11}

\begin{align*}
  \rho & = \frac{m}{n e^2 \tau} \\
  \tau & = \frac{m}{n e^2 \rho} \\
       & = \qty{2.43e-14}{s}
\end{align*}

\subsection{Guided Practice}

\subsubsection{VP25.3.1}

\begin{enumerate}[(a)]
  \item $E = V / d = \qty{4.14e-2}{N/C}$

  \item $R = V / I = \qty{1.21}{\Omega}$

  \item $A = \rho L / R = \qty{7.95e-7}{m^2} = \qty{0.795}{mm^2}$
\end{enumerate}

\subsubsection{VP25.3.2}

\begin{enumerate}[(a)]
  \item $R = V / I = E L / I = \qty{0.596}{\Omega}$

  \item $\rho = A R / L = \qty{2.44e-8}{\Omega m}$
\end{enumerate}

\subsubsection{VP25.3.3}

\begin{enumerate}[(a)]
  \item \[\frac{R_2}{R_1} = \frac{\rho \frac{2L}{\pi (2 r)^2}}{\rho \frac{L}{\pi r^2}} = \frac{1}{2}\]

  \item \[\frac{I_2}{I_1} = \frac{\frac{V}{R_2}}{\frac{V}{R_1}} = \frac{R_1}{R_2} = 2\]

  \item \[\frac{J_2}{J_1} = \frac{\frac{I_2}{\pi (2 r)^2}}{\frac{I_1}{\pi r^2}} = \frac{1}{4} \frac{I_2}{I_1} = \frac{1}{2}\]

  \item \[\frac{E_2}{E_1} = \frac{\frac{V}{2 L}}{\frac{V}{L}} = \frac{1}{2}\]
\end{enumerate}

\subsubsection{VP25.3.4}

\begin{enumerate}[(a)]
  \item

        \begin{align*}
          R_{20} & = \frac{V}{I_{20}} = \qty{0.625}{\Omega} \\
          R_{80} & = \frac{V}{I_{80}} = \qty{0.750}{\Omega}
        \end{align*}

  \item

        \begin{align*}
          R      & = R_0 [1 + \alpha (T - T_0)]         \\
          \alpha & = \frac{(R / R_0) - 1}{T - T_0}      \\
                 & = \qty{3.33e-3}{\degreeCelsius^{-1}}
        \end{align*}
\end{enumerate}

\subsubsection{VP25.5.1}

\begin{enumerate}[(a)]
  \item $\mathcal{E} - I r = I R \Rightarrow I = \frac{\mathcal{E}}{r + R} = \qty{0.549}{A}$

  \item $V = I R = \qty{8.40}{V}$

  \item $V = \mathcal{E} - I r = \qty{8.40}{V}$
\end{enumerate}

\subsubsection{VP25.5.2}

\begin{enumerate}[(a)]
  \item $V = I R = \qty{12}{V}$

  \item $V = \mathcal{E} - I r \Rightarrow r = (\mathcal{E} - V) / I = \qty{1.25}{\Omega}$
\end{enumerate}

\subsubsection{VP25.5.3}

\begin{enumerate}[(a)]
  \item $I = V / R = \qty{0.0650}{A}$

  \item $V = \mathcal{E} - I r \Rightarrow r = (\mathcal{E} - V) / I = \qty{3.08}{\Omega}$
\end{enumerate}

\subsubsection{VP25.5.4}

\begin{enumerate}[(a)]
  \item $\mathcal{E} - I r = I R \Rightarrow R = (\mathcal{E} / I) - r = \qty{0.59}{\Omega}$

  \item $V = I R = \qty{1.18}{V}$
\end{enumerate}

\subsubsection{VP25.9.1}

\begin{enumerate}[(a)]
  \item $\mathcal{E} - I r = I R \Rightarrow I = \mathcal{E} / (r + R) = \qty{1.24}{A}$

  \item $P = \mathcal{E} I = \qty{29.8}{W}$

  \item $P = I^2 r = \qty{2.00}{W}$

  \item $P = (\mathcal{E} - I r) I = \qty{27.8}{W}$

  \item $P = V I = I^2 R = \qty{27.8}{W}$
\end{enumerate}

\subsubsection{VP25.9.2}

\begin{enumerate}[(a)]
  \item $P = V I = I^2 R \Rightarrow I = \sqrt{P / R} = \qty{0.738}{A}$

  \item $P = \qty{6.54}{W}$

  \item $P = V I \Rightarrow V = P / I = \qty{8.86}{V}$

  \item $V = \mathcal{E} - I r \Rightarrow r = (\mathcal{E} - V) / I = \qty{0.190}{\Omega}$
\end{enumerate}

\subsubsection{VP25.9.3}

\begin{enumerate}[(a)]
  \item $V = I R = \qty{11.5}{V}$

  \item $V = \mathcal{E} - I r \Rightarrow r = (\mathcal{E} - V) / I = \qty{0.694}{\Omega}$

  \item $P = V I = I^2 R = \qty{8.29}{W}$

  \item $P = \mathcal{E} I = \qty{8.64}{W}$

  \item $P = I^2 r = \qty{0.360}{W}$
\end{enumerate}

\subsubsection{VP25.9.4}

\begin{enumerate}[(a)]
  \item \[\frac{I_2}{I_1} = \frac{\mathcal{E} / (r + 2R)}{\mathcal{E} / (r + R)} = \frac{r + R}{r + 2 R}\]

  \item \[\frac{P_2}{P_1} = \frac{I_2^2 2 R}{I_1^2 R} = \frac{2 (r + R)^2}{(r + 2 R)^2}\]

  \item The current is greater in $R_1$, more power is dissipated in $R_1$
\end{enumerate}

\subsection{Bridging Problem}

\begin{enumerate}[(a)]
  \item The toaster's initial resistance is \[R_0 = \frac{V}{I_0} = \qty{88.9}{\Omega}\] and its final resistance is \[R = \frac{V}{I} = \qty{97.6}{\Omega}\] so it final temperature is

        \begin{align*}
          R             & = R_0 [1 + \alpha (T - T_0)]       \\
          \Rightarrow T & = T_0 + \frac{R - R_0}{\alpha R_0} \\
                        & = \qty{237}{\degreeCelsius}
        \end{align*}

  \item Assuming all of the toaster's resistance comes from the heating element

        \begin{align*}
          P_0 & = V I_0 = \qty{162}{W} \\
          P   & = V I = \qty{148}{W}
        \end{align*}
\end{enumerate}

\subsection{Exercises}

\subsubsection{25.1}

\[Q = I t = \qty{1}{C}\]

\subsubsection{25.3}

\begin{enumerate}[(a)]
  \item $\num{3.12e19}$

  \item $J = I / A = \qty{1.51e6}{A/m^2}$

  \item $J = n q v_d \Rightarrow v_d = J / n q = \qty{0.111}{mm/s}$
\end{enumerate}

\subsubsection{25.5}

\begin{enumerate}[(a)]
  \item

        \begin{align*}
          J              & = n q v_d            \\
          \frac{E}{\rho} & = n q v_d            \\
          v_d            & = \frac{E}{\rho n q} \\
                         & = \qty{0.26}{mm/s}
        \end{align*}

  \item $V = E d = \qty{0.012}{V}$
\end{enumerate}

\subsubsection{25.7}

\begin{enumerate}[(a)]
  \item $Q = \int I \,dt = \int_0^{8.0} (55 - 0.65 t^2) \,dt = [55t - \frac{0.65}{3} t^3]_0^{8.0} = \qty{330}{C}$

  \item $I = Q / t = \qty{41}{A}$
\end{enumerate}

\subsubsection{25.9}

$I = Q / t = n q / t = \qty{9.0}{\mu A}$

\subsubsection{25.11}

The resistance of portion of the wire between the battery's terminals is \[R = \rho \frac{d}{A} = \rho \frac{d}{\pi r^2}\] so the current is \[I = \frac{\mathcal{E}}{R} = \frac{\mathcal{E} \pi r^2}{\rho} \frac{1}{d}.\]

If the slope of the line is $\qty{600}{A m}$ then

\begin{align*}
  \frac{\mathcal{E} \pi r^2}{\rho} & = 600                             \\
  \rho                             & = \frac{\mathcal{E} \pi r^2}{600} \\
                                   & = \qty{4.02e-8}{\Omega.m}
\end{align*}

\subsubsection{25.13}

\begin{enumerate}[(a)]
  \item The resistance of the rod is \[R = \frac{V}{I} = \qty{0.811}{\Omega}\] so the resistivity is \[\rho = R \frac{A}{L} = \qty{1.06e-5}{\Omega.m}\]

  \item \[R = R_0 [1 + \alpha (T - T_0)] \Rightarrow \alpha = \frac{R - R_0}{R_0 (T - T_0)} = \qty{1.05e-3}{\degreeCelsius^{-1}}\]
\end{enumerate}

\subsubsection{25.15}

\begin{enumerate}[(a)]
  \item The resistance of the $\qty{2.00}{m}$ wire is \[R = \rho \frac{L}{A} = \qty{1.65e-2}{\Omega}\] so its potential difference is \[V = I R = \qty{0.206}{mV}\]

  \item The potential difference would be \[V = I R = I \rho \frac{L}{A} = \qty{0.177}{mV}\]
\end{enumerate}

\subsubsection{25.17}

The electric field in a wire is \[\rho = \frac{E}{J} = E \frac{A}{I} = E \frac{\pi r^2}{I} \Rightarrow E = \frac{\rho I}{\pi r^2}.\]

So the ratio of the two wires' electric fields is

\begin{align*}
  \frac{E_\textrm{Cu}}{E_\textrm{Ag}} & = \frac{\frac{\rho_\textrm{Cu} I}{\pi r_\textrm{Cu}^2}}{\frac{\rho_\textrm{Ag} I}{\pi r_\textrm{Ag}^2}} \\
                                      & = \frac{\rho_\textrm{Cu} I}{\pi r_\textrm{Cu}^2} \frac{\pi r_\textrm{Ag}^2}{\rho_\textrm{Ag} I}         \\
                                      & = \frac{\rho_\textrm{Cu}}{\rho_\textrm{Ag}} \left( \frac{r_\textrm{Ag}}{r_\textrm{Cu}} \right)^2        \\
                                      & = 0.457
\end{align*}

\subsubsection{25.19}

\[R = \rho \frac{L}{A} = \qty{0.125}{\Omega}\]

\subsubsection{25.21}

\begin{enumerate}[(a)]
  \item \[I = A J = \pi r^2 \frac{E}{\rho} = \qty{11}{A}\]

  \item $V = E d = \qty{3.1}{V}$

  \item $R = \rho \frac{L}{A} = \qty{0.28}{\Omega}$
\end{enumerate}

\subsubsection{25.23}

\begin{enumerate}[(a)]
  \item $R = R_0 [1 + \alpha (T - T_0)] = \qty{99.5}{\Omega}$

  \item $R = R_0 [1 + \alpha (T - T_0)] = \qty{0.0158}{\Omega}$
\end{enumerate}

\subsubsection{25.25}

\begin{enumerate}[(a)]
  \item The resistance of the cable is \[R = \rho \frac{L}{A} = \qty{0.219}{\Omega}\] so the voltage drop is \[V = I R = \qty{27.4}{V}\]

  \item $P = V I \times 60 \times 60 = \qty{12.3}{MJ}$
\end{enumerate}

\subsubsection{25.27}

\begin{enumerate}[(a)]
  \item The circuit isn't complete so $I = 0$

  \item No current is flowing so $V = \mathcal{E} = \qty{5.0}{V}$

  \item $V = \qty{5.0}{V}$
\end{enumerate}

\subsubsection{25.29}

When the switch is open no current flows so $\mathcal{E} = V = \qty{3.08}{V}$.

When the switch is closed $V = \mathcal{E} - I r$ so $r = (\mathcal{E} - V) / I = \qty{0.0667}{\Omega}$.

Also, $V = I R$ so $R = V / I = \qty{1.8}{\Omega}$.

\subsubsection{25.31}

\begin{enumerate}[(a)]
  \item Both batteries contribute their voltage so the current flows clockwise and has value $I = V / R = \qty{1.4}{A}$
\end{enumerate}

\subsubsection{25.33}

The current in the circuit is \[\mathcal{E} - I r = I R \Rightarrow I = \frac{\mathcal{E}}{R + r}\] so the power consumed by the resistor is

\begin{align*}
  P & = I^2 R                                        \\
    & = \left( \frac{\mathcal{E}}{R + r} \right)^2 R \\
    & = \frac{\mathcal{E}^2}{R^2 + 2 r R + r^2} R
\end{align*}

which can be rewritten as a quadratic \[P R^2 + (2 P r - \mathcal{E}^2) R + P r^2 = 0\] which has solutions

\begin{align*}
  R & = \frac{-b \pm \sqrt{b^2 - 4 a c}}{2 a}                                              \\
    & = \frac{\mathcal{E}^2 - 2 P r \pm \sqrt{(2 P r - \mathcal{E}^2)^2 - 4 P^2 r^2}}{2 P} \\
    & = \qty{0.429}{\Omega}\textrm{ or }\qty{21}{\Omega}
\end{align*}

\subsubsection{25.35}

\begin{enumerate}[(a)]
  \item $P = V^2 / R \Rightarrow R = V^2 / P = \qty{144}{\Omega}$

  \item $R = \qty{240}{\Omega}$

  \item $I_{100} = P / V = \qty{0.833}{A}$ and $I_{60} = P / V = \qty{0.500}{A}$
\end{enumerate}

\subsubsection{25.37}

\begin{enumerate}[(a)]
  \item $R = V^2 / P = \qty{484}{\Omega}$ so $P_\textrm{US} = V^2 / R = \qty{29.8}{W}$

  \item $I = V / R = \qty{0.248}{A}$
\end{enumerate}

\subsubsection{25.41}

\begin{enumerate}[(a)]
  \item $P = V I = \qty{300}{W}$

  \item $E = P t = \qty{0.90}{J}$
\end{enumerate}

\subsubsection{25.43}

\begin{enumerate}[(a)]
  \item $E = P t = V I t = \qty{2.6}{MJ}$

        \setcounter{enumi}{2}
  \item $\qty{1.6}{h}$
\end{enumerate}

\subsubsection{25.45}

The current in the circuit is \[\mathcal{E} - I r = I R \Rightarrow I = \frac{\mathcal{E}}{r + R} = \qty{0.421}{A}\] so the power generated by the battery is \[P = V I = \qty{5.05}{W}\] and the power consumed by internal resistance is \[P = I^2 R = \qty{0.621}{W}\] which is $12.3\%$ of the generated power

\subsubsection{25.49}

\begin{enumerate}[(a)]
  \item \[\tau = \frac{m}{n e^2 \rho} = \qty{1.55}{ps}\]
\end{enumerate}

\subsection{Problems}

\subsubsection{25.51}

\begin{enumerate}[(a)]
  \item \[R = \rho L / A \Rightarrow \rho = R A / L = \qty{3.65e-8}{\Omega.m}\]

  \item \[I = V / R = E d / R = \qty{172}{A}\]

  \item \[I = n |q| v_d A \Rightarrow v_d = I / (n |q| A) = \qty{2.58}{mm/s}\]
\end{enumerate}

\subsubsection{25.53}

When the leads of the voltmeter are placed across the terminals of the battery no current flows so the internal resistance has no effect and the voltmeter reads the battery's emf \[\mathcal{E} = \qty{12.6}{V}.\]

The resistance of $\qty{1}{m}$ of wire is given by \[\mathcal{E} - I r = I R L \Rightarrow R = \frac{\mathcal{E} - I r}{I L}.\]

We can find $r$ by equating $R$ from the two measurements

\begin{align*}
  \frac{\mathcal{E} - I_1 r}{I_1 L_1} & = \frac{\mathcal{E} - I_2 r}{I_2 L_2}                         \\
  I_2 L_2 (\mathcal{E} - I_1 r)       & = I_1 L_1 (\mathcal{E} - I_2 r)                               \\
  (L_1 - L_2) I_1 I_2 r               & = (I_1 L_1 - I_2 L_2) \mathcal{E}                             \\
  r                                   & = \frac{(I_1 L_1 - I_2 L_2) \mathcal{E}}{(L_1 - L_2) I_1 I_2} \\
                                      & = \qty{0.600}{\Omega}.
\end{align*}

Using this and the values from a single measurement we can find $R$

\[R = \frac{\mathcal{E} - I_1 r}{I_1 L_1} = \qty{0.0600}{\Omega}.\]

\subsubsection{25.55}

\begin{enumerate}[(a)]
  \item $I = \qty{2.5}{mA}$

  \item $E = \rho J = \rho I / A = \qty{2.14e-5}{N/C}$

  \item $E = \qty{8.55e-5}{N/C}$

  \item $V = E_1 L_1 + E_2 L_2 = \qty{1.80e-4}{V}$
\end{enumerate}

\subsubsection{25.59}

\begin{enumerate}[(a)]
  \item

        \begin{align*}
          R & = \int dR                                                                                   \\
            & = \int_0^h \rho \frac{dy}{\pi (r_2 + (r_1 - r_2) \frac{y}{h})^2}                            \\
            & = \frac{\rho}{\pi} \left[ -\frac{h}{(r_1 - r_2)(r_2 + (r_1 - r_2) \frac{y}{h})} \right]_0^h \\
            & = \frac{\rho h}{\pi (r_1 - r_2)} \left( \frac{1}{r_2} - \frac{1}{r_1} \right)               \\
            & = \frac{\rho h}{\pi (r_1 - r_2)} \frac{r_1 - r_2}{r_1 r_2}                                  \\
            & = \frac{\rho h}{\pi r_1 r_2}
        \end{align*}
\end{enumerate}

\subsubsection{25.61}

\begin{enumerate}[(a)]
  \item When current flows from the negative to the positive terminal the emf is \[V_1 = \mathcal{E} - I_1 r \Rightarrow \mathcal{E} = V_1 + I_1 r.\] When current flows from the positive to the negative terminal it is \[V_2 = \mathcal{E} + I_2 r \Rightarrow \mathcal{E} = V_2 - I_2 r.\] By equating these values we can find $r$

        \begin{align*}
          V_1 + I_1 r & = V_2 - I_2 r                 \\
          r           & = \frac{V_2 - V_1}{I_1 + I_2} \\
                      & = \qty{0.360}{\Omega}
        \end{align*}

  \item $\mathcal{E} = V_1 + I_1 r = \qty{8.94}{V}$
\end{enumerate}

\subsubsection{25.63}

\begin{enumerate}[(a)]
  \item $R = \rho \frac{L}{A} = \qty{1.0}{k \Omega}$

  \item $V = I R = \qty{100}{V}$

  \item $P = I^2 R = \qty{10}{W}$
\end{enumerate}

\subsubsection{25.65}

\begin{enumerate}[(a)]
  \item $\frac{75 \times 60 \times 60 \times 24 \times 365}{10^3 \times 60 \times 60} \times 0.120 = \$78.90$

  \item $\frac{400 \times 60 \times 60 \times 8 \times 365}{10^3 \times 60 \times 60} \times 0.120 = \$140.00$
\end{enumerate}

\subsubsection{25.67}

\begin{enumerate}[(a)]
  \item From the resistivity at the left end we find \[\rho(0) = a + b (0)^2 = \qty{2.25e-8}{\Omega.m} \Rightarrow a = \qty{2.25e-8}{\Omega.m}\] and from the resistivity at the right end we find \[\rho(1.50) = \num{2.25e-8} + b (1.50)^2 = \qty{8.50e-8}{\Omega.m} \Rightarrow b = \qty{2.78e-8}{\Omega/m}.\]

        If we treat the rod as a series of resistors of resistivit $\rho(x)$ and length $dx$ we can integrate to find the total resistance

        \begin{align*}
          R & = \int_0^L \rho(x) \frac{dx}{A}                               \\
            & = \frac{1}{\pi r^2} \int_0^L (a + b x^2) \,dx                 \\
            & = \frac{1}{\pi r^2} \left[ ax + \frac{1}{3} b x^3 \right]_0^L \\
            & = \frac{1}{\pi r^2} \left( a L + \frac{1}{3} b L^3 \right)    \\
            & = \qty{171}{\mu \Omega}
        \end{align*}

  \item

        \begin{align*}
          \rho \left( \frac{L}{2} \right)          & = \frac{E \left( \frac{L}{2} \right)}{J}                        \\
          \Rightarrow E \left( \frac{L}{2} \right) & = \rho \left( \frac{L}{2} \right) J                             \\
                                                   & = \left( a + b \left( \frac{L}{2} \right)^2 \right) \frac{I}{A} \\
                                                   & = \qty{176}{\mu N/C}
        \end{align*}

  \item

        \begin{align*}
          R_1 & = \int_0^{L / 2} \rho (x) \frac{dx}{A}                                                                                  \\
              & = \qty{54.7}{\mu \Omega}                                                                                                \\
          R_2 & = \int_{L / 2}^L \rho (x) \frac{dx}{A}                                                                                  \\
              & = \frac{1}{\pi r^2} \left( a L + \frac{1}{3} b L^3 - a \frac{L}{2} - \frac{1}{3} b \left( \frac{L}{2} \right)^3 \right) \\
              & = \frac{1}{\pi r^2} \left( a \frac{L}{2} + \frac{7}{24} b L^3 \right)                                                   \\
              & = \qty{116}{\mu \Omega}
        \end{align*}
\end{enumerate}

\subsubsection{25.69}

\begin{enumerate}[(a)]
  \item The resistance of the cylinder containing the saltwater solution is \[r = \rho \frac{L}{A} = \frac{s_0}{s} \frac{L}{\pi r^2}.\]

        No current flows through the cylinder containing pure distilled water because it has infinite resistivity, so we can use Ohm's law to determine the salinity of the saltwater solution

        \begin{align*}
          \mathcal{E}       & = I (r + R)                                            \\
                            & = I \left( \frac{s_0}{s} \frac{L}{\pi r^2} + R \right) \\
          \mathcal{E} - I R & = \frac{I L s_0}{\pi r^2 s}                            \\
          s                 & = \frac{I L s_0}{\pi r^2 (\mathcal{E} - I R)}          \\
                            & = \qty{4.25}{ppt}.
        \end{align*}

  \item The capacitance of the cylinder contining pure distilled water is \[C = K \epsilon_0 \frac{A}{d} = \qty{1.86e-10}{F}.\]

        The voltage across it is equal to $\mathcal{E}$ minus the voltage drop across $R$ \[V = \mathcal{E} - I R = \qty{2.74}{V}\] so the charge on its plates is \[Q = C V = \qty{510}{pC}.\]

  \item $P = I^2 R = I^2 \frac{s_0}{s} \frac{L}{A} = \qty{1.33}{W}$

  \item The cylinder would need to have equal resistance so

        \begin{align*}
          R & = \frac{s_0}{s} \frac{L}{A} \\
          s & = \frac{L s_0}{A R}         \\
            & = \qty{1.60}{ppt}.
        \end{align*}
\end{enumerate}

\subsubsection{25.71}

The current in the circuit is given by Ohm's law \[I = \frac{\mathcal{E}}{R_1 + R_2}.\]

The potential difference across the capacitor is given by

\begin{align*}
  V & = \mathcal{E} - I R_2                                  \\
    & = \mathcal{E} \left( 1 - \frac{R_2}{R_1 + R_2} \right) \\
    & = \mathcal{E} \frac{R_1}{R_1 + R_2}
\end{align*}

and also \[V = \frac{Q}{C}.\]

Equating these two expressions for $V$ we find that $\mathcal{E}$ is

\begin{align*}
  \mathcal{E} \frac{R_1}{R_1 + R_2} & = \frac{Q}{C}                 \\
  \mathcal{E}                       & = \frac{Q (R_1 + R_2)}{C R_1} \\
                                    & = \qty{6.67}{V}.
\end{align*}

\subsubsection{25.73}

\begin{enumerate}[(a)]
  \setcounter{enumi}{1}
  \item No, because it's not a linear relationship between $V$ and $I$

  \item Yes, because it's a linear relationship between $V$ and $I$
\end{enumerate}

\subsubsection{25.75}

\begin{enumerate}[(a)]
  \item Inwards

  \item

        \begin{align*}
          V             & = \int \mathbf{E} \cdot d\mathbf{l}                                                                                      \\
                        & = \int_{r_\textrm{outer}}^{r_\textrm{inner}} \left( -\frac{c}{r} \hat{\mathbf{r}} \right) \cdot (dr \, \hat{\mathbf{r}}) \\
                        & = \int_{r_\textrm{outer}}^{r_\textrm{inner}} -\frac{c}{r} \,dr                                                           \\
                        & = -c \ln \frac{r_\textrm{inner}}{r_\textrm{outer}}                                                                       \\
                        & = c \ln \frac{r_\textrm{outer}}{r_\textrm{inner}}                                                                        \\
          \Rightarrow c & = \frac{V}{\ln (r_\textrm{outer} / r_\textrm{inner})}
        \end{align*}

  \item

        \begin{align*}
          R & = \int dR                                                              \\
            & = \int_{r_\textrm{inner}}^{r_\textrm{outer}} \rho \frac{dr}{2 \pi r L} \\
            & = \frac{\rho}{2 \pi L} \ln \frac{r_\textrm{outer}}{r_\textrm{inner}}
        \end{align*}

  \item \[\rho = \frac{2 \pi R L}{\ln (r_\textrm{outer} / r_\textrm{inner})} = \qty{616}{\Omega.m}\]
\end{enumerate}

\subsubsection{25.77}

\begin{enumerate}[(a)]
  \item The total resistance of the rod is

        \begin{align*}
          R & = \int_0^L \rho \frac{dx}{A}                     \\
            & = \frac{p_0}{A} \int_0^L e^{-x / L} \,dx         \\
            & = \frac{p_0}{A} [-L e^{-x / L}]_0^L              \\
            & = \frac{p_0 L}{A} \left( 1 - \frac{1}{e} \right)
        \end{align*}

        and the current in it is

        \begin{align*}
          I & = V / R                                              \\
            & = \frac{A V_0}{p_0 L \left( 1 - \frac{1}{e} \right)}
        \end{align*}

  \item The current density in the rod is \[J = I / A = \frac{V_0}{p_0 L \left( 1 - \frac{1}{e} \right)}\] so the electric field magnitude is

        \begin{align*}
          \rho          & = \frac{E}{J}                                                     \\
          \Rightarrow E & = \rho J                                                          \\
                        & = p_0 e^{-x / L} \frac{V_0}{p_0 L \left( 1 - \frac{1}{e} \right)} \\
                        & = \frac{V_0 e^{-x / L}}{L \left( 1 - \frac{1}{e} \right)}
        \end{align*}

  \item

        \begin{align*}
          V(x) & = V_0 - \int_0^x E \,dx'                                                        \\
               & = V_0 - \int_0^x \frac{V_0 e^{-x' / L}}{L \left( 1 - \frac{1}{e} \right)} \,dx' \\
               & = V_0 - \frac{V_0}{L \left( 1 - \frac{1}{e} \right)} [-L e^{-x' / L}]_0^x       \\
               & = \frac{V_0 \left( e^{-x / L} - \frac{1}{e} \right)}{1 - \frac{1}{e}}
        \end{align*}
\end{enumerate}

\section{Direct-Current Circuits}

\subsection{Resistors in Series and Parallel}

\subsubsection{Example 26.1}

The equivalent resistance of the two resistors in parallel is \[R_{cb} = \frac{1}{(1 / 6) + (1 / 3)} = \qty{2}{\Omega}.\]

The equivalent resistance of the two resulting resistors in series is \[R_{ab} = 4 + 2 = \qty{6}{\Omega}.\]

The current in the circuit (and thus the $\qty{4}{\Omega}$ resistor) is \[I = \frac{V}{R} = \qty{3}{A}.\]

The potential difference across the $\qty{4}{\Omega}$ resistor is \[V_{ac} = I R = \qty{12}{V}\] so the potential difference across the two resistors in parallel is \[V_{cb} = \mathcal{E} - V_{ac} = \qty{6}{V}.\]

Thus the current through the $\qty{6}{\Omega}$ resistor is \[I = \frac{V_{cb}}{R} = \qty{1}{A}\] and the current through the $\qty{3}{\Omega}$ resistor is \[I = \frac{V_{cb}}{R} = \qty{2}{A}.\]

\subsubsection{Example 26.2}

\begin{enumerate}[(a)]
  \item The total resistance of the circuit is $2 R = \qty{4}{\Omega}$ so the current is $I = V / R = \qty{2}{A}$, the potential difference across each bulb is $V = I R = \qty{4}{V}$, the power delivered to each bulb is $P = V I = \qty{8}{W}$, and the power delivered to the entire network is $2 P = \qty{16}{W}$

  \item The potential difference across each bulb is equal to that of the battery $\qty{8}{V}$ so the current in each bulb is $I = V / R = \qty{4}{A}$, the power delivered to each bulb is $P = V I = \qty{32}{W}$, and the power delivered to the entire network is $2 P = \qty{64}{W}$

  \item In the series case the circuit will no longer be complete and current will stop flowing. In the parallel case there will be no difference to the other bulb but the circuit will consume less energy
\end{enumerate}

\subsection{Kirchhoff's Rules}

\subsubsection{Example 26.3}

\begin{enumerate}[(a)]
  \item Following the circuit anticlockwise from $a$ and using Kirchhoff's loop rule we find

        \begin{align*}
          -4 I - 4 - 7 I + 12 - 2 I - 3 I & = 0            \\
          -16 I                           & = -8           \\
          I                               & = \qty{0.5}{A}
        \end{align*}

  \item Following the circuit clockwise from $b$ and using Kirchhoff's loop rule we find

        \begin{align*}
          V_{ab} & = 7 I + 4 + 4 I \\
                 & = \qty{9.5}{V}
        \end{align*}

  \item

        \begin{align*}
          P_{12} & = V I = \qty{6}{W}   \\
          P_{4}  & = -V I = \qty{-2}{W}
        \end{align*}
\end{enumerate}

\subsection{Electrical Measuring Instruments}

\subsubsection{Example 26.8}

\begin{align*}
  (I_\textrm{a} - I_\textrm{fs}) R_\textrm{sh} & = I_\textrm{fs} R_c                                               \\
  R_\textrm{sh}                                & = \frac{I_\textrm{fs} R_\textrm{c}}{I_\textrm{a} - I_\textrm{fs}} \\
                                               & = \qty{0.408}{\Omega}
\end{align*}

\subsubsection{Example 26.9}

\begin{align*}
  V_\textrm{V} & = I_\textrm{fs} (R_\textrm{C} + R_\textrm{S})       \\
  R_\textrm{S} & = \frac{V_\textrm{V}}{I_\textrm{fs}} - R_\textrm{C} \\
               & = \qty{9980}{\Omega}
\end{align*}

\subsubsection{Example 26.10}

\begin{align*}
  V & = I (R + R_\textrm{A})       \\
  R & = \frac{V}{I} - R_\textrm{A} \\
    & = \qty{118}{\Omega}
\end{align*}

\[P = I^2 R = \qty{1.18}{W}\]

\subsubsection{Example 26.11}

\begin{align*}
  I           & = \frac{V}{R} + \frac{V}{R_\textrm{V}}      \\
  \frac{V}{R} & = I - \frac{V}{R_\textrm{V}}                \\
              & = \frac{I R_\textrm{V} - V}{R_\textrm{V}}   \\
  R           & = \frac{R_\textrm{V} V}{I R_\textrm{V} - V} \\
              & = \qty{121}{\Omega}
\end{align*}

\[P = \frac{V^2}{R} = \qty{1.19}{W}\]

\subsection{R-C Circuits}

\subsubsection{Example 26.12}

\begin{enumerate}[(a)]
  \item $\tau = R C = \qty{10}{s}$

  \item \[\frac{q(46)}{Q_\textrm{f}} = \frac{Q_\textrm{f} (1 - e^{-46 / R C})}{Q_\textrm{f}} = 0.99\]

  \item \[\frac{i(46)}{I_0} = \frac{I_0 e^{-46 / R C}}{I_0} = 0.010\]
\end{enumerate}

\subsubsection{Example 26.13}

\begin{enumerate}[(a)]
  \item

        \begin{align*}
          q              & = Q_0 e^{-t / RC}        \\
          -\frac{t}{R C} & = \ln \frac{q}{Q_0}      \\
          t              & = -R C \ln \frac{q}{Q_0} \\
                         & = \qty{23}{s}
        \end{align*}

  \item $i(23) = -\frac{Q_0}{R C} e^{-24 / R C} = \qty{-4.54e-8}{A}$
\end{enumerate}

\subsection{Power Distribution Systems}

\subsubsection{Example 26.14}

\begin{enumerate}[(a)]
  \item

        \begin{align*}
          I_\textrm{toaster} & = \qty{15}{A}       \\
          R_\textrm{toaster} & = \qty{8}{\Omega}   \\
          I_\textrm{pan}     & = \qty{11}{A}       \\
          R_\textrm{pan}     & = \qty{11}{\Omega}  \\
          I_\textrm{lamp}    & = \qty{0.83}{A}     \\
          R_\textrm{lamp}    & = \qty{145}{\Omega}
        \end{align*}

  \item The total current is $15 + 11 + 0.83 = \qty{27}{A}$ which will trip the circuit breaker
\end{enumerate}

\subsection{Guided Practice}

\subsubsection{VP26.2.1}

\begin{enumerate}[(a)]
  \item $R_\textrm{eq} = \qty{7.00}{\Omega}$

  \item $R_\textrm{eq} = \frac{1}{(1 / R_1) + (1 / R_2) + (1 / R_3)} = \qty{0.571}{\Omega}$

  \item $R_\textrm{eq} = R_1 + \frac{1}{(1 / R_2) + (1 / R_3)} = \qty{2.33}{\Omega}$

  \item $R_\textrm{eq} = \frac{1}{(1 / R_1) + (1 / (R_2 + R_3))} = \qty{0.857}{\Omega}$
\end{enumerate}

\subsubsection{VP26.2.2}

\begin{enumerate}[(a)]
  \item $I = V / R_\textrm{eq} = V / \frac{1}{(1 / R_1) + (1 / (R_2 + R_3))} = \qty{8.18}{A}$

  \item $I = V / R_1 = \qty{6.00}{A}$

  \item $I = V / (R_2 + R_3) = \qty{2.18}{A}$

  \item $I = V / (R_2 + R_3) = \qty{2.18}{A}$
\end{enumerate}

\subsubsection{VP26.2.3}

\begin{enumerate}[(a)]
  \item $P = V I = V^2 / R_\textrm{eq} = V^2 / \frac{1}{(1 / R_1) + (1 / R_2) + (1 / R_3)} = \qty{54.6}{W}$

  \item $P = V I = V^2 / R_1 = \qty{20.6}{W}$

  \item $P = \qty{18.0}{W}$

  \item $P = \qty{16.0}{W}$
\end{enumerate}

\subsubsection{VP26.2.4}

\begin{enumerate}[(a)]
  \item $P = V I = V^2 / R = V^2 / (R_1 + \frac{1}{(1 / R_2) + (1 / R_3)}) = \qty{9.84}{W}$

  \item $I = V / R = P / V = \qty{1.09}{A}$, $P = I^2 R = \qty{5.98}{W}$

  \item $V = \mathcal{E} - I R_1 = \qty{3.53}{V}$, $P = V^2 / R = \qty{2.08}{W}$

  \item $P = V^2 / R = \qty{1.78}{W}$
\end{enumerate}

\subsubsection{VP26.7.1}

\begin{enumerate}[(a)]
  \item $I = V / R = \qty{1}{A}$

  \item $V_\textrm{ab} = \qty{7}{V}$

  \item $P_{12} = V I = \qty{12}{W}$, $P_4 = V I = \qty{4}{W}$
\end{enumerate}

\subsubsection{VP26.7.2}

\begin{enumerate}[(a)]
  \item The current experienced by the resistor is $I = I_1 + I_2 = \qty{1.55}{A}$ so $V_\textrm{ab} = I R = \qty{7.75}{V}$

  \item Using Kirchhoff's loop rule clockwise around the outer loop we find

        \begin{align*}
          0   & = \mathcal{E}_1 - I_1 r_1 - (I_1 + I_2) R           \\
          r_1 & = \frac{\mathcal{E}}{I_1} - \frac{I_1 + I_2}{I_1} R \\
              & = \qty{1.25}{\Omega}
        \end{align*}

  \item Using Kirchhoff's loop rule clockwise around the right loop we find

        \begin{align*}
          0   & = \mathcal{E}_2 - I_2 r_2 - (I_1 + I_2) R             \\
          r_2 & = \frac{\mathcal{E}_2}{I_2} - \frac{I_1 + I_2}{I_2} R \\
              & = \qty{0.926}{\Omega}
        \end{align*}
\end{enumerate}

\subsubsection{VP26.7.3}

\begin{enumerate}[(a)]
  \item $P = \qty{33.4}{W}$

  \item $P = \qty{18.8}{W}$

  \item $P = \qty{2.09}{W}$

  \item $P = \qty{52.2}{W}$

  \item $P = \qty{25.0}{W}$
\end{enumerate}

\subsubsection{VP26.13.1}

\begin{enumerate}[(a)]
  \item $Q_\textrm{f} = C \mathcal{E} = \qty{2.88e-5}{C}$

  \item $I_0 = \mathcal{E} / R = \qty{9e-7}{A}$

  \item $\tau = \qty{32}{s}$

  \item $q(18.0) / Q_\textrm{f} = 1 - e^{-18.0 / \tau} = 0.430$

  \item $i(18.0) / I_0 = e^{-18.0 / \tau} = 0.570$
\end{enumerate}

\subsubsection{VP26.13.2}

\begin{enumerate}[(a)]
  \item

        \begin{align*}
          q & = Q_0 e^{-t / R C}       \\
          t & = -R C \ln \frac{q}{Q_0} \\
            & = \qty{11.0}{s}
        \end{align*}

  \item $i = \qty{-1.37e-7}{A}$
\end{enumerate}

\subsubsection{VP26.13.3}

\begin{enumerate}[(a)]
  \item

        \begin{align*}
          q              & = Q_0 e^{-t / R C}           \\
          -\frac{t}{R C} & = \ln \frac{q}{Q_0}          \\
          R              & = -\frac{t}{C \ln (q / Q_0)} \\
                         & = \qty{1.32}{M\Omega}
        \end{align*}

  \item $I_0 = -\frac{Q_0}{R C} = \qty{-5.21e-7}{A}$

  \item $i(17.0) = I_0 e^{-t / R C} = \qty{-1.04e-7}{A}$
\end{enumerate}

\subsubsection{VP26.13.4}

\begin{enumerate}[(a)]
  \item $q = C \mathcal{E} (1 - e^{-t / R C}) = \qty{6.12e-5}{C}$

  \item $q = Q_0 e^{-t / R C} = \qty{9.89e-6}{C}$
\end{enumerate}

\subsection{Bridging Problem}

\begin{enumerate}[(a)]
  \item The equivalent capacitance of the two capacitors is \[C_\textrm{eq} = \frac{1}{(1 / C_1) + (1 / C_2)} = \qty{1.44}{\mu F}\] so the energy stored in the capacitors is \[E = \frac{Q^2}{2 C_\textrm{eq}} = \qty{9.39}{J}\]

  \item The equivalent resistance of the resistor and the voltmeter is \[R_\textrm{eq} = \frac{1}{(1 / R_1) + (1 / R_2)} = \qty{646}{\Omega}.\] This is an R-C circuit so the rate of change of the energy stored in the capacitors just after the connection is made is

        \begin{align*}
          \frac{d E}{d t} & = \frac{d}{dt} \left( \frac{Q^2}{2 C_\textrm{eq}} \right)              \\
                          & = \frac{2 Q I}{2 C_\textrm{eq}}                                        \\
                          & = \frac{2 Q (-\frac{Q}{R_\textrm{eq} C_\textrm{eq}})}{2 C_\textrm{eq}} \\
                          & = -\frac{Q^2}{R_\textrm{eq} C_\textrm{eq}^2}                           \\
                          & = \qty{-2.02e4}{W}
        \end{align*}

  \item The energy stored in the capacitors is proportional to $Q^2$ so it will have reduced by a factor of $1 / e$ when the charge has reduced by a factor of $1 / \sqrt{e}$ which happens when

        \begin{align*}
          e^{-t / R C}   & = \frac{1}{\sqrt{e}}          \\
          \frac{-t}{R C} & = \ln \frac{1}{\sqrt{e}}      \\
          t              & = -R C \ln \frac{1}{\sqrt{e}} \\
                         & = \qty{0.465}{ms}
        \end{align*}

  \item From part (b)

        \begin{align*}
          \frac{d E}{d t} & = \frac{q i}{C}                                                  \\
                          & = \frac{Q_0 e^{-t / R C} \cdot -\frac{Q_0}{R C} e^{-t / R C}}{C} \\
                          & = -\frac{Q_0^2 e^{-2 t / R C}}{R C^2}                            \\
                          & = \qty{-7.43}{kW}
        \end{align*}
\end{enumerate}

\subsection{Exercises}

\subsubsection{26.1}

Each segment has resistance $R / 3$. The circle is equivalent to two wires of resistance $R / 6$ in parallel which have an equivalent resistance of \[R_\textrm{circle} = \frac{1}{(1 / (R / 6)) + (1 / (R / 6))} = \frac{R}{12}\] so the total resistance is \[\frac{R}{3} + \frac{R}{12} + \frac{R}{3} = \frac{3}{4} R.\]

\subsubsection{26.3}

We can find the battery's emf by rearranging the equation for the power dissipated by a resistor

\begin{align*}
  P           & = \frac{\mathcal{E}^2}{R} \\
  \mathcal{E} & = \sqrt{P R}              \\
              & = \qty{30}{V}.
\end{align*}

If a second resistor is connected in series with the first, the total power dissipated by them is

\[P = \frac{\mathcal{E}^2}{R_1 + R_2} = \qty{22.5}{W}.\]

\subsubsection{26.5}

\begin{enumerate}[(a)]
  \item \[I = \frac{\mathcal{E}}{R} = \mathcal{E} \left( \frac{1}{R_\textrm{ab}} + \frac{1}{R_\textrm{ac} + R_\textrm{bc}} \right) = \qty{3.50}{A}\]

  \item \[I = \frac{\mathcal{E}}{R} = \mathcal{E} \left( \frac{1}{R_\textrm{bc}} + \frac{1}{R_\textrm{ab} + R_\textrm{ac}} \right) = \qty{4.50}{A}\]

  \item \[I = \frac{\mathcal{E}}{R} = \mathcal{E} \left( \frac{1}{R_\textrm{ac}} + \frac{1}{R_\textrm{ab} + R_\textrm{bc}} \right) = \qty{3.15}{A}\]

  \item

        \begin{align*}
          \mathcal{E} - I r & = I R                                                                                           \\
          I                 & = \frac{\mathcal{E}}{r + R}                                                                     \\
                            & = \frac{\mathcal{E}}{r + \frac{1}{(1 / R_\textrm{bc}) + (1 / (R_\textrm{ab} + R_\textrm{ac}))}} \\
                            & = \qty{3.25}{A}
        \end{align*}
\end{enumerate}

\subsubsection{26.7}

The top and bottom resistors are in parallel with an equivalent resistance of \[R_{13} = \frac{1}{(1 / R_1) + (1 / R_2)} = \qty{11.25}{\Omega}.\] They are in series with the middle resistor with an equivalent resistance of \[R_{123} = R_2 + R_{13} = \qty{29.25}{\Omega}.\] The current in the circuit is thus \[I = \frac{\mathcal{E}}{r + R_{123}} = \qty{0.769}{A}.\]

\subsubsection{26.9}

\begin{enumerate}[(a)]
  \item $I = \mathcal{E} / R$

  \item $I = \mathcal{E} / 6 R$

  \item Parallel
\end{enumerate}

\subsubsection{26.11}

\begin{enumerate}[(a)]
  \item The voltage drop across $R_2$ is \[V_2 = I_2 R_2 = \qty{24.0}{V}\] so \[I_1 = V_2 / R_1 = \qty{8.00}{A}\] and \[I_3 = I_1 + I_2 = \qty{12.0}{A}.\]

  \item The equivalent resistance of the three resistors is \[R = \frac{1}{(1 / R_1) + (1 / R_2)} + R_3 = \qty{7}{\Omega}\] so \[\mathcal{E} = I R = \qty{84.0}{V}.\]
\end{enumerate}

\subsubsection{26.13}

The two resistors on the left have an equivalent resistance of \[\frac{1}{(1 / 3.00) + (1 / 6.00)} = \qty{2.00}{\Omega}.\] The two resistors on the right have an equivalent resistance of \[\frac{1}{(1 / 12.0) + (1 / 4.00)} = \qty{3.00}{\Omega}.\] All four resistors have an equivalent resistance of \[2.00 + 3.00 = \qty{5.00}{\Omega}.\] The current through the circuit is \[I = \frac{\mathcal{E}}{R} = \qty{12.0}{A}.\] The voltage across the two resistors on the left is \[V = I R = \qty{24.0}{V}\] so the current through  the $\qty{3.00}{\Omega}$ resistor is \[I = \frac{V}{R} = \qty{8.00}{A}\] and the current through the $\qty{6.00}{\Omega}$ resistor is \[I = \frac{V}{R} = \qty{4.00}{A}.\] The voltage across the two resistors on the right is \[V = I R = \qty{36}{V}\] so the current through the $\qty{12.0}{\Omega}$ resistor is \[I = \frac{V}{R} = \qty{3.00}{A}\] and the current through the $\qty{4.00}{\Omega}$ resistor is \[I = \frac{V}{R} = \qty{9.00}{A}.\]

\subsubsection{26.23}

\begin{enumerate}[(a)]
  \item $I_R = \qty{2.00}{A}$

  \item $R = \qty{5.00}{\Omega}$

  \item $\mathcal{E} = \qty{42.0}{V}$

  \item $I = \qty{3.50}{A}$
\end{enumerate}

\subsubsection{26.25}

\begin{enumerate}[(a)]
  \item $I = \qty{8.00}{A}$

  \item $\mathcal{E}_1 = \qty{36.0}{V}$, $\mathcal{E}_2 = \qty{54.0}{V}$

  \item $R = \qty{9.00}{\Omega}$
\end{enumerate}

\subsubsection{26.27}

\begin{enumerate}[(a)]
  \item $I_1 = \qty{1.60}{A}$, $I_2 = \qty{1.40}{A}$, $I_3 = \qty{0.200}{A}$

  \item $V_\textrm{ab} = \qty{10.4}{V}$
\end{enumerate}

\subsubsection{26.29}

\begin{enumerate}[(a)]
  \item $\mathcal{E} = \qty{36.4}{V}$
\end{enumerate}

\subsubsection{26.33}

\begin{enumerate}[(a)]
  \item In parallel with the galvanometer place a resistor of resistance \[R_\textrm{sh} = \frac{I_\textrm{fs} R_\textrm{c}}{I_\textrm{a} - I_\textrm{fs}} = \qty{0.641}{\Omega}\]

  \item In series with the galvanometer place a resistor of resistance \[R_\textrm{s} = \frac{V_\textrm{V}}{I_\textrm{fs}} - R_\textrm{C} = \qty{975}{\Omega}\]
\end{enumerate}

\subsubsection{26.35}

\begin{enumerate}[(a)]
  \item The equivalent resistance of the $\qty{5.00}{k \Omega}$ and the voltmeter is \[\frac{1}{(1 / \num{5.00e3}) + (1 / \num{10.0e3})} = \qty{3.33}{k \Omega}.\] The equivalent resistance of the two resistors and the voltmeter is \[\num{3.33e3} + \num{6.00e3} = \qty{9.33}{k \Omega}.\] The current in the circuit is \[I = \mathcal{E} / R = \qty{5.36}{mA}\] so the potential difference across the $\qty{5.00}{k \Omega}$ resistor and voltmeter is \[V = I R = \qty{17.8}{V}.\]

  \item The current in the circuit is \[I = \mathcal{E} / R = \qty{4.55}{mA}\] so the voltage across the $\qty{5.00}{k \Omega}$ resistor is \[V = I R = \qty{22.7}{V}.\]

  \item $21.2\%$
\end{enumerate}

\subsubsection{26.37}

\begin{enumerate}[(a)]
  \item At time $t = 0$ \[V_0 = \frac{Q_0}{C} \Rightarrow C = \frac{Q_0}{V_0} = \frac{Q_0}{12}.\] At time $t = 4$

        \begin{align*}
          V                & = \frac{Q_0 e^{-4 / R C}}{C} \\
                           & = 12 e^{-48 / Q_0 R}         \\
          \ln \frac{V}{12} & = -\frac{48}{Q_0 R}          \\
          Q_0              & = -\frac{48}{R \ln (V / 12)} \\
                           & = \qty{10.2}{\mu C}.
        \end{align*}

        So the capacitance is \[C = \frac{Q_0}{12} = \qty{0.849}{\mu F}.\]

  \item $\tau = R C = \qty{2.86}{s}$
\end{enumerate}

\subsubsection{26.39}

\begin{enumerate}[(a)]
  \item $V = \qty{0}{V}$

  \item $V = \qty{245}{V}$

  \item $q = \qty{0}{C}$

  \item $I = \mathcal{E} / R = \qty{32.7}{mA}$

  \item $V_\textrm{capacitor} = \qty{245}{V}$, $V_\textrm{resistor} = \qty{0}{V}$, $q = \qty{1.13}{mC}$, $I = 0$
\end{enumerate}

\subsubsection{26.41}

\begin{enumerate}[(a)]
  \item The equivalent capacitance of the two capacitors is \[C = \qty{35.0}{\mu F}\] and the equivalent resistance of the two resistors is \[R = \qty{80.0}{\Omega}.\] The voltage across the capacitors will be $\qty{10.0}{V}$ at

        \begin{align*}
          V & = V_0 e^{-t / R C}       \\
          t & = -R C \ln \frac{V}{V_0} \\
            & = \qty{4.21}{ms}.
        \end{align*}

  \item $i = -\frac{V_0}{R} e^{-t / R C} = \qty{-125}{mA}$
\end{enumerate}

\subsubsection{26.45}

Because the three capacitors are in series and all have the same charge we can replace them with a single capacitor with capacitance \[C = \frac{1}{(1 / C_1) + (1 / C_2) + (1 / C_3)} = \qty{4.62}{pF}.\]

It will have lost 80\% of its stored energy at

\begin{align*}
  \frac{q^2}{2 C}                   & = 0.2 \frac{Q_0^2}{2 C}  \\
  q^2                               & = 0.2 Q_0^2              \\
  \left( Q_0 e^{-t / R C} \right)^2 & = 0.2 Q_0^2              \\
  Q_0^2 e^{-2 t / R C}              & = 0.2 Q_0^2              \\
  e^{-2 t / R C}                    & = 0.2                    \\
  t                                 & = -\frac{R C \ln 0.2}{2} \\
                                    & = \qty{9.29e-11}{s}
\end{align*}

at which point the current will be \[i = -\frac{Q_0}{R C} e^{-t / R C} = \qty{13.6}{A}.\]

\subsubsection{26.47}

\begin{enumerate}[(a)]
  \item Just after the switch is closed the capacitors have no charge, no potential drop, and it's as if they're not there so we can calculate the current from the resisitors alone. The equivalent resistance is $R = \qty{107}{\Omega}$ so the current is $I = \qty{0.938}{A}$.

  \item After the switch has been closed for a long time no current flows through the capacitors so we can again calculate the current from the resistors alone. The equivalent resistance is $R = \qty{165}{\Omega}$ so the current is $I = \qty{0.606}{A}$.
\end{enumerate}

\subsubsection{26.49}

\begin{enumerate}[(a)]
  \item $Q = C \mathcal{E} = \qty{165}{\mu C}$

  \item

        \begin{align*}
          q              & = C \mathcal{E} (1 - e^{-t / R C})               \\
          e^{-t / RC}    & = 1 - \frac{q}{C \mathcal{E}}                    \\
          -\frac{t}{R C} & = \ln \left( 1 - \frac{q}{C \mathcal{E}} \right) \\
          R              & = -\frac{t}{C \ln (1 - q / C \mathcal{E})}       \\
                         & = \qty{464}{\Omega}
        \end{align*}

  \item

        \begin{align*}
          C \mathcal{E} (1 - e^{-t / R C}) & = 0.99 C \mathcal{E} \\
          1 - e^{-t / R C}                 & = 0.99               \\
          e^{-t / R C}                     & = 0.01               \\
          t                                & = -R C \ln 0.01      \\
                                           & = \qty{12.6}{ms}
        \end{align*}
\end{enumerate}

\subsubsection{26.51}

The heater draws \[I_\textrm{heater} = P_\textrm{header} / V = \qty{12.5}{A}\] and the various hair dryer settings draw

\begin{align*}
  I_{600}  & = \qty{5}{A}    \\
  I_{900}  & = \qty{7.5}{A}  \\
  I_{1200} & = \qty{10}{A}   \\
  I_{1500} & = \qty{12.5}{A}
\end{align*}

so the circuit breaker will trip if the hair dryer's $\qty{900}{W}$ mode is used.

\subsection{Problems}

\subsubsection{26.53}

\begin{enumerate}[(a)]
  \item $C \mathcal{E}^2 / 2$

  \item

        \begin{align*}
          E & = \int_0^\infty \frac{\mathcal{E}^2}{R} e^{-t / R C} \,dt \\
            & = \frac{\mathcal{E}^2}{R} [-R C e^{-e / R C}]_0^\infty    \\
            & = C \mathcal{E}^2
        \end{align*}

  \item

        \begin{align*}
          E & = \int_0^\infty \left( \frac{\mathcal{E}}{R} e^{-t / R C} \right)^2 R \,dt      \\
            & = \int_0^\infty \frac{\mathcal{E}^2}{R} e^{-2 t / R C} \,dt                     \\
            & = \frac{\mathcal{E}^2}{R} \left[ -\frac{R C}{2} e^{-2 t / R C} \right]_0^\infty \\
            & = \frac{1}{2} C \mathcal{E}^2
        \end{align*}

  \item Half is stored in each
\end{enumerate}

\subsubsection{26.55}

\begin{enumerate}[(a)]
  \item $I_1 = \qty{-2.29}{A}$

  \item $I_2 = \qty{3.71}{A}$

  \item $I_\textrm{R} = \qty{1.42}{A}$
\end{enumerate}

\subsubsection{26.57}

\begin{enumerate}[(a)]
  \item $\qty{0.222}{V}$

  \item $\qty{0.464}{A}$
\end{enumerate}

\subsubsection{26.63}

\begin{enumerate}[(a)]
  \item $\qty{109}{V}$, no
\end{enumerate}

\subsubsection{26.67}

\begin{enumerate}[(a)]
  \item The current through both branches is \[I = \mathcal{E} / R = \qty{4.00}{A}.\] The potential difference between points $a$ and $b$ is \[V_\textrm{ab} = 3 I - 6 I = -3 I = \qty{-12.0}{V}.\]

  \item $I = \qty{1.71}{A}$

  \item $R = \qty{4.20}{\Omega}$
\end{enumerate}

\subsubsection{26.69}

\begin{enumerate}[(a)]
  \item Modelling the device as a capacitor and resistor in parallel, the capacitor has capacitance \[C = K \epsilon_0 \frac{A}{d}\] and the resistor has resistance \[R = \rho \frac{d}{A}\] so the time constant is \[\tau = R C = \epsilon_0 \rho K\]

  \item The capacitor has capacitance \[C = K \epsilon_0 \frac{A}{d} = \qty{0.257}{nF}\] so its charge at $t = 0$ is \[Q_0 = C \mathcal{E} = \qty{1.28}{nC}\]

  \item The resistor has resistance \[R = \rho \frac{d}{A} = \qty{9.30e11}{\Omega}\] so the capacitor will have half of its original charge when

        \begin{align*}
          q                & = \frac{1}{2} Q_0      \\
          Q_0 e^{-t / R C} & = \frac{1}{2} Q_0      \\
          t                & = -R C \ln \frac{1}{2} \\
                           & = \qty{166}{s}
        \end{align*}

  \item $i = -\frac{\mathcal{E}}{R} e^{-t / R C} = \qty{-4.68}{pA}$
\end{enumerate}

\subsubsection{26.71}

\begin{enumerate}[(a)]
  \item $Q_0 = C \mathcal{E} = \epsilon_0 A \mathcal{E} / d = \qty{11.8}{pC}$

  \item The capacitor's new capacitance is \[C = K \epsilon_0 \frac{A}{d} = \qty{14.2}{pF}\] so the time constant is \[\tau = R C = \qty{142}{ps}.\] The ``new'' capacitor would have the same charge at

        \begin{align*}
          Q_0           & = C \mathcal{E} (1 - e^{-t / \tau})                      \\
          e^{-t / \tau} & = 1 - \frac{Q_0}{C \mathcal{E}}                          \\
          t             & = -\tau \ln \left( 1 - \frac{Q_0}{C \mathcal{E}} \right) \\
                        & = \qty{12.3}{ps}
        \end{align*}

        at which point the current would be

        \begin{align*}
          i & = \frac{\mathcal{E}}{R} e^{-t / \tau} \\
            & = \qty{0.917}{A}
        \end{align*}

  \item $E = \frac{Q_0^2}{2 C} = \qty{4.90}{pJ}$

  \item $E = \frac{1}{2} C \mathcal{E}^2 = \qty{710}{pJ}$

  \item

        \begin{align*}
          E & = \int_{t_0}^\infty \mathcal{E} I \,dt                         \\
            & = \int_{t_0}^\infty \frac{\mathcal{E}^2}{R} e^{-t / \tau} \,dt \\
            & = \frac{\mathcal{E}^2}{R} [-\tau e^{-t / \tau}]_{t_0}^\infty   \\
            & = \frac{\mathcal{E}^2 \tau}{R} e^{-t_0 / \tau}                 \\
            & = \qty{1300}{pJ}
        \end{align*}

  \item $E = \qty{590}{pJ}$
\end{enumerate}

\subsubsection{26.73}

\begin{enumerate}[(a)]
  \item The equivalent resistance of the two resistors and voltmeter (in $\unit{k \Omega}$) is \[R = 100 + \frac{1}{(1 / 200) + (1 / 50)} = \qty{140}{k \Omega}\] so the current through the circuit is \[I = \frac{V}{R} = \qty{2.86}{mA},\] the voltage drop from $a$ to $b$ is \[V_\textrm{ab} = I R = \qty{286}{V},\] and the voltage drop from $b$ to ground is \[V_\textrm{b} = V - V_\textrm{ab} = \qty{114}{V}\]

  \item The equivalent resistance of the two resistors and voltmeter (in $\unit{k \Omega}$) is \[R = 100 + \frac{1}{(1 / 200) + (1 / 5000)} = \qty{292}{k \Omega}\] so the current through the circuit is \[I = \frac{V}{R} = \qty{1.37}{mA},\] the voltage drop from $a$ to $b$ is \[V_\textrm{ab} = I R = \qty{137}{V},\] and the voltage drop from $b$ to ground is \[V_\textrm{b} = V - V_\textrm{ab} = \qty{263}{V}\]

  \item The equivalent resistance of the two resistors and voltmeter (in $\unit{k \Omega}$) is \[R = 100 + 200 = \qty{300}{k \Omega}\] so the current through the circuit is \[I = \frac{V}{R} = \qty{1.33}{mA},\] the voltage drop from $a$ to $b$ is \[V_\textrm{ab} = I R = \qty{133}{V},\] and the voltage drop from $b$ to ground is \[V_\textrm{b} = V - V_\textrm{ab} = \qty{267}{V}\]
\end{enumerate}

\subsubsection{26.75}

\begin{enumerate}[(a)]
  \item $V_\textrm{ab} = \qty{18.0}{V}$

  \item $a$

  \item The current through the switch is \[I = \frac{V}{R} = \qty{2.00}{A}\] so the potential difference between the battery and the $\qty{6.00}{\Omega}$ resistor is \[V_6 = I R = \qty{12.0}{V}\] meaning the potential difference between $b$ and ground is \[V_b = V - V_6 = \qty{6.00}{V}\]

  \item

        \begin{align*}
          Q_{3\textrm{,before}} & = \qty{54}{\mu C}  \\
          Q_{3\textrm{,after}}  & = \qty{18}{\mu C}  \\
          \Delta Q_3            & = \qty{-36}{\mu C} \\
          Q_{6\textrm{,before}} & = \qty{108}{\mu C} \\
          Q_{6\textrm{,after}}  & = \qty{72}{\mu C}  \\
          \Delta Q_6            & = \qty{-36}{\mu C}
        \end{align*}
\end{enumerate}

\subsubsection{26.77}

The resistance of the resistor is

\begin{align*}
  i & = \frac{\mathcal{E}}{R} e^{-0 / \tau} \\
  R & = \frac{\mathcal{E}}{i}               \\
    & = \qty{1.7}{M \Omega}
\end{align*}

so the capacitance of the capacitor is

\[C = \frac{\tau}{R} = \qty{3.1}{\mu F}\]

\subsubsection{26.81}

\begin{enumerate}[(a)]
  \item $V_\textrm{out} = 21 - 10 V_\textrm{in}$

  \item $\qty{1.35}{V}$

  \item ?
\end{enumerate}

\subsubsection{26.83}

The total resistance of the network to the right of points $c$ and $d$ is equal to $R_\textrm{T}$ and it is parallel with $R_2$ so the equivalent resistance of the whole network is

\begin{align*}
  R_\textrm{T}                                                       & = R_1 + \frac{1}{(1 / R_2) + (1 / R_T)} + R_1                     \\
  R_\textrm{T} \left( \frac{1}{R_2} + \frac{1}{R_\textrm{T}} \right) & = 1 + 2 R_1 \left( \frac{1}{R_2} + \frac{1}{R_\textrm{T}} \right) \\
  \frac{R_\textrm{T}}{R_2} + 1                                       & = 1 + 2 \frac{R_1}{R_2} + 2 \frac{R_1}{R_\textrm{T}}              \\
  R_\textrm{T}^2                                                     & = 2 R_1 R_\textrm{T} + 2 R_1 R_2                                  \\
  R_\textrm{T}^2 - 2 R_1 R_\textrm{T} - 2 R_1 R_2                    & = 0
\end{align*}

This is a quadratic in $R_\textrm{T}$ with solutions

\begin{align*}
  R_\textrm{T} & = \frac{2 R_1 \pm \sqrt{(-2 R_1)^2 - 4 (1) (-2 R_1 R_2)}}{2} \\
               & = \frac{2 R_1 \pm \sqrt{4 R_1^2 + 8 R_1 R_2}}{2}             \\
               & = R_1 \pm \sqrt{R_1^2 + 2 R_1 R_2}
\end{align*}

The minus value can be discarded because it would result in a negative resistance, leaving \[R_\textrm{T} = R_1 + \sqrt{R_1^2 + 2 R_1 R_2}\]

\section{Magnetic Field and Magnetic Forces}

\subsection{Magnetic Field}

\subsubsection{Example 27.1}

The force is directed in the negative $y$ direction and has magnitude

\begin{align*}
  F & = q v B \sin \theta \\
    & = \qty{4.8e-14}{N}
\end{align*}

\subsection{Magnetic Field Lines and Magnetic Flux}

\subsubsection{Example 27.2}

\begin{align*}
  \Phi_B        & = \int \mathbf{B} \cdot d \mathbf{A} \\
                & = B A \cos \theta                    \\
  \Rightarrow B & = \frac{\Phi_B}{A \cos \theta}       \\
                & = \qty{6.0}{T}
\end{align*}

\subsection{Motion of Charged Particles in a Magnetic Field}

\subsubsection{Example 27.3}

\begin{align*}
  f             & = \frac{|q| B}{2 \pi m} \\
  \Rightarrow B & = \frac{2 \pi f m}{|q|} \\
                & = \qty{0.0876}{T}
\end{align*}

\subsubsection{Example 27.4}

\begin{enumerate}[(a)]
  \item The force acts in the negative $y$ direction and has magnitude \[F = q v_z B = \qty{1.60e-14}{N}\] so the proton is accelerating in the negative $y$ direction with magnitude \[a = \frac{F}{m} = \qty{9.58e12}{m/s^2}\]

  \item \[R = \frac{m v}{|q| B} = \qty{4.18}{mm}\] \[\omega = \frac{|q| B}{m} = \qty{4.79e7}{rad/s}\] \[v_x T = v_x \frac{2 \pi}{\omega} = \qty{1.97}{cm}\]
\end{enumerate}

\subsection{Applications of Motion of Charged Particles}

\subsubsection{Example 27.5}

\begin{enumerate}[(a)]
  \item \[v = \sqrt{\frac{2 e V}{m}} = \qty{7.26e6}{m/s} = 0.024c\]

  \item \[\frac{E}{B} = v \Rightarrow B = \frac{E}{v} = \qty{0.83}{T}\]

  \item The beam will move downward
\end{enumerate}

\subsubsection{Example 27.6}

\begin{align*}
  R             & = \frac{m v}{|q| B} \\
  \Rightarrow B & = \frac{m v}{|q| R} \\
                & = \qty{0.0818}{T}
\end{align*}

\subsection{Magnetic Force on a Current-Carrying Conductor}

\begin{enumerate}[(a)]
  \item The magnetic force is directed upwards and has magnitude \[F = I l B \sin \theta = \qty{42.4}{N}\]

  \item Rotate the rod such that it's perpendicular to the magnetic field, i.e. the current is moving South-East at which point the magnitude of the magnetic force will be \[F = I l B = \qty{60.0}{N}\]
\end{enumerate}

\subsubsection{Example 27.8}

The total force experienced by the conductor is equal to the vector sum of the force experienced by each of its segments \[\mathbf{F} = \mathbf{F}_1 + \mathbf{F}_2 + \mathbf{F}_3.\]

The first segment is parallel to the magnetic field and thus experiences no magnetic force \[\mathbf{F}_1 = \mathbf{0}.\]

Each infinitesimal part of the second segment experiences a magnetic force directed radially outward. By symmetry the horizontal components of these forces cancel meaning the magnetic force experienced by the second segment is directed in the positive $y$ direction and has magnitude

\begin{align*}
  F_2 & = I \int_0^\pi R \,d \theta B \sin \theta \\
      & = I R B [-\cos \theta]_0^\pi              \\
      & = 2 I R B.
\end{align*}

The magnetic force experienced by the third segment is directed in the positive $y$ direction and has magnitude \[F_3 = I L B.\]

The total magnetic force experienced by the conductor is thus \[\mathbf{F} = \mathbf{F}_1 + \mathbf{F}_2 + \mathbf{F}_3 = \mathbf{0} + 2 I R B \hat{\mathbf{j}} + I L B \hat{\mathbf{j}} = I B (2 R + L) \hat{\mathbf{j}}.\]

\subsection{Force and Torque on a Current Loop}

\subsubsection{Example 27.9}

\[\mu = N I A = \qty{1.18}{A.m^2}\]

\[\tau = \mu B \sin \phi = \qty{1.41}{N.m}\]

\subsubsection{Example 27.10}

\begin{align*}
  \Delta U & = U_1 - U_0      \\
           & = (-\mu B) - (0) \\
           & = \qty{-1.41}{J}
\end{align*}

\subsection{The Direct-Current Motor}

\subsubsection{Example 27.11}

\begin{enumerate}[(a)]
  \item $V = \mathcal{E} + I r \Rightarrow \mathcal{E} = V - I r = \qty{112}{V}$

  \item $P_\textrm{in} = V I = \qty{480}{W}$

  \item $P_\textrm{resistor} = V I = I^2 R = \qty{32}{W}$

  \item $P_\textrm{motor} = P_\textrm{in} - P_\textrm{resistor} = \qty{448}{W}$

  \item $e = \frac{P_\textrm{motor}}{P_\textrm{in}} = 93.3\%$

  \item The back emf becomes zero, the current jumps to $I = V / R = \qty{60}{A}$ and the power dissipated by the resistor becomes $P = I^2 R = \qty{7200}{W}$
\end{enumerate}

\subsection{The Hall Effect}

\subsubsection{Example 27.12}

\begin{align*}
  n & = \frac{-J_x B_y}{E_z q}                 \\
    & = \frac{-I}{A} \frac{d}{V} \frac{B_y}{q} \\
    & = \qty{11.6}{m^{-3}}
\end{align*}

\subsection{Guided Practice}

\subsubsection{VP27.1.1}

\begin{enumerate}[(a)]
  \item $F = q v B \sin \theta = \qty{4.18e-14}{N}$

  \item The force is in the negative $z$ direction
\end{enumerate}

\subsubsection{VP27.1.2}

\begin{align*}
  F             & = q v B \sin \theta         \\
  \Rightarrow q & = \frac{F}{v B \sin \theta} \\
                & = \qty{2.91e-15}{C}
\end{align*}

\subsubsection{VP27.1.3}

\begin{enumerate}[(a)]
  \item

        \begin{align*}
          F             & = q v B \sin \theta         \\
          \Rightarrow B & = \frac{F}{q v \sin \theta} \\
                        & = \qty{0.0175}{T}
        \end{align*}

  \item The magnetic field is in the positive $z$ direction
\end{enumerate}

\subsubsection{VP27.1.4}

\begin{enumerate}[(a)]
  \item

        \begin{align*}
          F             & = q v B \sin \theta         \\
          \Rightarrow B & = \frac{F}{q v \sin \theta} \\
                        & = \qty{0.189}{T}
        \end{align*}

  \item The force is in the negative $y$ direction
\end{enumerate}

\subsubsection{VP27.6.1}

\begin{enumerate}[(a)]
  \item \[R = \frac{m v}{|q| B} = \qty{0.435}{mm}\]

  \item \[\omega = \frac{|q| B}{m} = \qty{2.87e7}{rad/s}\]

  \item \[f = \frac{\omega}{2 \pi} = \qty{4.57e6}{Hz}\]
\end{enumerate}

\subsubsection{VP27.6.2}

\begin{enumerate}[(a)]
  \item \[R = \frac{m v}{|q| B} = \qty{0.348}{mm}\]

  \item \[y = T v_y = \frac{2 \pi m v_y}{|q| B} = \qty{1.64}{mm}\]

  \item \[F = q v B = \qty{4.80e-16}{N}\]
\end{enumerate}

\subsubsection{VP27.6.3}

\begin{enumerate}[(a)]
  \item $v = E / B = \qty{8.00e5}{m/s}$

  \item Along the $y$ axis in the negative direction

  \item In the negative $x$ direction
\end{enumerate}

\subsubsection{VP27.6.4}

\begin{enumerate}[(a)]
  \item \[R = \frac{m v}{|q| B} = \qty{6.65}{cm}\]

  \item \[R = \frac{m v}{|q| B} = \qty{7.48}{cm}\]
\end{enumerate}

\subsubsection{VP27.7.1}

\begin{enumerate}[(a)]
  \item $F = I l B \sin \theta = \qty{2.44e-3}{N}$

  \item $+z$
\end{enumerate}

\subsubsection{VP27.7.2}

\begin{enumerate}[(a)]
  \item \[F = I l B \Rightarrow I = \frac{F}{l B} = \qty{1.60}{A}\]

  \item $+z$
\end{enumerate}

\subsubsection{VP27.7.3}

\begin{enumerate}[(a)]
  \item

        \begin{align*}
          \mathbf{F} & = I \mathbf{l} \times \mathbf{B}                                                                     \\
                     & = I \begin{vmatrix}
                             \hat{\mathbf{i}} & \hat{\mathbf{j}} & \hat{\mathbf{k}} \\
                             0.200            & -0.120           & 0                \\
                             0                & 0                & 0.0175
                           \end{vmatrix} \\
                     & = -0.00252 \hat{\mathbf{i}} - 0.00420 \hat{\mathbf{j}}
        \end{align*}

  \item $F = \qty{4.90e-3}{N}$
\end{enumerate}

\subsubsection{VP27.7.4}

\begin{align*}
  F                  & = I l B \sin \theta                 \\
  \Rightarrow \theta & = \arcsin \frac{F}{I l B}           \\
                     & = \ang{41.0}\textrm{ and }\ang{139}
\end{align*}

\subsection{Bridging Problem}

\begin{enumerate}[(a)]
  \item

        \begin{align*}
          \boldsymbol{\tau} & = \boldsymbol{\mu} \times \mathbf{B}                                                                \\
                            & = \begin{vmatrix}
                                  \hat{\mathbf{i}} & \hat{\mathbf{j}} & \hat{\mathbf{k}} \\
                                  -0.00445         & 0.00334          & 0                \\
                                  0.138            & 0.0345           & -0.0460
                                \end{vmatrix}  \\
                            & = \num{-1.54e-4} \hat{\mathbf{i}} - \num{2.05e-4} \hat{\mathbf{j}} - \num{6.14e-4} \hat{\mathbf{k}}
        \end{align*}

  \item

        \begin{align*}
          \Delta U & = U_1 - U_0                                                                  \\
                   & = -\boldsymbol{\mu}_1 \cdot \mathbf{B} + \boldsymbol{\mu}_0 \cdot \mathbf{B} \\
                   & = \num{-2.56e-4} - \num{4.99e-4}                                             \\
                   & = \qty{-7.55e-4}{J}
        \end{align*}

  \item

        \begin{align*}
          \Delta U             & = K_0 - K_1                                           \\
                               & = \frac{1}{2} I \omega_0^2 - \frac{1}{2} I \omega_1^2 \\
          \Rightarrow \omega_1 & = \sqrt{\frac{-2 \Delta U}{I}}                        \\
                               & = \qty{42.1}{rad/s}
        \end{align*}
\end{enumerate}

\subsection{Exercises}

\subsubsection{27.1}

\begin{enumerate}[(a)]
  \item $\mathbf{F} = q \mathbf{v} \times \mathbf{B} = (\qty{-6.68e-4}{N}) \hat{\mathbf{k}}$

  \item $\mathbf{F} = q \mathbf{v} \times \mathbf{B} = (\qty{6.68}{N}) \hat{\mathbf{i}} + (\qty{7.27e-4}{N}) \hat{\mathbf{j}}$
\end{enumerate}

\subsubsection{27.3}

\begin{enumerate}[(a)]
  \item Positive

  \item $F = q v B = \qty{0.0505}{N}$
\end{enumerate}

\subsubsection{27.5}

The velocity is in the negative $x$ direction with magnitude \[v = \frac{F}{q B} = \qty{172}{m/s}\]

\subsubsection{27.7}

\begin{enumerate}[(a)]
  \item

        \begin{align*}
          F_+                       & = e v_+ B \sin \theta \\
          \Rightarrow B \sin \theta & = \frac{F_+}{e v_+}
        \end{align*}

        \begin{align*}
          F_-                       & = e v_- B \sin (90 - \theta) \\
                                    & = e v_- B \cos \theta        \\
          \Rightarrow B \cos \theta & = \frac{F_-}{e v_-}
        \end{align*}

        \begin{align*}
          \tan \theta        & = \frac{F_+ v_-}{F_- v_+}         \\
          \Rightarrow \theta & = \arctan \frac{F_+ v_-}{F_- v_+} \\
                             & = \ang{40.0}
        \end{align*}

        \begin{align*}
          F_+           & = e v_+ B \sin \theta           \\
          \Rightarrow B & = \frac{F_+}{e v_+ \sin \theta} \\
                        & = \qty{1.46}{T}
        \end{align*}

        The magnetic field is in the $x-z$ plane, directed $\ang{40.0}$ below the positive $x$ axis and has magnitude $\qty{1.46}{T}$

  \item The force is in the $x-z$ plane and is directed $\ang{40.0}$ from the negative $z$ axis towards the negative $x$ axis and has magnitude \[F = q v B = \qty{7.48e-16}{N}\]
\end{enumerate}

\subsubsection{27.9}

\begin{enumerate}[(a)]
  \item $\Phi_B = A B = \qty{3.05}{mWb}$

  \item $\Phi_B = A B \sin \theta = \qty{1.83}{mWb}$

  \item $0$
\end{enumerate}

\subsubsection{27.11}

If the bottle was a closed surface the flux would be 0, however it has an opening. Some of the magnetic field goes through the side of the bottle (which contributes negative flux) then passes through the opening such that it's not cancelled out. The total flux is thus the negative of however much flux makes it through the opening \[\Phi_B = -A B \sin \theta = \qty{-0.78}{mWb}\]

\subsubsection{27.13}

\begin{enumerate}[(a)]
  \item $0$

  \item $\qty{-0.0115}{Wb}$

  \item $\qty{0.0115}{Wb}$

  \item $0$
\end{enumerate}

\subsubsection{27.15}

\begin{enumerate}[(a)]
  \item The magnetic field points out of the page and has magnitude \[B = \frac{m v}{|q| R} = \qty{0.294}{T}\]

  \item $\frac{T}{2} = \frac{\pi m}{|q| B} = \qty{0.112}{\mu s}$
\end{enumerate}

\subsubsection{27.17}

At the bottom of the shaft the ball's velocity is

\begin{align*}
  v^2 & = v_0^2 + 2 a (y - y_0) \\
  v   & = \sqrt{2 a y}          \\
      & = \qty{49.5}{m/s}.
\end{align*}

Its net charge is $-(\num{4.00e8}) e = \qty{-6.40e-11}{C}$ so the magnetic force it experiences is directed to the South and has magnitude \[F = q v B = \qty{7.92e-10}{N}.\]

\subsubsection{27.19}

\begin{enumerate}[(a)]
  \item The particles have negative charge and are moving with velocity \[v = \frac{B |q| R}{m} = \qty{2.84e6}{m/s}\]
\end{enumerate}

\subsubsection{27.21}

The electric potential energy is converted into kinetic energy so the particle's speed is

\begin{align*}
  K                 & = U                      \\
  \frac{1}{2} m v^2 & = q V                    \\
  v                 & = \sqrt{\frac{2 q V}{m}} \\
                    & = \qty{2.65e7}{m/s}.
\end{align*}

The magnitude of the magnetic field is thus

\begin{align*}
  B & = \frac{m v}{|q| R} \\
    & = \qty{0.838}{mT}.
\end{align*}

\subsubsection{27.23}

\begin{enumerate}[(a)]
  \item The electric field is directed in the $+x$ direction with magnitude \[E = v B = \qty{7.90e3}{N/C}\]

  \item The electric field is directed in the $+x$ direction with magnitude \[E = v B = \qty{7.90e3}{N/C}\]
\end{enumerate}

\subsubsection{27.25}

The electric field between the plates is \[E = V / d = \qty{1.83e4}{N/C}.\] The speed of the alpha particles is

\begin{align*}
  K                 & = U                      \\
  \frac{1}{2} m v^2 & = 2 e V                  \\
  v                 & = \sqrt{\frac{4 e V}{m}} \\
                    & = \qty{4.11e5}{m/s}.
\end{align*}

In order for them to emerge undeflected the magnetic field would need to be directed out of the page and have magnitude \[B = \frac{E}{v} = \qty{44.5}{mT}.\]

\subsubsection{27.27}

\begin{enumerate}[(a)]
  \item \[v = \frac{E}{B} = \qty{4.92e3}{m/s}\]

  \item \[m = \frac{B |q| R}{v} = \qty{9.96e-26}{kg}\]
\end{enumerate}

\subsubsection{27.29}

The magnetic force experienced by the horizontal segments of the wire is directed downwards with magnitude \[F = I l B = \qty{0.648}{N}.\] The magnetic force experienced by the vertical segment of the wire is directed to the right with magnitude \[F = I l B = \qty{0.324}{N}.\] The total force experienced by the wire is thus \[\mathbf{F} = 0.324 \hat{\mathbf{i}} - 0.648 \hat{\mathbf{j}}.\]

\subsubsection{27.31}

\begin{enumerate}[(a)]
  \item When current is flowing through the bar it experiences an upwards force. If the current is too great the bar will move upwards and break the circuit.

        The current in the circuit is $I = V / R$ so the upwards force experienced by the bar is \[F = I l B = V l B / R = 0.00900 V.\] In order for the bar to move upwards this must be greater than the bar's weight \[0.00900 V = m g \Rightarrow V = \frac{m g}{0.00900} = \qty{817}{V}.\]

  \item The bar's acceleration is

        \begin{align*}
          a & = \frac{F}{m}                     \\
            & = \frac{I l B - m g}{m}           \\
            & = \frac{\frac{V}{R} l B - m g}{m} \\
            & = \qty{113}{m/s^2}
        \end{align*}
\end{enumerate}

\subsubsection{27.33}

\begin{enumerate}[(a)]
  \item $a$

  \item The maximum current is $I = V / R = \qty{35}{A}$ which results in an upwards force on the bar of $F = I l B = \qty{31.5}{N}$ which can support a mass of $m = F / g = \qty{3.21}{kg}$.
\end{enumerate}

\subsubsection{27.35}

\begin{enumerate}[(a)]
  \item $\mu = N I A \Rightarrow N = \mu / I A = 55$

  \item Counterclockwise

  \item The torque is directed upwards and has magnitude $\tau = \mu B = \qty{8.73}{N.m}$
\end{enumerate}

\subsubsection{27.37}

\begin{enumerate}[(a)]
  \item $ab$ no force, $bc$ no force, $cd$ out of the page, $da$ no force

  \item $F_{cd} = I l B = \qty{1.20}{N}$

  \item The torque is directed downwards with magnitude $\tau = r F = \qty{0.420}{N.m}$
\end{enumerate}

\subsubsection{27.39}

\begin{enumerate}[(a)]
  \item $A_2$ because the sides of length $\qty{0.500}{m}$ are parallel to the magnetic field and thus experience no magnetic force

  \item The force experienced by the sides of length $\qty{1.00}{m}$ is $F = I l B = \qty{6.00}{N}$ so the magnitude of the total torque is $\tau = 2 r F = \qty{3.00}{N.m}$. The moment of inertia of the coil around the $A_2$ axis is

        \begin{align*}
          I & = 2 \lambda (0.250)^2 + 4 \int_0^{0.250} \lambda y^2 \,dy                   \\
            & = \left( 0.125 + 4 \left[ \frac{1}{3} y^3 \right]_0^{0.250} \right) \lambda \\
            & = 0.146 \lambda                                                             \\
            & = 0.0103
        \end{align*}

        so the initial angular acceleration is \[\alpha = \frac{\tau}{I} = \qty{291}{rad/s^2}\]
\end{enumerate}

\subsubsection{27.43}

\begin{enumerate}[(a)]
  \item $I = V / R = \qty{1.13}{A}$

  \item $I = \qty{3.69}{A}$

  \item $V - \mathcal{E} = I R \Rightarrow \mathcal{E} = V - I R = \qty{98.2}{V}$

  \item $P = V I - I^2 R = \qty{362}{W}$
\end{enumerate}

\subsubsection{27.45}

\begin{enumerate}[(a)]
  \item

        \begin{align*}
          J = \frac{I}{A} & = n q v_d         \\
          v_d             & = \frac{I}{A n q} \\
                          & = \qty{4.7}{mm/s}
        \end{align*}

  \item The electric field is in the $+z$ direction and has magnitude \[E = \frac{B J}{n q} = \qty{4.5e-3}{N/C}\]

  \item $\mathcal{E} = E z_1 = \qty{53}{\mu V}$
\end{enumerate}

\subsection{Problems}

\subsubsection{27.47}

\begin{enumerate}[(a)]
  \item The particle accelerates in the $+z$ direction due to the electric field. This is parallel with the magnetic field so it doesn't affect the magnetic force. The acceleration has magnitude \[a_z = \frac{F_z}{m} = \frac{q E}{m} = \qty{450}{m/s^2}\] so at $t = \qty{0.0200}{s}$ its $z$ coordinate will be \[z = \frac{1}{2} a_z t^2 = \qty{0.0900}{m}.\]

        The particle moves in a circle in the $x-y$ plane with radius \[R = \frac{m v}{|q| B} = \qty{1.00}{m}\] and angular speed \[\omega = \frac{|q| B}{m} = \qty{30.0}{rad/s}\] so at $t = \qty{0.0200}{s}$ its $x$ coordinate is \[x = R \sin (\omega t) = \qty{0.565}{m}\] and its $y$ coordinate is \[y = R \cos (\omega t) = \qty{0.825}{m}\]

  \item $v = \sqrt{v_x^2 + v_z^2} = \qty{31.3}{m/s}$
\end{enumerate}

\subsubsection{27.49}

\begin{enumerate}[(a)]
  \item

        \begin{align*}
          K     & = U                                                 \\
          m v^2 & = \frac{1}{4 \pi \epsilon_0} \frac{e^2}{r}          \\
          v     & = \sqrt{\frac{1}{4 \pi \epsilon_0} \frac{e^2}{m r}} \\
                & = \qty{8.3e6}{m/s}
        \end{align*}

  \item \[B = \frac{m v}{|q| R} = \qty{0.14}{T}\]
\end{enumerate}

\subsubsection{27.51}

The particle's velocity is

\begin{align*}
  \frac{1}{2} m v^2 & = q \Delta V                     \\
  q                 & = \sqrt{\frac{2 q \Delta V}{m}}.
\end{align*}

Using the equation for the radius of a particle in a uniform magnetic field we find

\begin{align*}
  R   & = \frac{m v}{q B}                                       \\
      & = \frac{m}{q B} \sqrt{\frac{2 q \Delta V}{m}}           \\
  R^2 & = \left( \frac{m}{q B} \right)^2 \frac{2 q \Delta V}{m} \\
      & = \frac{2 m}{q B^2} \Delta V
\end{align*}

and we know the slope of that line so

\begin{align*}
  (\num{1.04e-6}) & = \frac{2 m}{q B^2}             \\
  \frac{q}{m}     & = \frac{2}{(\num{1.04e-6}) B^2} \\
                  & = \qty{4.81e7}{C/kg}
\end{align*}

\subsubsection{27.53}

The velocity of the particles is \[v = \frac{E}{B} = \qty{2.76e4}{m/s}.\] The distance between where the particles are emitted and the line for $^{82}$Kr is \[2 R = 2 \frac{m v}{|q| B} = \qty{6.89}{cm}.\] The distance to the line for $^{84}$Kr is \[2 R = \qty{7.05}{cm}\] which is an additional $\qty{0.16}{cm}$. Finally, the distance to the line for $^{86}$Kr is \[2 R = \qty{7.22}{cm}\] which is an additional $\qty{0.17}{cm}$.

\subsubsection{27.55}

The current should flow from right to left in the diagram such that the resulting magnetic force is directed into the incline. The wire experiences three forces: the magnetic force, the normal force, and its weight. Decomposing these forces into their components parallel to the incline we find the equation of motion:

\begin{align*}
  0 & = M g \sin \theta - I L B \cos \theta \\
  I & = \frac{M g}{L B} \tan \theta
\end{align*}

\subsubsection{27.57}

\begin{enumerate}[(a)]
  \item \[B = \frac{m v}{|q| R} = \qty{8.46}{mT}\]

  \item \[2 R = 2 \frac{m v}{|q| B} = \qty{27.1}{cm}\]

  \item $\qty{2.1}{cm}$, yes
\end{enumerate}

\subsubsection{27.59}

\begin{enumerate}[(a)]
  \item $F = I L B$ directed towards the open end of the device

  \item \[v^2 = 2 a x \Rightarrow x = \frac{m v^2}{2 I L B}\]

  \item $x = \qty{1.96e6}{m}$
\end{enumerate}

\subsubsection{27.61}

\begin{enumerate}[(a)]
  \item The force will be directed toward the origin and have magnitude

        \begin{align*}
          F & = \int_{-\pi / 4}^{\pi / 4} I B R \cos \theta \,d\theta \\
            & = I B R [\sin \theta]_{-\pi / 4}^{\pi / 4}              \\
            & = \qty{1.13}{N}
        \end{align*}

  \item The force will be directed in the positive $z$ direction and have magnitude

        \begin{align*}
          F & = \int_0^{\pi / 2} I B R \sin \theta \,d\theta \\
            & = I B R [-\cos \theta]_0^{\pi / 2}             \\
            & = \qty{0.800}{N}
        \end{align*}
\end{enumerate}

\subsubsection{27.63}

\begin{enumerate}[(a)]
  \item The mass of the copper bar is \[m = V \rho = \pi \left( \frac{d}{2} \right)^2 L \rho = \frac{1}{4} \pi d^2 L \rho\] so its weight is \[W = m g = \frac{1}{4} \pi d^2 L \rho g.\] It experiences an upwards magnetic force of \[F = \frac{I L B}{\sqrt{2}}.\] Equating these two forces and solving for $I$ finds \[\frac{1}{4} \pi d^2 L \rho g = \frac{I L B}{\sqrt{2}} \Rightarrow I = \frac{\sqrt{2} \pi d^2 \rho g}{4 B} = \qty{4.8e4}{A}.\]

  \item Not feasible

  \item \[B = \frac{\pi d^2 \rho g}{4 I} = \qty{19}{G}\]

  \item The weight of the bar is \[m g = V \rho g = \pi \left( \frac{d}{2} \right)^2 L \rho g = \qty{0.17}{N}\] and the total weight it can support is \[mg = I L B = \qty{1.0}{N}\] so we could levitate and extra $\qty{0.83}{N}$ or $\qty{85}{g}$.
\end{enumerate}

\subsubsection{27.65}

The magnetic field should be directed in the $+y$ direction in order to generate an outwards torque in the bottom wire segment, the magnitude of which is \[R F \cos \theta = R I L B \cos \theta = 0.0394 B \cos \theta \,\unit{N.m}.\] The torque generated by the weight of the loop is in the opposite direction and has magnitude \[\frac{R}{2} m g \sin \theta = 0.00165 \sin \theta \,\unit{N.m}.\] In equilibrium at $\ang{30.0}$ these torques are equal so

\begin{align*}
  0.0394 B \cos \theta & = 0.00165 \sin \theta                \\
  B                    & = \frac{0.00165}{0.0394} \tan \theta \\
                       & = \qty{24}{mT}
\end{align*}

\subsubsection{27.67}

\begin{enumerate}[(a)]
  \item

        \begin{align*}
          F_{PQ} & = 0                                                            \\
          F_{PR} & = I L B = \qty{12.0}{N} \textrm{, into the page}               \\
          F_{QR} & = I L B \sin \theta = \qty{12.0}{N} \textrm{, out of the page}
        \end{align*}

  \item 0

  \item

        \begin{align*}
          \tau_{PQ} & = 0                                                              \\
          \tau_{PR} & = 0                                                              \\
          \tau_{QR} & = \int_0^1 r F \,dt                                              \\
                    & = \int_0^1 0.600 (1 - t) I L B \sin \theta \,dt                  \\
                    & = 0.600 I L B \sin \theta \left[ t - \frac{1}{2} t^2 \right]_0^1 \\
                    & = 0.300 I L B \sin \theta                                        \\
                    & = \qty{3.60}{N.m}
        \end{align*}

  \item $\tau = I A B \sin \phi = \qty{3.60}{N.m}$, yes

  \item Out
\end{enumerate}

\subsubsection{27.69}

By symmetry the radial components of the magnetic forces cancel, leaving only the $y$ components. The total magnetic force is directed in the negative $y$ direction and has magnitude

\begin{align*}
  F & = N \int_0^{2 \pi} I R B \sin \phi \,d\theta \\
    & = 2 \pi N I R B \sin \phi                    \\
    & = \qty{0.444}{N}
\end{align*}

\subsubsection{27.71}

\begin{enumerate}[(a)]
  \setcounter{enumi}{1}
  \item

        \begin{align*}
          \mathbf{F}_{(0, 0) \rightarrow (0, L)} & = \int d \mathbf{F}                                                      \\
                                                 & = \int I \,d \mathbf{l} \times \mathbf{B}                                \\
                                                 & = \frac{I B_0}{L} \int_0^L dy \hat{\mathbf{j}} \times y \hat{\mathbf{k}} \\
                                                 & = \frac{I B_0}{L} \int_0^L y \,dy \hat{\mathbf{i}}                       \\
                                                 & = \frac{1}{2} I B_0 L \hat{\mathbf{i}}                                   \\
          \mathbf{F}_{(0, L) \rightarrow (L, L)} & = \frac{I B_0}{L} \int_0^L dx \hat{\mathbf{i}} \times L \hat{\mathbf{k}} \\
                                                 & = -I B_0 L \hat{\mathbf{j}}                                              \\
          \mathbf{F}_{(L, L) \rightarrow (L, 0)} & = \frac{I B_0}{L} \int_L^0 dy \hat{\mathbf{j}} \times y \hat{\mathbf{k}} \\
                                                 & = -\frac{1}{2} I B_0 L \hat{\mathbf{i}}                                  \\
          \mathbf{F}_{(L, 0) \rightarrow (0, 0)} & = \mathbf{0}
        \end{align*}

  \item $\mathbf{F} = -I B_0 L \hat{\mathbf{j}}$
\end{enumerate}

\subsubsection{27.73}

\begin{enumerate}[(a)]
  \item The magnitude of the wire's magnetic dipole moment is \[\mu = N I A = 2 N I R W\] so the torque from the wire is \[\tau_\textrm{wire} = \mu B \sin \theta = 2 N I R W B \sin \theta\] where $\theta$ is the cylinder's angle from the position where $h = 0$.

        The torque from the weight is \[\tau_\textrm{weight} = M g R \sin \theta.\] If $h > 0$ then $\sin \theta \ne 0$ and \[2 N I R W B \sin \theta = M g R \sin \theta \Rightarrow I = \frac{M g}{2 N W B}.\]

  \item $h_\textrm{top} = R (1 + \pi / 2)$

  \item $\tau = M g (\sigma - 1) \sqrt{h (2 R - h)}$

  \item $\tau = M g R \left[ \sigma \cos \left( \frac{h}{R} - 1 \right) - 1 \right]$

  \item

        \begin{align*}
          U(h) & = M g h + \mu B \cos \theta - \mu B                      \\
               & = M g h + 2 N I R W B \left( \frac{R - h}{R} - 1 \right) \\
               & = M g h - \sigma M g h                                   \\
               & = M g h (1 - \sigma)
        \end{align*}

  \item The gravitational potential energy is \[U_\textrm{gravity} = M g h\] and the magnetic dipole potential energy is

        \begin{align*}
          U_\textrm{magnetic} & = \mu B \cos \theta                                         \\
                              & = \mu B \cos \left( \frac{h}{R} - 1 + \frac{\pi}{2} \right) \\
                              & = -\mu B \sin \left( \frac{h}{R} - 1 \right)                \\
                              & = -\sigma M g R \sin \left( \frac{h}{R} - 1 \right)
        \end{align*}

        At $h = R$ the potential energy functions for $0 \le h \le R$ and $R \le h \le h_\textrm{top}$ must be equal so

        \begin{align*}
          M g R (1 - \sigma) & = k + M g R - \mu B \sin \left( \frac{R}{R} - 1 \right) \\
          \Rightarrow k      & = -M g R \sigma.
        \end{align*}

        Therefore the potential energy function for $R \le h \le h_\textrm{top}$ is

        \begin{align*}
          U(h) & = M g h - \sigma M g R \sin \left( \frac{h}{R} - 1 \right) - M g R \sigma                             \\
               & = M g R \left\{ \frac{h}{R} - \sigma \left[ \sin \left( \frac{h}{R} - 1 \right) + 1 \right] \right\}.
        \end{align*}

  \item The mass will experience no net force and thus be stationary when the derivative of the potential energy is 0. The derivative of the potential energy function for $0 \le h \le R$ is \[\frac{d U(h)}{d h} = M g (1 - \sigma)\] which can't equal $0$, so the stationary point is at $h > R$. The derivative of the potential energy function for $R \le h \le h_\textrm{top}$ is

        \begin{align*}
          \frac{d U(h)}{d h} & = M g R \left\{ \frac{1}{R} - \sigma \left[ \frac{1}{R} \cos \left( \frac{h}{R} - 1 \right) \right] \right\} \\
                             & = M g \left[ 1 - \sigma \cos \left( \frac{h}{R} - 1 \right) \right]
        \end{align*}

        and this will equal $0$ when

        \begin{align*}
          \sigma \cos \left( \frac{h}{R} - 1 \right) & = 1                                                            \\
          h                                          & = R \left[ \arccos \left( \frac{1}{\sigma} \right) + 1 \right]
        \end{align*}

  \item

        \begin{align*}
          U(h_\textrm{top})                                                                                                            & > 0                                                                                            \\
          M g R \left\{ \frac{h_\textrm{top}}{R} - \sigma \left[ \sin \left( \frac{h_\textrm{top}}{R} - 1 \right) + 1 \right] \right\} & > 0                                                                                            \\
          \sigma                                                                                                                       & < \frac{h_\textrm{top}}{R \left[ \sin \left( \frac{h_\textrm{top}}{R} - 1 \right) + 1 \right]} \\
                                                                                                                                       & < \frac{1}{2} \left( 1 + \frac{\pi}{2} \right)
        \end{align*}
\end{enumerate}

\subsubsection{27.75}

\begin{enumerate}[(a)]
  \item $\boldsymbol{\mu} = -I A \hat{\mathbf{k}}$

  \item $B_x = \frac{3 D}{I A}$, $B_y = \frac{4 D}{I A}$, $B_z = -\frac{12 D}{I A}$
\end{enumerate}

\subsubsection{27.77}

\begin{enumerate}[(a)]
  \setcounter{enumi}{1}
  \item \[\frac{2 B^2 e}{m} = \num{2e8} \Rightarrow m = \frac{B^2 e}{\num{1e8}} = \qty{1.85e-28}{kg}\]

  \item $V = \qty{1.20}{kV}$

  \item

        \begin{align*}
          K                 & = e V                    \\
          \frac{1}{2} m v^2 & = e V                    \\
          v                 & = \sqrt{\frac{2 e V}{m}} \\
                            & = \qty{8.32e5}{m/s}
        \end{align*}
\end{enumerate}

\subsubsection{27.79}

\begin{enumerate}[(a)]
  \item $r = R \sin \theta$

  \item

        \begin{align*}
          dI & = \sigma v \,dW                           \\
             & = \sigma \omega R \sin \theta R \,d\theta \\
             & = \sigma \omega R^2 \sin \theta \,d\theta
        \end{align*}

  \item

        \begin{align*}
          d\mu & = A \,dI                                                        \\
               & = \pi (R \sin \theta)^2 \sigma \omega R^2 \sin \theta \,d\theta \\
               & = \pi \sigma \omega R^4 \sin^3 \theta \,d\theta
        \end{align*}

  \item

        \begin{align*}
          \boldsymbol{\mu} & = \int_0^\pi \pi \sigma \omega R^4 \sin^3 \theta \,d\theta \hat{\mathbf{k}} \\
                           & = \frac{4}{3} \pi \frac{Q}{4 \pi R^2} \omega R^4 \hat{\mathbf{k}}           \\
                           & = \frac{1}{3} \omega Q R^2 \hat{\mathbf{k}}
        \end{align*}

  \item \[\tau = \frac{1}{3} \omega Q R^2 B (-\cos \alpha \hat{\mathbf{i}} + \sin \alpha \hat{\mathbf{j}})\]
\end{enumerate}

\subsubsection{27.81}

\begin{enumerate}[(a)]
  \item \[R = \frac{m v}{|q| B} = \qty{5.14}{m}\]

  \item The particle exits the field when

        \begin{align*}
          R \sin \theta & = 0.250                      \\
          \theta        & = \arcsin \frac{0.250}{5.14} \\
                        & = \qty{0.0487}{rad}
        \end{align*}

        which occurs when

        \begin{align*}
          \omega t_1 & = \theta                 \\
          t_1        & = \frac{m \theta}{|q| B} \\
                     & = \qty{1.73}{\mu s}
        \end{align*}

  \item $\Delta x_1 = R (1 - \cos \theta) = \qty{6.09}{mm}$

  \item $\Delta x = \Delta x_1 + (D - d) \cos \theta = \qty{3.05}{cm}$
\end{enumerate}

\subsubsection{27.83}

c

\subsubsection{27.85}

The resistance of the nerve is \[R = \rho \frac{L}{A} = \rho \frac{L}{\pi (d / 2)^2} = \qty{340}{\Omega}\] so the current in the nerve is \[I = \frac{V}{R} = \qty{294}{\mu A}\] and the maximum magnetic force it can experience is roughly \[F_\textrm{max} = I L B = \qty{5.88e-7}{N}\] which is closest to option a.

\section{Sources of Magnetic Field}

\subsection{Magnetic Field of a Moving Charge}

\subsubsection{Example 28.1}

The upper proton experiences an electric force of \[\mathbf{F_E} = \ke \frac{q^2}{r^2} \hat{\mathbf{j}}\] and a magnetic force of \[\mathbf{F_B} = q \mathbf{v} \times \mathbf{B} = \frac{\mu_0}{4 \pi} \frac{q^2 v^2}{r^2} \hat{\mathbf{j}}\] so \[\frac{F_B}{F_E} = \epsilon_0 \mu_0 v^2 = \frac{v^2}{c^2}\]

\subsection{Magnetic Field of a Current Element}

\subsubsection{Example 28.2}

\begin{enumerate}[(a)]
  \item

        \begin{align*}
          \mathbf{B} & = \frac{\mu_0}{4 \pi} \frac{I \,dl (-\hat{\mathbf{i}}) \times \hat{\mathbf{j}}}{r^2} \\
                     & = -(\qty{8.7e-8}{T}) \hat{\mathbf{k}}
        \end{align*}

  \item

        \begin{align*}
          \mathbf{B} & = -\frac{\mu_0}{4 \pi} \frac{I \,dl \sin \theta}{r^2} \hat{\mathbf{k}} \\
                     & = -(\qty{4.3e-8}{T}) \hat{\mathbf{k}}
        \end{align*}
\end{enumerate}

\subsection{Magnetic Field of a Straight Current-Carrying Conductor}

\subsubsection{Example 28.3}

\begin{align*}
  B & = \frac{\mu_0 I}{2 \pi r} \\
  r & = \frac{\mu_0 I}{2 \pi B} \\
    & = \qty{4.0}{mm}
\end{align*}

\subsubsection{Example 28.4}

\begin{enumerate}[(a)]
  \item

        \[\mathbf{B}_{P_1} = \frac{\mu_0 I}{2 \pi} \left( \frac{1}{4 d} - \frac{1}{2 d} \right) \hat{\mathbf{j}} = -\frac{\mu_0 I}{8 \pi d} \hat{\mathbf{j}}\]

        \[\mathbf{B}_{P_2} = \frac{\mu_0 I}{\pi d} \hat{\mathbf{j}}\]

        \[\mathbf{B}_{P_3} = \frac{\mu_0 I}{2 \pi} \left( \frac{1}{3 d} - \frac{1}{d} \right) \hat{\mathbf{j}} = -\frac{\mu_0 I}{3 \pi d} \hat{\mathbf{j}}\]

  \item \[\mathbf{B} = \frac{\mu_0 I}{2 \pi} \left( \frac{1}{x + d} - \frac{1}{x - d} \right) \hat{\mathbf{j}} = -\frac{\mu_0 I d}{\pi (x^2 - d^2)} \hat{\mathbf{j}}\]
\end{enumerate}

\subsection{Force Between Parallel Conductors}

\subsubsection{Example 28.5}

\[\frac{F}{L} = \frac{\mu_0 I^2}{2 \pi r} = \qty{1.0e4}{N/m}\]

\subsection{Magnetic Field of a Circular Current Loop}

\subsubsection{Example 28.6}

\begin{enumerate}[(a)]
  \item \[B = \frac{\mu_0 I a^2}{2 \left( x^2 + a^2 \right)^{3/2}} = \qty{1.1e-4}{T}\]

  \item

        \begin{align*}
          \frac{\mu_0 I a^2}{2 \left( x^2 + a^2 \right)^{3/2}} & = \frac{1}{8} \frac{\mu_0 I a^2}{2 \left( (0)^2 + a^2 \right)^{3/2}} \\
          \left( x^2 + a^2 \right)^{3/2}                       & = 8 a^3                                                              \\
          x^2 + a^2                                            & = 4 a^2                                                              \\
          x                                                    & = \pm \sqrt{3} a                                                     \\
                                                               & = \pm \qty{1.04}{m}
        \end{align*}
\end{enumerate}

\setcounter{subsection}{6}
\subsection{Applications of Ampere's Law}

\subsubsection{Example 28.7}

\begin{align*}
  \oint \mathbf{B} \cdot d\mathbf{l} & = \mu_0 I                 \\
  2 \pi r B                          & = \mu_0 I                 \\
  B                                  & = \frac{\mu_0 I}{2 \pi r}
\end{align*}

\subsubsection{Example 28.8}

For points inside the conductor

\begin{align*}
  \oint \mathbf{B} \cdot d\mathbf{l} & = \mu_0 I_\textrm{encl}                \\
  B (2 \pi r)                        & = \mu_0 I \left( \frac{r}{R} \right)^2 \\
  B                                  & = \frac{\mu_0 I r}{2 \pi R^2}
\end{align*}

and for points outside the conductor

\[B = \frac{\mu_0 I}{2 \pi r}\]

\subsubsection{Example 28.9}

\begin{align*}
  \oint \mathbf{B} \cdot d\mathbf{l} & = \mu_0 I_\textrm{encl} \\
  B L                                & = \mu_0 I n L           \\
  B                                  & = \mu_0 I n
\end{align*}

\subsubsection{Example 28.10}

By the right-hand rule the magnetic field must be directed circumferentially and by symmetry it must be radially symmetrical. The hole of the solenoid contains no current so by Ampere's law the magnitude of the magnetic field must be $0$. Outside the solenoid the net current is $0$ so the magntiude of the magnetic field must also be $0$. Finally, using Ampere's law with an integration path around the radial axis of the solenoid finds

\begin{align*}
  \oint \mathbf{B} \cdot d\mathbf{l} & = \mu_0 I_\textrm{encl}     \\
  B (2 \pi r)                        & = \mu_0 N I                 \\
  B                                  & = \frac{\mu_0 N I}{2 \pi r}
\end{align*}

\subsection{Magnetic Materials}

\subsubsection{Example 28.12}

\begin{enumerate}[(a)]
  \item \[M = \frac{\mu_\textrm{total}}{V} \Rightarrow \mu_\textrm{total} = M V = \qty{6}{A.m^2}\]

  \item \[B = \frac{\mu_0 \mu}{2 \pi \left( x^2 + a^2 \right)^{3/2}} = \qty{1e-3}{T} = \qty{1}{G}\]
\end{enumerate}

\subsection{Guided Practice}

\subsubsection{VP28.2.1}

\[B = \frac{\mu_0}{4 \pi} \frac{q \mathbf{v} \times \hat{\mathbf{r}}}{r^2}\]

\begin{enumerate}[(a)]
  \item $B = (\qty{3.20e-15}{T}) \hat{\mathbf{k}}$

  \item $B = -(\qty{8.00e-16}{T}) \hat{\mathbf{j}}$

  \item $B = 0$

  \item $B = -(\qty{1.13e-15}{T}) \hat{\mathbf{k}}$
\end{enumerate}

\subsubsection{VP28.2.2}

\begin{enumerate}[(a)]
  \item

        \begin{align*}
          F & = -q \mathbf{v}_\textrm{electron} \times \mathbf{B}_\textrm{proton}                                                                                                              \\
            & = -q \mathbf{v}_\textrm{electron} \times \left( \frac{\mu_0}{4 \pi} \frac{q \mathbf{v}_\textrm{proton} \times \hat{\mathbf{r}}_\textrm{electron}}{r_\textrm{electron}^2} \right) \\
            & = -\frac{\mu_0 q^2}{4 \pi r_\textrm{electron}^2} \mathbf{v}_\textrm{electron} \times \left( -v_\textrm{proton} \hat{\mathbf{k}} \right)                                          \\
            & = -\frac{\mu_0 q^2}{4 \pi r_\textrm{electron}^2} v_\textrm{electron} v_\textrm{proton} \hat{\mathbf{i}}                                                                          \\
            & = -(\qty{1.13e-28}{N}) \hat{\mathbf{i}}
        \end{align*}

  \item By Newton's third law the force is $(\qty{1.13e-28}{N}) \hat{\mathbf{i}}$
\end{enumerate}

\subsubsection{VP28.2.3}

\begin{enumerate}[(a)]
  \item \[\hat{\mathbf{r}} = \frac{\mathbf{r}}{r} = \frac{3 \hat{\mathbf{i}} + 4 \hat{\mathbf{j}}}{\sqrt{3^2 + 4^2}} = (\qty{0.600}{m}) \hat{\mathbf{i}} + (\qty{0.800}{m}) \hat{\mathbf{j}}\]

  \item

        \[d\mathbf{l} = (\qty{2.00}{mm}) \hat{\mathbf{j}}\]
        \[d\mathbf{l} \times \hat{\mathbf{r}} = -(\qty{12.0}{mm}) \hat{\mathbf{k}}\]
        \[d\mathbf{B} = \frac{\mu_0}{4 \pi} \frac{I \,d\mathbf{l} \times \hat{\mathbf{r}}}{r^2} = -(\qty{2.88e-11}{T}) \hat{\mathbf{k}}\]
\end{enumerate}

\subsubsection{VP28.2.4}

\begin{align*}
  d\mathbf{B} & = \frac{\mu_0}{4 \pi} \frac{I \,d\mathbf{l} \times \hat{\mathbf{r}}}{r^2} \\
              & = \frac{\mu_0}{4 \pi} \frac{I}{r^2} (0.00800 \hat{\mathbf{j}})            \\
              & = (\qty{2.05e-9}{T}) \hat{\mathbf{j}}
\end{align*}

\begin{align*}
  \mathbf{F} & = -q \mathbf{v} \times d \mathbf{B}    \\
             & = q v \,dB \hat{\mathbf{k}}            \\
             & = (\qty{9.84e-23}{N}) \hat{\mathbf{k}}
\end{align*}

\subsubsection{VP28.5.1}

\[B = \frac{\mu_0 I}{2 \pi r} \Rightarrow I = \frac{2 \pi r B}{\mu_0} = \qty{2.24}{A}\]

\subsubsection{VP28.5.2}

\begin{enumerate}[(a)]
  \item \[\mathbf{B} = \frac{\mu_0}{2 \pi} \left( \frac{I_1}{r_1} - \frac{I_2}{r_2} \right) \hat{\mathbf{i}} = -(\qty{6.00e-5}{T}) \hat{\mathbf{i}}\]

  \item \[\mathbf{B} = -\frac{\mu_0}{2 \pi} \left( \frac{I_1}{r_1} + \frac{I_2}{r_2} \right) \hat{\mathbf{i}} = -(\qty{4.67e-5}{T}) \hat{\mathbf{i}}\]

  \item \[\mathbf{B} = \frac{\mu_0}{2 \pi} \left( \frac{I_1}{r_1} + \frac{I_2}{r_2} \right) = (\qty{8.67e-5}{T}) \hat{\mathbf{i}}\]
\end{enumerate}

\subsubsection{VP28.5.3}

\begin{enumerate}[(a)]
  \item The magnitude of the magnetic field generated by the first wire alone is \[B = \frac{\mu_0 I}{2 \pi r} = \qty{2.67e-5}{T}.\] The current in the secon wire will generate a magnetic field in the opposite direction meaning the final magnetic field must also be in the opposite direction. The current required to achieve the desired magnetic field strength is \[-B = \frac{\mu_0}{\pi r} (I_1 - I_2) \Rightarrow I_2 = I_1 + \frac{\pi r B}{\mu_0} = \qty{5.00}{A}\]

  \item The current in the second wire will generate a magnetic field in the same direction as the first so \[B = \frac{\mu_0}{\pi r} (I_1 + I_2) \Rightarrow I_2 = \frac{\pi r B}{\mu_0} - I_1 = \qty{1.00}{A}\]
\end{enumerate}

\subsubsection{VP28.5.4}

The wires are attracted to each other so their currents are in the same direction. The current in the other wire is given by \[\frac{F}{L} = \frac{\mu_0 I I'}{2 \pi r} \Rightarrow I' = \frac{2 \pi r}{\mu_0 I} \frac{F}{L} = \qty{3.29e4}{A}\]

\subsubsection{VP28.10.1}

Inside the conductor \[B = \frac{\mu_0 I}{2 \pi} \frac{r}{R^2}\] and outside the conductor \[B = \frac{\mu_0 I}{2 \pi r}\]

\begin{enumerate}[(a)]
  \item $B = \qty{4.94e-6}{T}$

  \item $B = \qty{8.89e-6}{T}$

  \item $B = \qty{6.67e-6}{T}$
\end{enumerate}

\subsubsection{VP28.10.2}

\begin{enumerate}[(a)]
  \item Due to the nature of the current the magnetic field must be circumferential. By Ampere's law the magnetic field in the region $r < R_1$ must be $0$ as there is no current there. Using Ampere's law in the region $R_1 < r < R_2$ with a circular integral path centred on the axis of the cylinder we find that \[2 \pi r B = \mu_0 \pi (r^2 - R_1^2) J \Rightarrow B = \frac{\mu_0 J (r^2 - R_1^2)}{2 r}.\] Using Ampere's law in the region $r > R_2$ with a circular integral path centred on the axis of the cylinder we find that \[2 \pi r B = \mu_0 \pi (R_2^2 - R_1^2) J \Rightarrow B = \frac{\mu_0 J (R_2^2 - R_1^2)}{2 r}.\]

  \item $r = R_2$
\end{enumerate}

\subsubsection{VP28.10.3}

\[B = \mu_0 n I \Rightarrow I = \frac{B}{\mu_0 n} = \qty{3.18}{A}\]

\subsubsection{VP28.10.4}

\begin{enumerate}[(a)]
  \item \[B = \frac{\mu_0 N I}{2 \pi r} \Rightarrow N = \frac{2 \pi r B}{\mu_0 I} = 833\]

  \item \[B_\textrm{max} = \qty{2.33e-3}{T}\] \[B_\textrm{min} = \qty{1.75e-3}{T}\]
\end{enumerate}

\subsection{Bridging Problem}

The charge of a loop of width $dr$ is \[dQ = 2 \pi r \,dr \frac{Q}{\pi a^2} = \frac{2 Q r}{a^2} \,dr.\] The loop rotates $n$ times per second so its current is \[dI = n \,dQ = \frac{2 n Q r}{a^2} \,dr.\] The magnetic field at the centre of a current-carrying loop is \[B = \frac{\mu_0 I}{2 r}\] so the magnetic field at the centre of the loop is \[dB = \frac{\mu_0 \,dI}{2 r} = \frac{\mu_0 n Q}{a^2} \,dr\] and thus the magnetic field at the centre of the disk is

\begin{align*}
  B & = \int dB                             \\
    & = \int_0^a \frac{\mu_0 n Q}{a^2} \,dr \\
    & = \frac{\mu_0 n Q}{a}
\end{align*}

\subsection{Exercises}

\subsubsection{28.1}

\begin{enumerate}[(a)]
  \item $\mathbf{B} = -(\qty{19.2}{\mu T}) \hat{\mathbf{k}}$

  \item $\mathbf{B} = \mathbf{0}$

  \item $\mathbf{B} = (\qty{19.2}{\mu T}) \hat{\mathbf{i}}$

  \item $\mathbf{B} = (\qty{6.79}{\mu T}) \hat{\mathbf{i}}$
\end{enumerate}

\subsubsection{28.3}

\begin{enumerate}[(a)]
  \item $B = \qty{6.00e-8}{T}$ out of the page

  \item $B = \qty{0.120}{\mu T}$ out of the page

  \item $B = 0$
\end{enumerate}

\subsubsection{28.9}

\[d \mathbf{B} = \frac{\mu_0}{4 \pi} \frac{I \,d \mathbf{l} \times \hat{\mathbf{r}}}{r^2}\]

\begin{enumerate}[(a)]
  \item $B = \qty{0.440}{\mu T}$ out of the page

  \item $B = \qty{16.7}{n T}$ out of the page

  \item $B = 0$
\end{enumerate}

\subsubsection{28.11}

\begin{enumerate}[(a)]
  \item $\mathbf{B} = (\qty{50.0}{p T}) \hat{\mathbf{j}}$

  \item $\mathbf{B} = -(\qty{50.0}{p T}) \hat{\mathbf{i}}$

  \item $\mathbf{B} = (\qty{25.0}{p T}) (-\frac{1}{\sqrt{2}} \hat{\mathbf{i}} + \frac{1}{\sqrt{2}} \hat{\mathbf{j}})$

  \item $B = 0$
\end{enumerate}

\subsubsection{28.13}

\begin{align*}
  B & = \frac{\mu_0 I l}{4 \pi} \left( \frac{2}{\sqrt{0.03^2 + 0.03^2}} \right)^2 \left( \sin \frac{\pi}{4} + \sin \frac{3 \pi}{4} \right) \\
    & = \qty{17.6}{\mu T} \textrm{ into the page}
\end{align*}

\subsubsection{28.15}

\begin{enumerate}[(a)]
  \item $B = \frac{\mu_0 I}{2 \pi r} = \qty{0.80}{m T}$

  \item $B = \qty{0.040}{m T}$ so the magnetic field of the lightning bolt is $20$ times stronger
\end{enumerate}

\subsubsection{28.17}

\[B = \frac{\mu_0 I}{2 \pi r} \Rightarrow I = \frac{2 \pi r B}{\mu_0} = \qty{25}{\mu A}\]

\subsubsection{28.19}

\begin{enumerate}[(a)]
  \item $\mathbf{B} = -(\qty{0.100}{\mu T}) \hat{\mathbf{i}}$

  \item $\mathbf{B} = (\qty{1.50}{\mu T}) \hat{\mathbf{i}} + (\qty{1.60}{\mu T}) \hat{\mathbf{k}}$ which has magnitude $\qty{2.19}{\mu T}$ and direction $\ang{46.8}$ from the $+x$ axis to the $+z$ axis

  \item $\mathbf{B} = (\qty{7.90}{\mu T}) \hat{\mathbf{i}}$
\end{enumerate}

\subsubsection{28.25}

\begin{enumerate}[(a)]
  \item $B_P = \qty{41}{\mu T}$ into the page, $B_Q = \qty{41}{\mu T}$ out of the page

  \item $B_P = \qty{9.0}{\mu T}$ out of the page, $B_Q = \qty{9.0}{\mu T}$ into the page
\end{enumerate}

\subsubsection{28.27}

\begin{align*}
  \lambda g & = \frac{F}{L}                       \\
            & = \frac{\mu_0 I^2}{2 \pi h}         \\
  h         & = \frac{\mu_0 I^2}{2 \pi \lambda g}
\end{align*}

\subsubsection{28.29}

\begin{enumerate}[(a)]
  \item $F = \qty{6.00}{\mu N}$ repulsive

  \item $F = \qty{24.0}{\mu N}$
\end{enumerate}

\subsubsection{28.31}

\[B = \frac{\mu_0 I}{2 a} \Rightarrow I = \frac{2 a B}{\mu_0} = \qty{0.38}{\mu A}\]

\subsubsection{28.33}

The magnetic field generated by one of the semicircular wires is \[B = \frac{\mu_0 I}{4 R}\] so the field generated by both of them is \[B = \frac{\mu_0}{4 R} (I_1 - I_2).\] When $I_1 = I_2$ the magnetic field at $P$ is $0$.

\subsubsection{28.35}

The magnetic field generated by the inner wire points into the table so the magnetic field generated by the outer wire must point out of the table, i.e. its current must be counter-clockwise.

The current in the outer wire must be

\begin{align*}
  0                & = \frac{\mu_0}{2} \left( \frac{I_\textrm{inner}}{R_\textrm{inner}} - \frac{I_\textrm{outer}}{R_\textrm{outer}} \right) \\
  I_\textrm{outer} & = I_\textrm{inner} \frac{R_\textrm{outer}}{R_\textrm{inner}}                                                           \\
                   & = \qty{18.0}{A}
\end{align*}

\subsubsection{28.37}

\begin{enumerate}[(a)]
  \item $I_\textrm{enc} = \qty{305}{A}$

  \item $\qty{-3.83e-4}{T.m}$
\end{enumerate}

\subsubsection{28.39}

\begin{enumerate}[(a)]
  \item $B = \frac{\mu_0 I}{2 \pi r}$

  \item $B = 0$
\end{enumerate}

\subsubsection{28.41}

\begin{enumerate}[(a)]
  \item $B = \frac{\mu_0 I_1}{2 \pi r}$

  \item $B = \frac{\mu_0 (I_1 + I_2)}{2 \pi r}$
\end{enumerate}

\subsubsection{28.43}

\begin{enumerate}[(a)]
  \item \[B = \mu_0 n I \Rightarrow n = \frac{B}{\mu_0 I} = 1790\]

  \item $\qty{63.0}{m}$
\end{enumerate}

\subsubsection{28.45}

\begin{enumerate}[(a)]
  \item \[I = \frac{2 \pi r B}{\mu_0} = \qty{3.72}{MA}\]

  \item \[I = \frac{2 \pi r B}{\mu_0 N} = \qty{124}{kA}\]

  \item \[I = \qty{237}{A}\]
\end{enumerate}

\subsubsection{28.47}

\begin{enumerate}[(a)]
  \item

        \begin{enumerate}[(i)]
          \item $B_0 = \mu_0 n I = \qty{1.13}{mT}$

          \item $K_m B_0 = B_0 + \mu_0 M \Rightarrow M = \frac{B_0}{\mu_0} (K_m - 1) = \qty{4.68}{MA/m}$

          \item $B = K_m B_0 = \qty{5.88}{T}$
        \end{enumerate}
\end{enumerate}

\subsection{Problems}

\subsubsection{28.49}

\begin{enumerate}[(a)]
  \item $B = \frac{\mu_0}{4 \pi} \left( \frac{q v}{y^2} - \frac{q' v'}{x^2} \right) = \qty{1.00}{\mu T}$ into the page

  \item The magnetic field produced by $q'$ at $q$ has magnitude \[B = \frac{\mu_0 q' v}{4 \pi r^2} = \qty{0.104}{\mu T}\] and is directed into the page so the force exerted on $q$ has magnitude \[F = q v B = \qty{74.9}{nN}\] and is in the $+y$ direction
\end{enumerate}

\subsubsection{28.51}

\begin{align*}
  0 & = \frac{\mu_0}{2 \pi} \left( \frac{I_1}{y} - \frac{I_2}{0.8 - y} \right) \\
    & = I_1 (0.8 - y) - I_2 y                                                  \\
    & = 0.8 I_1 - (I_1 + I_2) y                                                \\
  y & = 0.8 \frac{I_1}{I_1 + I_2}                                              \\
    & = \qty{0.200}{m}
\end{align*}

\subsubsection{28.53}

\begin{enumerate}[(a)]
  \item \[F = \frac{\mu_0 I^2 R}{d}\]

  \item \[I = \frac{2 B R}{\mu_0}\] \[F = \frac{\mu_0 R}{d} \left( \frac{2 B R}{\mu_0} \right)^2 = \frac{4 B^2 R^3}{\mu_0 d} = \frac{4 B^2 (A / \pi)^{3/2}}{\mu_0 d}\]

  \item \[B = \sqrt{\frac{\mu_0 d F}{4 (A / \pi)^{3/2}}}\]
\end{enumerate}

\subsubsection{28.55}

\begin{enumerate}[(a)]
  \item The electron experiences a magnetic field of magnitude \[B = \frac{\mu_0 I}{2 \pi r} = \qty{0.130}{mT}\] circumferential to the wire so it experiences a magnetic force of magnitude \[F = q v B = \qty{5.20e-18}{N}\] directed away from the wire, resulting in an initial acceleration of magnitude \[a = \frac{F}{m} = \qty{5.71e12}{m/s^2}\] directed away from the wire.

  \item The electric field should be directed away from the wire and have magnitude \[E = \frac{F}{q} = \qty{32.5}{N/C}\]

  \item The electron experiences a gravitational force of $F = m g = \qty{8.92e-30}{N}$ which is significantly smaller than the electric and magnetic forces, so it can be ignored
\end{enumerate}

\subsubsection{28.59}

\begin{enumerate}[(a)]
  \item The current must be directed out of the page and have magnitude

        \begin{align*}
          0   & = \frac{\mu_0}{2 \pi} \left( \frac{I_2}{0.50} - \frac{I_1}{1.5} \right) \\
          I_2 & = I_1 \frac{0.50}{1.5}                                                  \\
              & = \qty{2.00}{A}
        \end{align*}

  \item \[B = \frac{\mu_0}{2 \pi} \left( \frac{I_1}{0.50} - \frac{I_2}{1.5} \right) = \qty{2.13}{\mu T}\] upwards
\end{enumerate}

\subsubsection{28.61}

Let the angle between the cords and the vertical be $\theta$, the length of the wire be $l$, the magnetic force per unit length be $f$, the mass of the wire per unit length be $\lambda$, the tension per unit length be $t$, and the weight per unit length be $w$.

The distance between the wires is \[r = 2 l \sin \theta.\] They're in equilibrium so they experience no net horizontal force

\begin{align*}
  f - t \sin \theta                                     & = 0 \\
  \frac{\mu_0 I^2}{2 \pi r} - t \sin \theta             & = 0 \\
  \frac{\mu_0 I^2}{4 \pi l \sin \theta} - t \sin \theta & = 0
\end{align*}

and no net vertical force

\begin{align*}
  t \cos \theta - w         & = 0                              \\
  t \cos \theta - \lambda g & = 0                              \\
  t                         & = \frac{\lambda g}{\cos \theta}.
\end{align*}

Subsituting this into the equation for horizontal force we find

\begin{align*}
  \frac{\mu_0 I^2}{4 \pi l \sin \theta} - \lambda g \tan \theta & = 0                                                              \\
  I                                                             & = \sqrt{\frac{4 \pi \lambda g l \sin \theta \tan \theta}{\mu_0}} \\
                                                                & = \qty{23.2}{A}.
\end{align*}

\subsubsection{28.63}

\begin{enumerate}[(a)]
  \item

        \begin{align*}
          I_0 & = \int_0^a \int_0^{2\pi} \frac{b}{r} e^{(r - a) / \delta} r \, d \theta \,dr \\
              & = 2 \pi b \int_0^a e^{(r - a) / \delta} \,dr                                 \\
              & = 2 \pi b \left[ \delta e^{(r - a) / \delta} \right]_0^a                     \\
              & = 2 \pi b \delta \left( 1 - e^{-a / \delta} \right)                          \\
              & = \qty{81.5}{A}
        \end{align*}

  \item \[B = \frac{\mu_0 I_0}{2 \pi r}\]

  \item

        \begin{align*}
          I & = \int_0^r \int_0^{2 \pi} \frac{b}{r'} e^{(r' - a) / \delta} r' \,d \theta \,dr' \\
            & = 2 \pi b \int_0^r e^{(r' - a) / \delta} \,dr'                                   \\
            & = 2 \pi b \left[ \delta e^{(r' - a) / \delta} \right]_0^r                        \\
            & = 2 \pi b \delta \left( e^{(r - a) / \delta} - e^{-a / \delta} \right)           \\
            & = \frac{e^{(r - a) / \delta} - e^{-a / \delta}}{1 - e^{-a / \delta}} I_0         \\
            & = \frac{e^{r / \delta} - 1}{e^{a / \delta} - 1} I_0
        \end{align*}

  \item \[B = \frac{\mu_0 I_0}{2 \pi r} \frac{e^{r / \delta} - 1}{e^{a / \delta} - 1}\]

  \item $\qty{0.175}{mT}$, $\qty{0.326}{mT}$, and $\qty{0.163}{mT}$
\end{enumerate}

\subsubsection{28.65}

\begin{enumerate}[(a)]
  \item

        \begin{align*}
          I      & = \int_0^R \int_0^{2 \pi} \alpha r^2 \,d\theta \,dr \\
                 & = 2 \pi \alpha \left[ \frac{1}{3} r^3 \right]_0^R   \\
                 & = \frac{2}{3} \pi \alpha R^3                        \\
          \alpha & = \frac{3 I}{2 \pi R^3}
        \end{align*}

  \item

        \begin{enumerate}[(i)]
          \item

                \begin{align*}
                  B & = \frac{\mu_0}{2 \pi r} \int_0^r \int_0^{2 \pi} \alpha r'^2 \,d \theta \,dr \\
                    & = \frac{\alpha \mu_0}{r} \left[ \frac{1}{3} r'^3 \right]_0^r                \\
                    & = \frac{1}{3} \alpha \mu_0 r^2                                              \\
                    & = \frac{\mu_0 I r^2}{2 \pi R^3}
                \end{align*}

          \item \[B = \frac{\mu_0 I}{2 \pi r}\]
        \end{enumerate}
\end{enumerate}

\subsubsection{28.67}

\begin{enumerate}[(a)]
  \item

        \begin{align*}
          I & = \int_0^a \int_0^{2 \pi} \frac{2 I_0}{\pi a^2} \left[ 1 - \left( \frac{r}{a} \right)^2 \right] r \,d \theta \,dr \\
            & = \frac{4 I_0}{a^2} \left[ \frac{1}{2} r^2 - \frac{r^4}{4 a^2} \right]_0^a                                        \\
            & = \frac{4 I_0}{a^2} \left( \frac{1}{2} a^2 - \frac{1}{4} a^2 \right)                                              \\
            & = I_0
        \end{align*}

  \item \[B = \frac{\mu_0 I_0}{2 \pi r}\]

  \item

        \begin{align*}
          I & = \int_0^r \int_0^{2 \pi} \frac{2 I_0}{\pi a^2} \left[ 1 - \left( \frac{r'}{a} \right)^2 \right] r' \,d \theta \,dr' \\
            & = \frac{4 I_0}{a^2} \left[ \frac{1}{2} r'^2 - \frac{r'^4}{4 a^2} \right]_0^r                                         \\
            & = \frac{4 I_0}{a^2} \left( \frac{1}{2} r^2 - \frac{r^4}{4 a^2} \right)                                               \\
            & = \frac{I_0 r^2}{a^2} \left( 2 - \frac{r^2}{a^2} \right)
        \end{align*}

  \item \[B = \frac{\mu_0 I}{2 \pi r} = \frac{\mu_0 I_0 r}{2 \pi a^2} \left( 2 - \frac{r^2}{a^2} \right)\]
\end{enumerate}

\subsubsection{28.69}

By the right hand rule we find that the magnetic field must only have an $x$ component and be directed in the $-x$ direction when $z < 0$ and $+x$ direction when $z > 0$. Using Ampere's law with a counter-clockwise square integral path in the $x$-$z$ plane straddling the current sheet and with side $L$ finds

\begin{align*}
  2 B L      & = \mu_0 I L n                                                 \\
  \mathbf{B} & = \begin{cases}
                   \frac{1}{2} \mu_0 I n \hat{\mathbf{i}}  & z > 0 \\
                   -\frac{1}{2} \mu_0 I n \hat{\mathbf{i}} & z < 0
                 \end{cases}
\end{align*}

\subsubsection{28.71}

\begin{enumerate}[(a)]
  \item \[I_\text{encl} = \frac{Q_1}{2 \pi R_1} \omega_1 R_1 = \frac{\omega_1 Q_1}{2 \pi}\]

  \item By the right hand rule the magnetic field must be directed along the axis of the cylinder. Using Ampere's law with a rectangular integral path of height $L$ with one side on the cylinder's axis and another very far outside the cylinder we find that \[B L = \mu_0 \frac{I_\text{encl}}{H} L \Rightarrow \mathbf{B} = \frac{\mu_0 I_\text{encl}}{H} \hat{\mathbf{k}} = \frac{\mu_0 \omega_1 Q_1}{2 \pi H} \hat{\mathbf{k}}\]

  \item \[\tau = \mu B \sin \theta = \frac{1}{4} \omega_2 Q_2 R_2^2 \frac{\mu_0 \omega_1 Q_1}{2 \pi H} \sin \theta = \frac{\mu_0 \omega_1 \omega_2 Q_1 Q_2 R^2}{8 \pi H} \sin \theta\]

  \item The moment of inertia of the disk about its axis is

        \begin{align*}
          I & = \int_0^{R_2} \int_0^{2 \pi} \frac{M}{\pi R_2^2} r^3 \,d \theta \,dr \\
            & = \frac{2 M}{R_2^2} \left[ \frac{1}{4} r^4 \right]_0^{R_2}            \\
            & = \frac{1}{2} M R_2^2
        \end{align*}

        so the magnitude of its angular momentum about $\boldsymbol{\omega}_2$ is \[L = I \omega_2 = \frac{1}{2} M R_2^2 \omega_2\]
\end{enumerate}

\subsubsection{28.75}

\begin{enumerate}[(a)]
  \item $\rho = n q$

  \item $\sigma = n^{2/3} q$

  \item By the right hand rule the magnetic field must be circumferential. Using Ampere's law with an integral path around the circumference of the cylinder finds

        \begin{align*}
          2 \pi R B  & = \mu_0 I_\text{encl}                               \\
                     & = \mu_0 \pi R^2 \rho v                              \\
                     & = \mu_0 \pi R^2 n q v                               \\
          \mathbf{B} & = \frac{1}{2} \mu_0 R n q v \hat{\boldsymbol{\phi}}
        \end{align*}

  \item $dI_\text{surface} = R \sigma v \,d \phi = n^{2 / 3} R q v \,d \phi$

  \item $d \mathbf{F} = dI_\text{surface} \mathbf{l} \times \mathbf{B} = -\frac{1}{2} \mu_0 L n^{5 / 3} q^2 R^2 v^2 \,d \phi \,\hat{\mathbf{r}}$

  \item \[F = \mu_0 \pi L n^{5 / 3} q^2 R^2 v^2\] \[p = \frac{F}{2 \pi R L} = \frac{1}{2} \mu_0 n^{5 / 3} q^2 R v^2\]

  \item $p = \qty{19.1}{\mu N/m^2}$
\end{enumerate}

\subsubsection{28.77}

\begin{enumerate}[(a)]
  \item

        \begin{align*}
          dI & = R \,d\theta \sigma \omega r                          \\
             & = R \,d\theta \frac{Q}{4 \pi R^2} \omega R \sin \theta \\
             & = \frac{\omega Q}{4 \pi} \sin \theta \,d\theta
        \end{align*}

  \item

        \begin{align*}
          d\mu & = A \,dI                                                             \\
               & = \pi r^2 \frac{\omega Q}{4 \pi} \sin \theta \,d\theta               \\
               & = \pi (R \sin \theta)^2 \frac{\omega Q}{4 \pi} \sin \theta \,d\theta \\
               & = \frac{1}{4} \omega Q R^2 \sin^3 \theta \,d\theta
        \end{align*}

  \item

        \begin{align*}
          \boldsymbol{\mu} & = \int d\mu \,\hat{\mathbf{k}}                                                                                  \\
                           & = \int_0^\pi \frac{1}{4} \omega Q R^2 \sin^3 \theta \,d\theta \,\hat{\mathbf{k}}                                \\
                           & = \frac{1}{4} \omega Q R^2 \left[ \frac{1}{12} (\cos (3 \theta) - 9 \cos \theta) \right]_0^\pi \hat{\mathbf{k}} \\
                           & = \frac{1}{3} \omega Q R^2
        \end{align*}

  \item \[L = \frac{2}{3} M R^2 \omega\]

  \item \[\gamma = \frac{Q}{2 m}\]

  \item $g = 1$
\end{enumerate}

\subsubsection{28.79}

\begin{enumerate}[(a)]
  \item Assuming the capacitor discharges entirely before the wires move, we can use the wires' change in momentum to determine their initial velocity.

        Rearranging the equation for momentum per unit length we find an expression for $v_0$ \[\frac{p}{L} = \lambda v_0 \Rightarrow v_0 = \frac{1}{\lambda} \frac{p}{L}.\]

        The wire's momentum per unit length is equal to the integral of the repulsive force per unit length over all time

        \begin{align*}
          v_0 & = \frac{1}{\lambda} \int_0^\infty \frac{F}{L} \,dt                \\
              & = \frac{1}{\lambda} \int_0^\infty \frac{\mu_0 I^2}{2 \pi d} \,dt.
        \end{align*}

        The current in a discharging R-C circuit is $Q_0 e^{-t / R C} / R C$ which lets us calculate the initial velocity

        \begin{align*}
          v_0 & = \frac{1}{\lambda} \int_0^\infty \frac{\mu_0}{2 \pi d} \left( \frac{Q_0}{R C} e^{-t / R C} \right)^2 \,dt \\
              & = \frac{\mu_0 Q_0^2}{2 \pi d \lambda R^2 C^2} \int_0^\infty e^{-2 t / R C} \,dt                            \\
              & = \frac{\mu_0 Q_0^2}{2 \pi d \lambda R^2 C^2} \left[ -\frac{R C}{2} e^{-2 t / R C} \right]_0^\infty        \\
              & = \frac{\mu_0 Q_0^2}{4 \pi d \lambda R C}.
        \end{align*}

  \item No non-conservative forces act on the wires so energy is conserved and

        \begin{align*}
          \lambda g h & = \frac{1}{2} \lambda v_0^2 \\
          h           & = \frac{v_0^2}{2 g}
        \end{align*}
\end{enumerate}

\subsubsection{28.81}

b

\subsubsection{28.83}

c

\end{document}