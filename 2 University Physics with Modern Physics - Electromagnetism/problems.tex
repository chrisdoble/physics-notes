\documentclass{article}
\usepackage{amsmath} % For align*
\usepackage{enumerate} % For customisable list labels
\usepackage{siunitx} % For units

\title{University Physics with Modern Physics Electromagnetism Problems}
\author{Chris Doble}
\date{December 2022}

\begin{document}

\maketitle

\tableofcontents

\setcounter{section}{20}
\section{Electric Charge and Electric Field}

\setcounter{subsection}{2}
\subsection{Coulomb's Law}

\subsubsection{Example 21.1}

The magnitude of electric repulsion between two $\alpha$ particles is given by \[F_e = \frac{1}{4\pi\epsilon_0}\frac{q^2}{r^2}\] and the magnitude of gravitational attraction is given by \[F_g = \frac{Gm^2}{r^2}\]. The ratio of the two values is

\begin{align*}
  \frac{F_e}{F_g} & = \frac{1}{4\pi\epsilon_0}\frac{q^2}{r^2}\frac{r^2}{Gm^2} \\
                  & = \frac{1}{4\pi\epsilon_0}\frac{q^2}{Gm^2}                \\
                  & = 3.1\times10^{35}
\end{align*}

showing that the electric repulsion is significantly stronger than the gravitational attraction.

\subsubsection{Example 21.2}

\begin{enumerate}[a)]
  \item The magnitude of the force that $q_1$ exerts on $q_2$ is
        \begin{align*}
          F & = \frac{1}{4\pi\epsilon_0}\frac{q_1q_2}{r^2}                                 \\
            & = (9.0 \times 10^9)\frac{|(25 \times 10^{-9})(-75 \times 10^{-9})|}{0.030^2} \\
            & = 1.9 \times 10^{-2}\,\textrm{N}.
        \end{align*}
        Since $q_1$ and $q_2$ have opposite charge, the force is attractive (from $q_2$ to $q_1$).

  \item The magnitude of the force that $q_2$ exerts on $q_1$ is the same as in part a, but the direction is reversed (from $q_1$ to $q_2$).
\end{enumerate}

\subsubsection{Example 21.3}

By the principle of superposition of forces, the net force exerted on $q_3$ is equal to the vector sum of the forces exerted on it by $q_1$ and $q_2$ separately.

Both $q_1$ and $q_3$ have positive charge so they repel each other. $q_1$ is to the right of $q_3$ so $q_3$ experiences a force to the left of magnitude

\begin{align*}
  F_\textrm{1 on 3} & = \frac{1}{4\pi\epsilon_0}\frac{|q_1q_3|}{r^2}                                \\
                    & = (9.0 \times 10^9)\frac{|(1.0 \times 10^{-9})(5.0 \times 10^{-9})|}{0.020^2} \\
                    & = 1.1 \times 10^{-4}\,\textrm{N}.
\end{align*}

However $q_2$ has a negative charge so it attracts $q_3$. It is also to the right of $q_3$ so $q_3$ experiences a force to the right of magnitude

\begin{align*}
  F_\textrm{2 on 3} & = \frac{1}{4\pi\epsilon_0}\frac{|q_2q_3|}{r^2}                                \\
                    & = (9.0 \times 10^9)\frac{|(-3.0 \times 10^{-9})(5.0 \times 10^{-9})}{0.040^2} \\
                    & = 8.4 \times 10^{-5}\,\textrm{N}.
\end{align*}

The net force experienced by $q_3$ is therefore

\begin{align*}
  F & = -F_\textrm{1 on 3} + F_\textrm{2 on 3}   \\
    & = -1.1 \times 10^{-4} + 8.4 \times 10^{-5} \\
    & = -2.6 \times 10^{-5}\,\textrm{N}.
\end{align*}

\subsubsection{Example 21.4}

Since $q_1$ and $q_2$ are of equal charge and are symmetric about the x axis on which $Q$ lies, the vertical components of their forces cancel leaving only the horizontal.

The horizontal component of $q_1$'s force on $Q$ is given by

\begin{align*}
  F_\textrm{1 on Q, x} & = \frac{1}{4\pi\epsilon_0} \frac{q_1Q}{r_\textrm{1,Q}^2} \cos\alpha                                             \\
                       & = (9.0 \times 10^9) \frac{(2.0 \times 10^{-6})(4.0 \times 10^{-6})}{\sqrt{0.30^2 + 0.40^2}^2} \frac{0.40}{0.50} \\
                       & = 0.23\,\textrm{N}.
\end{align*}

Again, since $q_1$ and $q_2$ are of equal charge and symmetric about the x axis, $F_\textrm{1 on Q, x} = F_\textrm{2 on Q, x}$ and the total force experienced by $Q$ is in the positive x direction of magnitude \[F = 2F_\textrm{1 on Q, x} = 0.46\,\textrm{N}.\]

\subsection{Electric Field and Electric Forces}

\subsubsection{Example 21.5}

The magnitude of the electric field vector is given by

\begin{align*}
  E & = \frac{1}{4\pi\epsilon_0} \frac{|q|}{r^2}             \\
    & = (9.0 \times 10^9) \frac{|4.0 \times 10^{-9}|}{2.0^2} \\
    & = 9.0\,\textrm{N}/\textrm{C}.
\end{align*}

\subsubsection{Example 21.6}

The magnitude of the electric field vector is given by

\begin{align*}
  E & = \frac{1}{4\pi\epsilon_0} \frac{|q|}{r^2}                      \\
    & = (9.0 \times 10^9) \frac{|-8.0 \times 10^{-9}|}{1.2^2 + 1.6^2} \\
    & = 18\,\textrm{N}/\textrm{C}
\end{align*}

and it is directed towards the origin. If $\theta$ is the angle between the positive x axis and $\hat{\mathbf{r}}$ then the component form of $\mathbf{E}$ is

\begin{align*}
  E & = -E\left(\cos\theta\hat{\mathbf{i}} + \sin\theta\hat{\mathbf{j}}\right)                      \\
    & = -E\left(\frac{x}{r}\hat{\mathbf{i}} + \frac{-y}{r}\hat{\mathbf{j}}\right)                   \\
    & = \frac{-18}{\sqrt{1.2^2 + 1.6^2}}\left(1.2\hat{\mathbf{i}} + 1.6\hat{\mathbf{j}}\right)      \\
    & = (-11\,\textrm{N}/\textrm{C})\hat{\mathbf{i}} - (14\,\textrm{N}/\textrm{C})\hat{\mathbf{j}}.
\end{align*}

\subsubsection{Example 21.7}

\begin{enumerate}[a)]
  \item Electrons have a negative charge and the electric field is directed upwards, so the electron will move downwards. The magnitude of its acceleration is

        \begin{align*}
          a & = \frac{F}{m}                                                           \\
            & = \frac{eE}{m}                                                          \\
            & = \frac{(1.60 \times 10^{-19})(1.00 \times 10^4)}{9.11 \times 10^{-31}} \\
            & = 1.76 \times 10^{15}\,\textrm{m}/\textrm{s}^2.
        \end{align*}

  \item Its acceleration is constant between the plates, so its final speed is

        \begin{align*}
          v^2 & = v_0^2 + 2a(x - x_0)                      \\
              & = 2ax                                      \\
          v   & = \sqrt{2ax}                               \\
              & = \sqrt{2(1.76 \times 10^{15})(0.01)}      \\
              & = 5.9 \times 10^6\,\textrm{m}/\textrm{s}^2
        \end{align*}

        and thus its final kinetic energy is

        \begin{align*}
          K & = \frac{1}{2}mv^2                                      \\
            & = \frac{1}{2}(9.11 \times 10^{-31})(5.9 \times 10^6)^2 \\
            & = 1.6 \times 10^{-17}\,\textrm{J}.
        \end{align*}

  \item We can find the time it takes for the electron to travel this distance by rearranging the kinematic equation \[v = v_0 + at\] to

        \begin{align*}
          t & = \frac{v - v_0}{a}                           \\
            & = \frac{5.9 \times 10^6}{1.76 \times 10^{15}} \\
            & = 3.4 \times 10^{-9}\,\textrm{s}.
        \end{align*}
\end{enumerate}

\subsection{Electric-Field Calculations}

\subsubsection{Example 21.8}

\begin{enumerate}[a)]
  \item At point $a$ the electric field caused by $q_1$ points to the right and has magnitude

        \begin{align*}
          E_1 & = \frac{1}{4\pi\epsilon_0} \frac{|q_1|}{r^2}            \\
              & = (9.0 \times 10^9) \frac{12 \times 10^{-9}}{(0.060)^2} \\
              & = 3.0 \times 10^4\,\textrm{N}/\textrm{C}.
        \end{align*}

        The electric field caused by $q_2$ also points to the right and it has magnitude

        \begin{align*}
          E_2 & = \frac{1}{4\pi\epsilon_0} \frac{|q_2|}{r^2}               \\
              & = (9.0 \times 10^9) \frac{|-12 \times 10^{-9}|}{(0.040)^2} \\
              & = 6.8 \times 10^4\,\textrm{N}/\textrm{C}.
        \end{align*}

        Thus the total field points to the right and has magnitude \[E = E_1 + E_2 = 9.8 \times 10^4\,\textrm{N}/\textrm{C}.\]

  \item At point $b$ the electric field caused by $q_1$ points to the left and has magnitude

        \begin{align*}
          E_1 & = \frac{1}{4\pi\epsilon_0} \frac{|q_1|}{r^2}            \\
              & = (9.0 \times 10^9) \frac{12 \times 10^{-9}}{(0.040)^2} \\
              & = 6.8 \times 10^4\,\textrm{N}/\textrm{C}.
        \end{align*}

        The electric field caused by $q_2$ points to the right and has magnitude

        \begin{align*}
          E_2 & = \frac{1}{4\pi\epsilon_0} \frac{|q_2|}{r^2}               \\
              & = (9.0 \times 10^9) \frac{|-12 \times 10^{-9}|}{(0.140)^2} \\
              & = 0.55 \times 10^4\,\textrm{N}/\textrm{C}.
        \end{align*}

        Thus the total electric field points to the left and has magnitude \[E = E_1 - E_2 = 6.3 \times 10^4\,\textrm{N}/\textrm{C}.\]

  \item At point $c$ the electric field caused by $q_1$ points from $q_1$ to $c$ and has magnitude

        \begin{align*}
          E_1 & = \frac{1}{4\pi\epsilon_0} \frac{|q_1|}{r^2}            \\
              & = (9.0 \times 10^9) \frac{|12 \times 10^{-9}|}{0.130^2} \\
              & = 6.4 \times 10^3\,\textrm{N}/\textrm{C}.
        \end{align*}

        The electric field caused by $q_2$ points from $c$ to $q_2$ and has magnitude

        \begin{align*}
          E_2 & = \frac{1}{4\pi\epsilon_0} \frac{|q_2|}{r^2}             \\
              & = (9.0 \times 10^9) \frac{|-12 \times 10^{-9}|}{0.130^2} \\
              & = 6.4 \times 10^3\,\textrm{N}/\textrm{C}                 \\
              & = E_1.
        \end{align*}

        The vertical components of $\mathbf{E_1}$ and $\mathbf{E_2}$ cancel, leaving only a horizontal component pointing to the right of magnitude

        \begin{align*}
          E & = 2E_1\cos\alpha                          \\
            & = 2(6.4 \times 10^3)\frac{0.050}{0.130}   \\
            & = 4.9 \times 10^3\,\textrm{N}/\textrm{C}.
        \end{align*}
\end{enumerate}

\subsubsection{Example 21.9}

By symmetry, each point on the ring has a corresponding point on the opposite side. The components of their electric fields perpendicular to the axis of the ring cancel, leaving only a component parallel to the axis of the ring. Thus the total magnetic field at $P$ is parallel to the axis of the ring and can be calculated as

\begin{align*}
  E & = \int_0^{2\pi} \frac{1}{4\pi\epsilon_0} \frac{\lambda}{r^2} \cos \alpha\,d\theta \\
    & = \frac{1}{4\pi\epsilon_0} \frac{Qx}{2\pi(a^2 + x^2)^{3/2}} \int_0^{2\pi} d\theta \\
    & = \frac{1}{4\pi\epsilon_0} \frac{Qx}{(a^2 + x^2)^{3/2}}.
\end{align*}

\subsubsection{Example 21.10}

By symmetry, each point on the line has a corresponding point on the opposite side of the $x$-axis. The $y$ components of their electric fields cancel, leaving only the $x$ components. Thus the total magnetic field at $P$ only has an $x$ component and can be calculated as

\begin{align*}
  E & = \int_{-a}^a \frac{1}{4\pi\epsilon_0} \frac{\lambda}{r^2} \cos\alpha\,dy                                        \\
    & = \frac{1}{4\pi\epsilon_0} \frac{Qx}{2a} \int_{-a}^a \frac{1}{(x^2 + y^2)^{3/2}}\,dy                             \\
    & = \frac{1}{4\pi\epsilon_0} \frac{Qx}{2a} \left[\frac{y}{x^2\sqrt{x^2+y^2}}\right]_{-a}^a                         \\
    & = \frac{1}{4\pi\epsilon_0} \frac{Q}{2ax} \left(\frac{a}{\sqrt{x^2 + a^2}} + \frac{a}{\sqrt{x^2 + (-a)^2}}\right) \\
    & = \frac{1}{4\pi\epsilon_0} \frac{Q}{x\sqrt{x^2 + a^2}}.
\end{align*}

\subsubsection{Example 21.11}

By symmetry, each point on the disk has a corresponding point 180° rotation around the $x$-axis. The $y$ and $z$ components of their electric fields cancel, leaving only the $x$ components. Thus the total magnetic field at $P$ only has an $x$ component and can be calculated as

\begin{align*}
  E & = \int_0^R \int_0^{2\pi} \frac{1}{4\pi\epsilon_0} \frac{\sigma}{r^2} s \cos\alpha\,d\theta\,ds                       \\
    & = \frac{\sigma}{4\pi\epsilon_0} \int_0^R \int_0^{2\pi} \frac{s}{s^2 + x^2} \frac{x}{\sqrt{s^2 + x^2}} \,d\theta \,ds \\
    & = \frac{\sigma x}{2\epsilon_0} \int_0^R \frac{s}{(s^2 + x^2)^{3/2}} \,ds                                             \\
    & = \frac{\sigma x}{2\epsilon_0} \left[-\frac{1}{\sqrt{s^2 + x^2}}\right]_0^R                                          \\
    & = \frac{\sigma x}{2\epsilon_0} \left(-\frac{1}{\sqrt{R^2 + x^2}} + \frac{1}{x}\right)                                \\
    & = \frac{\sigma}{2\epsilon_0} \left(1 - \frac{1}{\sqrt{(R/x)^2 + 1}}\right).
\end{align*}

\subsubsection{Example 21.12}

From Example 21.11 we know that the electric field produced by an infinite plane sheet of charge is \[E= \frac{\sigma}{2\epsilon_0}.\] Therefore the electric field outside the sheets is $\mathbf{0}$ and between the sheets is $\sigma/\epsilon_0$ towards the negative sheet.

\setcounter{subsection}{6}
\subsection{Electric Dipoles}

\subsubsection{Example 21.13}

\begin{enumerate}[a)]
  \item The electric field is uniform so the net force exerted on the dipole is $\mathbf{0}$

  \item The electric dipole moment is directed from the negative charge to the positive charge and has magnitude \[p = qd = (1.6 \times 10^{-19})(0.125 \times 10^{-9}) = 2.0 \times 10^{-29}\,\textrm{C}\cdot\textrm{m}\]

  \item The torque aligns the electric dipole moment with the electric field so it is directed out of the page and has magnitude \[\tau = qEd\sin\phi = (1.6 \times 10^{-19})(5.0 \times 10^5)(0.125 \times 10^{-9})\sin 35 = 5.7 \times 10^{-24}\,\textrm{N}\cdot\textrm{m}\]

  \item The potential energy of an electric dipole in a uniform electric field is given by \[U = -qdE\cos\phi = (2.0 \times 10^{-29})(5.0 \times 10^5)\cos 35 = 8.2 \times 10^{-24}\,\textrm{J}\]
\end{enumerate}

\subsubsection{Example 21.14}

As $P$ is on the $y$-axis, the electric fields of the electric dipole's point charges have no $x$ component and thus the net electric field is directed along the $y$-axis.

By the principle of superposition of electric fields, the magnitude of the electric field at $P$ is

\begin{align*}
  E & = E_- + E_+                                                                                                                         \\
    & = \frac{1}{4\pi\epsilon_0} \frac{-q}{(y - (-d/2))^2} + \frac{1}{4\pi\epsilon_0} \frac{q}{(y - d/2)^2}                               \\
    & = \frac{1}{4\pi\epsilon_0}q\left(\frac{1}{(y-d/2)^2} - \frac{1}{(y + d/2)^2}\right)                                                 \\
    & = \frac{1}{4\pi\epsilon_0} \frac{q}{y^2} \left( \left( 1 - \frac{d}{2y} \right)^{-2} - \left( 1 + \frac{d}{2y} \right)^{-2} \right) \\
    & \approx \frac{1}{4\pi\epsilon_0} \frac{q}{y^2} \left( 1 + \frac{d}{y} - 1 + \frac{d}{y} \right)                                     \\
    & = \frac{qd}{2\pi\epsilon_0y^3}                                                                                                      \\
    & = \frac{p}{2\pi\epsilon_0y^3}.
\end{align*}

\subsection{Guided Practice}

\subsubsection{VP21.4.1}

$q_1$ attracts $q_3$ to the left with magnitude

\begin{align*}
  F_1 & = \frac{1}{4\pi\epsilon_0} \frac{|q_1q_3|}{r^2}                                   \\
      & = (9.0 \times 10^9) \frac{|(4.00 \times 10^{-9})(-2.00 \times 10^{-9})}{0.0400^2} \\
      & = 4.5 \times 10^{-5} \,\textrm{N}.
\end{align*}

$q_2$ repels $q_3$ to the left with magnitude

\begin{align*}
  F_2 & = \frac{1}{4\pi\epsilon_0} \frac{|q_2 q_3|}{r^2}                                              \\
      & = (9.0 \times 10^9) \frac{|(-1.20 \times 10^{-9})(-2.00 \times 10^{-9}|}{(0.0600 - 0.0400)^2} \\
      & = 5.4 \times 10^{-5} \,\textrm{N}.
\end{align*}

By the principle of superposition of forces, the net force on $q_3$ is \[\mathbf{F} = (-F_1 - F_2)\hat{\mathbf{i}} = (-9.9 \times 10^{-5} \,\textrm{N}) \hat{\mathbf{i}}.\]

\subsubsection{VP21.4.2}

\begin{enumerate}[a)]
  \item $q_1$ repels $q_2$ in the positive $x$ direction with magnitude

        \begin{align*}
          F_1 & = \frac{1}{4\pi\epsilon_0} \frac{|q_1 q_2|}{r^2}                                                       \\
              & = \frac{1}{4 \pi (8.854 \times 10^{-12})} \frac{|(3.60 \times 10^{-9})(2.00 \times 10^{-9})}{0.0400^2} \\
              & = 40.4 \,\mu\textrm{N}.
        \end{align*}

  \item By the superposition of forces

        \begin{align*}
          F   & = F_1 + F_2            \\
          F_2 & = F - F_1              \\
              & = 54.0 - 40.4          \\
              & = 13.6 \,\mu\textrm{N}
        \end{align*}

        in the positive $x$ direction.

  \item $q_2$ repels $q_3$ so it must also have a positive charge of magnitude

        \begin{align*}
          F   & = \frac{1}{4\pi\epsilon_0} \frac{q_2 q_3}{r^2}                                           \\
          q_2 & = \frac{4\pi\epsilon_0 F r^2}{q_3}                                                       \\
              & = \frac{4\pi(8.854 \times 10^{-12})(1.36 \times 10^{-5})(0.0800)^2}{2.00 \times 10^{-9}} \\
              & = 4.84 \times 10^{-9} \,\textrm{C}.
        \end{align*}
\end{enumerate}

\subsubsection{VP21.4.3}

By symmetry the $x$ components of $q_1$ and $q_2$'s electric fields cancel leaving only their $y$ components which are directed in the negative $y$ direction and equal. $q_3$ is negative and thus experiences a net force in the positive $y$ direction of magnitude

\begin{align*}
  F & = 2 \frac{1}{4\pi\epsilon_0} \frac{q_1q_3}{r^2} \sin \alpha                                                                                            \\
    & = 2 \frac{1}{4\pi (8.854 \times 10^{-12})} \frac{(6.00 \times 10^{-9})(2.50 \times 10^{-9})}{0.150^2 + 0.200^2} \frac{0.200}{\sqrt{0.150^2 + 0.200^2}} \\
    & = 3.45 \times 10^{-6} \,\textrm{N}.
\end{align*}

\subsubsection{VP21.4.4}

The magnitude of the electric force exerted by $q_1$ on $q_3$ is

\begin{align*}
  F_1 & = \frac{1}{4 \pi \epsilon_0} \frac{|q_1 q_3|}{r^2}                                                                \\
      & = \frac{1}{4 \pi (8.854 \times 10^{-12})} \frac{|(4.00 \times 10^{-9}) (-1.50 \times 10^{-9})}{0.250^2 + 0.200^2} \\
      & = 5.26 \times 10^{-7} \,\textrm{N}.
\end{align*}

They have opposite charges so the force is directed from $q_3$ to $q_1$. In component form the force is

\begin{align*}
  \mathbf{F_1} & = -F_1\cos\alpha\hat{\mathbf{i}} + F_1\sin\alpha\hat{\mathbf{j}}                                                     \\
               & = F_1 \left( -\frac{x}{r}\hat{\mathbf{i}} + \frac{y}{r}\hat{\mathbf{j}} \right)                                      \\
               & = \frac{5.26 \times 10^{-7}}{\sqrt{0.250^2 + 0.200^2}} \left( -0.250\hat{\mathbf{i}} + 0.200\hat{\mathbf{j}} \right) \\
               & = (-4.11 \times 10^{-7}\,\textrm{N}) \hat{\mathbf{i}} + (3.29 \times 10^{-7}\,\textrm{N}) \hat{\mathbf{j}}.
\end{align*}

The magnitude of the electric force exerted by $q_2$ on $q_3$ is

\begin{align*}
  F_2 & = \frac{1}{4\pi\epsilon_0} \frac{|q_2 q_3|}{r^2}                                                          \\
      & = \frac{1}{4 \pi (8.854 \times 10^{-12})} \frac{|(-4.00 \times 10^{-9}) (-1.50 \times 10^{-9})|}{0.250^2} \\
      & = 8.63 \times 10^{-7} \,\textrm{N}.
\end{align*}

The have like charges so the force is directed from $q_2$ to $q_3$, i.e. along the positive $x$-axis. In component form the force is \[\mathbf{F_2} = (8.64 \times 10^{-7} \,\textrm{N})\hat{\mathbf{i}}.\]

Thus the net force experienced by $q_3$ is

\begin{align*}
  \mathbf{F} & = \mathbf{F_1} + \mathbf{F_2}                                                                                \\
             & = (4.53 \times 10^{-7} \,\textrm{N}) \hat{\mathbf{i}} + (3.29 \times 10^{-7} \,\textrm{N}) \hat{\mathbf{j}}.
\end{align*}

\subsubsection{VP21.10.1}

\begin{enumerate}[a)]
  \item The source points and field point all lie on the $y$-axis, so the source points' electric fields have no $x$ components. $q_1$ is positive and the field point is below it, so its contribution is negative. $q_2$ is negative and the field point is above it, so its contribution is also negative. Thus the $y$ component of the net electric field is

        \begin{align*}
          E_y & = \frac{1}{4 \pi \epsilon_0} \left( -\frac{q_1}{(y_1 - y)^2} + \frac{q_2}{(y_2 - y)^2} \right)                                                     \\
              & = \frac{1}{4 \pi (8.854 \times 10^{-12})} \left( \frac{4.00 \times 10^{-9}}{(0.200 - 0.100)^2} - \frac{5.00 \times 10^{-9}}{(0 - 0.100)^2} \right) \\
              & = -8.09 \times 10^3 \,\textrm{N}/\textrm{C}.
        \end{align*}

  \item The source points and field point all lie on the $y$-axis, so the source points' electric fields have no $x$ components. $q_1$ is positive and the field point is above it, so its contribution is positive. $q_2$ is negative and the field point is above it, so its contribution is also negative. Thus the $y$ component of the net electric field is

        \begin{align*}
          E_y & = \frac{1}{4 \pi \epsilon_0} \left( \frac{q_1}{(y_1 - y)^2} + \frac{q_2}{(y_2 - y)^2} \right)                                                      \\
              & = \frac{1}{4 \pi (8.854 \times 10^{-12})} \left( \frac{4.00 \times 10^{-9}}{(0.200 - 0.400)^2} - \frac{5.00 \times 10^{-9}}{(0 - 0.400)^2} \right) \\
              & = 618 \,\textrm{N}/\textrm{C}.
        \end{align*}

  \item The electric field of $q_1$ has magnitude

        \begin{align*}
          E_1 & = \frac{1}{4 \pi \epsilon_0} \frac{q_1}{r^2}                                            \\
              & = \frac{1}{4 \pi (8.854 \times 10^{-12})} \frac{4.00 \times 10^{-9}}{0.200^2 + 0.200^2} \\
              & = 449 \,\textrm{N}/\textrm{C}.
        \end{align*}

        It is directed from $q_1$ to the field point and thus in component form is

        \begin{align*}
          \mathbf{E_1} & = E_1(\cos\phi\hat{\mathbf{i}} - \sin\phi\hat{\mathbf{j}})                                       \\
                       & = \frac{449}{\sqrt{0.200^2 + 0.200^2}}(0.200\hat{\mathbf{i}} - 0.200\hat{\mathbf{j}})            \\
                       & = (317 \,\textrm{N}/\textrm{C})\hat{\mathbf{i}} - (317 \,\textrm{N}/\textrm{C})\hat{\mathbf{j}}.
        \end{align*}

        $q_2$ and the field point both lie on the $x$-axis, and thus its electric field has no $y$ component. In component form it is

        \begin{align*}
          \mathbf{E_2} & = \frac{1}{4 \pi \epsilon_0} \frac{q_2}{r^2} \hat{\mathbf{i}}                                   \\
                       & = \frac{1}{4 \pi (8.854 \times 10^{-12})} \frac{-5.00 \times 10^{-9}}{0.200^2} \hat{\mathbf{i}} \\
                       & = (-1.12 \times 10^{3} \,\textrm{N}/\textrm{C})\hat{\mathbf{i}}.
        \end{align*}

        The total electric field is thus

        \begin{align*}
          \mathbf{E} & = \mathbf{E_1} + \mathbf{E_2}                                                                                                \\
                     & = (-8.03 \times 10^2 \,\textrm{N}/\textrm{C})\hat{\mathbf{i}} + (-3.17 \times 10^2 \,\textrm{N}/\textrm{C})\hat{\mathbf{j}}.
        \end{align*}
\end{enumerate}

\subsubsection{VP21.10.2}

\begin{enumerate}[a)]
  \item Both source points and the field point are on the $x$-axis, so the electric fields at $P$ have no $y$ components.

        $q_1$ is positive and $P$ is to the right of $q_1$, so its electric field points to the right and has magnitude

        \begin{align*}
          E_1 & = \frac{1}{4 \pi \epsilon_0} \frac{q_1}{r^2}                                   \\
              & = \frac{1}{4 \pi (8.854 \times 10^{-12})} \frac{1.80 \times 10^{-9}}{0.0200^2} \\
              & = 4.04 \times 10^4 \,\textrm{N}/\textrm{C}.
        \end{align*}

  \item The magnitude and direction of the electric field that $q_2$ causes at $P$ can be calculated as

        \begin{align*}
          E   & = E_1 + E_2                                 \\
          E_2 & = E - E_1                                   \\
              & = 6.75 \times 10^4 - 4.04 \times 10^4       \\
              & = 2.71 \times 10^4 \,\textrm{N}/\textrm{C}.
        \end{align*}

  \item $E_2$ is positive at $P$, so $q_2$ must be negative. Its value is

        \begin{align*}
          E_2 & = -\frac{1}{4 \pi \epsilon_0} \frac{q_2}{r^2}                  \\
          q_2 & = -4 \pi \epsilon_0 E_2 r^2                                    \\
              & = -4 \pi (8.854 \times 10^{-12}) (2.71 \times 10^4) (0.0200)^2 \\
              & = -1.21 \times 10^{-9} \,\textrm{C}.
        \end{align*}
\end{enumerate}

\subsubsection{VP21.10.3}

\newcommand{\ke}{\frac{1}{4 \pi \epsilon_0}}

\begin{enumerate}[a)]
  \item From Example 21.9 we know that the electric field of a charged ring of radius $a$ at a distance $x$ along the ring's axis is directed away from the ring along its axis and has magnitude \[E = \frac{1}{4 \pi \epsilon_0} \frac{Qx}{(x^2 + a^2)^{3/2}}.\]

        By the principle of superposition of electric fields, the electric field of the hydrogen atom is

        \begin{align*}
          E & = \ke \left( \frac{e}{a^2} - \frac{ea}{(a^2 + a^2)^{3/2}} \right) \\
            & = \ke e \left( \frac{1}{a^2} - \frac{a}{2\sqrt{2}a^3} \right)     \\
            & = \ke \frac{e}{a^2} \left( 1 - \frac{1}{2\sqrt{2}} \right).
        \end{align*}

  \item $1 - 1/(2\sqrt{2}) \approx 0.65$ so the field points away from the proton.
\end{enumerate}

\subsubsection{VP21.10.4}

\begin{enumerate}[a)]
  \item The charge per unit length is \[\lambda = \frac{Q}{L}\] so the charge contained in a segment of length $dx$ is \[\lambda\,dx = \frac{Q}{L}\,dx.\]

  \item The field and source points both lie on the $x$-axis, so the differential electric field has no $y$ component. The $x$ component is \[dE_x = -\ke \frac{\lambda \, dx}{x^2} = -\ke \frac{Q}{Lx^2} \, dx.\]

  \item The total electric field at the origin is

        \begin{align*}
          E & = \int dE_x                                                   \\
            & = \int_L^{2L} -\ke \frac{Q}{Lx^2} \, dx                       \\
            & = -\ke \frac{Q}{L} \left[ -\frac{1}{x} \right]_L^{2L}         \\
            & = -\ke \frac{Q}{L} \left( -\frac{1}{2L} + \frac{1}{L} \right) \\
            & = -\ke \frac{Q}{2L^2}.
        \end{align*}
\end{enumerate}

\subsubsection{VP21.14.1}

\begin{enumerate}[a)]
  \item The magnitude of the torque is given by

        \begin{align*}
          \tau & = p E \sin \theta                                 \\
               & = (\num{6.13e-30}) (\num{3.00e5}) \sin \ang{50.0} \\
               & = \qty{1.41e-24}{N.m}.
        \end{align*}

  \item The potential energy is given by

        \begin{align*}
          U & = -p E \cos \theta                                 \\
            & = -(\num{6.13e-30}) (\num{3.00e5}) \cos \ang{50.0} \\
            & = \qty{-1.18e-24}{J}.
        \end{align*}
\end{enumerate}

\subsubsection{VP21.14.2}

To find the magnitude of the charges we can rearrange the torque equation

\begin{align*}
  \tau & = p E \sin \theta                                                      \\
       & = q d E \sin \theta                                                    \\
  q    & = \frac{\tau}{d E \sin \theta}                                         \\
       & = \frac{\num{6.60e-26}}{(\num{1.10e-10}) (\num{8.50e4}) \sin \ang{90}} \\
       & = \qty{7.06e-21}{C}.
\end{align*}

\subsubsection{VP21.14.3}

When the dipole moment is parallel to the field its potential energy is $-p E$ and when it is antiparallel its potential energy is $p E$. Thus, the work required to perform the rotation is $2 p E$ and

\begin{align*}
  W & = 2 p E                                   \\
  p & = \frac{W}{2 E}                           \\
    & = \frac{\num{4.60e-25}}{2 (\num{1.20e5})} \\
    & = \qty{1.92e-30}{C.m}.
\end{align*}

\subsubsection{VP21.14.4}

\newcommand{\kev}{\num{8.988e9}}

\begin{enumerate}[a)]
  \item Rearranging the equation for the dipole moment gives

        \begin{align*}
          p & = q d                                   \\
          d & = \frac{p}{q}                           \\
            & = \frac{\num{3.50e-29}}{\num{1.60e-19}} \\
            & = \qty{2.19e-10}{m}.
        \end{align*}

  \item From Example 21.14, the electric field of the molecule along its axis is \[E_y = \ke \frac{2p}{y^3}.\]

        Rearranging for $y$ and substituting in the desired field strength gives

        \begin{align*}
          y & = \left( \ke \frac{2 p}{E_y} \right)^{1/3}                                             \\
            & = \left( (\kev) \frac{2 (\num{1.60e-19}) (\num{2.19e-10})}{\num{8.00e4}} \right)^{1/3} \\
            & = \qty{1.99e-8}{m}.
        \end{align*}
\end{enumerate}

\subsubsection{Bridging Problem}

By symmetry, each point on the semicircle has a corresponding point on the opposite side of the $y$-axis. The $x$ components of their electric fields cancel, leaving only the $y$ components. Thus, the total electric field at $P$ points in the negative $y$ direction and has magnitude

\begin{align*}
  E & = \int_0^\pi \ke \frac{\lambda a \, d\theta}{a^2} \sin \theta \\
    & = \ke \frac{Q}{\pi a^2} \int_0^\pi \sin \theta \, d\theta     \\
    & = \ke \frac{Q}{\pi a^2} \left[ -\cos \theta \right]_0^\pi     \\
    & = \ke \frac{2Q}{\pi a^2}.
\end{align*}

\section{Gauss's Law}

\setcounter{subsection}{1}
\subsection{Calculating Electric Flux}

\subsubsection{Example 22.1}

\begin{enumerate}[a)]
  \item The electric flux is given by

        \begin{align*}
          \Phi_E & = A E \cos \theta                          \\
                 & = \pi (0.10)^2 (\num{2.0e3}) \cos \ang{30} \\
                 & = \qty{54}{N.m^2/C}.
        \end{align*}

  \item $\Phi_E = 0$

  \item $\Phi_E = A E = \qty{63}{N.m^2/C}$
\end{enumerate}

\subsubsection{Example 22.2}

\begin{enumerate}[a)]
  \item Four of the six sides are parallel to $E$ and thus contribute no flux. The remaining two surfaces contribute $E L^2$ and $-E L^2$ which add to $0$ and also contribute no flux. The total flux is $0$.

  \item Two of the six sides are parallel to $E$ and thus contribute no flux. Two of the remaining four sides contribute $E L^2 \cos \theta$ and $-E L^2 \cos \theta$ which add to $0$ and also contribute no flux. The final two sides contribute $E L^2 \sin \theta$ and $-E L^2 \sin \theta$ which also add to $0$ and contribute no flux. The total flux is $0$.
\end{enumerate}

\subsubsection{Example 22.3}

At all points the electric field is normal to the sphere and its magnitude is \[E = \ke \frac{q}{r^2} = (\kev) \frac{\num{3.0e-6}}{0.20^2} = \qty{6.74e5}{N/C}.\]

The total flux is thus \[\Phi_E = AE = 4 \pi r^2 E = 4 \pi (0.20)^2 (\num{6.74e5}) = \qty{3.4e5}{N.m^2/C}.\]

\setcounter{subsection}{3}
\subsection{Applications of Gauss's Law}

\subsubsection{Example 22.5}

Under electrostatics the electric field inside a conductor is always $\mathbf{0}$.

Using a Gaussian sphere of radius $r > R$ centred on the conducting sphere we can use Gauss's law to calculate the electric field because $\mathbf{E}$ will be perpendicular to $\mathbf{dA}$ and have a constant magnitude

\begin{align*}
  \oint \mathbf{E} \cdot \mathbf{dA} & = \frac{Q_\textrm{enc}}{\epsilon_0}                \\
  E \oint dA                         & = \frac{Q_\textrm{enc}}{\epsilon_0}                \\
  E                                  & = \ke \frac{Q_\textrm{enc}}{r^2}                   \\
  \mathbf{E}                         & = \ke \frac{Q_\textrm{enc}}{r^2} \hat{\mathbf{r}}.
\end{align*}

\subsubsection{Example 22.6}

By symmetry the electric field must only have a radial component. Using a Gaussian cylinder of length $L$ and radius $r$ centred on the line $E$ will be perpendicular to $dA$ and have constant magnitude so we can use Gauss's law to derive the electric field

\begin{align*}
  \oint \mathbf{E} \cdot \mathbf{dA} & = \frac{Q_\textrm{enc}}{\epsilon_0}         \\
  E \oint dA                         & = \frac{L \lambda}{\epsilon_0}              \\
  E                                  & = \frac{L \lambda}{2 \pi r L \epsilon_0}    \\
  \mathbf{E}                         & = \ke \frac{2 \lambda}{r} \hat{\mathbf{r}}.
\end{align*}

\subsubsection{Example 22.7}

By symmetry, the electric field must be directed away from the sheet. Using a Gaussian cube of side $L$, $E$ will be perpendicular to $dA$ on the far sides of the box and have constant magnitude so we can use Gauss's law to dervice the electric field

\begin{align*}
  \oint \mathbf{E} \cdot \mathbf{dA} & = \frac{Q_\textrm{enc}}{\epsilon_0} \\
  2 E L^2                            & = \frac{\sigma L^2}{\epsilon_0}     \\
  E                                  & = \frac{\sigma}{2 \epsilon_0}.
\end{align*}

\subsubsection{Example 22.8}

First we position a Gaussian cylinder such that one end is within the plate and the other is between the plates. The end that is within the plate experiences no flux because there are no electric fields within conductors. We can use Gauss's law to find the electric field at the other end and thus between the plates

\begin{align*}
  \oint \mathbf{E} \cdot \mathbf{dA} & = \frac{Q_\textrm{enc}}{\epsilon_0} \\
  E A                                & = \frac{\sigma A}{\epsilon_0}       \\
  E                                  & = \frac{\sigma}{\epsilon_0}.
\end{align*}

\subsubsection{Example 22.9}

From previous examples we know that for $r > R$ the sphere's electric field is equivalent to that of a point charge \[E = \ke \frac{Q}{r^2}.\]

The sphere is insulating so for $r < R$ its electric field isn't necessarily $\mathbf{0}$. However by symmetry its electric field must only have a radial component so we can use Gauss's law

\begin{align*}
  \oint \mathbf{E} \cdot \mathbf{dA} & = \frac{Q_\textrm{enc}}{\epsilon_0}                                \\
  E \oint dA                         & = \frac{V \rho}{\epsilon_0}                                        \\
  E (4 \pi r^2)                      & = \frac{4}{3} \pi r^3 \frac{Q}{(4/3) \pi R^3} \frac{1}{\epsilon_0} \\
  \mathbf{E}                         & = \ke \frac{Q r}{R^3} \hat{\mathbf{r}}.
\end{align*}

\subsubsection{Example 22.10}

For $r > R$ the electric field is as if all the charge were concentrated at the centre of the sphere. We can rearrange the equation for the electric field of a point charge to find the magnitude of the charge

\begin{align*}
  E & = \ke \frac{q}{r^2}      \\
  q & = 4 \pi \epsilon_0 E r^2 \\
    & = \qty{-1.80e-9}{C}.
\end{align*}

\subsection{Charges on Conductors}

\subsubsection{Example 22.12}

\begin{enumerate}[a)]
  \item Rearranging the equation for the electric field at the surface of a conductor gives

        \begin{align*}
          E_\perp & = \frac{\sigma}{\epsilon_0} \\
          \sigma  & = E_\perp \epsilon_0        \\
                  & = (-150) (\num{8.85e-12})   \\
                  & = -\qty{1.33e-9}{C/m^2}.
        \end{align*}

  \item Multiplying $\sigma$ by the surface area of the Earth gives

        \begin{align*}
          Q & = 4 \pi R^2 \sigma                        \\
            & = 4 \pi (\num{6.38e6})^2 (\qty{-1.33e-9}) \\
            & = \qty{-6.80e5}{C}.
        \end{align*}
\end{enumerate}

\subsection{Guided Practice}

\subsubsection{VP22.4.4}

\begin{enumerate}[a)]
  \item $q_1$ is within the Gaussian surface, so \[\Phi_E = \frac{Q_\textrm{enc}}{\epsilon_0} = \frac{\num{3.00e-9}}{8.854e-12} = \qty{339}{N.m^2/C}.\]

  \item $q_2$ is also within the Guassian surface, so \[\Phi_E = \frac{(\num{3.00e-9}) + (\num{-8.00e-9})}{\num{8.854e-12}} = \qty{-565}{N.m^2/C}.\]

  \item $q_3$ is outside the Guassian surface so the flux is unchanged from part b.
\end{enumerate}

\subsubsection{VP22.10.4}

From Examples 21.12 and 22.8 we know that the electric field between the plates of a capacitor is \[E = \frac{\sigma}{\epsilon_0}.\]

Rearranging the equation for electric force we find

\begin{align*}
  F                  & = qE                                                      \\
                     & = q\frac{\sigma}{\epsilon_0}                              \\
  \Rightarrow \sigma & = \frac{\epsilon_0 F}{q}                                  \\
                     & = \frac{(\num{8.854e-12}) (\num{22.0e-6})}{\num{3.60e-9}} \\
                     & = \qty{5.41e-8}{C/m^2}.
\end{align*}

\subsubsection{VP22.12.1}

The inner surface of the block must have a total charge of $\qty{-3.00}{nC}$ to balance the charge within the cavity, leaving $\qty{-5.00}{nC}$ for the outer surface.

\subsubsection{VP22.12.4}

\begin{enumerate}[a)]
  \item The inner surface must have a total charge of $\qty{-4.00}{nC}$ to balance charge within the cavity, leaving $\qty{-2.00}{nC}$ for the outer surface.

  \item Found in part a: $\qty{-2.00}{nC}$.

  \item $E = \qty{2.50e3}{N/C}$

  \item $E = \qty{1.65e2}{N/C}$
\end{enumerate}

\subsubsection{Bridging Problem}

\begin{enumerate}[a)]
  \item The sphere will include the proton's charge ($+Q$) plus a portion of the electron's charge.

        \begin{align*}
          q & = Q + \int_0^r \rho \, dV                                          \\
            & = Q + \int_0^r -\frac{Q}{\pi a_0^3} e^{-2r'/a_0} 4 \pi r'^2 \, dr' \\
            & = Q - \frac{4Q}{a_0^3} \int_0^r e^{-2r'/a_0} r'^2 \, dr'.          \\
        \end{align*}

        Using integration by parts

        \begin{align*}
          \int e^{-2 r / a_0} r^2 \, dr                    & = -\frac{a_0}{2} e^{-2 r / a_0} r^2 - \int -a_0 e^{-2 r / a_0} r \, dr                                                            \\
                                                           & = -\frac{a_0}{2} e^{-2 r / a_0} r^2 - \left( \frac{a_0^2}{2} e^{-2 r / a_0} r - \int \frac{a_0^2}{2} e^{-2 r / a_0} \, dr \right) \\
                                                           & = -\frac{a_0}{2} e^{-2 r / a_0} r^2 - \frac{a_0^2}{2} e^{-2 r / a_0} r - \frac{a_0^3}{4} e^{-2 r / a_0} + c                       \\
                                                           & = -\frac{a_0}{4} e^{-2 r / a_0} \left( 2 r^2 + 2 a_0 r + a_0^2 \right) + c                                                        \\
          \Rightarrow \int_0^r e^{-2 r' / a_0} r'^2 \, dr' & = \frac{1}{4} a_0 \left( a_0^2 - e^{-2 r / a_0} \left( 2 r^2 + 2 a_0 r + a_0^2 \right) \right)
        \end{align*}

        Therefore

        \begin{align*}
          q & = Q - \frac{4 Q}{a_0^3} \frac{1}{4} a_0 \left( a_0^2 - e^{-2 r / a_0} \left( 2 r^2 + 2 a_0 r + a_0^2 \right) \right) \\
            & = Q \left( 1 - 1 + \frac{e^{-2 r / a_0}}{a_0^2} \left( 2 r^2 + 2 a_0 r + a_0^2 \right) \right)                       \\
            & = Q e^{-2 r / a_0} \left( 2 \left( \frac{r}{a_0} \right)^2 + 2 \frac{r}{a_0} + 1 \right).
        \end{align*}

  \item By Gauss's law and symmetry we know that the electric field will only have a radial component and will have magnitude \[E = \ke \frac{q}{r^2}.\]
\end{enumerate}

\subsection{Exercises}

\subsubsection{22.2}

The electric field is constant so we can calculate the electric flux as

\begin{align*}
  \Phi_E & = \mathbf{A} \cdot \mathbf {E}       \\
         & = A E \cos \theta                    \\
         & = (0.400)(0.600)(90.0) \cos \ang{70} \\
         & = \qty{7.4}{N.m^2/C}.
\end{align*}

\subsubsection{22.9}

\begin{enumerate}[a)]
  \item Assuming the plastic sphere has no charge, the electric field is $\mathbf{0}$.

  \item By Gauss's law and symmetry the electric field is as if all charge were concentrated at the centre of the sphere, giving $E = \qty{-1.22e8}{N/C}$.

  \item Using the same logic as in part b, $E = \qty{-3.64e7}{N/C}$.
\end{enumerate}

\subsubsection{22.13}

The electric field of an infinite line charge is \[\mathbf{E} = \ke \frac{2 \lambda}{r} \hat{\mathbf{r}}.\]

The force experienced by the section is

\begin{align*}
  F & = \int dF                                                   \\
    & = \int E \, dq                                              \\
    & = \int_0^L \ke \frac{2 \lambda}{r} \lambda \, dx            \\
    & = \ke \frac{2 \lambda^2}{r} \int_0^L dx                     \\
    & = (\num{8.988e9}) \frac{2 (\num{5.20e-6})^2}{0.300}(0.0500) \\
    & = \qty{0.0810}{N}.
\end{align*}

\subsubsection{22.23}

\begin{enumerate}[a)]
  \item The electric field of an infinite plane is \[E = \frac{\sigma}{2 \epsilon_0}.\]

        The point charge and sheet have opposite charges so are attracted to each other. The work done on the electron by the electric field of the sheet is \[W = F d = q \frac{\sigma}{2 \epsilon_0} d = (\num{1.60e-19}) \frac{\num{2.90e-12}}{2 (\num{8.85e-12})} (0.250) = \qty{6.56e-21}{J}.\]

  \item By the work-energy theorem, we can use the kinetic energy of the electron to calculate its velocity

        \begin{align*}
          W             & = \frac{1}{2} m v^2                                \\
          \Rightarrow v & = \sqrt{\frac{2 W}{m}}                             \\
                        & = \sqrt{\frac{2 (\num{6.56e-21})}{\num{9.11e-31}}} \\
                        & = \qty{1.20e5}{m/s}.
        \end{align*}
\end{enumerate}

\subsubsection{22.34}

The electric field is non-uniform so to calculate the electric flux we must perform an integral

\begin{align*}
  \Phi_E & = \int E \, dA                           \\
         & = \int_0^L \int_0^L kx \, dx \, dy       \\
         & = k L \left[ \frac{1}{2} x^2 \right]_0^L \\
         & = \frac{1}{2} k L^3                      \\
         & = \frac{1}{2} (964) (0.350)^3            \\
         & = \qty{20.7}{N.m^2/C}.
\end{align*}

\subsubsection{22.45}

\begin{enumerate}[a)]
  \item The charge contained within a sphere of radius $r$ is given by

        \begin{align*}
          q & = \int_a^r \rho \, dV                              \\
            & = \int_a^r \frac{\alpha}{r} 4 \pi r'^2 \, dr'      \\
            & = 4 \pi \alpha \int_a^r r' \, dr'                  \\
            & = 4 \pi \alpha \left[ \frac{1}{2} r'^2 \right]_a^r \\
            & = 2 \pi \alpha (r^2 - a^2).
        \end{align*}

        By symmetry and Gauss's law, the electric field at radius $r$ is the same as if all the charge were concentrated at the centre of the sphere so

        \begin{align*}
          E & = \ke \frac{q}{r^2}                         \\
            & = \ke \frac{2 \pi \alpha (r^2 - a^2)}{r^2}.
        \end{align*}

  \item The net electric field must be equal at $r=a$ and $r=b$

        \begin{align*}
          \ke \frac{1}{a^2} (2 \pi \alpha (a^2 - a^2) + q) & = \ke \frac{1}{b^2} (2 \pi \alpha (b^2 - a^2) + q) \\
          q                                                & = \frac{a^2}{b^2} (2 \pi \alpha (b^2 - a^2) + q)   \\
          q \left(1 - \frac{a^2}{b^2}\right)               & = \frac{a^2}{b^2} 2 \pi \alpha (b^2 - a^2)         \\
          q                                                & = 2 \pi \alpha a^2.
        \end{align*}

  \item The net electric field between $r=a$ and $r=b$ will be

        \begin{align*}
          E & = \ke \frac{1}{r^2} (2 \pi \alpha (r^2 - a^2) + 2 \pi \alpha a^2) \\
            & = \frac{\alpha}{2 \epsilon_0}.
        \end{align*}
\end{enumerate}

\subsubsection{22.61}

\begin{enumerate}[a)]
  \item By Gauss's law and symmetry, the electric field of the sphere is the same as if the charge were concentrated at its centre. Thus the electric flux through the round side of the cylinder is

        \begin{align*}
          \Phi_a & = \int \mathbf{E} \cdot \mathbf{dA}                                                                                              \\
                 & = \int_{-L / 2}^{L / 2} \ke \frac{Q}{r'^2} \cos \theta 2 \pi R \, dz                                                             \\
                 & = \frac{Q R^2}{2 \epsilon_0} \int_{-L / 2}^{L / 2} (R^2 + z^2)^{-3/2} \, dz                                                      \\
                 & = \frac{Q R^2}{2 \epsilon_0} \left[ \frac{z}{R^2 \sqrt{R^2 + z^2}} \right]_{-L / 2}^{L / 2}                                      \\
                 & = \frac{Q R^2}{2 \epsilon_0} \left( \frac{L / 2}{R^2 \sqrt{R^2 + (L / 2)^2}} + \frac{L / 2}{R^2 \sqrt{R^2 + (-L / 2)^2}} \right) \\
                 & = \frac{Q L}{2 \epsilon_0 \sqrt{R^2 + (L / 2)^2}}.
        \end{align*}

  \item The electric flux through the top of the cylinder is

        \begin{align*}
          \Phi_b & = \int \mathbf{E} \cdot \mathbf{dA}                                               \\
                 & = \int_0^R \ke \frac{Q}{r'^2} \cos \theta 2 \pi x \, dx                           \\
                 & = \frac{Q}{2 \epsilon_0} \int_0^R \frac{(L / 2) x}{((L / 2)^2 + x^2)^{3/2}} \, dx \\
                 & = \frac{Q L}{4 \epsilon_0} \int_0^R x ((L / 2)^2 + x^2)^{-3/2} \, dx
        \end{align*}

        Let $u = (L / 2)^2 + x^2$ so $du = 2 x \, dx$ and thus

        \begin{align*}
          \Phi_b & = \frac{Q L}{8 \epsilon_0} \int_{(L / 2)^2}^{(L / 2)^2 + R^2} u^{-3/2} \, du                           \\
                 & = \frac{Q L}{8 \epsilon_0} \left[ -2 u^{-1/2} \right]_{(L / 2)^2}^{(L / 2)^2 + R^2}                    \\
                 & = \frac{Q L}{4 \epsilon_0} \left( \frac{1}{\sqrt{(L / 2)^2}} -\frac{1}{\sqrt{(L / 2)^2 + R^2}} \right) \\
                 & = \frac{Q L}{4 \epsilon_0} \left( \frac{2}{L} - \frac{1}{\sqrt{(L / 2)^2 + R^2}} \right).
        \end{align*}

  \item The same as in part b.
\end{enumerate}

\end{document}