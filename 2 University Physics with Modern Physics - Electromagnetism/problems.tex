\documentclass{article}
\usepackage{amsmath} % For align*
\usepackage{enumerate} % For customisable list labels
\usepackage{siunitx} % For units

\title{University Physics with Modern Physics Electromagnetism Problems}
\author{Chris Doble}
\date{December 2022}

\begin{document}

\maketitle

\tableofcontents

\setcounter{section}{20}
\section{Electric Charge and Electric Field}

\setcounter{subsection}{2}
\subsection{Coulomb's Law}

\subsubsection{Example 21.1}

The magnitude of electric repulsion between two $\alpha$ particles is given by \[F_e = \frac{1}{4\pi\epsilon_0}\frac{q^2}{r^2}\] and the magnitude of gravitational attraction is given by \[F_g = \frac{Gm^2}{r^2}\]. The ratio of the two values is

\begin{align*}
  \frac{F_e}{F_g} & = \frac{1}{4\pi\epsilon_0}\frac{q^2}{r^2}\frac{r^2}{Gm^2} \\
                  & = \frac{1}{4\pi\epsilon_0}\frac{q^2}{Gm^2}                \\
                  & = 3.1\times10^{35}
\end{align*}

showing that the electric repulsion is significantly stronger than the gravitational attraction.

\subsubsection{Example 21.2}

\begin{enumerate}[a)]
  \item The magnitude of the force that $q_1$ exerts on $q_2$ is
        \begin{align*}
          F & = \frac{1}{4\pi\epsilon_0}\frac{q_1q_2}{r^2}                                 \\
            & = (9.0 \times 10^9)\frac{|(25 \times 10^{-9})(-75 \times 10^{-9})|}{0.030^2} \\
            & = 1.9 \times 10^{-2}\,\textrm{N}.
        \end{align*}
        Since $q_1$ and $q_2$ have opposite charge, the force is attractive (from $q_2$ to $q_1$).

  \item The magnitude of the force that $q_2$ exerts on $q_1$ is the same as in part a, but the direction is reversed (from $q_1$ to $q_2$).
\end{enumerate}

\subsubsection{Example 21.3}

By the principle of superposition of forces, the net force exerted on $q_3$ is equal to the vector sum of the forces exerted on it by $q_1$ and $q_2$ separately.

Both $q_1$ and $q_3$ have positive charge so they repel each other. $q_1$ is to the right of $q_3$ so $q_3$ experiences a force to the left of magnitude

\begin{align*}
  F_\textrm{1 on 3} & = \frac{1}{4\pi\epsilon_0}\frac{|q_1q_3|}{r^2}                                \\
                    & = (9.0 \times 10^9)\frac{|(1.0 \times 10^{-9})(5.0 \times 10^{-9})|}{0.020^2} \\
                    & = 1.1 \times 10^{-4}\,\textrm{N}.
\end{align*}

However $q_2$ has a negative charge so it attracts $q_3$. It is also to the right of $q_3$ so $q_3$ experiences a force to the right of magnitude

\begin{align*}
  F_\textrm{2 on 3} & = \frac{1}{4\pi\epsilon_0}\frac{|q_2q_3|}{r^2}                                \\
                    & = (9.0 \times 10^9)\frac{|(-3.0 \times 10^{-9})(5.0 \times 10^{-9})}{0.040^2} \\
                    & = 8.4 \times 10^{-5}\,\textrm{N}.
\end{align*}

The net force experienced by $q_3$ is therefore

\begin{align*}
  F & = -F_\textrm{1 on 3} + F_\textrm{2 on 3}   \\
    & = -1.1 \times 10^{-4} + 8.4 \times 10^{-5} \\
    & = -2.6 \times 10^{-5}\,\textrm{N}.
\end{align*}

\subsubsection{Example 21.4}

Since $q_1$ and $q_2$ are of equal charge and are symmetric about the x axis on which $Q$ lies, the vertical components of their forces cancel leaving only the horizontal.

The horizontal component of $q_1$'s force on $Q$ is given by

\begin{align*}
  F_\textrm{1 on Q, x} & = \frac{1}{4\pi\epsilon_0} \frac{q_1Q}{r_\textrm{1,Q}^2} \cos\alpha                                             \\
                       & = (9.0 \times 10^9) \frac{(2.0 \times 10^{-6})(4.0 \times 10^{-6})}{\sqrt{0.30^2 + 0.40^2}^2} \frac{0.40}{0.50} \\
                       & = 0.23\,\textrm{N}.
\end{align*}

Again, since $q_1$ and $q_2$ are of equal charge and symmetric about the x axis, $F_\textrm{1 on Q, x} = F_\textrm{2 on Q, x}$ and the total force experienced by $Q$ is in the positive x direction of magnitude \[F = 2F_\textrm{1 on Q, x} = 0.46\,\textrm{N}.\]

\subsection{Electric Field and Electric Forces}

\subsubsection{Example 21.5}

The magnitude of the electric field vector is given by

\begin{align*}
  E & = \frac{1}{4\pi\epsilon_0} \frac{|q|}{r^2}             \\
    & = (9.0 \times 10^9) \frac{|4.0 \times 10^{-9}|}{2.0^2} \\
    & = 9.0\,\textrm{N}/\textrm{C}.
\end{align*}

\subsubsection{Example 21.6}

The magnitude of the electric field vector is given by

\begin{align*}
  E & = \frac{1}{4\pi\epsilon_0} \frac{|q|}{r^2}                      \\
    & = (9.0 \times 10^9) \frac{|-8.0 \times 10^{-9}|}{1.2^2 + 1.6^2} \\
    & = 18\,\textrm{N}/\textrm{C}
\end{align*}

and it is directed towards the origin. If $\theta$ is the angle between the positive x axis and $\hat{\mathbf{r}}$ then the component form of $\mathbf{E}$ is

\begin{align*}
  E & = -E\left(\cos\theta\hat{\mathbf{i}} + \sin\theta\hat{\mathbf{j}}\right)                      \\
    & = -E\left(\frac{x}{r}\hat{\mathbf{i}} + \frac{-y}{r}\hat{\mathbf{j}}\right)                   \\
    & = \frac{-18}{\sqrt{1.2^2 + 1.6^2}}\left(1.2\hat{\mathbf{i}} + 1.6\hat{\mathbf{j}}\right)      \\
    & = (-11\,\textrm{N}/\textrm{C})\hat{\mathbf{i}} - (14\,\textrm{N}/\textrm{C})\hat{\mathbf{j}}.
\end{align*}

\subsubsection{Example 21.7}

\begin{enumerate}[a)]
  \item Electrons have a negative charge and the electric field is directed upwards, so the electron will move downwards. The magnitude of its acceleration is

        \begin{align*}
          a & = \frac{F}{m}                                                           \\
            & = \frac{eE}{m}                                                          \\
            & = \frac{(1.60 \times 10^{-19})(1.00 \times 10^4)}{9.11 \times 10^{-31}} \\
            & = 1.76 \times 10^{15}\,\textrm{m}/\textrm{s}^2.
        \end{align*}

  \item Its acceleration is constant between the plates, so its final speed is

        \begin{align*}
          v^2 & = v_0^2 + 2a(x - x_0)                      \\
              & = 2ax                                      \\
          v   & = \sqrt{2ax}                               \\
              & = \sqrt{2(1.76 \times 10^{15})(0.01)}      \\
              & = 5.9 \times 10^6\,\textrm{m}/\textrm{s}^2
        \end{align*}

        and thus its final kinetic energy is

        \begin{align*}
          K & = \frac{1}{2}mv^2                                      \\
            & = \frac{1}{2}(9.11 \times 10^{-31})(5.9 \times 10^6)^2 \\
            & = 1.6 \times 10^{-17}\,\textrm{J}.
        \end{align*}

  \item We can find the time it takes for the electron to travel this distance by rearranging the kinematic equation \[v = v_0 + at\] to

        \begin{align*}
          t & = \frac{v - v_0}{a}                           \\
            & = \frac{5.9 \times 10^6}{1.76 \times 10^{15}} \\
            & = 3.4 \times 10^{-9}\,\textrm{s}.
        \end{align*}
\end{enumerate}

\subsection{Electric-Field Calculations}

\subsubsection{Example 21.8}

\begin{enumerate}[a)]
  \item At point $a$ the electric field caused by $q_1$ points to the right and has magnitude

        \begin{align*}
          E_1 & = \frac{1}{4\pi\epsilon_0} \frac{|q_1|}{r^2}            \\
              & = (9.0 \times 10^9) \frac{12 \times 10^{-9}}{(0.060)^2} \\
              & = 3.0 \times 10^4\,\textrm{N}/\textrm{C}.
        \end{align*}

        The electric field caused by $q_2$ also points to the right and it has magnitude

        \begin{align*}
          E_2 & = \frac{1}{4\pi\epsilon_0} \frac{|q_2|}{r^2}               \\
              & = (9.0 \times 10^9) \frac{|-12 \times 10^{-9}|}{(0.040)^2} \\
              & = 6.8 \times 10^4\,\textrm{N}/\textrm{C}.
        \end{align*}

        Thus the total field points to the right and has magnitude \[E = E_1 + E_2 = 9.8 \times 10^4\,\textrm{N}/\textrm{C}.\]

  \item At point $b$ the electric field caused by $q_1$ points to the left and has magnitude

        \begin{align*}
          E_1 & = \frac{1}{4\pi\epsilon_0} \frac{|q_1|}{r^2}            \\
              & = (9.0 \times 10^9) \frac{12 \times 10^{-9}}{(0.040)^2} \\
              & = 6.8 \times 10^4\,\textrm{N}/\textrm{C}.
        \end{align*}

        The electric field caused by $q_2$ points to the right and has magnitude

        \begin{align*}
          E_2 & = \frac{1}{4\pi\epsilon_0} \frac{|q_2|}{r^2}               \\
              & = (9.0 \times 10^9) \frac{|-12 \times 10^{-9}|}{(0.140)^2} \\
              & = 0.55 \times 10^4\,\textrm{N}/\textrm{C}.
        \end{align*}

        Thus the total electric field points to the left and has magnitude \[E = E_1 - E_2 = 6.3 \times 10^4\,\textrm{N}/\textrm{C}.\]

  \item At point $c$ the electric field caused by $q_1$ points from $q_1$ to $c$ and has magnitude

        \begin{align*}
          E_1 & = \frac{1}{4\pi\epsilon_0} \frac{|q_1|}{r^2}            \\
              & = (9.0 \times 10^9) \frac{|12 \times 10^{-9}|}{0.130^2} \\
              & = 6.4 \times 10^3\,\textrm{N}/\textrm{C}.
        \end{align*}

        The electric field caused by $q_2$ points from $c$ to $q_2$ and has magnitude

        \begin{align*}
          E_2 & = \frac{1}{4\pi\epsilon_0} \frac{|q_2|}{r^2}             \\
              & = (9.0 \times 10^9) \frac{|-12 \times 10^{-9}|}{0.130^2} \\
              & = 6.4 \times 10^3\,\textrm{N}/\textrm{C}                 \\
              & = E_1.
        \end{align*}

        The vertical components of $\mathbf{E_1}$ and $\mathbf{E_2}$ cancel, leaving only a horizontal component pointing to the right of magnitude

        \begin{align*}
          E & = 2E_1\cos\alpha                          \\
            & = 2(6.4 \times 10^3)\frac{0.050}{0.130}   \\
            & = 4.9 \times 10^3\,\textrm{N}/\textrm{C}.
        \end{align*}
\end{enumerate}

\subsubsection{Example 21.9}

By symmetry, each point on the ring has a corresponding point on the opposite side. The components of their electric fields perpendicular to the axis of the ring cancel, leaving only a component parallel to the axis of the ring. Thus the total magnetic field at $P$ is parallel to the axis of the ring and can be calculated as

\begin{align*}
  E & = \int_0^{2\pi} \frac{1}{4\pi\epsilon_0} \frac{\lambda}{r^2} \cos \alpha\,d\theta \\
    & = \frac{1}{4\pi\epsilon_0} \frac{Qx}{2\pi(a^2 + x^2)^{3/2}} \int_0^{2\pi} d\theta \\
    & = \frac{1}{4\pi\epsilon_0} \frac{Qx}{(a^2 + x^2)^{3/2}}.
\end{align*}

\subsubsection{Example 21.10}

By symmetry, each point on the line has a corresponding point on the opposite side of the $x$-axis. The $y$ components of their electric fields cancel, leaving only the $x$ components. Thus the total magnetic field at $P$ only has an $x$ component and can be calculated as

\begin{align*}
  E & = \int_{-a}^a \frac{1}{4\pi\epsilon_0} \frac{\lambda}{r^2} \cos\alpha\,dy                                        \\
    & = \frac{1}{4\pi\epsilon_0} \frac{Qx}{2a} \int_{-a}^a \frac{1}{(x^2 + y^2)^{3/2}}\,dy                             \\
    & = \frac{1}{4\pi\epsilon_0} \frac{Qx}{2a} \left[\frac{y}{x^2\sqrt{x^2+y^2}}\right]_{-a}^a                         \\
    & = \frac{1}{4\pi\epsilon_0} \frac{Q}{2ax} \left(\frac{a}{\sqrt{x^2 + a^2}} + \frac{a}{\sqrt{x^2 + (-a)^2}}\right) \\
    & = \frac{1}{4\pi\epsilon_0} \frac{Q}{x\sqrt{x^2 + a^2}}.
\end{align*}

\subsubsection{Example 21.11}

By symmetry, each point on the disk has a corresponding point 180° rotation around the $x$-axis. The $y$ and $z$ components of their electric fields cancel, leaving only the $x$ components. Thus the total magnetic field at $P$ only has an $x$ component and can be calculated as

\begin{align*}
  E & = \int_0^R \int_0^{2\pi} \frac{1}{4\pi\epsilon_0} \frac{\sigma}{r^2} s \cos\alpha\,d\theta\,ds                       \\
    & = \frac{\sigma}{4\pi\epsilon_0} \int_0^R \int_0^{2\pi} \frac{s}{s^2 + x^2} \frac{x}{\sqrt{s^2 + x^2}} \,d\theta \,ds \\
    & = \frac{\sigma x}{2\epsilon_0} \int_0^R \frac{s}{(s^2 + x^2)^{3/2}} \,ds                                             \\
    & = \frac{\sigma x}{2\epsilon_0} \left[-\frac{1}{\sqrt{s^2 + x^2}}\right]_0^R                                          \\
    & = \frac{\sigma x}{2\epsilon_0} \left(-\frac{1}{\sqrt{R^2 + x^2}} + \frac{1}{x}\right)                                \\
    & = \frac{\sigma}{2\epsilon_0} \left(1 - \frac{1}{\sqrt{(R/x)^2 + 1}}\right).
\end{align*}

\subsubsection{Example 21.12}

From Example 21.11 we know that the electric field produced by an infinite plane sheet of charge is \[E= \frac{\sigma}{2\epsilon_0}.\] Therefore the electric field outside the sheets is $\mathbf{0}$ and between the sheets is $\sigma/\epsilon_0$ towards the negative sheet.

\subsection{Electric Dipoles}

\subsubsection{Example 21.13}

\begin{enumerate}[a)]
  \item The electric field is uniform so the net force exerted on the dipole is $\mathbf{0}$

  \item The electric dipole moment is directed from the negative charge to the positive charge and has magnitude \[p = qd = (1.6 \times 10^{-19})(0.125 \times 10^{-9}) = 2.0 \times 10^{-29}\,\textrm{C}\cdot\textrm{m}\]

  \item The torque aligns the electric dipole moment with the electric field so it is directed out of the page and has magnitude \[\tau = qEd\sin\phi = (1.6 \times 10^{-19})(5.0 \times 10^5)(0.125 \times 10^{-9})\sin 35 = 5.7 \times 10^{-24}\,\textrm{N}\cdot\textrm{m}\]

  \item The potential energy of an electric dipole in a uniform electric field is given by \[U = -qdE\cos\phi = (2.0 \times 10^{-29})(5.0 \times 10^5)\cos 35 = 8.2 \times 10^{-24}\,\textrm{J}\]
\end{enumerate}

\subsubsection{Example 21.14}

As $P$ is on the $y$-axis, the electric fields of the electric dipole's point charges have no $x$ component and thus the net electric field is directed along the $y$-axis.

By the principle of superposition of electric fields, the magnitude of the electric field at $P$ is

\begin{align*}
  E & = E_- + E_+                                                                                                                         \\
    & = \frac{1}{4\pi\epsilon_0} \frac{-q}{(y - (-d/2))^2} + \frac{1}{4\pi\epsilon_0} \frac{q}{(y - d/2)^2}                               \\
    & = \frac{1}{4\pi\epsilon_0}q\left(\frac{1}{(y-d/2)^2} - \frac{1}{(y + d/2)^2}\right)                                                 \\
    & = \frac{1}{4\pi\epsilon_0} \frac{q}{y^2} \left( \left( 1 - \frac{d}{2y} \right)^{-2} - \left( 1 + \frac{d}{2y} \right)^{-2} \right) \\
    & \approx \frac{1}{4\pi\epsilon_0} \frac{q}{y^2} \left( 1 + \frac{d}{y} - 1 + \frac{d}{y} \right)                                     \\
    & = \frac{qd}{2\pi\epsilon_0y^3}                                                                                                      \\
    & = \frac{p}{2\pi\epsilon_0y^3}.
\end{align*}

\subsection{Guided Practice}

\subsubsection{VP21.4.1}

$q_1$ attracts $q_3$ to the left with magnitude

\begin{align*}
  F_1 & = \frac{1}{4\pi\epsilon_0} \frac{|q_1q_3|}{r^2}                                   \\
      & = (9.0 \times 10^9) \frac{|(4.00 \times 10^{-9})(-2.00 \times 10^{-9})}{0.0400^2} \\
      & = 4.5 \times 10^{-5} \,\textrm{N}.
\end{align*}

$q_2$ repels $q_3$ to the left with magnitude

\begin{align*}
  F_2 & = \frac{1}{4\pi\epsilon_0} \frac{|q_2 q_3|}{r^2}                                              \\
      & = (9.0 \times 10^9) \frac{|(-1.20 \times 10^{-9})(-2.00 \times 10^{-9}|}{(0.0600 - 0.0400)^2} \\
      & = 5.4 \times 10^{-5} \,\textrm{N}.
\end{align*}

By the principle of superposition of forces, the net force on $q_3$ is \[\mathbf{F} = (-F_1 - F_2)\hat{\mathbf{i}} = (-9.9 \times 10^{-5} \,\textrm{N}) \hat{\mathbf{i}}.\]

\subsubsection{VP21.4.2}

\begin{enumerate}[a)]
  \item $q_1$ repels $q_2$ in the positive $x$ direction with magnitude

        \begin{align*}
          F_1 & = \frac{1}{4\pi\epsilon_0} \frac{|q_1 q_2|}{r^2}                                                       \\
              & = \frac{1}{4 \pi (8.854 \times 10^{-12})} \frac{|(3.60 \times 10^{-9})(2.00 \times 10^{-9})}{0.0400^2} \\
              & = 40.4 \,\mu\textrm{N}.
        \end{align*}

  \item By the superposition of forces

        \begin{align*}
          F   & = F_1 + F_2            \\
          F_2 & = F - F_1              \\
              & = 54.0 - 40.4          \\
              & = 13.6 \,\mu\textrm{N}
        \end{align*}

        in the positive $x$ direction.

  \item $q_2$ repels $q_3$ so it must also have a positive charge of magnitude

        \begin{align*}
          F   & = \frac{1}{4\pi\epsilon_0} \frac{q_2 q_3}{r^2}                                           \\
          q_2 & = \frac{4\pi\epsilon_0 F r^2}{q_3}                                                       \\
              & = \frac{4\pi(8.854 \times 10^{-12})(1.36 \times 10^{-5})(0.0800)^2}{2.00 \times 10^{-9}} \\
              & = 4.84 \times 10^{-9} \,\textrm{C}.
        \end{align*}
\end{enumerate}

\subsubsection{VP21.4.3}

By symmetry the $x$ components of $q_1$ and $q_2$'s electric fields cancel leaving only their $y$ components which are directed in the negative $y$ direction and equal. $q_3$ is negative and thus experiences a net force in the positive $y$ direction of magnitude

\begin{align*}
  F & = 2 \frac{1}{4\pi\epsilon_0} \frac{q_1q_3}{r^2} \sin \alpha                                                                                            \\
    & = 2 \frac{1}{4\pi (8.854 \times 10^{-12})} \frac{(6.00 \times 10^{-9})(2.50 \times 10^{-9})}{0.150^2 + 0.200^2} \frac{0.200}{\sqrt{0.150^2 + 0.200^2}} \\
    & = 3.45 \times 10^{-6} \,\textrm{N}.
\end{align*}

\subsubsection{VP21.4.4}

The magnitude of the electric force exerted by $q_1$ on $q_3$ is

\begin{align*}
  F_1 & = \frac{1}{4 \pi \epsilon_0} \frac{|q_1 q_3|}{r^2}                                                                \\
      & = \frac{1}{4 \pi (8.854 \times 10^{-12})} \frac{|(4.00 \times 10^{-9}) (-1.50 \times 10^{-9})}{0.250^2 + 0.200^2} \\
      & = 5.26 \times 10^{-7} \,\textrm{N}.
\end{align*}

They have opposite charges so the force is directed from $q_3$ to $q_1$. In component form the force is

\begin{align*}
  \mathbf{F_1} & = -F_1\cos\alpha\hat{\mathbf{i}} + F_1\sin\alpha\hat{\mathbf{j}}                                                     \\
               & = F_1 \left( -\frac{x}{r}\hat{\mathbf{i}} + \frac{y}{r}\hat{\mathbf{j}} \right)                                      \\
               & = \frac{5.26 \times 10^{-7}}{\sqrt{0.250^2 + 0.200^2}} \left( -0.250\hat{\mathbf{i}} + 0.200\hat{\mathbf{j}} \right) \\
               & = (-4.11 \times 10^{-7}\,\textrm{N}) \hat{\mathbf{i}} + (3.29 \times 10^{-7}\,\textrm{N}) \hat{\mathbf{j}}.
\end{align*}

The magnitude of the electric force exerted by $q_2$ on $q_3$ is

\begin{align*}
  F_2 & = \frac{1}{4\pi\epsilon_0} \frac{|q_2 q_3|}{r^2}                                                          \\
      & = \frac{1}{4 \pi (8.854 \times 10^{-12})} \frac{|(-4.00 \times 10^{-9}) (-1.50 \times 10^{-9})|}{0.250^2} \\
      & = 8.63 \times 10^{-7} \,\textrm{N}.
\end{align*}

The have like charges so the force is directed from $q_2$ to $q_3$, i.e. along the positive $x$-axis. In component form the force is \[\mathbf{F_2} = (8.64 \times 10^{-7} \,\textrm{N})\hat{\mathbf{i}}.\]

Thus the net force experienced by $q_3$ is

\begin{align*}
  \mathbf{F} & = \mathbf{F_1} + \mathbf{F_2}                                                                                \\
             & = (4.53 \times 10^{-7} \,\textrm{N}) \hat{\mathbf{i}} + (3.29 \times 10^{-7} \,\textrm{N}) \hat{\mathbf{j}}.
\end{align*}

\subsubsection{VP21.10.1}

\begin{enumerate}[a)]
  \item The source points and field point all lie on the $y$-axis, so the source points' electric fields have no $x$ components. $q_1$ is positive and the field point is below it, so its contribution is negative. $q_2$ is negative and the field point is above it, so its contribution is also negative. Thus the $y$ component of the net electric field is

        \begin{align*}
          E_y & = \frac{1}{4 \pi \epsilon_0} \left( -\frac{q_1}{(y_1 - y)^2} + \frac{q_2}{(y_2 - y)^2} \right)                                                     \\
              & = \frac{1}{4 \pi (8.854 \times 10^{-12})} \left( \frac{4.00 \times 10^{-9}}{(0.200 - 0.100)^2} - \frac{5.00 \times 10^{-9}}{(0 - 0.100)^2} \right) \\
              & = -8.09 \times 10^3 \,\textrm{N}/\textrm{C}.
        \end{align*}

  \item The source points and field point all lie on the $y$-axis, so the source points' electric fields have no $x$ components. $q_1$ is positive and the field point is above it, so its contribution is positive. $q_2$ is negative and the field point is above it, so its contribution is also negative. Thus the $y$ component of the net electric field is

        \begin{align*}
          E_y & = \frac{1}{4 \pi \epsilon_0} \left( \frac{q_1}{(y_1 - y)^2} + \frac{q_2}{(y_2 - y)^2} \right)                                                      \\
              & = \frac{1}{4 \pi (8.854 \times 10^{-12})} \left( \frac{4.00 \times 10^{-9}}{(0.200 - 0.400)^2} - \frac{5.00 \times 10^{-9}}{(0 - 0.400)^2} \right) \\
              & = 618 \,\textrm{N}/\textrm{C}.
        \end{align*}

  \item The electric field of $q_1$ has magnitude

        \begin{align*}
          E_1 & = \frac{1}{4 \pi \epsilon_0} \frac{q_1}{r^2}                                            \\
              & = \frac{1}{4 \pi (8.854 \times 10^{-12})} \frac{4.00 \times 10^{-9}}{0.200^2 + 0.200^2} \\
              & = 449 \,\textrm{N}/\textrm{C}.
        \end{align*}

        It is directed from $q_1$ to the field point and thus in component form is

        \begin{align*}
          \mathbf{E_1} & = E_1(\cos\phi\hat{\mathbf{i}} - \sin\phi\hat{\mathbf{j}})                                       \\
                       & = \frac{449}{\sqrt{0.200^2 + 0.200^2}}(0.200\hat{\mathbf{i}} - 0.200\hat{\mathbf{j}})            \\
                       & = (317 \,\textrm{N}/\textrm{C})\hat{\mathbf{i}} - (317 \,\textrm{N}/\textrm{C})\hat{\mathbf{j}}.
        \end{align*}

        $q_2$ and the field point both lie on the $x$-axis, and thus its electric field has no $y$ component. In component form it is

        \begin{align*}
          \mathbf{E_2} & = \frac{1}{4 \pi \epsilon_0} \frac{q_2}{r^2} \hat{\mathbf{i}}                                   \\
                       & = \frac{1}{4 \pi (8.854 \times 10^{-12})} \frac{-5.00 \times 10^{-9}}{0.200^2} \hat{\mathbf{i}} \\
                       & = (-1.12 \times 10^{3} \,\textrm{N}/\textrm{C})\hat{\mathbf{i}}.
        \end{align*}

        The total electric field is thus

        \begin{align*}
          \mathbf{E} & = \mathbf{E_1} + \mathbf{E_2}                                                                                                \\
                     & = (-8.03 \times 10^2 \,\textrm{N}/\textrm{C})\hat{\mathbf{i}} + (-3.17 \times 10^2 \,\textrm{N}/\textrm{C})\hat{\mathbf{j}}.
        \end{align*}
\end{enumerate}

\subsubsection{VP21.10.2}

\begin{enumerate}[a)]
  \item Both source points and the field point are on the $x$-axis, so the electric fields at $P$ have no $y$ components.

        $q_1$ is positive and $P$ is to the right of $q_1$, so its electric field points to the right and has magnitude

        \begin{align*}
          E_1 & = \frac{1}{4 \pi \epsilon_0} \frac{q_1}{r^2}                                   \\
              & = \frac{1}{4 \pi (8.854 \times 10^{-12})} \frac{1.80 \times 10^{-9}}{0.0200^2} \\
              & = 4.04 \times 10^4 \,\textrm{N}/\textrm{C}.
        \end{align*}

  \item The magnitude and direction of the electric field that $q_2$ causes at $P$ can be calculated as

        \begin{align*}
          E   & = E_1 + E_2                                 \\
          E_2 & = E - E_1                                   \\
              & = 6.75 \times 10^4 - 4.04 \times 10^4       \\
              & = 2.71 \times 10^4 \,\textrm{N}/\textrm{C}.
        \end{align*}

  \item $E_2$ is positive at $P$, so $q_2$ must be negative. Its value is

        \begin{align*}
          E_2 & = -\frac{1}{4 \pi \epsilon_0} \frac{q_2}{r^2}                  \\
          q_2 & = -4 \pi \epsilon_0 E_2 r^2                                    \\
              & = -4 \pi (8.854 \times 10^{-12}) (2.71 \times 10^4) (0.0200)^2 \\
              & = -1.21 \times 10^{-9} \,\textrm{C}.
        \end{align*}
\end{enumerate}

\subsubsection{21.10.3}

\newcommand{\ke}{\frac{1}{4 \pi \epsilon_0}}

\begin{enumerate}[a)]
  \item From Example 21.9 we know that the electric field of a charged ring of radius $a$ at a distance $x$ along the ring's axis is directed away from the ring along its axis and has magnitude \[E = \frac{1}{4 \pi \epsilon_0} \frac{Qx}{(x^2 + a^2)^{3/2}}.\]

        By the principle of superposition of electric fields, the electric field of the hydrogen atom is

        \begin{align*}
          E & = \ke \left( \frac{e}{a^2} - \frac{ea}{(a^2 + a^2)^{3/2}} \right) \\
            & = \ke e \left( \frac{1}{a^2} - \frac{a}{2\sqrt{2}a^3} \right)     \\
            & = \ke \frac{e}{a^2} \left( 1 - \frac{1}{2\sqrt{2}} \right).
        \end{align*}

  \item $1 - 1/(2\sqrt{2}) \approx 0.65$ so the field points away from the proton.
\end{enumerate}

\subsubsection{21.10.4}

\begin{enumerate}[a)]
  \item The charge per unit length is \[\lambda = \frac{Q}{L}\] so the charge contained in a segment of length $dx$ is \[\lambda\,dx = \frac{Q}{L}\,dx.\]

  \item The field and source points both lie on the $x$-axis, so the differential electric field has no $y$ component. The $x$ component is \[dE_x = -\ke \frac{\lambda \, dx}{x^2} = -\ke \frac{Q}{Lx^2} \, dx.\]

  \item The total electric field at the origin is

        \begin{align*}
          E & = \int dE_x                                                   \\
            & = \int_L^{2L} -\ke \frac{Q}{Lx^2} \, dx                       \\
            & = -\ke \frac{Q}{L} \left[ -\frac{1}{x} \right]_L^{2L}         \\
            & = -\ke \frac{Q}{L} \left( -\frac{1}{2L} + \frac{1}{L} \right) \\
            & = -\ke \frac{Q}{2L^2}.
        \end{align*}
\end{enumerate}

\subsubsection{VP21.14.1}

\begin{enumerate}[a)]
  \item The magnitude of the torque is given by

        \begin{align*}
          \tau & = p E \sin \theta                                 \\
               & = (\num{6.13e-30}) (\num{3.00e5}) \sin \ang{50.0} \\
               & = \qty{1.41e-24}{N.m}.
        \end{align*}

  \item The potential energy is given by

        \begin{align*}
          U & = -p E \cos \theta                                 \\
            & = -(\num{6.13e-30}) (\num{3.00e5}) \cos \ang{50.0} \\
            & = \qty{-1.18e-24}{J}.
        \end{align*}
\end{enumerate}

\subsubsection{VP21.14.2}

To find the magnitude of the charges we can rearrange the torque equation

\begin{align*}
  \tau & = p E \sin \theta                                                      \\
       & = q d E \sin \theta                                                    \\
  q    & = \frac{\tau}{d E \sin \theta}                                         \\
       & = \frac{\num{6.60e-26}}{(\num{1.10e-10}) (\num{8.50e4}) \sin \ang{90}} \\
       & = \qty{7.06e-21}{C}.
\end{align*}

\subsubsection{VP21.14.3}

When the dipole moment is parallel to the field its potential energy is $-p E$ and when it is antiparallel its potential energy is $p E$. Thus, the work required to perform the rotation is $2 p E$ and

\begin{align*}
  W & = 2 p E                                   \\
  p & = \frac{W}{2 E}                           \\
    & = \frac{\num{4.60e-25}}{2 (\num{1.20e5})} \\
    & = \qty{1.92e-30}{C.m}.
\end{align*}

\subsubsection{VP21.14.4}

\newcommand{\kev}{\num{8.988e9}}

\begin{enumerate}[a)]
  \item Rearranging the equation for the dipole moment gives

        \begin{align*}
          p & = q d                                   \\
          d & = \frac{p}{q}                           \\
            & = \frac{\num{3.50e-29}}{\num{1.60e-19}} \\
            & = \qty{2.19e-10}{m}.
        \end{align*}

  \item From Example 21.14, the electric field of the molecule along its axis is \[E_y = \ke \frac{2p}{y^3}.\]

        Rearranging for $y$ and substituting in the desired field strength gives

        \begin{align*}
          y & = \left( \ke \frac{2 p}{E_y} \right)^{1/3}                                             \\
            & = \left( (\kev) \frac{2 (\num{1.60e-19}) (\num{2.19e-10})}{\num{8.00e4}} \right)^{1/3} \\
            & = \qty{1.99e-8}{m}.
        \end{align*}
\end{enumerate}

\end{document}