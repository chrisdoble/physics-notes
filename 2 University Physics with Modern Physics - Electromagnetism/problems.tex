\documentclass{article}
\usepackage{amsmath}
\usepackage{enumerate}

\title{University Physics with Modern Physics Electromagnetism Problems}
\author{Chris Doble}
\date{December 2022}

\begin{document}

\maketitle

\tableofcontents

\setcounter{section}{20}
\section{Electric Charge and Electric Field}

\setcounter{subsection}{2}
\subsection{Coulomb's Law}

\subsubsection{Example 21.1}

The magnitude of electric repulsion between two $\alpha$ particles is given by \[F_e = \frac{1}{4\pi\epsilon_0}\frac{q^2}{r^2}\] and the magnitude of gravitational attraction is given by \[F_g = \frac{Gm^2}{r^2}\]. The ratio of the two values is

\begin{align*}
  \frac{F_e}{F_g} & = \frac{1}{4\pi\epsilon_0}\frac{q^2}{r^2}\frac{r^2}{Gm^2} \\
                  & = \frac{1}{4\pi\epsilon_0}\frac{q^2}{Gm^2}                \\
                  & = 3.1\times10^{35}
\end{align*}

showing that the electric repulsion is significantly stronger than the gravitational attraction.

\subsubsection{Example 21.2}

\begin{enumerate}[a)]
  \item The magnitude of the force that $q_1$ exerts on $q_2$ is
        \begin{align*}
          F & = \frac{1}{4\pi\epsilon_0}\frac{q_1q_2}{r^2}                                 \\
            & = (9.0 \times 10^9)\frac{|(25 \times 10^{-9})(-75 \times 10^{-9})|}{0.030^2} \\
            & = 1.9 \times 10^{-2}\,\textrm{N}.
        \end{align*}
        Since $q_1$ and $q_2$ have opposite charge, the force is attractive (from $q_2$ to $q_1$).

  \item The magnitude of the force that $q_2$ exerts on $q_1$ is the same as in part a, but the direction is reversed (from $q_1$ to $q_2$).
\end{enumerate}

\subsubsection{Example 21.3}

By the principle of superposition of forces, the net force exerted on $q_3$ is equal to the vector sum of the forces exerted on it by $q_1$ and $q_2$ separately.

Both $q_1$ and $q_3$ have positive charge so they repel each other. $q_1$ is to the right of $q_3$ so $q_3$ experiences a force to the left of magnitude

\begin{align*}
  F_\textrm{1 on 3} & = \frac{1}{4\pi\epsilon_0}\frac{|q_1q_3|}{r^2}                                \\
                    & = (9.0 \times 10^9)\frac{|(1.0 \times 10^{-9})(5.0 \times 10^{-9})|}{0.020^2} \\
                    & = 1.1 \times 10^{-4}\,\textrm{N}.
\end{align*}

However $q_2$ has a negative charge so it attracts $q_3$. It is also to the right of $q_3$ so $q_3$ experiences a force to the right of magnitude

\begin{align*}
  F_\textrm{2 on 3} & = \frac{1}{4\pi\epsilon_0}\frac{|q_2q_3|}{r^2}                                \\
                    & = (9.0 \times 10^9)\frac{|(-3.0 \times 10^{-9})(5.0 \times 10^{-9})}{0.040^2} \\
                    & = 8.4 \times 10^{-5}\,\textrm{N}.
\end{align*}

The net force experienced by $q_3$ is therefore

\begin{align*}
  F & = -F_\textrm{1 on 3} + F_\textrm{2 on 3}   \\
    & = -1.1 \times 10^{-4} + 8.4 \times 10^{-5} \\
    & = -2.6 \times 10^{-5}\,\textrm{N}.
\end{align*}

\subsubsection{Example 21.4}

Since $q_1$ and $q_2$ are of equal charge and are symmetric about the x axis on which $Q$ lies, the vertical components of their forces cancel leaving only the horizontal.

The horizontal component of $q_1$'s force on $Q$ is given by

\begin{align*}
  F_\textrm{1 on Q, x} & = \frac{1}{4\pi\epsilon_0} \frac{q_1Q}{r_\textrm{1,Q}^2} \cos\alpha                                             \\
                       & = (9.0 \times 10^9) \frac{(2.0 \times 10^{-6})(4.0 \times 10^{-6})}{\sqrt{0.30^2 + 0.40^2}^2} \frac{0.40}{0.50} \\
                       & = 0.23\,\textrm{N}.
\end{align*}

Again, since $q_1$ and $q_2$ are of equal charge and symmetric about the x axis, $F_\textrm{1 on Q, x} = F_\textrm{2 on Q, x}$ and the total force experienced by $Q$ is in the positive x direction of magnitude \[F = 2F_\textrm{1 on Q, x} = 0.46\,\textrm{N}.\]

\end{document}