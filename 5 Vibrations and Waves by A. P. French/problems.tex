\documentclass{article}
\usepackage{amsmath} % For align*
\usepackage{amsfonts} % For open face letters
\usepackage{enumitem} % For customisable list labels
\usepackage{graphicx} % For images
\usepackage{siunitx} % For units
\graphicspath{{./images/}}

\renewcommand{\Im}{\operatorname{Im}}
\renewcommand{\Re}{\operatorname{Re}}

\setlist[enumerate, 1]{label={(\alph*)}}
\setlist[enumerate, 2]{label={(\roman*)}}

\title{Vibrations and Waves by A. P. French Problems}
\author{Chris Doble}
\date{May 2022}

\begin{document}

\maketitle

\tableofcontents

\section{Periodic motions}

\setcounter{subsection}{3}
\subsection{}

\begin{enumerate}
  \item

        \begin{align*}
          z   & = A e^{j \theta}              \\
          d z & = j A e^{j \theta} \,d \theta \\
              & = j z \,d \theta
        \end{align*}

        The motion of the point is always perpendicular to its position.

  \item

        \begin{align*}
          |2 + j \sqrt{3}|        & = \sqrt{2^2 + \sqrt{3}^2}     \\
                                  & = \sqrt{7}                    \\
          \arg (2 + j \sqrt{3})   & = \arctan \frac{\sqrt{3}}{2}  \\
                                  & = \ang{41}                    \\ \\
          (2 - j \sqrt{3})^2      & = 4 - j 4 \sqrt{3} - 3        \\
                                  & = 1 - j 4 \sqrt{3}            \\
          |1 - j 4 \sqrt{3}|      & = \sqrt{1^2 + (4 \sqrt{3})^2} \\
                                  & = 7                           \\
          \arg (1 - j 4 \sqrt{3}) & = -\arctan 4 \sqrt{3}
        \end{align*}
\end{enumerate}

\setcounter{subsection}{8}
\subsection{}

\begin{align*}
  \cos \theta + j \sin \theta               & = e^{j \theta}            \\
  \cos \frac{\pi}{2} + j \sin \frac{\pi}{2} & = e^{j \frac{\pi}{2}}     \\
  j                                         & = e^{j \frac{\pi}{2}}     \\
  j^j                                       & = (e^{j \frac{\pi}{2}})^j \\
                                            & = e^{-\frac{\pi}{2}}      \\
                                            & \approx 0.208
\end{align*}

Yes, I would be willing to pay 20 cents because I could sell it to the mathematician and gain 0.8 cents.

\subsection{}

\begin{align*}
  y                   & = A \cos k x + B \sin k x             \\
  \frac{d y}{d x}     & = -A k \sin k x + B k \cos k x        \\
  \frac{d^2 y}{d x^2} & = -A k^2 \cos k x - B k^2 \sin k x    \\
                      & = -k^2 y                              \\ \\
  C                   & = \sqrt{A^2 + B^2}                    \\
  \alpha              & = \arctan \left( -\frac{B}{A} \right) \\
  y                   & = C \cos (k x + \alpha)               \\
                      & = C \Re [e^{j (k x + \alpha)}]        \\
                      & = Re[(C e^{j \alpha}) e^{j k x}]
\end{align*}

\subsection{}

\begin{enumerate}
  \item

        \begin{align*}
          x      & = A \cos (\omega t + \alpha) \\
          A      & = \qty{5}{cm}                \\
          f      & = \qty{1}{Hz}                \\
          \omega & = 2 \pi f                    \\
                 & = 2 \pi \,\unit{rad/s}       \\
          \alpha & = \pm \frac{\pi}{2}
        \end{align*}

  \item

        \begin{align*}
          x \left( \frac{8}{3} \right)                   & = 5 \cos \left( 2 \pi \frac{8}{3} + \alpha \right) \\
                                                         & = \pm \qty{4.33}{cm}                               \\
          \frac{d x}{d t}                                & = -A \omega \sin (\omega t + \alpha)               \\
          \frac{d x}{d t} \left( \frac{8}{3} \right)     & = \pm \qty{15.7}{cm/s}                             \\
          \frac{d^2 x}{d t^2}                            & = -A \omega^2 \cos (\omega t + \alpha)             \\
          \frac{d^2 x}{d t^2} \left( \frac{8}{3} \right) & = \mp \qty{171}{cm/s^2}
        \end{align*}
\end{enumerate}

\subsection{}

\begin{enumerate}
  \item

        \begin{align*}
          v        & = \qty{50}{cm/s}                                                      \\
          T        & = \qty{6}{s}                                                          \\
          \theta_0 & = \ang{30}                                                            \\
          c        & = \qty{300}{cm}                                                       \\
          A        & = \frac{c}{2 \pi}                                                     \\
                   & = \frac{150}{\pi} \,\unit{cm}                                         \\
          \omega   & = \frac{2 \pi}{T}                                                     \\
                   & = \frac{\pi}{3} \,\unit{rad/s}                                        \\
          \alpha   & = \frac{\pi}{6} \,\unit{rad}                                          \\
          x        & = \frac{150}{\pi} \cos \left( \frac{\pi}{3} t + \frac{\pi}{6} \right)
        \end{align*}

  \item

        \begin{align*}
          x(\qty{2}{s})                    & = \qty{-41.3}{cm}                                                       \\
          \frac{d x}{d t}                  & = -50 \sin \left( \frac{\pi}{3} t + \frac{\pi}{6} \right)               \\
          \frac{d x}{d t} (\qty{2}{s})     & = \qty{-25}{cm/s}                                                       \\
          \frac{d^2 x}{d t^2}              & = -\frac{50 \pi}{3} \cos \left( \frac{\pi}{3} t + \frac{\pi}{6} \right) \\
          \frac{d^2 x}{d t^2} (\qty{2}{s}) & = \qty{45}{cm/s^2}
        \end{align*}
\end{enumerate}

\section{The superposition of periodic motions}

\subsection{}

\begin{enumerate}
  \item

        \begin{align*}
          z & = \sin \omega t + \cos \omega t                          \\
            & = \sqrt{2} \cos \left( \omega t - \frac{\pi}{4} \right)  \\
            & = \sqrt{2} e^{j \left( \omega t - \frac{\pi}{4} \right)}
        \end{align*}

  \item

        \begin{align*}
          z & = \cos (\omega t - \pi / 3) - \cos \omega t                                           \\
            & = \cos \omega t \cos \frac{\pi}{3} + \sin \omega t \sin \frac{\pi}{3} - \cos \omega t \\
            & = -\frac{1}{2} \cos \omega t + \frac{\sqrt{3}}{2} \sin \omega t                       \\
            & = \cos (\omega t + 2 \pi / 3)                                                         \\
            & = e^{j (\omega t + 2 \pi / 3)}
        \end{align*}

  \item

        \begin{align*}
          z & = 3 \cos \omega t + 2 \sin \omega t          \\
            & = \sqrt{13} \cos (\omega t + \arctan -2 / 3) \\
        \end{align*}

  \item

        \begin{align*}
          z & = \sin \omega t - 2 \cos (\omega t - \pi / 4) + \cos \omega t                                 \\
            & = \sin \omega t - 2 (\cos \omega t \cos \pi / 4 + \sin \omega t \sin \pi / 4) + \cos \omega t \\
            & = \sin \omega t - \sqrt{2} \cos \omega t - \sqrt{2} \sin \omega t + \cos \omega t             \\
            & = (1 - \sqrt{2}) \cos \omega t + (1 - \sqrt{2}) \sin \omega t                                 \\
            & = (1 - \sqrt{2}) \sqrt{2} \cos (\omega t - \pi / 4)                                           \\
            & = (\sqrt{2} - 2) \cos (\omega t - \pi / 4)                                                    \\
            & = (2 - \sqrt{2}) \cos (\omega t + 3 \pi / 4)
        \end{align*}
\end{enumerate}

\subsection{}

\begin{align*}
  x      & = A_1 \cos \omega t + A_2 \cos (\omega t + \alpha_1) + A_3 \cos (\omega t + \alpha_1 + \alpha_2)                               \\
         & = A_1 \cos \omega t + A_2 (\cos \omega t \cos \alpha_1 - \sin \omega t \sin \alpha_1)                                          \\
         & \qquad + A_3 (\cos \omega t \cos (\alpha_1 + \alpha_2) - \sin \omega t \sin (\alpha_1 + \alpha_2))                             \\
         & = (A_1 + A_2 \cos \alpha_1 + A_3 \cos (\alpha_1 + \alpha_2)) \cos \omega t                                                     \\
         & \qquad - (A_2 \sin \alpha_1 + A_3 \sin (\alpha_1 + \alpha_2)) \sin \omega t                                                    \\
  A      & = \sqrt{(A_1 + A_2 \cos \alpha_1 + A_3 \cos (\alpha_1 + \alpha_2))^2 + (A_2 \sin \alpha_1 + A_3 \sin (\alpha_1 + \alpha_2))^2} \\
         & \approx \qty{0.52}{mm}                                                                                                         \\
  \alpha & = \arctan \frac{A_2 \sin \alpha_1 + A_3 \sin (\alpha_1 + \alpha_2)}{A_1 + A_2 \cos \alpha_1 + A_3 \cos (\alpha_1 + \alpha_2)}  \\
         & \approx \qty{0.59}{rad}                                                                                                        \\
         & \approx \ang{34}
\end{align*}

\subsection{}

The equation of motion is \[x = 2 A \cos \left( \frac{12 \pi - 10 \pi}{2} t \right) \cos \left( \frac{12 \pi + 10 \pi}{2} t \right)\] with the variation in amplitude given by the term \[2 A \cos \pi t\] so the beat period is $\qty{1}{s}$.

\subsection{}

\begin{enumerate}
  \item \[\omega = 2 \pi, \unit{rad/s} \Rightarrow f = \qty{1}{Hz}\]

  \item \[\omega = \frac{25 \pi}{2} \,\unit{rad/s} \Rightarrow f = \frac{25}{4} \,\unit{Hz}\]

  \item \[\omega = \frac{3 + \pi}{2} \,\unit{rad/s} \Rightarrow f = \frac{3 + \pi}{4 \pi} \,\unit{Hz}\]
\end{enumerate}

\section{The free vibrations of physical systems}

\subsection{}

\begin{align*}
  F   & = -k x              \\
  m a & = -k x              \\
  k   & = -\frac{m a}{x}    \\
      & = \qty{4.0e-5}{N/m}
\end{align*}

\subsection{}

\begin{enumerate}
  \item \[T_0 = 2 \pi \sqrt{\frac{m}{k}}\]

  \item

        \begin{enumerate}
          \item

                \begin{align*}
                  m x'' & = -2 k x                     \\
                  x''   & = -\frac{2 k}{m} x           \\
                  T     & = 2 \pi \sqrt{\frac{m}{2 k}} \\
                        & = \frac{T_0}{\sqrt{2}}
                \end{align*}

          \item

                \begin{align*}
                  m x'' & = -k \frac{x}{2}             \\
                  x''   & = -\frac{k}{2 m} x           \\
                  T     & = 2 \pi \sqrt{\frac{2 m}{k}} \\
                        & = \sqrt{2} T_0
                \end{align*}
        \end{enumerate}
\end{enumerate}

\subsection{}

\begin{enumerate}
  \item

        \begin{align*}
          y        & = A \cos \omega t                               \\
          y'       & = -\omega A \sin \omega t                       \\
          y''      & = -\omega^2 A \cos \omega t                     \\
          g        & = \omega^2 A \cos \omega t                      \\
          \omega t & = \arccos \frac{g}{\omega^2 A}                  \\
          t        & = \frac{1}{\omega} \arccos \frac{g}{\omega^2 A} \\
          y        & = A \cos \arccos \frac{g}{\omega^2 A}           \\
                   & = \frac{g}{\omega^2}                            \\
                   & = \qty{2.5}{cm}
        \end{align*}

  \item

        \begin{align*}
          v                 & = -\omega A \sin \omega t                     \\
                            & = -\omega A \sin \arccos \frac{g}{\omega^2 A} \\
                            & \approx \qty{0.87}{m/s}                       \\
          \frac{1}{2} m v^2 & = m g h                                       \\
          h                 & = \frac{v^2}{2 g}                             \\
                            & \approx \qty{3.8}{cm}                         \\
          \Delta h          & \approx \qty{1.3}{cm}
        \end{align*}
\end{enumerate}

\subsection{}

\begin{enumerate}
  \item

        \begin{align*}
          m y''  & = -g \rho A y               \\
          y''    & = -\frac{g \rho A}{m} y     \\
          \omega & = \sqrt{\frac{g \rho A}{m}} \\
                 & = \sqrt{\frac{g}{l}}
        \end{align*}
\end{enumerate}

\subsection{}

\[T = 2 \pi \sqrt{\frac{2 L}{3 g}}\]

\subsection{}

\[T = 2 \pi \sqrt{\frac{d}{g}}\]

\setcounter{subsection}{7}
\subsection{}

\begin{enumerate}
  \item

        \begin{align*}
          m g & = \frac{A Y}{l_0} x               \\
          x   & = \frac{m g l_0}{A Y}             \\
              & = \frac{m g l_0}{\pi (d / 2)^2 Y} \\
              & = \qty{0.25}{mm}
        \end{align*}

  \item

        \begin{align*}
          F_u   & = u \pi (d / 2)^2                             \\
                & \approx \qty{215.98}{N}                       \\
          k     & = \frac{A Y}{L}                               \\
                & = \frac{\pi (d / 2)^2 Y}{L}                   \\
                & = \frac{\pi d^2 Y}{4 L}                       \\
                & \approx \qty{19634.95}{N/m}                   \\
          F_u   & = k x_u                                       \\
          x_u   & = \frac{F_u}{k}                               \\
                & \approx \qty{1.1}{cm}                         \\
          m g h & = \frac{1}{2} \frac{A Y}{L} x_u^2 - m g x_u   \\
          h     & = \frac{\pi (d / 2)^2 Y x_u^2}{2 m g L} - x_u \\
                & = \frac{\pi d^2 Y x_u^2}{8 m g L} - x_u       \\
                & = \qty{0.23}{m}
        \end{align*}
\end{enumerate}

\subsection{}

\begin{enumerate}
  \item

        \begin{align*}
          \rho_\text{steel}        & = \qty{7850}{kg/m^3}               \\
          V_\text{sphere}          & = \frac{4}{3} \pi r^3              \\
          F_u                      & = A u                              \\
                                   & = \pi r^2 u                        \\
                                   & \approx \qty{3455.75}{N}           \\
          m g                      & = F_u                              \\
          m                        & = \frac{F_u}{g}                    \\
                                   & \approx \qty{352.3}{kg}            \\
          \rho V                   & = m                                \\
          \rho \frac{4}{3} \pi r^3 & = m                                \\
          r                        & = \sqrt[3]{\frac{3 m}{4 \pi \rho}} \\
                                   & = \qty{22}{cm}
        \end{align*}

  \item

        \begin{align*}
          M & = -\frac{\pi n r^4}{2 l} \theta                    \\
          c & = \frac{\pi n r^4}{2 l}                            \\
          T & = 2 \pi \sqrt{\frac{I}{c}}                         \\
            & = 2 \pi \sqrt{\frac{2 M R^2 / 5}{\pi n r^4 / 2 l}} \\
            & = 2 \pi \sqrt{\frac{4 l M R^2}{5 \pi n r^4}}       \\
            & = \qty{66}{s}
        \end{align*}
\end{enumerate}

\subsection{}

\begin{enumerate}
  \item

        \begin{align*}
          Y & = \frac{\text{stress}}{\text{strain}} \\
            & = \frac{F / A}{\Delta l / l_0}        \\
            & = \frac{m g / A}{\Delta l / l_0}      \\
            & = \frac{m g l_0}{\Delta l A}          \\
            & = \qty{5.9e11}{N/m^2}
        \end{align*}

  \item

        \begin{align*}
          y                         & = \frac{4 L^3}{Y a b^3} F        \\
          F                         & = \frac{Y a b^3}{4 L^3} y        \\
          k                         & = \frac{Y a b^3}{4 L^3}          \\
          \omega_y                  & = \sqrt{\frac{k}{m}}             \\
                                    & = \sqrt{\frac{Y a b^3}{4 L^3 m}} \\
          \omega_x                  & = \sqrt{\frac{Y a^3 b}{4 L^3 m}} \\
          \frac{\omega_y}{\omega_x} & = \sqrt{\frac{a b^3}{a^3 b}}     \\
                                    & = \frac{b}{a}
        \end{align*}

  \item $3 / 2$
\end{enumerate}

\subsection{}

\begin{enumerate}
  \item \[\omega = \sqrt{\frac{A \gamma p}{l m}}\]
\end{enumerate}

\setcounter{subsection}{13}
\subsection{}

\begin{enumerate}
  \item \[m \frac{d^2 y}{d t^2} + b \frac{d y}{d t} + k y = 0\]

  \item

        \begin{align*}
          \omega                          & = \frac{\sqrt{3}}{2} \omega_0 \\
          \omega^2                        & = \frac{3}{4} \omega_0^2      \\
          \omega_0^2 - \frac{\gamma^2}{4} & = \frac{3}{4} \omega_0^2      \\
          \frac{1}{4} \omega_0^2          & = \frac{\gamma^2}{4}          \\
          \omega_0^2                      & = \gamma^2                    \\
          \omega_0                        & = \gamma                      \\
                                          & = \frac{b}{m}                 \\
          b                               & = m \omega_0                  \\
                                          & = m \sqrt{\frac{k}{m}}        \\
                                          & = \qty{4}{N/(m/s)}
        \end{align*}
\end{enumerate}

\subsection{}

\begin{enumerate}
  \item

        \begin{align*}
          \overline{E}_0 e^{-\gamma} & = \frac{1}{2} \overline{E}_0 \\
          e^{-\gamma}                & = \frac{1}{2}                \\
          -\gamma                    & = \ln \frac{1}{2}            \\
          \gamma                     & = \ln 2                      \\
          Q_0                        & = \frac{\omega_0}{\gamma}    \\
                                     & = \frac{2 \pi f}{\gamma}     \\
                                     & = \frac{512 \pi}{\ln 2}      \\
                                     & \approx 2321
        \end{align*}

  \item \[Q = 2 Q_0\]

  \item

        \begin{align*}
          \gamma & = \frac{1}{4}             \\
          Q      & = \frac{\omega_0}{\gamma} \\
                 & = 4 \sqrt{\frac{k}{m}}    \\
                 & = 12                      \\
          \gamma & = \frac{b}{m}             \\
          b      & = \gamma m                \\
                 & = \qty{0.025}{N/(m/s)}
        \end{align*}
\end{enumerate}

\subsection{}

\begin{enumerate}
  \item

        \begin{align*}
          x & = A \sin \omega t                                                                                            \\
          v & = \omega A \cos \omega t                                                                                     \\
          a & = -\omega^2 A \sin \omega t                                                                                  \\
          E & = \int_0^{1 / f} \frac{K e^2}{c^3} (-\omega^2 A \sin \omega t)^2 \,dt                                        \\
            & = \frac{K e^2 \omega^4 A^2}{c^3} \int_0^{1 / f} \sin^2 \omega t \,dt                                         \\
            & = \frac{K e^2 \omega^4 A^2}{c^3} \left[ \frac{t}{2} - \frac{1}{4 \omega} \sin 2 \omega t \right]_0^{1 / f}   \\
            & = \frac{K e^2 \omega^4 A^2}{c^3} \left( \frac{1}{2 f} - \frac{1}{4 \omega} \sin 2 \omega \frac{1}{f} \right) \\
            & = \frac{K e^2 (2 \pi f)^4 A^2}{2 f c^3}                                                                      \\
            & = \frac{8 \pi^4 K e^2 f^3 A^2}{c^3}
        \end{align*}

  \item

        \begin{align*}
          E_0                                             & = \frac{1}{2} m v^2                                          \\
                                                          & = \frac{m (\omega A)^2}{2}                                   \\
                                                          & = 2 \pi^2 A^2 f^2 m                                          \\
          \frac{Q}{\pi} E                                 & = E_0 \left( 1 - \frac{1}{e} \right)                         \\
          \frac{Q}{\pi} \frac{8 \pi^4 K q^2 f^3 A^2}{c^3} & = 2 \pi^2 A^2 f^2 m \left( 1 - \frac{1}{e} \right)           \\
          Q \frac{4 \pi K q^2 f}{c^3}                     & = m \left( 1 - \frac{1}{e} \right)                           \\
          Q                                               & = \frac{c^3 m}{4 \pi f K q^2} \left( 1 - \frac{1}{e} \right)
        \end{align*}
\end{enumerate}

\subsection{}

\begin{enumerate}
  \item

        \begin{align*}
          V                                                  & = \pi r^2 y_\text{left}                             \\
          V                                                  & = \pi (2 r)^2 y_\text{right}                        \\
          \pi r^2 y_\text{left}                              & = \pi (2 r)^2 y_\text{right}                        \\
          y_\text{right}                                     & = \frac{1}{4} y_\text{left}                         \\
          \frac{y_\text{left}}{2} + \frac{y_\text{right}}{2} & = \frac{y_\text{left}}{2} + \frac{y_\text{left}}{8} \\
                                                             & = \frac{5}{8} y_\text{left}                         \\
          U                                                  & = m g \frac{5}{8} y                                 \\
                                                             & = \frac{5}{8} \rho \pi r^2 y g y                    \\
                                                             & = \frac{5}{8} g \rho \pi r^2 y^2
        \end{align*}

  \item

        \begin{align*}
          r(x)                    & = r + \frac{x}{l} r                                                                                                                          \\
                                  & = r \left( 1 + \frac{x}{l} \right)                                                                                                           \\
          \frac{d y}{d t} \pi r^2 & = v \pi r(x)^2                                                                                                                               \\
                                  & = v \pi \left[ r \left( 1 + \frac{x}{l} \right) \right]^2                                                                                    \\
          v                       & = \frac{d y}{d t} \frac{1}{\left( 1 + \frac{x}{l} \right)^2}                                                                                 \\
          m                       & = \rho \pi r(x)^2 \,dx                                                                                                                       \\
                                  & = \rho \pi \left[ r \left( 1 + \frac{x}{l} \right) \right]^2 \,dx                                                                            \\
                                  & = \rho \pi r^2 \left( 1 + \frac{x}{l} \right)^2 \,dx                                                                                         \\
          dK                      & = \frac{1}{2} m v^2                                                                                                                          \\
                                  & = \frac{1}{2} \rho \pi r^2 \left( 1 + \frac{x}{l} \right)^2 \,dx \left( \frac{d y}{d t} \frac{1}{\left( 1 + \frac{x}{l} \right)^2} \right)^2 \\
                                  & = \frac{1}{2} \rho \frac{\pi r^2 \,dx}{(1 + x / l)^2} \left( \frac{d y}{d t} \right)^2
        \end{align*}

  \item

        \begin{align*}
          K & = \frac{1}{2} \rho \pi r^2 h \left( \frac{d y}{d t} \right)^2 + \frac{1}{2} \rho \pi (2 r)^2 h \left( \frac{d y}{d t} \right)^2 + \int_0^l \,dK                 \\
            & = \frac{5}{2} \rho \pi r^2 h \left( \frac{d y}{d t} \right)^2 + \int_0^l \frac{1}{2} \rho \frac{\pi r^2 \,dx}{(1 + x / l)^2} \left( \frac{d y}{d t} \right)^2   \\
            & = \frac{5}{2} \rho \pi r^2 h \left( \frac{d y}{d t} \right)^2 + \frac{1}{2} \rho \pi r^2 \int_0^l \frac{1}{(1 + x / l)^2} \,dx \left( \frac{d y}{d t} \right)^2 \\
            & = \frac{1}{4} \rho \pi r^2 \left( l + \frac{5}{2} h \right) \left( \frac{d y}{d t} \right)^2
        \end{align*}

  \item

        \begin{align*}
          K + U                                                                                                                       & = E                                                                                                         \\
          \frac{1}{4} \rho \pi r^2 \left( l + \frac{5}{2} h \right) \left( \frac{d y}{d t} \right)^2 + \frac{5}{8} g \rho \pi r^2 y^2 & = E                                                                                                         \\
          m                                                                                                                           & = \frac{1}{2} \rho \pi r^2 \left( l + \frac{5}{2} h \right)                                                 \\
          k                                                                                                                           & = \frac{5}{4} g \rho \pi r^2                                                                                \\
          T                                                                                                                           & = 2 \pi \sqrt{\frac{m}{k}}                                                                                  \\
                                                                                                                                      & = 2 \pi \sqrt{\frac{\frac{1}{2} \rho \pi r^2 \left( l + \frac{5}{2} h \right)}{\frac{5}{4} g \rho \pi r^2}} \\
                                                                                                                                      & = 2 \pi \sqrt{\frac{2 h}{g}}
        \end{align*}
\end{enumerate}

\setcounter{subsection}{18}
\subsection{}

\begin{enumerate}
  \item \[m \frac{d^2 x}{d t^2} + 2 k (x + l - l_0) = 0\]

  \item

        \begin{align*}
          T                                                              & = k (l' - l_0)                                            \\
                                                                         & = k (\sqrt{l^2 + y^2} - l_0)                              \\
          F                                                              & = 2 T \sin \theta                                         \\
                                                                         & = 2 k (\sqrt{l^2 + y^2} - l_0) \frac{y}{\sqrt{l^2 + y^2}} \\
                                                                         & = 2 k \left( 1 - \frac{l_0}{\sqrt{l^2 + y^2}} \right) y   \\
                                                                         & \approx 2 k \left( 1 - \frac{l_0}{l} \right) y            \\
          m \frac{d^2 y}{d t^2} + 2 k \left( 1 - \frac{l_0}{l} \right) y & = 0
        \end{align*}

  \item

        \begin{align*}
          T_x             & = 2 \pi \sqrt{\frac{m}{2 k}}                                            \\
          T_y             & = 2 \pi \sqrt{\frac{m}{2 k \left( 1 - \frac{l_0}{l} \right)}}           \\
          \frac{T_x}{T_y} & = \frac{2 \pi \sqrt{m / 2 k}}{2 \pi \sqrt{\frac{m}{2 k (1 - l / l_0)}}} \\
                          & = \sqrt{\frac{m}{2 k} \frac{2 k (1 - l / l_0)}{m}}                      \\
                          & = \sqrt{1 - l / l_0}
        \end{align*}

  \item

        \begin{align*}
          x           & = A_x \cos \left( \sqrt{\frac{2 k}{m}} t + \phi_x \right)               \\
          A_0         & = A_x \cos \phi_x                                                       \\
          0           & = -\sqrt{\frac{2 k}{m}} A_x \sin \phi_x                                 \\
          \tan \phi_x & = 0                                                                     \\
          \phi_x      & = 0                                                                     \\
          A_x         & = A_0                                                                   \\
          x           & = A_0 \cos \sqrt{\frac{2 k}{m}} t                                       \\ \\
          y           & = A_y \cos \left( \sqrt{\frac{2 k (1 - l_0 / l)}{m}} t + \phi_y \right) \\
          A_0         & = A_y \cos \phi_y                                                       \\
          0           & = -\sqrt{\frac{2 k (1 - l_0 / l)}{m}} A_y \sin \phi_y                   \\
          \tan \phi_y & = 0                                                                     \\
          \phi_y      & = 0                                                                     \\
          A_y         & = A_0                                                                   \\
          y           & = A_0 \cos \sqrt{\frac{2 k (1 - l_0 / l)}{m}} t
        \end{align*}
\end{enumerate}

\end{document}