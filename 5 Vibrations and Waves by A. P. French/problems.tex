\documentclass{article}
\usepackage{amsmath} % For align*
\usepackage{amsfonts} % For open face letters
\usepackage{enumitem} % For customisable list labels
\usepackage{graphicx} % For images
\usepackage{siunitx} % For units
\graphicspath{{./images/}}

\renewcommand{\Im}{\operatorname{Im}}
\renewcommand{\Re}{\operatorname{Re}}

\setlist[enumerate, 1]{label={(\alph*)}}
\setlist[enumerate, 2]{label={(\roman*)}}

\title{Vibrations and Waves by A. P. French Problems}
\author{Chris Doble}
\date{May 2022}

\begin{document}

\maketitle

\tableofcontents

\section{Periodic motions}

\setcounter{subsection}{3}
\subsection{}

\begin{enumerate}
  \item

        \begin{align*}
          z   & = A e^{j \theta}              \\
          d z & = j A e^{j \theta} \,d \theta \\
              & = j z \,d \theta
        \end{align*}

        The motion of the point is always perpendicular to its position.

  \item

        \begin{align*}
          |2 + j \sqrt{3}|        & = \sqrt{2^2 + \sqrt{3}^2}     \\
                                  & = \sqrt{7}                    \\
          \arg (2 + j \sqrt{3})   & = \arctan \frac{\sqrt{3}}{2}  \\
                                  & = \ang{41}                    \\ \\
          (2 - j \sqrt{3})^2      & = 4 - j 4 \sqrt{3} - 3        \\
                                  & = 1 - j 4 \sqrt{3}            \\
          |1 - j 4 \sqrt{3}|      & = \sqrt{1^2 + (4 \sqrt{3})^2} \\
                                  & = 7                           \\
          \arg (1 - j 4 \sqrt{3}) & = -\arctan 4 \sqrt{3}
        \end{align*}
\end{enumerate}

\setcounter{subsection}{8}
\subsection{}

\begin{align*}
  \cos \theta + j \sin \theta               & = e^{j \theta}            \\
  \cos \frac{\pi}{2} + j \sin \frac{\pi}{2} & = e^{j \frac{\pi}{2}}     \\
  j                                         & = e^{j \frac{\pi}{2}}     \\
  j^j                                       & = (e^{j \frac{\pi}{2}})^j \\
                                            & = e^{-\frac{\pi}{2}}      \\
                                            & \approx 0.208
\end{align*}

Yes, I would be willing to pay 20 cents because I could sell it to the mathematician and gain 0.8 cents.

\subsection{}

\begin{align*}
  y                   & = A \cos k x + B \sin k x             \\
  \frac{d y}{d x}     & = -A k \sin k x + B k \cos k x        \\
  \frac{d^2 y}{d x^2} & = -A k^2 \cos k x - B k^2 \sin k x    \\
                      & = -k^2 y                              \\ \\
  C                   & = \sqrt{A^2 + B^2}                    \\
  \alpha              & = \arctan \left( -\frac{B}{A} \right) \\
  y                   & = C \cos (k x + \alpha)               \\
                      & = C \Re [e^{j (k x + \alpha)}]        \\
                      & = Re[(C e^{j \alpha}) e^{j k x}]
\end{align*}

\subsection{}

\begin{enumerate}
  \item

        \begin{align*}
          x      & = A \cos (\omega t + \alpha) \\
          A      & = \qty{5}{cm}                \\
          f      & = \qty{1}{Hz}                \\
          \omega & = 2 \pi f                    \\
                 & = 2 \pi \,\unit{rad/s}       \\
          \alpha & = \pm \frac{\pi}{2}
        \end{align*}

  \item

        \begin{align*}
          x \left( \frac{8}{3} \right)                   & = 5 \cos \left( 2 \pi \frac{8}{3} + \alpha \right) \\
                                                         & = \pm \qty{4.33}{cm}                               \\
          \frac{d x}{d t}                                & = -A \omega \sin (\omega t + \alpha)               \\
          \frac{d x}{d t} \left( \frac{8}{3} \right)     & = \pm \qty{15.7}{cm/s}                             \\
          \frac{d^2 x}{d t^2}                            & = -A \omega^2 \cos (\omega t + \alpha)             \\
          \frac{d^2 x}{d t^2} \left( \frac{8}{3} \right) & = \mp \qty{171}{cm/s^2}
        \end{align*}
\end{enumerate}

\subsection{}

\begin{enumerate}
  \item

        \begin{align*}
          v        & = \qty{50}{cm/s}                                                      \\
          T        & = \qty{6}{s}                                                          \\
          \theta_0 & = \ang{30}                                                            \\
          c        & = \qty{300}{cm}                                                       \\
          A        & = \frac{c}{2 \pi}                                                     \\
                   & = \frac{150}{\pi} \,\unit{cm}                                         \\
          \omega   & = \frac{2 \pi}{T}                                                     \\
                   & = \frac{\pi}{3} \,\unit{rad/s}                                        \\
          \alpha   & = \frac{\pi}{6} \,\unit{rad}                                          \\
          x        & = \frac{150}{\pi} \cos \left( \frac{\pi}{3} t + \frac{\pi}{6} \right)
        \end{align*}

  \item

        \begin{align*}
          x(\qty{2}{s})                    & = \qty{-41.3}{cm}                                                       \\
          \frac{d x}{d t}                  & = -50 \sin \left( \frac{\pi}{3} t + \frac{\pi}{6} \right)               \\
          \frac{d x}{d t} (\qty{2}{s})     & = \qty{-25}{cm/s}                                                       \\
          \frac{d^2 x}{d t^2}              & = -\frac{50 \pi}{3} \cos \left( \frac{\pi}{3} t + \frac{\pi}{6} \right) \\
          \frac{d^2 x}{d t^2} (\qty{2}{s}) & = \qty{45}{cm/s^2}
        \end{align*}
\end{enumerate}

\section{The superposition of periodic motions}

\subsection{}

\begin{enumerate}
  \item

        \begin{align*}
          z & = \sin \omega t + \cos \omega t                          \\
            & = \sqrt{2} \cos \left( \omega t - \frac{\pi}{4} \right)  \\
            & = \sqrt{2} e^{j \left( \omega t - \frac{\pi}{4} \right)}
        \end{align*}

  \item

        \begin{align*}
          z & = \cos (\omega t - \pi / 3) - \cos \omega t                                           \\
            & = \cos \omega t \cos \frac{\pi}{3} + \sin \omega t \sin \frac{\pi}{3} - \cos \omega t \\
            & = -\frac{1}{2} \cos \omega t + \frac{\sqrt{3}}{2} \sin \omega t                       \\
            & = \cos (\omega t + 2 \pi / 3)                                                         \\
            & = e^{j (\omega t + 2 \pi / 3)}
        \end{align*}

  \item

        \begin{align*}
          z & = 3 \cos \omega t + 2 \sin \omega t          \\
            & = \sqrt{13} \cos (\omega t + \arctan -2 / 3) \\
        \end{align*}

  \item

        \begin{align*}
          z & = \sin \omega t - 2 \cos (\omega t - \pi / 4) + \cos \omega t                                 \\
            & = \sin \omega t - 2 (\cos \omega t \cos \pi / 4 + \sin \omega t \sin \pi / 4) + \cos \omega t \\
            & = \sin \omega t - \sqrt{2} \cos \omega t - \sqrt{2} \sin \omega t + \cos \omega t             \\
            & = (1 - \sqrt{2}) \cos \omega t + (1 - \sqrt{2}) \sin \omega t                                 \\
            & = (1 - \sqrt{2}) \sqrt{2} \cos (\omega t - \pi / 4)                                           \\
            & = (\sqrt{2} - 2) \cos (\omega t - \pi / 4)                                                    \\
            & = (2 - \sqrt{2}) \cos (\omega t + 3 \pi / 4)
        \end{align*}
\end{enumerate}

\subsection{}

\begin{align*}
  x      & = A_1 \cos \omega t + A_2 \cos (\omega t + \alpha_1) + A_3 \cos (\omega t + \alpha_1 + \alpha_2)                               \\
         & = A_1 \cos \omega t + A_2 (\cos \omega t \cos \alpha_1 - \sin \omega t \sin \alpha_1)                                          \\
         & \qquad + A_3 (\cos \omega t \cos (\alpha_1 + \alpha_2) - \sin \omega t \sin (\alpha_1 + \alpha_2))                             \\
         & = (A_1 + A_2 \cos \alpha_1 + A_3 \cos (\alpha_1 + \alpha_2)) \cos \omega t                                                     \\
         & \qquad - (A_2 \sin \alpha_1 + A_3 \sin (\alpha_1 + \alpha_2)) \sin \omega t                                                    \\
  A      & = \sqrt{(A_1 + A_2 \cos \alpha_1 + A_3 \cos (\alpha_1 + \alpha_2))^2 + (A_2 \sin \alpha_1 + A_3 \sin (\alpha_1 + \alpha_2))^2} \\
         & \approx \qty{0.52}{mm}                                                                                                         \\
  \alpha & = \arctan \frac{A_2 \sin \alpha_1 + A_3 \sin (\alpha_1 + \alpha_2)}{A_1 + A_2 \cos \alpha_1 + A_3 \cos (\alpha_1 + \alpha_2)}  \\
         & \approx \qty{0.59}{rad}                                                                                                        \\
         & \approx \ang{34}
\end{align*}

\subsection{}

The equation of motion is \[x = 2 A \cos \left( \frac{12 \pi - 10 \pi}{2} t \right) \cos \left( \frac{12 \pi + 10 \pi}{2} t \right)\] with the variation in amplitude given by the term \[2 A \cos \pi t\] so the beat period is $\qty{1}{s}$.

\subsection{}

\begin{enumerate}
  \item \[\omega = 2 \pi, \unit{rad/s} \Rightarrow f = \qty{1}{Hz}\]

  \item \[\omega = \frac{25 \pi}{2} \,\unit{rad/s} \Rightarrow f = \frac{25}{4} \,\unit{Hz}\]

  \item \[\omega = \frac{3 + \pi}{2} \,\unit{rad/s} \Rightarrow f = \frac{3 + \pi}{4 \pi} \,\unit{Hz}\]
\end{enumerate}

\end{document}