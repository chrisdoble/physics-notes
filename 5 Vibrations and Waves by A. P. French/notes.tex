\documentclass{article}
\usepackage{amsmath} % For align*
\usepackage{enumitem} % For customisable list labels
\usepackage{graphicx} % For images
\usepackage{siunitx} % For units
\graphicspath{{./images/}}

\title{Vibrations and Waves by A. P. French Notes}
\author{Chris Doble}
\date{May 2022}

\begin{document}

\maketitle

\tableofcontents

\section{Periodic motions}

\begin{itemize}
  \item Fouriers theorem states that any repeating signal of period $T$ can be expressed as a sum of $\sin$ waves with periods $T$, $T / 2$, etc.

  \item It's important to define the domain of a SHM equation, e.g. for what values of $t$ is the motion defined?

  \item SHM can be considered a projection of uniform circular motion

  \item That uniform circular motion can be represented by a number in the complex plane, with the projection being its real part

  \item Multiplication by $j$ can be considered a counter-clockwise rotation of $\ang{90}$ in the complex plane

  \item Euler's formula states \[e^{j \theta} = \cos \theta + j \sin \theta\]

  \item Multiplication of a complex number $z$ by $e^{j \theta}$ is equivalent to a counter-clockwise rotation of $z$ by an angle of $\theta$
\end{itemize}

\section{The superposition of periodic motions}

\begin{itemize}
  \item The combination of two SHM's of the same period

        \[x_1 = A_1 \cos (\omega t + \alpha_1)\]
        \[x_2 = A_2 \cos (\omega t + \alpha_2)\]

        is given by \[x = A \cos (\omega t + \alpha)\] where \[A^2 = A_1^2 + A_2^2 + 2 A_1 A_2 \cos (\alpha_2 - \alpha_1),\] \[A \sin \beta = A_2 \sin (\alpha_2 - \alpha_1),\] and \[\alpha = \alpha_1 + \beta.\]

  \item The combination in complex representation

        \[z_1 = A_1 e^{j (\omega t + \alpha_1)}\]
        \[z_2 = A_2 e^{j (\omega t + \alpha_2)}\]

        is given by

        \[z = e^{j (\omega t + \alpha_1)} [A_1 + A_2 e^{j (\alpha_2 - \alpha_1)}]\]

  \item In the case where $A_1 = A_2$ if we denote $\delta = \alpha_2 - \alpha_1$ then \[\beta = \frac{\delta}{2}\] and \[A = 2 A_1 \cos \beta = 2 A_1 \cos \frac{\delta}{2}\]

  \item The superposition of two sinusoids with different periods will itself be periodic if there exist integers $n_1$ and $n_2$ such that \[T = n_1 T_1 = n_2 T_2\] where $T_1$ and $T_2$ are the periods of the two sinusoids

  \item Periodic motion in two or more dimensions can be represented by extending the ``projection of a rotating vector'' approach, with one vector for each axis, e.g. \[x = A_1 \cos \omega t\] \[y = A_2 \cos \omega t\] where differing amplitudes, frequencies, and phase differences product different curves called \textbf{Lissajous curves}
\end{itemize}

\section{The free vibrations of physical systems}

\begin{itemize}
  \item When a tensile force is applied to a material it elongates. The ratio of the elongation to the original length $x / l_0$ is known as the \textbf{tensile strain}

  \item The ratio of the tensile force to the cross sectional area of the material $F / A$ is known as the \textbf{tensile stress}

  \item The ratio of stress and strain is a constant known as \textbf{Young's modulus} $Y$

  \item The force exerted by the stretched material on another object is given by \[\frac{F / A}{x / l_0} = -Y \Rightarrow F = -\frac{A Y}{l_0} x\] which is in the form of Hooke's law with $k = -\frac{A Y}{l_0}$
\end{itemize}

\section{Forced vibrations and resonance}

\begin{itemize}
  \item Periodic motion that isn't simple harmonic is \textbf{anharmonic}
\end{itemize}

\end{document}