\documentclass{article}
\usepackage{amsmath} % For align*
\usepackage{enumitem} % For customisable list labels
\usepackage{graphicx} % For images
\usepackage{siunitx} % For units
\graphicspath{{./images/}}

\title{Vibrations and Waves by A. P. French Notes}
\author{Chris Doble}
\date{May 2022}

\begin{document}

\maketitle

\tableofcontents

\section{Periodic motions}

\begin{itemize}
  \item Fouriers theorem states that any repeating signal of period $T$ can be expressed as a sum of $\sin$ waves with periods $T$, $T / 2$, etc.

  \item It's important to define the domain of a SHM equation, e.g. for what values of $t$ is the motion defined?

  \item SHM can be considered a projection of uniform circular motion

  \item That uniform circular motion can be represented by a number in the complex plane, with the projection being its real part

  \item Multiplication by $j$ can be considered a counter-clockwise rotation of $\ang{90}$ in the complex plane

  \item Euler's formula states \[e^{j \theta} = \cos \theta + j \sin \theta\]

  \item Multiplication of a complex number $z$ by $e^{j \theta}$ is equivalent to a counter-clockwise rotation of $z$ by an angle of $\theta$
\end{itemize}

\section{The superposition of periodic motions}

\begin{itemize}
  \item The combination of two SHM's of the same period

        \[x_1 = A_1 \cos (\omega t + \alpha_1)\]
        \[x_2 = A_2 \cos (\omega t + \alpha_2)\]

        is given by \[x = A \cos (\omega t + \alpha)\] where \[A^2 = A_1^2 + A_2^2 + 2 A_1 A_2 \cos (\alpha_2 - \alpha_1),\] \[A \sin \beta = A_2 \sin (\alpha_2 - \alpha_1),\] and \[\alpha = \alpha_1 + \beta.\]

  \item The combination in complex representation

        \[z_1 = A_1 e^{j (\omega t + \alpha_1)}\]
        \[z_2 = A_2 e^{j (\omega t + \alpha_2)}\]

        is given by

        \[z = e^{j (\omega t + \alpha_1)} [A_1 + A_2 e^{j (\alpha_2 - \alpha_1)}]\]

  \item In the case where $A_1 = A_2$ if we denote $\delta = \alpha_2 - \alpha_1$ then \[\beta = \frac{\delta}{2}\] and \[A = 2 A_1 \cos \beta = 2 A_1 \cos \frac{\delta}{2}\]

  \item The superposition of two sinusoids with different periods will itself be periodic if there exist integers $n_1$ and $n_2$ such that \[T = n_1 T_1 = n_2 T_2\] where $T_1$ and $T_2$ are the periods of the two sinusoids

  \item Periodic motion in two or more dimensions can be represented by extending the ``projection of a rotating vector'' approach, with one vector for each axis, e.g. \[x = A_1 \cos \omega t\] \[y = A_2 \cos \omega t\] where differing amplitudes, frequencies, and phase differences product different curves called \textbf{Lissajous curves}
\end{itemize}

\section{The free vibrations of physical systems}

\begin{itemize}
  \item When a tensile force is applied to a material it elongates. The ratio of the elongation to the original length $x / l_0$ is known as the \textbf{tensile strain}

  \item The ratio of the tensile force to the cross sectional area of the material $F / A$ is known as the \textbf{tensile stress}

  \item The ratio of stress and strain is a constant known as \textbf{Young's modulus} $Y$

  \item The force exerted by the stretched material on another object is given by \[\frac{F / A}{x / l_0} = -Y \Rightarrow F = -\frac{A Y}{l_0} x\] which is in the form of Hooke's law with $k = -\frac{A Y}{l_0}$
\end{itemize}

\section{Forced vibrations and resonance}

\begin{itemize}
  \item Periodic motion that isn't simple harmonic is \textbf{anharmonic}
\end{itemize}

\section{Coupled oscillators and normal modes}

\begin{itemize}
  \item A property of a normal mode is that all objects oscillate at the same frequency
\end{itemize}

\section{Normal modes of continuous systems. Fourier analysis}

\begin{itemize}
\item If a medium is vibrating at a natural frequency with only one end fixed (e.g. the pressure in a tube with one end open), the length of the medium must be an integer multiple of quarter wavelengths

\item In one-dimensional systems, the frequency of a normal mode $f_n$ is proportional to the mode number $n$ for small $n$

\item In higher-dimensional systems, the frequency of a normal mode $f_n$ is not proportional to the mode number $n$

\item In higher-dimensional systems, one frequency may correspond to multiple normal modes and is said to be \textbf{degenerate}

\item The process of determining the coefficients of a Fourier series is called \textbf{harmonic analysis}

\item One way to think of orthogonal functions is as vectors of infinite dimension. Two $n$-dimensional vectors $\mathbf{a}$ and $\mathbf{b}$ are orthogonal if their scalar product is $0$, i.e. \[\mathbf{a} \cdot \mathbf{b} = 0 \text{ if } \sum_0^n a_n b_n = 0.\] If two functions $f(x)$ and $g(x)$ are considered vectors of infinite dimension then the expression is similar \[\int_0^L f(x) g(x) \,dx = 0 \text{ is approximately } \sum_{n = 0}^\infty f(x_n) g(x_n) = 0\]
\end{itemize}

\section{Progressive waves}

\begin{itemize}
  \item A normal mode of vibration of a stretched string can be described as the superposition of two sine waves, identical to one another, traveling in opposite directions

        \begin{align*}
          y_n(x, t) & = A \sin \left( \frac{n \pi x}{L} \right) \cos \omega_n t                                                                                  \\
                    & = \frac{1}{2} A_n \left[ \sin \left( \frac{n \pi x}{L} - \omega_n t \right) + \sin \left( \frac{n \pi x}{L} + \omega_n t \right) \right]   \\
                    & = \frac{1}{2} A_n \left[ \sin \left( \frac{2 \pi}{\lambda} (x - v t) \right) + \sin \left( \frac{2 \pi}{\lambda} (x + v t) \right) \right]
        \end{align*}

  \item In reality the wave velocity $v$ is typically a function of the frequency $f$ / the wavelength $\lambda$

  \item When deriving the wave equation it's possible to deal only with first derivatives, i.e.

        \begin{align*}
          y(x, t)                       & = A \sin \left[ \frac{2 \pi}{\lambda} (x - v t) \right]                          \\
          \frac{\partial y}{\partial x} & = \frac{2 \pi}{\lambda} A \cos \left[ \frac{2 \pi}{\lambda} (x - v t) \right]    \\
          \frac{\partial y}{\partial t} & = -\frac{2 \pi v}{\lambda} A \cos \left[ \frac{2 \pi}{\lambda} (x - v t) \right] \\
          \frac{\partial y}{\partial x} & = -\frac{1}{v} \frac{\partial y}{\partial t}
        \end{align*}

        however this only applies to waves travelling in the positive $x$ direction. By taking the second derivative we arrive at a relation that also applies to waves travelling in he negative $x$ direction \[\frac{\partial^2 y}{\partial x^2} = \frac{1}{v^2} \frac{\partial^2 y}{\partial t^2}\]

  \item If a function $y(t)$ is even with respect to its midpoint in time, i.e. $y(-t) = +y(t)$, then it can be represented by a Fourier series of cosine functions alone. If it is odd, i.e. $y(-t) = -y(t)$ then it can be represented by sine functions alone. Otherwise its Fourier series contains both sine and cosine functions.

  \item In performing the frequency analysis of a short pulse of a particular frequency $f$, we find that the longer the pulse the better it is represented by a single sinusoidal wave of frequency $f$ — the width of its frequency spectrum narrows. Inversely, as the pulse shortens the width of its frequency spectrum broadens.

  \item \textbf{Cut-off} is the inability of a dispersive medium to transmit waves above (or possibly below) a certain critical frequency. The rate at which waves above the maximum frequency attenuate is proportional to the frequency.

  \item Energy per unit length is also known as energy density

  \item The kinetic energy per unit length of a sinusoidal wave on a stretched string given by $y(x, t) = f(x \pm vt) = f(z)$ is \[\frac{d K}{d x} = \frac{1}{2} \mu \left( \frac{\partial y}{\partial t} \right)^2 = \frac{1}{2} \mu v^2 [f'(z)]^2\] and the potential energy per unit length is \[\frac{d U}{d x} = \frac{1}{2} T \left( \frac{\partial y}{\partial x} \right)^2 = \frac{1}{2} T [f'(z)]^2.\] Given that \[v = \sqrt{\frac{T}{\mu}} \Rightarrow \mu v^2 = T\] these values are equal.

  \item The total energy in a wavelength of a sinusoidal wave on a stretched string is given by \[E = \frac{1}{2} (\lambda \mu) \mu_0^2\] where \[\mu_0 = 2 \pi f A.\] This is also the amount of work that must be done on the string to establish that wavelength

  \item The rate at which work is done on a stretched string to establish a sinusoidal wave is \[P = \frac{1}{2} \mu \mu_0^2 v\] which is the amount of energy in the wave per unit length $\frac{1}{2} \mu \mu_0^2$ times the velocity at which the wave propagates $v$
\end{itemize}

\section{Boundary effects and interference}

\begin{itemize}
  \item When a travelling wave reaches a fixed end, a wave of opposite displacement is reflected back to the source

  \item When a travelling wave reaches a free end, a wave of the same displacement is reflected back to the source

  \item Reflection can be considered the superposition of the travelling wave and a corresponding ``virtual'' travelling wave that is initially past the end of the string and travelling in the opposite direction

  \item If a pulse of the form $f_1(t - x / v_1)$ is moving along a stretched string of linear density $\mu_1$ connected to a string of linear density $\mu_2$, the displacements of the strings can be represented as

        \begin{align*}
          y_1(x, t) & = f_1 \left( t - \frac{x}{v_1} \right) + g_1 \left( t + \frac{x}{v_1} \right) \text{ and} \\
          y_2(x, t) & = f_2 \left( t - \frac{x}{v_2} \right).
        \end{align*}

        Where

        \begin{align*}
          g_1 \left( t + \frac{-x}{v_1} \right)          & = \frac{v_2 - v_1}{v_2 + v_1} f_1 \left( t - \frac{x}{v_1} \right) \text{ and} \\
          f_2 \left( t - \frac{v_2 x / v_1}{v_2} \right) & = \frac{2 v_2}{v_2 + v_1} f_1 \left( t - \frac{x}{v_1} \right)
        \end{align*}

        i.e. the reflected pulse $g_1$ moves in the negative $x$ direction at the same speed as $f_1$ and its amplitude is scaled by a factor of $\frac{v_2 - v_1}{v_2 + v_1}$ (which may be negative) whereas the transmitted pulse $f_2$ is scaled horizontally, moves in the positive $x$ direction at a different speed, and its amplitude is scaled by a factor of $\frac{2 v_2}{v_2 + v_1}$

  \item Electrical impedance has both a magnitude and phase so it can be represented as a complex quantity. These impedances can then be added to calculate the impedance of a circuit.

  \item The \textbf{mechanical impedance} of a physical system is defined as the ratio of the driving force to the associated velocity of displacement

  \item The mechanical impedance of a stretched string is \[Z = \frac{T}{v}\] where $v$ is the velocity at which waves propagate along the string

  \item The \textbf{characteristic impedance} of a stretched string is \[Z = \sqrt{T \mu}\]

  \item A stretched string can only carry transverse waves

  \item A long spring can carry both transverse and longitudinal waves

  \item A column of gas or liquid has no elastic resistance to change of shape, only density, and thus can only carry longitudinal waves unless gravity or surface tension provide an elastic restoring force

  \item Three-dimensional objects may have two polarizations of transverse waves — perpendicular to one another and the direction of travel. They may also have different wave speeds.

  \item A given interface may behave differently between longitudinal and transverse waves. For example, in a tank of water with smooth vertical walls the walls will act as a rigid boundary for longitudinal waves but completely free for transverse waves. If standing waves are set up the wall will act as a node for longitudinal waves and an antinode for transverse waves.

  \item The angle of a reflected wave equals the angle of incidence

  \item \textbf{Snell's law} relates the angles of incidence and refraction of two waves to the refractive index or wave speed of the two mediums \[\frac{\sin \theta_1}{\sin \theta_2} = \frac{n_2}{n_1} = \frac{v_1}{v_2}\]

  \item The \textbf{Doppler effect} is the apparent change in frequency or wavelength of a wave in relation to an observer moving relative to the wave source

  \item If a wave source is moving in a straight line at speed $v$ and the waves have speed $u$ then the wavefronts emitted in the direction of motion will be closer together and have effective wavelength \[\lambda_\text{min} = \lambda_0 \left( 1 - \frac{u}{v} \right)\] while those emitted $\ang{180}$ from the direction of motion will be further apart and have effective wavelength \[\lambda_\text{max} = \lambda_0 \left( 1 + \frac{u}{v} \right)\]

  \item More generally, if the observer is at an angle $\theta$ to the direction of motion of the wave source, the effective wavelength will be \[\lambda(\theta) = \lambda_0 \left( 1 - \frac{u \cos \theta}{v} \right)\]

  \item Interference patterns resulting from multiple slits (a \textbf{diffraction grating}) are more complicated. Let a diffraction grating contain $N$ slits with a distance between neighboring slits $d$ and the amplitude of the resulting wave be measured at a point $P$. If $P$ is at an angle $\theta$ from the first slit, the difference in path length between the slits is $d \sin \theta$. This means the phase difference between waves originating from the slits is $k d \sin \theta = 2 \pi d \sin \theta / \lambda$. This results in an amplitude at $P$ of \[A = A_0 \frac{\sin N \delta / 2}{\sin \delta / 2}.\] This results in a single \textbf{principal maxima} at $\delta = 2n \pi$ for $n = 0, \,1, \,\ldots$, zeroes at $\delta = 2 n \pi / N$ for $n = 1, \,2, \,\ldots$, and \textbf{subsidiary maxima} between these zeroes.
\end{itemize}

\end{document}